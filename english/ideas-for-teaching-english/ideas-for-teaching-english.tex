\documentclass[12pt,a4paper]{report}
%\usepackage{english-style}

%\author{Carol et al}
%\title{Ideas for Teaching English}
\setcounter{secnumdepth}{-1}

\begin{document}
%==============================================================================

\chapter{Ideas for Teaching English}

\section{Importance of English}

The most obvious reason to teach English is that the medium of instruction and examinations are English. Thus, if students know English they will be able to excel in the classroom and pass their final exams. 

However, there are other reasons why teaching English is important. English is the international language. If our students know English, they are more likely to find work, travel, develop a worldly view, and increase their standard of living. 

Learning another language also stimulates the brain and can encourage development in analytical thinking. Learning the grammar of English also helps students understand the grammar and intricacies of their own language. 

%------------------------------------------------------------------------------
\section{Introducing English}

\subsubsection{Facilitate a positive atmosphere for speaking and learning English:}
\begin{itemize}
\item{Share these and other ideas with the teachers at your school}

\item{Work with your counterpart/PCV to convince teachers and staff of the importance of English}

\item{English only policy or a written contract for teachers and students to sign and follow.}

\item{Enforce language rules (positively) by giving prizes, movie nights, stickers, baked goods, etc. to English speakers.}

\item{Detention as punishment and written English essay to get out of detention.}

\item{Give stickers to teachers to give to students who speak to them in English.}
\end{itemize}

\subsubsection{Teachers set an example by speaking in English only when at school}
\begin{itemize}
\item{Teach or help your teachers (especially English teachers and form 6 leavers) improve their English}

\item{Make a Swahili necklace or hat for Swahili speakers to wear.}

\item{Alternative: make a necklace or hat for English speakers.}

\item{At the end of each day make the students pass the necklace or hat saying I gave the necklace to so-and-so until all Swahili speakers are announced. Give these students detention, written assignments in English or make them say something at parade in English.}

\item{Encourage teachers and students to attend school-wide activities: debate, morning speech, spelling-bee, English club, etc.}
\end{itemize}

\subsubsection{Inspire students to understand the importance of understanding versus memorizing English:}
\begin{itemize}
\item{Bring a map of the world to discuss languages emphasizing the global use of English. Where is Kiswahili spoken/understood? What language do they speak in (country)? Where do they speak (language)?}

\item{Rewrite simple exam questions with words they won’t know to show them how sad it would be if they missed the question because they don’t know the English.}
\end{itemize}

\subsubsection{Train your students to organize their English learning and practice regularly.}
\begin{itemize}
\item{Review the parts of speech - article, verb, noun, etc. Make fill-in-the blank multiple choice with science or math words they do not know. Does the blank sound like a verb, noun, adjective? Which part of speech does each choice sound like?}

\item{Make a list of common prefixes and suffixes. When new vocabulary has prefixes or suffixes, ask students to define parts of words they know before writing the definition.}

\item{Relate words which are opposites or binary (multicellular/unicellular, parallel/perpendicular, cation/anion). Have students say the opposite of a word whenever you say “opposite”.}

\item{Show students how to create bubble maps connecting all the concepts they learned in one topic or subtopic.}

\item{Vocabulary lists for each topic/subtopic}

\item{Teach peer editing and encourage students to edit each other’s notes, essays, news articles and other writing.}

\end{itemize}

%------------------------------------------------------------------------------
\section{In Class}

\subsection{General Strategies}

\subsubsection{On the board}
\begin{itemize}
\item{Write important words on one side of the board – define in simple English or write Swahili}

\item{Pronounce new words as you write the syllables}

\item{When students come to the board, have them do work in steps to catch mistakes}
\end{itemize}

\subsubsection{Assessment}
\begin{itemize}
\item{Group test - When you get tired of students failing, test their collective knowledge with a group test. Write one question on the board at a time and select a student at random to answer it. All students get the collective grade. To avoid cheating, make sure the class is silent and have the student come to the board before you write the question.}

\item{Group work – spread out the capable students and give the whole group the same grade or let them grade each other.}

\item{Have students underline the words they don’t know on assignments so you will know what to teach or which words to use on a test.}

\item{Put everything away and write a problem.}

\item{EYKA quiz – write down everything you know about (topic)}

\item{Dictation - Read a couple sentences related to a recent topic and have students write them as you read aloud. Write any new vocabulary in the dictation on the board. Read a couple times slowly enough for students to write.}

\end{itemize}

\subsubsection{Lecturing}
\begin{itemize}

\item{Special English = slow + repetition}

\item{Repeat instructions and commands}

\item{Fill-in-the-blank notes for students to fill in as you are speaking}

\item{Check for understanding by thumbs up/down, “raise your hand if…” or “do you understand?” If students say they understand, ask one to explain it to the class}

\item{Ask specific questions/ask why?}

\item{For difficult concepts, team teach - invite a teacher to watch you lecture in English and explain in Swahili}

\item{Team teaching is possible using a student or two from form 3 or 4. It helps them learn by teaching; also, they are more equipped with Swahili and familiar Tanzanian methods to explain difficult ideas.}

\item{Leave an exercise book with the monitor/monitrice for students to anonymously write questions and list unknown words. Check these books every afternoon.}

\end{itemize}

\subsection{Interactive Teaching Techniques}

\subsubsection{Students speak!}
\begin{itemize}
\item{Give students participation points for speaking in class}

\item{Post a schedule for students to sign up and present on a particular topic or question.}

\item{Get students speaking in front of class 5-10 minutes before you teach}

\item{See Play! for games to get your students speaking}
\end{itemize}

\subsubsection{Draw!}
\begin{itemize}
\item{Use pictures and examples. If you are teaching about classification, draw the organisms and quiz students. Common accidents in the chemistry lab could also be drawn and used to teach students the hazards and new English vocabulary. Teach students shapes by drawing them rather than their Kiswahili name.}

\item{Draw or have students draw examples on pieces of paper and write what they are on other pieces paper. Handout all of the papers at the beginning of class and have students find the matching drawing or description, then come to the front and describe the example.}
\end{itemize}

\subsubsection{Act!} 
\begin{itemize}
\item{If you are talking about hearing, ask students to touch their ears. When describing a lever, move your arm. If you are discussing temperature regulation, have students act like they are in a hot/cold environment. Teach form IV physics students to do the wave.}

\item{Give students scenarios and have them make skits for topics like first aid, fire fighting, lab safety, gravity/no gravity, perimeters/area or ratios.}
\end{itemize}

\subsubsection{Create!}
\begin{itemize}
\item{Have students create projects. By re-reading, summarizing, and presenting from lessons, students reinforce what they have learned and increase their confidence in writing and speaking English.}

\item{Ask students to bring or make the teaching aids}

\item{Give out pieces of paper explaining a process and have students put them in order (English NECTAs have a section on putting sentences in order)}

\end{itemize}

\subsubsection{Play!} 
\begin{itemize}
\item{Kitimoto – one student sits in front while others ask questions to review subject material}

\item{Morning speech – randomly invite a student or pair of students to the front of the room to give a speech or ask each other questions about a specific topic. Other students ask questions or challenge the points made by speaker.}

\item{Question ball – toss a ball or stuffed animal around the room; thrower (or teacher) asks a question, catcher answers and asks another question or asks for help (or ball returns to teacher)}

\item{20 questions – try to guess a noun or concept by asking mwenye noun 20 or less yes or no questions.}

\item{Spell a long word (ex. Extraction) that your students should know. How many words can they find using only the letters of that word (ex. Cat, reaction, taxi, etc.)?}

\item{Nouns in a bag (sharades, pictionary, taboo) – write concepts on small paper and put them in a bag. Students select a piece of paper and act, draw or describe the word without saying what is on the paper. Example: lab safety/procedure/apparatus or first aid.}
\end{itemize}

\subsubsection{Translate!}
\begin{itemize}
\item{Before teaching a topic or lesson, choose 5-10 words that students need to know. For example, when teaching about circulation, give students a core vocabulary list for blood, heart, circulate, pump, and tube. Do not translate every word in the notes, but rather be sure students add to their own English vocabulary and increase their understanding of the subject material.}

\item{Students take turns to spell words (can be vocabulary from science/math class) while others translate, give definitions, opposites or examples.}

\end{itemize}

%------------------------------------------------------------------------------
\section{After School}

\subsubsection{Read!}
\begin{itemize}
\item{If you have a library or books}

\item{Book club - students read a book a month and discuss (prepare questions in advance)}

\item{Reading incentive - encourage students to read by a sticker chart or clothes-pinning each name to a string that goes around the room (up high so sneaking their clothespin along the line creates a distraction) move a certain distance for each book read or give the books points by length and difficulty. Award prizes for most points or books read at the end of each month/term.}

\end{itemize}

\subsubsection{Without books}
\begin{itemize}
\item{Wikipedia has a simple English website: simple.wikipedia.org}

\item{Students do writing exercises and read each other’s work.}

\item{Write short stories for students to read}

\item{Make books of familiar children’s stories or your own.}

\item{Reading hour - invite students to come with a book (bring books, magazines and newspapers if you have them, and bring your dictionaries be around to explain if they don’t understand what they're reading.}

\item{FEMA publishes guides for how to use Fema magazines that have great lesson plans for life skills, reading, writing, debate, discussion, etc.}

\end{itemize}

\subsubsection{Listen!}
\begin{itemize}

\item{Listen to popular songs and write the lyrics.}

\item{Books on tape can be downloaded from Librivox.org or similar sites listed here}

%\item{$\mathrm{http://www.how-to-learn-any-language.com/forum/forum_posts.asp?TID\=6386}$ or record your own.}

\item{Dictation (see IN CLASS – Assessment)}

\item{Read aloud from a book, newspaper, students' writing or your own.}

\end{itemize}

\subsubsection{Talk!}
\begin{itemize}

\item{Informal English time – Ask students questions about what they’re doing or what they learned after greeting them, on breaks, after school or in the village.}

\item{English conversation group - PSKK (or Piga Story Kwa Kiingereza) Club - Format can be very flexible; meeting bi-weekly, students are presented with a topic or theme of discussion which may be decided in advance so they can prepare (see writing/discussion ideas).}

\item{Interviews - students to interview each other or members of the community who speak English.}

\item{Read aloud - students to read aloud from subject material, story books or stories they have written.}

\item{If you coach sports, enforce your school’s English policy on the field, track or court.}
\end{itemize}

\subsubsection{Create!}
\begin{itemize}
\item{School newspaper - make a template for student journalists to fill in with weekly articles, editorials, advertisements, and comics. (Alternative: science journal)}

\item{Literary zine - pay to publish (i.e. printed and bound in a stationery or printed and stapled) biannual collection of students' fiction, non-fiction, poetry with pictures. In some regions it might be cheaper to have this printed in and sent from the States.)}

\item{If you have a computer lab with internet access, a personal, class or school blog project might work; or set up a private social network on a site like www.xanga.com}

\item{World Wise media exchange – type up your students’ letters in an email with pictures, make video clips of your students making ugali, in class, doing cleanliness, etc. Have students draw pictures, write stories, or make short movies and invite the American class to react with a story, poem, movie, drawing, photo or song.}

\item{Group story - write the introductions of several stories (a sentence or paragraph about anything) on different pieces of paper paper. Have students pass the papers around adding one sentence to each story. Write a story on the board and edit it using coloured chalk and editing symbols.}

\item{Comic strips - students love the photo comic section of FEMA publications, introduce them to drawn comics with storylines.}

\item{Music – free-style rapping/church choir/school welcome party song and dance in English}

\end{itemize}

\subsubsection{Write!}
\begin{itemize}
\item{Keep a class journal (or several) for students to write stories or questions. Encourage students to answer each other’s questions or comment on each other’s stories. If they’re short on ideas, write a prompt every couple pages or bind envelopes together and write a prompt on each one, have students write answers on their own paper and put them in the envelopes.}

\item{Letter writing - if you don't have pen pals in the USA, try being pen pals with a neighboring school. Students will be on the same language level, have better chances of meeting each other, and won't have any reasons to beg.}
\end{itemize}

\subsubsection{Compete!}
\begin{itemize}
\item{Essay contest - descriptive or analytical writing about a prompt. A good enough prize will get the whole school writing. Beware: you might get tired of reading the same thing.}

\item{Spelling Bees! Students take turns spelling words with increasing difficulty. They are “out" if they misspell a word until Each week a different stream competes and winners are selected from each. Then have the winners from each stream compete for the school title.}

\item{School debate between forms. Let the students choose the motion and be the judge, being sure they score students on points, explanation, grammar, pronunciation, and confidence.}

\item{Make a weekly contest, one for stories, another week for songs, another week for poems or comics, throughout the year. Then compile the best into a literary magazine for the students at the end of the semester or year.}

\end{itemize}

\subsubsection{Play!}
\begin{itemize}
\item{Same games you play in class}

\item{Speed scrabble (aka Bananagrams) – Start with all caps face down. Players start with 7 letters each and work independently to spell connected (like in scrabble or on a crossword puzzle) words. When a player has used up all their letters, they say 'go' and everyone picks another letter. Letters are rearraged to include the new letter. Game continues until all letters are used. Winner places his or her last letter first. Everyone reads their words aloud. Paint the bottle caps or use the same soda to avoid cheating.}

\item{Silly sentences - Write phrases (verb, subject, time, and prepositional, see examples below) on cards and draw symbols (or use matching stickers) on the back of each card for each type of phrase. With the cards symbol side up have students select one of each and make a sentence, conjugating the verb to match the time. Have one student write the sentence, another translate, and a third draw a picture..:}

\item{time phrases: occasionally, yesterday, this afternoon, last night}

\item{subject phrases: my (brother/sister), the headmaster, a long, poisonous, hungry snake named Juma}

\item{verb phrases: (wash) my clothes, (eat) rice and beans, (wear) a purple dress, (get) pregnant}

\item{prepositional phrases: on a deserted island, at a birthday party, on TV}

\item{A sentence might be: Last night the headmaster washed my clothes at a birthday party.}

\end{itemize}

\subsubsection{Debate topics}

\begin{itemize}

\item{Globalization}
\item{Use nightly news for topics (kipimo joto)}
\item{Based on local village issues}
\item{Tanzania is being taken over by the free masons}
\item{Loliondo medicine man}
\item{Western health vs. medicine}
\item{African nations will never develop without foreign aid}
\item{Subject debates}
\item{Arsenal vs. Mann U}
\item{Tanzania vs. Kenya}
\item{Bongo flava is better than rap}
\item{Education}
\item{Tanzanian education should be in Kiswahili}
\item{Gender}
\item{A woman should always cook}
\item{Studetns should wear uniforms to school}
\item{Birth control is a woman’s responsibility}
\item{corporal punishment}
\item{day vs. boarding}
\item{education is better than money}
\item{private better than government}
\item{HIV will never be eradicatd}
\item{Students results are related to teachers}
\item{English as medium for primary}

\end{itemize}

\subsubsection{Writing/Discussion prompts}

\begin{itemize}
\item{auto/biography}
\item{describe people, places, hobbies, interests, culture, and custom}
\item{myths and fables}
\item{how to}
\item{poetry and song lyrics}
\item{hot topic editorials (abortion, witch doctor)}
\item{book review}
\item{write or talk about a photograph or item of sentimental value}
\item{interview with a family member, friend or local professional}
\item{religion}
\item{philosophy}
\item{What or whom inspires you to be a better student? Explain how to (ex. cook pilau)}
\item{Draw a family tree and tell your family history or describe the people in your family. Where have you been? What made a particular trip or school break memorable? FEMA articles}
\item{current events}
\end{itemize}
%==============================================================================
\end{document}
