
\chapter{Volumetric Analysis}
\section{Introduction}

Volumetric analysis is always one of the practicals on the Form Four national examination. Students must be able to perform accurate titrations and a few different calculations. Teachers must be able to not only teach these calculations and titration technique, but at most schools they must also be able to prepare the solutions themselves.

Accurate volumetric analysis may be performed with locally available materials. This chapter discusses the theory of volumetric analysis, how to prepare apparatus, indicators, and solutions, basic titration methods, and how to perform three common practical exercises.


\section{Indicators}

Indicators are essential for acid-base titrations. Local indicators, as prepared in the Acids and Bases section above, may be used when first teaching volumetric analysis. For the national exams, students should use commercial MO and POP. Instructions follow for preparing these indicators.

\subsubsection{Preparation of MO}
Dissolve 1 g of MO in 1 litre of distilled or rain water. Store MO in a plastic bottle with a secure cap. The solution should last for over a year.

\subsubsection{Preparation of POP}
Dissolve 1 g of POP in 700 mL of clear methylated spirit. Add 300 mL of distilled or rain water. Store POP solution in a plastic bottle with a secure cap. The solution should last for over a year as long as the bottle remains well sealed.

\section{Water}
The primary component of the solutions used in volumetric analysis is water. This section discusses what kind of water to use and how to measure specific volumes of it.

\subsection{Sources of Water}
While the preparation of analytic solutions for scientific research requires a pure source of water, the technique of relative standardization (discussed below) allows the teacher to use any source of water for preparing solutions for classroom volumetric analysis without affecting the result.

The best source of water for volumetric analysis at a school is rain water. Collect as much as possible during the rainy season. If rain water is scarce, however, what little there is should be used for qualitative analysis.

Tap or river water may also be used for volumetric analysis. Any acidic or basic compounds present in the water will not affect the result from volumetric analysis as long as relative standardization is used. If the water is hard water, dissolved calcium and magnesium will precipitate with the addition of strong bases, causing the solution to form a precipitate and turn cloudy. Water from old pipes often contains iron which precipitates with strong bases, causing the colour of the solution to change to yellow or brown. In these situations, the use of rain water is strongly encouraged.

\subsection{Measuring Volume}
Common plastic water bottles are made in an industrial process that ensures that each is exactly the same. These bottles usually have marks on the plastic surface, some of which correspond to specific volumes. For example, in half litre Maji Africa bottles, there is a series on curved lines on the lower half of the bottle and then a series of straight lines. The lowest of the straight lines corresponds to 250 mL. Therefore it is possible to use a common Maji Africa bottle to measure exactly 250 mL of water. (ILLUSTRATION)

Many other bottles have useful marking, including 
On the half litre Kilimanjaro bottle, there are marks for 100 mL, 200 mL, 300 mL, 400 mL, and 450 mL. (ILLUSTRATION)
On the 1.5 L Kilimanjaro bottle, there are marks for 300 mL and 1.5 L. (ILLUSTRATION)
On the Ndanda Springs bottle, there is a mark for 350 mL.

Common plastic bottles therefore let us measure directly 100 mL, 200 mL, 250 mL, 300 mL, 350 mL, 400 mL, 450 mL, and 1.5 L. Other volumes can be measured by addition and subtraction of these values. For example, 600 mL can be measured by using the 300 mL mark on the half litre Kilimanjaro bottles two times or by using the 350 mL mark on an Ndanda Springs bottle followed by 250 mL measured with a Maji Africa bottle. As another example, 1.3 L can be measured by filling the 1.5 L Kilimanjaro bottle to the 1.5 L mark and then pouring 200 mL of that water into the half litre Kilimanjaro bottle.

Different water bottles are available in different regions. Science teachers in every region should test with volumetric glassware water bottles common to that region and report the result to other teachers in the regions, perhaps through In-Service Training.

Note that bottles not in their proper shape (dented, crushed, over-inflated) will give incorrect measurements.

\section{Relative Standardization}

Preparing large volumes of solution is difficult with great accuracy. Relative standardization is a technique to correct the concentration of solutions so that they give the correct results for practical exercises. Note that this technique is only useful in educational situations where the purpose is to prepare a pair of solutions for titration that give an answer known by the teacher. In scientific research, the aforementioned technique -- absolute standardization -- is used because the concentration of one of the solutions is truly unknown.

All schools should use relative standardization to check the concentration of the solutions they prepare for the national examinations. This ensures that the tests measure the ability of the students to perform the practical, and not the quality of the school's balance, water supply, glassware, etc. While useful for all schools, relative standardization is particularly helpful for schools with few resources, as it allows the preparation of high quality solutions with low cost apparatus and chemicals.

The first part of this section explains the theory behind relative standardization. The second part of this section explains the procedure to follow to perform relative standardization.

\subsection{General Theory}

The principle of a titration is that the chemical in the burette is added until it exactly neutralizes the chemical in the flask. If the two chemicals react 1:1, e.g. 

\[ \mathrm{HCl}_{(aq)} + \mathrm{NaOH}_{(aq)} \longrightarrow \mathrm{NaCl}_{(aq)} + \mathrm{H}_{2}\mathrm{O}_{(l)} \]

then exactly one mole of the burette chemical is required to neutralize one mole of the chemical in the flask. If the two chemicals react 2:1, e.g. 

\[ 2\mathrm{HCl}_{(aq)} + \mathrm{Na}_{2}\mathrm{CO}_{3(aq)} \longrightarrow 2\mathrm{NaCl}_{(aq)} + \mathrm{H}_{2}\mathrm{O}_{(l)} + \mathrm{CO}_{2(g)} \]

then exactly two moles of the burette chemical is required to neutralize one mole of the chemical in the flask. Let us think of this reaction as a mole ratio.

\[ \frac{\mbox{moles of }A}{\mbox{moles of }B} = \frac{n_{A}}{n_{B}} \]

Where $ n_{A} $ and $ n_{B} $ are the stoichiometric coefficients of A and B respectively.

\[ \mathrm{moles} = \mathrm{molarity} \times \mathrm{volume} = M \times V \mbox{ (so long as V is measured in liters)} \]

By substitution,

\[ \frac{(M_{A})(V_{A})}{(M_{B})(V_{B})} = \frac{n_{A}}{n_{B}} \]

A student performing a titration might rearrange this equation to get

\[ M_{A} = \frac{(n_{a})(M_{B})(V_{B})}{(n_{B})(V_{A})} \]

or

\[ M_{B} = \frac{(n_{B})(M_{A})(V_{A})}{(n_{A})(V_{B})} \]

As teachers, however, we care with something else: making sure that our students find the required volume in the burette. Solving the equation for $ V_{A} $ we find that

\[ V_{A} = \frac{(n_{a})(M_{B})(V_{B})}{(n_{B})(M_{A})} \]

As $ n_{A} $ and $ n_{B} $ are both set by the reaction, as long as we use the correct chemicals there is no problem here.

$ V_{B} $ is measured by the students -- it is the volume they transfer into the flask. As long as the students know how to use plastic syringes accurately, they should get this value almost perfectly correct.

The remaining term, $ \frac{M_{B}}{M_{A}} $ is for the teacher, not the student, to make correct. If we prepare the solutions poorly, our students can do everything right but still get the wrong value for $ V_{A} $. It is very important that we ensure that our solutions have the correct ratio of $ \frac{M_{B}}{M_{A}} $ so that the exercise properly assesses the ability of our students.

Many people look at this ratio and decide that they therefore need to prepare both solutions perfectly, so that $ M_{B} $ and $ M_{A} $ are exactly what is required. This not true. The actual values for $ M_{B} $ and $ M_{A} $ are not important; what matters is the ratio $ M_{B} $ to $ M_{A} $!

For example, if the titration requires 0.10~M HCl and 0.10~M NaOH, our expected mole ratio is:

\[ \frac{M_{HCl}}{M_{NaOH}} = \frac{0.10}{0.10} = 1 \]

Preparing 0.11 M HCl and 0.09 M NaOH will cause the students to get the wrong answer:

\[ \frac{M_{HCl}}{M_{NaOH}} = \frac{0.11}{0.09} = 1.22 \]

However, preparing exactly 0.05 M HCl and 0.05 M NaOH results in the same molar ratio:

\[ \frac{M_{HCl}}{M_{NaOH}} = \frac{0.05}{0.05} = 1 \]

Thus the students can get exactly the right answer if they use the right technique even though neither solution was actually the correct concentration.

How can we ensure that we have the correct molar ratio between our solutions? Titrate your solutions against each other. If the volume is not the expected value, one of your solutions is too concentrated relative to the other. You can calculate exactly how much too concentrated and add the exact amount of water necessary to perfect the ratio. This process is called relative standardization, because you are standardizing one solution relative to the other.

\subsection{Procedure for Relative Standardization}
\label{sub:relstand}
In some titrations the acid is in the burette and in some it is the base is in the burette. So let us not use ``acid'' and ``base'' to refer to the solutions, but rather ``solution 1'' and ``solution 2'' where solution 1 is the solution measured in the burette and solution 2 is measured by pipette (syringe).

You should have prepared a bucket or so of each. The volume you have prepared is $ V_{1} $ liters of solution 1 and $ V_{2} $ liters of solution 2.

Titrate the solutions against each other. Call the volume you measure in the burette ``actual titration volume.''  You know the desired molarity of each solution, so from the above student equations you can calculate the burette volume you expect, which you might call ``theoretical titration volume.''

After the titration, there are three possibilities. If the actual titration volume equals the theoretical titration volume, your solutions are perfect. Well done.

If the actual titration volume is smaller than the theoretical titration volume, solution 1 is too concentrated and must be diluted. Use the ratio:

\[ \frac{V_{1} \mbox{ (before dilution)}}{V_{1} \mbox{ (after dilution)}} = \frac{\mbox{actual titration volume}}{\mbox{theoretical titration volume}} \]

If the actual titration volume is larger than the theoretical titration volume, solution 2 is too concentrated and must be diluted. Use the ratio:

\[ \frac{V_{2} \mbox{ (before dilution)}}{V_{2} \mbox{ (after dilution)}} = \frac{\mbox{theoretical titration volume}}{\mbox{actual titration volume}} \]

After diluting one of your solutions, repeat the process. After a few cycles, the solutions should be perfect. Remember that the volume ``before dilution'' is the volume actually in the bucket, so the amount you made less the amount used for these test titrations.

\section{Simple Acid-Base Titration}

\subsection{Theory}

Burettes are not necessary to perform volumetric analysis with reasonable precision. Students may use plastic syringes in place of burettes. These should be the most precise syringes available, which as of late 2010 were the 10~mL NeoJect brand plastic syringes. These syringes are more accurate than the low cost glass pipettes that many school purchase. As the accuracy of the titration is no better than its least accurate instrument, a titration with two plastic syringes is more accurate than a titration with a burette and a cheap glass pipette.

If use of these syringes is new to you, please read Use of Plastic Syringes before proceeding.

To get maximum precision from plastic syringes, students should learn how to estimate values between the lines on the syringe body. The NeoJect syringes are marked with lines every 0.2~mL. Students should observe the top of the fluid and decide if it is on the line exactly, half way in between, or in between half way and one of the lines. This allows them to divide the space between lines into four parts, giving them a precision of 0.05~mL. Estimation between gradations is standard practice with scientific instruments; even students using burettes should estimate the fluid height between the lines to at least 0.05~mL. Syringes have the capacity to deliver the precision required by most if not all national exams.

If students are using syringes in place of burettes, they require two syringes for the practical, one as a burette and a different one as the pipette. We recommend that you label the syringes, for example, on one syringe writing ‘Burette’ with a permanent pen to help students remember which is which.

\subsection{Titration Procedure without Burettes}

\begin{enumerate}

\item{Clean the ‘pipette’ syringe with water. Then rinse it with the acid or base solution you will be putting in the flask.}

\item{Use a syringe to transfer the required amount of acid or base to the flask. To do this transfer accurately, add first 1 mL of air to the syringe and then suck up the fluid to beyond the desired amount. Push back the plunger until the top of the fluid is exactly the volume required. ((ILLUSTRATION)) Delivering the required volume to the flask may take multiple transfers with the single syringe. Record the total volume transferred to the flask as the ‘volume of pipette used’}

\item{Add one or two drops of indicator to the flask.}

\item{Clean the ‘burette’ syringe with water. Then rinse it with the acid or base solution you will be using to titrate.}

\item{Add 1~mL of air to the syringe and then suck up the acid or base solution to beyond the 10~mL mark. Slowly push back the plunger until the top of the fluid is exactly at the 10 mL line.}

\item{Slowly add the solution from the syringe to the flask. As you titrate, swirl the flask to mix. As described above, swirl instead of shaking to keep all of the liquid together. Make sure that each drop from the syringe hits the liquid rather than getting suck on the edge of the container. Stop titration when the indicator starts a permanent color change. Just as with a burette, this is the endpoint.}

\item{Often the volume required from the ‘burette’ is greater than 10~mL. This is no problem – after finishing the syringe students should simply fill it again as they did the first time and continue. On their rough paper (scratch paper), they should note that they have already consumed 10~mL.}

\end{enumerate}

\subsection{Table of Results when using syringes in place of burettes}

At present, many national exam marking boards expect students to use burettes. The obvious problem is that while the top line on a burette is 0~mL, the top of the syringe reads 10~mL. For students to get the marks their careful technique deserves, they must record their results in a manner consistent with traditional reporting. On rough paper, students should calculate the volume of solution used in their titration. This is easy -- if the syringe started at 10.00 mL and ended at 2.55~mL, the student used $10.00 \mathrm{mL} - 2.55 \mathrm{mL} = 7.45 \mathrm{mL}$ of solution. If the student used two full syringes and the third finished at 4.65~mL, then the student used $10.00 \mathrm{mL} - 4.65 \mathrm{mL} = 5.35 \mathrm{mL}$ in the last syringe plus 10~mL in each of the first two syringes, so $5.35 \mathrm{mL} + 10 \mathrm{mL} + 10 \mathrm{mL} = 25.35 \mathrm{mL}$ total.

In the Table of Results, the student should then write 25.35~mL for the Volume Used. If this volume had been used in a burette, the student would have found an initial volume of 0.00~mL and a final volume of 25.35~mL. The rest of the table should be filled in this manner. When using a syringe as a burette, the student should always write 0.00~mL for the Initial Volume and then for Final Volume they should write the total number they calculated for Volume Used. This method will ensure that the students gets the marks he or she deserves for careful titration. 

\section{Volumetric Analysis Practical Activities}

There are three volumetic analysis activities commonly found on national examinations: determining the percent purity of a substance, determining relative molecular mass, and determining water of crystallization. Each of these activities uses the simple acid-base titration procedure above. The differences are in the preparation for the teacher and the calculations for the student.

\subsection{Determining Percent Purity}
Volumetric analysis may be used to find the purity of an acid or a base. In this kind of practical, students are told the mass of impure solute dissolved in a solution and the concentration of the other solution. 
For example: 

"You are provided with:
\begin{itemize}
\item{ Solution P containing 28.60 g per litre of impure sodium carbonate}
\item{Solution Q containing 0.20 mol of hydrochloric acid in a litre of solution}
\item{Methyl orange as an indicator}"
\end{itemize}
The student is then required to perform a standard acid base titration to calculate the molarity of sodium carbonate. From the calcuated molarity and the given concentration, the percent purity can be found. For a teacher, the trick is to make an ``impure" solution by dissolving the correct mass of pure substance. 
\subsubsection{Materials}
Hydrochloric acid*, sodium carbonate*, pipette*, burette*, acid/base indicator*, titration flask*, and beakers*.
\subsubsection{Preparation Procedure}
\begin{enumerate}
\item{Prepare an ``impure solution" of sodium carbonate. For example, if the sodium carbonate sample above is supposed to be 70\% pure, the teacher would dissolve $0.70*28.60g=20.02$ grams of sodium carbonate in one litre of water.  This number does not need to be exact because later it will be corrected using relative standardization.}
\item{Prepare a 0.2 M solution hydrochloric acid (or 0.1M solution of sulphuric acid, refer to the section above on substituting chemicals).}
\item{Prepare an indicator solution.}
\item{Use relative standardization as explained in Section \ref{sub:relstand} to standardize the acid and base solutions together. In this case, we know the concentration of acid is 0.20 \textit{M}. To calculate the concentration of base, we must first go from concentration to molarity: 
$$\frac{c}{M_r}=M$$
$$\frac{20.02 g/L}{86g/mol}=0.23\frac{mol}{L}$$
The volume of the base is dependent on which volume of ``pipette" the student uses. In the following example, 25 mL of base were used. Now, the volume and concentration of base is known and the concentration of acid is known so it is easy to find the expected volume of acid that should be required for the titration from the equation:
$$\frac{M_A V_A}{M_B V_B}=\frac{n_A}{n_B}$$
the mole fraction$\frac{n_A}{n_B}$  can be determined from the balanced chemical equation: 

2HCl$_(aq)$ + \ce{Na2CO3}$_(aq)$ $\longrightarrow$ NaCl + \ce{CO2} + \ce{H2O}

$$\frac{n_A}{n_B}=\frac{n_{HCl}}{n_{\ce{Na2CO3}}}=2$$
Thus,
$$\frac{M_A V_A}{M_B V_B}=2$$
solving for $V_A$
$$V_A=\frac{2 M_B V_B}{M_A}$$
$$V_A=\frac{2 M_B V_B}{M_A}$$
and substituting in the numbers: 
$$V_A=\frac{0.23 \frac{mol \ce{Na2CO3}}{L} \times 0.025L}{0.2\frac{mol HCl}{L}}=0.02875$$
This means that the solutions should be relatively standardized such that 25 mL of sodium carbonate requires 28.75 mL of HCl for neutralization. If on the first rount, the titration requires, for example, 30 mL of HCl you must dilute the carbonate solution and then titrate again.}
\end{enumerate}

\subsubsection{Activity Procedure}
\begin{enumerate}
\item {Use the provided solutions to perform a standard acid base titration.}
\item{Use the results to find the actual molarity of base solution. This can be done using the equation 
$$\frac{M_A V_A}{M_B V_B}=2$$
from above. Solving for $M_B$:
$$M_B=\frac{M_A V_A}{2V_B}$$}
\item {Calculate the percent purity using the equation
$$Percent Purity = \frac{Actual  Molarity}{Given  Molarity}\times 100$$}
\end{enumerate}


\subsection{Determining Relative Atomic Mass}

Volumetric analysis may also be used to determine the identity of an unknown element in a compound. For example, it might be written ``You are provided with solution A containing a monovalent metal P hydroxide (POH). Solution A was made by dissolving 1.00 g of POH and making it up to 250 mL of solution". For this kind of practical, the teacher will be told what metal is present and can use relative standardization so that students get the correct answer.

\subsubsection{Preparation Procedure}
\begin{enumerate}
\item{Prepare acid and base solutions and use relative standardization to correct the volumes as is shown in the previous example.}
\end{enumerate}
\subsubsection{Activity Procedure}
\begin{enumerate}
\item{Use the provided solutions to perform a standard acid base titration.}
\item{Use the result to calculate the molarity of the unknown compound.}
\item{ From the molarity and the given mass concentration, the molar mass of the compound can be calculated using the equation: 
$$\frac{concentration(g/L)}{molarity(mol/L)}={molar mass(g/mol)}$$
This is the mass of P+O+H, so the molar mass of P is the calculated molar mass minus the masses of oxygen and hydrogen.}
\end{enumerate}


\subsection{Determining Water of Crystallization}

Many salts crystallize together with a predictable number of water molecules trapped in the crystal. For example, sodium carbonate usually forms a decahydrate, $\ce{Na2CO3}\cdot10\ce{H2O}$, where the crystals have ten molecules of water for every molecule of \ce{Na2CO3}. 

``You are provided with the folloiwing:
\begin{itemize}
\item {Solution AA containing 3.65 g of HCl per dm$^3$ of the solution.}
\item {Solution BB containing 7.15 g of hydrated sodium carbonate (\ce{Na2CO3xH2O}) in 0.5 dm$^3$}
\end{itemize}


The purpose of this activity is to determine number of waters of crystallization for the given compound.

\subsubsection{Preparation}
\begin{enumerate}
\item{The preparation for this activity is the same as for the previous activities. The solutions should be made with the approximate concentrations needed and then standardized using relative standardization. In this case, the teacher will know that the sodium carbonate is decahydrate, meaning that its molar mass is
$$M_r=2\times23+12+3\times16+10\times18=286g/mol$$
Which means the molarity of the sodium carbonate solution should be:
$$M=\frac{c}{M_r}=\frac{\frac{7.15g}{0.5dm^{3}}}{286g/mol}=0.025mol/L$$}
\end{enumerate}
\subsubsection{Activity Procedure}
\begin{enumerate}
\item {Use the provided solutions to perform a standard acid base titration.}
\item{Use the result to calculate the molarity of the sodium carbonate as is shown above.}
\item{Calculate the molar mass of the carbonate compound as is shownin the previouse example the water of crystallization.} \item {After the molar mass has been found, calculate the number of waters from the equation 

$$\mbox{Molar mass of compound without water} +18n = \mbox{Calculated  molar mass}$$}

Where n is the number of waters of crystallization.
\end{enumerate} 