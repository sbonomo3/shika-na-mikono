\section{Manufacturing Apparatus}

\subsection{Beakers}
Tools Required: Knife, scissors, or razor blade\\
Materials Required: Plastic water bottle\\
Manufacturing Procedure: Cut the bottom off of a plastic water bottle. Note that the top (with the cap) may also be used as a beaker.

\subsection{Burettes (option I)}
Tools Required: none\\
Materials Required: 10 mL plastic syringe\\
Manufacturing Procedure: Use the plastic syringe as a burette, to add a solution drop by drop.

\subsection{Burettes (option II)}
Tools Required: Knife, scissors, or razor blade\\
Materials Required: Plastic syringe, IV giving set, superglue\\
Manufacturing Procedure: Cut off the part of the IV tube with the flow control slider. Remove the plunger from the syringe and use superglue to attach the tube to the nozzle of the syringe.

\subsection{Evaporating dish}
Tools Required: none\\
Materials Required: metal take-away tray\\
Manufacturing Procedure: Pour the chemical to be evaporated into the dish and either leave in the sun (slow) or heat gently over a heat source (fast)

\subsection{Filter funnel}
Tools Required: none\\
Materials Required: cotton wool, plastic funnel\\
Manufacturing Procedure: Push cotton wool into the small part of the funnel. Pour the solution to be filtered into the funnel and allow gravity to pull the liquid through.

\subsection{Flame}
Motopoa stove - metal bottle cap filled with Motopoas

\subsection{Flasks}
glass liquor bottles, plastic water bottles

\subsection{Gas generator}
Tools Required: Knife or scissors, marker pen\\
Materials Required: Two plastic water bottles with caps, IV ``giving set'', pen biro, tape\\
Manufacturing Procedure:
\begin{enumerate}
\item{Cut approximately 25 cm of IV tube}
\item{Cut two sections of pen biro, each about 1 cm long}
\item{Use the knife of scissors to bore a very small hole in each pen cap. The holes should be exactly the same size as the pen biro.}
\item{Insert a cut piece of biro into each cap. Most of the length should be outside of the cap.}
\item{Force each end of the tubing over one of the biro ends, thus joining the two caps. ((ILLUSTRATION))}
\item{Use tape and the pen to label one bottle "reaction bottle" and the other "collection bottle."}
\end{enumerate}

\subsection{Heat source}
Motopoa stoves are generally the best, also kerosene and charcoal. See the section on Sources of Heat.

\subsection{Heating vessel}
metal spoons, opened light bulb ((ILLUSTRATIONS))

\subsection{Petri dishes}
Tools Required: knife, scissors, or razor blade\\
Materials Required: plastic water bottle\\
Manufacturing Procedure: Cut the bottom 2 cm from a plastic water bottle.

\subsection{Pipettes}
Plastic syringes are simply better than glass pipettes. Plastic syringes are easier to use, faster to use, more accurate, more safe, harder to break, and less expensive than most glass pipettes available in Tanzania.

\subsection{Test tube}
Tools Required: flame\\
Materials Required: 10 mL plastic syringe\\
Manufacturing Procedure: Remove the plunger from the syringe. Head the nozzle until it starts to melt and then press against a hard surface. If the resulting tube leaks water, heat and press again.

\subsection{Test tube holder}
Clothes pin, piece of paper

\subsection{Test tube rack}
Tools Required: knife or razor blade\\
Materials Required: cardboard\\
Manufacturing Procedure: Make two parallel folds in the cardboard to elevate the central section. ((ILLUSTRATION)) Make "X" cuts in the central section. A test tube can be pushed through these cuts to hold it when not in use.

\subsection{Water bathes}
If using a kerosene or charcoal stove, this is just a heated pot of water. If using Motopoa, read the instructions in the section on Heat Sources.