\chapter{Fire}
Fire is used frequently in the chemistry laboratory. It is also used quite a bit in the home. Therefore, a good understanding of fire is very important.
The activities in this chapter give students the chance to investigate the factors necessary for combustion to occur and the products of combustion. Finally, this section gives an activity for students to investigate the factors affecting rusting of metal.

\subsection{Investigating the Requirements for Combustion}
Combustion is the rapid chemical combination of a substance with oxygen to produce heat and light. There are three things required for combustion to occur: fuel, oxygen and a high enough temperature (called the kindling temperature). This activity gives students the chance to observe what happens when the oxygen supply is cut off from a flame. It is a very simple and quick activity that is best done as a demonstration for the class.
\subsubsection*{Learning Objectives}
\begin{itemize}
\item{To identify the factors necessary for combustion to take place.}
\end{itemize}

\subsubsection*{Materials}
A candle, A match box, a glass or metal cover and two beakers.

\subsubsection*{Preparation Procedure}
\begin{enumerate}
\item{Cut the candle into two pieces about 5-7 cm each and place them in each of the beakers. [ILLUSTRATION NEEDED]}
\end{enumerate}

\subsubsection*{Activity Procedure}
\begin{enumerate}
\item{Light the candle and leave them to burn for some time.}
\item{Cover the first beaker while the candle is still burning with a fire proof material (glass or metal) so as prevent air from entering.}
\item{Leave the second beaker open so there is a continuous supply of air. Observe what happen.}
\end{enumerate}

\subsubsection*{Results and Conclusion}
In the first beaker, lack of oxygen extinguishes the candle flame after some time. In the second beaker the flame will continue to burn due to the constant supply of oxygen gas. 

\subsubsection*{Clean Up Procedure}
\begin{enumerate}
\item{Collect all the used materials, cleaning and storing items that will be used later. No special waste disposal is required.}
\end{enumerate}

\subsubsection*{Discussion Questions}
\begin{enumerate}
\item{Explain what happened in the two beakers and why.}
\end{enumerate}


\subsection{Investigating the Products of Combustion}

Combustion is the rapid chemical combination of a substance with oxygen, producing heat and light. Most combustion activities involve burning compounds containing carbon and hydrogen. The carbon in the compound combines with oxygen to form carbon dioxide and the hydrogen combines with oxygen to form water. Therefore we expect carbon dioxide and water to be the products of most combustion activities.

The following activity is to demonstrate the production of carbon dioxide in one case of combustion: the burning of a candle. This activity requires about half an hour to show results, should it should be started at the beginning of the lesson and discussed near the end.

\subsubsection*{Learning Objectives}
\begin{itemize}
\item{To identify carbon dioxide as a products of combustion.}
\end{itemize}

\subsubsection*{Materials}
Candle, match box, beaker*, lime water*, deflagrating spoon*

\subsubsection*{Activity Procedure}
\begin{enumerate}
\item{Pour 10-20 mL of lime water in the beaker.}
\item{By using a deflagrating spoon lower the candle into the lime water container so the flame will sit above the surface of the solution. ((ILLUSTRATION NEEDED))}
\item{Light the candle and leave it to stay there for about 30 minutes and observe what happens to the lime water.}
\end{enumerate}

\subsubsection*{Results and Conclusion}
When the candle burns the lime water turns milky in colour, showing that there is production of carbon dioxide gas.

\subsubsection*{Clean Up Procedure}
\begin{enumerate}
\item{Collect all the used materials, cleaning and storing items that will be used later. No special waste disposal is required.}
\end{enumerate}

\subsubsection*{Discussion Questions}
\begin{enumerate}
\item{Why does lime water turn milky when the candle burns?}
\item{What will happen to the lime water when the candle burns for more than 30 minutes in production of excess carbon dioxide?}
\end{enumerate}

\subsubsection*{Notes}
The products of combustion are light, heat energy, carbon dioxide gas and soot. These products can be observed experimentally as demonstrated.

\subsection{Rusting}

Rusting is not combustion because the reaction is not rapid and no light is produced. Chemically, however, rusting is very similar to combustion. Both are examples of a materials undergoing destructive oxidation through reaction with oxygen gas. Iron rusts by reacting with oxygen and water to form a hydrated iron oxide called rust.

This activity is useful for showing students the conditions required for rusting - metal, oxygen (air), and water. After students learn about the requirements of rusting, they can participate in a discussion about how to prevent rusting. This activity should be performed by students working in small groups. The experiment should be started in one period and discussed in the next as at least 24 hours are required for results.
\subsubsection*{Objectives}
\begin{itemize}
\item{To demonstrate the conditions necessary for rusting}
\end{itemize}}

\subsubsection*{Materials}
Oil, water, sealed syringes, syringe plungers, iron (non-galvanized) nails, iron wool* and paint.

\subsubsection*{Activity Steps}
\begin{enumerate}
\item{Label 6 sealed syringes with the numbers 1-6}
\item{Place a nail in syringe 1 and fill with water. Leave open to air. This is the control.}
\item{Place a nail in syringe 2 and fill with water. Place the plunger back to seal the syringe to prevent oxygen from entering the shell.}
\item{Place a nail in syringe 3 and cover with oil. Leave open to air.}
\item{Place a nail in syringe 4 and cover with oil. Replace plunger to seal the syringe.}
\item{Place a nail in syringe 5 cover with water. Make an oil layer with a thickness of at least 2 cm.}
\item{Take a nail and paint the nail to cover the metal completely. Now place in syringe 5 and fill with water.}
\end{enumerate}


\subsubsection*{Results and Conclusion}  Syringe 1 is the control experiment. This will be the normal rusting of the iron nail. Syringe 2 now prevents any oxygen from participating in the rusting reaction. This nail should have little or no rust. Syring 3 allows oxygen to encounter the oil for rusting; however, there is no water since it is filled with oil. In addition, oxygen migrates through oil very poorly that also limits the available oxygen for rusting on the surface of the nail. There should be little or no rusting on the nail. Syringe 4 has no oxygen or water available for rusting. This nail should have no rusting. Experiment 5 allows water available to react, but now there is an oil layer. This oil layer prevents oxygen from entering the water layer and rusting the iron nail. This nail should have little or no rust. Syring 6 is much like the control, but since the nail is painted, there is no iron available to react since it is protected by the paint. In each of these experiments, the availability of the iron, oxygen, and water change and the effects of the lack of availability of each component slow down or even prevent the rusting reaction.}
\subsubsection*{Clean Up Procedure}
\begin{enumerate}
\item{Collect all the used materials, cleaning and storing items that will be used later. No special waste disposal is required.}
\end{enumerate}
\subsubsection*{Discussion questions}
\begin{enumerate}
\item{Which nail had the most rust at the end? Which nail had the least?}
\item{What was the purpose of oil in this experiment? What was the function of the plunger?}
\item{Why are ships painted even on surfaces that cannot be seen?}
\end{enumerate}}