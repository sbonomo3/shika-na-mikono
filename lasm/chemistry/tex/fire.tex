\chapter{Fire}

The investigation of chemistry often involves heating. In most laboratories, substances are heating over a fire. Therefore, it is important to understand how fires work. Fires are also part of every day life.

\subsection{Investigating the Requirements for Combustion}

\subsubsection*{Learning Objectives}
\begin{itemize}
\item{SWBAT identify the factors necessary for combustion to take place.}
\end{itemize}

\subsubsection*{Background Information}


\subsubsection*{Materials}
A candle, A match box, two beakers.

\subsubsection*{Preparation Procedure}
\begin{enumerate}
\item{Cut the candle into two pieces about 5-7 cm each and place them in each of the beakers. [ILLUSTRATION NEEDED]}
\end{enumerate}

\subsubsection*{Activity Procedure}
\begin{enumerate}
\item{Light the candle and leave them to light for some time.}
\item{Cover one beaker while candle still lighting so as prevent entering of air (oxygen).}
\item{Leave the second beaker open so there is a continuos supply of air (oxygen). Observe what happen.}
\end{enumerate}

\subsubsection*{Results and Conclusion}
In the second beaker left opened the candle will go on burning for a long time because there is a continuous oxygen supply to the flame. In the second covered beaker the candle will go out just after covering because there is no oxygen supply.

\subsubsection*{Clean Up Procedure}
\begin{enumerate}
\item{Collect all the used materials, cleaning and storing items that will be used later. No special waste disposal is required.}
\end{enumerate}

\subsubsection*{Discussion Questions}
\begin{enumerate}
\item{Explain what happened in the two beakers and why.}
\item{Write down the condition necessary for combustion to take place.}
\item{What are the product of  combustion? Justify your answer.}
\end{enumerate}

\subsubsection*{Notes}


\subsection{Investigating the Products of Combustion}

Combustion is the rapid chemical combination of a substance with oxygen, producing heat and light. Most combustion activities involve burning compounds containing carbon and hydrogen. The carbon in the compound combines with oxygen to form carbon dioxide and the hydrogen combines with oxygen to form water. Therefore we expect carbon dioxide and water to be the products of  
most combustion activities.

The following activity is to demonstrate the production of carbon dioxide in one case of combustion: the burning of a candle. This activity requires about half an hour to show results, should it should be started at the beginning of the lesson and discussed near the end.

\subsubsection*{Objectives}
\begin{itemize}
\item{To identify carbon dioxide as a products of combustion.}
\end{itemize}

\subsubsection*{Materials}
Candle, match box, beaker*, lime water*, deflagrating spoon*

\subsubsection*{Activity Procedure}
\begin{enumerate}
\item{Pour 10-20 mL of lime water in the beaker.}
\item{By using a deflagrating spoon lower the candle into the lime water container so the flame will sit above the surface of the solution.}
\item{Light the candle and leave it to stay there for about 30 minutes and observe what happens to the lime water.}
\end{enumerate}

\subsubsection*{Results and Conclusion}
When the candle burns the lime water turns milky in colour, showing that there is production of carbon dioxide gas.

\subsubsection*{Clean Up Procedure}
\begin{enumerate}
\item{Collect all the used materials, cleaning and storing items that will be used later. No special waste disposal is required.}
\end{enumerate}

\subsubsection*{Discussion Questions}
\begin{enumerate}
\item{Why does lime water turn milky when the candle burns?}
\item{What will happen to the lime water when the candle burns for more than 30 minutes in production of excess carbon dioxide?}
\end{enumerate}

\subsubsection*{Notes}
The products of combustion are light, heat energy, carbon dioxide gas and soot.These products can be observed experimentally as demonstrated.

\subsection{Rusting}

Rusting is not combustion because the reaction is not rapid and no light is produced. Chemically, however, rusting is very similiar to combustion. Both are examples of a materials undergoing destructive oxidation through reaction with oxygen gas. Iron rusts by reacting with oxygen and water to form a hydrated iron oxide called rust.

This activity is useful for showing students the conditions required for rusting - metal, oxygen (air), and water. After students learn about the requirements of rusting, they can participate in a discussion about how to prevent rusting. This activity should be performed by students working in small groups. The experiment should be started in one period and discussed in the next as at least 24 hours are required for results.

[INSERT ACTIVITY HERE]
