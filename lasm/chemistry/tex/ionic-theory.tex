\chapter{Ionic Theory and Electrolysis}
\subsection{Reactivity Series: Displacement Reaction}

\subsubsection*{Learning Objectives}
\begin{itemize}
\item{To demonstrate the displacement a metal with a more reactive metal.}
\end{itemize}

\subsubsection*{Background Information}
Metals can be arranged in order according to their reactivity. The most reactive metals are placed at the top of the list while the least reactive metals are near the bottom. A metal higher in the reactivity series will displace a metal lower in the series from a solution. [DIAGRAM OF REACTIVITY SERIES]

\subsubsection*{Materials}
Steel wool*, copper (II) sulphate*, water, *beaker.

\subsubsection*{Preparation}
\begin{enumerate}
\item{Prepare a copper (II) sulphate solution by dissolving one spoonful of crystals in 500 mL of water.}
\end{enumerate}

\subsubsection*{Activity Procedure}
\begin{enumerate}
\item{Arrange students in groups of 2-3 and to each group give a beaker containing 20 mL of copper sulphate solution and a small piece of steel wool.}
\item{Instruct students to dip the steel wool into the copper sulphate solution and observe what happens.}
\end{enumerate}

\subsubsection*{Results and Conclusion}
When the steel wool is dipped in the copper sulphate solution, a layer of brown copper metal forms on the solution. Iron ions displace copper ions in solution and the copper ions are deposited as copper metal.

\subsubsection*{Clean Up Procedure}
\begin{enumerate}
\item{Collect all the used materials, cleaning and storing items that will be used later.}
\item{Copper sulphate solution can be stored and labelled for future use-it will contain dissolved iron ions. If copper is not stored, it should be disposed of in a pit latrine.}
\end{enumerate}

\subsubsection*{Discussion Questions}
\begin{enumerate}
\item{Explain what happens when steel wool is dipped in a solution of copper (ii) sulphate?}
\item{Write the chemical equation for the reaction.}
\end{enumerate}

\subsubsection*{Notes}
In the electrochemical series, the more reactive metals displace the less reactive metals. Hence Iron displaces copper in the solution of CuSO4.
Equation.
    CuSO4(aq)   +   Fe(s)  ->  FeSO4(aq)  +  Cu(s)


\subsection{Electrolytes}

\subsubsection*{Learning Objectives}
\begin{itemize}
\item{To distinguish between strong, weak and non-electrolytes.}
\item{To test the conductivity of electrolytes in solution and in solid form.}
\end{itemize}

\subsubsection*{Background Information}
An electrolyte is a substance that forms ions when in solution and thus conducts electricity when dissolved in water. An electrolyte does not conduct electricity when in a solid state because the ions are fixed in a crystal lattice and cannot move.  A strong electrolyte is a substance which is a good conductor of electricity, many ions are formed in solution. A weak electrolyte is a poor conductor of electricity, only few ions are formed in solution. A non-electrolyte is a substance which does not conduct electricity.

\subsubsection*{Materials}
Sodium chloride*, distilled water*, kerosene, copper (II) sulphate*, sugar, 1 M weak acid solution*, 1 M sulphuric acid*, beakers*, stirrer*, 2 carbon electrodes*, connecting wires*, 2-6 V power supply*, 1.5 V bulb*

\subsubsection*{Hazards and Safety}
\begin{itemize}
\item{Concentrated sulphuric acid is corrosive to skin and clothes. Avoid contact with skin and eyes. Neutralize spills with sodium hydrogen carbonate (baking powder).}
\end{itemize}

\subsubsection*{Preparation}
\begin{enumerate}
\item{Set up a circuit with the electrodes, bulb and power source in series. [DIAGRAM] Test the circuit by touching the electrodes and confirming that the bulb lights.}
\item{Prepare about 100 mL of approximately 1 M solutions sulphuric acid and citric or acetic acid.}
\end{enumerate}

\subsubsection*{Activity Procedure}
\begin{enumerate}
\item{Explain to students how the circuit is set up and demonstrate electric flow by touching the electrodes together to light the bulb.}
\item{Instruct a student to come to the front and test the conductivity of pure water by putting both electrodes in a beaker of pure water.}
\item{On a piece of paper put a small pile of sodium chloride. Instruct one student to come up and test if the solid sample conducts electricity but putting both electrodes in the pile. Note: Make sure the electrodes do not touch each other.}
\item{Pour the sodium chloride into the pure water and stir until dissolved. Instruct the student to now test the salt solution for conductivity.}
\item{Clean the electrodes with distilled water.}
\item{Repeat the test with citric acid crystals, first testing the conductivity of the solid crystals and then adding to pure water and testing the solution.}
\item{Clean the electrodes with distilled water.}
\item{Repeat the test with sugar crystals, first testing the conductivity of the solid crystals and then adding to pure water and testing the solution. Guide students to discuss their observations for the three solutions.}
\item{In one beaker put the 1 M sulphuric acid, in another beaker put 1 M citric acid. Put the electrodes first in the sulphuric acid. Rinse the electrodes and then put them in the citric acid solution. Note the difference in the brightness of the bulb.}
\end{enumerate}

\subsubsection*{Results and Conclusion}
The sodium chloride is a strong electrolyte. The bulb will light and will be bright.
Citric acid is a weak electrolyte. The bulb will light but it will be dimmer than in the sodium chloride solution.
Sugar is a non-electrolyte. The bulb will not light.
The bulb should burn brighter in the sulphuric acid solution than in the citric acid solution even though they are the exact same concentration. This is because the sulphuric acid dissociates completely, you have 1 M solution of hydrogen ions. The citric acid is weak, thus will only partially dissociate.

\subsubsection*{Clean Up Procedure}
\begin{enumerate}
\item{Collect all the used materials, cleaning and storing items that will be used later. No special waste disposal is required.}
\end{enumerate}

\subsubsection*{Discussion Questions}
\begin{enumerate}
\item{Classify the substances tested as strong electrolyte, weak electrolytes or non-electrolytes.}
\item{How does this experiment demonstrate the difference between a strong electrolyte and a weak electrolyte?}
\item{Explain using a diagram why solid sodium chloride does not conduct electricity while sodium chloride solution does.}
\end{enumerate}

\subsubsection*{Notes}
Water does not need to be distilled, but prior to the experiment you should confirm that your tap water is not hard enough to conduct electricity.
If time and resources are available, different electrolytes and non electrolytes can be tested such as.
Strong electrolytes: copper (II) chloride, potassium iodide, magnesium sulphate, sodium carbonate
Weak electrolytes: acetic acid (vinegar), lemon juice
Non-electrolytes: ethanol, kerosene


\subsection{Conservation of Energy}

\subsubsection*{Learning Objectives}
\begin{itemize}
\item{To explain the law of conservation of energy.}
\item{To demonstrate the conversion of chemical energy to electrical energy.}
\end{itemize}

\subsubsection*{Background Information}
Chemical energy is potential energy stored in the bonds of chemical compounds. Electrical energy is energy from the movement of charge. It is possible to convert from chemical energy to electrical energy and vise versa.

\subsubsection*{Materials}
copper (II) sulphate, zinc plate*, copper metal*, sand paper

\subsubsection*{Preparation}
\begin{enumerate}
\item{Prepare a 2 M solution of copper (II) sulphate.}
\item{Clean the solid piece of copper and zinc using steel wool or sand paper.}
\end{enumerate}

\subsubsection*{Activity Procedure}
\begin{enumerate}
\item{Connect the following materials in a series: zinc cathode-connecting wire-ammeter-copper anode. [DIAGRAM]}
\item{Dip the zinc and copper electrodes into the copper (II) sulphate solution.}
\item{Read the current on the ammeter.}
\end{enumerate}

\subsubsection*{Results and Conclusion}
When the circuit is connected a current of 0.06 A is indicated on the ammeter. The current on the ammeter is likely to read 0.06A but may vary according to the concentration of solution and size of electrodes. The electric current produced indicates that the chemical energy inherent in the electrodes and the electrolyte solution is converted to electrical energy.

\subsubsection*{Clean Up Procedure}
\begin{enumerate}
\item{Collect all the used materials, cleaning and storing items that will be used later.}
\item{Copper sulphate solution can be stored for later use.}
\end{enumerate}

\subsubsection*{Discussion Questions}
\begin{enumerate}
\item{Is it possible to convert electrical energy to chemical energy? How?}
\item{What energy changes occur when: (a) Iron or steel is electroplated with chromium in industry? (b) Lightning strikes in the sky? (c) Water changes to steam?}
\end{enumerate}

\subsubsection*{Notes}


\subsection{Mechanism of Electrolysis}

[INSERT ACTIVITY HERE]

\subsection{Electrolysis: Effect of Electrolyte}

\subsubsection*{Learning Objectives}
\begin{itemize}
\item{To set up electrolytic cells using different electrolytes.}
\item{To explain how choice of electrolyte can affect the products of electrolysis.}
\end{itemize}

\subsubsection*{Background Information}
Electrolysis is the use of electricity to drive an otherwise unfavourable electrochemical reaction. When electricity is passed through an electrolyte, ions move towards the electrode which attracts them strongest. At each electrode there is an electrochemical reaction which creates a product at that electrode. The nature of ions present in the electrolyte solution affect which products are formed at the cathode and anode.

\subsubsection*{Materials}
 copper (II) sulphate*, potassium iodide solution, magnesium sulphate*, beaker*, carbon electrodes*, connection wire, masking tape, 3 to 6V power source*, sand paper

\subsubsection*{Preparation}
\begin{enumerate}
\item{Prepare a solution of copper (II) sulphate by adding approximately one teaspoon of copper (II) sulphate crystals to about 200 mL of water (or reuse the copper sulphate solution from the Electrolysis: Effect of Electrode experiment)}
\item{Prepare about 200 mL of 10\% of potassium iodide solution.}
\item{Prepare a solution of magnesium sulphate by dissolving approximately one teaspoon of crystals in about 200 mL of water.}
\item{Use wire to connect one carbon electrode to either end of the power supply. [electrode diagram]}
\end{enumerate}

\subsubsection*{Activity Procedure}
\begin{enumerate}
\item{In one clear beaker put copper (II) sulphate solution}
\item{Clean the electrodes using sand paper and place them into the beaker of copper sulphate, making sure that the electrodes do not touch each other.  Have students record their observations. After 3 minutes, remove the electrodes and allow students to examine them carefully.}
\item{Clean the electrodes well using water and sand paper. Repeat steps 1 and 2, this time with potassium iodide solution instead of copper sulphate.}
\item{Repeat with magnesium sulphate solution. Make sure to clean the electrodes well each time and record all of your observations.}
\end{enumerate}

\subsubsection*{Results and Conclusion}
When copper sulphate is electrolysed solid brown copper metal will form at the cathode while at the anode oxygen gas will form. This gas can be collected and tested using a glowing splint.
When potassium iodide is electrolysed, hydrogen gas will form at the cathode while at the anode dark iodine solid will form.
When magnesium sulphate is electrolysed, hydrogen gas will form at the cathode while at the anode oxygen gas will form.

\subsubsection*{Clean Up Procedure}
\begin{enumerate}
\item{Collect all the used materials, cleaning and storing items that will be used later. Electrodes should be cleaned well using sand paper and water.}
\item{Copper (II) sulphate solution that is still blue should be kept in a bottle "copper (II) sulphate" to be used in future experiments. Copper (II) sulphate solution that has lost its colour is now a solution of dilute sulphuric acid. Label and store for later use.}
\item{Potassium iodide solution now contains iodine. Ascorbic acid can be used to reduce the iodine back to iodine ions and the solution can be used again.}
\end{enumerate}

\subsubsection*{Discussion Questions}
\begin{enumerate}
\item{Write half reactions for the reactions occurring at the anode and cathode for each of the electrolytes tested.}
\item{What effect did changing the electrolyte have on the products of the electrolysis?}
\item{Explain why changing the electrolyte in this experiment changed the products formed.}
\end{enumerate}

\subsubsection*{Notes}
The products of electrolysis depend on three conditions:
1. Concentration of the electrolyte solution
2. Nature of the electrode (especially anode)
3. Position of the ions in the electrochemical series
DIAGRAM OF ELECTROCHEMICAL SERIES
This experiment examines the third factor. Ions lower in the electrochemical series will be discharged in preference to ions higher in the series. For example, SO42- ions will remain in solution while OH- is dischared. H+ ions will be discharged in preference to K+ ions.
In this experiment because water is present, OH- and H+ ions are available in each solution. The salt dissolved provides competing ions. When copper sulphate solution is present, copper is deposited at the cathode in preference to hydrogen gas because it is lower in reactivity series than hydrogen. On the other hand, when potassium iodide is electrolysed, the potassium remains in solution while H+ ions are reduced at the cathode creating hydrgen gas.
If halogen ions compete with hydroxide ions, the halogen ions will be discharged at the anode if they are in high enough concentration.


\subsection{Electrolysis: Effect of Electrode}

\subsubsection*{Learning Objectives}
\begin{itemize}
\item{To set up an electrolytic cell.}
\item{To explain how choice of electrode can affect the product of electrolysis.}
\end{itemize}

\subsubsection*{Background Information}
Electrolysis is the use of electricity to drive an otherwise unfavourable electrochemical reaction. When electricity is passed through an electrolyte, ions move towards the electrode which attracts them strongest. At each electrode there is an electrochemical reaction which creates a product at that electrode. The material that the electrode is made of can affect which product is formed at that electrode.

\subsubsection*{Materials}
 copper (II) sulphate*, beaker*, carbon electrodes*, copper electrodes*, connection wire, masking tape, 3 to 6V power source*, sand paper

\subsubsection*{Preparation}
\begin{enumerate}
\item{Prepare a solution of copper (II) sulphate by adding approximately one teaspoon of copper (II) sulphate crystals to about 200 mL of water.}
\item{Use wire to connect one electrode to either end of the power supply. [DIAGRAM]}
\item{Use sand paper and water to clean the electrodes.}
\end{enumerate}

\subsubsection*{Activity Procedure}
\begin{enumerate}
\item{In one clear beaker put copper (II) sulphate solution.}
\item{Put both carbon electrodes into the beaker of copper sulphate, making sure that the electrodes do not touch each other.}
\item{Identify the cathode and anode. Record observations about what happens at each electrode. After 3 minutes remove the electrodes from solution and allow students to examine them carefully. Have students record their observations.}
\item{Repeat the experiment using the copper electrodes instead of the carbon electrodes. Guide students to observe the appearance of both electrodes before and after the electrolysis.}
\end{enumerate}

\subsubsection*{Results and Conclusion}
(1) When copper (II) sulphate is electrolysed using the carbon electrodes the blue colour should fade as copper ions are deposited as copper metal on the cathode.  At the anode gas bubbles will be observed forming. The gas produced is oxygen. This can be proven if it is collected in a test tube and tested with a glowing splint (oxygen will relight the glowing splint).
(2) When copper (II) sulphate is electrolysed using copper electrodes the solution remains blue. Copper ions are deposited as metal at the cathode and at the anode reduction of the copper regenerates the ions in solution. After some time, it should be observed that the anode becomes smaller as the copper is oxidized while the anode becomes larger as the copper ions are deposited (reduced). Concentration and pH of the solution should remain unchanged.

\subsubsection*{Clean Up Procedure}
\begin{enumerate}
\item{Collect all the used materials, cleaning and storing items that will be used later. Electrodes should be cleaned with sand paper and water. No special waste disposal is required.}
\item{Copper (II) sulphate solution that is still blue should be kept in a bottle "copper (II) sulphate" to be used in future experiments.}
\item{Copper (II) sulphate solution that has lost its colour is now a solution of dilute sulphuric acid. Label and store for later use.}
\end{enumerate}

\subsubsection*{Discussion Questions}
\begin{enumerate}
\item{Write half reactions for the reactions occurring at the anode and cathode (a) for the carbon electrodes (b) for the copper electrodes.}
\item{What effect did changing the electrodes from carbon to copper have on the products of the electrolysis?}
\item{Explain why changing the electrode in this experiment changed the products formed.}
\item{Explain which products you would expect to see in an experiment with (a) a carbon anode and a copper cathode and (b) a copper anode and a carbon cathode.}
\end{enumerate}

\subsubsection*{Notes}



\subsection{Faraday's First Law of Electrolysis}

\subsubsection*{Learning Objectives}
\begin{itemize}
\item{To explain the meaning and significance of Faraday's First Law of Electrolysis.}
\item{To verify Faraday's First Law experimentally.}
\end{itemize}

\subsubsection*{Background Information}


\subsubsection*{Materials}
magnesium sulphate*, water, 2 burettes of 50 mL capacity, two clean carbon electrodes*, connecting wires*, masking tape, 4-6V power supply*, ammeter, stopwatch*, 1.5 L plastic water bottle, 10 L bucket

\subsubsection*{Preparation}
\begin{enumerate}
\item{Cut the bottom 6 cm from the bottle of the water bottle to form a short beaker.}
\item{Use tape to connect the power supply, ammeter, and 2 carbon electrodes in the following circuit: (ILLUSTRATION)
Leave one of the wires unconnected from the ammeter. During the activity you will connect it.}
\end{enumerate}

\subsubsection*{Activity Procedure}
\begin{enumerate}
\item{Dissolve 5 spoons of magnesium sulphate in about 300 mL water.}
\item{Close the stopcock of each burette and fill both with magnesium sulphate solution.}
\item{Insert one electrode into the top of each burette.}
\item{Add the rest of the magnesium sulphate solution to the beaker.}
\item{Turn the bucket upside-down and place it next to the beaker so that the two containers touch.}
\item{Place your thumb over the open end of one burette. Invert the burette so that the end with the stopcock is pointing up and lower the burette into the beaker. Do not remove your thumb until the open end of the burette it well below the top of the solution in the beaker. The solution in the burette should remain in the burette as long as the stopcock is closed.}
\item{Tape the burette to the bucket of water. Keep the open end under the solution in the beaker! The burette should be pointed almost straight up. (ILLUSTRATION)}
\item{Repeat the previous two steps for the other burette.}
\item{Connect the remaining wire to the ammeter to complete the circuit and immediately start the stopwatch. Bubbles should form at both electrodes and all gas produced should enter into the burettes for collection.}
\item{Record the reading on the ammeter and the volume of gas produced at the cathode after every minute for ten minutes.}
\item{ADD CALCULATIONS IN THIS SECTION}
\end{enumerate}

\subsubsection*{Results and Conclusion}


\subsubsection*{Clean Up Procedure}
\begin{enumerate}
\item{The chemicals used in this experiment do not require any special disposal.}
\end{enumerate}

\subsubsection*{Notes}
If magnesium sulphate is unavailable, dilute sulphuric acid* may be used instead. This is more dangerous, however. If using sulphuric acid, save the final solution - it is still dilute sulphuric acid and may be used for other experiments. While other salts can be used as electrolytes, many will produce other products and are therefore not advised. The calculations for this experiment depend on knowing the amount of electricity passing in the system; therefore the ammeter is required. If an ammeter is unavailable, the teacher should still demonstrate electrolysis, and should still verify Faraday's Second Law.

\subsection{Faraday's Second Law}

\subsection{Electroplating}

\subsubsection*{Learning Objectives}
\begin{itemize}
\item{To explain the meaning and significance of electroplating.}
\item{To electroplate a metal with another metal.}
\end{itemize}

\subsubsection*{Background Information}
Electroplating is the process of coating a metal by depositing another metal through an electrolytic process. Electroplating is used to coat steel materials with other metals such as silver, gold or chromium. Electroplating is popular in automobile manufacture, jewelry making and the making of many materials used in hospitals and in the kitchen.

\subsubsection*{Materials}
copper (II) sulphate*, iron nail, sand paper, 4-6V  power source*, connection wires, masking tape, beakers*, carbon electrode*

\subsubsection*{Preparation}
\begin{enumerate}
\item{Clean an iron nail thoroughly using sand paper.}
\item{Connect the power source in series with the iron nail and the carbon electrode with the iron nail as the cathode (connected to the + end of the battery) and the carbon electrode as the anode (connected to the - end of the battery).}
\item{Prepare a 2 M solution of copper (II) sulphate and acidify it by adding some dilute sulphuric acid.}
\end{enumerate}

\subsubsection*{Activity Procedure}
\begin{enumerate}
\item{Insert the nail and the carbon electrode into the acidified copper (II) sulphate solution.}
\item{Observe and record the changes on both the electrodes after 30 minutes.}
\end{enumerate}

\subsubsection*{Results and Conclusion}
The nail will become coated in a thin layer of brown copper metals-we have electroplated iron with copper. At the carbon electrode (anode) oxygen gas will be formed.

\subsubsection*{Clean Up Procedure}
\begin{enumerate}
\item{The copper sulphate can be kept and labelled as "acidified copper sulphate" for future use.}
\item{The nail may be cleaned using sand paper.}
\item{Collect all the used materials, cleaning and storing items that will be used later.}
\end{enumerate}

\subsubsection*{Notes}
If the copper coat does not stick well to the nail, it may be necessary to soak it in dilute sulphuric acid and then cleaned with sand paper or steel wool prior to the experiment.
