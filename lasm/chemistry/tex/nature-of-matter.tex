\chapter{The Nature of Matter}

Chemistry is the study of matter. All of the higher lessons depend on students' understanding of the basics of matter. Much of the study of chemistry is based on three simple postulates:
\begin{itemize}
\item{Matter is composed of extremely small particles called molecules.}
\item{Attraction forces cause these molecules to stick together.}
\item{Thermal energy keeps these particles in constant motion, sometimes shaking them apart from each other.}
\end{itemize}

%The rest of the study of chemistry is based on the structure of the atom, the manner in which electrons attract other atoms to form chemical bonds, and the nature of these bonds.

\section{Concept of Matter}

Everything we see is made up of extremely tiny particles called molecules. A molecule is the smallest unit of a substance that retains the chemical properties of the substance. Emphasis should be given to the extremely small nature of molecules: in 1 litre of water there are more than 33,000,000,000,000,000,000,000,000,000 (`33 billion billion billion') molecules of water! Write this number on the board for students to think about.


\section{States of Matter}

Molecules are always in motion, although their movements are usually restricted by forces attraction to other molecules. The balance between thermal energy and intermolecular attraction determines the state of matter.
There are three states of matter: solid, liquid, and gas. A solid has fixed volume and fixed shape. A liquid has fixed volume but will take the shape of its container. A gas will take the volume and the shape of its container. A `vapour' is another word for gas often used for compounds that are normally liquid or solid at room temperature. For example, we might talk about `water vapour' or `zinc vapour,' but generally not `oxygen vapour.'


\subsection{Changes of State}
By changing the temperature of a substance, we can change the state it exists in. As a solid object is heated the molecules begin to shake faster and faster until they break free of the rigid solid substance and become a free-flowing liquid. This is called melting. A change of state is a physical change which involves a substance going from one state to another. 
((ILLUSTRATION))

Changes of state are something that students see in everyday life (a melting candle, drying clothes). This activity is good for getting students to think about these everyday changes in terms of the molecules which make up every substance. The activity demonstrates melting, solidifying, vaporization, and condensing. If enough candles are available, the students can do the first activity and make observations in small groups. Because of materials, time and safety it might be ideal to do the vapourization and condensing of water as a class demonstration and discussion.

\subsubsection*{Learning Objectives}
\begin{itemize}
\item{To describe the states of matter.}
\item{To demonstrate changes in state of various material.}
\end{itemize}

\subsubsection*{Materials}
candle, heating vessels*, water, condenser, knife, heat source

\subsubsection*{Hazards and Safety}
\begin{itemize}
\item{Never heat a tightly sealed container.}
\end{itemize}

\subsubsection*{Preparation}
\begin{enumerate}
\item{Fill a condenser with cold water.}
\end{enumerate}

\subsubsection*{Activity Procedure}
\begin{enumerate}
\item{Take a candle and light it using a match box. Have students make observations about any change they observe.}
\item{Put about 20 mL of water in the heating vessel and heat for 5-10 minutes. Have students make observations.}
\item{Put about 50 mL in a heating vessel and connect with the condenser. Heat until a liquid is noticed at the other end of the condenser. Collect this liquid. Have students make observations.}
\end{enumerate}

\subsubsection*{Results and Conclusion}
When the candle burns it changes from solid to liquid. When the candle cools the wax will harden, returning to solid state.
Boiling water is a change from a liquid to a vapour in gaseous state. Condensing water vapour changes it back from gas to liquid.

\subsubsection*{Clean Up Procedure}
\begin{enumerate}
\item{Collect all the used materials, cleaning and storing items that will be used later. No special waste disposal is required.}
\end{enumerate}

\subsubsection*{Discussion Questions}
\begin{enumerate}
\item{Explain what change of state is involved when burning candle wax and boiling water.}
\item{what happens when water is heated and passed through a condenser?}
\end{enumerate}

\subsubsection*{Notes}

\section{Physical and Chemical Changes}

Changes in state are an example of a physical change. A physical change is a change that only involves rearranging molecules. Chemists also study chemical changes, in which chemical bonds are broken and remade, and molecules change from one kind to another. The following activity may be used with students to illustrate the difference between physical and chemical changes.

\subsection{Physical and Chemical Change}

\subsubsection*{Learning Objectives}
\begin{itemize}
\item{SWBAT demonstrate physical and chemical changes of matter experimentally.}
\end{itemize}


\subsubsection*{Materials}
A piece of paper, sugar, spoon, bar magnet, two iron nails, steel wool, kerosene stove and match box.

\subsubsection*{Activity Procedure}
\begin{enumerate}
\item{Take a piece of paper and light it on fire using a match box. Have students record their observations.}
\item{Take a small amount of sugar into the spoon and heat until a clear chemical change is observed. Have students record their observations.}
\item{Take a bar magnet and rub one nail in one direction only. [DIAGRAM]}
\item{ Place the second (un-magnetized) nail hear the iron fillings . Repeat with the nail that has been rubbed with a magnet.}
\item{Take the magnetized nail and rub it again in the opposite direction and place in iron fillings observe what will happen.}
\end{enumerate}

\subsubsection*{Results and Conclusion}
Burning paper and burning sugar are examples of chemical changes. These means that chemical bonds are broken and remade and new products are formed. But when the nail is rubbed with the bar magnet no chemical bonds are broken or made. The nail becomes magnetized and thus attracts the iron fillings. This is an example of a physical change.

\subsubsection*{Clean Up Procedure}
\begin{enumerate}
\item{Collect all the used materials, cleaning and storing items that will be used later. No special waste disposal is required.}
\end{enumerate}

\subsubsection*{Discussion Questions}
\begin{enumerate}
\item{Explain the changes in the paper and sugar. Name the type of change.}
\item{Explain the process of rubbing a nail with a bar magnet and what happen when it was rerubbed?}
\item{Why did the two nail behave differently? Name the change that happened in the rubbed nail.}
\end{enumerate}

\subsubsection*{Notes}

\section{Compounds and Mixtures}

The mixture of iron and sulphur before heating in the above activity is a good example of a mixture. Mixtures can be separated by purely physical means - no chemical bonds need to be broken or made. After heating the mixture of iron and sulphur, a new chemical compound formed - chemical bonds were broken and new ones were made. This section deals with mixtures, where chemicals are mixed together but chemical reactions have not yet occurred.

\subsection{Solutions, Suspensions and Emulsions}
When a liquid is combined with another substance, three kinds of mixtures are possible: solution, suspension or emulsion. 
A solution is a mixture where the one component (solute) dissolves completely in the other (solvent) to form a transparent liquid. A suspension is formed when a small pieces of solid are suspended or dispersed throughout a liquid without dissolving in it. The solute particles in the suspension remain visible.  An emulsion is a fine dispersion of small droplets of one liquid in another in which it is not soluble or miscible. 
Solutions, suspensions and emulsions are something students encounter in every day life- salt water is a solution, river water is a suspension and milk is an emulsion.
This activity gives students a chance to make their own examples of each type of mixture and compare them side by side. This activity is best done by students in small groups so that each students gets a chance to make clear observations-it is hard to see the difference between a solution and a suspension from far away.

\subsubsection*{Learning Objectives}
\begin{itemize}
\item{To prepare examples of solution, suspension and emulsion.}
\item{Explain the properties of solutions, suspensions and emulsion.}
\end{itemize}

\subsubsection*{Materials}
Cooking oil, water, methylated spirit, test tubes*, beakers, plastic bottles with lids, copper (II) sulphate, clay soil.

\subsubsection*{Preparation}
\begin{enumerate}
\item{Grind the solid clay to make a fine powder.}
\end{enumerate}

\subsubsection*{Activity Procedure}
\begin{enumerate}
\item{Instruct students to mix equal volumes of cooking oil and water in a test tube and shake it vigorously.}
\item{Direct students to allow the solution in the test tube to settle. Have them record their observations.}
\item{Instruct students to add about a mL of methylated spirit to the test tube and again shake vigorously.}
\item{Students should again allow the test tube to settle and record their results. The mixture formed should be an emulsion.}
\item{Tell students to add a spoon of clay to a test tube of water and shake vigorously then set the test tube down and allow the solution to settle.}
\item{Direct students to add half of a spoonful of copper (II) sulphate to about 200 mL of water and shake vigorously until no more crystals are visible.}
\item{Instruct student to allow the mixture to sit for some minutes and record their observations.}
\end{enumerate}

\subsubsection*{Results and Conclusion}
In the first experiment oil and water clearly form two layers because they are immiscible liquids. Upon addition of methylated spirit and shaking, the two liquids form an emulsion. The emulsion is not transparent, you can not be seen in it.
The clay mixes with water to form suspension which after some time will settle to the bottom of the container. The suspension is not transparent and individual clay particles can clearly be seen.
The copper sulphate crystals dissolve in water completely to form a blue solution which does not settle no matter how long it is allowed to sit. The solution is transparent and individual particles can not be seen.

\subsubsection*{Clean Up Procedure}
\begin{enumerate}
\item{Collect the copper (II) sulphate solution, store in a labelled bottle and save for later use.}
\item{Collect all the used materials, cleaning and storing items that will be used later. No special waste disposal is required.}
\end{enumerate}

\subsubsection*{Discussion Questions}
\begin{enumerate}
\item{Did the oil and water separate after the addition of methylated spirit?}
\item{Can you see through the emulsion formed? Why?}
\item{Did the solid clay settle out of the clay mixture?}
\item{Can you see through the clay mixture?}
\end{enumerate}

\subsubsection*{Notes}
An emulsion can also be created by shaking a sodium hydroxide solution with cooking oil, however this is more dangerous than the procedure listed.

\section{Separation of Mixtures}

There are many methods of separating mixtures. Many are used in daily life. This topic is a good way to teach students that the chemistry they use in the classroom has application when they go home. This section presents different methods for separating mixtures and an activity to use for teaching each. Many of these methods will be important for more complicated experiments, so both students and teachers should be familiar with them.

\subsection{Decantation}
Decantation is an easy way to separate a suspension. Leave the suspension to settle, so the solid falls to the bottom of the container. Then, the liquid can be poured off the top. This method is applied in the home, for example, when rice is rinsed and the excess water is poured off. Students can perform this activity in a very short time; it is best combined with the filtration activity.


\subsection{Filtration}

Filtration is a more complicated but more effective method for separating a suspension. A suspension is filtered by passing it through a material with holes so small that only the liquid can pass through. The solid is left behind.

\subsection{Evaporation}

Evaporation is a simple method to separate solutions where one chemical is most more volatile than the other, for example a solution of salt in water.

\subsection{Separation of Mixtures: Decantation, Filtration and Evaporation}
This activity combines the principles of decantation, filtration and evaporation to produce a pure sample of salt from a mixture of salt, water, sand and stones. This activity is best done by student in small groups of 2-4.

\subsubsection*{Learning Objectives}
\begin{itemize}
\item{SWBAT separate  a mixture using the decantation, filtration, evaporation.}
\end{itemize}

\subsubsection*{Materials}
Water, sand, salt, small pebbles, beakers*, funnel*, clean cloth and a plastic bottle
\subsubsection*{Preparation}
\begin{enumerate}
\item{Make a mixture by combing water, salt, sand and pebbles an a plastic bottle. Close the lid and shake}
\end{enumerate}
\subsubsection*{Activity Procedure}
\begin{enumerate}
\item Shake the bottle well and observe the solution. Allow the bottle to sit for a few minutes so that the pebbles and some dirt settle to the bottom.
\item Pour the solution into a clean beaker, leaving the pebbles and dirt in the bottom of the bottle. You should now have a sample of water with some dirt particles.
\item Put the clean cloth in the filter and pour the dirty water into the filter. Make sure that all of the water passes through the cloth and not around the edges. Collect the water in a clean beaker. If dirt is still present in the water, filter it again through a fresh piece of cloth.
\item Put a small amount of this water into a clean beaker and leave it in the sun for a few hours. Make sure it is put in a place that is protected from wind so dirt does not enter. Once all of the water has evaporated collect the beaker and examine its contents.
\end{enumerate}

\subsubsection*{Results and Conclusion}
Decantation should separate the large pebbles and pieces of rocks. The filtration should remove all of the dirt particles and leave a clear salt water solution. After evaporation, salt crystals should remain in the beaker.

\subsubsection*{Clean Up Procedure}
\begin{enumerate}
\item{Collect all the used materials, cleaning and storing items that will be used later. No special waste disposal is required.}
\end{enumerate}

\subsubsection*{Discussion Questions}
\begin{enumerate}
\item{Give 3 examples each of decantation and filtration being used in daily life.}
\item{One problem with the method of evaporation is that the liquid is lost and cannot be collected again. Which method of separation could be used to separate salt water if pure water was the desired product.}
\end{enumerate}


\subsection{Construction of a Sand Filter}
In many villages there is no access to clean drinking water. People are forced to drink water that is unsafe or use a lot of fuel for boiling water. Many students are already aware of the process of boiling and filtering water to make it safe for drinking. On a larger scale, sand filters can be used to purify water for a group of people or even a village. When built properly on a large scale sand filters are even effective for removing micro-organisms which cause disease. In this activity students build small-scale sand filters and attempt to filter their own dirty water.  This activity is best done by students in small groups.

\textbf{NOTE}: The sand filters built in the laboratory are only an example, they are \textit{not} effective for removing bacteria from dirty water.
\subsubsection*{Learning Objectives}
\begin{itemize}
\item{To demonstrate the purification of water by using a sand filter.}
\end{itemize}

\subsubsection*{Materials}
Fine sand, coarse sand, small pebbles, large pebbles, charcoal, empty water bottle, beaker, screen material to act as a seive, and dirty water.

\subsubsection*{Preparation}
\begin{enumerate}
\item{Use water to rinse all of the dirt off of the pebbles.}
\item{Put all of the sand in a sieve and shake to remove excess dirt. In the sieve, rinse the pebbles with water.}
\item{In a beaker, rinse the coarse sand with clean water and decant off the dirty water. Repeat until the water decanted is clear.}
\item{Cut the bottom off of a water bottle so that is it shaped like a funnel.}
\item{Invert the water bottle and fill the bottom with large pebbles, making sure that the botom pebble is larger than the hole in the bottle so that it doesnt fall out. Put a layer of smaller pebbles on top of the large pebbles.}
\item{Put the coarse sand particles on top of the pebbles, followed by charcoal and then small sand particles at the top (see diagram).}
((ILLUSTRATION))
\item{Run clean water through the filter until it comes out clear at the bottom.}
\end{enumerate}

\subsubsection*{Activity Procedure}
\begin{enumerate}
\item{Pour some dirty water into the top of the sand filter.}
\item{Collect the water in a clean beaker from the bottom of the filter.}
\end{enumerate}

\subsubsection*{Results and Conclusion}
When dirty water passes through sand particles, impurities are trapped and remain above. The smallest particles and some micro-organisms are stopped by the charcoal layer. Ideally, the water coming out of the bottom of the filter should be clear. 

\subsubsection*{Clean Up Procedure}
\begin{enumerate}
\item{Collect all the used materials, cleaning and storing items that will be used later. No special waste disposal is required.}
\end{enumerate}

\subsubsection*{Discussion Questions}
\begin{enumerate}
\item{How can you distinguish water before purification and water after purification?}
\item{List 3 substances which can make water impure.}
\item{Why is charcoal used in this process?}
\item{Why is water purification important?}
\end{enumerate}



\subsection{Separation of Mixtures: Simple Distillation}
Simple distillation is a more complicated method for separating solutions that allows the liquid to be collected. This method involves vaporizing the liquid, then condensing it by cooling the vapor, and collecting the resulting liquid.  The following activity involves a slightly complicates set-up so it is best done as a demonstration. If food colour is added to the water, even students in the back will be able to see the process.
\subsubsection*{Learning Objectives}
\begin{itemize}
\item{To be able to separate a solution by application of simple distillation}
\end{itemize}

\subsubsection*{Materials}
Condenser, water, food colouring, heating vessel*, beaker, and heat source*.

\textbf{Notes}:

\begin{itemize}
\item{If food colouring powder is not available it is also possible to use a coloured compound like potassium permanganate or a salt. If a salt is used, it might be difficult to show that the distallate is salt free.}
\item{If a condenser is not available it is still possible to do this experiment. Connect a long delivery tube to a stopper on the heating vessel. The air temperature surrounding the tube will be cool enough to cause some water to condense which can then be collected in a beaker. If this method is used, a large amount of water will also be lost as steam.}
\end{itemize}

\subsubsection*{Preparation}
\begin{enumerate}
\item{Make a coloured solution by adding a small amount of colour to a sample of water.}
\item{Put cold water into the condenser.}
\end{enumerate}

\subsubsection*{Activity Procedure}
\begin{enumerate}
\item{Heat the container holding coloured water over the heat source for some time.}
\item{Collect any water out of the condenser and compare it to the original solution.}
\end{enumerate}

\subsubsection*{Results and Conclusion}
When the solution is heated, the water is vaporized while the coloured particles remain in solution. As the water vapour passes through the condenser tube is it cooled and condenses back to liquid form. The water will drip out of the end of the tube and will be pure, that is, containing no dissolved solutes. The final solution should contain no colour.

\subsubsection*{Clean Up Procedure}
\begin{enumerate}
\item{Collect all the used materials, cleaning and storing items that will be used later. No special waste disposal is required.}
\end{enumerate}

\subsubsection*{Discussion Questions}
\begin{enumerate}
\item{Why does the final solution appear different than the initial solution?}
\item{In what situation is simple distillation a better method of separation than evaporation?}

\end{enumerate}

\subsection{Separation of Mixtures: Chromatography}
Chromatography is the separation of a mixture by passing it in solution or suspension through a medium in which the components move at different rates. This method can be used to separate different coloured pigments in a coloured ink. Because each pigment has a different molecular shape, they can be dissolved in a solvent and separated. As the solvents moves through a medium (such as paper) the different molecules have different attraction to the paper and the solvent. Molecules with a higher attraction to the paper will move slower. After some time, the molecules will be separated, giving distinct bands of colour.

In this activity students are able to separate the different colour pigments present in the ink of a marker or pen. This activity should be done by students in small groups.
\subsubsection*{Learning Objectives}
\begin{itemize}

\item{To separate colours using paper chromatography.}

\end{itemize}

\subsubsection*{Materials}
Piece of white paper, cotton wick (utambi wa jiko), petri dish*, colourless methylated spirit, small beaker* and one red marker or ball pen.

\subsubsection*{Preparation Procedure}
\begin{enumerate}
\item{Cut piece of paper to a size a little bigger than the top of the beaker. Put a small hole in the center of it.}
\item{Prepare a wick that is long enough to reach to the bottom of the beaker.}
\end{enumerate}

\subsubsection*{Activity Procedure}
\begin{enumerate}
\item{Pour the colourless spirit into the beaker.}
\item{Make a hole in the center of the piece of paper you have prepared and insert the thread through the hole. ((ILLUSTRATION NEEDED)) }
\item{Using a red marker/ball pen draw a circle around the hole and put the piece of paper on the beaker containing methylated spirit and make sure the end of the wick is immersed in the liquid.}
\item{Leave it to sit until the spirit has caused separation of the colours (this should be about 10 minutes).}
\item{Repeat the experiment using different markers or pens.}
\end{enumerate}

\subsubsection*{Results and Conclusion}
Methylated spirit acts as a solvent. The solvent climbs up the wick and then spreads across the paper.  When the solvent reaches the ink, the different pigments dissolve and move at different speeds and thus are separated. The colours observed for a red marker might be yellow, pink and purple.

\subsubsection*{Clean Up Procedure}
\begin{enumerate}
\item{Collect all the used materials, cleaning and storing items that will be used later. No special waste disposal is required.}
\end{enumerate}

\subsubsection*{Discussion Questions}
\begin{enumerate}
\item{Explain the use of the wick, methylated spirit and marker pen to this experiment.}
\item{How many colors are there on the piece of paper at the end of the experiment.}
\end{enumerate}


\subsection{Separation of Mixtures: Layer Separation}
Layer separation is a method of separation used to separate two liquids that are immiscible. Immiscible means that the liquids are mutually insoluble and thus form two distinct layers when mixed. The following activity give students a chance to separate a mixture of kerosene. The activity is very short and can be done in combination with another separation activity in a single class period.

\subsubsection*{Learning Objectives}
\begin{itemize}
\item{To separate two immiscible liquids using layer separation.}
\end{itemize}

\subsubsection*{Materials}
Kerosene, water, plastic water bottle, IV giving set, super glue

\subsubsection*{Preparation}
\begin{enumerate}
\item{Make a separatory funnel. ((ILLUSTRATION))}
\end{enumerate}

\subsubsection*{Activity Procedure}
\begin{enumerate}
\item{Measure approximately equal volume of water and kerosene using the small beaker and pour them in the separating funnel and shake them well.}
\item{Leave the mixture for some time to settle and run off the lower layer.}
\end{enumerate}

\subsubsection*{Results and Conclusion}
When water and kerosene are mixed and left to settle the two liquids form two immiscible layers. Kerosene forms the upper layer while water forms the lower layer because water is denser than kerosene. Water can be slowly drained from the separatory funnel and collected in a clean beaker.

\subsubsection*{Clean Up Procedure}
\begin{enumerate}
\item{Collect all the used materials, cleaning and storing items that will be used later. Kerosene should not be disposed. Put it in a well-labelled bottle and store for later use.}
\end{enumerate}

\subsubsection*{Discussion Questions}
\begin{enumerate}
\item{What are immiscible liquids?}
\item{In a kerosene/water mixture, which liquid is on top? Why?}
\end{enumerate}

\subsection{Separation of Mixtures: Solvent Extraction}

[INSERT ACTIVITY HERE]
