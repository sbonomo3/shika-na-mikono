\chapter{Acids and Bases}

Acids and bases are both categories of reactive chemicals. Acids react by giving the hydrogen ion to other compounds. Bases react by taking the hydrogen ion from other compounds. Reactions between acids and bases therefore involve the transfer of a hydrogen ion from the acid to the base.

This chapter explains how to prepare indicators to test whether a given compound is an acid, a base, or neither. Then this chapter gives an activity where students use indicators to identify common acids and bases. Finally, this chapter considers the reaction of acids and bases with a variety of other compounds.

\subsection{Preparation of Indicators}

Acid-base indicators are chemicals that are different colours in acids and bases. Methyl orange, for example, is red in acid and yellow in base. Indicators are very useful, because they tell us if something is an acid or a base. They are also used during reactions to show when a solution changes from acidic to basic or basic to acidic, as in volumetric analysis.

There are two types of indicators: liquid indicators and paper indicators. Liquid indicators are added dropwise until a distinct colour is observed. Paper indicators are made from liquid indicators and are used to either quickly test solutions or to test substances like gases that are not easily tested with liquid indicators.

In this activity students identify local flowers that may be used to produce acid-base indicators. The teacher should guide them to identify the best flowers, and together they should produce both liquid and paper indicators. These indicators may then be used for future experiments.

\subsubsection*{Objectives}
\begin{itemize}
\item{To identify local flowers that function as acid-base indicators.}
\item{To prepare a liquid acid-base indicator from local flowers.}
\item{To prepare both blue and red indicator paper from local flowers.}
\end{itemize}

\subsubsection*{Materials}
Pink, red, orange, and purple flowers, water, beakers (many), colourless methylated spirits, empty water bottle with cap, cooking pan (sufuria), heat source, white A4 paper, pair of scissors, 5M sulphuric acid, sodium hydroxide

\subsubsection*{Hazards and Safety}
\begin{itemize}
\item{((battery acid))}
\item{((sodium hydroxide))}
\end{itemize}

\subsubsection*{Preparation}
\begin{enumerate}
\item{Fill a 1.5 L water bottle half way with water. Add about 100 mL of battery acid. Label the bottle "dilute sulphuric acid."}
\item{Fill a second 1.5 L water bottle half way with water. Add 4 spoons of sodium hydroxide. Label this solution "sodium hydroxide solution."}
\end{enumerate}

\subsubsection*{Activity Procedure}
\begin{enumerate}
\item{Collect many pink, red, orange, and purple flowers flowers.}
\item{Crush each different flower in a small amount of water to obtain an extract of its pigment. The water should become the colour of the flower.}
\item{Divide the extract from each flower into three portions.}
\item{To one portion of extract from each flower, add about 1 mL of sulphuric acid solution. Note any colour change.}
\item{To the second portion of extract from each flower, add about 1 mL of sodium hydroxide solution. Note any colour change.}
\item{Place all three portions of the extracts next to each other to observe the three colours: the extract in acidic solution, the extract in neutral solution, and the extract in basic solution.}
\item{Select the best three flowers of all of the samples present. The best flowers will be available in a large quantity near the school and produce a very large colour difference in acid and in base.}
\item{Collect a large quantity of the three best flowers.}
\item{Cush these flowers in water to make approximately 1 litre total of extract of each flower. Divide this extract into three portions.}
\item{To the first portion, add an equal volume of colourless methylated spirits. Label the bottle "Indicator Solution."}
\item{Pour the second portion into a pot and start heating on a stove.}
\item{Leave the third portion at room temperature.}
\item{Cut A4 paper into pieces approximately 1 cm by 4 cm. Prepare about 100 pieces total.}
\item{Put half of the papers into the room temperature extract.}
\item{Put the other half of the papers into the pot with the flower extract and heat until boiling. Leave the papers in the boiling extract for at least ten minutes. Then take them out to dry.}
\item{After the papers dry, test one not-boiled and one boiled strip from each flower in both acid and in base. Select the best papers for using in place of litmus paper. The best substitute for red litmus will change colour in base but not in acid. The best substitute for blue litmus will change colour in acid but not in base.}
\end{enumerate}

\subsubsection*{Clean Up Procedure}
\begin{enumerate}
\item{Remove the unwanted materials and dip them in a pit latrine.}
\item{Wash hands with clean water and soap.}
\end{enumerate}

\subsubsection*{Discussion Questions}
\begin{enumerate}
\item{How do you think the first indicators were discovered?}
\end{enumerate}

\subsubsection*{Notes}
One very effective flower for making indicator solution and indicator paper is rosella. The extract is red in acid and blue or green in base. The papers act as red litmus when made in cool extract and as blue litmus when made in boiling extract. In general redish flowers are particularly useful for making indicators because most contain a pigment called anthrocyanins that are redish in acid and bluish in base. These pigments are much more complicated than methyl orange and each is different, so it is not possible to give students a chemical formula for these compounds.

\subsection{Acids and Bases in Daily Life}

\subsubsection*{Learning Objectives}
\begin{itemize}
\item{To site natural sources of acids and bases.}
\end{itemize}

\subsubsection*{Materials}
Citric acid*, ash from burning charcoal or a banana plant*, pure water, citrus fruits, beakers*, droppers*, and Litmus Paper*

\subsubsection*{Preparation}
\begin{enumerate}
\item{Squeeze lemon juice into a beaker.}
\item{Mix ashes with water in a beaker.}
\end{enumerate}

\subsubsection*{Activity Procedure}
\begin{enumerate}
\item{Arrange students into groups of 4-6.  Give each group 5 beakers, litmus paper, and a dropper.}
\item{Instruct students to add a few drops of lemon juice to the first beaker. Then instruct students to dip red and blue litmus paper into the beaker.}
\item{Instruct students to add a few drops of vinegar into the second beaker. Then instruct students to dip red and blue litmus paper into the beaker.}
\item{Instruct students to add a few drops of the wood ash solution into the third beaker. Then instruct students to dip red and blue litmus paper into the beaker.}
\item{Instruct students to add a few drops of the ash solution into the fourth beaker. Then instruct students to dip red and blue litmus paper into the beaker.}
\item{Instruct students to add a few drops of the soda ash solution into the fifth beaker. Then instruct students to dip red and blue litmus paper into the beaker.}
\end{enumerate}

\subsubsection*{Results and Conclusion}
Lemon juice and vinegar both turn blue litmus paper red. They have no effect on red litmus paper. Ash solution turns red litmus paper blue and has no effect on blue litmus paper.

\subsubsection*{Clean Up Procedure}
\begin{enumerate}
\item{Collect all the used materials, cleaning and storing items that will be used later.  No special waste disposal is required.}
\end{enumerate}

\subsubsection*{Discussion Questions}
\begin{enumerate}
\item{Which of these substances are acidic? How do you know?}
\item{Which of these substances are basic? How do you know?}
\end{enumerate}

\subsubsection*{Notes}
Citrus fruits contain citric acid while vinegar is a dilute solution of ethanoic (acetic) acid. Ashes contain metal oxides, hydroxides, and carbonates - these produce an alkaline solution in water. Rosella contains pigments called anthrocyanins that are red in acid and blue in base.


%-------

\subsection{Reaction of Acids}

\subsubsection*{Learning Objectives}
\begin{itemize}
\item{To demonstrate the reactions of acids with various inorganic compounds}
\end{itemize}

\subsubsection*{Materials}
test tubes*, test tube rack*, spatulas*, beakers*, dilute weak acid*, copper wire, flame, calcium oxide*, calcium hydroxide*, sodium hydrogen carbonate*

\subsubsection*{Hazards and Safety}
\begin{itemize}
\item{((rxn in test tube))}
\end{itemize}

\subsubsection*{Preparation}
\begin{enumerate}
\item{Pass a long piece of copper wire slowly through a flame until all of the surface has oxidized. The copper should become dark in colour. This colour is caused by a layer of copper oxide formed by reaction with air at high temperature.}
\item{Cut the oxidixed copper wire into 5 cm segments, one for each group of students}
\end{enumerate}

\subsubsection*{Activity Procedure}
\begin{enumerate}
\item{Divide the students into groups of 4 - 6 students each}
\item{Instruct each group to take four test tubes, one test tube rack, three petri dishes, three spatulas, and one piece of oxidized wire.}
\item{Provide each group with a very small amount each of sodium hydrogen carbonate, calcium hydroxide, and calcium oxide. Put each sample in a separate petri dish.}
\item{Instruct students to add three drops of indicator solution to each test tube. The students should observe that all of the tubes contain basic materials except the one with copper oxide.}
\item{Provide each group with about 20 mL of dilute weak acid.}
\item{Instruct the students to pour about one quarter of their weak acid into the first test tube (calcium oxide). They should observe the reaction and any colour change in the indicator. They should also observe any change in temperature by holding the outside of the test tube.}
\item{Instruct the students to pour another quarter of their weak acid into the second test tube (calcium hydroxide). They should observe the reaction and any colour change in the indicator. They should also observe any change in temperature by holding the outside of the test tube.}
\item{Instruct the students to pour another quarter of their weak acid into the third test tube (copper oxide). They should observe the reaction and any colour change in the indicator. After one minute, they should observe any change in the colour of the metal.}
\item{Instruct the students to pour the last quarter of their weak acid into the fourth test tube (sodium hydrogen carbonate). They should observe the reaction and any colour change in the indicator. They should also observe any change in temperature by holding the outside of the test tube.}
\end{enumerate}

\subsubsection*{Results and Conclusion}
In the fourth test tube, the students should observe rapid effervescence - bubbles should form and the solution may bubble out of the top of the test tube. This effervesence is caused by the release of carbon dioxide from the carbonate by the action of the acid. The indicator will also change colour from basic to acidic.

\subsubsection*{Clean Up Procedure}
\begin{enumerate}
\item{((no waste treatment))}
\item{((clean up))}
\end{enumerate}

\subsubsection*{Notes}

\subsection{Reaction of Bases}

[INSERT ACTIVITY HERE]
