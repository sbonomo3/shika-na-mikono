\chapter{Chemical Change and Chemical Reactions}
\section{Physical and Chemical Changes}

Changes in state are an example of a physical change. A physical change is a change that only involves rearranging molecules. Chemists also study chemical changes, in which chemical bonds are broken and remade, and molecules change from one kind to another. The following activity may be used with students to illustrate the difference between physical and chemical changes.

\subsection{Physical and Chemical Change}

\subsubsection*{Learning Objectives}
\begin{itemize}
\item{SWBAT demonstrate physical and chemical changes of matter experimentally.}
\end{itemize}


\subsubsection*{Materials}
A piece of paper, sugar, spoon, bar magnet, two iron nails, steel wool, kerosene stove and match box.

\subsubsection*{Activity Procedure}
\begin{enumerate}
\item{Take a piece of paper and light it on fire using a match box. Have students record their observations.}
\item{Take a small amount of sugar into the spoon and heat until a clear chemical change is observed. Have students record their observations.}
\item{Take a bar magnet and rub one nail in one direction only. [DIAGRAM]}
\item{ Place the second (un-magnetized) nail hear the iron fillings . Repeat with the nail that has been rubbed with a magnet.}
\item{Take the magnetized nail and rub it again in the opposite direction and place in iron fillings observe what will happen.}
\end{enumerate}

\subsubsection*{Results and Conclusion}
Burning paper and burning sugar are examples of chemical changes. These means that chemical bonds are broken and remade and new products are formed. But when the nail is rubbed with the bar magnet no chemical bonds are broken or made. The nail becomes magnetized and thus attracts the iron fillings. This is an example of a physical change.

\subsubsection*{Clean Up Procedure}
\begin{enumerate}
\item{Collect all the used materials, cleaning and storing items that will be used later. No special waste disposal is required.}
\end{enumerate}

\subsubsection*{Discussion Questions}
\begin{enumerate}
\item{Explain the changes in the paper and sugar. Name the type of change.}
\item{Explain the process of rubbing a nail with a bar magnet and what happen when it was rerubbed?}
\item{Why did the two nail behave differently? Name the change that happened in the rubbed nail.}
\end{enumerate}

\subsubsection*{Notes}

\section{ Chemical Reactions}
Chemical reactions are chemical changes - chemical bonds are broken and remade. This section includes examples of three common types of reactions - thermal decomposition, binary combination, and precipitation - and activities to guide students to better understand these reactions. There is also a discussion of solubility and another activity to do with students.

\subsection{Thermal Decomposition}

[INSERT ACTIVITY HERE]

\subsection{Binary Combination}
Binary combination is the kind of chemical reaction where two elements come together to form a compound. The elements alone are often quite different from the compounds they make up. Iron is a hard, magnetic, silver metal and sulphur is a yellow powder. When heated, the two combine to form iron (II) Sulphide FeS which is a black, insoluble compound. 

This activity gives students a chance to observe a binary combination reaction by heating sulphur and iron together. The students will investigate the properties of the elements before the reaction as well as the final product. Because this activity might lead to the production of a gas that could be harmful in large amounts, it is best as a teacher demonstration.

\subsubsection*{Learning Objectives}
\begin{itemize}
\item{To demonstrate a binary combination reaction.}
\item{To write a balanced equation for the binary combination of iron and sulphur.}
\end{itemize}

\subsubsection*{Materials}
Sulphur*, steel wool, source of heat*, tea spoon, heating vessel*, bar magnet and aluminium plate.

\subsubsection*{Hazards and Safety}
\begin{itemize}
\item{Perform this experiment in a well ventilated room and prevent inhalation of any fumes formed. Sulphur dioxide will be formed which is poisonous.}
\end{itemize}

\subsubsection*{Preparation}
\begin{enumerate}
\item{Grind the steel wool using a spoon on a hard surface so that fine particles are obtained.}
\end{enumerate}

\subsubsection*{Activity Procedure}
\begin{enumerate}
\item{Put half a teaspoon of iron particles in a beaker and add the same amount of sulphur powder. Mix them well.}
\item{Put a bar magnet into the mixture and observe the results.}
\item{Put some iron fillings in the heating vessel, add twice as much sulphur powder to it and mix them well. Heat the mixture until the sulphur powder is gone.}
\item{Again use a bar magnet to try to separate the two components physically from the mixture and record the results.}
\item{To another heating vessel put few iron fillings, add sulphur powder to it and mix them. Heat the mixture strongly for an extended time.}
\item{Use a a bar magnet and to try separate the two components physically from the mixture and record the results. If nothing is observed on the bar magnet, continue heating and try again.}
\end{enumerate}

\subsubsection*{Results and Conclusion}
The mixture of sulphur and iron before heating can easily be separated by physical means because iron has magnetic properties while sulphur does not. When the second mixture is heated, the iron and sulphur combine to form iron (II) sulphide. Because the sulphur and iron are chemically bound together, they cannot be separated by physical means. In the third trial the iron sulphide is heated extensively, liberating the sulphur as sulphur dioxide. This leaves iron metal behind (or an iron oxide) which can be picked up using a magnet.

\subsubsection*{Clean Up Procedure}
\begin{enumerate}
\item{Collect all the used materials, cleaning and storing items that will be used later. No special waste disposal is required. Solid waste should not be put in the drain.}
\end{enumerate}

\subsubsection*{Discussion Questions}
\begin{enumerate}
\item{What do you understand by the term binary combination?}
\item{Write the equation for the binary combination of Iron and sulphur.}
\item{Why was there no effect of the bar magnet in mixture 2?}
\end{enumerate}


\subsection{Precipitation Reaction}
A precipitate reaction is the formation of an insoluble salt by mixing solutions which contain its two components. These reactions are great for teaching students because of the speed and dramatic changes involved. The principles or precipitation are used in many future practicals-including identifying unknowns in qualitative analysis-so this activity is a good introductions to the principles. This activity is best done individually by students or in small groups of 2-3. 

\subsubsection*{Learning Objectives}
\begin{itemize}
\item{To explain the meaning of a precipitation reaction.}
\item{To write chemical equations for the precipitation of insoluble salt.}
\end{itemize}

\subsubsection*{Materials}
copper sulphate* or magnesium sulphate*, sodium carbonate*, beakers*, funnel*, filter paper*

\subsubsection*{Preparation}
\begin{enumerate}
\item{Make a solution of copper or magnesium sulphate by dissolving about 1 spoonful in 500 mL of water.}
\item{Make a solution of sodium carbonate by dissolving about 1 spoonful in 500 mL of water.}
\end{enumerate}

\subsubsection*{Activity Procedure}
\begin{enumerate}
\item{Take one empty beaker and add about 10 mL of copper or magnesium sulphate solution.}
\item{To the same beaker add about 10 mL of sodium carbonate solution.}
\item{Leave the mixture to settle for 5-10 minutes.}
\end{enumerate}

\subsubsection*{Results and Conclusion}
When copper sulphate or magnesium sulphate solution is mixed with sodium carbonate a precipitate will form. Copper carbonate is a blue precipitate, magnesium carbonate is a white precipitate. This reaction is useful for preparing insoluble salts.
The chemical reactions are 
CuSO4(aq) Na2CO3(aq)--CaCO3(s) Na2SO4(aq)
MgSO4(aq) Na2CO3(aq)--MgCO3(s) Na2SO4(aq)

\subsubsection*{Clean Up Procedure}
\begin{enumerate}
\item{Collect all the used materials, cleaning and storing items that will be used later. No special waste disposal is required.}
\end{enumerate}

\subsubsection*{Discussion Questions}
\begin{enumerate}
\item{What precipitate is formed in this reaction?}
\item{Write a balanced chemical equation for this reaction.}
\end{enumerate}
