\chapter{Chemical Reactions}

Chemical reactions are chemical changes - chemical bonds are broken and remade. This section includes examples of three common types of reactions - thermal decomposition, binary combination, and precipitation - and activities to guide students to better understand these reactions. There is also a discussion of solubility and another activity to do with students.

\subsection{Thermal Decomposition}

[INSERT ACTIVITY HERE]

\subsection{Binary Combination}
Binary combination is the kind of chemical reaction where two elements come together to form a compound. The elements alone are often quite different from the compounds they make up. Iron is a hard, magnetic, silver metal and sulphur is a yellow powder. When heated, the two combine to form iron (II) Sulphide FeS which is a black, insoluble compound. 

This activity gives students a chance to observe a binary combination reaction by heating sulphur and iron together. The students will investigate the properties of the elements before the reaction as well as the final product. Because this activity might lead to the production of a gas that could be harmful in large amounts, it is best as a teacher demonstration.

\subsubsection*{Learning Objectives}
\begin{itemize}
\item{To demonstrate a binary combination reaction.}
\item{To write a balanced equation for the binary combination of iron and sulphur.}
\end{itemize}

\subsubsection*{Materials}
Sulphur*, steel wool, source of heat*, tea spoon, heating vessel*, bar magnet and aluminium plate.

\subsubsection*{Hazards and Safety}
\begin{itemize}
\item{Perform this experiment in a well ventilated room and prevent inhalation of any fumes formed. Sulphur dioxide will be formed which is poisonous.}
\end{itemize}

\subsubsection*{Preparation}
\begin{enumerate}
\item{Grind the steel wool using a spoon on a hard surface so that fine particles are obtained.}
\end{enumerate}

\subsubsection*{Activity Procedure}
\begin{enumerate}
\item{Put half a teaspoon of iron particles in a beaker and add the same amount of sulphur powder. Mix them well.}
\item{Put a bar magnet into the mixture and observe the results.}
\item{Put some iron fillings in the heating vessel, add twice as much sulphur powder to it and mix them well. Heat the mixture until the sulphur powder is gone.}
\item{Again use a bar magnet to try to separate the two components physically from the mixture and record the results.}
\item{To another heating vessel put few iron fillings, add sulphur powder to it and mix them. Heat the mixture strongly for an extended time.}
\item{Use a a bar magnet and to try separate the two components physically from the mixture and record the results. If nothing is observed on the bar magnet, continue heating and try again.}
\end{enumerate}

\subsubsection*{Results and Conclusion}
The mixture of sulphur and iron before heating can easily be separated by physical means because iron has magnetic properties while sulphur does not. When the second mixture is heated, the iron and sulphur combine to form iron (II) sulphide. Because the sulphur and iron are chemically bound together, they cannot be separated by physical means. In the third trial the iron sulphide is heated extensively, liberating the sulphur as sulphur dioxide. This leaves iron metal behind (or an iron oxide) which can be picked up using a magnet.

\subsubsection*{Clean Up Procedure}
\begin{enumerate}
\item{Collect all the used materials, cleaning and storing items that will be used later. No special waste disposal is required. Solid waste should not be put in the drain.}
\end{enumerate}

\subsubsection*{Discussion Questions}
\begin{enumerate}
\item{What do you understand by the term binary combination?}
\item{Write the equation for the binary combination of Iron and sulphur.}
\item{Why was there no effect of the bar magnet in mixture 2?}
\end{enumerate}


\subsection{Precipitation Reaction}
A precipitate reaction is the formation of an insoluble salt by mixing solutions which contain its two components. These reactions are great for teaching students because of the speed and dramatic changes involved. The principles or precipitation are used in many future practicals-including identifying unknowns in qualitative analysis-so this activity is a good introductions to the principles. This activity is best done individually by students or in small groups of 2-3. 

\subsubsection*{Learning Objectives}
\begin{itemize}
\item{To explain the meaning of a precipitation reaction.}
\item{To write chemical equations for the precipitation of insoluble salt.}
\end{itemize}

\subsubsection*{Materials}
copper sulphate* or magnesium sulphate*, sodium carbonate*, beakers*, funnel*, filter paper*

\subsubsection*{Preparation}
\begin{enumerate}
\item{Make a solution of copper or magnesium sulphate by dissolving about 1 spoonful in 500 mL of water.}
\item{Make a solution of sodium carbonate by dissolving about 1 spoonful in 500 mL of water.}
\end{enumerate}

\subsubsection*{Activity Procedure}
\begin{enumerate}
\item{Take one empty beaker and add about 10 mL of copper or magnesium sulphate solution.}
\item{To the same beaker add about 10 mL of sodium carbonate solution.}
\item{Leave the mixture to settle for 5-10 minutes.}
\end{enumerate}

\subsubsection*{Results and Conclusion}
When copper sulphate or magnesium sulphate solution is mixed with sodium carbonate a precipitate will form. Copper carbonate is a blue precipitate, magnesium carbonate is a white precipitate. This reaction is useful for preparing insoluble salts.
The chemical reactions are 
CuSO4(aq) Na2CO3(aq)--CaCO3(s) Na2SO4(aq)
MgSO4(aq) Na2CO3(aq)--MgCO3(s) Na2SO4(aq)

\subsubsection*{Clean Up Procedure}
\begin{enumerate}
\item{Collect all the used materials, cleaning and storing items that will be used later. No special waste disposal is required.}
\end{enumerate}

\subsubsection*{Discussion Questions}
\begin{enumerate}
\item{What precipitate is formed in this reaction?}
\item{Write a balanced chemical equation for this reaction.}
\end{enumerate}

\subsubsection*{Notes}


\subsection{Solubility Experiments}

\subsection{Solubility}

\subsubsection*{Learning Objectives}
\begin{itemize}
\item{SWBAT compare the solubility of different substances in water and organic solvents.}
\end{itemize}

\subsubsection*{Background Information}


\subsubsection*{Materials}
Sugar, kerosene, sodium chloride*, steel wool, iodine crystals*, potassium permanganate crystals*, water, beaker*, test tube*.

\subsubsection*{Preparation Procedure}
\begin{enumerate}
\item{Break the steel wool into small piece.}
\end{enumerate}

\subsubsection*{Activity Procedure}
\begin{enumerate}
\item{In a test tube, combine about 1 mL of kerosene and 1 mL of water. Write down your observations.}
\item{Put half a spoonful of sugar into about 10 mL of kerosene in beaker labeled kerosene. Make observations.}
\item{Put half a spoonful of sugar into about 10 mL of water in a beaker labeled water. Compare the solubility of sugar in kerosene and water.}
\item{Dump out the sugar water solution and decant the kerosene into a beaker for later use.}
\item{Repeat the test using salt and steel wool, each time dumping the water solution and saving the kerosene.}
\item{Repeat with iodine and potassium permanganate, this time using only a few crystals. Record observations.}
\end{enumerate}

\subsubsection*{Results and Conclusion}
Sugar and salt dissolve in water but not kerosene. Iron does not dissolve in either solvent. Potassium permanganate dissolves in water to make a deep purple solution but does not dissolve in kerosene. Iodine dissolves only a small amount in water but in kerosene dissolves to make a deep red solution.

\subsubsection*{Clean Up Procedure}
\begin{enumerate}
\item{Kerosene can be decanted to remove any solid and stored for later use. Kerosene containing iodide should be left in an open container away from people until it has completely evaporated.}
\item{Potassium permanganate can be stored in a container for later use. If it is disposed, it should be reduced with vitamin c (ascorbic acid) prior to being dumped in the drain.}
\end{enumerate}

\subsubsection*{Discussion Questions}
\begin{enumerate}
\item{What is the solubility of sugar, iron and salt in water and organic solvent?}
\item{Compare the solubilities of potassium permanganate and iodine in water and kerosene.}
\end{enumerate}

\subsubsection*{Notes}
Water is a polar substance. Kerosene is non-polar. In solubility the general rule is "like dissolves like". Polar substances like water dissolve polar or ionic solutes like salt, sugar and potassium permanganate. Non-polar solvents like kerosene dissolve non-polar substances like iodine. Most organic substances are non-polar hence can not dissolve in water. Kerosene does not dissolve water and water does not dissolve kerosene, hence they form two layers when mixed.
