\section{Sources of Chemicals}
\label{cha:sourcesofchemicals}
%The following is a list of most of the chemicals used in ordinary science laboratories. For each we note local sources of these chemicals, low cost industrial sources of these chemicals, methods to manufacture these chemicals at your school, and/or functional alternatives to these chemicals.  We also list information like other names, common uses, and hazards. Finally, we include descriptions of many of the compounds and confirmatory tests for some to assist with identification of unlabelled chemicals. For more information on this, 
%see \nameref{cha:unknownchemicals}.

Chemicals are generally listed alphabetically by IUPAC name, 
although many compounds are also cross listed by their common name (e.g. 
acetone (common) / propanone (IUPAC)).

\subsection{Acetic acid}
See \nameref{sec:ethanoic}.
\subsection{Acetone}
See \nameref{sec:propanone}.
\subsection{Ammonia solution}
\label{sec:ammoniasol}
Formula: \ce{NH3}$_{(aq)}$\\
Other names: ammonium hydroxide, 
ammonium hydroxide solution\\
Description: clear liquid less dense than water, 
completely miscible in water, 
strong biting smell similar to old urine\\
Use: qualitative analysis, various experiments\\
Source: released from an aqueous mixture of ammonium salt and hydroxide, 
for example calcium ammonium nitrate and sodium hydroxide. 
The gas can be trapped and dissolved in water.\\
Alternative: to distinguish between zinc and lead cations, 
add dilute sulfuric acid dropwise. 
The formation of a white precipitate -- lead sulfate -- confirms lead.
Note: ammonia solution also is called ammonium hydroxide 
because ammonia undergoes autoionization to form ammonium and hydroxide ions. 
Just like water, 
there is an equilibrium concentration of the ions in an ammonia solution.
\subsection{Ammonium hydroxide solution}
See \nameref{sec:ammoniasol}.
\subsection{Ascorbic acid}
Other names: vitamin C\\
Formula: \ce{C6H7O7}\\
Description: white powder, 
but pharmacy tablets often colored\\
Confirm: aqueous solution turns blue litmus red 
AND decolorizes dilute iodine or potassium permanganate solution\\
Use: all-purpose reducing agent, 
may substitute for sodium thiosulfate in redox titrations, 
removes iodine and permanganate stains from clothing\\
Source: pharmacies
\subsection{Calcium ammonium nitrate}
Other names: \ce{CAN}\\
Description: small pellets, 
often with brown coating; 
endothermic heat of solvation\\
Use: low cost ammonium salt for teaching qualitative analysis; 
not as useful for teaching about nitrates 
as no red/brown gas released when heated. 
May be used for the preparation of ammonia and sodium nitrate.\\
Source: agricultural shops (fertilizer)\\
\subsection{Calcium carbonate}
Formula: \ce{CaCO3}\\
Description: white powder, 
insoluble in water
Confirm: brick red flame test and acid causes effervescence\\
Use: demonstration of reactivity of carbonates, 
rates of reaction, 
qualitative analysis\\
Source: coral rock, 
sea shells, 
egg shells, 
limestone, 
marble, 
white residue from boiling water\\
Local manufacture: prepare a solution of aqueous calcium 
from either calcium ammonium nitrate or calcium hydroxide 
and add a solution of sodium carbonate.\\ 
Calcium carbonate will precipitate and may be filtered and dried.
\subsection{Calcium hydroxide}
Formula: \ce{Ca(OH)2}\\
Other names: quicklime\\
Description: white to off white powder, 
sparingly soluble in water\\
Use: dissolve in carbonate-free water to make limewater\\
Source: building supply shops\\
Alternative: add a small amount of cement to water, 
let settle, 
and decant the clear solution; 
this is limewater.
\subsection{Calcium oxide}
Formula: \ce{CaO}\\
Other names: lime\\
Use: reacts with water to form calcium hydroxide, 
thus forming limewater\\
Source: cement is mostly calcium oxide
\subsection{Calcium sulfate}
Formula: \ce{CaSO4\cdot} 2\ce{H2O}\\
Other names: gypsum, 
plaster of Paris\\
Description: white powder, 
insoluble in cold water but soluble in hot water\\
Use: qualitative analysis\\
Source: building supply companies (as gypsum powder)
\subsection{Carbon (amorphous)}
Source: soot, 
charcoal (impure)
\subsection{Carbon (graphite)}
\label{sec:carbongraphite}
Use: element, \\
inert electrodes for chemistry and physics
Source: dry cell battery electrodes, 
pencil cores (impure)
\subsection{Carbon dioxide}
Preparation: react an aqueous weak acid 
(citric acid or ethanoic acid) with a soluble carbonate 
(sodium carbonate or sodium hydrogen carbonate)
\subsection{Citric acid}
Formula: \ce{C6H8O7} = \ce{CH2(COOH)COH(CHOOH)CH2COOH}\\
Description: white crystals soluble in water, 
endothermic heat of solvation\\
Use: all purpose weak acid, 
volumetric analysis, 
melting demonstration, 
preparation of carbon dioxide, 
manufacture of Benedict's solution\\
Hazard: acid – keep out of eyes!\\
Source: markets (sold as a spice often with a local name), 
supermarkets
\subsection{Copper}
\label{sec:copper}
Use: element, 
preparation of copper sulfate, 
electrochemical reactions\\
Description: dull red/orange metal\\
Source: electrical wire -- e.g. 
2.5~mm gray insulated wire has 50~g of high purity copper per meter.\\
Note: modern earthing rods are only copper plated, 
and thus no longer a good source of copper
\subsection{Copper sulfate}
Formula: \ce{CuSO4} (anhydrous), 
\ce{CuSO4\cdot} 5\ce{H2O} (pentahydrate)\\
Description: white (anhydrous) or blue (pentahydrate) crystals\\
Confirm: blue/green flame test 
and aqueous solution gives a white precipitate 
when mixed with lead or barium solution\\
Use: qualitative analysis, 
demonstration of the reactivity series, 
manufacture of Benedict's solution, 
test for water\\
Source: imported ``local'' medicine (manufactured in India).\\ 
Local manufacture: Electrolyze dilute (1-2~M) sulfuric acid 
with a copper anode and inert (e.g. 
graphite) cathode. 
Evaporate final solution until 
blue crystals of copper sulfate pentahydrate precipitate. 
To prepare anhydrous copper sulfate from copper sulfate pentahydrate, 
gently heat until the blue color has faded. 
Strong heating will irreversibly form black copper oxide. 
Store anhydrous copper sulfate in an air-tight container -- 
otherwise atmospheric moisture will reform the pentahydrate.
\subsection{Distilled water}
Formula: \ce{H2O} and nothing else!\\
Use: qualitative analysis\\
Source: rain water.\\
Allow the first 15 minutes of rain to clean off the roof 
and then start collecting water. 
In schools in dry climates, 
collect as much rain water as possible during the rainy season. 
Use it only for qualitative analysis, 
preparation of qualitative analysis reagents, 
and manufacture of qualitative analysis salts.\\ 
Distilled water may also be purchased at most petrol stations 
and automotive shops.\\
Local manufacture: Heat water in a kettle 
and use a rubber hose to bring the steam through a container of cold water. 
Collect the condensate -- pure water.\\
Alternative: river or tap water is almost always sufficient. 
Volumetric analysis never needs distilled water 
if you follow the instructions in Relative Standardization. 
Also, 
the tap water in many places is sufficient for even qualitative analysis.
\subsection{Ethanoic acid}
\label{sec:ethanoic}
Formula: \ce{CH3COOH}\\
Other names: acetic acid\\
Description: clear liquid, 
completely miscible with water, 
strong vinegar smell\\
Use: all purpose weak acid, 
volumetric analysis\\
Source: 96\% solution available from village industry supply shops, 
vinegar (5\% solution) available in small shops and supermarkets\\
Safety for 96\% ethanoic acid: HARMFUL VAPORS. 
Use outside or in a well ventilated space. 
CORROSIVE ACID. 
Always have dilute weak base solution (e.g. 
sodium hydrogen carbonate) available to neutralize spills. 
Wear gloves and goggles when handling. 
Do not induce vomiting if ingested.\\
Alternative: for a weak acid, 
citric acid. 
\subsection{Ethanol}
Formula: \ce{CH3CH2OH}\\
Description: clear liquid, 
completely miscible with water, 
strong and sweet alcohol smell\\
Use: solvent, 
extraction of chlorophyll, 
removes permanent marker, 
preparation of POP solution, 
distillation, 
preservation of biological specimens\\
Hazard: ethanol itself is a mild poison, 
and methylated spirits and other industrial alcohol contain 
additional poisonous impurities (methanol) 
specifically so that no one drinks it\\
Sources: methylated spirits are 70\% ethanol, 
hard liquor is often 30-40\%, 
village-brewed concentrated alcohol varies 
and may contain toxic quantities of methanol\\
Local manufacture: fermentation of sugar by yeast will produce 
up to a 15\% solution -- at that point, 
the yeast dies; 
distillation can in theory concentrate this to up to 95\%, 
but this is hard with simple materials. 
Nevertheless, 
preparing ethanol of sufficient concentration to dissolve POP (50-60\%) 
is quite possible.\\
Note: the color of most methylated spirits makes them undesirable 
for preparation of POP; 
hard liquor will suffice, 
but poorly because of its relatively low ethanol content. 
Colored methylated spirits can be run 
through a simple distillation apparatus to produce colorless spirits, 
as the pigment is less volatile than the ethanol. 
Of course, 
methanol and other poisons remain, 
but the clear solution works beautifully for dissolving POP.\\ 
Beware that ethanol vapors are flammable -- 
a poorly constructed distillation setup may explode.
\subsection{Graphite}
See \nameref{sec:carbongraphite}.
\subsection{Hydrochloric acid}
\label{sec:hydroacid}
Formula: \ce{HCl}, 
36.5~g/mol, 
density 1.18~g/cm$^{3}$ when concentrated ($\sim$12~M)\\
Other names: muriatic acid, 
pH decreasing compound for swimming pools\\
Description: clear liquid, 
may be discolored by contamination, 
distinct smell similar to chlorine 
although sometimes smells strongly of vinegar\\
Confirm: decolorizes weak solutions of potassium permanganate; 
white precipitate in silver nitrate solution 
and effervescence with (hydrogen) carbonates\\
Use: volumetric analysis, 
qualitative analysis\\
Source: swimming pool chemical suppliers, industrial chemical\\ 
Safety: HARMFUL VAPORS. 
Use outside or in a well ventilated space. 
CORROSIVE ACID. 
Always have dilute weak base solution (e.g. 
sodium hydrogen carbonate) available to neutralize spills. 
Wear gloves and goggles when handling. 
Extremely toxic hydrogen cyanide gas formed 
on mixing with cyanides or hexacyanoferrate compounds. 
Toxic chlorine gas formed on reaction with oxidizing agents, 
especially bleach. 
Do not induce vomiting if ingested.\\
Alternative (strong acid): sulfuric acid\\
Alternative (acid): citric acid\\
Alternative (qualitative analysis): for the test for carbonates, 
use dilute sulfuric acid; 
to dissolve insoluble carbonates, 
nitric acid may be used instead
\subsection{Hydrogen}
Formula: \ce{H2}\\
Confirm: ``pop sound,'' i.e. 
ignites with a bang; 
in an inverted test tube the rapid movement of air 
near the mouth creates a rapid, 
high pitch ``whoosh'' that gives the ``pop'' name\\
Preparation: combine dilute acid (e.g. 
battery acid) and a reactive metal (steel wool or zinc) 
in a plastic water bottle. 
Attach a balloon to the top of the water bottle; 
being less dense than air, 
hydrogen will migrate up and slowly fill the balloon. 
Specific instructions for various alternatives are available 
in the Hands-On activities section. 
Before ignition, 
always move the balloon away from the container of acid.
\subsection{Hydrogen peroxide}
Formula: \ce{H2O2}\\
Description: solutions are colorless liquids 
appearing exactly like water\\
Confirm: decolorizes potassium manganate (VII) solution 
in the absence of acid, 
neutral pH\\
Use: preparation of oxygen, 
general oxidizer and also may act as a reducing agent (e.g. 
with potassium permanganate)\\
Source: pharmacies sell 3\% (10 volume) and 6\% (20 volume) solutions 
as medicine for cleaning sores\\
Note: `20 volume' means it will produce 20 times its liquid volume in oxygen gas.
\subsection{Indicator}
\label{sec:indicator}
Source: red flowers\\
Preparation: Crush flower petals in water. 
Some effective flowers include rosella, 
bougainvillea, 
and hibiscus. 
Test other flowers near your school.\\
Note: For bougainvillea and some other flowers, 
extract the pigment with ethanol 
or hard alcohol to get a better color. 
Color will change from pink (acidic) to colorless (basic). 
Rosella will change from red (acidic) to green (basic).
For an indicator in redox titrations involving iodine, 
see starch solution.
\subsection{Iodine}
Formula: \ce{I2}$_{(s)}$\\
Description: purple/black crystals\\
Local manufacture: add a little dilute sulfuric acid 
to iodine solution from a pharmacy. 
Then add sodium hypochlorite solution (bleach) dropwise 
until the solution turns colorless with solid iodine resting on the bottom. 
The solid iodine can be removed by filtration or decantation. 
If pure iodine is necessary, 
the solid may be purified by sublimation.\\
Note: this reaction produces poisonous chlorine gas. 
Therefore, 
produce iodine in a well ventilated area and stand upwind.
\subsection{Iodine solution}
\label{sec:iodinesol}
Composition: \ce{I2} + \ce{KI} dissolved in water and sometimes ethanol\\
Description: light brown solution\\
Confirm: turns starch blue or black\\
Use: food tests for detection of starch and fats\\
Source: pharmacies sell a ‘weak iodine solution’ 
or ‘tincture of iodine’ that is really about 50\% by mass iodine. 
To prepare a useful solution for food tests, 
dilute this 10:1 in ordinary water.\\
Note: to use this solution for detection of fats, 
it must be made without ethanol, 
spirits, 
alcohol and the like. 
Either kind works for detection of starch.
\subsection{Iron}
\label{sec:iron}
Use: element, 
demonstration of reactivity series, 
preparation of hydrogen, 
preparation of iron sulfide, 
preparation of iron sulfate\\
Source: for samples of the element 
and for use in electrochemical experiments, 
buy non-galvanized nails at a hardware store, 
or find them on the ground. 
You can tell they are not galvanized because they are starting to rust. 
Clean off the rust with steel wool prior to use. 
For samples of the element for preparation of other compounds, 
buy steel wool from small shops or supermarkets. 
This has a very high surface area / mass ratio, 
allowing for faster reactions.
\subsection{Magnesium sulfate}
\label{sec:magsulfate}
Formula: \ce{MgSO4\cdot} 7\ce{H2O}\\
Other names: epsom salts\\
Description: white or clear crystals\\
Use: crystallization experiments, 
qualitative analysis test reagent 
(confirmation of hydrogen carbonate and carbonate), 
precipitation reactions\\
Source: livestock and veterinary supply shops sell Epsom salts 
to treat constipation in cattle
\subsection{Manganese (IV) oxide}
Formula: \ce{MnO2}\\
Other names: manganese dioxide\\
Description: black powder\\
Confirm: liberates oxygen from hydrogen peroxide\\
Use: preparation of oxygen, 
qualitative analysis (confirmation of chlorides)\\
Source: old dry cell batteries (radio batteries)\\
Extraction: smash a dry cell battery with a rock 
and scrape out the black powder. 
This is a mixture of manganese dioxide, 
zinc chloride, 
and ammonium chloride. 
This impure mixture is suitable for the preparation of oxygen. 
To purify manganese dioxide for use in qualitative analysis, 
boil the powder in water to dissolve away the chlorides. 
Filter the solution after boiling 
and repeat if the test gives false positives (e.g. 
confirms chlorides in samples that lack chlorides)\\
Note: Wash your hands with soap if you accidentally touch the powder. 
Do not get it on your clothes or into cuts on your hands. 
\ce{MnO2} causes metal to corrode; 
if you use a metal tool to scrap out the powder, 
be sure to clean it off afterwards. 
Better: use non-metal tools. 
\subsection{Organic solvents}
Sources: kerosene, 
petrol, 
paint remover, 
paint thinner and the safest: cooking oil
\subsection{Oxygen}
Confirm: oxygen gas relights a glowing splint, 
i.e. 
a piece of wood or paper glowing red / orange 
will flame when put in a container 
containing much more oxygen than the typical 20\% in air\\
Preparation: combine hydrogen peroxide 
and manganese (IV) oxide in a plastic water bottle. 
Immediately crush the bottle to remove all other air and then cap the top. 
The bottle will re-inflate with oxygen gas.
\subsection{Potassium iodide}
\label{sec:potiodide}
Formula: \ce{KI}\\
Description: white crystals very similar in appearance to common salt, 
endothermic heat of solvation\\
Confirm: addition of weak potassium permanganate 
or bleach solution causes a clear KI solution to turn yellow/brown 
due to the formation of \ce{I2} (which then reacts with \ce{I-} to form soluble \ce{I3-})\\
Use: preparation of iodine solution for food tests in biology, 
preparation of iodine solutions for redox titrations, 
confirmatory test for lead in qualitative analysis\\
Local manufacture: Heat a pharmacy iodine tincture strongly until 
only clear crystals remain. 
In this process, 
the \ce{I2} will sublimate -- 
placing a cold dish above the iodine solution should cause must of the iodine 
to deposit as solid purple crystals. 
Note that the iodine vapors are harmful to inhale.
If you need \ce{KI} for a solution that may contain impurities, 
add ascorbic acid solution to dilute iodine tincture 
until the solution exactly decolorized.\\
Alternative (food tests): see \nameref{sec:iodinesol}\\
Alternative (redox titrations): 
often you can also use iodine solution for this; 
just calibrate the dilution of pharmacy tincture 
and the other reagents to create a useful titration\\
Alternative (qualitative analysis): 
confirm lead by the addition of dilute sulfuric acid -- 
white lead sulfate precipitates
\subsection{Potassium manganate (VII)}
Formula: \ce{KMnO4}\\
Other names: potassium permanganate, 
permanganate\\
Description: purple/black crystals, 
sometimes with a yellow/brown glint, 
very soluble in water -- 
a few crystals will create a strongly purple colored solution\\
Hazard: powerful oxidizing agent -- 
may react violently with various compounds; 
solutions stain clothing (remove stains with ascorbic acid solution); 
crystals and concentrated solution discolor skin 
(the effect subsides after a few hours, 
but it is better to not touch the chemical!)\\
Use: strong oxidizer, 
self-indicating redox titrations, 
identification of various unknown compounds, 
diffusion experiments\\
Source: imported ``local'' medicine. 
Also sold in very small quantities in many pharmacies. 
May be available in larger quantities from hospitals.\\
Alternative (oxidizer): bleach (sodium hypochlorite), 
hydrogen peroxide\\
Alternative (diffusion experiments): solid or liquid food coloring, 
available in markets and small shops
\subsection{Propanone}
\label{sec:propanone}
Formula: \ce{H3CCOCH3}\\
Other names: acetone\\
Description: clear liquid miscible in water, 
smells like nail polish remover, 
quickly evaporates\\
Use: all-purpose lab solvent, 
iodoform reaction (kinetics, organic chemistry)\\
Hazard: highly flammable\\
Source: nail polish remover (mixture with ethyl ethanoate)\\
Alternative (volatile polar solvent): ethanol, 
including methylated spirits
\subsection{Sodium carbonate}
Formula: \ce{Na2CO3\cdot} 10\ce{H2O} (hydrated), 
\ce{Na2CO3} (anhydrous)\\
Other names: soda ash, washing soda\\
Description: white powder completely soluble in water\\
Use: all-purpose cheap base, 
volumetric analysis, 
qualitative analysis, 
manufacture of other carbonates\\
Safety: rather caustic, keep off of hands and definitely out of eyes!\\
Source: commercial and industrial chemical supply -- 
should be very inexpensive\\
Local manufacture: dissolve sodium hydrogen carbonate in distilled water 
and boil for five or ten minutes 
to convert the hydrogen carbonate to carbonate. 
Let evaporate until crystals form. 
For volumetric analysis, 
the hydrated salt may always substitute 
for the anhydrous with a correction to the concentration -- 
see Chemical Substitutions for Volumetric Analysis
\subsection{Sodium chloride}
Formula: \ce{NaCl}\\
Other names: common salt\\
Use: all-purpose cheap salt, 
qualitative analysis\\
Source: the highest quality salt in markets (white, 
finely powdered) is best. 
The iodine salts added to prevent goiter 
do not generally affect experimental results.
\subsection{Sodium hydrogen carbonate}
Formula: \ce{NaHCO3}\\
Description: white powder, 
in theory completely soluble in cold water 
in practice often dissolves poorly\\
Other names: sodium bicarbonate, 
bicarbonate of soda\\
Use: all-purpose weak base, 
preparation of carbon dioxide, 
qualitative analysis\\
Source: small shops \\
Note: may contain ammonium hydrogen carbonate
\subsection{Sodium hydroxide}
Formula: \ce{NaOH}\\
Other names: caustic soda\\
Description: white deliquescent crystals -- 
will look wet after a minute in contact with air 
and will fully dissolve after some time, 
depending on humidity and particle size\\ 
Use: all-purpose strong base, 
volumetric analysis, 
food tests in biology, 
qualitative analysis, 
preparation of sodium salts of weak acids\\
Hazard: corrodes metal, 
burns skin, 
and can blind if it gets in eyes\\
Source: industrial supply shops, 
supermarkets, 
hardware stores (drain cleaner)\\
Local manufacture: mix wood ashes in water, 
let settle, 
and decant; 
the resulting solution is mixed sodium and potassium hydroxides 
and carbonates and will work for practicing volumetric analysis\\
Note: ash extracts are about 0.1~M base and may be concentrated by boiling; 
this is dangerous, 
however, 
and industrial caustic soda is so inexpensive 
and so pure that there is little reason to use ash extract 
other than to show that ashes are alkaline 
and that sodium hydroxide is not exotic.
\subsection{Sodium hypochlorite solution}
Formula: \ce{NaOCl}$_{(aq)}$\\
Other names: bleach\\
Use: oxidizing agent\\
Source: small shops, 
supermarkets\\
Local manufacture: electrolysis of concentrated salt water solution 
with inert (e.g. 
graphite) electrodes; 
4-5~V (three regular batteries) is best for maximum yield\\
Note: commercial bleach is usually 3.5\% sodium hypochlorite by weight
\subsection{Sulfur}
Description: light yellow powder with distinct sulfurous smell\\
Use: element, 
preparation of iron sulfide\\
Source: large agricultural shops (fungicide, 
e.g. 
for dusting crops), 
imported ``local'' medicine
\subsection{Sulfuric acid}
Formula: \ce{H2SO4}\\
Other names: battery acid\\
Description: clear liquid with increasing viscosity at higher concentrations; 
fully concentrated sulfuric acid ($\sim$18~M) is almost twice as dense as water 
and may take on a yellow, 
brown, 
or even black color from contamination\\
Use: all-purpose strong acid, 
volumetric analysis, 
qualitative analysis, 
preparation of hydrogen and various salts\\
Source: battery acid from petrol stations 
is about 4.5~M sulfuric acid and one of the least expensive sources of acid\\
Hazard: battery acid is dangerous; 
it will blind if it gets in eyes and will put holes in clothing. 
Fully concentrated sulfuric acid is monstrous, 
but fortunately never required. 
For qualitative analysis, 
``concentrated'' sulfuric acid means $\sim$5~M -- battery acid will suffice.\\
Note: ``dilute'' sulfuric acid should be about 1~M. 
To prepare this from battery acid, 
add one volume of battery acid to four volumes of water (e.g. 
100~mL battery acid + 400~mL water)
\subsection{Zinc}
\label{sec:zinc}
Description: firm silvery metal, 
usually coated with a dull oxide\\
Use: element, 
preparation of hydrogen, 
preparation of zinc carbonate and zinc sulfate\\
Source: dry cell batteries; 
under the outer steel shell is an inner cylinder of zinc. 
In new batteries, 
this whole shell may be extracted. 
In used batteries, 
the battery has consumed most of the zinc during the reaction, 
but there is generally an unused ring of zinc around the top 
that easily breaks off. 
Note that alkaline batteries, 
unlike dry cells, 
are unsafe to open -- and much more difficult besides.