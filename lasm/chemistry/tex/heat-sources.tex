\section{Heat Sources}

Heat and flames are important for teaching chemistry. This section discusses low cost options for both heating solutions and higher temperature flames. This section also discusses safety specific to heating and fire. Finally, this section includes activities to teach students about luminous and non-luminous flames as well as basic fire fighting technique.

\subsection{Options for Heat Sources}

\subsubsection{Motopoa Burners}

These are the simplest stoves and the most practical for the chemistry laboratory. Motopoa is a heavy oil mixed with an organic alcohol. The fuel lights easily and burns hot, generally with a non-luminous (invisible) flame.

While there are commercial Motopoa stoves for cooking, a laboratory burner is only a small metal container. For small heating experiments - qualitative analysis, thermal decomposition, binary combination - the best container is the metal cap from a liquor bottle, e.g. Konyagi brand. Remove the plastic from the cap and fill it half way with Motopoa. Light the Motopoa with a match or gas lighter. No wick is required. The flame is difficult to see in bright light; carefully feel for heat to see if it has lit. To extinguish (put out) a Motopoa flame, blow on it like a candle.

Larger burners may also be constructed for heating water bathes, etc. Find an approximately 1 litre metal can and cut out the bottom and ventillation holes in the sides. Place this over a smaller can of Motopoa and rest a cooking pot on the top of the larger can. ((ILLUSTRATION))

\subsubsection{Kerosene Stoves}

Kerosene stoves may also be used for heating water bathes. Kerosene is less expensive than Motopoa but produces unpleasant smoke when burned and is more dangerous as a fuel because spills easily spread. The stoves are also more difficult to light and put out, and must be purchased rather than locally manufactured.

\subsubsection{Charcoal Stoves}

Charcoal stoves are useful when heat is required for a long time. Charcoal stoves take a long time to heat up. Charcoal burning properly should not produce smoke, though a stove with insufficient airflow may produce carbon monoxide.

\subsection{Types of Flames}
Many chemistry practicals involve the use of flame for heating, thus it is important for students to understand the different ways to produce flames in the laboratory. There are two kinds of flame: luminous and non-luminous. A luminous flame is one that produces a lot of light. The best flame in the chemistry lab is a non-luminous flame because it is hotter and does not produce soot. Unfortunately, in books the most common method given for producing the two types of flame is a bunsen burner. This activity is written to introduce students to different types of flames without the use of a bunsen burner. This activity is best done by students in small groups but the teacher should supervise the groups closely as they will be dealing with fires.
\subsubsection*{Learning Objectives}
\begin{itemize}
\item{To produce luminous and non-luminous flames from different fuel burners.}
\item{To demonstrate the combustion of various substances in air.}
\end{itemize}

\subsubsection*{Materials}
kerosene burner*, spirit burner*, motopoa burner*, candle, spirit, kerosene, paper, motopoa, metal spoon, matches, metal jam-jar lid

\subsubsection*{Hazards and Safety}
\begin{itemize}
\item{((FIRE))}
\end{itemize}

\subsubsection*{Activity Procedure}
\begin{enumerate}
\item{Light a kerosene burner and observe the flame. Adjust the heights of the wicks to change the appearance of the flame. Hold the metal spoon over the flame for a few seconds and then examine it for soot.}
\item{Light the spirit burner (if no spirit burner is available, a small amount of spirit can be poured into a soda bottle lid and lit). Observe the flame. Hold the metal spoon over the flame for a few seconds and examine it for soot.}
\item{Light the motopoa burner. Observe the flame. Hold the metal spoon over the flame for a few seconds and examine it for soot.}
\item{Light the candle. Observe the flame. Hold the metal spoon over the flame for a few seconds and examine it for soot.}
\item{Put a small piece of paper into the metal lid and light it. Observe the flame. Hold the spoon over the flame for a few seconds and examine it for soot.}
\end{enumerate}

\subsubsection*{Results and Conclusion}
The kerosene produces a luminous flame. Adjusting the wick length can change the nature of the flame. When the wicks are very long you shold observe a bigger and brighter flame that is more likely to produce soot. If the metal cover is removed, the flame will be luminous and produce a lot smoke.
Spirit burns as a non-luminous flame that is colourless. The flame will be nearly invisible and will not produce soot.
Motopoa burns as a non-luminous flame. The flame will not produce soot.
A candle produces a luminous flame that deposits soot on the spoon.
A piece of paper produces a luminous flame that deposits soot on the spoon.
\subsubsection*{Discussion Questions}
\begin{enumerate}
\item{Which substances produced soot when burned?}
\item{Identify which substances produce luminous flames? Which substances produce non-luminous flames?}
\item{Which of the materials tested is best for heating in the chemistry laboaratory? Give reasons.}
\item{Which of the materials tested is worst for heating in the chemistry laboaratory? Give reasons.}
\end{enumerate}

\subsection{Fire Fighting}
Many chemistry activities involve the use of fire as a heat source. It is very important that students have an understanding of how to extinguish various fires in the laboratory. The most common sources of fire in the chemistry laboratory are kerosene, spirit, motopoa and paper/wood. The common available materials for extinguishing fires are water, sand and exhaled carbon dioxide. This activity give students the chance to explore and identify the best methods for extinguishing fires. This information will be very useful in case of an accident during later activities. This activity can be a useful demonstration but it is most effective if performed by students in small groups so they gain experience and knowledge on how to respond in case of a fire.
\subsubsection*{Learning Objectives}
\begin{itemize}
\item{To demonstrate the proper use of available materials in extinguishing various types of fire in the laboratory.}
\end{itemize}

\subsubsection*{Materials}
Ethanol*, motopoa*, candle, paper, sand, water, syringe, soda bottle tops, matches, glass jar, beaker

\subsubsection*{Hazards and Safety}
\begin{itemize}
\item{((FIRE))}
\end{itemize}

\subsubsection*{Activity Procedure}
\begin{enumerate}
\item{Put a small amount of ethanol into a bottle cap and light it with a match. Pour water onto the flame.}
\item{Dry the cap and add a small amount of ethanol and light it again with a match. Put a handful of sand onto the flame.}
\item{Clean the cap and add a small amount of ethanol and light it again with a match. invert a glass jar over the flame and wait.}
\item{Put a small amount of kerosene into a bottle cap and light it with a match. Using a syringe, \textit{carefully} add a few drops of water near the base of the flame.}
\item{To the kerosene flame ad a handful of sand.}
\item{Clean the cap and add a small amount of kerosene and light it again with a match. Invert a glass jar over the flame and wait.}
\item{Fill a beaker with water.}
\item{Light a piece of paper on fire. Immediately dip it in the beaker of water.}
\end{enumerate}

\subsubsection*{Results and Conclusion}
A flame caused by burning ethanol is extinguished by the addition of water or sand. When the glass jar is inverted over the flame it will be deprived of oxygen and slowly go out. This is a useful demonstration because there a flame in a container, it can be extinguished by closing or covering the container.
A kerosene fire can NOT be extinguished by water. Because water is immiscible with kerosene, addition of water will only cause the fire to spread. Sand can be used to extinguish the kerosene flame.
Paper is easily extinguished by water.
