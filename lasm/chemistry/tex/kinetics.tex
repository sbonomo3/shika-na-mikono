\chapter{Kinetics, Equilibrium, and Energetics}

\section{Kinetics}

Kinetics is the study of the rate of chemical reactions. Some reactions have a high rate, that is, they happen very quickly, for example the burning of paper. Other reactions have a low rate, that is, they happen very slowly, for example the rusting of metal. In order for a chemical reaction to happen, the molecules concerned must collide with each other. Anything that increases the frequency and intensity of these collisions will increase the rate of reaction.

Students should learn six factors that affect the rate of reaction: concentration, pressure, temperature, surface area, catalyst and light. This chapter discusses each of these factors and activities for presenting several of them.

\subsection{The Effect of Concentration of Reaction Rate}

Concentration has a positive effect on reaction rate. A higher concentration means that the molecules are more crowded and collide more often, thus increasing the rate of chemical reaction. In a solution of low concentration, molecules are less likely to collide with each other and hence have a slower rate of reaction. 

The effect of concentration is only observed in reactions that occur in solution. In gases, the effective concentration is directly related to the pressure of the gas - a higher pressure means a higher concentration means a faster rate of reaction. Because experiments with reacting gases are more difficult to set up and generally more dangerous to perform, students generally do experiments on the effect of concentration and use logic to extend their conclusions to the effect of pressure.

The following activity is useful for students to observe the effect of concentration on reaction rate. This activity should be performed by students working in small groups so that the effects may be easily seen.

\subsubsection*{Learning Objectives}
\begin{itemize}
\item{To demonstrate the effect of concentration on the rate of the reaction}
\end{itemize}

\subsubsection*{Materials}
dilute weak acid, sodium hydrogen carbonate, water, test tubes (6pcs), test tube rack

\subsubsection*{Activity Procedure}
\begin{enumerate}
\item{In 3 test tubes labeled A, B and C put approximately 10mL, 5mL and 2.5mL of vinegar respectively. Dilute each solution to approximately 20 mL.}
\item{In 3 new test tubes labeled  D, E, and F place 10g, 5g and 2.5g of sodium bicarbonate respectively. Dilute each solution to 20 mL with water.}
\item{Mix solution A with solution D and observe what happens.}
\item{Mix solution B with solution E and observe what happens.}
\item{Mix solution C with solutionF and observe what happend.Note any difference between the reactions in steps 3, 4, and 5.}
\end{enumerate}

\subsubsection*{Results and Conclusion}
Bubbles (carbon dioxide gas) will be formed most quickly in the reaction between solutions A and D. The bubbles will form more slowly in the reaction solutions B and E and the reaction will be slowest in the reaction between C and F. 
This experiment shows that higher the concentration of reactants, the faster the reaction will proceed.

\subsubsection*{Clean Up Procedure}
\begin{enumerate}
\item{Collect and clean all the used materials, storing items that will be used later. No special waste disposal required.}
\end{enumerate}

\subsubsection*{Discussion Questions}
\begin{enumerate}
\item{Why would this experiment be dangerous with concentrated acetic acid? (hint: vinegar is about 6\% acetic acid)}
\item{What will happen to the rate of a chemical reaction as the reaction proceeds?}
\end{enumerate}

\subsection{Kinetics: Effect of Temperature}

An increase in temperature increases the rate of chemical reactions by increasing both the frequency of collisions between molecules and the intensity of these collisions. A higher temperature means that the molecules are moving with a higher velocity and thus collide both more often and more forcefully. Faster, harder collisions means a much faster chemical reaction.

The following activity is useful for students to observe this effect. They may reasonably perform this experiment on the same day as the experiment regarding the effect of concentration. Students should perform this experiment themselves in groups.

\subsubsection*{Learning Objectives}
\begin{itemize}
\item{To demonstrate the effect of temperature on the rate of reaction.}
\end{itemize}

\subsubsection*{Materials}
any dilute weak acid (acetic or citric acid)*, sodium hydrogen carbonate*, beakers*, test tubes*, water, source of heat*

\subsubsection*{Preparation}
\begin{enumerate}
\item{A few minutes prior to class light the heat source and put some water on to heat.}
\item{Prepare a solution of soidium hydrogen carbonate by dissolving approximately 3 teaspoons per litre of water.}
\end{enumerate}

\subsubsection*{Activity Procedure}
\begin{enumerate}
\item{Arrange the students into groups of 4-6. To each group give 4 test tubes, a beaker containing approximately 10 mL of acid and a second beaker containing approximately 10 mL of base.}
\item{Instruct students to arrange test tubes in the rack and label them with numbers 1, 2, 3 and 4. Instruct them to put approximately 3 mL of acid into test tubes 1 and 2 and 3 mL of base into test tubes 3 and 4.}
\item{Instruct students to heat test tubes 2 and 4 in the boiling water bath until they are nearly boiling.}
\item{Instruct students to pour the solution from test tube 3 into test tube 1.}
\item{Instruct students to then pour the solution from test tube 4 into test tube 2. Have students record their observations about the differences between the two reactions.}
\end{enumerate}

\subsubsection*{Results and Conclusion}
In the reaction between test tube 2 and 4 the reaction will be notably faster than in the reaction between test tube 1 and 3. The students should see that the bubbles are formed more quickly--the reaction is more vigorous. The hot solutions will react faster than the cold soluitons.

\subsubsection*{Clean Up Procedure}
\begin{enumerate}
\item{Collect all the used materials, cleaning and storing items that will be used later. No special waste disposal is required.}
\item{Unused acid and base solutions can be stored and labelled for later use.}
\end{enumerate}

\subsubsection*{Discussion Questions}
\begin{enumerate}
\item{In which reaction did the reaction happen faster? How do you know?}
\item{Explain what effect temperature has on the rate of a chemical reaction.}
\item{Explain why it is important to keep vegetables in a cool place during the day rather than in the sun.}
\item{The human body must be kept at a constant temperature. Explain what would happen in the temperature gets to high or low.}
\end{enumerate}

\subsubsection*{Notes}
This experiment can be expanded by setting up a gas collection apparatus and recording the volume of gas collected per unit of time for each reaction.

\subsection{Kinetics: Effect of Surface Area}

The surface area of a solid can effect the rate of chemical reactions by increasing the rate of collisions between molecules. A larger surface area means that there are more molecule available at any moment for reacting thus there are more frequent collisions. This is because in a solid, only molecules on the outer surface are able to react. If the solid is divided into many smaller pieces, the total surface area increases, the frequency of collisions increases, and therefore the rate of reaction increases.

The following activitity uses the reaction between iron metal and dilute sulphuric acid to allow students to observe the effect of surface area on reaction rate. These materials are available everywhere. Students may perform this activity in groups, with close supervision as the use of a strong acid is required.

In parts of the country with carbonate rocks - coral rock, limestone, and marble - an alternative activity is the reaction of calcium carbonate and a dilute weak acid. Students should react citric acid solution or ethanoic acid solution with both large pieces and powders of these rocks. A clear difference in the rate of effervesence will be observed.

\subsubsection*{Learning Objectives}
\begin{itemize}
\item{To show the effect of surface area of a reacting solid on the rate of reaction.}
\end{itemize}

\subsubsection*{Materials}
dilute sulphuric acid*, iron nail (or any other solid iron object like mbulumbulu), iron wool, test tubes*

\subsubsection*{Hazards and Safety}
\begin{itemize}
\item{((dilute strong acid))}
\end{itemize}

\subsubsection*{Activity Procedure}
\begin{enumerate}
\item{Fill two test tubes half way with dilute sulphuric acid.}
\item{At the same time, put a nail in one test tube and a piece of steel wool into the other.}
\end{enumerate}

\subsubsection*{Results and Conclusion}
When the iron is placed in the acid bubbles should clearly be seen on the surface. The bubbles form much more quickly from the steel wool than from the iron nail because it has a much higher surface area.

\subsubsection*{Clean Up Procedure}
\begin{enumerate}
\item{Collect the solutions of acid from each group. Use tweezers to remove the iron metal pieces, rinse them well in water and store for later use.}
\item{Store the sulphuric acid in a bottle labelled "Impure sulphuric acid" and save for later use. If no bottle is available, neutralise the acid with baking powder until effervescence stops and then dispose down the drain.}
\item{Collect all the used materials, cleaning and storing items that will be used later.}
\end{enumerate}

\subsubsection*{Discussion Questions}
\begin{enumerate}
\item{Which object has a higher surface area a nail or a piece of steel wool?}
\item{In which test tube did the reaction occur faster? How do you know.}
\item{Why do people usually grind salt before using it to cook?}
\end{enumerate}

\subsection{Effect of catalyst on reaction rate}

A catalyst is any substance that increases the rate of a chemical reaction without being consumed in the reaction. 

Note that a substance is a catalyst only if it increases the rate of reaction. A substance that decreases the rate of reaction is called an inhibitor.

\section{Equilibrium}

\subsection{Reversible Chemical Reaction}

\subsubsection*{Learning Objectives}
\begin{itemize}
\item{To demonstrate a reversible reaction.}
\end{itemize}

\subsubsection*{Background Information}
A reversible reaction is a reaction that can occur in both forward and backward direction. Reversible reactions can be shown with a double arrow. Copper (II) sulphate exists in two forms: hydrated (with water) and anhydrous (without water). Hydrated copper (II) sulphate is blue while anhydrous copper (II) sulphate is white.

\subsubsection*{Materials}
Heat source*, copper (II) sulphate*, water, metal spoon
s
\subsubsection*{Hazards and Safety}
\begin{itemize}
\item{Motopoa flame may be invisible - be careful to avoid burns.}
\end{itemize}

\subsubsection*{Preparation}
\begin{enumerate}
\item{Grind the copper (II) sulphate crystals into a powder. If the powder looks white, leave it exposed to the air until it regains a blue colour.}
\end{enumerate}

\subsubsection*{Activity Procedure}
\begin{enumerate}
\item{Instruct students to place a very small amount of blue copper (II) sulphate in a metal spoon.}
\item{Supervise students heating the spoon gently over a heat source. Students should stop heating when the crystals have changed from blue to white.}
\item{Instruct students to add a few drops of water to the white crystals. Students should observe and record any colour change.}
\end{enumerate}

\subsubsection*{Results and Conclusion}
On heating blue hydrated copper (II) sulphate, the colour changes from blue (CuSO4-5H2O) to white (CuSO4). On addition of a few drops of water, CuSO4 returns to its original hydrated state (blue).

\subsubsection*{Clean Up Procedure}
\begin{enumerate}
\item{Copper (II) sulphate crystals can be left in the air to dry the excess water and then used again in future experiments.}
\item{Collect all the used material, cleaning and storing items that will be used again later. No special waste disposal is required.}
\end{enumerate}

\subsubsection*{Discussion Questions}
\begin{enumerate}
\item{Copper sulphate can be used as a test for water. Explain how this is possible.}
\item{Give an example of another reversible reaction that you have seen before.}
\end{enumerate}

\subsubsection*{Notes}
Continued heating will convert white anhydrous copper (II) sulphate to black copper oxide. This reaction is irreversible.

\subsection{Endothermic and Exothermic Reactions}

[INSERT ACTIVITY HERE]
