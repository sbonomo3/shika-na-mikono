\chapter{Gases}

There are several gases that are often discussed in chemistry. This chapter presents various activities to help students observe the preparation and properties of some of these gases.

\subsection{Oxygen}

Oxygen is a colourless, odourless, reactive gas essential for aerobic respiration. 20\% of the air is oxygen. Oxygen is most easily prepared by the catalytic decomposition of hydrogen peroxide. Oxygen supports combustion and will re-light a glowing splint. Oxygen also reacts vigorously with both metals and non-metals.

Traditionally the preparation of oxygen is a demonstration but with low cost materials it is possible and therefore desirable for students to prepare oxygen in groups. Students can perform the glowing splint test to confirm their result. The reaction between burning sulphur and oxygen, however, should performed by the teacher as a demonstration due to the poisonous nature of the sulphur dioxide produced.

\subsubsection*{Objectives}
\begin{itemize}
\item{To prepare a sample of oxygen gas.}
\item{To demonstrate the properties of oxygen gas.}
\end{itemize}

\subsubsection*{Materials}
6\% hydrogen peroxide*, manganese (IV) oxide*, water, sulphur powder*, gas generator*, spoon*, match box, thin dry stick, deflagrating spoon*, beaker*

\subsubsection*{Hazards and Safety}
\begin{itemize}
\item{Manganese (IV) oxide is poisonous. Avoid contact and wash with soap after this activity. Also, it is corrosive to metal - wash all tools thoroughly.}
\item{When sulphur burns, the gas produced (SO2) is poisonous. Avoid inhalation of the gas.}
\end{itemize}

\subsubsection*{Preparation of Oxygen}
\begin{enumerate}
\item{Put about one tea spoon full of manganese (IV) oxide into the reaction bottle of the gas generator.}
\item{Squeeze the second bottle ("collection bottle") to remove some of the air.}
\item{Add in the first bottle about 50 mL of dilute hydrogen peroxide. Close the bottle tightly and pass the gas into the second bottle.}
\item{Collect two bottles of oxygen this way ready for testing.}
\end{enumerate}

\subsubsection*{Glowing splint test for oxygen}
\begin{enumerate}
\item{In one of the bottles full of oxygen, insert a glowing splint.}
\item{Observe what happens.}
\end{enumerate}

\subsubsection*{Reaction between sulphur and oxygen}
\textit{This experiment should only be performed by the teacher}
\begin{enumerate}
\item{Fill a beaker with water and keep it available to put out the burning sulphur at the end of this experiment.}
\item{Put some sulphur in a deflagrating spoon and set it on fire.}
\item{Insert the burning sulphur into the second bottle of oxygen.}
\item{Observe the colour of the flame. Record your observations.}
\item{After the flame has been observed, remove the burning sulphur and immediately put it into the beaker of water to put out the flame.}
\item{Close the bottle of oxygen where the sulphur reacted to prevent the sulphur dioxide from continuing to enter the room.}
\end{enumerate}

\subsubsection*{Results and Conclusion}
The gas produced relights a glowing splint. This is the test for oxygen. Sulphur is a non-metal that burns in oxygen to form sulphur dioxide gas. Manganese (IV) oxide acts as a catalyst (i.e. it increases the rate of the decomposition of hydrogen peroxide). When hydrogen peroxide decomposes, it gives out oxygen gas and water:
2H2O2 --> 2H2O + O2
This is the safest way of producing oxygen because it requires no heating.

\subsubsection*{Clean Up Procedure}
\begin{enumerate}
\item{Take the bottle from the sulphur test outside and away from people. Open the bottle and leave it with the open end pointing down. After one hour or longer, dispose of the bottle as with other plastic trash.}
\item{Collect all the used materials, cleaning and storing items that will be used later.}
\item{Reserve the unused materials from the dry cell for other experiments.}
\item{Clean any metal tools that touched manganese (IV) oxide to prevent corrosion.}
\end{enumerate}

\subsubsection*{Discussion Questions}
\begin{enumerate}
\item{What is the use of manganese (IV) oxide in this reaction?}
\item{List the physical properties of the gas produced in this reaction.}
\item{How do you distinguish oxygen gas from other gases?}
\item{What is the name of the gas produced when sulphur burns in oxygen?}
\end{enumerate}

\subsection{Carbon Dioxide}

Carbon dioxide is a colourless, ordourless, non-reactive gas. Carbon dioxide is less than 1\% of the air but it is about 4\% of the air exhaled by people. Carbon dioxide is a product of combustion and most forms of respiration. The gas may be prepared by the reaction of acids and carbonates. Carbon dioxide does not support combustion and is more dense than air.

\subsubsection*{Objectives}
\begin{itemize}
\item{To prepare a sample of carbon dioxide gas.}
\item{To investigate the properties of carbon dioxide gas.}
\end{itemize}

\subsubsection*{Materials}
Three limes, empty water bottles, water, sodium hydrogen carbonate, gas generator*, candle, match box, lime water*, knife or razor blade

\subsubsection*{Preparation}
\begin{enumerate}
\item{Prepare a solution of lime water.}
\end{enumerate}

\subsubsection*{Activity Procedure}
\begin{enumerate}
\item{Cut the limes and squeeze them to collect the juice.}
\item{Add approximately 10 ml of water to the lime solution.}
\item{In the gas generator bottle, add half a tea spoon of sodium carbonate.}
\item{Add the diluted lime to juice to the sodium hydrogen carbonate. Close the bottle with the cap joined to the bottle of lime water.}
\item{Allow the gas to pass through the delivery tube and into a container holding about 5 mL of lime water. Continue passing until a change is noted.}
\item{Observe the lime water and record observations. Continue to pass carbon dioxide until the white precipitate disappears.}
\item{Collect another bottle of carbon dioxide gas from the gas generator.}
\item{Light a candle.}
\item{Pour the produced carbon dioxide over the burning candle.}
\item{Record observations.}
\item{Test the gas produced in this experiment with wet blue litmus paper. You must use a strong stream of gas, right as it is produced.}
\item{Record observations.}
\end{enumerate}

\subsubsection*{Results and Conclusion}
Carbon dioxide gas is prepared by the action of dilute acid on any carbonate. The gas is colourless and odourless. It turns lime water milky due to formation of calcium carbonate.  
CO2(g)  + Ca(OH)2(aq) ---> CaCO3(s)  +  H2O(l)
Passing excess carbon dioxide causes the milky colour to disappear due to formation calcium hydrogen carbonate which is soluble.
CaCO3(s)  +  H2O(l)  +  CO2(g)-->Ca(HCO3)2(aq)
Carbon dioxide gas turns blue litmus paper pink, showing that it is slightly acidic. It is used in extinguishing fire because it is denser than air and does not support combustion.
The gas is denser than air so you can pour from one bottle into another just like water. It does not support combustion though materials which burn to giving a lot of heat can split the gas and give out oxygen which can support combustion. For example burning Magnesium.
Mg(s) + CO2(g) --> MgO(s)  +  C(s)

\subsubsection*{Clean Up Procedure}
\begin{enumerate}
\item{Wash all bottles with clean water and soap.}
\item{Collect all the used materials, cleaning and storing items that will be used later. No special waste disposal is required.}
\end{enumerate}

\subsubsection*{Discussion Questions}
\begin{enumerate}
\item{What colour is carbon dioxide gas?}
\item{Why did lime water turn milky?}
\item{When excess carbon dioxide was passed through the lime water, why did the white colour disappear?}
\item{Does carbon dioxide gas support burning?}
\item{Discuss the density of carbon dioxide and the air.}
\item{Why is carbon dioxide gas used in fire extinguishers?}
\end{enumerate}

\subsubsection*{Notes}
Limes produce citric acid. White cement produces calcium hydroxide.

\subsection{Nitrogen}

Nitrogen is a colourless, ordourless, non-reactive gas. 78\% of the air is nitrogen and this is the primary source of nitrogen for industry. In the laboratory, nitrogen may be purified from the air by removing oxygen and carbon dioxide. This activity requires a more complicated set-up than other gas experiments and so is best performed as a demonstration by the teacher.

\subsubsection*{Objectives}
\begin{itemize}
\item{To prepare nitrogen gas from air.}
\end{itemize}

\subsubsection*{Background Information}
Air is 78\% nitrogen. One method of preparation of nitrogen gas is to remove all of the other components of air (oxygen, carbon dioxide, water, etc).

\subsubsection*{Materials}
Large water bottle (6L), sodium hydroxide*, 5 M sulphuric acid*, delivery tubes*, source of heat, piece of glass tube (about 20 cm), 4 empty water bottles (1 or 1.5 l), very thin copper wire

\subsubsection*{Hazards and Safety}
\begin{itemize}
\item{((Sodium Hydroxide))}
\item{((Battery Acid))}
\end{itemize}

\subsubsection*{Preparation}
\begin{enumerate}
\item{In the lids from two water bottles poke two holes. In the remaining water bottle lid poke a single hole.}
\item{Connect the delivery tubes in these holes using brio (pen) tubes as junctions.}
\item{Insert the copper foils or copper turnings inside the glass tube.}
\item{Prepare a 2M solution of caustic soda in one 1 L bottle with two holes in the cap.}
\item{Put sulphuric acid in the other bottle with two holes in the cap.}
\item{Prepare the heat source by lighting the stove.}
\item{Arrange the apparatus set up as in the diagram below (ILLUSTRATION) making sure that bottle C is squeezed (compressed) to remove air before compression.}
\end{enumerate}

\subsubsection*{Activity Procedure}
\begin{enumerate}
\item{Add water through the funnel into the 6 L bottle so as to displace air present in the bottle. This is done after the copper turning/foil starts to be red hot.}
\item{Observe what happens in the two water bottles A and B as well as the changes of the red hot copper turning in the combustion tube.}
\item{Observe the expansion of the bottle C as water fills the 6 L bottle.}
\item{Collect the gas in the bottle C by tightening the delivery tube.}
\end{enumerate}

\subsubsection*{Results and Conclusion}
Copper turnings turn red hot when heated in the absence of air (i.e. before water was added to the 6 L bottle. After the addition of water to 6 L bottle, bubbles were observed in both A and B and the red hot copper turnings turn black. The collection bottle (C) expanded. The black colour of copper indicates oxidation of copper by atmospheric oxygen. Copper oxide is black. The bubbles observed in the bottles indicates the passage of air into the solution.

\subsubsection*{Clean Up Procedure}
\begin{enumerate}
\item{Collect all the used materials, cleaning and storing items that will be used later. No special waste disposal is required.}
\end{enumerate}

\subsubsection*{Discussion Questions}
\begin{enumerate}
\item{What is observed in bottles A and B?}
\item{What do you think is the purpose of 2M NaOH and conc. H2SO4?}
\item{What is the function of hot copper in the combustion tube?}
\item{Nitrogen collected by this method is said to be impure. What do you think is the chief impurity?}
\item{What is the colour of the copper turnings after the collection of gas in bottle C? With the help of chemical equations explain the changes which occurred in the combustion tube.}
\end{enumerate}

\subsubsection*{Notes}
When air passes through the solution of sodium hydroxide, CO2 is absorbed to form carbonates:
2NaOH(aq) + CO2(g)-->Na2CO3(aq)+H2O(l)
Concentrated H2SO4 is a good drying agent for gases. It absorbs the atmospheric water vapour.
Heated copper absorbs the atmospheric oxygen to become copper oxide.
2Cu(s) + O2(g)-->2CuO(s)
The nitrogen obtained is not 100\% pure. It contains about 1\% by volume of the noble gases.

\subsection{Hydrogen}

Hydrogen is a colourless, ordourless, reactive gas. This activity presents a method for preparating and testing hydrogen gas and should be done by the teacher because hydrogen is highly flammable.

\subsubsection*{Objectives}
\begin{itemize}
\item{To prepare a sample of hydrogen gas.}
\item{To demonstrate the properties of hydrogen gas.}
\end{itemize}

\subsubsection*{Materials}
Zinc metal, empty water bottles, balloon, match box, 5 M sulphuric acid, and gas generator*.

\subsubsection*{Hazards and Safety}
\begin{itemize}
\item{Collect a small amount of hydrogen gas. When the gas is lit, point both ends of the bottle away from people.}
\item{ ((battery acid))}
\item{The bottle may fly in the air like a rocket - take care!}
\end{itemize}

\subsubsection*{Preparation}
\begin{enumerate}
\item{Take a piece of zinc and cut it into small pieces.}
\end{enumerate}

\subsubsection*{Activity Procedure}
\begin{enumerate}
\item{Put the zinc pieces into the bottle of a gas generator.}
\item{Add about 100 mL of battery acid.}
\item{Collect the gas by downward displacement of air in the collection bottle.}
\item{Remove the collection bottle, keeping the open end pointed downward.}
\item{Bring a flame to the mouth of the bottle and observe what happens.}
\end{enumerate}

\subsubsection*{Results and Conclusion}
When metals react with acids, hydrogen gas is liberated. The gas is colourless, ordourless, and inert to litmus. It is the lightest gas known and burns in air to produce water vapour with a "pop" sound. The gas is identified by this explosion.

\subsubsection*{Clean Up Procedure}
\begin{enumerate}
\item{Wash the bottles.}
\item{Dispose of all unwanted materials. Neutralize the left over battery acid with sodium hydrogen carbonate before putting in a sink.}
\end{enumerate}

\subsubsection*{Discussion Questions}
\begin{enumerate}
\item{What is the colour of the gas produced?}
\item{How do you identify hydrogen gas?}
\item{What is the product when hydrogen burns in the air?}
\end{enumerate}

\subsubsection*{Notes}
Battery acid is 5 M sulphuric acid. Other common metals that may be used for this experiment include magnesium (more reactive) and iron (less reactive). Copper metal will not displace hydrogen gas from normal acids because it is below hydrogen in the reactivity series.

\subsection{Chlorine}

Chlorine is a yellow-green gas that smells like bleach. The gas is poisonous and highly reactive. Chlorine is soluble in water and forms an acidic solution of chloric (I) acid (also called hypochlorous acid). Both the gas and the solution oxidizes pigments and therefore decolourizes both flowers and fabrics.

Below is an activity to prepare and test chlorine gas. This activity should be performed by the teacher due to the poisonous nature of the gas.

\subsubsection*{Objectives}
\begin{itemize}
\item{To prepare chlorine gas}
\item{To investigate the properties of chlorine gas}
\end{itemize}

\subsubsection*{Materials}
Bleach*, 5M sulphuric acid*, colored flowers, string or thread, large empty plastic water bottle.

\subsubsection*{Hazards and Safety}
\begin{itemize}
\item{Chlorine is very poisonous! Avoid inhalation. Only prepare chlorine gas in small quantities and in well ventilated areas or outside.}
\item{Emphasize the poisonous nature of the gas so no students attempt to prepare chlorine outside of school.}
\item{Bleach decolourizes clothing - take care to not spill any outside of the bottle.}
\item{((battery acid))}
\end{itemize}

\subsubsection*{Preparation}
\begin{enumerate}
\item{Collect coloured flowers.}
\item{Cut about 30 cm of string for each flower.}
\item{Tie one end of each string to each flower.}
\end{enumerate}

\subsubsection*{Activity Procedure}
\textit{This activity should be performed by the teacher}
\begin{enumerate}
\item{Put about 100 ml of Jik in the reaction bottle}
\item{Hang the flowers into the bottle using the strings. Tie the free end of the string around the neck of the bottle.}
\item{Add about 10 ml of battery acid to the reaction bottle. Quickly close the cap.}
\item{Put the bottle aside for 5 minutes.}
\item{Record your observations.}
\item{Test the gas evolved using moist litmus paper.}
\end{enumerate}

\subsubsection*{Results and Conclusion}
The greenish-yellow colour of chlorine should be clearly visible, especially against a white backgrounds. The chlorine gas will bleach the flowers in the bottle. The bleaching action is due to the formation of hypochlorite ion, formed when chlorine dissolves in water:

$$ \mathrm{Cl}_{2(g)} + \mathrm{H|_2\mathrm{O{ \longrightarrow \mathrm{HCl}_{(aq)} + \mathrm{HClO}_{(aq)} $$

$$ \mathrm{HClO} + \mathrm{'dye'} \longrightarrow \mathrm{'dye-O'} + \mathrm{HCl} $$

\subsubsection*{Clean Up Procedure}
\begin{enumerate}
\item{Move the bottle outside, far from people. Open it and leave it for several hours. Dispose of the solution inside far from people.}
\item{Collect all the used materials, cleaning and storing items that will be used later. No special waste disposal is required.}
\end{enumerate}

\subsubsection*{Discussion Questions}
\begin{enumerate}
\item{What colour is chlorine gas?}
\item{What happened to the flowers?}
\item{Why can we not collect chlorine gas over water?}
\item{Why is it important to always be aware of which chemicals you are mixing in a chemistry lab?}
\end{enumerate}

\subsection{Ammonia}

Ammonia is a colourless gas with a sharp, choking smell like that of old urine. The gas is reactive and harmful to inhale. Ammonia is highly soluble in water in which it forms basic solution of ammonium hydroxide.

Students will produce small quantities of ammonia themselves during qualitative analysis. The activity below is the demonstrate the preparation of a larger quantity of ammonia. Because of the amount of the gas and its harmful nature, this activity should be a demonstration by the teacher. If it is desired for students to perform this activity, reduce the amount of ammonium sulphate and have students perform this experiment in a test tube.

\subsubsection*{Objectives}
\begin{itemize}
\item{To prepare Ammonia gas}
\end{itemize}

\subsubsection*{Materials}
Gas generator*, ammonium sulphate*, sodium hydroxide*, red litmus paper*, match box, conc. HCl(optional), heat source*, heating vessel*

\subsubsection*{Hazards and Safety}
\begin{itemize}
\item{Avoid inhaling poisonous ammonia and hydrogen chloride gases.}
\item{((sodium hydroxide))}
\end{itemize}

\subsubsection*{Preparation}
\begin{enumerate}
\item{Prepare sodium hydroxide solution by dissolving approximately 1 spoon of sodium hydroxide in about 200 mL of water.}
\end{enumerate}

\subsubsection*{Activity Procedure}
\begin{enumerate}
\item{Put one tea spoon of ammonium sulphate into a heating vessel.}
\item{Add about 100 mL of sodium hydroxide solution to the ammonium sulphate and mix.}
\item{Warm the mixture using a heat source.}
\item{Test the gas evolved using moist red litmus paper.}
\item{Record the observations. Note the smell of the gas.}
\item{Optional: Bring a bottle containing conc. HCl acid, open it and allow fumes coming out of the bottle to react with the gas evolved from the gas generator.}
\item{Record the observations. Note the intensity of the fumes.}
\end{enumerate}

\subsubsection*{Results and Conclusion}
Ammonia is a colourless gas with pungent smell (smell of urine), it turns red litmus blue. It also forms white fumes when it comes into contact with HCl vapors.
Ammonia gas is the only alkaline gas known, it reacts with hydrogen chloride gas to form thick/dense white fumes of ammonium chloride(This is the identification of the gas). It is highly soluble in water, which is why it can't be collected over water. It is less dense than air that is why it is collected by downward displacement of air(upward delivery).

\subsubsection*{Clean Up Procedure}
\begin{enumerate}
\item{Collect all the used materials, cleaning and storing items that will be used later. No special waste disposal is required.}
\end{enumerate}

\subsubsection*{Discussion Questions}
\begin{enumerate}
\item{What is the colour of ammonia?}
\item{How do you identify the ammonia gas?}
\item{Write balanced chemical equation showing the reaction between ammonia and hydrogen chloride gas.}
\item{Why do you think ammonia is not collected over water?}
\end{enumerate}

\subsection{Sulphur Dioxide}

Sulphur dioxide is a colourless gas with a irritating sulphurous smell. The smell is different from that of hydrogen sulphide (rotten eggs) but has similar characteristics. Sulphur dioxide is both reactive and poisonous.

The following activity may be used to demonstrate for students the preparation and reactivity of sulphur dioxide. Due to the dangerous nature of the gas, this activity should be a demonstration, and not performed by the stuents themselves.

\subsubsection{Objectives}
\begin{itemize}
\item{To prepare sulphur dioxide gas}
\item{To obseve the acidity of sulphur dioxide gas and its solubility in water}
\item{To observe the oxidizing nature of sulphur dioxide gas}
\end{itemize}

\subsubsection{Hazards}
\begin{itemize}
\item{Sulphur dioxide gas is poisonous -- only prepare small quantities in closed containers and avoid breathing the gas}
\item{Burning sulphur dioxide}
\end{itemize}

\subsubsection{Preparation}
\begin{enumerate}
\item{Fill a beaker with water and keep it near the reaction bottle.}
\end{enumerate}

\subsubsection{Activity Procedure}
\begin{enumerate}
\item{In a 1.5 L water bottle, add about 50 mL of water followed by 3 drops of indicator.}
\item{Tie a flower to a string and suspend it in the bottle, just above the water.}
\item{Place a small amount of sulphur powder on a deflagrating spoon.}
\item{Light the sulphur powder on fire and promptly lower it into the bottle. Keep the burning sulphur above the flower and away from the plastic walls.}
\item{When the sulphur finishes burning, or if the gas produced becomes noticable by smell, remove the deflagrating spoon and put the sulphur end into the beaker of water. Do not touch the sides of the beaker with the hot deflagrating spoon.}
\item{Cap the bottle as soon as possible.}
\item{Let the bottle stand and watch for changes in the colour of the flower.}
\item{After ten minutes, shake the bottle to thoroughly mix the gas and the water. Observe the indicator for a colour change.}
\end{enumerate}

\subsubsection{Results and Conclusion}
The flower should be bleached by the sulphur dioxide. The demonstrates that oxidizing nature of sulphur dioxide. The indicator should turn an acid colour due to the formation of sulphuric (IV) acid, \ce{H2SO3}, from the reaction

$$\mathrm{SO}_{2(g)} + \mathrm{H}_2\mathrm{O} \longrightarrow \mathrm{H}_2\mathrm{SO}_{3(aq)}$$

\subsubsection*{Clean Up Procedure}
\begin{enumerate}
\item{Take the bottle of sulphur dioxide outside away from people. Open the bottle and lean it against a rock or tree with the open end down. After several hours, move the bottle to a rubbish pit.}
\end{enumerate}

\subsubsection*{Discussion Questions}
\begin{enumerate}
\item{Sulphur is found in coal. How does the burning of coal cause acid rain?}
\end{enumerate}
