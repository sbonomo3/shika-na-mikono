\documentclass[12pt,a4paper]{report}
\usepackage{amsmath}
\usepackage{amsfonts}
\usepackage{amssymb}
\usepackage{mhchem}
\usepackage{hyperref}
\author{Ministry of Education and Vocational Training}
\title{Using Local Available Science Materials in Chemistry Teaching Activities}
\setcounter{secnumdepth}{0}

\begin{document}

\begin{titlepage}
\begin{center}
\begin{huge}
ORDINARY LEVEL SECONDARY\\[4pt]
EDUCATION CHEMISTRY\\[4pt]
PRACTICALS\\[4pt]
\end{huge}
\end{center}
\vspace{.5in}

\textbf{\begin{center}
\begin{Huge}
USING LOCALLY\\[6pt]
AVAILABLE MATERIALS 
\end{Huge}
\end{center}}

\vfill
\begin{center}
\begin{Large}
TEACHER'S GUIDE
\end{Large}
\end{center}

\end{titlepage}

\chapter*{Preface}

\chapter*{Background}

\tableofcontents
\chapter{The Chemistry Laboratory}

\section{The Importance of Experimentation}

%==============================================================================
\section{Sources of Chemicals}
\label{cha:sourcesofchemicals}
%The following is a list of most of the chemicals used in ordinary science laboratories. For each we note local sources of these chemicals, low cost industrial sources of these chemicals, methods to manufacture these chemicals at your school, and/or functional alternatives to these chemicals.  We also list information like other names, common uses, and hazards. Finally, we include descriptions of many of the compounds and confirmatory tests for some to assist with identification of unlabelled chemicals. For more information on this, 
%see \nameref{cha:unknownchemicals}.

Chemicals are generally listed alphabetically by IUPAC name, 
although many compounds are also cross listed by their common name (e.g. 
acetone (common) / propanone (IUPAC)).

\subsection{Acetic acid}
See \nameref{sec:ethanoic}.
\subsection{Acetone}
See \nameref{sec:propanone}.
\subsection{Ammonia solution}
\label{sec:ammoniasol}
Formula: \ce{NH3}$_{(aq)}$\\
Other names: ammonium hydroxide, 
ammonium hydroxide solution\\
Description: clear liquid less dense than water, 
completely miscible in water, 
strong biting smell similar to old urine\\
Use: qualitative analysis, various experiments\\
Source: released from an aqueous mixture of ammonium salt and hydroxide, 
for example calcium ammonium nitrate and sodium hydroxide. 
The gas can be trapped and dissolved in water.\\
Alternative: to distinguish between zinc and lead cations, 
add dilute sulfuric acid dropwise. 
The formation of a white precipitate -- lead sulfate -- confirms lead.
Note: ammonia solution also is called ammonium hydroxide 
because ammonia undergoes autoionization to form ammonium and hydroxide ions. 
Just like water, 
there is an equilibrium concentration of the ions in an ammonia solution.
\subsection{Ammonium hydroxide solution}
See \nameref{sec:ammoniasol}.
\subsection{Ascorbic acid}
Other names: vitamin C\\
Formula: \ce{C6H7O7}\\
Description: white powder, 
but pharmacy tablets often colored\\
Confirm: aqueous solution turns blue litmus red 
AND decolorizes dilute iodine or potassium permanganate solution\\
Use: all-purpose reducing agent, 
may substitute for sodium thiosulfate in redox titrations, 
removes iodine and permanganate stains from clothing\\
Source: pharmacies
\subsection{Calcium ammonium nitrate}
Other names: \ce{CAN}\\
Description: small pellets, 
often with brown coating; 
endothermic heat of solvation\\
Use: low cost ammonium salt for teaching qualitative analysis; 
not as useful for teaching about nitrates 
as no red/brown gas released when heated. 
May be used for the preparation of ammonia and sodium nitrate.\\
Source: agricultural shops (fertilizer)\\
\subsection{Calcium carbonate}
Formula: \ce{CaCO3}\\
Description: white powder, 
insoluble in water
Confirm: brick red flame test and acid causes effervescence\\
Use: demonstration of reactivity of carbonates, 
rates of reaction, 
qualitative analysis\\
Source: coral rock, 
sea shells, 
egg shells, 
limestone, 
marble, 
white residue from boiling water\\
Local manufacture: prepare a solution of aqueous calcium 
from either calcium ammonium nitrate or calcium hydroxide 
and add a solution of sodium carbonate.\\ 
Calcium carbonate will precipitate and may be filtered and dried.
\subsection{Calcium hydroxide}
Formula: \ce{Ca(OH)2}\\
Other names: quicklime\\
Description: white to off white powder, 
sparingly soluble in water\\
Use: dissolve in carbonate-free water to make limewater\\
Source: building supply shops\\
Alternative: add a small amount of cement to water, 
let settle, 
and decant the clear solution; 
this is limewater.
\subsection{Calcium oxide}
Formula: \ce{CaO}\\
Other names: lime\\
Use: reacts with water to form calcium hydroxide, 
thus forming limewater\\
Source: cement is mostly calcium oxide
\subsection{Calcium sulfate}
Formula: \ce{CaSO4\cdot} 2\ce{H2O}\\
Other names: gypsum, 
plaster of Paris\\
Description: white powder, 
insoluble in cold water but soluble in hot water\\
Use: qualitative analysis\\
Source: building supply companies (as gypsum powder)
\subsection{Carbon (amorphous)}
Source: soot, 
charcoal (impure)
\subsection{Carbon (graphite)}
\label{sec:carbongraphite}
Use: element, \\
inert electrodes for chemistry and physics
Source: dry cell battery electrodes, 
pencil cores (impure)
\subsection{Carbon dioxide}
Preparation: react an aqueous weak acid 
(citric acid or ethanoic acid) with a soluble carbonate 
(sodium carbonate or sodium hydrogen carbonate)
\subsection{Citric acid}
Formula: \ce{C6H8O7} = \ce{CH2(COOH)COH(CHOOH)CH2COOH}\\
Description: white crystals soluble in water, 
endothermic heat of solvation\\
Use: all purpose weak acid, 
volumetric analysis, 
melting demonstration, 
preparation of carbon dioxide, 
manufacture of Benedict's solution\\
Hazard: acid – keep out of eyes!\\
Source: markets (sold as a spice often with a local name), 
supermarkets
\subsection{Copper}
\label{sec:copper}
Use: element, 
preparation of copper sulfate, 
electrochemical reactions\\
Description: dull red/orange metal\\
Source: electrical wire -- e.g. 
2.5~mm gray insulated wire has 50~g of high purity copper per meter.\\
Note: modern earthing rods are only copper plated, 
and thus no longer a good source of copper
\subsection{Copper sulfate}
Formula: \ce{CuSO4} (anhydrous), 
\ce{CuSO4\cdot} 5\ce{H2O} (pentahydrate)\\
Description: white (anhydrous) or blue (pentahydrate) crystals\\
Confirm: blue/green flame test 
and aqueous solution gives a white precipitate 
when mixed with lead or barium solution\\
Use: qualitative analysis, 
demonstration of the reactivity series, 
manufacture of Benedict's solution, 
test for water\\
Source: imported ``local'' medicine (manufactured in India).\\ 
Local manufacture: Electrolyze dilute (1-2~M) sulfuric acid 
with a copper anode and inert (e.g. 
graphite) cathode. 
Evaporate final solution until 
blue crystals of copper sulfate pentahydrate precipitate. 
To prepare anhydrous copper sulfate from copper sulfate pentahydrate, 
gently heat until the blue color has faded. 
Strong heating will irreversibly form black copper oxide. 
Store anhydrous copper sulfate in an air-tight container -- 
otherwise atmospheric moisture will reform the pentahydrate.
\subsection{Distilled water}
Formula: \ce{H2O} and nothing else!\\
Use: qualitative analysis\\
Source: rain water.\\
Allow the first 15 minutes of rain to clean off the roof 
and then start collecting water. 
In schools in dry climates, 
collect as much rain water as possible during the rainy season. 
Use it only for qualitative analysis, 
preparation of qualitative analysis reagents, 
and manufacture of qualitative analysis salts.\\ 
Distilled water may also be purchased at most petrol stations 
and automotive shops.\\
Local manufacture: Heat water in a kettle 
and use a rubber hose to bring the steam through a container of cold water. 
Collect the condensate -- pure water.\\
Alternative: river or tap water is almost always sufficient. 
Volumetric analysis never needs distilled water 
if you follow the instructions in Relative Standardization. 
Also, 
the tap water in many places is sufficient for even qualitative analysis.
\subsection{Ethanoic acid}
\label{sec:ethanoic}
Formula: \ce{CH3COOH}\\
Other names: acetic acid\\
Description: clear liquid, 
completely miscible with water, 
strong vinegar smell\\
Use: all purpose weak acid, 
volumetric analysis\\
Source: 96\% solution available from village industry supply shops, 
vinegar (5\% solution) available in small shops and supermarkets\\
Safety for 96\% ethanoic acid: HARMFUL VAPORS. 
Use outside or in a well ventilated space. 
CORROSIVE ACID. 
Always have dilute weak base solution (e.g. 
sodium hydrogen carbonate) available to neutralize spills. 
Wear gloves and goggles when handling. 
Do not induce vomiting if ingested.\\
Alternative: for a weak acid, 
citric acid. 
\subsection{Ethanol}
Formula: \ce{CH3CH2OH}\\
Description: clear liquid, 
completely miscible with water, 
strong and sweet alcohol smell\\
Use: solvent, 
extraction of chlorophyll, 
removes permanent marker, 
preparation of POP solution, 
distillation, 
preservation of biological specimens\\
Hazard: ethanol itself is a mild poison, 
and methylated spirits and other industrial alcohol contain 
additional poisonous impurities (methanol) 
specifically so that no one drinks it\\
Sources: methylated spirits are 70\% ethanol, 
hard liquor is often 30-40\%, 
village-brewed concentrated alcohol varies 
and may contain toxic quantities of methanol\\
Local manufacture: fermentation of sugar by yeast will produce 
up to a 15\% solution -- at that point, 
the yeast dies; 
distillation can in theory concentrate this to up to 95\%, 
but this is hard with simple materials. 
Nevertheless, 
preparing ethanol of sufficient concentration to dissolve POP (50-60\%) 
is quite possible.\\
Note: the color of most methylated spirits makes them undesirable 
for preparation of POP; 
hard liquor will suffice, 
but poorly because of its relatively low ethanol content. 
Colored methylated spirits can be run 
through a simple distillation apparatus to produce colorless spirits, 
as the pigment is less volatile than the ethanol. 
Of course, 
methanol and other poisons remain, 
but the clear solution works beautifully for dissolving POP.\\ 
Beware that ethanol vapors are flammable -- 
a poorly constructed distillation setup may explode.
\subsection{Graphite}
See \nameref{sec:carbongraphite}.
\subsection{Hydrochloric acid}
\label{sec:hydroacid}
Formula: \ce{HCl}, 
36.5~g/mol, 
density 1.18~g/cm$^{3}$ when concentrated ($\sim$12~M)\\
Other names: muriatic acid, 
pH decreasing compound for swimming pools\\
Description: clear liquid, 
may be discolored by contamination, 
distinct smell similar to chlorine 
although sometimes smells strongly of vinegar\\
Confirm: decolorizes weak solutions of potassium permanganate; 
white precipitate in silver nitrate solution 
and effervescence with (hydrogen) carbonates\\
Use: volumetric analysis, 
qualitative analysis\\
Source: swimming pool chemical suppliers, industrial chemical\\ 
Safety: HARMFUL VAPORS. 
Use outside or in a well ventilated space. 
CORROSIVE ACID. 
Always have dilute weak base solution (e.g. 
sodium hydrogen carbonate) available to neutralize spills. 
Wear gloves and goggles when handling. 
Extremely toxic hydrogen cyanide gas formed 
on mixing with cyanides or hexacyanoferrate compounds. 
Toxic chlorine gas formed on reaction with oxidizing agents, 
especially bleach. 
Do not induce vomiting if ingested.\\
Alternative (strong acid): sulfuric acid\\
Alternative (acid): citric acid\\
Alternative (qualitative analysis): for the test for carbonates, 
use dilute sulfuric acid; 
to dissolve insoluble carbonates, 
nitric acid may be used instead
\subsection{Hydrogen}
Formula: \ce{H2}\\
Confirm: ``pop sound,'' i.e. 
ignites with a bang; 
in an inverted test tube the rapid movement of air 
near the mouth creates a rapid, 
high pitch ``whoosh'' that gives the ``pop'' name\\
Preparation: combine dilute acid (e.g. 
battery acid) and a reactive metal (steel wool or zinc) 
in a plastic water bottle. 
Attach a balloon to the top of the water bottle; 
being less dense than air, 
hydrogen will migrate up and slowly fill the balloon. 
Specific instructions for various alternatives are available 
in the Hands-On activities section. 
Before ignition, 
always move the balloon away from the container of acid.
\subsection{Hydrogen peroxide}
Formula: \ce{H2O2}\\
Description: solutions are colorless liquids 
appearing exactly like water\\
Confirm: decolorizes potassium manganate (VII) solution 
in the absence of acid, 
neutral pH\\
Use: preparation of oxygen, 
general oxidizer and also may act as a reducing agent (e.g. 
with potassium permanganate)\\
Source: pharmacies sell 3\% (10 volume) and 6\% (20 volume) solutions 
as medicine for cleaning sores\\
Note: `20 volume' means it will produce 20 times its liquid volume in oxygen gas.
\subsection{Indicator}
\label{sec:indicator}
Source: red flowers\\
Preparation: Crush flower petals in water. 
Some effective flowers include rosella, 
bougainvillea, 
and hibiscus. 
Test other flowers near your school.\\
Note: For bougainvillea and some other flowers, 
extract the pigment with ethanol 
or hard alcohol to get a better color. 
Color will change from pink (acidic) to colorless (basic). 
Rosella will change from red (acidic) to green (basic).
For an indicator in redox titrations involving iodine, 
see starch solution.
\subsection{Iodine}
Formula: \ce{I2}$_{(s)}$\\
Description: purple/black crystals\\
Local manufacture: add a little dilute sulfuric acid 
to iodine solution from a pharmacy. 
Then add sodium hypochlorite solution (bleach) dropwise 
until the solution turns colorless with solid iodine resting on the bottom. 
The solid iodine can be removed by filtration or decantation. 
If pure iodine is necessary, 
the solid may be purified by sublimation.\\
Note: this reaction produces poisonous chlorine gas. 
Therefore, 
produce iodine in a well ventilated area and stand upwind.
\subsection{Iodine solution}
\label{sec:iodinesol}
Composition: \ce{I2} + \ce{KI} dissolved in water and sometimes ethanol\\
Description: light brown solution\\
Confirm: turns starch blue or black\\
Use: food tests for detection of starch and fats\\
Source: pharmacies sell a ‘weak iodine solution’ 
or ‘tincture of iodine’ that is really about 50\% by mass iodine. 
To prepare a useful solution for food tests, 
dilute this 10:1 in ordinary water.\\
Note: to use this solution for detection of fats, 
it must be made without ethanol, 
spirits, 
alcohol and the like. 
Either kind works for detection of starch.
\subsection{Iron}
\label{sec:iron}
Use: element, 
demonstration of reactivity series, 
preparation of hydrogen, 
preparation of iron sulfide, 
preparation of iron sulfate\\
Source: for samples of the element 
and for use in electrochemical experiments, 
buy non-galvanized nails at a hardware store, 
or find them on the ground. 
You can tell they are not galvanized because they are starting to rust. 
Clean off the rust with steel wool prior to use. 
For samples of the element for preparation of other compounds, 
buy steel wool from small shops or supermarkets. 
This has a very high surface area / mass ratio, 
allowing for faster reactions.
\subsection{Magnesium sulfate}
\label{sec:magsulfate}
Formula: \ce{MgSO4\cdot} 7\ce{H2O}\\
Other names: epsom salts\\
Description: white or clear crystals\\
Use: crystallization experiments, 
qualitative analysis test reagent 
(confirmation of hydrogen carbonate and carbonate), 
precipitation reactions\\
Source: livestock and veterinary supply shops sell Epsom salts 
to treat constipation in cattle
\subsection{Manganese (IV) oxide}
Formula: \ce{MnO2}\\
Other names: manganese dioxide\\
Description: black powder\\
Confirm: liberates oxygen from hydrogen peroxide\\
Use: preparation of oxygen, 
qualitative analysis (confirmation of chlorides)\\
Source: old dry cell batteries (radio batteries)\\
Extraction: smash a dry cell battery with a rock 
and scrape out the black powder. 
This is a mixture of manganese dioxide, 
zinc chloride, 
and ammonium chloride. 
This impure mixture is suitable for the preparation of oxygen. 
To purify manganese dioxide for use in qualitative analysis, 
boil the powder in water to dissolve away the chlorides. 
Filter the solution after boiling 
and repeat if the test gives false positives (e.g. 
confirms chlorides in samples that lack chlorides)\\
Note: Wash your hands with soap if you accidentally touch the powder. 
Do not get it on your clothes or into cuts on your hands. 
\ce{MnO2} causes metal to corrode; 
if you use a metal tool to scrap out the powder, 
be sure to clean it off afterwards. 
Better: use non-metal tools. 
\subsection{Organic solvents}
Sources: kerosene, 
petrol, 
paint remover, 
paint thinner and the safest: cooking oil
\subsection{Oxygen}
Confirm: oxygen gas relights a glowing splint, 
i.e. 
a piece of wood or paper glowing red / orange 
will flame when put in a container 
containing much more oxygen than the typical 20\% in air\\
Preparation: combine hydrogen peroxide 
and manganese (IV) oxide in a plastic water bottle. 
Immediately crush the bottle to remove all other air and then cap the top. 
The bottle will re-inflate with oxygen gas.
\subsection{Potassium iodide}
\label{sec:potiodide}
Formula: \ce{KI}\\
Description: white crystals very similar in appearance to common salt, 
endothermic heat of solvation\\
Confirm: addition of weak potassium permanganate 
or bleach solution causes a clear KI solution to turn yellow/brown 
due to the formation of \ce{I2} (which then reacts with \ce{I-} to form soluble \ce{I3-})\\
Use: preparation of iodine solution for food tests in biology, 
preparation of iodine solutions for redox titrations, 
confirmatory test for lead in qualitative analysis\\
Local manufacture: Heat a pharmacy iodine tincture strongly until 
only clear crystals remain. 
In this process, 
the \ce{I2} will sublimate -- 
placing a cold dish above the iodine solution should cause must of the iodine 
to deposit as solid purple crystals. 
Note that the iodine vapors are harmful to inhale.
If you need \ce{KI} for a solution that may contain impurities, 
add ascorbic acid solution to dilute iodine tincture 
until the solution exactly decolorized.\\
Alternative (food tests): see \nameref{sec:iodinesol}\\
Alternative (redox titrations): 
often you can also use iodine solution for this; 
just calibrate the dilution of pharmacy tincture 
and the other reagents to create a useful titration\\
Alternative (qualitative analysis): 
confirm lead by the addition of dilute sulfuric acid -- 
white lead sulfate precipitates
\subsection{Potassium manganate (VII)}
Formula: \ce{KMnO4}\\
Other names: potassium permanganate, 
permanganate\\
Description: purple/black crystals, 
sometimes with a yellow/brown glint, 
very soluble in water -- 
a few crystals will create a strongly purple colored solution\\
Hazard: powerful oxidizing agent -- 
may react violently with various compounds; 
solutions stain clothing (remove stains with ascorbic acid solution); 
crystals and concentrated solution discolor skin 
(the effect subsides after a few hours, 
but it is better to not touch the chemical!)\\
Use: strong oxidizer, 
self-indicating redox titrations, 
identification of various unknown compounds, 
diffusion experiments\\
Source: imported ``local'' medicine. 
Also sold in very small quantities in many pharmacies. 
May be available in larger quantities from hospitals.\\
Alternative (oxidizer): bleach (sodium hypochlorite), 
hydrogen peroxide\\
Alternative (diffusion experiments): solid or liquid food coloring, 
available in markets and small shops
\subsection{Propanone}
\label{sec:propanone}
Formula: \ce{H3CCOCH3}\\
Other names: acetone\\
Description: clear liquid miscible in water, 
smells like nail polish remover, 
quickly evaporates\\
Use: all-purpose lab solvent, 
iodoform reaction (kinetics, organic chemistry)\\
Hazard: highly flammable\\
Source: nail polish remover (mixture with ethyl ethanoate)\\
Alternative (volatile polar solvent): ethanol, 
including methylated spirits
\subsection{Sodium carbonate}
Formula: \ce{Na2CO3\cdot} 10\ce{H2O} (hydrated), 
\ce{Na2CO3} (anhydrous)\\
Other names: soda ash, washing soda\\
Description: white powder completely soluble in water\\
Use: all-purpose cheap base, 
volumetric analysis, 
qualitative analysis, 
manufacture of other carbonates\\
Safety: rather caustic, keep off of hands and definitely out of eyes!\\
Source: commercial and industrial chemical supply -- 
should be very inexpensive\\
Local manufacture: dissolve sodium hydrogen carbonate in distilled water 
and boil for five or ten minutes 
to convert the hydrogen carbonate to carbonate. 
Let evaporate until crystals form. 
For volumetric analysis, 
the hydrated salt may always substitute 
for the anhydrous with a correction to the concentration -- 
see Chemical Substitutions for Volumetric Analysis
\subsection{Sodium chloride}
Formula: \ce{NaCl}\\
Other names: common salt\\
Use: all-purpose cheap salt, 
qualitative analysis\\
Source: the highest quality salt in markets (white, 
finely powdered) is best. 
The iodine salts added to prevent goiter 
do not generally affect experimental results.
\subsection{Sodium hydrogen carbonate}
Formula: \ce{NaHCO3}\\
Description: white powder, 
in theory completely soluble in cold water 
in practice often dissolves poorly\\
Other names: sodium bicarbonate, 
bicarbonate of soda\\
Use: all-purpose weak base, 
preparation of carbon dioxide, 
qualitative analysis\\
Source: small shops \\
Note: may contain ammonium hydrogen carbonate
\subsection{Sodium hydroxide}
Formula: \ce{NaOH}\\
Other names: caustic soda\\
Description: white deliquescent crystals -- 
will look wet after a minute in contact with air 
and will fully dissolve after some time, 
depending on humidity and particle size\\ 
Use: all-purpose strong base, 
volumetric analysis, 
food tests in biology, 
qualitative analysis, 
preparation of sodium salts of weak acids\\
Hazard: corrodes metal, 
burns skin, 
and can blind if it gets in eyes\\
Source: industrial supply shops, 
supermarkets, 
hardware stores (drain cleaner)\\
Local manufacture: mix wood ashes in water, 
let settle, 
and decant; 
the resulting solution is mixed sodium and potassium hydroxides 
and carbonates and will work for practicing volumetric analysis\\
Note: ash extracts are about 0.1~M base and may be concentrated by boiling; 
this is dangerous, 
however, 
and industrial caustic soda is so inexpensive 
and so pure that there is little reason to use ash extract 
other than to show that ashes are alkaline 
and that sodium hydroxide is not exotic.
\subsection{Sodium hypochlorite solution}
Formula: \ce{NaOCl}$_{(aq)}$\\
Other names: bleach\\
Use: oxidizing agent\\
Source: small shops, 
supermarkets\\
Local manufacture: electrolysis of concentrated salt water solution 
with inert (e.g. 
graphite) electrodes; 
4-5~V (three regular batteries) is best for maximum yield\\
Note: commercial bleach is usually 3.5\% sodium hypochlorite by weight
\subsection{Sulfur}
Description: light yellow powder with distinct sulfurous smell\\
Use: element, 
preparation of iron sulfide\\
Source: large agricultural shops (fungicide, 
e.g. 
for dusting crops), 
imported ``local'' medicine
\subsection{Sulfuric acid}
Formula: \ce{H2SO4}\\
Other names: battery acid\\
Description: clear liquid with increasing viscosity at higher concentrations; 
fully concentrated sulfuric acid ($\sim$18~M) is almost twice as dense as water 
and may take on a yellow, 
brown, 
or even black color from contamination\\
Use: all-purpose strong acid, 
volumetric analysis, 
qualitative analysis, 
preparation of hydrogen and various salts\\
Source: battery acid from petrol stations 
is about 4.5~M sulfuric acid and one of the least expensive sources of acid\\
Hazard: battery acid is dangerous; 
it will blind if it gets in eyes and will put holes in clothing. 
Fully concentrated sulfuric acid is monstrous, 
but fortunately never required. 
For qualitative analysis, 
``concentrated'' sulfuric acid means $\sim$5~M -- battery acid will suffice.\\
Note: ``dilute'' sulfuric acid should be about 1~M. 
To prepare this from battery acid, 
add one volume of battery acid to four volumes of water (e.g. 
100~mL battery acid + 400~mL water)
\subsection{Zinc}
\label{sec:zinc}
Description: firm silvery metal, 
usually coated with a dull oxide\\
Use: element, 
preparation of hydrogen, 
preparation of zinc carbonate and zinc sulfate\\
Source: dry cell batteries; 
under the outer steel shell is an inner cylinder of zinc. 
In new batteries, 
this whole shell may be extracted. 
In used batteries, 
the battery has consumed most of the zinc during the reaction, 
but there is generally an unused ring of zinc around the top 
that easily breaks off. 
Note that alkaline batteries, 
unlike dry cells, 
are unsafe to open -- and much more difficult besides.

%==============================================================================
\section{Laboratory Organization}

\subsection{Organizing and Labelling Chemicals}

\subsection{Warning Signs}
%==============================================================================
\section{Manufacturing Apparatus}

((Read through activities are list here every apparatus with a * and instructions for its manufacture from locally available materials))

%==============================================================================
\section{Laboratory Safety}

\subsection{Laboratory Rules}

\subsection{First Aid}

[INSERT ACTIVITY HERE]

%==============================================================================
\section{Heat Sources}

Heat and flames are important for teaching chemistry. This section discusses low cost options for both heating solutions and higher temperature flames. This section also discusses safety specific to heating and fire. Finally, this section includes activities to teach students about luminous and non-luminous flames as well as basic fire fighting technique.

\subsection{Options for Heat Sources}

((Describe kerosene and motopoa stoves, how to use them, etc))


\subsection{Safety with Heat Sources}

((Describe simple safety precautions for working with fire))

\subsection{Types of Flames}
Many chemistry practicals involve the use of flame for heating, thus it is important for students to understand the different ways to produce flames in the laboratory. There are two kinds of flame: luminous and non-luminous. A luminous flame is one that produces a lot of light. The best flame in the chemistry lab is a non-luminous flame because it is hotter and does not produce soot. Unfortunately, in books the most common method given for producing the two types of flame is a bunsen burner. This activity is written to introduce students to different types of flames without the use of a bunsen burner. This activity is best done by students in small groups but the teacher should supervise the groups closely as they will be dealing with fires.
\subsubsection*{Learning Objectives}
\begin{itemize}
\item{To produce luminous and non-luminous flames from different fuel burners.}
\item{To demonstrate the combustion of various substances in air.}
\end{itemize}

\subsubsection*{Materials}
kerosene burner*, spirit burner*, motopoa burner*, candle, spirit, kerosene, paper, motopoa, metal spoon, matches, metal jam-jar lid

\subsubsection*{Hazards and Safety}
\begin{itemize}
\item{((FIRE))}
\end{itemize}

\subsubsection*{Activity Procedure}
\begin{enumerate}
\item{Light a kerosene burner and observe the flame. Adjust the heights of the wicks to change the appearance of the flame. Hold the metal spoon over the flame for a few seconds and then examine it for soot.}
\item{Light the spirit burner (if no spirit burner is available, a small amount of spirit can be poured into a soda bottle lid and lit). Observe the flame. Hold the metal spoon over the flame for a few seconds and examine it for soot.}
\item{Light the motopoa burner. Observe the flame. Hold the metal spoon over the flame for a few seconds and examine it for soot.}
\item{Light the candle. Observe the flame. Hold the metal spoon over the flame for a few seconds and examine it for soot.}
\item{Put a small piece of paper into the metal lid and light it. Observe the flame. Hold the spoon over the flame for a few seconds and examine it for soot.}
\end{enumerate}

\subsubsection*{Results and Conclusion}
The kerosene produces a luminous flame. Adjusting the wick length can change the nature of the flame. When the wicks are very long you shold observe a bigger and brighter flame that is more likely to produce soot. If the metal cover is removed, the flame will be luminous and produce a lot smoke.
Spirit burns as a non-luminous flame that is colourless. The flame will be nearly invisible and will not produce soot.
Motopoa burns as a non-luminous flame. The flame will not produce soot.
A candle produces a luminous flame that deposits soot on the spoon.
A piece of paper produces a luminous flame that deposits soot on the spoon.
\subsubsection*{Discussion Questions}
\begin{enumerate}
\item{Which substances produced soot when burned?}
\item{Identify which substances produce luminous flames? Which substances produce non-luminous flames?}
\item{Which of the materials tested is best for heating in the chemistry laboaratory? Give reasons.}
\item{Which of the materials tested is worst for heating in the chemistry laboaratory? Give reasons.}
\end{enumerate}

\subsection{Fire Fighting}
Many chemistry activities involve the use of fire as a heat source. It is very important that students have an understanding of how to extinguish various fires in the laboratory. The most common sources of fire in the chemistry laboratory are kerosene, spirit, motopoa and paper/wood. The common available materials for extinguishing fires are water, sand and exhaled carbon dioxide. This activity give students the chance to explore and identify the best methods for extinguishing fires. This information will be very useful in case of an accident during later activities. This activity can be a useful demonstration but it is most effective if performed by students in small groups so they gain experience and knowledge on how to respond in case of a fire.
\subsubsection*{Learning Objectives}
\begin{itemize}
\item{SWBAT demonstrate the proper use of available materials in extinguishing various types of fire in the laboratory.}
\end{itemize}

\subsubsection*{Materials}
kerosene, ethanol*, motopoa*, candle, paper, sand, water, syringe, soda bottle tops, matches, glass jar, beaker

\subsubsection*{Hazards and Safety}
\begin{itemize}
\item{((FIRE))}
\end{itemize}

\subsubsection*{Activity Procedure}
\begin{enumerate}
\item{Put a small amount of ethanol into a bottle cap and light it with a match. Pour water onto the flame.}
\item{Dry the cap and add a small amount of ethanol and light it again with a match. Put a handful of sand onto the flame.}
\item{Clean the cap and add a small amount of ethanol and light it again with a match. invert a glass jar over the flame and wait.}
\item{Put a small amount of kerosene into a bottle cap and light it with a match. Using a syringe, \textit{carefully} add a few drops of water near the base of the flame.}
\item{To the kerosene flame ad a handful of sand.}
\item{Clean the cap and add a small amount of kerosene and light it again with a match. Invert a glass jar over the flame and wait.}
\item{Fill a beaker with water.}
\item{Light a piece of paper on fire. Immediately dip it in the beaker of water.}
\end{enumerate}

\subsubsection*{Results and Conclusion}
A flame caused by burning ethanol is extinguished by the addition of water or sand. When the glass jar is inverted over the flame it will be deprived of oxygen and slowly go out. This is a useful demonstration because there a flame in a container, it can be extinguished by closing or covering the container.
A kerosene fire can NOT be extinguished by water. Because water is immiscible with kerosene, addition of water will only cause the fire to spread. Sand can be used to extinguish the kerosene flame.
Paper is easily extinguished by water.

%==============================================================================
\chapter{The Nature of Matter}

Chemistry is the study of matter. All of the higher lessons depend on students' understanding of the basics of matter. Much of the study of chemistry is based on three simple postulates:
\begin{itemize}
\item{Matter is composed of extremely small particles called molecules.}
\item{Attraction forces cause these molecules to stick together.}
\item{Thermal energy keeps these particles in constant motion, sometimes shaking them apart from each other.}
\end{itemize}

%The rest of the study of chemistry is based on the structure of the atom, the manner in which electrons attract other atoms to form chemical bonds, and the nature of these bonds.

\section{Concept of Matter}

Everything we see is made up of extremely tiny particles called molecules. A molecule is the smallest unit of a substance that retains the chemical properties of the substance. Emphasis should be given to the extremely small nature of molecules: in 1 litre of water there are more than 33,000,000,000,000,000,000,000,000,000 (`33 billion billion billion') molecules of water! Write this number on the board for students to think about.


\section{States of Matter}

Molecules are always in motion, although their movements are usually restricted by forces attraction to other molecules. The balance between thermal energy and intermolecular attraction determines the state of matter.
There are three states of matter: solid, liquid, and gas. A solid has fixed volume and fixed shape. A liquid has fixed volume but will take the shape of its container. A gas will take the volume and the shape of its container. A `vapour' is another word for gas often used for compounds that are normally liquid or solid at room temperature. For example, we might talk about `water vapour' or `zinc vapour,' but generally not `oxygen vapour.'


\subsection{Changes of State}
By changing the temperature of a substance, we can change the state it exists in. As a solid object is heated the molecules begin to shake faster and faster until they break free of the rigid solid substance and become a free-flowing liquid. This is called melting. A change of state is a physical change which involves a substance going from one state to another. 
((ILLUSTRATION))

Changes of state are something that students see in everyday life (a melting candle, drying clothes). This activity is good for getting students to think about these everyday changes in terms of the molecules which make up every substance. The activity demonstrates melting, solidifying, vaporization, and condensing. If enough candles are available, the students can do the first activity and make observations in small groups. Because of materials, time and safety it might be ideal to do the vapourization and condensing of water as a class demonstration and discussion.

\subsubsection*{Learning Objectives}
\begin{itemize}
\item{To describe the states of matter.}
\item{To demonstrate changes in state of various material.}
\end{itemize}

\subsubsection*{Materials}
candle, heating vessels*, water, condenser, knife, heat source

\subsubsection*{Hazards and Safety}
\begin{itemize}
\item{Never heat a tightly sealed container.}
\end{itemize}

\subsubsection*{Preparation}
\begin{enumerate}
\item{Fill a condenser with cold water.}
\end{enumerate}

\subsubsection*{Activity Procedure}
\begin{enumerate}
\item{Take a candle and light it using a match box. Have students make observations about any change they observe.}
\item{Put about 20 mL of water in the heating vessel and heat for 5-10 minutes. Have students make observations.}
\item{Put about 50 mL in a heating vessel and connect with the condenser. Heat until a liquid is noticed at the other end of the condenser. Collect this liquid. Have students make observations.}
\end{enumerate}

\subsubsection*{Results and Conclusion}
When the candle burns it changes from solid to liquid. When the candle cools the wax will harden, returning to solid state.
Boiling water is a change from a liquid to a vapour in gaseous state. Condensing water vapour changes it back from gas to liquid.

\subsubsection*{Clean Up Procedure}
\begin{enumerate}
\item{Collect all the used materials, cleaning and storing items that will be used later. No special waste disposal is required.}
\end{enumerate}

\subsubsection*{Discussion Questions}
\begin{enumerate}
\item{Explain what change of state is involved when burning candle wax and boiling water.}
\item{what happens when water is heated and passed through a condenser?}
\end{enumerate}

\subsubsection*{Notes}

\section{Physical and Chemical Changes}

Changes in state are an example of a physical change. A physical change is a change that only involves rearranging molecules. Chemists also study chemical changes, in which chemical bonds are broken and remade, and molecules change from one kind to another. The following activity may be used with students to illustrate the difference between physical and chemical changes.

\subsection{Physical and Chemical Change}

\subsubsection*{Learning Objectives}
\begin{itemize}
\item{SWBAT demonstrate physical and chemical changes of matter experimentally.}
\end{itemize}


\subsubsection*{Materials}
A piece of paper, sugar, spoon, bar magnet, two iron nails, steel wool, kerosene stove and match box.

\subsubsection*{Activity Procedure}
\begin{enumerate}
\item{Take a piece of paper and light it on fire using a match box. Have students record their observations.}
\item{Take a small amount of sugar into the spoon and heat until a clear chemical change is observed. Have students record their observations.}
\item{Take a bar magnet and rub one nail in one direction only. [DIAGRAM]}
\item{ Place the second (un-magnetized) nail hear the iron fillings . Repeat with the nail that has been rubbed with a magnet.}
\item{Take the magnetized nail and rub it again in the opposite direction and place in iron fillings observe what will happen.}
\end{enumerate}

\subsubsection*{Results and Conclusion}
Burning paper and burning sugar are examples of chemical changes. These means that chemical bonds are broken and remade and new products are formed. But when the nail is rubbed with the bar magnet no chemical bonds are broken or made. The nail becomes magnetized and thus attracts the iron fillings. This is an example of a physical change.

\subsubsection*{Clean Up Procedure}
\begin{enumerate}
\item{Collect all the used materials, cleaning and storing items that will be used later. No special waste disposal is required.}
\end{enumerate}

\subsubsection*{Discussion Questions}
\begin{enumerate}
\item{Explain the changes in the paper and sugar. Name the type of change.}
\item{Explain the process of rubbing a nail with a bar magnet and what happen when it was rerubbed?}
\item{Why did the two nail behave differently? Name the change that happened in the rubbed nail.}
\end{enumerate}

\subsubsection*{Notes}

\section{Compounds and Mixtures}

The mixture of iron and sulphur before heating in the above activity is a good example of a mixture. Mixtures can be separated by purely physical means - no chemical bonds need to be broken or made. After heating the mixture of iron and sulphur, a new chemical compound formed - chemical bonds were broken and new ones were made. This section deals with mixtures, where chemicals are mixed together but chemical reactions have not yet occurred.

\subsection{Solutions, Suspensions and Emulsions}
When a liquid is combined with another substance, three kinds of mixtures are possible: solution, suspension or emulsion. 
A solution is a mixture where the one component (solute) dissolves completely in the other (solvent) to form a transparent liquid. A suspension is formed when a small pieces of solid are suspended or dispersed throughout a liquid without dissolving in it. The solute particles in the suspension remain visible.  An emulsion is a fine dispersion of small droplets of one liquid in another in which it is not soluble or miscible. 
Solutions, suspensions and emulsions are something students encounter in every day life- salt water is a solution, river water is a suspension and milk is an emulsion.
This activity gives students a chance to make their own examples of each type of mixture and compare them side by side. This activity is best done by students in small groups so that each students gets a chance to make clear observations-it is hard to see the difference between a solution and a suspension from far away.

\subsubsection*{Learning Objectives}
\begin{itemize}
\item{To prepare examples of solution, suspension and emulsion.}
\item{Explain the properties of solutions, suspensions and emulsion.}
\end{itemize}

\subsubsection*{Materials}
Cooking oil, water, methylated spirit, test tubes*, beakers, plastic bottles with lids, copper (II) sulphate, clay soil.

\subsubsection*{Preparation}
\begin{enumerate}
\item{Grind the solid clay to make a fine powder.}
\end{enumerate}

\subsubsection*{Activity Procedure}
\begin{enumerate}
\item{Instruct students to mix equal volumes of cooking oil and water in a test tube and shake it vigorously.}
\item{Direct students to allow the solution in the test tube to settle. Have them record their observations.}
\item{Instruct students to add about a mL of methylated spirit to the test tube and again shake vigorously.}
\item{Students should again allow the test tube to settle and record their results. The mixture formed should be an emulsion.}
\item{Tell students to add a spoon of clay to a test tube of water and shake vigorously then set the test tube down and allow the solution to settle.}
\item{Direct students to add half of a spoonful of copper (II) sulphate to about 200 mL of water and shake vigorously until no more crystals are visible.}
\item{Instruct student to allow the mixture to sit for some minutes and record their observations.}
\end{enumerate}

\subsubsection*{Results and Conclusion}
In the first experiment oil and water clearly form two layers because they are immiscible liquids. Upon addition of methylated spirit and shaking, the two liquids form an emulsion. The emulsion is not transparent, you can not be seen in it.
The clay mixes with water to form suspension which after some time will settle to the bottom of the container. The suspension is not transparent and individual clay particles can clearly be seen.
The copper sulphate crystals dissolve in water completely to form a blue solution which does not settle no matter how long it is allowed to sit. The solution is transparent and individual particles can not be seen.

\subsubsection*{Clean Up Procedure}
\begin{enumerate}
\item{Collect the copper (II) sulphate solution, store in a labelled bottle and save for later use.}
\item{Collect all the used materials, cleaning and storing items that will be used later. No special waste disposal is required.}
\end{enumerate}

\subsubsection*{Discussion Questions}
\begin{enumerate}
\item{Did the oil and water separate after the addition of methylated spirit?}
\item{Can you see through the emulsion formed? Why?}
\item{Did the solid clay settle out of the clay mixture?}
\item{Can you see through the clay mixture?}
\end{enumerate}

\subsubsection*{Notes}
An emulsion can also be created by shaking a sodium hydroxide solution with cooking oil, however this is more dangerous than the procedure listed.

\section{Separation of Mixtures}

There are many methods of separating mixtures. Many are used in daily life. This topic is a good way to teach students that the chemistry they use in the classroom has application when they go home. This section presents different methods for separating mixtures and an activity to use for teaching each. Many of these methods will be important for more complicated experiments, so both students and teachers should be familiar with them.

\subsection{Decantation}
Decantation is an easy way to separate a suspension. Leave the suspension to settle, so the solid falls to the bottom of the container. Then, the liquid can be poured off the top. This method is applied in the home, for example, when rice is rinsed and the excess water is poured off. Students can perform this activity in a very short time; it is best combined with the filtration activity.


\subsection{Filtration}

Filtration is a more complicated but more effective method for separating a suspension. A suspension is filtered by passing it through a material with holes so small that only the liquid can pass through. The solid is left behind.

\subsection{Evaporation}

Evaporation is a simple method to separate solutions where one chemical is most more volatile than the other, for example a solution of salt in water.

\subsection{Separation of Mixtures: Decantation, Filtration and Evaporation}
This activity combines the principles of decantation, filtration and evaporation to produce a pure sample of salt from a mixture of salt, water, sand and stones. This activity is best done by student in small groups of 2-4.

\subsubsection*{Learning Objectives}
\begin{itemize}
\item{SWBAT separate  a mixture using the decantation, filtration, evaporation.}
\end{itemize}

\subsubsection*{Materials}
Water, sand, salt, small pebbles, beakers*, funnel*, clean cloth and a plastic bottle
\subsubsection*{Preparation}
\begin{enumerate}
\item{Make a mixture by combing water, salt, sand and pebbles an a plastic bottle. Close the lid and shake}
\end{enumerate}
\subsubsection*{Activity Procedure}
\begin{enumerate}
\item Shake the bottle well and observe the solution. Allow the bottle to sit for a few minutes so that the pebbles and some dirt settle to the bottom.
\item Pour the solution into a clean beaker, leaving the pebbles and dirt in the bottom of the bottle. You should now have a sample of water with some dirt particles.
\item Put the clean cloth in the filter and pour the dirty water into the filter. Make sure that all of the water passes through the cloth and not around the edges. Collect the water in a clean beaker. If dirt is still present in the water, filter it again through a fresh piece of cloth.
\item Put a small amount of this water into a clean beaker and leave it in the sun for a few hours. Make sure it is put in a place that is protected from wind so dirt does not enter. Once all of the water has evaporated collect the beaker and examine its contents.
\end{enumerate}

\subsubsection*{Results and Conclusion}
Decantation should separate the large pebbles and pieces of rocks. The filtration should remove all of the dirt particles and leave a clear salt water solution. After evaporation, salt crystals should remain in the beaker.

\subsubsection*{Clean Up Procedure}
\begin{enumerate}
\item{Collect all the used materials, cleaning and storing items that will be used later. No special waste disposal is required.}
\end{enumerate}

\subsubsection*{Discussion Questions}
\begin{enumerate}
\item{Give 3 examples each of decantation and filtration being used in daily life.}
\item{One problem with the method of evaporation is that the liquid is lost and cannot be collected again. Which method of separation could be used to separate salt water if pure water was the desired product.}
\end{enumerate}


\subsection{Construction of a Sand Filter}
In many villages there is no access to clean drinking water. People are forced to drink water that is unsafe or use a lot of fuel for boiling water. Many students are already aware of the process of boiling and filtering water to make it safe for drinking. On a larger scale, sand filters can be used to purify water for a group of people or even a village. When built properly on a large scale sand filters are even effective for removing micro-organisms which cause disease. In this activity students build small-scale sand filters and attempt to filter their own dirty water.  This activity is best done by students in small groups.

\textbf{NOTE}: The sand filters built in the laboratory are only an example, they are \textit{not} effective for removing bacteria from dirty water.
\subsubsection*{Learning Objectives}
\begin{itemize}
\item{To demonstrate the purification of water by using a sand filter.}
\end{itemize}

\subsubsection*{Materials}
Fine sand, coarse sand, small pebbles, large pebbles, charcoal, empty water bottle, beaker, sieve* and dirty water.

\subsubsection*{Preparation}
\begin{enumerate}
\item{Use water to rinse all of the dirt off of the pebbles.}
\item{Put all of the sand in a sieve and shake to remove excess dirt. In the sieve, rinse the pebbles with water.}
\item{In a beaker, rinse the coarse sand with clean water and decant off the dirty water. Repeat until the water decanted is clear.}
\item{Cut the bottom off of a water bottle so that is it shaped like a funnel.}
\item{Invert the water bottle and fill the bottom with large pebbles, making sure that the botom pebble is larger than the hole in the bottle so that it doesnt fall out. Put a layer of smaller pebbles on top of the large pebbles.}
\item{Put the coarse sand particles on top of the pebbles, followed by charcoal and then small sand particles at the top (see diagram).}
((ILLUSTRATION))
\item{Run clean water through the filter until it comes out clear at the bottom.]
\end{enumerate}

\subsubsection*{Activity Procedure}
\begin{enumerate}
\item{Pour some dirty water into the top of the sand filter.}
\item{Collect the water in a clean beaker from the bottom of the filter.}
\end{enumerate}

\subsubsection*{Results and Conclusion}
When dirty water passes through sand particles, impurities are trapped and remain above. The smallest particles and some micro-organisms are stopped by the charcoal layer. Ideally, the water coming out of the bottom of the filter should be clear. 

\subsubsection*{Clean Up Procedure}
\begin{enumerate}
\item{Collect all the used materials, cleaning and storing items that will be used later. No special waste disposal is required.}
\end{enumerate}

\subsubsection*{Discussion Questions}
\begin{enumerate}
\item{How can you distinguish water before purification and water after purification?}
\item{List 3 substances which can make water impure.}
\item{Why is charcoal used in this process?}
\item{Why is water purification important?}
\end{enumerate}



\subsection{Separation of Mixtures: Simple Distillation}
Simple distillation is a more complicated method for separating solutions that allows the liquid to be collected. This method involves vaporizing the liquid, then condensing it by cooling the vapor, and collecting the resulting liquid.  The following activity involves a slightly complicates set-up so it is best done as a demonstration. If food colour is added to the water, even students in the back will be able to see the process.
\subsubsection*{Learning Objectives}
\begin{itemize}
\item{To be able to separate a solution by application of simple distillation}
\end{itemize}

\subsubsection*{Materials}
Condenser, water, food colouring, heating vessel*, beaker, and heat source*.

\textbf{Notes}:

\begin{itemize}
\item{If food colouring powder is not available it is also possible to use a coloured compound like potassium permanganate or a salt. If a salt is used, it might be difficult to show that the distallate is salt free.}
\item{If a condenser is not available it is still possible to do this experiment. Connect a long delivery tube to a stopper on the heating vessel. The air temperature surrounding the tube will be cool enough to cause some water to condense which can then be collected in a beaker. If this method is used, a large amount of water will also be lost as steam.
\end{itemize}

\subsubsection*{Preparation}
\begin{enumerate}
\item{Make a coloured solution by adding a small amount of colour to a sample of water.}
\item{Put cold water into the condenser.}
\end{enumerate}

\subsubsection*{Activity Procedure}
\begin{enumerate}
\item{Heat the container holding coloured water over the heat source for some time.}
\item{Collect any water out of the condenser and compare it to the original solution.}
\end{enumerate}

\subsubsection*{Results and Conclusion}
When the solution is heated, the water is vaporized while the coloured particles remain in solution. As the water vapour passes through the condenser tube is it cooled and condenses back to liquid form. The water will drip out of the end of the tube and will be pure, that is, containing no dissolved solutes. The final solution should contain no colour.

\subsubsection*{Clean Up Procedure}
\begin{enumerate}
\item{Collect all the used materials, cleaning and storing items that will be used later. No special waste disposal is required.}
\end{enumerate}

\subsubsection*{Discussion Questions}
\begin{enumerate}
\item{Why does the final solution appear different than the initial solution?}
\item{In what situation is simple distillation a better method of separation than evaporation?}

\end{enumerate}

\subsection{Separation of Mixtures: Chromatography}
Chromatography is the separation of a mixture by passing it in solution or suspension through a medium in which the components move at different rates. This method can be used to separate different coloured pigments in a coloured ink. Because each pigment has a different molecular shape, they can be dissolved in a solvent and separated. As the solvents moves through a medium (such as paper) the different molecules have different attraction to the paper and the solvent. Molecules with a higher attraction to the paper will move slower. After some time, the molecules will be separated, giving distinct bands of colour.

In this activity students are able to separate the different colour pigments present in the ink of a marker or pen. This activity should be done by students in small groups.
\subsubsection*{Learning Objectives}
\begin{itemize}

\item{To separate colours using paper chromatography.}

\end{itemize}

\subsubsection*{Materials}
Piece of white paper, cotton wick (utambi wa jiko), petri dish*, colourless methylated spirit, small beaker* and one red marker or ball pen.

\subsubsection*{Preparation Procedure}
\begin{enumerate}
\item{Cut piece of paper to a size a little bigger than the top of the beaker. Put a small hole in the center of it.}
\item{Prepare a wick that is long enough to reach to the bottom of the beaker.}
\end{enumerate}

\subsubsection*{Activity Procedure}
\begin{enumerate}
\item{Pour the colourless spirit into the beaker.}
\item{Make a hole in the center of the piece of paper you have prepared and insert the thread through the hole. ((ILLUSTRATION NEEDED)) }
\item{Using a red marker/ball pen draw a circle around the hole and put the piece of paper on the beaker containing methylated spirit and make sure the end of the wick is immersed in the liquid.}
\item{Leave it to sit until the spirit has caused separation of the colours (this should be about 10 minutes).}
\item{Repeat the experiment using different markers or pens.}
\end{enumerate}

\subsubsection*{Results and Conclusion}
Methylated spirit acts as a solvent. The solvent climbs up the wick and then spreads across the paper.  When the solvent reaches the ink, the different pigments dissolve and move at different speeds and thus are spearated. The colours observed for a red pen may be yellow, pink and purple.


\subsubsection*{Clean Up Procedure}
\begin{enumerate}
\item{Collect all the used materials, cleaning and storing items that will be used later. No special waste disposal is required.}
\end{enumerate}

\subsubsection*{Discussion Questions}
\begin{enumerate}
\item{Explain the use of the thread, methylated spirit and marker pen to this experiment.}
\item{How many colors are there on the piece of paper at the end of the experiment.}
\end{enumerate}


\subsection{Separation of Mixtures: Layer Separation}

\subsection{Layer Separation}

\subsubsection*{Learning Objectives}
\begin{itemize}
\item{SWBAT separate two immiscible liquids using layer separation.}
\end{itemize}

\subsubsection*{Background Information}


\subsubsection*{Materials}
Kerosene, water, separatory funnel*.

\subsubsection*{Activity Procedure}
\begin{enumerate}
\item{Measure the approximately equal volume of water and kerosene using the small beaker and pour them in the separating funnel and shake them well.}
\item{Leave the mixture for some time to settle and run off the lower layer.}
\end{enumerate}

\subsubsection*{Results and Conclusion}
When water and kerosene was mixed and left to settle the two solution formed the immiscible solution. Kerosene forms the upper layer while water forms the lower layer because water is denser than kerosene.

\subsubsection*{Clean Up Procedure}
\begin{enumerate}
\item{Collect all the used materials, cleaning and storing items that will be used later. Kerosene should not be disposed. Put it in a well-labelled bottle and store for later use.}
\end{enumerate}

\subsubsection*{Discussion Questions}
\begin{enumerate}
\item{What are immiscible liquids?}
\item{In a kerosene/water mixture, which liquid is on top? Why?}
\end{enumerate}

\subsubsection*{Notes}
Polar liquids dissolve other polar liquids-for example ethanol can easily be dissolve in water. Non-polar liquids dissolve non-polar liquids.  Polar and non-polar liquids do not mix-when combined together they form immiscible layers of liquid. Kerosene is non-polar and water is polar, thus when mixed they form two layers.

\subsection{Separation of Mixtures: Solvent Extraction}

[INSERT ACTIVITY HERE]

%==============================================================================

\chapter{Chemical Reactions}

Chemical reactions are chemical changes - chemical bonds are broken and remade. This section includes exampls of three common types of reactions - thermal decomposion, decomposition, and precipitation - and activities to guide students to better understand these reactions. There is also a discussion of what happens when something dissolves and another activity to do with students.

\subsection{Thermal Decomposition}

[INSERT ACTIVITY HERE]

\subsection{Binary Combination}

\subsubsection*{Learning Objectives}
\begin{itemize}
\item{To demonstrate a binary combination reaction.}
\item{To write a balanced equations for the binary combination of iron and sulphur.}
\end{itemize}

\subsubsection*{Background Information}


\subsubsection*{Materials}
sulphur*, steel wool, bar magnet, source of heat*, tea spoon, open light bulb and aluminium plate.

\subsubsection*{Hazards and Safety}
\begin{itemize}
\item{Perform this experiment in a well ventilated room and prevent inhalation of fume formed. Sulphur dioxide will be formed which is poisonous.}
\end{itemize}

\subsubsection*{Preparation}
\begin{enumerate}
\item{Grind the steel wool using a spoon on a hard surface so that fine particles are obtained.}
\item{Carefully open a light bulb so that it can be used for heating.}
\end{enumerate}

\subsubsection*{Activity Procedure}
\begin{enumerate}
\item{Put half a teaspoon of iron particles in a beaker and add the same amount of sulphur powder. Mix them well. Label this mixture 1.}
\item{Instruct one student to put a bar magnet into the mixture and allow the class to observe the results. The iron fillings should separate from the mixture.}
\item{In an open light bulb put a few iron fillings, add sulphur powder to it and mix them well. Label this mixture 2. Heat the mixture until the sulphur powder sublimes.}
\item{Again have a student come to the front and use a bar magnet to try to separate the two components physically from the mixture and record the results.}
\item{To another open bulb put few iron fillings, add sulphur powder to it and mix them. Label this mixture 3. Heat the mixture strongly for an extended time.}
\item{Use a a bar magnet and to try separate the two components physically from the mixture and record the results. If nothing is observed on the bar magnet, continue heating and try again.}
\end{enumerate}

\subsubsection*{Results and Conclusion}
The components in the open light bulb 1 were able to be separated by physical means using a bar magnet. After heating mixture 2 the magnet was unable to separate the components. After heating mixture three, the bar magnet was able to separate out some iron.
In mixture 1 it was able to separate the components from the mixture by physical means using a bar magnet since the components were not reacted. In the mixture 2 it was impossible to separate the components physically because the reaction occurred forming a compound FeS.
Furthermore in the bulb 3 it was possible for the bar magnet to show its effect because during heating all the sulphur turned to sulphur dioxide gas SO2, leaving behind Iron fillings.

\subsubsection*{Clean Up Procedure}
\begin{enumerate}
\item{Collect all the used materials, cleaning and storing items that will be used later. No special waste disposal is required.}
\end{enumerate}

\subsubsection*{Discussion Questions}
\begin{enumerate}
\item{What do you understand by the term binary combination?}
\item{Write the equation for the binary combination of Iron and sulphur.}
\item{Why was there no effect of the bar magnet in mixture 2?}
\end{enumerate}

\subsubsection*{Notes}
Iron (II) Sulphide FeS is a black, insoluble compound usually used for the preparation of hydrogen sulphide. It is prepared by heating Iron with sulphur in the calculated quantities.
Iron (II) sulphide occurs as a hard, brassy mineral. Often known as fool's gold. It is the cheapest source of sulphur dioxide which givesn off when burn in air.

\subsection{Precipitation Reaction}

\subsubsection*{Learning Objectives}
\begin{itemize}
\item{To write chemical equations for the precipitation of insoluble salt.}
\item{To explain the meaning of a precipitation reaction.}
\end{itemize}

\subsubsection*{Background Information}


\subsubsection*{Materials}
copper sulphate* or magnesium sulphate*, sodium carbonate*, beakers*, funnel*, filter paper*

\subsubsection*{Preparation}
\begin{enumerate}
\item{Make a solution of copper or magnesium sulphate by dissolving about 1 spoonful in 500 mL of water.}
\item{Make a solution of sodium carbonate by dissolving about 1 spoonful in 500 mL of water.}
\end{enumerate}

\subsubsection*{Activity Procedure}
\begin{enumerate}
\item{Take one empty beaker and add about 10 mL of sulphate solution.}
\item{To the same beaker add about 10 mL of sodium carbonate solution.}
\item{Leave the mixture to settle for 5-10 minutes.}
\end{enumerate}

\subsubsection*{Results and Conclusion}
When copper sulphate or magnesium sulphate solution is mixed with sodium carbonate a precipitate will form. Copper carbonate is a blue precipitate, magnesium carbonate is a white precipitate. This reaction is useful for preparing insoluble salts.
The chemical reactions are 
CuSO4(aq) Na2CO3(aq)--CaCO3(s) Na2SO4(aq)
MgSO4(aq) Na2CO3(aq)--MgCO3(s) Na2SO4(aq)

\subsubsection*{Clean Up Procedure}
\begin{enumerate}
\item{Collect all the used materials, cleaning and storing items that will be used later. No special waste disposal is required.}
\end{enumerate}

\subsubsection*{Discussion Questions}
\begin{enumerate}
\item{What precipitate is formed in this reaction?}
\item{Write a balanced chemical equation for this reaction.}
\end{enumerate}

\subsubsection*{Notes}


\subsection{Solubility Experiments}

\subsection{Solubility}

\subsubsection*{Learning Objectives}
\begin{itemize}
\item{SWBAT compare the solubility of different substances in water and organic solvents.}
\end{itemize}

\subsubsection*{Background Information}


\subsubsection*{Materials}
Sugar, kerosene, sodium chloride*, steel wool, iodine crystals*, potassium permanganate crystals*, water, beaker*, test tube*.

\subsubsection*{Preparation Procedure}
\begin{enumerate}
\item{Break the steel wool into small piece.}
\end{enumerate}

\subsubsection*{Activity Procedure}
\begin{enumerate}
\item{In a test tube, combine about 1 mL of kerosene and 1 mL of water. Write down your observations.}
\item{Put half a spoonful of sugar into about 10 mL of kerosene in beaker labeled kerosene. Make observations.}
\item{Put half a spoonful of sugar into about 10 mL of water in a beaker labeled water. Compare the solubility of sugar in kerosene and water.}
\item{Dump out the sugar water solution and decant the kerosene into a beaker for later use.}
\item{Repeat the test using salt and steel wool, each time dumping the water solution and saving the kerosene.}
\item{Repeat with iodine and potassium permanganate, this time using only a few crystals. Record observations.}
\end{enumerate}

\subsubsection*{Results and Conclusion}
Sugar and salt dissolve in water but not kerosene. Iron does not dissolve in either solvent. Potassium permanganate dissolves in water to make a deep purple solution but does not dissolve in kerosene. Iodine dissolves only a small amount in water but in kerosene dissolves to make a deep red solution.

\subsubsection*{Clean Up Procedure}
\begin{enumerate}
\item{Kerosene can be decanted to remove any solid and stored for later use. Kerosene containing iodide should be left in an open container away from people until it has completely evaporated.}
\item{Potassium permanganate can be stored in a container for later use. If it is disposed, it should be reduced with vitamin c (ascorbic acid) prior to being dumped in the drain.}
\end{enumerate}

\subsubsection*{Discussion Questions}
\begin{enumerate}
\item{What is the solubility of sugar, iron and salt in water and organic solvent?}
\item{Compare the solubilities of potassium permanganate and iodine in water and kerosene.}
\end{enumerate}

\subsubsection*{Notes}
Water is a polar substance. Kerosene is non-polar. In solubility the general rule is "like dissolves like". Polar substances like water dissolve polar or ionic solutes like salt, sugar and potassium permanganate. Non-polar solvents like kerosene dissolve non-polar substances like iodine. Most organic substances are non-polar hence can not dissolve in water. Kerosene does not dissolve water and water does not dissolve kerosene, hence they form two layers when mixed.


%==============================================================================

\chapter{Water}

Water is fundamental to life. It is also essential in the chemistry laboratory. This section includes a series of activities to teach students about water - how to test for it, how to purify it, and much about hard and soft water.

\subsection{Test for Water}

\subsubsection*{Learning Objectives}
\begin{itemize}
\item{Students will understanding the meaning of the words `hydrated' and `anhydrous.'}
\item{To test for the presence of water.}
\end{itemize}

\subsubsection*{Background Information}
Copper (II) sulphate exists in two forms: hydrated (with water) and anhydrous (without water). Hydrated copper (II) sulphate is blue while anhydrous copper (II) sulphate is white. Thus, white copper sulphate can be used as a test for water.

\subsubsection*{Materials}
Heat source*, copper (II) sulphate*, water, metal spoon

\subsubsection*{Activity Procedure}
\begin{enumerate}
\item{Instruct students to place a very small amount of blue copper (II) sulphate in a metal spoon.}
\item{Supervise students heating the spoon gently over a heat source. Students will stop heating when the crystals have changed from blue to white.}
\item{Instruct students to add a few drops of water to the white crystals. Students will observe and record any colour change.}
\end{enumerate}

\subsubsection*{Results and Conclusion}
On heating blue hydrated copper (II) sulphate, the colour changes from blue (CuSO4-5H2O) to white (CuSO4). On addition of a few drops of water CuSO4 returns to its original hydrated state (blue), i.e. copper sulphate pentahydrated.

\subsubsection*{Clean Up Procedure}
\begin{enumerate}
\item{Copper (II) sulphate crystals can be left in the air to dry the excess water and then used again in future experiments.}
\item{Collect all the used material, cleaning and storing items that will be used again later. No special waste disposal is required.}
\end{enumerate}

\subsubsection*{Discussion Questions}
\begin{enumerate}
\item{Explain what happens to the blue copper (II) sulphate when it is heated. What happened when water was added.}
\item{Write a balanced chemical equation for this reaction.}
\item{When anhydrous copper (II) sulphate is allowed to sit out for 30 minutes its colour changes from white to light blue. Explain why this happens.}
\end{enumerate}

\subsubsection*{Notes}

\subsection{Water Purification}

Natural water may contain harmful organisms and substances. Water purification is necessary before drinking in order to removed harmful micro-organisms and dirt particles. Water treatment is the process of making water safe and usable for either domestic, industrial or medical purpose. This activity allows students to practice proper water treatment.



\subsubsection*{Objectives}
\begin{itemize}
\item{To demonstrate proper treatment of drinking water.}
\end{itemize}

\subsubsection*{Materials}
clean piece of white cloth, boiling vessel (sufuria), heat source, impure water, bucket

\subsubsection*{Activity Procedure}
\begin{enumerate}
\item{Make the source of heat by lighting the stove.}
\item{Heat the water to boiling. Allow water to boil for 5 minutes.}
\item{Allow water to cool.}
\item{Use the clean white cloth to filter the water into a clean bucket. Water is now safe for drinking.}
\end{enumerate}

\subsubsection*{Results and Conclusion}
The act of boiling water at the boiling point (100 C) kills the germs and bacteria which may cause disease. Filtration with a piece of clean white cloth removes any solid impurities. The filtrate optioned is drinkable water.

\subsubsection*{Discussion Questions}
\begin{enumerate}
\item{Define the following terms (a) water treatment (b) water purification.}
\item{It is advised to let water boil for at least 5 minutes. Why?}
\item{Explain how the filtered water is different from the original water.}
\end{enumerate}

\subsubsection*{Notes}
Instead of a cloth, it is possible to use chemical purifiers for filtration such as "water guard" or "aqua guard".

\subsection{Differences between Soft Water and Hard Water}

The names soft water and hard water were developed before people understood chemistry. Even today the average person is familiar with the difference between water from place to place. In some places it is easy to use soap - it forms a soft lather easily and cleans hands and clothes effectively. In these places the water is called soft water. In other places, it is hard to use soap - much effort is required to get the lather required for cleaning. In these places the water is said to be hard water.

Now we know what causes some water to be hard water and other water to be soft water. Hard water contains dissolved magnesium or calcium ions. Soft water does not. While we cannot feel the difference between plain soft water and plain hard water, the difference is clear when using soap. Dissolved magnesium and calcium ions bind to the soap, forming a `scum' that is not effective for cleaning. Only with much soap can a useful lather be formed.

The following activity is useful for students to experience the difference between soft water and hard water. Students should perform this activity in small groups.

\subsubsection*{Objectives}
\begin{itemize}
\item{To differentiate between soft water and hard water.}
\end{itemize}

\subsubsection*{Materials}
Gypsum powder*, epsom salt*, table salt, beakers*, sponge, and a bar of soap

\subsubsection*{Preparation}
\begin{enumerate}
\item{Label 5 beakers: CaSO4 solution, MgSO4 solution, NaCl solution, MgSO4 + CaSO4 solution and soft water.}
\end{enumerate}

\subsubsection*{Activity Procedure}
\begin{enumerate}
\item{Dip a sponge and soap into one solution after another while rubbing them together for several seconds.}
\item{Record the result from each beaker.}
\end{enumerate}

\subsubsection*{Results and Conclusion}
The three beakers labelled CaSO4 solution, MgSO4 solution and CaSO4 + MgSO4 solution all form soap scum, while the other two beakers containing soft water and NaCl solution easily formed a lather with soap.

\subsubsection*{Clean Up}
\begin{enumerate}
\item{Collect all the used materials, cleaning and storing items that will be used later. No special waste disposal is required.}
\end{enumerate}

\subsubsection*{Discussion Questions}
\begin{enumerate}
\item{What causes hardness of water?}
\item{How can you differentiate hard water from soft water?}
\end{enumerate}

\subsubsection*{Notes}
Permanent hardness of water is caused by the presence of Ca+2 and Mg+2 ions in solution. The two ions react with soap to form insoluble substance called scum. The formation of scum destroy the soap and prevent the formation of lather. Ca+2 and Mg+2 ions present in soluton and destroys the soap as follows;
CaSO4 (aq) + 2 NaCl (aq)...... CaSt (aq) + Mg2SO4
MgSO4(aq) + 2NaSt(aq) ........ MgSt(aq) + NaSO4.

\subsection{Temporary Hard Water}

There are two kinds of hard water: temporary hard water and permanent hard water. A long time ago, people thought that only temporary hard water could be treated, whereas permanent hard water would always be hard. Now we know how to treat both kinds of hard water.

Temporary hard water has hydrogen carbonate anions. When temporary hard water is boiled, the hydrogen carbonate decomposes to form carbonate. Neither magnesium carbonate nor calcuim carbonate is soluble in water, so these salts precipitate. Therefore temporary hard water can be boiled to produce soft water.

The following activity is useful for students to observe the treatment of temporary hard water by boiling. Students should perform this activity in small groups, one for each available heat source.

\subsubsection*{Objectives}
\begin{itemize}
\item{To treat hard water by heating.}
\end{itemize}

\subsubsection*{Materials}
beakers*, heating vessel*, magnesium sulphate*, sodium hydrogen carbonate*, kerosene stove, match box, soft water (rain/distilled water), filter paper*, funnel* and sodium carbonate*.

\subsubsection*{Hazards and Safety}
\begin{itemize}
\item{((FIRE))}
\end{itemize}

\subsubsection*{Preparation}
\begin{enumerate}
\item{Make an open bulb and use it as heating vessel.}
\end{enumerate}

\subsubsection*{Activity Procedure}
\begin{enumerate}
\item{In a beaker labeled hard water then put one spoon full of magnesium sulphate salts followed by two spoon full of sodium hydrogen carbonate and water up to height of 3-4 cm high and stir well to mix.}
\item{In a second beaker fill about 2-3 cm height of beaker full of water and add two spoon full of washing soda and stir well to dissolve.}
\item{Put soft water in the third beaker.}
\item{Take a small amount (about 5-10 mL) of hard water and boil until no more precipitate is formed. Remove from heat and let it cool.}
\item{Fold the filter paper once and the middle, repeat again second fold the make it fit to the funnel for the filtration.}
\item{Filter the boiled solution into a beaker labeled filtrate.}
\item{Put a few drops of sodium carbonate into soft water, hard water and filtrate. Note the changes.}
\end{enumerate}

\subsubsection*{Results and Conclusion}
Magnesium sulphate mixed with sodium hydrogen carbonate solution to produce temporary hard water which is magnesium hydrogen carbonate. Boiling temporary hard water causes precipitation of magnesium carbonate which is the way of removing the magnesium ion which are the ones causing hard water. This type of hardness can also be removed by precipitation method by addition of washing soda. When the washing soda was added to hard water there was precipitate of magnesium carbonate.

\subsubsection*{Clean Up Procedure}
\begin{enumerate}
\item{Collect all the used materials, cleaning and storing items that will be used later. No special waste disposal is required.}
\end{enumerate}

\subsubsection*{Discussion Questions}
\begin{enumerate}
\item{Define temporary hardness of water.}
\item{Why were magnesium sulphate and sodium hydrogen carbonate mixed? Write the chemical formulae of the resulting solution.}
\item{What is the precipitate formed during boiling?}
\item{Describe why there was precipitate in only one beaker.}
\item{What are other ways of removing hardness of water?}
\end{enumerate}

\subsubsection*{Notes}
Temporary hard water is often described as containing calcium hydrogen carbonate and magnesium hydrogen carbonate. This is a source of confusion. Temporary hard water indeed contains calcium/magnesium ions as well as hydrogen carbonate ions and thus can correctly be called a solution of calcium hydrogen carbonate (or magnesium hydrogen carbonate). Note however that neither calcium hydrogen carbonate nor magnesium hydrogen carbonate exists as a solid chemical. Thus the preparation of temporary hard water in the laboratory is generally accomplished by adding calcium and magnesium with one salt and hydrogen carbonate with another.

\subsection{Treating Hard Water by Precipitation}

Permanent hard water contains magnesium and calcium ions without hydrogen carbonate ions. Generally the charge of the magnesium and calcium ions is balanced with chlorides and/or sulphates. Because there are no hydrogen carbonate ions, boiling permanent hard water has no effect on the hardness. To treat permanent hard water, one must add carbonate ions directly, generally as sodium carbonate. Because many people use sodium carbonate for softening water for use with soap, sodium carbonate is often called `washing soda.'

The following activity is useful for students to experience treating hard water by addition of sodium carbonate, also called treatment by precipitation.

\subsubsection{Notes}
The addition of sodium carbonate will treat both permanent hard water and also temporary soft water.

%==============================================================================

\chapter{Fire}

The investigation of chemistry often involves heating. In most laboratories, substances are heating over a fire. Therefore, it is important to understand how fires work. Fires are also part of every day life.

\subsection{Investigating the Requirements for Combustion}

\subsubsection*{Learning Objectives}
\begin{itemize}
\item{SWBAT identify the factors necessary for combustion to take place.}
\end{itemize}

\subsubsection*{Background Information}


\subsubsection*{Materials}
A candle, A match box, two beakers.

\subsubsection*{Preparation Procedure}
\begin{enumerate}
\item{Cut the candle into two pieces about 5-7 cm each and place them in each of the beakers. [ILLUSTRATION NEEDED]}
\end{enumerate}

\subsubsection*{Activity Procedure}
\begin{enumerate}
\item{Light the candle and leave them to light for some time.}
\item{Cover one beaker while candle still lighting so as prevent entering of air (oxygen).}
\item{Leave the second beaker open so there is a continuos supply of air (oxygen). Observe what happen.}
\end{enumerate}

\subsubsection*{Results and Conclusion}
In the second beaker left opened the candle will go on burning for a long time because there is a continuous oxygen supply to the flame. In the second covered beaker the candle will go out just after covering because there is no oxygen supply.

\subsubsection*{Clean Up Procedure}
\begin{enumerate}
\item{Collect all the used materials, cleaning and storing items that will be used later. No special waste disposal is required.}
\end{enumerate}

\subsubsection*{Discussion Questions}
\begin{enumerate}
\item{Explain what happened in the two beakers and why.}
\item{Write down the condition necessary for combustion to take place.}
\item{What are the product of  combustion? Justify your answer.}
\end{enumerate}

\subsubsection*{Notes}


\subsection{Investigating the Products of Combustion}

Combustion is the rapid chemical combination of a substance with oxygen, producing heat and light. Most combustion activities involve burning compounds containing carbon and hydrogen. The carbon in the compound combines with oxygen to form carbon dioxide and the hydrogen combines with oxygen to form water. Therefore we expect carbon dioxide and water to be the products of  
most combustion activities.

The following activity is to demonstrate the production of carbon dioxide in one case of combustion: the burning of a candle. This activity requires about half an hour to show results, should it should be started at the beginning of the lesson and discussed near the end.

\subsubsection*{Objectives}
\begin{itemize}
\item{To identify carbon dioxide as a products of combustion.}
\end{itemize}

\subsubsection*{Materials}
Candle, match box, beaker*, lime water*, deflagrating spoon*

\subsubsection*{Activity Procedure}
\begin{enumerate}
\item{Pour 10-20 mL of lime water in the beaker.}
\item{By using a deflagrating spoon lower the candle into the lime water container so the flame will sit above the surface of the solution.}
\item{Light the candle and leave it to stay there for about 30 minutes and observe what happens to the lime water.}
\end{enumerate}

\subsubsection*{Results and Conclusion}
When the candle burns the lime water turns milky in colour, showing that there is production of carbon dioxide gas.

\subsubsection*{Clean Up Procedure}
\begin{enumerate}
\item{Collect all the used materials, cleaning and storing items that will be used later. No special waste disposal is required.}
\end{enumerate}

\subsubsection*{Discussion Questions}
\begin{enumerate}
\item{Why does lime water turn milky when the candle burns?}
\item{What will happen to the lime water when the candle burns for more than 30 minutes in production of excess carbon dioxide?}
\end{enumerate}

\subsubsection*{Notes}
The products of combustion are light, heat energy, carbon dioxide gas and soot.These products can be observed experimentally as demonstrated.
\subsection{Finding the Heat of Combustion for Different Fuels}

[INSERT ACTIVITY HERE]

\subsection{Rusting}

Rusting is not combustion because the reaction is not rapid and no light is produced. Chemically, however, rusting is very similiar to combustion. Both are examples of a materials undergoing destructive oxidation through reaction with oxygen gas. Iron rusts by reacting with oxygen and water to form a hydrated iron oxide called rust.

This activity is useful for showing students the conditions required for rusting - metal, oxygen (air), and water. After students learn about the requirements of rusting, they can participate in a discussion about how to prevent rusting. This activity should be performed by students working in small groups. The experiment should be started in one period and discussed in the next as at least 24 hours are required for results.

[INSERT ACTIVITY HERE]

%==============================================================================
\chapter{Acids and Bases}

Acids and bases are both categories of reactive chemicals. Acids react by giving the hydrogen ion to other compounds. Bases react by taking the hydrogen ion from other compounds. Reactions between acids and bases therefore involve the transfer of a hydrogen ion from the acid to the base.

This chapter explains how to prepare indicators to test whether a given compound is an acid, a base, or neither. Then this chapter gives an activity where students use indicators to identify common acids and bases. Finally, this chapter considers the reaction of acids and bases with a variety of other compounds.

\subsection{Preparation of Indicators}

Acid-base indicators are chemicals that are different colours in acids and bases. Methyl orange, for example, is red in acid and yellow in base. Indicators are very useful, because they tell us if something is an acid or a base. They are also used during reactions to show when a solution changes from acidic to basic or basic to acidic, as in volumetric analysis.

There are two types of indicators: liquid indicators and paper indicators. Liquid indicators are added dropwise until a distinct colour is observed. Paper indicators are made from liquid indicators and are used to either quickly test solutions or to test substances like gases that are not easily tested with liquid indicators.

In this activity students identify local flowers that may be used to produce acid-base indicators. The teacher should guide them to identify the best flowers, and together they should produce both liquid and paper indicators. These indicators may then be used for future experiments.

\subsubsection*{Objectives}
\begin{itemize}
\item{To identify local flowers that function as acid-base indicators.}
\item{To prepare a liquid acid-base indicator from local flowers.}
\item{To prepare both blue and red indicator paper from local flowers.}
\end{itemize}

\subsubsection*{Materials}
Pink, red, orange, and purple flowers, water, beakers (many), colourless methylated spirits, empty water bottle with cap, cooking pan (sufuria), heat source, white A4 paper, pair of scissors, 5M sulphuric acid, sodium hydroxide

\subsubsection*{Hazards and Safety}
\begin{itemize}
\item{((battery acid))}
\item{((sodium hydroxide))}
\end{itemize}

\subsubsection*{Preparation}
\begin{enumerate}
\item{Fill a 1.5 L water bottle half way with water. Add about 100 mL of battery acid. Label the bottle "dilute sulphuric acid."}
\item{Fill a second 1.5 L water bottle half way with water. Add 4 spoons of sodium hydroxide. Label this solution "sodium hydroxide solution."}
\end{enumerate}

\subsubsection*{Activity Procedure}
\begin{enumerate}
\item{Collect many pink, red, orange, and purple flowers flowers.}
\item{Crush each different flower in a small amount of water to obtain an extract of its pigment. The water should become the colour of the flower.}
\item{Divide the extract from each flower into three portions.}
\item{To one portion of extract from each flower, add about 1 mL of sulphuric acid solution. Note any colour change.}
\item{To the second portion of extract from each flower, add about 1 mL of sodium hydroxide solution. Note any colour change.}
\item{Place all three portions of the extracts next to each other to observe the three colours: the extract in acidic solution, the extract in neutral solution, and the extract in basic solution.}
\item{Select the best three flowers of all of the samples present. The best flowers will be available in a large quantity near the school and produce a very large colour difference in acid and in base.}
\item{Collect a large quantity of the three best flowers.}
\item{Cush these flowers in water to make approximately 1 litre total of extract of each flower. Divide this extract into three portions.}
\item{To the first portion, add an equal volume of colourless methylated spirits. Label the bottle "Indicator Solution."}
\item{Pour the second portion into a pot and start heating on a stove.}
\item{Leave the third portion at room temperature.}
\item{Cut A4 paper into pieces approximately 1 cm by 4 cm. Prepare about 100 pieces total.}
\item{Put half of the papers into the room temperature extract.}
\item{Put the other half of the papers into the pot with the flower extract and heat until boiling. Leave the papers in the boiling extract for at least ten minutes. Then take them out to dry.}
\item{After the papers dry, test one not-boiled and one boiled strip from each flower in both acid and in base. Select the best papers for using in place of litmus paper. The best substitute for red litmus will change colour in base but not in acid. The best substitute for blue litmus will change colour in acid but not in base.}
\end{enumerate}

\subsubsection*{Clean Up Procedure}
\begin{enumerate}
\item{Remove the unwanted materials and dip them in a pit latrine.}
\item{Wash hands with clean water and soap.}
\end{enumerate}

\subsubsection*{Discussion Questions}
\begin{enumerate}
\item{How do you think the first indicators were discovered?}
\end{enumerate}

\subsubsection*{Notes}
One very effective flower for making indicator solution and indicator paper is rosella. The extract is red in acid and blue or green in base. The papers act as red litmus when made in cool extract and as blue litmus when made in boiling extract. In general redish flowers are particularly useful for making indicators because most contain a pigment called anthrocyanins that are redish in acid and bluish in base. These pigments are much more complicated than methyl orange and each is different, so it is not possible to give students a chemical formula for these compounds.

-----------
\subsection{Acids and Bases in Daily Life}

\subsubsection*{Learning Objectives}
\begin{itemize}
\item{To site natural sources of acids and bases.}
\end{itemize}

\subsubsection*{Materials}
Citric acid*, ash from burning charcoal or a banana plant*, pure water, citrus fruits, beakers*, droppers*, and Litmus Paper*

\subsubsection*{Preparation}
\begin{enumerate}
\item{Squeeze lemon juice into a beaker.}
\item{Mix ashes with water in a beaker.}
\end{enumerate}

\subsubsection*{Activity Procedure}
\begin{enumerate}
\item{Arrange students into groups of 4-6.  Give each group 5 beakers, litmus paper, and a dropper.}
\item{Instruct students to add a few drops of lemon juice to the first beaker. Then instruct students to dip red and blue litmus paper into the beaker.}
\item{Instruct students to add a few drops of vinegar into the second beaker. Then instruct students to dip red and blue litmus paper into the beaker.}
\item{Instruct students to add a few drops of the wood ash solution into the third beaker. Then instruct students to dip red and blue litmus paper into the beaker.}
\item{Instruct students to add a few drops of the ash solution into the fourth beaker. Then instruct students to dip red and blue litmus paper into the beaker.}
\item{Instruct students to add a few drops of the soda ash solution into the fifth beaker. Then instruct students to dip red and blue litmus paper into the beaker.}
\end{enumerate}

\subsubsection*{Results and Conclusion}
Lemon juice and vinegar both turn blue litmus paper red. They have no effect on red litmus paper. Ash solution turns red litmus paper blue and has no effect on blue litmus paper.

\subsubsection*{Clean Up Procedure}
\begin{enumerate}
\item{Collect all the used materials, cleaning and storing items that will be used later.  No special waste disposal is required.}
\end{enumerate}

\subsubsection*{Discussion Questions}
\begin{enumerate}
\item{Which of these substances are acidic? How do you know?}
\item{Which of these substances are basic? How do you know?}
\end{enumerate}

\subsubsection*{Notes}
Citrus fruits contain citric acid while vinegar is a dilute solution of ethanoic (acetic) acid. Ashes contain metal oxides, hydroxides, and carbonates - these produce an alkaline solution in water. Rosella contains pigments called anthrocyanins that are red in acid and blue in base.


%-------

\subsection{Reaction of Acids}

\subsubsection*{Learning Objectives}
\begin{itemize}
\item{To demonstrate the reactions of acids with various inorganic compounds}
\end{itemize}

\subsubsection*{Materials}
test tubes*, test tube rack*, spatulas*, beakers*, dilute weak acid*, copper wire, flame, calcium oxide*, calcium hydroxide*, sodium hydrogen carbonate*

\subsubsection*{Hazards and Safety}
\begin{itemize}
\item{((rxn in test tube))}
\end{itemize}

\subsubsection*{Preparation}
\begin{enumerate}
\item{Pass a long piece of copper wire slowly through a flame until all of the surface has oxidized. The copper should become dark in colour. This colour is caused by a layer of copper oxide formed by reaction with air at high temperature.}
\item{Cut the oxidixed copper wire into 5 cm segments, one for each group of students}
\end{enumerate}

\subsubsection*{Activity Procedure}
\begin{enumerate}
\item{Divide the students into groups of 4 - 6 students each}
\item{Instruct each group to take four test tubes, one test tube rack, three petri dishes, three spatulas, and one piece of oxidized wire.}
\item{Provide each group with a very small amount each of sodium hydrogen carbonate, calcium hydroxide, and calcium oxide. Put each sample in a separate petri dish.}
\item{Instruct students to add three drops of indicator solution to each test tube. The students should observe that all of the tubes contain basic materials except the one with copper oxide.}
\item{Provide each group with about 20 mL of dilute weak acid.}
\item{Instruct the students to pour about one quarter of their weak acid into the first test tube (calcium oxide). They should observe the reaction and any colour change in the indicator. They should also observe any change in temperature by holding the outside of the test tube.}
\item{Instruct the students to pour another quarter of their weak acid into the second test tube (calcium hydroxide). They should observe the reaction and any colour change in the indicator. They should also observe any change in temperature by holding the outside of the test tube.}
\item{Instruct the students to pour another quarter of their weak acid into the third test tube (copper oxide). They should observe the reaction and any colour change in the indicator. After one minute, they should observe any change in the colour of the metal.}
\item{Instruct the students to pour the last quarter of their weak acid into the fourth test tube (sodium hydrogen carbonate). They should observe the reaction and any colour change in the indicator. They should also observe any change in temperature by holding the outside of the test tube.}
\end{enumerate}

\subsubsection*{Results and Conclusion}
In the fourth test tube, the students should observe rapid effervescence - bubbles should form and the solution may bubble out of the top of the test tube. This effervesence is caused by the release of carbon dioxide from the carbonate by the action of the acid. The indicator will also change colour from basic to acidic.

\subsubsection*{Clean Up Procedure}
\begin{enumerate}
\item{((no waste treatment))}
\item{((clean up))}
\end{enumerate}

\subsubsection*{Notes}

\subsection{Reaction of Bases}

[INSERT ACTIVITY HERE]

%==============================================================================

\chapter{Moles}

\subsection{A Mole of Water}

[INSERT ACTIVITY HERE]

\subsection{A Mole of Gas}

\subsubsection*{Learning Objectives}
\begin{itemize}
\item{To construct a box holding a mole of gas}
\end{itemize}

\subsubsection*{Materials}
Card bord, knife, ruler, masking tape and super glue.

\subsubsection*{Hazards and Safety}
\begin{itemize}
\item{Be careful with the knife and glue to avoid accidents.}
\end{itemize}

\subsubsection*{Preparation}
\begin{enumerate}
\item{Collect all the needed materials and put them on the bench.}
\end{enumerate}

\subsubsection*{Activity Procedure}
\begin{enumerate}
\item{Draw three side of a box each having 28.2cm long.}
\item{Cut the pieces to obtain equal size.}
\item{Fold the edges to obtain the cube.}
\item{Bind the edges of the box with glue or masking tape.}
\end{enumerate}

\subsubsection*{Results and Conclusion}
The volume of the box is the product of the length, width and height. (ie: 28.2cm * 28.2cm * 28.2cm = 22425.768cm3 = 22.4dm3).

\subsubsection*{Clean Up Procedure}
\begin{enumerate}
\item{Remove all the unwanted materials and dispose them safely.}
\end{enumerate}

\subsubsection*{Discussion Questions}
\begin{enumerate}
\item{Calculate the volume of the box in cubic centimeter.}
\item{Converte the volume obtained in cubic decimeter.}
\end{enumerate}

\subsubsection*{Notes}
The volume of any gas at s.t.p contain the volume equal to the volume of this box, ie: 22.4dm3. This is called gramme molecular volume(G.M.V.) when the molecular weight of a given gas is expressed in grams. Eg: 2g of H2(g), 32g of O2(g), 17g of NH3(g) and 44g of CO2(g).


%==============================================================================

\chapter{Volumetric Analysis}

[SECTION STILL UNDER CONSTRUCTION]

%tools for volumetric analysis, quantitative precision, volumetric measurements, rel stand.
%[3.5.2] Titration
%[3.5.3] Calculations wit Volumetric Analysis
%(rel atomic mass)
%(percent purity)
%(water of xtal)

%==============================================================================

\chapter{Gases}
\subsection{Oxygen}

\subsubsection*{Learning Objectives}
\begin{itemize}
\item{To prepare a sample of oxygen gas.}
\item{To demonstrate the properties of oxygen gas.}
\end{itemize}

\subsubsection*{Background Information}


\subsubsection*{Materials}
6 percent hydrogen peroxide (pharmacies), manganese (IV) oxide*, water, sulphur powder*, gas generator*, spoon*,  match box, thin dry stick, deflagrating spoon*

\subsubsection*{Hazards and Safety}
\begin{itemize}
\item{When sulphur burns, the gas produced (SO2) is poisonous. Avoid inhalation of the gas.}
\item{Manganese (IV) oxide is poisonous. Avoid contact and wash immediately. Also, it is corrosive to metal-wash all tools thoroughly.}
\end{itemize}

\subsubsection*{Activity Procedure}
\begin{enumerate}
\item{Put about one tea spoon full of manganese (IV) oxide into the reaction bottle of the gas generator.}
\item{Squeeze the second bottle ("collection bottle") to remove some of the air.}
\item{Add in the first bottle about 50 mL of dilute hydrogen peroxide. Close the bottle tightly and pass the gas into the second bottle.}
\item{Collect two bottles of oxygen this way ready for testing.}
\item{TEST 1: In one of the bottles full of oxygen, insert a glowing splint.}
\item{Observe what happens.}
\item{TEST 2: Put some sulphur in a deflagrating spoon and set it on fire.}
\item{Insert the burning sulphur into the second bottle of oxygen.}
\item{Observe the colour of the flame. Record your observations.}
\end{enumerate}

\subsubsection*{Results and Conclusion}
The gas produced relights a glowing glowing splint. This is the test for oxygen. Sulphur is a non-metal that burns in oxygen to form sulphur dioxide gas. Manganese (IV) oxide acts as a catalyst (i.e. it increases the rate of the decomposition of hydrogen peroxide). When hydrogen peroxide decomposes, it gives out oxygen gas and water:
2H2O2 --> 2H2O + O2
This is the safest way of producing oxygen because it requires no heating.

\subsubsection*{Clean Up Procedure}
\begin{enumerate}
\item{Collect all the used materials, cleaning and storing items that will be used later. No special waste disposal is required.}
\item{Reserve the unused materials from the dry cell for other experiments.}
\end{enumerate}

\subsubsection*{Discussion Questions}
\begin{enumerate}
\item{What is the use of manganese (IV) oxide in this reaction?}
\item{List the physical properties of the gas produced in this reaction.}
\item{How do you distinguish oxygen gas from other gases?}
\item{What is the name of the gas produced when sulphur burns in oxygen?}
\end{enumerate}

\subsubsection*{Notes}


\subsection{Carbon Dioxide}

\subsubsection*{Learning Objectives}
\begin{itemize}
\item{To prepare a sample of carbon dioxide gas.}
\item{To investigate the properties of carbon dioxide gas.}
\end{itemize}

\subsubsection*{Background Information}


\subsubsection*{Materials}
Three limes, empty water bottles, water, sodium hydrogen carbonate, gas generator*, candle, match box, lime water*, knife or razor blade

\subsubsection*{Preparation}
\begin{enumerate}
\item{Prepare a solution of lime water.}
\end{enumerate}

\subsubsection*{Activity Procedure}
\begin{enumerate}
\item{Cut the limes and squeeze them to collect the juice.}
\item{Add approximately 10 ml of water to the lime solution.}
\item{In the gas generator bottle, add half a tea spoon of sodium carbonate.}
\item{Add the diluted lime to juice to the sodium hydrogen carbonate. Close the bottle with the cap joined to the bottle of lime water.}
\item{Allow the gas to pass through the delivery tube and into a container holding about 5 mL of lime water. Continue passing until a change is noted.}
\item{Observe the lime water and record observations. Continue to pass carbon dioxide until the white precipitate disappears.}
\item{Collect another bottle of carbon dioxide gas from the gas generator.}
\item{Light a candle.}
\item{Pour the produced carbon dioxide over the burning candle.}
\item{Record observations.}
\item{Test the gas produced in this experiment with wet blue litmus paper. You must use a strong stream of gas, right as it is produced.}
\item{Record observations.}
\end{enumerate}

\subsubsection*{Results and Conclusion}
Carbon dioxide gas is prepared by the action of dilute acid on any carbonate.The gas is colourless and odourless. It turns lime water milky due to formation of calcium carbonate.    
CO2(g)   + ca(OH)2(aq) ---> CaCO3(s)   +   H2O(l)
Passing excess carbon dioxide the milky colour disappears due to formation calcium hydrogen carbonate which is soluble.
CaCO3(s)   +   H2O(l)   +   CO2(g)-->Ca(HCO3)2(aq)
Carbon dioxide gas turns blue litmus paper pink, showing that it is slightly acidic. It is used in extinguishing fire because it is denser than air and does not support combustion.
The gas is denser than air so you can pour from one bottle into another just like water. It does not support combustion though materials which burn to giving a lot of heat can split the gas and give out oxygen which can support combustion. For example burning Magnesium.
Mg(s)  +  CO2(g)  --> MgO(s)    +   C(s)

\subsubsection*{Clean Up Procedure}
\begin{enumerate}
\item{Wash all bottles with clean water and soap.}
\item{Collect all the used materials, cleaning and storing items that will be used later. No special waste disposal is required.}
\end{enumerate}

\subsubsection*{Discussion Questions}
\begin{enumerate}
\item{What colour is carbon dioxide gas?}
\item{Why did lime water turn milky?}
\item{When excess carbon dioxide was passed through the lime water, why did the white colour disappear?}
\item{Does carbon dioxide gas support burning?}
\item{Discuss about the density of carbon dioxide and the air.}
\item{Why is carbon dioxide gas used in fire extinguishers?}
\end{enumerate}

\subsubsection*{Notes}
Limes produce citric acid. White cement produces calcium hydroxide.
\subsection{Nitrogen}

\subsubsection*{Learning Objectives}
\begin{itemize}
\item{To prepare nitrogen gas using separation methods (i.e. separation of air components)}
\end{itemize}

\subsubsection*{Background Information}
Air is  78\% nitrogen. One method of preparation of nitrogen gas is to remove all of the other components of air (oxygen, carbon dioxide, water, etc).

\subsubsection*{Materials}
Large water bottle (6L), sodium hydroxide*, 5 M sulphuric acid*, delivery tubes*, source of heat, piece of glass tube (about 20 cm), 4 empty water bottles (1 or 1.5 l), copper foil (or turnings)

\subsubsection*{Hazards and Safety}
\begin{itemize}
\item{((Sodium Hydroxide))}
\item{((Battery Acid))}
\end{itemize}

\subsubsection*{Preparation}
\begin{enumerate}
\item{In the lids from two water bottles poke two holes. In the remaining water bottle lid poke a single hole.}
\item{Connect the delivery tubes in these holes using brio (pen) tubes as junctions.}
\item{Insert the copper foils or copper turnings inside the glass tube.}
\item{Prepare a 2M solution of caustic soda in one 1 L bottles with two holes in the cap.}
\item{Put sulphuric acid in the other bottle with two holes in the cap.}
\item{Prepare the heat source by lighting the stove.}
\item{Arrange the apparatus set up as in the diagram below (ILLUSTRATION) making sure that bottle C is squeezed (compressed) to remove air before compression.}
\end{enumerate}

\subsubsection*{Activity Procedure}
\begin{enumerate}
\item{Add water through the funnel into the 6 L bottle so as to displace air present in the bottle. This is done after the copper turning/foil starts to be red hot.}
\item{Observe what happens in the two water bottles A and B as well as the changes ofthe red hot copper turning in the combustion tube.}
\item{Observe the expansion of the bottle C as water fills the 6 L bottle.}
\item{Collect the gas in the bottle C by tightening the delivery tube.}
\end{enumerate}

\subsubsection*{Results and Conclusion}
Copper turning turn red hot when heated in the absence of air (i.e. before water was added to the 6L bottle. After the addition of water to 6L bottle, bubbles were observed in both A and B and the red hot copper turnings turn black. The collection bottle (C) expanded. The black colour of copper indicates oxidation of copper by atmospheric oxygen. Copper oxide is black. The bubbles observed in the bottles indicates the passage of air into the solution.

\subsubsection*{Clean Up Procedure}
\begin{enumerate}
\item{Collect all the used materials, cleaning and storing items that will be used later. No special waste disposal is required.}
\end{enumerate}

\subsubsection*{Discussion Questions}
\begin{enumerate}
\item{What is observed in bottles A and B?}
\item{What do you think is the purpose of 2M NaOH and conc. H2SO4?}
\item{What is the function of hot copper i the combustion tube?}
\item{Nitrogen collected by this method is said to be impure. What do you think is the chief impurity?}
\item{What is the colour of the copper turnings after the collection of gas in bottle C? With the help of chemical reaction equation explain the changes which occurred in the combustion tube.}
\end{enumerate}

\subsubsection*{Notes}
When air passes through the solution of sodium hydroxide, CO2 is absorbed to form carbonates:
2NaOH(aq)+CO2(g)-->Na2CO3(aq)+H2O(l)
Concentrated H2SO4 is a good drying agent for gases. It absorbs the atmospheric water vapour.
Heated copper absorbs the atmospheric oxygen to become copper oxide.
2Cu(s)_O2(g)-->2CuO(s)
The nitrogen obtained is not 100% pure. It contains about 1% by volume of the noble gases.


\subsection{Hydrogen}

\subsubsection*{Learning Objectives}
\begin{itemize}
\item{To prepare a sample of hydrogen gas.}
\item{To demonstrate the properties of hydrogen gas.}
\end{itemize}

\subsubsection*{Background Information}


\subsubsection*{Materials}
Zinc metal, empty water bottles, balloon, match box, 5 M sulphuric acid, and gas generator*.

\subsubsection*{Hazards and Safety}
\begin{itemize}
\item{Collect a small amount of hydrogen gas. When the gas is lit, point both ends of the bottle away from people.}
\item{ ((battery acid))}
\item{The bottle may fly in the air like a rocket - take care!}
\end{itemize}

\subsubsection*{Preparation}
\begin{enumerate}
\item{Take a piece of zinc and cut it into small pieces.}
\end{enumerate}

\subsubsection*{Activity Procedure}
\begin{enumerate}
\item{Put the zinc pieces into the bottle of a gas generator.}
\item{Add about 100 mL of battery acid.}
\item{Collect the gas by downward displacement of air in the collection bottle.}
\item{Remove the collection bottle, keeping the open end pointed downward.}
\item{Bring a flame to the mouth of the bottle and observe what happens.}
\end{enumerate}

\subsubsection*{Results and Conclusion}
When metals react with acids, hydrogen gas is liberated. The gas is colourless, ordourless, and inert to litmus. It is the lightest gas known and burns in air to produce water vapour with a "pop" sound. The gas is identified by this explosion.

\subsubsection*{Clean Up Procedure}
\begin{enumerate}
\item{Wash the bottles.}
\item{Dispose of all unwanted materials. Neutralize the left over battery acid with sodium hydrogen carbonate before putting in a sink.}
\end{enumerate}

\subsubsection*{Discussion Questions}
\begin{enumerate}
\item{What is the colour of the gas produced?}
\item{How do you identify hydrogen gas?}
\item{What is the product when hydrogen burns in the air?}
\end{enumerate}

\subsubsection*{Notes}
Battery acid is 5 M sulphuric acid. Other common metals that may be used for this experiment include magnesium (more reactive) and iron (less reactive). Copper metal will not displace hydrogen gas from normal acids because it is below hydrogen in the reactivity series.


\subsection{Chlorine}

\subsubsection*{Learning Objectives}
\begin{itemize}
\item{To prepare chlorine gas.}
\item{To investigate the properties of chlorine gas.}
\end{itemize}

\subsubsection*{Background Information}


\subsubsection*{Materials}
Bleach*, 5M sulphuric acid*, colored flowers, string or thread, large empty plastic water bottle.

\subsubsection*{Hazards and Safety}
\begin{itemize}
\item{Chlorine is very poisonous! Avoid inhalation. Only prepare chlorine gas is small quantities and in well ventilated areas or outside.}
\item{Bleach bleaches clothing - take care to not spill any outside of the bottle.}
\item{Concentrated sulphuric acid is corrosive to skin and clothes. Avoid contact with skin and eyes. Neutralize spills with sodium hydrogen carbonate (baking powder).}
\end{itemize}

\subsubsection*{Preparation}
\begin{enumerate}
\item{Collect coloured flowers.}
\item{Cut about 30 cm of string for each flower.}
\item{Tie one end of each string to each flower.}
\end{enumerate}

\subsubsection*{Activity Procedure}
\begin{enumerate}
\item{Put about 100 ml of Jik in the reaction bottle}
\item{Hang the flowers into the bottle using the strings. Tie the free end of the string around the neck of the bottle.}
\item{Add about 10 ml of battery acid to the reaction bottle. Quickly close the cap.}
\item{Put the bottle aside for 5 minutes.}
\item{Record your observations.}
\item{Test the gas evolved using moist litmus paper.}
\end{enumerate}

\subsubsection*{Results and Conclusion}
Chlorine gas is greenish-yellow with an irritating choking smell. It turns blue litmus red and then bleaches it. The chlorine gas will bleach the flowers in the bottle. The bleaching action is due to the formation of hypochlorite ion, which is formed when chlorine dissolves in water.

Cl2(g) + H2O --> HCl(aq) + HClO(aq)

HClO + 'dye' --> 'dye-O' + HCl

\subsubsection*{Clean Up Procedure}
\begin{enumerate}
\item{Open bottle outside prior to clean up.}
\item{Collect all the used materials, cleaning and storing items that will be used later. No special waste disposal is required.}
\end{enumerate}

\subsubsection*{Discussion Questions}
\begin{enumerate}
\item{What colour is chlorine gas?}
\item{What happened to the flowers?}
\item{Why can we not collect chlorine gas over water?}
\item{Why is it important to always be aware of which chemicals you are mixing in a chemistry lab?}
\end{enumerate}

\subsubsection*{Notes}




\subsection{Ammonia}

\subsubsection*{Learning Objectives}
\begin{itemize}
\item{To prepare Ammonia gas.}
\end{itemize}

\subsubsection*{Background Information}


\subsubsection*{Materials}
Gas generator*, ammonium sulphate*, sodium hydroxide*, red litmus paper*, match box, conc. HCl(optional) , heat source*, heating vessel*

\subsubsection*{Hazards and Safety}
\begin{itemize}
\item{Avoid inhaling poisonous ammonia and hydrogen chloride gases.}
\item{((sodium hydroxide))}
\end{itemize}

\subsubsection*{Preparation}
\begin{enumerate}
\item{Prepare a caustic soda solution by dissolving approximately 1 spoonful in 200 mL of water.}
\end{enumerate}

\subsubsection*{Activity Procedure}
\begin{enumerate}
\item{Put one tea spoonful of ammonium sulphate into a heating vessel.}
\item{Add about 100 mL of sodium hydroxide solution to the ammonium sulphate and mix.}
\item{Warm the mixture using a heat source.}
\item{Test the gas evolved using moos red litmus paper.}
\item{Record the observations. Note the smell of the gas.}
\item{Optional: Bring a bottle containing conc. HCl acid, open it and allow fumes coming out of the bottle to react with the gas evolved from the gas generator.}
\item{Record the observations. Note the intensity of the fumes.}
\end{enumerate}

\subsubsection*{Results and Conclusion}
Ammonia is a colourless gas with pungent smell(smell of urine), it turns red litmus blue. It also forms white fumes when it comes into contact with HCl vapors.
Ammonia gas is the only alkaline gas known, it reacts with hydrogen chloride gas to form thick/dense white fumes of ammonium chloride(This is the identification of the gas). It is highly soluble in water that's why it can't be collected over water. It is less dense than air that's why it is  collected by downward displacement of air(upward delivery).

\subsubsection*{Clean Up Procedure}
\begin{enumerate}
\item{Collect all the used materials, cleaning and storing items that will be used later. No special waste disposal is required.}
\end{enumerate}

\subsubsection*{Discussion Questions}
\begin{enumerate}
\item{What is the colour of ammonia?}
\item{How do you identify the ammonia gas?}
\item{Write balanced chemical equation showing the reaction between ammonia and hydrogen chloride gas.}
\item{Why do you think ammonia is not collected over water?}
\end{enumerate}

\subsubsection*{Notes}

\subsection{Sulphur Dioxide}

[INSERT ACTIVITY HERE]

%==============================================================================

\chapter{Inorganic Chemistry/Preparation of Metal Compounds}
[
\section{Preparation of Oxides}

\subsection{Direct Method}

\subsection{Direct Preparation of a Metal Oxide.}

\subsubsection*{Learning Objectives}
\begin{itemize}
\item{SWBAT prepare an oxide of a metal by the direct method of using heat.}
\end{itemize}

\subsubsection*{Background Information}


\subsubsection*{Materials}
Zinc metal*, source of heat*, match box, spoon.

\subsubsection*{Hazards and Safety}
\begin{itemize}
\item{((FIRE))}
\end{itemize}

\subsubsection*{Activity Procedure}
\begin{enumerate}
\item{Take small piece of zinc case and put on the spoon.}
\item{Heat the zinc strongly on the spoon and observe the changes.}
\item{Leave it to cool and note another change.}
\end{enumerate}

\subsubsection*{Results and Conclusion}
Zinc metal turns yellow when heated strongly in air. The yellow colour turns white when allowed to cool. This product which is yellow when hot and white when cold is zinc oxide (ZnO).

\subsubsection*{Clean Up Procedure}
\begin{enumerate}
\item{Collect all the used materials, cleaning and storing items that will be used later. No special waste disposal is required.}
\end{enumerate}

\subsubsection*{Discussion Questions}
\begin{enumerate}
\item{What happened when the zinc was heated strongly?}
\item{Explain the changes observed when the heated sample was cooled. Name the sample.}
\item{What other metal oxides can be prepared by direct method?}
\end{enumerate}

\subsubsection*{Notes}


\subsection{Indirect Method}

\subsection{Indirect Preparation of a Metal Oxide}

\subsubsection*{Learning Objectives}
\begin{itemize}
\item{SWBAT prepare a metal oxide by the reaction of metal with acid followed by heating.}
\end{itemize}

\subsubsection*{Background Information}


\subsubsection*{Materials}
Zinc metal*, 5M sulphuric acid*, beaker*, heating vessel*, heat source*, spoon.

\subsubsection*{Hazards and Safety}
\begin{itemize}
\item{((battery acid))}
\end{itemize}

\subsubsection*{Activity Procedure}
\begin{enumerate}
\item{Put about 10 mL of sulphuric acid into a beaker.}
\item{Add a small piece of zinc metal and then allow the reaction to take place.}
\item{After zinc granule has completely dissolved, take the solution and pour it 3 mL of it into a heating vessel.}
\item{Heat the solution to dryness to obtain the residue formed.}
\item{Put small amount of this solid into a spoon and heat it until the compound decomposes and a new residue forms.}
\end{enumerate}

\subsubsection*{Results and Conclusion}
When zinc reacts with dilute sulphuric acid, a soluble zinc sulphate salt forms. This is formed by displacement reaction. The salt can be obtained only by evaporating water to dryness. ZnSO4 is white colour. When heating the compound (ZnSO4) the gas SO2 is evolved and the residue is ZnO. The ZnO is yellow when hot and white when cold.

\subsubsection*{Clean Up Procedure}
\begin{enumerate}
\item{Collect all the used materials, cleaning and storing items that will be used later. No special waste disposal is required.}
\end{enumerate}

\subsubsection*{Discussion Questions}
\begin{enumerate}
\item{Write the equation and name the compound formed when zinc is put into acid and heated to dryness.}
\item{What is the purpose of heating the solution to dryness?}
\item{When heating the compound formed after dryness explain the changes in the residue.}
\end{enumerate}

\subsubsection*{Notes}


\section{Preparation of Hydroxides}

\subsection{Preparation of Metal Hydroxide}

\subsubsection*{Learning Objectives}
\begin{itemize}
\item{To prepare metal hydroxide by an indirect method.}
\end{itemize}

\subsubsection*{Background Information}


\subsubsection*{Materials}
Steel wool, strong acid*, sodium hydroxide, filter funnel*, beakers*.

\subsubsection*{Hazards and Safety}
\begin{itemize}
\item{((battery acid))}
\end{itemize}

\subsubsection*{Preparation}
\begin{enumerate}
\item{Prepare a sodium hydroxide solution by adding 1 spoon of sodium hydroxide to 100 mL of water. Provide each group with 10 mL of solution in a beaker.}
\end{enumerate}

\subsubsection*{Activity Procedure}
\begin{enumerate}
\item{Instruct students to take a small amount of steel wool and put in one of the beakers.}
\item{Instruct students to add about 10 mL of strong acid.}
\item{Guide students to observe the reaction. Bubbles of hydrogen gas should form. Once students have observed the reaction, they should place the reacting beaker in a well ventilated space and not breathe the gas produced. While most is hydrogen, something else unpleasant also forms.}
\item{The reaction is finished when there are no more bubbles. If the steel wool is completely consumed, advised students to add more steel wool to allow the reaction to continue. The goal is for all of the acid to be consumed. Observe the colour of the final solution.}
\item{When the reaction is finished, instruct students to decant the contents of the reaction beaker into their beaker of sodium hydroxide. A precipitate should form immediately. Observe the colour of the precipitate.}
\item{Pour the mixture with the precipitate into the filter funnel. Leave to filter. Observe any change in colour.}
\item{Once most of the liquid has passed through the filter, remove the solid from the filter funnel and leave to dry.}
\end{enumerate}

\subsubsection*{Results and Conclusion}
The steel wool react with strong acid to form iron (II) soution. If sulphuric acid is used as the strong acid, the product of the first reaction will be iron (II) sulphate. This solution reacts with sodium hydroxide solution to produce a green, gelatinous precipitate of iron (II) hydroxide. On exposure to air, this precipitate oxidizes to iron (III) oxide.

\subsubsection*{Clean Up Procedure}
\begin{enumerate}
\item{Pour the wastes to the toilet or soak pit tank and clean the table.}
\end{enumerate}

\subsubsection*{Discussion Questions}
\begin{enumerate}
\item{What are the products of the first reaction?}
\item{What are the products of the second reaction?}
\item{What caused the change in colour as the precipitate is exposed to the air?}
\item{What is the chemical formula of the final product?}
\end{enumerate}

\subsubsection*{Notes}

%----


\section{Preparation of Carbonates}

\subsection{Preparation of metal carbonate by precipitation.}

\subsubsection*{Learning Objectives}
\begin{itemize}
\item{To prepare metal carbonate by precipitation reaction.}
\end{itemize}

\subsubsection*{Background Information}


\subsubsection*{Materials}
Magnesium sulphate* and/or copper sulphate*, sodium carbonate*, funnel*, cottom wool, beakers*, spatulas

\subsubsection*{Preparation}
\begin{enumerate}
\item{Stuff cotton wool into the funnel to plug the hole at the bottom.}
\end{enumerate}

\subsubsection*{Activity Procedure}
\begin{enumerate}
\item{In one beaker, add 2 spoons of magnesium sulphate or 1 spoon of copper sulphate to about 100 mL of water. Stir with a spatula until completely dissolved.}
\item{In a second beaker, add 2 spoons of sodium carbonate to about 100 mL of water. Stir with a spatula until completely dissolved.}
\item{Add the sodium carbonate solution to the magnesium sulphate / copper sulphate solution. A precipitate should form immediately.}
\item{Place the funnel on top of another beaker and pour the solution with the precipitate into the funnel. A clean liquid (filtrate) should collect in the beaker below the funnel. A solid precipitate should collect in the funnel.}
\item{Leave the funnel to sit until all of the liquid has moved through. Then remove the solid accumulated in the funnel and leave to dry.}
\end{enumerate}

\subsubsection*{Results and Conclusion}
When magnesium sulphate solution is mixed with sodium carbonate solution, magnesium carbonate precipitates. When copper sulphate solution is mixed with sodium carbonate solution, copper carbonate precipitates. This demonstrates the preparation of metal carbonates by precipitation reactions.

\subsubsection*{Clean Up Procedure}
\begin{enumerate}
\item{Save the dried precipitate (magnesium carbonate or copper carbonate) for a future experiment.}
\end{enumerate}

\subsubsection*{Discussion Questions}
\begin{enumerate}
\item{What did you observe then the solutions were mixed?}
\item{What is chemical precipitated?}
\item{Write the ionic equation for the formtion of the product.}
\end{enumerate}

\subsubsection*{Notes}
This experiment deals with the solubility of different compounds. Magnesium sulphate and copper sulphate are both soluble in water, as is sodium carbonate. Magnesium carbonate and copper carbonate are both insoluble. As soon as magnesium and carbonate ions meet in solution, they form a white precipitate. As soon as copper and carbonate ions meet in solution, they form a blue precipitate. Many other metal carbonates may be prepared with this method.

\subsection{Preparation of Metal Carbonate by Reaction of CO2 with Alkali}

\subsubsection*{Learning Objectives}
\begin{itemize}
\item{To understand how carbon dioxide reacts with an alkali solution.}
\end{itemize}

\subsubsection*{Background Information}
Carbon dioxide reacts with alkalis to form metal carbonates. Some carbonates are soluble while others are insoluble. Carbonates of alkali metals (e.g. Na and K) are solution while those of alkaline earth metals are insoluble (e.g. Ca and Mg). When CO2 is in excess, the soluble hydrogen carbonate forms.
CO2+Ca(OH)2CaCO3+H2O
andCaCO3+H2O+CO2Ca(HCO3)2

\subsubsection*{Materials}
clean drinking straw, lime water*, beaker*, test tube*

\subsubsection*{Activity Procedure}
\begin{enumerate}
\item{Give each student a test tube or beaker containing approximately 3 mL of lime water and a clean drinking straw.}
\item{Instruct students to blow exhaled air from the mouth into the solution in the test tube and note the changes.}
\end{enumerate}

\subsubsection*{Results and Conclusion}
Carbon dioxide blown from the straw will react with lime water Ca(OH)2 to form a white precipitate of CaCO3. This shows that carbon dioxide reacts with an alkali solution.

\subsubsection*{Clean Up Procedure}
\begin{enumerate}
\item{Collect all the used materials, cleaning and storing items that will be used later. No special waste disposal is required.}
\end{enumerate}

\subsubsection*{Notes}
Hydrogen carbonates of alkaline earth metals only exist in solution - they cannot be prepared as solids.


%-----


\section{Preparation of Sulphates}

\subsection{ Preparation of Salts by Reaction of Metal with Acid}

\subsubsection*{Learning Objectives}
\begin{itemize}
\item{SWBAT prepare salts in the laboratory by the replacement reaction of hydrogen from an acid by a metal.}
\end{itemize}

\subsubsection*{Background Information}


\subsubsection*{Materials}
zinc metal*, dil. H2SO4*, beakers*, evaporating dish*, kerosene stove and steel wool.

\subsubsection*{Hazards and Safety}
\begin{itemize}
\item{((battery acid))}
\end{itemize}

\subsubsection*{Preparation Procedure}
\begin{enumerate}
\item{Clean up the zinc metal by a steel wool and cut it into small pieces to increase surface area for the reaction.}
\end{enumerate}

\subsubsection*{Activity Procedure}
\begin{enumerate}
\item{Take zinc granules into one of the beaker followed by small amount of sulphuricsulphuric acid and leave for the reaction to take place.}
\item{After all the zinc granules have reacted take the solution into the evaporating plate.}
\item{Heat the solution to evaporate to dryness and collect the remains.}
\end{enumerate}

\subsubsection*{Results and Conclusion}
Zinc reacts with the sulphuric acid and replace hydrogen gas and form soluble zinc sulphate. The product formed from the evaporation of the zinc sulphate solution is the white solid zinc sulphate.

\subsubsection*{Clean Up Procedure}
\begin{enumerate}
\item{Collect all the used materials, cleaning and storing items that will be used later. No special waste disposal is required.}
\end{enumerate}

\subsubsection*{Discussion Questions}
\begin{enumerate}
\item{What happened when the sulphuric acid was added to the zinc granules?}
\item{What are the products of the solution resulting when the zinc granules reacted to completion? Write the word and chemical reaction equation.}
\item{Give the colour, name and chemical formula of the product formed after evaporation.}
\end{enumerate}

\subsubsection*{Notes}
When sulphuric acid react with zinc metal produces hydrogen gas and zinc sulphate, this is because zinc is more reactive than hydrogen.
Zn(s) + H2SO4(aq) ==> ZnSO4(aq) + H2(g)
Zinc sulphate is soluble salt, to get it in a solid form you evaporate to dryness. It will appears as a white powder.

\section{Preparation of Sulphides}

\subsection{Investigating the Reaction of Sulphur with Metal}

\subsubsection*{Learning Objectives}
\begin{itemize}
\item{To explain the oxidizing properties of sulphur.}
\end{itemize}

\subsubsection*{Background Information}


\subsubsection*{Materials}
Sulphur powder*, copper wire*, two spoons, source of heat*, heavy scissors, and a match box

\subsubsection*{Hazards and Safety}
\begin{itemize}
\item{Sulphur vapours are harmful, therefore avoid inhaling while heating.}
\item{Perform this experiment in a well-ventilated room or outside.}
\end{itemize}

\subsubsection*{Preparation}
\begin{enumerate}
\item{Cut the copper wire into small pieces using the scissors.}
\end{enumerate}

\subsubsection*{Activity Procedure}
\begin{enumerate}
\item{Instruct students to put a few copper cutting into a spoon.}
\item{Instruct students to add a small amount of sulphur to the copper cutting and mix. The volume of sulphur should be large compared to the volume of copper because copper is much more dense.}
\item{Heat the mixture in the spoon over a flame until the mixture turns black.}
\end{enumerate}

\subsubsection*{Results and Conclusion}
Copper and sulphur reacted to form black copper sulphide.

\subsubsection*{Clean Up Procedure}
\begin{enumerate}
\item{Collect all the used materials, cleaning and storing items that will be used later. It may be necessary to clean the spoon with a piece of steel wool.}
\item{Dispose of chemical waste in the pit latrine.}
\end{enumerate}

\subsubsection*{Discussion Questions}
\begin{enumerate}
\item{Why does the mixture of copper and sulphur turn black? Write the chemical equation for this reaction.}
\item{Name the black compound formed.}
\end{enumerate}

\subsubsection*{Notes}



%==============================================================================

\chapter{Qualitative Analysis}

[SECTION STILL UNDER CONSTRUCTION]
%concept
%chemicals // procedure
%procedure
%confirmatory tests

%==============================================================================

\chapter{Kinetics, Equilibrium, and Energetics}

\section{Kinetics}

Kinetics is the study of the rate of chemical reactions. Some reactions have a high rate, that is, they happen very quickly, for example the burning of paper. Other reactions have a low rate, that is, they happen very slowly, for example the rusting of metal. In order for a chemical reaction to happen, the molecules concerned must collide with each other. Anything that increases the frequency and intensity of these collisions will increase the rate of reaction.

Students should learn six factors that affect the rate of reaction: concentration, pressure, temperature, surface area, catalyst and light. This chapter discusses each of these factors and activities for presenting several of them.

\subsection{The Effect of Concentration of Reaction Rate}

Concentration has a positive effect on reaction rate. A higher concentration means that the molecules are more crowded and collide more often, thus increasing the rate of chemical reaction. In a solution of low concentration, molecules are less likely to collide with each other and hence have a slower rate of reaction. 

The effect of concentration is only observed in reactions that occur in solution. In gases, the effective concentration is directly related to the pressure of the gas - a higher pressure means a higher concentration means a faster rate of reaction. Because experiments with reacting gases are more difficult to set up and generally more dangerous to perform, students generally do experiments on the effect of concentration and use logic to extend their conclusions to the effect of pressure.

The following activity is useful for students to observe the effect of concentration on reaction rate. This activity should be performed by students working in small groups so that the effects may be easily seen.

\subsubsection*{Learning Objectives}
\begin{itemize}
\item{To demonstrate the effect of concentration on the rate of the reaction}
\end{itemize}

\subsubsection*{Materials}
dilute weak acid, sodium hydrogen carbonate, water, test tubes (6pcs), test tube rack

\subsubsection*{Activity Procedure}
\begin{enumerate}
\item{In 3 test tubes labeled A, B and C put approximately 10mL, 5mL and 2.5mL of vinegar respectively. Dilute each solution to approximately 20 mL.}
\item{In 3 new test tubes labeled  D, E, and F place 10g, 5g and 2.5g of sodium bicarbonate respectively. Dilute each solution to 20 mL with water.}
\item{Mix solution A with solution D and observe what happens.}
\item{Mix solution B with solution E and observe what happens.}
\item{Mix solution C with solutionF and observe what happend.Note any difference between the reactions in steps 3, 4, and 5.}
\end{enumerate}

\subsubsection*{Results and Conclusion}
Bubbles (carbon dioxide gas) will be formed most quickly in the reaction between solutions A and D. The bubbles will form more slowly in the reaction solutions B and E and the reaction will be slowest in the reaction between C and F. 
This experiment shows that higher the concentration of reactants, the faster the reaction will proceed.

\subsubsection*{Clean Up Procedure}
\begin{enumerate}
\item{Collect and clean all the used materials, storing items that will be used later. No special waste disposal required.}
\end{enumerate}

\subsubsection*{Discussion Questions}
\begin{enumerate}
\item{Why would this experiment be dangerous with concentrated acetic acid? (hint: vinegar is about 6\% acetic acid)}
\item{What will happen to the rate of a chemical reaction as the reaction proceeds?}
\end{enumerate}

\subsubsection*{Notes}

\subsection{Kinetics: Effect of Temperature}

An increase in temperature increases the rate of chemical reactions by increasing both the frequency of collisions between molecules and the intensity of these collisions. A higher temperature means that the molecules are moving with a higher velocity and thus collide both more often and more forcefully. Faster, harder collisions means a much faster chemical reaction.

The following activity is useful for students to observe this effect. They may reasonably perform this experiment on the same day as the experiment regarding the effect of concentration. Students should perform this experiment themselves in groups.

\subsubsection*{Learning Objectives}
\begin{itemize}
\item{To demonstrate the effect of temperature on the rate of reaction.}
\end{itemize}

\subsubsection*{Materials}
any dilute weak acid (acetic or citric acid)*, sodium hydrogen carbonate*, beakers*, test tubes*, water, source of heat*

\subsubsection*{Preparation}
\begin{enumerate}
\item{A few minutes prior to class light the heat source and put some water on to heat.}
\item{Prepare a solution of soidium hydrogen carbonate by dissolving approximately 3 teaspoons per litre of water.}
\end{enumerate}

\subsubsection*{Activity Procedure}
\begin{enumerate}
\item{Arrange the students into groups of 4-6. To each group give 4 test tubes, a beaker containing approximately 10 mL of acid and a second beaker containing approximately 10 mL of base.}
\item{Instruct students to arrange test tubes in the rack and label them with numbers 1, 2, 3 and 4. Instruct them to put approximately 3 mL of acid into test tubes 1 and 2 and 3 mL of base into test tubes 3 and 4.}
\item{Instruct students to heat test tubes 2 and 4 in the boiling water bath until they are nearly boiling.}
\item{Instruct students to pour the solution from test tube 3 into test tube 1.}
\item{Instruct students to then pour the solution from test tube 4 into test tube 2. Have students record their observations about the differences between the two reactions.}
\end{enumerate}

\subsubsection*{Results and Conclusion}
In the reaction between test tube 2 and 4 the reaction will be notably faster than in the reaction between test tube 1 and 3. The students should see that the bubbles are formed more quickly--the reaction is more vigorous. The hot solutions will react faster than the cold soluitons.

\subsubsection*{Clean Up Procedure}
\begin{enumerate}
\item{Collect all the used materials, cleaning and storing items that will be used later. No special waste disposal is required.}
\item{Unused acid and base solutions can be stored and labelled for later use.}
\end{enumerate}

\subsubsection*{Discussion Questions}
\begin{enumerate}
\item{In which reaction did the reaction happen faster? How do you know?}
\item{Explain what effect temperature has on the rate of a chemical reaction.}
\item{Explain why it is important to keep vegetables in a cool place during the day rather than in the sun.}
\item{The human body must be kept at a constant temperature. Explain what would happen in the temperature gets to high or low.}
\end{enumerate}

\subsubsection*{Notes}
This experiment can be expanded by setting up a gas collection apparatus and recording the volume of gas collected per unit of time for each reaction.

\subsection{Kinetics: Effect of Surface Area}

The surface area of a solid can effect the rate of chemical reactions by increasing the rate of collisions between molecules. A larger surface area means that there are more molecule available at any moment for reacting thus there are more frequent collisions. This is because in a solid, only molecules on the outer surface are able to react. If the solid is divided into many smaller pieces, the total surface area increases, the frequency of collisions increases, and therefore the rate of reaction increases.

The following activitity uses the reaction between iron metal and dilute sulphuric acid to allow students to observe the effect of surface area on reaction rate. These materials are available everywhere. Students may perform this activity in groups, with close supervision as the use of a strong acid is required.

In parts of the country with carbonate rocks - coral rock, limestone, and marble - an alternative activity is the reaction of calcium carbonate and a dilute weak acid. Students should react citric acid solution or ethanoic acid solution with both large pieces and powders of these rocks. A clear difference in the rate of effervesence will be observed.

\subsubsection*{Learning Objectives}
\begin{itemize}
\item{To show the effect of surface area of a reacting solid on the rate of reaction.}
\end{itemize}

\subsubsection*{Materials}
dilute sulphuric acid*, iron nail (or any other solid iron object like mbulumbulu), iron wool, test tubes*

\subsubsection*{Hazards and Safety}
\begin{itemize}
\item{((dilute strong acid))}
\end{itemize}

\subsubsection*{Activity Procedure}
\begin{enumerate}
\item{Fill two test tubes half way with dilute sulphuric acid.}
\item{At the same time, put a nail in one test tube and a piece of steel wool into the other.}
\item{
\end{enumerate}

\subsubsection*{Results and Conclusion}
When the iron is placed in the acid bubbles should clearly be seen on the surface. The bubbles form much more quickly from the steel wool than from the iron nail because it has a much higher surface area.

\subsubsection*{Clean Up Procedure}
\begin{enumerate}
\item{Collect the solutions of acid from each group. Use tweezers to remove the iron metal pieces, rinse them well in water and store for later use.}
\item{Store the sulphuric acid in a bottle labelled "Impure sulphuric acid" and save for later use. If no bottle is available, neutralise the acid with baking powder until effervescence stops and then dispose down the drain.}
\item{Collect all the used materials, cleaning and storing items that will be used later.}
\end{enumerate}

\subsubsection*{Discussion Questions}
\begin{enumerate}
\item{Which object has a higher surface area a nail or a piece of steel wool?}
\item{In which test tube did the reaction occur faster? How do you know.}
\item{Why do people usually grind salt before using it to cook?}
\end{enumerate}

\subsubsection*{Notes}


\subsection{Effect of catalyst on reaction rate}

A catalyst is any substance that increases the rate of a chemical reaction without being consumed in the reaction. 

Note that a substance is a catalyst only if it increases the rate of reaction. A substance that decreases the rate of reaction is called an inhibitor.

\section{Equilibrium}

\subsection{Reversible Chemical Reaction}

\subsubsection*{Learning Objectives}
\begin{itemize}
\item{To demonstrate a reversible reaction.}
\end{itemize}

\subsubsection*{Background Information}
A reversible reaction is a reaction that can occur in both forward and backward direction. Reversible reactions can be shown with a double arrow. Copper (II) sulphate exists in two forms: hydrated (with water) and anhydrous (without water). Hydrated copper (II) sulphate is blue while anhydrous copper (II) sulphate is white.

\subsubsection*{Materials}
Heat source*, copper (II) sulphate*, water, metal spoon

\subsubsection*{Hazards and Safety}
\begin{itemize}
\item{Motopoa flame may be invisible - be careful to avoid burns.}
\end{itemize}

\subsubsection*{Preparation}
\begin{enumerate}
\item{Grind the copper (II) sulphate crystals into a powder. If the powder looks white, leave it exposed to the air until it regains a blue colour.}
\end{enumerate}

\subsubsection*{Activity Procedure}
\begin{enumerate}
\item{Instruct students to place a very small amount of blue copper (II) sulphate in a metal spoon.}
\item{Supervise students heating the spoon gently over a heat source. Students should stop heating when the crystals have changed from blue to white.}
\item{Instruct students to add a few drops of water to the white crystals. Students should observe and record any colour change.}
\end{enumerate}

\subsubsection*{Results and Conclusion}
On heating blue hydrated copper (II) sulphate, the colour changes from blue (CuSO4-5H2O) to white (CuSO4). On addition of a few drops of water, CuSO4 returns to its original hydrated state (blue).

\subsubsection*{Clean Up Procedure}
\begin{enumerate}
\item{Copper (II) sulphate crystals can be left in the air to dry the excess water and then used again in future experiments.}
\item{Collect all the used material, cleaning and storing items that will be used again later. No special waste disposal is required.}
\end{enumerate}

\subsubsection*{Discussion Questions}
\begin{enumerate}
\item{Copper sulphate can be used as a test for water. Explain how this is possible.}
\item{Give an example of another reversible reaction that you have seen before.}
\end{enumerate}

\subsubsection*{Notes}
Continued heating will convert white anhydrous copper (II) sulphate to black copper oxide. This reaction is irreversible.

\subsection{Endothermic and Exothermic Reactions}

[INSERT ACTIVITY HERE]

%==============================================================================

\chapter{Electrochemistry}
\subsection{Reactivity Series: Displacement Reaction}

\subsubsection*{Learning Objectives}
\begin{itemize}
\item{To demonstrate the displacement a metal with a more reactive metal.}
\end{itemize}

\subsubsection*{Background Information}
Metals can be arranged in order according to their reactivity. The most reactive metals are placed at the top of the list while the least reactive metals are near the bottom. A metal higher in the reactivity series will displace a metal lower in the series from a solution. [DIAGRAM OF REACTIVITY SERIES]

\subsubsection*{Materials}
Steel wool*, copper (II) sulphate*, water, *beaker.

\subsubsection*{Preparation}
\begin{enumerate}
\item{Prepare a copper (II) sulphate solution by dissolving one spoonful of crystals in 500 mL of water.}
\end{enumerate}

\subsubsection*{Activity Procedure}
\begin{enumerate}
\item{Arrange students in groups of 2-3 and to each group give a beaker containing 20 mL of copper sulphate solution and a small piece of steel wool.}
\item{Instruct students to dip the steel wool into the copper sulphate solution and observe what happens.}
\end{enumerate}

\subsubsection*{Results and Conclusion}
When the steel wool is dipped in the copper sulphate solution, a layer of brown copper metal forms on the solution. Iron ions displace copper ions in solution and the copper ions are deposited as copper metal.

\subsubsection*{Clean Up Procedure}
\begin{enumerate}
\item{Collect all the used materials, cleaning and storing items that will be used later.}
\item{Copper sulphate solution can be stored and labelled for future use-it will contain dissolved iron ions. If copper is not stored, it should be disposed of in a pit latrine.}
\end{enumerate}

\subsubsection*{Discussion Questions}
\begin{enumerate}
\item{Explain what happens when steel wool is dipped in a solution of copper (ii) sulphate?}
\item{Write the chemical equation for the reaction.}
\end{enumerate}

\subsubsection*{Notes}
In the electrochemical series, the more reactive metals displace the less reactive metals. Hence Iron displaces copper in the solution of CuSO4.
Equation.
    CuSO4(aq)   +   Fe(s)  ->  FeSO4(aq)  +  Cu(s)


\subsection{Electrolytes}

\subsubsection*{Learning Objectives}
\begin{itemize}
\item{To distinguish between strong, weak and non-electrolytes.}
\item{To test the conductivity of electrolytes in solution and in solid form.}
\end{itemize}

\subsubsection*{Background Information}
An electrolyte is a substance that forms ions when in solution and thus conducts electricity when dissolved in water. An electrolyte does not conduct electricity when in a solid state because the ions are fixed in a crystal lattice and cannot move.  A strong electrolyte is a substance which is a good conductor of electricity, many ions are formed in solution. A weak electrolyte is a poor conductor of electricity, only few ions are formed in solution. A non-electrolyte is a substance which does not conduct electricity.

\subsubsection*{Materials}
Sodium chloride*, distilled water*, kerosene, copper (II) sulphate*, sugar, 1 M weak acid solution*, 1 M sulphuric acid*, beakers*, stirrer*, 2 carbon electrodes*, connecting wires*, 2-6 V power supply*, 1.5 V bulb*

\subsubsection*{Hazards and Safety}
\begin{itemize}
\item{Concentrated sulphuric acid is corrosive to skin and clothes. Avoid contact with skin and eyes. Neutralize spills with sodium hydrogen carbonate (baking powder).}
\end{itemize}

\subsubsection*{Preparation}
\begin{enumerate}
\item{Set up a circuit with the electrodes, bulb and power source in series. [DIAGRAM] Test the circuit by touching the electrodes and confirming that the bulb lights.}
\item{Prepare about 100 mL of approximately 1 M solutions sulphuric acid and citric or acetic acid.}
\end{enumerate}

\subsubsection*{Activity Procedure}
\begin{enumerate}
\item{Explain to students how the circuit is set up and demonstrate electric flow by touching the electrodes together to light the bulb.}
\item{Instruct a student to come to the front and test the conductivity of pure water by putting both electrodes in a beaker of pure water.}
\item{On a piece of paper put a small pile of sodium chloride. Instruct one student to come up and test if the solid sample conducts electricity but putting both electrodes in the pile. Note: Make sure the electrodes do not touch each other.}
\item{Pour the sodium chloride into the pure water and stir until dissolved. Instruct the student to now test the salt solution for conductivity.}
\item{Clean the electrodes with distilled water.}
\item{Repeat the test with citric acid crystals, first testing the conductivity of the solid crystals and then adding to pure water and testing the solution.}
\item{Clean the electrodes with distilled water.}
\item{Repeat the test with sugar crystals, first testing the conductivity of the solid crystals and then adding to pure water and testing the solution. Guide students to discuss their observations for the three solutions.}
\item{In one beaker put the 1 M sulphuric acid, in another beaker put 1 M citric acid. Put the electrodes first in the sulphuric acid. Rinse the electrodes and then put them in the citric acid solution. Note the difference in the brightness of the bulb.}
\end{enumerate}

\subsubsection*{Results and Conclusion}
The sodium chloride is a strong electrolyte. The bulb will light and will be bright.
Citric acid is a weak electrolyte. The bulb will light but it will be dimmer than in the sodium chloride solution.
Sugar is a non-electrolyte. The bulb will not light.
The bulb should burn brighter in the sulphuric acid solution than in the citric acid solution even though they are the exact same concentration. This is because the sulphuric acid dissociates completely, you have 1 M solution of hydrogen ions. The citric acid is weak, thus will only partially dissociate.

\subsubsection*{Clean Up Procedure}
\begin{enumerate}
\item{Collect all the used materials, cleaning and storing items that will be used later. No special waste disposal is required.}
\end{enumerate}

\subsubsection*{Discussion Questions}
\begin{enumerate}
\item{Classify the substances tested as strong electrolyte, weak electrolytes or non-electrolytes.}
\item{How does this experiment demonstrate the difference between a strong electrolyte and a weak electrolyte?}
\item{Explain using a diagram why solid sodium chloride does not conduct electricity while sodium chloride solution does.}
\end{enumerate}

\subsubsection*{Notes}
Water does not need to be distilled, but prior to the experiment you should confirm that your tap water is not hard enough to conduct electricity.
If time and resources are available, different electrolytes and non electrolytes can be tested such as.
Strong electrolytes: copper (II) chloride, potassium iodide, magnesium sulphate, sodium carbonate
Weak electrolytes: acetic acid (vinegar), lemon juice
Non-electrolytes: ethanol, kerosene


\subsection{Conservation of Energy}

\subsubsection*{Learning Objectives}
\begin{itemize}
\item{To explain the law of conservation of energy.}
\item{To demonstrate the conversion of chemical energy to electrical energy.}
\end{itemize}

\subsubsection*{Background Information}
Chemical energy is potential energy stored in the bonds of chemical compounds. Electrical energy is energy from the movement of charge. It is possible to convert from chemical energy to electrical energy and vise versa.

\subsubsection*{Materials}
copper (II) sulphate, zinc plate*, copper metal*, sand paper

\subsubsection*{Preparation}
\begin{enumerate}
\item{Prepare a 2 M solution of copper (II) sulphate.}
\item{Clean the solid piece of copper and zinc using steel wool or sand paper.}
\end{enumerate}

\subsubsection*{Activity Procedure}
\begin{enumerate}
\item{Connect the following materials in a series: zinc cathode-connecting wire-ammeter-copper anode. [DIAGRAM]}
\item{Dip the zinc and copper electrodes into the copper (II) sulphate solution.}
\item{Read the current on the ammeter.}
\end{enumerate}

\subsubsection*{Results and Conclusion}
When the circuit is connected a current of 0.06 A is indicated on the ammeter. The current on the ammeter is likely to read 0.06A but may vary according to the concentration of solution and size of electrodes. The electric current produced indicates that the chemical energy inherent in the electrodes and the electrolyte solution is converted to electrical energy.

\subsubsection*{Clean Up Procedure}
\begin{enumerate}
\item{Collect all the used materials, cleaning and storing items that will be used later.}
\item{Copper sulphate solution can be stored for later use.}
\end{enumerate}

\subsubsection*{Discussion Questions}
\begin{enumerate}
\item{Is it possible to convert electrical energy to chemical energy? How?}
\item{What energy changes occur when: (a) Iron or steel is electroplated with chromium in industry? (b) Lightning strikes in the sky? (c) Water changes to steam?}
\end{enumerate}

\subsubsection*{Notes}


\subsection{Mechanism of Electrolysis}

[INSERT ACTIVITY HERE]

\subsection{Electrolysis: Effect of Electrolyte}

\subsubsection*{Learning Objectives}
\begin{itemize}
\item{To set up electrolytic cells using different electrolytes.}
\item{To explain how choice of electrolyte can affect the products of electrolysis.}
\end{itemize}

\subsubsection*{Background Information}
Electrolysis is the use of electricity to drive an otherwise unfavourable electrochemical reaction. When electricity is passed through an electrolyte, ions move towards the electrode which attracts them strongest. At each electrode there is an electrochemical reaction which creates a product at that electrode. The nature of ions present in the electrolyte solution affect which products are formed at the cathode and anode.

\subsubsection*{Materials}
 copper (II) sulphate*, potassium iodide solution, magnesium sulphate*, beaker*, carbon electrodes*, connection wire, masking tape, 3 to 6V power source*, sand paper

\subsubsection*{Preparation}
\begin{enumerate}
\item{Prepare a solution of copper (II) sulphate by adding approximately one teaspoon of copper (II) sulphate crystals to about 200 mL of water (or reuse the copper sulphate solution from the Electrolysis: Effect of Electrode experiment)}
\item{Prepare about 200 mL of 10% of potassium iodide solution.}
\item{Prepare a solution of magnesium sulphate by dissolving approximately one teaspoon of crystals in about 200 mL of water.}
\item{Use wire to connect one carbon electrode to either end of the power supply. [electrode diagram]}
\end{enumerate}

\subsubsection*{Activity Procedure}
\begin{enumerate}
\item{In one clear beaker put copper (II) sulphate solution}
\item{Clean the electrodes using sand paper and place them into the beaker of copper sulphate, making sure that the electrodes do not touch each other.  Have students record their observations. After 3 minutes, remove the electrodes and allow students to examine them carefully.}
\item{Clean the electrodes well using water and sand paper. Repeat steps 1 and 2, this time with potassium iodide solution instead of copper sulphate.}
\item{Repeat with magnesium sulphate solution. Make sure to clean the electrodes well each time and record all of your observations.}
\end{enumerate}

\subsubsection*{Results and Conclusion}
When copper sulphate is electrolysed solid brown copper metal will form at the cathode while at the anode oxygen gas will form. This gas can be collected and tested using a glowing splint.
When potassium iodide is electrolysed, hydrogen gas will form at the cathode while at the anode dark iodine solid will form.
When magnesium sulphate is electrolysed, hydrogen gas will form at the cathode while at the anode oxygen gas will form.

\subsubsection*{Clean Up Procedure}
\begin{enumerate}
\item{Collect all the used materials, cleaning and storing items that will be used later. Electrodes should be cleaned well using sand paper and water.}
\item{Copper (II) sulphate solution that is still blue should be kept in a bottle "copper (II) sulphate" to be used in future experiments. Copper (II) sulphate solution that has lost its colour is now a solution of dilute sulphuric acid. Label and store for later use.}
\item{Potassium iodide solution now contains iodine. Ascorbic acid can be used to reduce the iodine back to iodine ions and the solution can be used again.}
\end{enumerate}

\subsubsection*{Discussion Questions}
\begin{enumerate}
\item{Write half reactions for the reactions occurring at the anode and cathode for each of the electrolytes tested.}
\item{What effect did changing the electrolyte have on the products of the electrolysis?}
\item{Explain why changing the electrolyte in this experiment changed the products formed.}
\end{enumerate}

\subsubsection*{Notes}
The products of electrolysis depend on three conditions:
1. Concentration of the electrolyte solution
2. Nature of the electrode (especially anode)
3. Position of the ions in the electrochemical series
DIAGRAM OF ELECTROCHEMICAL SERIES
This experiment examines the third factor. Ions lower in the electrochemical series will be discharged in preference to ions higher in the series. For example, SO42- ions will remain in solution while OH- is dischared. H+ ions will be discharged in preference to K+ ions.
In this experiment because water is present, OH- and H+ ions are available in each solution. The salt dissolved provides competing ions. When copper sulphate solution is present, copper is deposited at the cathode in preference to hydrogen gas because it is lower in reactivity series than hydrogen. On the other hand, when potassium iodide is electrolysed, the potassium remains in solution while H+ ions are reduced at the cathode creating hydrgen gas.
If halogen ions compete with hydroxide ions, the halogen ions will be discharged at the anode if they are in high enough concentration.


\subsection{Electrolysis: Effect of Electrode}

\subsubsection*{Learning Objectives}
\begin{itemize}
\item{To set up an electrolytic cell.}
\item{To explain how choice of electrode can affect the product of electrolysis.}
\end{itemize}

\subsubsection*{Background Information}
Electrolysis is the use of electricity to drive an otherwise unfavourable electrochemical reaction. When electricity is passed through an electrolyte, ions move towards the electrode which attracts them strongest. At each electrode there is an electrochemical reaction which creates a product at that electrode. The material that the electrode is made of can affect which product is formed at that electrode.

\subsubsection*{Materials}
 copper (II) sulphate*, beaker*, carbon electrodes*, copper electrodes*, connection wire, masking tape, 3 to 6V power source*, sand paper

\subsubsection*{Preparation}
\begin{enumerate}
\item{Prepare a solution of copper (II) sulphate by adding approximately one teaspoon of copper (II) sulphate crystals to about 200 mL of water.}
\item{Use wire to connect one electrode to either end of the power supply. [DIAGRAM]}
\item{Use sand paper and water to clean the electrodes.}
\end{enumerate}

\subsubsection*{Activity Procedure}
\begin{enumerate}
\item{In one clear beaker put copper (II) sulphate solution.}
\item{Put both carbon electrodes into the beaker of copper sulphate, making sure that the electrodes do not touch each other.}
\item{Identify the cathode and anode. Record observations about what happens at each electrode. After 3 minutes remove the electrodes from solution and allow students to examine them carefully. Have students record their observations.}
\item{Repeat the experiment using the copper electrodes instead of the carbon electrodes. Guide students to observe the appearance of both electrodes before and after the electrolysis.}
\end{enumerate}

\subsubsection*{Results and Conclusion}
(1) When copper (II) sulphate is electrolysed using the carbon electrodes the blue colour should fade as copper ions are deposited as copper metal on the cathode.  At the anode gas bubbles will be observed forming. The gas produced is oxygen. This can be proven if it is collected in a test tube and tested with a glowing splint (oxygen will relight the glowing splint).
(2) When copper (II) sulphate is electrolysed using copper electrodes the solution remains blue. Copper ions are deposited as metal at the cathode and at the anode reduction of the copper regenerates the ions in solution. After some time, it should be observed that the anode becomes smaller as the copper is oxidized while the anode becomes larger as the copper ions are deposited (reduced). Concentration and pH of the solution should remain unchanged.

\subsubsection*{Clean Up Procedure}
\begin{enumerate}
\item{Collect all the used materials, cleaning and storing items that will be used later. Electrodes should be cleaned with sand paper and water. No special waste disposal is required.}
\item{Copper (II) sulphate solution that is still blue should be kept in a bottle "copper (II) sulphate" to be used in future experiments.}
\item{Copper (II) sulphate solution that has lost its colour is now a solution of dilute sulphuric acid. Label and store for later use.}
\end{enumerate}

\subsubsection*{Discussion Questions}
\begin{enumerate}
\item{Write half reactions for the reactions occurring at the anode and cathode (a) for the carbon electrodes (b) for the copper electrodes.}
\item{What effect did changing the electrodes from carbon to copper have on the products of the electrolysis?}
\item{Explain why changing the electrode in this experiment changed the products formed.}
\item{Explain which products you would expect to see in an experiment with (a) a carbon anode and a copper cathode and (b) a copper anode and a carbon cathode.}
\end{enumerate}

\subsubsection*{Notes}



\subsection{Faraday's First Law of Electrolysis}

\subsubsection*{Learning Objectives}
\begin{itemize}
\item{To explain the meaning and significance of Faraday's First Law of Electrolysis.}
\item{To verify Faraday's First Law experimentally.}
\end{itemize}

\subsubsection*{Background Information}


\subsubsection*{Materials}
magnesium sulphate*, water, 2 burettes of 50 mL capacity, two clean carbon electrodes*, connecting wires*, masking tape, 4-6V power supply*, ammeter, stopwatch*, 1.5 L plastic water bottle, 10 L bucket

\subsubsection*{Preparation}
\begin{enumerate}
\item{Cut the bottom 6 cm from the bottle of the water bottle to form a short beaker.}
\item{Use tape to connect the power supply, ammeter, and 2 carbon electrodes in the following circuit: (ILLUSTRATION)
Leave one of the wires unconnected from the ammeter. During the activity you will connect it.}
\end{enumerate}

\subsubsection*{Activity Procedure}
\begin{enumerate}
\item{Dissolve 5 spoons of magnesium sulphate in about 300 mL water.}
\item{Close the stopcock of each burette and fill both with magnesium sulphate solution.}
\item{Insert one electrode into the top of each burette.}
\item{Add the rest of the magnesium sulphate solution to the beaker.}
\item{Turn the bucket upside-down and place it next to the beaker so that the two containers touch.}
\item{Place your thumb over the open end of one burette. Invert the burette so that the end with the stopcock is pointing up and lower the burette into the beaker. Do not remove your thumb until the open end of the burette it well below the top of the solution in the beaker. The solution in the burette should remain in the burette as long as the stopcock is closed.}
\item{Tape the burette to the bucket of water. Keep the open end under the solution in the beaker! The burette should be pointed almost straight up. (ILLUSTRATION)}
\item{Repeat the previous two steps for the other burette.}
\item{Connect the remaining wire to the ammeter to complete the circuit and immediately start the stopwatch. Bubbles should form at both electrodes and all gas produced should enter into the burettes for collection.}
\item{Record the reading on the ammeter and the volume of gas produced at the cathode after every minute for ten minutes.}
\item{ADD CALCULATIONS IN THIS SECTION}
\end{enumerate}

\subsubsection*{Results and Conclusion}


\subsubsection*{Clean Up Procedure}
\begin{enumerate}
\item{The chemicals used in this experiment do not require any special disposal.}
\end{enumerate}

\subsubsection*{Notes}
If magnesium sulphate is unavailable, dilute sulphuric acid* may be used instead. This is more dangerous, however. If using sulphuric acid, save the final solution - it is still dilute sulphuric acid and may be used for other experiments. While other salts can be used as electrolytes, many will produce other products and are therefore not advised. The calculations for this experiment depend on knowing the amount of electricity passing in the system; therefore the ammeter is required. If an ammeter is unavailable, the teacher should still demonstrate electrolysis, and should still verify Faraday's Second Law.

\subsection{Faraday's Second Law}

\subsection{Electroplating}

\subsubsection*{Learning Objectives}
\begin{itemize}
\item{To explain the meaning and significance of electroplating.}
\item{To electroplate a metal with another metal.}
\end{itemize}

\subsubsection*{Background Information}
Electroplating is the process of coating a metal by depositing another metal through an electrolytic process. Electroplating is used to coat steel materials with other metals such as silver, gold or chromium. Electroplating is popular in automobile manufacture, jewelry making and the making of many materials used in hospitals and in the kitchen.

\subsubsection*{Materials}
copper (II) sulphate*, iron nail, sand paper, 4-6V  power source*, connection wires, masking tape, beakers*, carbon electrode*

\subsubsection*{Preparation}
\begin{enumerate}
\item{Clean an iron nail thoroughly using sand paper.}
\item{Connect the power source in series with the iron nail and the carbon electrode with the iron nail as the cathode (connected to the + end of the battery) and the carbon electrode as the anode (connected to the - end of the battery).}
\item{Prepare a 2 M solution of copper (II) sulphate and acidify it by adding some dilute sulphuric acid.}
\end{enumerate}

\subsubsection*{Activity Procedure}
\begin{enumerate}
\item{Insert the nail and the carbon electrode into the acidified copper (II) sulphate solution.}
\item{Observe and record the changes on both the electrodes after 30 minutes.}
\end{enumerate}

\subsubsection*{Results and Conclusion}
The nail will become coated in a thin layer of brown copper metals-we have electroplated iron with copper. At the carbon electrode (anode) oxygen gas will be formed.

\subsubsection*{Clean Up Procedure}
\begin{enumerate}
\item{The copper sulphate can be kept and labelled as "acidified copper sulphate" for future use.}
\item{The nail may be cleaned using sand paper.}
\item{Collect all the used materials, cleaning and storing items that will be used later.}
\end{enumerate}

\subsubsection*{Notes}
If the copper coat does not stick well to the nail, it may be necessary to soak it in dilute sulphuric acid and then cleaned with sand paper or steel wool prior to the experiment.


%==============================================================================

\chapter{Organic Chemistry}

Organic chemistry is the study of complex carbon-based compounds. In ordinary level, students should become familiar with basic naming, properties, and reactions of simple organic compounds. This chapter presents a variety of activities to make these topics more hands-on and more connected to daily life.

\subsection{Alkanes}

Students first study alkanes, hydrocarbon chains of various length. Many alkanes are common in daily life and a useful activity is to present them to students to observe their properties.

\subsubsection*{Learning Objectives}
\begin{itemize}
\item{To observe common alkanes}
\item{To observe a correlation between chain length and state of matter in alkanes}
\end{itemize}

\subsubsection*{Materials}
butane lighter (kibiriti cha gasi), petrol (mafuta ya gari), kerosene (mafuta ya taa), vaseline (mafuta ya kupaka), candle wax (mabaki ya mishumaa), pin or syringe needle

[FINISH ACTIVITY]


\subsection{Preparation of Ethanol by Fermentation of Sugar}

After learning about the simple hydrocarbons (alkanes, alkenes, and alkynes), students should learn about substituted hydrocarbons. Substituted hydrocarbons are simple hydrocarbons with functional groups. Functional groups are more reactive parts of organic molecules, usually involving atoms such as oxygen, nitrogen, sulphur, and the halogens. One of the functional groups is the hydroxyl group, -OH. Organic molecules with a hydroxyl group are called alcohols, e.g. methanol, ethanol, butanol.

In daily life, the most common of the alcohols is ethanol, \ce{CH3CH2OH}. Ethanol is a colourless liquid with a characteristic odour. Ethanol may be produced directly from petroleum but ethanol produced for human consumption is prepared by a biological process, the fermentation of carbohydrates. Fermentation is the chemical breakdown of a substance by bacteria, yeasts, or other microorganisms. Yeast works on sugar to break it down into alcohol and carbon dioxide gas. Fermentation of starch or sugar produces many common alcoholic beverages, e.g. beer and wines.

The following activity may be used to show the preparation of ethanol by fermentation.

\subsubsection*{Learning Objectives}
\begin{itemize}
\item{To prepare ethanol in the laboratory}
\end{itemize}

\subsubsection*{Materials}
empty bottle*, delivery tube*, spatula*, yeast, sugar, lime water*

\subsubsection*{Activity Procedure}
\begin{enumerate}
\item{Arrange students into groups of 4-6. To each group give one plastic bottle with lid, one delivery tube, and a beaker.}
\item{Prepare a gas generator by making a hole in the lid of a plastic bottle and passing a delivery tube through the hole. Make sure there is an airtight seal.}
\item{Give each group a beaker or test tube containing 10-20 mL of lime water and instruct them to put the open end of the delivery tube into this solution.}
\item{Instruct students to make a sugar solution by dissolving about 3 full spoonfuls of sugar into about 100 mL of water.}
\item{Have students put about 1.5 teaspoons of yeast into the sugar solution and seal the bottle. Guide students to make observations.}
\item{Direct students to label their containers and set them aside in a safe place. Have students come back each day to make observations about any change in appearance in the lime water.}
\item{After 3 days, instruct students to write down observation about the appearance of the lime water. Over time, the lime water should turn milky.}
\item{Guide students to investigate the products formed inside the bottle by observing appearance and smell of the solution remaining.}
\end{enumerate}

\subsubsection*{Results and Conclusion}
When yeast is added to the sugar soluiton and left for some time a gas is generated. The gas turns the limewater milky which confirms that it is carbon dioxide. The solution remaining in the bottle with have a faint smell of alcohol (pombe) which shows the presence of ethanol.

\subsubsection*{Clean Up Procedure}
\begin{enumerate}
\item{Gas generators can be collected and stored for later use.}
\item{Collect all the used materials, cleaning and storing items that will be used later. No special waste disposal is required.}
\end{enumerate}

\subsubsection*{Discussion Questions}
\begin{enumerate}
\item{What is the role of yeast in this experiment?}
\item{What happens to the lime water in this experiment? What does this indicate?}
\item{Where is this process applied in the village?}
\end{enumerate}

\subsubsection*{Notes}
If gas generators have been made in a previous experiment, these can be used here.

\subsection{Reaction of Ethanol with Oxygen}

One property of alcohols is that they readily combust. Ethanol burns readily in air with an almost colourless flame, producing carbon dioxide and water.

\subsubsection*{Learning Objectives}
\begin{itemize}
\item{To describe the properties of alcohols.}
\end{itemize}

\subsubsection*{Materials}
Colourless spirit, match box, soda cap with plastic removed or lid of jam jar, knife.

\subsubsection*{Hazards and Safety}
\begin{itemize}
\item{Ethanol is flammable and the flame is hot. This activity should not be done on or around plastic or cloth material. Advise students that the flame may be colourless, thus special care must be taken. An ethanol flame can be extinguished with water if needed.}
\end{itemize}

\subsubsection*{Activity Procedure}
\begin{enumerate}
\item{Arrange students in groups of 2-3. To each group give a soda cap with approximately 1 mL of ethanol.}
\item{Instruct students to light a match and touch it to the ethanol and write their observations. Students should see and feel a flame, this is the combustion of ethanol.}
\end{enumerate}

\subsubsection*{Results and Conclusion}
When the lit match was brought to the flame the ethanol burns in the presence of oxygen. A colourless flame will form.  The balanced chemical equation for this reaction is
C2H5OH(l)+3O2(g)-->3H2O(g)+2CO2(g)

\subsubsection*{Clean Up Procedure}
\begin{enumerate}
\item{Collect all the used materials, cleaning and storing items that will be used later. No special waste disposal is required.}
\end{enumerate}

\subsubsection*{Discussion Questions}
\begin{enumerate}
\item{Explain what happens when alcohol is lit with a  match.}
\item{Write a balanced chemical equation for the combustion of ethanol.}
\item{This experiment would not have worked if you used beer, although it does contain some ethanol. Explain.}
\end{enumerate}

\subsubsection*{Notes}

\subsection{Oxidation of Ethanol}

The oxidation of ethanol is a good example of a reaction in organic chemistry where one organic compound is converted into another. There are two methods for oxidizing ethanol that are viable in a simple laboratory - each are presented with an activity. The first activity shows the biological oxidation of ethanol to ethanoic (acetic) acid. This activity requires very little class time but the experiment itself may take many days. The second activity shows the chemical oxidation of ethanol to ethanal (acetaldehyde) with potassium permanganate. This activity is much faster, but the product is difficult to prove, and the reaction produces a very bad ordour.

\subsection{reaction of alcohol and carboxylic acid}

One organic reaction of major importance is esterification. Esterification is the formation of an ester (ROOR') group through the reaction of an alcohol and a carboxylic acid. Many esters are volatile and their production can be observed by smell.

This activity presents an opportunity for students to perform their own organic reaction. Ideally students perform this activity in pairs.

\subsubsection*{Learning Objectives}
\begin{itemize}
\item{To describe the reaction between spirit(ethanol) and citric acid}
\end{itemize}

\subsubsection*{Materials}
citric acid powder*, ethanol*, 5 M sulphuric acid*, beakers*, tea spoon.

\subsubsection*{Hazards and Safety}
\begin{itemize}
\item{((battery acid))}
\end{itemize}

\subsubsection*{Preparation}
\begin{enumerate}
\item{Make a saturated solution of citric acid and put in one of the beakers.}
\end{enumerate}

\subsubsection*{Activity Procedure}
\begin{enumerate}
\item{Take a small amount of the citric acid solution about three water cap and pour in the second beaker.}
\item{Into the second beaker add about one capful of battery acid and mix.}
\item{To the mixture above add about three capfuls of spirit and mix. Observe the smell.}
\end{enumerate}

\subsubsection*{Results and Conclusion}
Spirit reacts like ethanol and reacted with citric acid in the presence of acidic medium to produce a esther with fragrant smell.

\subsubsection*{Clean Up Procedure}
\begin{enumerate}
\item{Collect all the used materials, cleaning and storing items that will be used later. No special waste disposal is required.}
\end{enumerate}

\subsubsection*{Discussion Questions}
\begin{enumerate}
\item{Why was battery acid added to the citric acid solution?}
\item{What smell did you detect in your experiment?}
\item{can you tell what the product is? write the chemical reaction equation assuming that the spirit contains only ethanol.}
\end{enumerate}

\subsubsection*{Notes}
Citric acid has three carboxylic acid groups. You might draw its structure for students on the board: ((ILLUSTRATION))

\subsection{Preparation of Soap}

An organic reaction of great practical value is saponification. Saponification is the reaction of a fatty acid with a strong base to form soap. The following activity is useful for demonstating to students both an organic reaction with a dramatic result as well as explaining the manufacture of a common substance.

[INSERT ACTIVITY HERE]

%==============================================================================

\chapter{Soil Chemistry}

[SECTION STILL UNDER CONSTRUCTION]

%soil formation
%soil pH
%liming
%ammonium sulphate and urea
%leaching

%==============================================================================
\end{document}
