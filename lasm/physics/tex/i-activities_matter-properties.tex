\section{Properties of Matter}

\subsection{Surface Tension in Bubbles}

\subsubsection*{Learning Objectives}
\begin{itemize}
\item{To observe the effect of surface tension in soapy water}
\end{itemize}

\subsubsection*{Background}
Surface tension is the force which holds the molecules together on the surface of a liquid.  It can be strong enough that insects can walk on water, or objects with high density can float on a liquid.  Bubbles form under the force of surface tension.  As the surface of a liquid naturally forms a shape with the smallest surface area, a thin film of liquid will form a sphere, held together by surface tension around air.  The air inside pushes out with equal force that air outside pushes in.  Bubbles don't form easily in water, so a soapy solution is needed.

\subsubsection*{Materials}
thin piece of wire (approximately 30cm), water, detergent, glycerin (optional)

\subsubsection*{Preparation Procedure}
\begin{enumerate}
\item{Bend the wire into a loop 2 to 3 cm in diameter.}
\item{Continue to bend the wire so that it circles around the circumference of this circle several times.}
\item{Leave a straight piece several centimeters long to use as a handle.  This is the bubble blower.}
\item{prepare a concentrated solution of detergent by mixing powdered soap in water with a small amount of glycerin.}
\end{enumerate}

\subsubsection*{Activity Procedure}
\begin{enumerate}
\item{Dip the circular part of the bubble blower into the detergent.}
\item{When you remove the bubble blower from the solution, a thin soapy film should remain across the circle.}
\item{Gently blow through the center of the circle. With a little practice, you should be able to cause a spherical bubble to separate from the blower and float away.}
\end{enumerate}

\subsubsection*{Results and Conclusion}
When you blow through the bubble blower, we can see that the tension is pulling it back towards a at surface. Once an independent bubble is formed, we see that it forms a nearly perfect sphere. This is because the surface is under tension. This tension forces the bubble to form the shape with the minimum surface area, a sphere.

\subsubsection*{Cleanup Procedure}
\begin{enumerate}
\item{Return all materials to their proper places.}
\end{enumerate}

\subsubsection*{Discussion Questions}
\begin{enumerate}
\item{What is inside the bubble?}
\item{Compare the pressure inside the bubble to the pressure outside the bubble.}
\item{Why do bubbles form easily in soapy water but not in fresh water?}
\end{enumerate}

\subsubsection*{Notes}
Try making the solution without glycerin and note any different results.  It will be seen that the bubbles have colours; this is due to thin film interference, or the refraction of white light in a thin liquid medium.

\subsection{Changing Surface Tension}

\subsubsection*{Learning Objectives}
\begin{itemize}
\item{To observe the effect of surface tension}
\item{To observe the effect of an impurity on the surface tension of water}
\end{itemize}

\subsubsection*{Background Information}
Surface tension holds the molecules of a liquid together at the surface.  However, the surface tension is not uniform; it depends on the composition of the liquid as well as the other forces present.  We can change the surface tension and observe the effects.

\subsubsection*{Materials}
cup or dish, water, ground black pepper, soap or detergent

\subsubsection*{Preparation Procedure}
\begin{enumerate}
\item{Make sure that the cup or dish is clean, and has no soap or detergent residue.}
\end{enumerate}

\subsubsection*{Activity Procedure}
\begin{enumerate}
\item{Fill the cup or dish with clean water.}
\item{Sprinkle ground black pepper over the surface of the water in a way that the pepper is distributed evenly and covers the whole surface.}
\item{Apply a small amount of soap or detergent to one finger.}
\item{Touch this finger to the surface of the water in the center of the cup or dish.}
\end{enumerate}
\subsubsection*{Results and Conclusion}
When you touch your finger to center of the surface of the water, the pepper moves quickly towards the edge of the water.  This is because as you touch your finger to the surface, you introduce a small amount of soap or detergent, lowering the surface tension at that point.  The surface of the water is now unbalanced; the surface tension near the edge is pulling the surface outwards more strongly than the surface tension at the center is pulling the surface inwards. As there is a net force on the surface outwards towards the edge, the surface moves, pulling the pepper along with it to the edges of the cup or dish.

\subsubsection*{Cleanup Procedure}
\begin{enumerate}
\item{Dispose of the water and return all materials to their proper places.}
\end{enumerate}

\subsubsection*{Discussion Questions}
\begin{enumerate}
\item{Why does the pepper move quickly away from the center?}
\item{What happens if you add other liquids instead of soap?}
\end{enumerate}

\subsubsection*{Notes}
Floating objects will tend towards the area with highest surface tension.  The pepper in this case is following the higher surface tension towards the side of the container.

subsection*{Water Dome}

\subsubsection*{Learning Objectives}
\begin{itemize}
\item{To observe the strength of surface tension and cohesion}
\end{itemize}

\subsubsection*{Background Information}
Surface tension can be surprisingly strong.  Insects and objects can float on top of water because of its tension; also water itself holds together in droplets.  Combining droplets allows cohesion to form more bonds, creating a larger droplet with the same surface tension.

\subsubsection*{Materials}
coin, water, syringe or eyedropper

\subsubsection*{Activity Procedure}
\begin{enumerate}
\item{Place a coin on the table.}
\item{Using the syringe or eyedropper, carefully drop individual water drops onto the coin. With some practice, you should be able to get quite a few drops onto the coin before the water spills over, creating a dome of water.}
\end{enumerate}

\subsubsection*{Results and Conclusion}
As you add more water to the coin, the water forms a larger and larger dome rather than spilling off the coin.  The surface tension of the water holds it together against the force of gravity, which is trying to pull the water off of the coin.  

\subsubsection*{Cleanup Procedure}
\begin{enumerate}
\item{Return all materials to their proper places.}
\end{enumerate}

\subsubsection*{Discussion Questions}
\begin{enumerate}
\item{How many drops did you think would fit on the coin?  How many actually did fit?}
\item{Why does the water dome eventually break?}
\item{Describe all the forces acting on the water.}
\end{enumerate}

\subsubsection*{Notes}
The most dramatic results can be see on small coins, though big coins can also be used.  You should be able to get at least 60 drops on a 200 shilling coin.

\subsection{Cohesion in a Moving Liquid}

\subsubsection*{Learning Objectives}
\begin{itemize}
\item{To observe the effect of cohesion on moving water}
\end{itemize}

\subsubsection*{Background Information}
Cohesion is the force between molecules in a liquid.  It holds liquids like water together if they are standing still or moving.

\subsubsection*{Materials}
empty 0.5 litre bottle, water, pin or small nail

\subsubsection*{Preparation Procedure}
\begin{enumerate}
\item{Make five small holes at the bottom of the side of the bottle.  Make the holes close together (about 5 mm apart) with the syringe needle or nail.}
\end{enumerate}

\subsubsection*{Activity Procedure}
\begin{enumerate}
\item{Pour water into the bottle and allow the water to start flowing out of the holes at the bottom.}
\item{Using your thumb and forefinger, pinch the streams of water together so that they form a single stream.}
\item{To separate the streams again, pass your hand in front of the holes and the five streams will appear again.}
\end{enumerate}

\subsubsection*{Results and Conclusion}
When you pinch the streams of water together, they form a single stream or a few streams.  Water has a tendency to cling to itself due to its surface tension and cohesion. As you bring the streams together, you allow the water to stick to itself forming a single stream. Passing your hand in front again stops the flow of water at the holes and allows it to start again, which it will do in five streams.

\subsubsection*{Cleanup Procedure}
\begin{enumerate}
\item{Return all materials to their proper places.}
\end{enumerate}

\subsubsection*{Discussion Questions}
\begin{itemize}
\item{How did the behavior of the water streams change as the level in the bottle decreased?}
\item{What force holds the water streams together?}
\item{Why do the streams eventually split?}
\end{itemize}

\subsubsection*{Notes}
Be careful not to let the holes overlap or be too far apart.  This will cause the water to form one stream from the beginning or make it impossible to pinch the streams.  Practice pinching the water and make new holes in necessary.

\subsection{Elasticity}

\subsubsection*{Learning Objectives}
\begin{itemize}
\item{To explain the concept of elasticity.} 
\item{To demonstrate elasticity.} 
\item{To deduce the relationship between the applied force (tension) and the increase in length (extension) from Hooke's law.} 
\end{itemize}

\subsubsection*{Background Information}
The force applied to a spring is related to the extension of the spring as it stretches under the force.  This is described by Hooke's Law.  Provided that the elastic limit of the spring is not exceeded (meaning that the spring is not stretched too much), the extension of a spring or other material is directly proportional to it load or tension force.  This is because of its elasticity.

\subsubsection*{Materials}
Rubber band, spring balance*, and sling shot

\subsubsection*{Hazards and Safety}
\begin{itemize}
\item{When making the small extension of the rubber/spring balance, be careful not to use too much force to avoid breaking the material.} 
\end{itemize}

\subsubsection*{Preparation Procedure}
\begin{enumerate}
\item{Collect all the required materials.}
\end{enumerate}

\subsubsection*{Activity Procedure}
\begin{enumerate}
\item{Take the rubber band or spiral spring and measure its original length.  Record this measurement.} 
\item{Hold the rubber band/spring balance and fix one end while slightly pulling on the other end. Check the increase in length after stretching and record the new length.} 
\end{enumerate}

\subsubsection*{Results and Conclusion}
There is an increase in length when a force is applied to the rubber band/spring balance.  
The increase in length of the material when there is an applied tension, along with the return of the body to its original length when the tension is removed, is known as Elasticity.  
There is a relationship between the amount of force(tension) applied and increase in length(extension).  This relationship is stated in Hooke's law.  

\subsubsection*{Clean Up Procedure}
\begin{enumerate}
\item{After finishing the demonstration, untie the rubber band/spring balance and leave it in the normal state as it was in the beginning.} 
\end{enumerate}

\subsubsection*{Discussion Questions}
\begin{enumerate}
\item{Explain why the rope or spring breaks when very high tension (force) is used.} 
\end{enumerate}

\subsubsection*{Notes}
Everything has an elestic limit where it will break or cease to be elestic.  Stretching a spring to far will cause it to excede its elastic limit and stop working.

\subsection{Determining Adhesion and Cohesion}

\subsubsection*{Learning Objectives}
\begin{itemize}
\item{To observe cohesion and adhesion forces of different liquids.}
\item{To determine adhesive forces between water molecules.}
\item{To determine cohesive forces between different liquids.}
\end{itemize}

\subsubsection*{Background Information}
Adhesion is the force of attraction between a material and its surrounds.  Cohesion is the force of the water molecules to stick to themselves. The concept of adhesive and cohesive forces can be used in different ways such as determination of  viscosity of the liquids, to explain the transportation of liquid in plant and animal. When a drop of water is placed on a sheet of glass, the water spreads because adhesive forces between glass and water are greater than the cohesive forces. When a drop of honey, cooking oil, or glycerine is placed on a sheet of glass, it remains almost spherical because the cohesive force is greater than the adhesive force.

\subsubsection*{Materials}
Sheet of glass, water, honey, glycerine, cooking oil, syringe, cotton wool, and 2 wooden blocks

\subsubsection*{Activity Procedure}
\begin{enumerate}
\item{Place 2 wooden blocks on the top of the table.}
\item{Place a piece of pane glass horizontally over the two wooden blocks.}
\item{Using a  syringe, place different drop of liquids on top glass. Record your observations.}
\end{enumerate}

\subsubsection*{Results and Conclusion}
When the drop of water is placed on top of the glass, water spreads and wets the glass. While material like honey, glycerine, and cooking oil remain in spherical shape. The adhesive forces between the water molecule and glass molecule is greater. While the  cohesive forces between the molecule of honey, glycerine and cooking oil is larger.

\subsubsection*{Clean Up Procedure}
\begin{enumerate}
\item{Use cotton wool to clean the sheet of glass.}
\item{Collect all the used materials, cleaning and storing items that will be used later.}
\end{enumerate}

\subsubsection*{Discussion Questions}
\begin{enumerate}
\item{Explain why water wets the glass while glycerine does not wet the glass.}
\item{Discuss where we apply the concept of adhesive and cohesive forces in other areas of science.}
\item{How are cohesion and viscosity related?}
\end{enumerate}

\subsubsection*{Notes}
The liquids which have the greatest forces of cohesion are also those with the highest viscosity.

\subsection{Capillarity}

\subsubsection*{Learning Objectives}
\begin{itemize}
\item{To observe the effect of capillarity in various liquids.} 
\item{To explain the mode of action of capillarity.} 
\end{itemize}

\subsubsection*{Background Information}
Capillarity, or capillary action, is possible because of the combined effects of cohesion and adhesion. Cohesion holds a liquid together, especially at the surface. Adhesion allows a liquid to attach itself to another material, like the vertical surface of a container.  

\subsubsection*{Materials}
Clear thin plastic straw, shallow container (bottom of a water bottle, jar cap), various liquids like water, spirit, cooking oil, and kerosene  

\subsubsection*{Activity Procedure}
\begin{enumerate}
\item{Pour some water into the container so that it is about 1 cm deep.} 
\item{Place one end of the straw into the liquid so that the end is completely submerged but not touching the bottom of the container.} 
\item{Observe the change in water level in the straw for about a minute.} 
\item{Repeat these steps for the other liquids and compare.} 
\end{enumerate}

\subsubsection*{Results and Conclusion}
Liquid in a thin tube will slowly climb up the tube without any visible force present.  Adhesion and cohesion are pulling the liquid up the tube.  The capillary action of the liquid depends on the diameter of the tube.  

\subsubsection*{Clean Up Procedure}
\begin{enumerate}
\item{Return all liquids to their proper containers and put them away.} 
\item{Wash the container and straw and put them away.} 
\end{enumerate}

\subsubsection*{Discussion Questions}
\begin{enumerate}
\item{What causes the liquid to move up the straw?}
\item{What causes the liquid at the edge to cling to the straw while the liquid in the middle remains lower?}
\item{Which liquid moved up the straw fastest? Which moved slowest?}
\item{What would happen if the diameter of the straw was increased? What would happen if it was decreased?}
\end{enumerate}

\subsubsection*{Notes}
Liquids are able to climb up a thin tube due to the combined effects of adhesion and cohesion. Cohesion allows a liquid to remain connected to itself while adhesion allows a liquid to remain connected to another surface. Adhesion causes a liquid to climb slightly up the side of any container; the surface tension of the liquid (cohesion) then pulls the remainder of the liquid up as well. In a normal container, the adhesive force at the side of the container is not strong enough to pull all the other liquid up. In a thin container, a larger proportion of liquid is attached to the side of the tube and a smaller proportion is being held by surface tension, so the adhesive force is strong enough to pull all the liquid up the tube.
