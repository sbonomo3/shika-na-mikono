\section{Change of State}

\subsection{Boiling at Room Temperature}

\subsubsection*{Learning Objectives}
\begin{itemize}
\item{To observe the effect of pressure on the boiling point of water} 
\item{To explain why the boiling point of water decreases with pressure} 
\end{itemize}

\subsubsection*{Background Information}
Boiling and melting points are usually assumed to be constant. For example, the melting point of ice is 0-degrees C and the boiling point is 100-degrees C. However, this is only true at STP, or Standard Temperature and Pressure. Standard pressure is 760 mm Hg, which is only true at sea level. Pressure decreases with elevation, therefore the boiling point of water also decreases. The boiling point of water at sea level will be measurably different from the boiling point on a mountain.  

\subsubsection*{Materials}
10 mL or 20 mL syringe without needle, water

\subsubsection*{Preparation Procedure}
\begin{enumerate}
\item{Collect the syringe and remove the needle.} 
\end{enumerate}

\subsubsection*{Activity Procedure}
\begin{enumerate}
\item{Fill the syringe with a small amount of water.} 
\item{Place your thumb over the opening of the syringe.} 
\item{Pull the plunger out as far as you can.} 
\item{Observe the behavior of the water in the syringe.} 
\end{enumerate}

\subsubsection*{Results and Conclusion}
When the plunger is pulled out as far as it will go (without removing it from the syringe), the water will begin to bubble, meaning that it is boiling. This is because the pressure inside the syringe is decreasing, and the boiling point of the water is decreasing with the pressure. When the boiling point is reduced to room temperature, the water begins to boil.  

\subsubsection*{Clean Up Procedure}
\begin{enumerate}
\item{Dispose of the water and return all materials to their proper places.} 
\end{enumerate}

\subsubsection*{Discussion Questions}
\begin{enumerate}
\item{Why is it difficult to pull the plunger out when your thumb is covering the syringe opening?}
\item{By pulling out the plunger, are you increasing or decreasing the pressure in the syringe?}
\item{What happens to the water when you pull the plunger as far as you can?}
\end{enumerate}

\subsubsection*{Notes}
When pulling water into the syringe, you only need to take a small amount. If you take too much water, you will not be able to reduce the pressure in the syringe before the plunger comes out.  
Before doing this activity, be sure that the students understand that the presence of bubbles means that a liquid is boiling.
