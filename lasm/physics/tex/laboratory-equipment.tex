\chapter{Laboratory Equipment}
Throughout this book you will see materials that have been marked with an asterisk (*). These are locally available materials which can be made or purchased for your laboratory. The guide for using and making these local materials are found in the following section.  

\textbf{Beakers}\\
Use: To hold liquids\\
Materials: Water bottles, juice containers, lids for bottles or jars, and a knife\\
Procedure:Take empty plastic battles of different sizes. Cut them in half. The base can be used as a beaker.  \\

\begin{flushleft}
\textbf{Delivery Tube}
\end{flushleft}
\vspace{-10pt}
Use: For the movement and collection of gases, capillary tubes, hydraulic press\\
Materials: Straws, pen tubes, or pawpaw petioles\\

\begin{flushleft}
\textbf{Needles}
\end{flushleft}
\vspace{-10pt}
Use: Compass needles, optical pins, making holes, flying wire\\
Materials: Office pins, sewing needles, needles from syringes\\

\begin{flushleft}
\textbf{Droppers}
\end{flushleft}
\vspace{-10pt}
Use: To add small amounts of liquid to something\\
Materials: 2 mL syringes\\
Procedure: Take a syringe. Remove the needle. \\

\begin{flushleft}
\textbf{Funnel}
\end{flushleft}
\vspace{-10pt}
Use: To guide liquid or powder into a small opening\\
Materials: Empty water bottles\\
Procedure: Take an empty water bottle and remove the cap. Cut them in half. The upper part of the bottle can be used as a funnel.  \\

\begin{flushleft}
\textbf{Heat Source}
\end{flushleft}
\vspace{-10pt}
Use: Heating substances\\
Materials: Candles, kerosene stoves, charcoal burners, or a moto poa stove\\
Procedure: Cut a metal can in half and add a small amount of moto poa\\


\begin{flushleft}
\textbf{Stopper}
\end{flushleft}
\vspace{-10pt}
Use: To cover the mouth of a bottle, hold a capillary tube\\
Materials: Rubber, cork, plastic water bottle cap\\
Procedure: Cut a circular piece of rubber.  If the stopper is being used to hold a capillary tube, a hole can be melted in a plastic cap or rubber stopper.\\


\begin{flushleft}
\textbf{Water Bath}
\end{flushleft}
\vspace{-10pt}
Use: To heat substances without using a direct flame\\
Materials: Heat source, water, and a cooking pot\\
Procedure: Bring water to a boil in a small aluminium pot, then place the test tubes in the water to heat the substance inside the test tube.  \\

\begin{flushleft}
\textbf{Circuit Components}
\end{flushleft}
\vspace{-10pt}
Use: Building simple circuits, Ohm's Law, amplifier, wave rectifiers\\
Materials: Broken radio, computer, stereo, other electrical devices\\
Procedure: Remove resistors, capacitors, transistors, diodes, motors, wires, transformers, inductors, rheostats, pulleys, gears, battery holders, switches, speakers and other components from the devices.\\

\begin{flushleft}
\textbf{Masses}
\end{flushleft}
\vspace{-10pt}
Use: Calibrating and using beam balance and spring balance, Hooke's Law\\
Materials: Known masses, beam balance, sand, stones, plastic bags, thread, paper, tape, pen\\
Procedure: Use a beam balance and known masses at a duka or nearby school to measure exact masses of sand or stones.  Use a marker pen to mark the masses on the stones.  If using sand, place a small piece of plastic bag on the scale pan and fill it with sand until you have the required mass.  Tie the sand in the plastic bag with thread.  Use paper and tape to make a label on the outside, marking the mass with pen.  These masses can be used in your school.
Water can also be as a known mass.  The density of water is 1.0 g/mL, so you can use a known volume of water in a bottle to create a known mass.  Be sure to account for the mass of the bottle also.\\

\begin{flushleft}
\textbf{Plane Mirror}
\end{flushleft}
\vspace{-10pt}
Use: Laws of Reflection, periscope, water prism\\
Materials: piece of thin glass, kibatari, Optional: small pieces of mirror glass are cheap or free at a glass cutter's shop\\
Procedure: Light the kibatari so that it creates a lot of smoke.  Pass one side of the glass repeatedly over the kibatari until that side is totally black.  The other side acts as a mirror.\\

\begin{flushleft}
\textbf{Iron Filings}
\end{flushleft}
\vspace{-10pt}
Use: To map magnetic fields\\
Materials: Steel wool / Iron wool used for cleaning pots\\
Procedure: Rub some steel wool between your thumb and fingers.  The small pieces that fall are iron filings.  Collect them in a matchbox or other container to use again.\\

\section{Checking Voltmeters and Ammeters/Galvanometers}
Needed: Meters to check, a couple of wires, some resistors and a fresh battery.  
Important note: There is a wrong way to hook up the meter. The needle will
try to detect down because negative and positive are swapped. If the reading
is zero, make sure that you try the opposite connection to be sure.  
\subsection{Voltmeters}
Hook up the voltmeter across the battery. The battery is probably 1.5~V, but
do not worry if you see 1.1, 1.2, even if using a brand new battery. Try not to
use a battery that reads much below 1~V on several different meters.  
\subsubsection{Unuseable Voltmeters}
\begin{itemize}
\item{Totally dead, no detection of the needle}
\item{Voltage reading jumps excessively. Ensure that the connections are solid
and test again.}
\item{Measured voltage is totally wrong, not close to 1.5~V}
\end{itemize}
\subsubsection{Useable Voltmeters}
\begin{itemize}
\item{Read a voltage close to 1.5}
\item{If the voltage if not 1.5 exactly, the voltmeter is probably working and the battery is just old a bit.}  
\end{itemize}
\subsection{Ammeters}
Hook up the ammeter in series with a resistor. Because you do not necessarily
know the condition of the ammeter before testing, be sure to have several different
resistors on hand. An ammeter may appear not to work if resistance is
too high or too low. Start testing different ammeters.  
\subsubsection{Unuseable Ammeters}
\begin{itemize}
\item{Totally dead, no detection of the needle}
\item{Current reading jumps excessively (but check connections)}
\item{Totally wrong, reads much different from other ammeters}
\end{itemize}

\section{Useable Ammeters}
Reads a current similar to other ammeters. \\
Hard to say exactly what current, 
but feel free to calculate based on your resistor using V=IR, although do not
forget that there is some internal resistance r of battery, so V = I(R + r). \\
The resistance of the resistor is usually coded on the resistor in a series in stripes -
see the instructions under Resistors in the Sources of Equipment section.  \\
Tip: You can hold the wires onto the battery with your fingers; the current
is far too low to shock you.  \\
Other: Now that you have tested to see if your voltmeters and ammeters
work, you can feel free to check all of them for accuracy, by calculating expected
values and comparing between meters. Most practicals will still work alright
with somewhat accurate meters.