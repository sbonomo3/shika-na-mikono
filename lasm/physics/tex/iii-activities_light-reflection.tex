\section{Reflection of light}

\subsection{Total Internal Reflection in Water}

\subsubsection*{Learning Objectives}
\begin{itemize}
\item{To observe the effect of Total Internal Reflection of light inside water}
\end{itemize}

\subsubsection*{Background Information}
When light passes from a high-density medium to a low-density medium, it can undergo total internal reflection if the angle of incidence is great enough.  If water is passing through water, some of it will pass into air but some will be reflected at the surface back into the water.  This is the mechanism behind fibre-optic cables, which use total internal reflection in a clear wire to move light over great distances..

\subsubsection*{Materials}
Opaque container like a Nido can, nail, torch, water, bucket

\subsubsection*{Preparation Procedure}
\begin{enumerate}
\item{Use the nail to poke a hole at the bottom of one side of the opaque container.}
\end{enumerate}

\subsubsection*{Activity Procedure}
\begin{enumerate}
\item{Pour water into the opaque container so that it can flow out of the hole near the bottom into the other container.}
\item{In a dark room, shine a torch down into the water.}
\item{Observe the light in the water falling from the hole.}
\end{enumerate}

\subsubsection*{Results and Conclusion}
The light is reflected at the surface of water (total internal re
ection), so when it travels through the stream of pouring water, it continues to be reflected inside the stream until it reaches the container below. The light that only escapes farther down in the pouring water is what we see as the glowing
effect.

\subsubsection*{Cleanup Procedure}
\begin{enumerate}
\item{Clean up the water and return materials to their proper places.}
\end{enumerate}

\subsubsection*{Discussion Questions}
\begin{enumerate}
\item{What causes the water to glow?}
\item{What is the light doing inside the water stream?}
\end{enumerate}

\subsubsection*{Notes}
While the light is being reflected inside the water stream, what we are seeing with our eyes is the light that escapes.  A certain percentage of light will escape each time it hits the surface of the water and air.  Light that enters the air at a small angle will escape and reach our eyes; light that hits at a large angle will be relected again inside the water.
