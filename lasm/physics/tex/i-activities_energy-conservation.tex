\section{Conservation of Energy}

\subsection{Potential Energy of a Spring}

\subsubsection*{Learning Objectives}
\begin{itemize}
\item{To observe the change in energy from potential to kinetic}
\end{itemize}

\subsubsection*{Background Information}
In a closed system, where no force acts on the objects, the total energy remains constant.  In the case of mechanical energy, this means that potential energy and kinetic energy can change, but their total remains the same.  Springs and other elastic materials also have potential energy in the form of elastic potential.

\subsubsection{Materials} 
Clothes pin, thread, two pencils\\

\subsubsection{Activity Procedure}
\begin{enumerate}
\item{Open the clothes pin and tie the closed end with thread so that the
clothes pin stays open against the tension of the spring.}
\item{Place the clothes pin on a table and place two pencils next to the clip, one on either side, so that the
eraser touches the tied end and the tips point out in opposite directions
along the table.}
\item{Cut the thread holding the clip open.}
\end{enumerate}

\subsubsection*{Cleanup Procedure}
\begin{enumerate}
\item{Return all materials to their proper places.}
\end{enumerate}

\subsubsection*{Discussion Questions}
\begin{enumerate}
\item{What two types of energy are being used here?}
\item{Describe the change in energy occuring here.}
\end{enumerate}
 
\subsubsection{Notes}
The spring inside the clip holds energy when it is forced to contract.  
When the clip is allowed to close, the potential energy of the spring
is converted into mechanical energy as the clip moves, forcing the pencils
away quickly.