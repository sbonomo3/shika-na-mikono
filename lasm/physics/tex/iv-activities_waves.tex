\section{Waves}

\subsection{Construction and Use of Slinky Spring}

\subsubsection*{Learning Objectives}
\begin{itemize}
\item{To construct a slinky spring} 
\item{To observe the propagation of transverse and longitudinal waves} 
\end{itemize}

\subsubsection*{Background Information}
A mechanical wave is the oscillation of particles in a medium, either side-to-side or back-and-forth along the direction of the wave's motion. A transverse mechanical wave is produced when particles oscillate side-to-side as on a guitar string. Its crests and troughs can be seen clearly. A longitudinal wave is produced when particles oscillate back-and-forth as passengers on a bus that stops and starts repeatedly. It can be seen because of the alternating compressions and rarefactions along the wave.  

\subsubsection*{Materials}
2 m of flexible steel or copper wire, long cylindrical object (rod, stick) at least 3 cm diameter

\subsubsection*{Hazards and Safety}
\begin{itemize}
\item{Every spring has a maximum length that it can be stretched. If you stretch it beyond this point, it will not work again. Don't stretch the slinky too far.} 
\end{itemize}

\subsubsection*{Preparation Procedure}
\begin{enumerate}
\item{Collect all materials.} 
\item{Hold one end of the wire against the cylindrical object.} 
\item{Use your other hand to coil the wire around the cylinder, keeping the coils close together. When you finish coiling all of the wire along the cylinder, you should have a slinky spring.} 
\end{enumerate}

\subsubsection*{Activity Procedure}
\begin{enumerate}
\item{Have one student hold one end of the spring.} 
\item{Hold the other end with your hand and stretch the slinky slightly so that the coils separate.} 
\item{Hold the slinky flat against the floor.} 
\item{Move your end of the slinky quickly from side to side (the student should keep her end stationary). Observe the motion of the slinky.} 
\item{Move your hand quickly back and forth, alternately pushing and pulling the spring. Observe the motion of the slinky.} 
\item{Move your hand quickly to one side and then back to the center only once. Observe the progression of the wave along the slinky.} 
\end{enumerate}

\subsubsection*{Results and Conclusion}
Students will see that a tansverse wave progresses by alternating crests and troughs as the points on the slinky oscillate back and forth perpendicular to the direction of the wavefront.  
Students will see that a longitudinal wave progresses as points on the slinky alternately push and pull each other in the direction of the wavefront.  
Students will see that a single transverse wave (one crest and one trough) is reflected on the opposite side, or that the crest and trough trade places when reflected.  
Students should understand that a wave is caused when particles oscillate.  

\subsubsection*{Clean Up Procedure}
\begin{enumerate}
\item{Return the slinky to its proper place.} 
\end{enumerate}

\subsubsection*{Discussion Questions}
\begin{enumerate}
\item{What is the type of wave produced when you move your hand from side to side? Describe this wave.} 
\item{What is the type of wave produced when you move your hand back and forth? Describe this wave.} 
\item{When there is only one wave crest moving along the slinky, how is it reflected at the opposite end?}
\end{enumerate}


\subsection{Construction of a Ripple Tank}

\subsubsection*{Learning Objectives}
\begin{itemize}
\item{To construct a ripple tank} 
\end{itemize}

\subsubsection*{Background Information}
A ripple tank uses a container or shallow water to create waves and observe their behavior. A light source projects the shadow of the waves onto the paper below so that all bahavior at obstacles and sources can be seen clearly.

\subsubsection*{Materials}
pane of glass 45 cm by 45 cm, four pieces of wood 50 cm long and 4 cm thick, nails or screws, putty or caulk, small ball of plastic or rubber, stiff wire, small motor from a car stereo or other device, connecting wires, two D-cell batteries, four thin pieces of wood about 30 cm long, two thin pieces of wood about 50 cm long, various small pieces of straight wood, glue


\subsubsection*{Preparation Procedure}
\begin{enumerate}
\item{Obtain the 45 cm x 45 cm piece of glass.} 
\item{Make a groove 1 cm thick along one side of each of the four pieces of wood.} 
\item{From the groove on each piece of wood, cut the wood back at a 45-degree angle.} 
\item{Fit the pane of glass into the grooves and join the pieces of wood at the corners with nails or screws. It should resemble a window frame.} 
\item{Use caulk or putty to secure the glass to the wood frame so that water cannot pass.} 
\item{Make or find a small ball of rubber or plastic (no more than 2 cm diameter).} 
\item{Bend a 6 cm piece of wire into an L shape.} 
\item{Attach one end of the wire to the small ball.} 
\item{Along two opposite sides of the frame, attach two pieces of wood 30 cm tall vertically. These should be about 10 cm from the side and directly across from each other.} 
\item{Attach the two vertical pieces of wood at the top with the 50 cm piece of wood. You should now have a beam across the ripple tank.} 
\item{Attach the motor to a piece of wood 30 cm long. Try to balance the motor so that the piece of wood stays flat.} 
\item{Attach the ball and wire to one side of the piece of wood just under the motor. If the wood is flat, the wire should extend out and down.} 
\item{Suspend the wood and motor from the beam with thread so that the ball rests about 1 cm above the glass pane.} 
\item{Connect the terminals of the motor to connecting wires and extend these wires up to the beam.} 
\item{Attach two D-cell batteries to the beam so that the connecting wires can be easily attached.} 
\item{Glue a screw to the motor so that it vibrates when it spins.} 
\end{enumerate}

\subsubsection*{Activity Procedure}
\begin{enumerate}
\item{Set up the ripple tank with water, a light source and plane paper to check that it works.}
\end{enumerate}

\subsubsection*{Results and Conclusion}
The motor on the hanging bar provides a vibration, which in turn produces waves in the water.  Different objects can be attached to the hanging bar to produce different types of waves, and the waves can be seen clearly on the paper under the tank. 

\subsubsection*{Clean Up Procedure}
\begin{enumerate}
\item{Disconnect the wires.} 
\item{Remove the water from the ripple tank.} 
\item{Return all materials to their proper places.} 
\end{enumerate}

\subsubsection*{Discussion Questions}
\begin{enumerate}
\item{Why is it necessary to attach an object to the motor?}
\item{What is the shape of a wave produced by the ball?}
\item{What is the shape of a wave produced by a ruler?}
\end{enumerate}

\subsubsection*{Notes}
It is easier to ask a carpenter to make this. Bring the glass to the shop and describe the construction of a ripple tank. The motor can be found easily in broken electrical devices.  


\subsection{Behaviour of Waves}

\subsubsection*{Learning Objectives}
\begin{itemize}
\item{To demonstrate and explain the reflection, refraction, diffraction and interference of waves} 
\end{itemize}

\subsubsection*{Background Information}
The periodic mechanism which transfers energy from one point to another is called a wave.  There are two types of waves: mechanical and electromagnetic.  While these are produced differently, they share similar properties of reflection, refraction, interference, diffraction, etc.  Reflection occurs when a wave hits a barrier and reverses or changes its direction.  Refraction occurs when a wave passes from one medium into another and changes its speed and direction.  If the wave is allowed to pass only through a small opening, it will undergo diffraction which changes the shape of the wave.  If two small openings are used, the two diffraction patterns will pass over each other, forming interference.

\subsubsection*{Materials}
Ripple tank (see the activity for constructing a ripple tank), two D-cell batteries, large white paper(flip chart), water, torch, various objects to place in the water

\subsubsection*{Hazards and Safety}
\begin{itemize}
\item{Use a small source of power to run the motor.} 
\end{itemize}

\subsubsection*{Preparation Procedure}
\begin{enumerate}
\item{Create the ripple tank as described in the construction activity.} 
\end{enumerate}

\subsubsection*{Activity Procedure}
\begin{enumerate}
\item{Place the ripple tank between stools.} 
\item{Pour some water into the ripple tank so that it is about 1.  5 cm deep.} 
\item{Place the dipper on the wood beam; it should just touch the surface of water.} 
\item{Connect the motor to the batteries.} 
\item{Place the white screen/paper underneath the stools.} 
\item{Light a torch above the ripple tank and observe the waves formed.} 
\item{Exchange the dippers so as to obtain spherical and plain waves. Using these waves you can now observe the different behaviors of waves, such as reflection, refraction, diffraction and interference. You can place obstacles in the water to see diffraction and interference.} 
\end{enumerate}

\subsubsection*{Results and Conclusion}
The manufactured and mounted ripple tank can be used to study the behavior of water waves.  For instance, when a round ball is used, a circular wavefront is formed, while a ruler produces a plane wave.  If a barrier is placed in front of the wave, the wave is reflected back on itself or in a new direction, depending on the shape of the barrier.  When the wave passes between two barriers, it diffracts and changes form.  A plane wave becomes a circular wave, and two diffracted waves interfere to form points of constructive and destructive interference.

\subsubsection*{Clean Up Procedure}
\begin{enumerate}
\item{Dismantle the ripple tank and remove the water after the experiment. If there is any water, it should be dried up.} 
\end{enumerate}

\subsubsection*{Discussion Questions}
\begin{enumerate}
\item{Besides the ripple tank, what other method can be used to propagate the water waves?}
\end{enumerate}

\subsubsection*{Notes}
A ripple tank can be used to demonstrate the properties or behavior of a wave as described above.  However, you can also use it to show the relationship between wavelength and frequency, and that the speed of a wave is constant in one medium.
