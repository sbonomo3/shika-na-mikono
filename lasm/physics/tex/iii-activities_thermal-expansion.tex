\section{Thermal Expansion}

\subsection{Bimetallic Strip}

\subsubsection*{Learning Objectives}
\begin{itemize}
\item{To identify components of a bimetallic strip} 
\item{To observe and explain the function of a bimetallic strip} 
\item{To understand the mode of action of a bimetallic strip} 
\end{itemize}

\subsubsection*{Background Information}
Solids expand when heated, but every solid has a different expansivity which depends on what material it is. For example, copper expands noticeably when heated, but wood barely expands at all.  

\subsubsection*{Materials}
Aluminium foil, paper, glue, candle, matches

\subsubsection*{Preparation Procedure}
\begin{enumerate}
\item{Cut one piece of aluminium foil and one piece of paper of equal size. About 3 cm by 10 cm works well.} 
\item{Lay the foil flat on the paper so that they are aligned.} 
\item{Glue the pieces together and leave them to dry.} 
\end{enumerate}

\subsubsection*{Activity Procedure}
\begin{enumerate}
\item{Light the candle.} 
\item{Hold the strip above the candle far enough away that it won't burn or blacken, but close enough that it can feel the heat of the flame.} 
\item{Observe any changes in the shape of the strip.} 
\end{enumerate}

\subsubsection*{Results and Conclusion}
The strip bends towards the side of the paper because the aluminium foil expands more than the paper, forcing the paper to bend in to accomodate the expanding foil.  

\subsubsection*{Clean Up Procedure}
\begin{enumerate}
\item{Snuff the candle and return all materials to their proper places.} 
\end{enumerate}

\subsubsection*{Discussion Questions}
\begin{enumerate}
\item{What happens when the strip is held above the flame with the paper on the bottom and foil on top?}
\item{What happens when the strip is held above the flame with the foil on the bottom and paper on top?}
\item{Why does this occur?}
\end{enumerate}

\subsubsection*{Notes}
A real bitmetallic strip contains two metals: usually copper and iron or two other strong metals. The metals, as with the paper and foil, have different expansivities. When they are heated, the metal with the larger expansivity expands more than the other. But as they are attached along their lengths, the more expansive metal pulls the other metal and therefore bends inward toward the metal with lower expansivity.  
In this activity, the paper acts as the metal with lower expansivity, bending inward to accomodate the expanding foil while note expanding much itself.  
It helps in this activity to hold just a piece of foil or piece of paper over the flame first to show that, alone, neither one will bend. They only bend when they are glued together.  

\subsection{Thermal Expansion of Gases}

\subsubsection*{Learning Objectives}
\begin{itemize}
\item{To observe the expansion of a gas when heated} 
\item{To explain why gases expand when heated} 
\end{itemize}

\subsubsection*{Background Information}
Unlike solids, gases expand quickly when heated. Because the molecules of a gas are spread out and moving, the volume increases quickly when more energy is added.  

\subsubsection*{Materials}
Source of heat, small cooking pot, water, clear thin plastic straw or tubing

\subsubsection*{Hazards and Safety}
\begin{itemize}
\item{The water will boil and the steam will hurt your hand, so use a long straw and keep your hand out of the way of the steam. Use tongs to help you hold the straw if necessary.} 
\end{itemize}

\subsubsection*{Preparation Procedure}
\begin{enumerate}
\item{Collect all materials.} 
\end{enumerate}

\subsubsection*{Activity Procedure}
\begin{enumerate}
\item{Light the heat source.} 
\item{Fill the small pot half-way with water and place it over the heat.} 
\item{Place one end of the straw or tubing just below the surface of the water.} 
\item{Place your thumb over the other end of the straw and remove it from the water so that a single drop remains in the bottom of the straw.} 
\item{Turn the straw over and remove your thumb so that the water drop moves down to the center of the straw.} 
\item{Place one end of the straw back in the water, leaving the other end open.} 
\item{Observe the motion of the water drop in the straw as the water in the pot is heated.} 
\item{Hold your thumb over the open end of the straw.} 
\item{Remove the straw from the water and insert it into cold water.} 
\item{Remove your thumb and observe again the motion of the water drop in the straw.} 
\end{enumerate}

\subsubsection*{Results and Conclusion}
As the gas in the straw is heated, the water drop rises. This is due to the expansion of the gas below the drop, as the gas can only expand up. It will be seen that as the gas in the straw is cooled, the drop falls. This is due to the contraction of the gas below the drop, as the gas decreases in volume when its temperature decreases.  

\subsubsection*{Clean Up Procedure}
\begin{enumerate}
\item{Turn off the heat source and return all materials to their proper places.} 
\end{enumerate}

\subsubsection*{Discussion Questions}
\begin{enumerate}
\item{Describe the behavior of the drop when the straw is heated.} 
\item{Describe the behavior of the drop when the straw is cooled.} 
\item{What causes the drop to move up the straw?}
\item{What causes the drop to move down the straw?}
\item{What is expanding in this experiment? What is expanding the fastest?}
\end{enumerate}

\subsubsection*{Notes}
The behavior of gases is actually more complicated; gases will expand as much as the pressure will allow.

\subsection{Thermal Expansion of Solids - Breaking Glass}

\subsubsection*{Learning Objectives}
\begin{itemize}
\item{To observe the effects of thermal expansion of solids}
\item{To break glass containers evenly}
\end{itemize}

\subsubsection*{Background Information}
Solids expand when heated, though not always noticeably.  However, it is enough that if an object is heated on one side but not on another, the side that is heated expands and the other side does not.  If the object is rigid, like glass, it will break instead of bending.  In this way, glass containers like soda bottles and beakers can be broken evenly.

\subsubsection*{Materials}
Soda bottle or other open glass container (bottles with uniform sides are best), water, heat source

\subsubsection*{Hazards and Safety}
\begin{itemize}
\item{The glass will break and may leave sharp edges.  It is important that the glass is properly disposed of.}
\item{Be sure that the bottle is not covered.  If you cover the bottle it could explode rather than just break evenly.}
\end{itemize}

\subsubsection*{Preparation Procedure}
\begin{enumerate}
\item{Collect all materials.}
\end{enumerate}

\subsubsection*{Activity Procedure}
\begin{enumerate}
\item{Place some water in the soda bottle so that it is about half full.  It is best if the water level is at a point where the bottle's side is uniform.}
\item{Place the bottle over a heat source and wait.}
\item{If the bottle does not break before the water boils, remove the bottle from the heat until the water stops boiling, then place it in a container of cold water.}
\end{enumerate}

\subsubsection*{Results and Conclusion}
The glass bottle will break evenly at the level of the water inside.  This is because the water is gaining more heat than the air above it, so the glass bottle touching the water gains more heat than the glass touching the air.  Therefore the glass below the water level expands more than the glass above the water level, so the glass breaks evenly.

\subsubsection*{Clean Up Procedure}
\begin{enumerate}
\item{Dispose of broken glass.}
\item{Return all materials to their proper places.}
\end{enumerate}

\subsubsection*{Discussion Questions}
\begin{enumerate}
\item{Why does the bottle break at the level of the water?}
\item{Which part of the glass bottle is expanding most?}
\end{enumerate}

\subsubsection*{Notes}
Glass is one of the few materials which does this.  Metal bends when heated, hence the bimetallic strip.  However, a similar technique is used in many places to break large stones, especially in villages where other equipment is not available.


\subsection{Thermal Expansion of Liquids}

\subsubsection*{Learning Objectives}
\begin{itemize}
\item{To observe the expansion of liquids when heated} 
\item{To observe that different liquids expand at different rates} 
\end{itemize}

\subsubsection*{Background Information}
All substances expand when heated. While solids expand with specific dimensions, liquids and gases expand in volume only.  

\subsubsection*{Materials}
500 mL plastic water bottle with cap, nail, source of heat, small cooking pot, water, food colouring (optional), clear plastic straw or tubing, super glue, various liquids (cooking oil, kerosene, milk, etc.  )

\subsubsection*{Hazards and Safety}
\begin{itemize}
\item{The water in the small pot will boil after a short time, so be careful not to touch it.} 
\end{itemize}

\subsubsection*{Preparation Procedure}
\begin{enumerate}
\item{Collect all materials.} 
\item{Heat the nail.} 
\item{Use the hot nail to put a hole through the center of the bottle cap.} 
\item{Insert the straw through the hole half way so that half of the straw is below the cap and half is above.} 
\item{Use the hot nail and super glue to seal the hole so that no air can pass through the bottle cap. Be careful not to seal the straw itself.} 
\end{enumerate}

\subsubsection*{Activity Procedure}
\begin{enumerate}
\item{Fill the bottle three quarters with water and add a pinch of food colouring.} 
\item{Close the cap on the bottle so that the bottom of the straw is in the water.} 
\item{Fill the small pot half way with water.} 
\item{Place the pot over the heat source and place the water bottle in the pot (water bath).} 
\item{Allow the water to heat and observe the water level in the straw.} 
\item{Remove the water bottle from the pot and place it in cold water.} 
\item{Observe again the water level in the straw.} 
\item{Repeat all of these steps for various other liquids.} 
\item{If possible, prepare one water bottle for each liquid and observe their levels at the same time in the water bath.} 
\end{enumerate}

\subsubsection*{Results and Conclusion}
Liquids expand when heated and contract when cooled. This is because their atoms are moving with more energy and so must increase in volume. Because the water is sealed in the bottle and is expanding, it can only move up the straw.  

\subsubsection*{Clean Up Procedure}
\begin{enumerate}
\item{Turn off the heat source.} 
\item{Empty all the water and return all materials to their proper places.} 
\end{enumerate}

\subsubsection*{Discussion Questions}
\begin{enumerate}
\item{What happens to the water level in the straw when the water is heated?}
\item{What happens to the water level in the straw when the water is cooled?}
\item{What causes the water level to rise and fall?}
\item{Which liquid expands the fastest?}
\item{Which liquid expands the slowest?}
\end{enumerate}

\subsubsection*{Notes}
Liquids expand noticeably when heated so it is simple to see the effect. By using a capillary tube, we can see the effect much more easily as the liquid level will rise in the tube only. By sealing the cap on the bottle, we allow the liquid to expand in only one direction: up through the capillary tube.  

\subsection{Thermal Switch}

\subsubsection*{Learning Objectives}
\begin{itemize}
\item{To observe the thermal expansion and contraction of solids} 
\item{To explain the mode of action of a thermal switch} 
\item{To explain the effect of heating on solids} 
\end{itemize}

\subsubsection*{Background Information}
All substances expand when heated. Solids expand only a small amount, but metals expand noticeably.  

\subsubsection*{Materials}
flat piece of wood, 2 thick sticks about 5 cm tall, several small nails, short piece of thick wire (about 8 cm), connecting wires, 2 dry cells, bulb or galvanometer, candle, matches

\subsubsection*{Hazards and Safety}
\begin{itemize}
\item{The thick wire will become very hot in the candle, so be sure not to touch it.} 
\end{itemize}

\subsubsection*{Preparation Procedure}
\begin{enumerate}
\item{Collect all materials.} 
\item{Nail the two sticks upright on the piece of wood, about 6 cm apart.} 
\item{Fix one nail into the side of one stick near the top so that it sticks out horizonatally towards the other stick.} 
\item{Bend the end (about half a cm) of the wire at a right angle.} 
\item{Place the wire on the top of the other stick so that the bent end just touches the nail fixed to the other stick.} 
\item{Move the wire back slightly so that there is a gap of about 1 mm between the bent end of the wire and the nail.} 
\item{Fix the wire where it is by putting a nail in the top of the stick to hold the wire in place.} 
\item{Attach one connecting wire to the back of the thick wire and the other connecting wire to the nail fixed to the top of the other stick.} 
\item{Attach the connecting wires to the bulb and batteries in series.} 
\item{You should now have a circuit in series which is disconnected by the small gap between the thick wire and the nail.} 
\end{enumerate}

\subsubsection*{Activity Procedure}
\begin{enumerate}
\item{Set up the switch circuit on the table.} 
\item{Check to make sure that the batteries and bulb are working and that there is a small gap between the thick wire and the nail.} 
\item{Light the candle.} 
\item{Hold the candle under the thick wire until the wire expands to touch the nail. The bulb should light.} 
\item{Take the candle away to let the wire cool. The bulb should turn off.} 
\item{Repeat these steps until it is clear what is causing the bulb to turn on and off.} 
\end{enumerate}

\subsubsection*{Results and Conclusion}
The wire's length increases when it is in the candle flame and decreases when the candle is removed and the wire is allowed to cool. This is because the wire (a solid) expands when it is heated. This application of thermal expansion is used in thermostats.  

\subsubsection*{Clean Up Procedure}
\begin{enumerate}
\item{Disconnect all wires and return all materials to their proper places.} 
\end{enumerate}

\subsubsection*{Discussion Questions}
\begin{enumerate}
\item{What happens to the wire when it is heated? What happens when it is cooled?}
\item{What causes the circuit to be completed and the bulb to turn on?}
\item{What causes the bulb to turn off?}
\end{enumerate}

\subsubsection*{Notes}
Solids typically expand very little so that it is difficult to see with the naked eye. Metals expand enough that they can be used to demonstrate thermal expansion. The wire in this activity expands in the candle flame to complete the circuit, thereby lighting the bulb. When the source of heat is removed, the metal cools quickly and contracts, breaking the circuit and turning off the bulb.
