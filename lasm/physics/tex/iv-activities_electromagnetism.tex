\section{Electromagnetism}

\subsection{Simple Motor}

\subsubsection*{Learning Objectives}
\begin{itemize}
\item{To use Fleming's left-hand rule to determine the motion of the coil in the circuit} 
\item{To explain the principle of magnetic induction, which explain the motion of the rotating coil} 
\end{itemize}

\subsubsection*{Background Information}
The magnitude of the force acting on a current loop inside a magnetic field depends on the stength of the magnetic field, the amount of current flowing through the wire and the direction of the current and magnetic field.  

\subsubsection*{Materials}
Insulated copper wire, cellotape, clothes pegs, magnet, dry cells, connecting wires, two pieces of magnetic materials, switch

\subsubsection*{Preparation Procedure}
\begin{enumerate}
\item{Construct a ring of copper wire with lots of loops by wrapping the wire around any circular object with the desired diameter. Tape the coils together using cellotape so that the circular object can be removed and the ring does not fall apart. Make sure that two ends of the copper wire stick out from two opposite sides of the ring.} 
\item{ Take one end of the wire and scrape off all of the insulation coating using a razor. Take the other side and ONLY scrape off the insulation coating from half the circumfrence of the wire.} 
\item{Attach the two magnetic materials to two opposite sides of the spearker magnet.} 
\item{Attach the clothes pegs to the magnetic materials and suspend the coil above the magnet by placing it between the two wire loops of the pegs.} 
\item{Join the two dry cell together}
\end{enumerate}

\subsubsection*{Activity Procedure}
\begin{enumerate}
\item{Connect the coils of the pegs to the terminals of the battery through the switch using the connecting wires.} 
\item{Close the switch to observe the effects}
\end{enumerate}

\subsubsection*{Results and Conclusion}
When the switch is on, the magnetic field applies a force to the curent carying wire following Fleming's left-hand rule and causes the loop to spin. If the current is increased the coil spins faster showing the force is proportional to the current. If the current is reversed the coil will rotate in the other direction, this is apparent from Flemmin'g left hand rule. If there is a stronger magnet the coil spins faster which shows the force is proportional to the magnetic field strength. If the insulation is left on the current from the battery can not flow into the wire so there will be no spinning. If all of the insulation is scratched off from both sides then the loop will not spin but will instead reach an equillibrium position where the force acting on the top and bottom of the loop are balanced.  

\subsubsection*{Clean Up Procedure}
\begin{enumerate}
\item{Disconnect the components and keep them in their proper places}
\end{enumerate}

\subsubsection*{Discussion Questions}
\begin{enumerate}
\item{Explain the motion of the coil of current in the magnetic field. What happens when the current in wire is increased? What happens when the current is reversed in the coil? What happens when if we use a stronger magnet? What would happen if we did not scrape off the insulation coating? What would happen if we scraped off all of the insulation coating on both sides?}
\end{enumerate}

\subsubsection*{Notes}
During the preparation of a commutator, on one end scratch around the whole wire, but on the other end only half the circumfrence should be scratched
Magnet must be pretty strong to have an obvious effect. Make the coil with as many turns as possible.  

\subsection{Construction of Galvanometer}

\subsubsection*{Learning Objectives}
\begin{itemize}
\item{To construct a galvanometer}
\item{To explain the uses of a galvanometer}
\item{To explain the mode of action of a galvanometer}
\end{itemize}

\subsubsection*{Background Information}
A galvanometer uses the principle of electromagnetism to detect the presence of electric current. Whenever there is a flow of current in a wire conductor, magnetic fields are created around it. The magnetic fields produced are perpendicular to the direction of current and concentric around the current itself. For this reason, a coil produces a strong magnetic field through its center, and a magnet suspended in that coil will pivot to show the direction of the magnetic field. A suspended magnet, when not placed in another field, will always face in the direction of the earth's magnetic field.  

\subsubsection*{Materials}
Magnet; sewing needle, pin or the metal part of a syringe needle; coated copper wire or speaker wire; dry cells; water; empty water bottle; small piece paper; knife, connecting wires

\subsubsection*{Preparation Procedure}
\begin{enumerate}
\item{Cut the empty water bottle so that the bottom 3 cm act as a shallow dish.} 
\end{enumerate}

\subsubsection*{Activity Procedure}
\begin{enumerate}
\item{Instruct students to rub the pin/needle on the magnet several times in one direction without scratching it. This is magnetization by stroking and may take a minute depending on the strength of the magnet.} 
\item{Have students coil the wire around the water bottle dish about 20 times and secure it with cellotape so that it is secure.} 
\item{Use a knife or razor blade to scrape at least 2 cm of the insulating coating off of each end of the copper wire.} 
\item{Have students pour water into the dish so that it is about half full.} 
\item{Stitch the pin into a small piece of paper so that the pin is secure against one side of the paper.} 
\item{Place the pin and paper gently onto the surface of the water in the dish so that it floats. The pin will rotate on the water to point North and South because it is a magnet and is following the earth's magnetic field.} 
\item{Use connecting wires to connect the dry cells to both ends of the coiled wire. Observe the reaction of the needle.} 
\end{enumerate}

\subsubsection*{Results and Conclusion}
The needle/pin will rest pointing in the N-S direction. After connecting the dry cells, the needle/pin deflects because the magnetic field produced by the coil is stronger than the earth's field. The galvanometer detects whenever there is the flow of current in the wire.  

\subsubsection*{Clean Up Procedure}
\begin{enumerate}
\item{Pick up magnetized pin/needle, keep it in a safe place. Pour the water from the cap.} 
\item{Return all materials to their proper places}
\end{enumerate}

\subsubsection*{Discussion Questions}
\begin{enumerate}
\item{What causes the needle/pin to deflect?}
\item{Why does the pin face one direction when there is no current in the coil?}
\item{Use the Left Hand Rule to predict explain which end of the pin/needle is North and which end is South.} 
\end{enumerate}

\subsubsection*{Notes}
The stitched magnetized pin should rest parallel to the coiled wire so that it can be easy to observe the deflection. The pin will turn until its direction is perpendicular to the turns in the coil. Also, the paper on which the pin is resting will eventually sink, so you will need to replace it.  

\subsection{Force on a Current-Carrying Wire in a Magnetic Field}

\subsubsection*{Learning Objectives}
\begin{itemize}
\item{To observe the deflection of a current-carrying wire in a magnetic field} 
\item{To use the Left Hand Rule to predict the direction of deflection or force on a current-carrying wire in a magnetic field} 
\item{Students will be able to explain the relationship between motion, electric current and a magnetic field}
\end{itemize}

\subsubsection*{Background Information}
Electromagnetism is the relationship between three quanitities: electric current, magnetic fields and motion. When two of these quantities are present and perpendicular to each other, the third quantity is created. When electric current is running perpendicular to a magnetic field, a force is produced which pushes the wire in a third perpendicular direction to the current and field. The Left Hand Rule can be used to predict the direction of force and therefore the direction that the wire will move.  

\subsubsection*{Materials}
speaker magnet from a broken radio or speaker, thin copper wire about 20 cm long, books or other objects to use as supports, two stones, two clothes pegs, knife, connecting wire, two D-cell batteries, white paper, pen.  

\subsubsection*{Preparation Procedure}
\begin{enumerate}
\item{Rip off a small piece of paper the size of the magnet or smaller.} 
\item{Use a straight-edge to draw a grid on the paper like graph paper.} 
\item{Place the speaker magnet flat on the table so that one of the poles is facing up.} 
\item{Stack books or other solid objects to either side of the speaker magnet to the same height of the magnet.} 
\item{Scrape the ends (about 2 cm) of the copper wire so that the conductor is showing.} 
\item{Stretch the copper wire across the magnet so that it is resting on top of the books on either side but not quite touching the magnet.} 
\item{Secure the copper wire on either end with clothes pegs.} 
\item{Place stones or other heavy object on the clothes pegs so that the wire is pulled tight and cannot move easily.} 
\item{Attach connecting wires to each end of the copper wire.} 
\item{Place the paper with the grid on top of the magnet just below the copper wire.} 
\end{enumerate}

\subsubsection*{Activity Procedure}
\begin{enumerate}
\item{Position yourself directly over the copper wire and magnet so that the wire's position in relation to the grid is clearly visible.} 
\item{Connect the connecting wires to the battery terminals to start a current in the wire.} 
\item{Observe any movement by the wire.} 
\end{enumerate}

\subsubsection*{Results and Conclusion}
It can be seen that the wire is deflected to one side the current is allowed to pass through it. When the current is disconnected, the wire returns to its original position. This deflection is the result of the force on the wire which is produced by a combination of the electric current and the magnetic field.  
If the direction of current is switched, the direction of the wire's deflection is also switched.  

\subsubsection*{Clean Up Procedure}
\begin{enumerate}
\item{Disconnect all wires and return all materials to their proper places.} 
\end{enumerate}

\subsubsection*{Discussion Questions}
\begin{enumerate}
\item{Explain the the sources and directions of both the magnetic field and the electric current that are present.} 
\item{In what direction is the wire deflected when the circuit is completed?}
\item{Use the Left Hand Rule to find the direction of the magnetic field assuming that the direction of force and the direction of electric current are known.} 
\end{enumerate}

\subsubsection*{Notes}
This activity demonstrates the same concept which powers a motor, namely that an electric current running perpendicular to a magnetic field feels a perpendicular force. The deflection of the wire, while small, is observable and obviously in a direction perpendicular to both the current and magnetic field.  
If the direction of the field is known, students can use the LHR to predict the direction of motion. If the direction of the field is not knows, students can find it by observing the direction of force and then using the LHR to find the magnetic field.  
It will be seen that the wire vibrates when it first deflects. This is simply because of the sudden motion to one side, not because the current is producing a wave. However, if alternating current is used, it will be seen that the repeated back-and-forth deflection of the wire does produce a standing wave. If an AC source is available, this is a good demonstration of standing waves.  

\subsection{Water Energy}

\subsubsection*{Learning Objectives}
\begin{itemize}
\item{To construct a simple water wheel and generator} 
\item{To explain and show the conversion of mechanical energy to electrical energy by use of a generator coil}
\item{Students will be able to show the direction of current produced by a generator}
\end{itemize}

\subsubsection*{Background Information}
Electromagnetism is the relationship between three quantities: magnetic force, electric current and motion.  When two of these quantities are present and perpendicular to each other, the third is produced.  In a motor, we use electric current in a magnetic field to produce motion: the rotation of the motor coil.  In a generator, we use motion in a magnetic field to produce electtric current.  The structure of a motor and generator are therefore almost the same.  This is the mechanism behind gas generators, wind turbines, tidal generators, and many others.

\subsubsection*{Materials}
Plastic water bottle (any size), small motor from a car stereo or other device, super glue, large nail or soldering iron, heat source, 9 water bottle caps, 8 syringe needle caps, scissors, water and pitcher, connecting wires, galvanometer.  

\subsubsection*{Hazards and Safety}
\begin{itemize}
\item{Be careful when melting the plastic pieces together.  Melted plastic can quickly cause second degree burns.  If it is easier, you can use super glue to connect the plastic pieces together.}
\end{itemize}

\subsubsection*{Preparation Procedure}
\begin{enumerate}
\item{Making the water wheel: Using a hot nail or soldering iron, melt the open end of a syringe needle cap to the side of a water bottle cap to create a sort of spoon.} 
\item{Repeat this step 7 more times to create a total of 8 identical pieces: these are the spokes and cups of the water wheel.} 
\item{Cut the top off of a water bottle just below the lip which holds the cap making sure that the edges are even.} 
\item{Melt a plastic cap over the cut end of the water bottle top so that the piece is closed on one end by the plastic bottle cap and open on the other end where a bottle cap would normally be screwed on.} 
\item{Use the hot nail or soldering iron to melt 8 holes evenly around the side of the bottle cap used in step 4. Each hole should be just wide enough to admit the thin end of a syringe needle cap; you can insert a needle cap each time you melt a hole in the bottle cap, removing the needle cap before the plastic cools.} 
\item{Insert the 8 spokes from steps 1 and 2 into the holes so that they create an 8-spoked wheel with all of the cups facing in one direction (either clockwise or anticlockwise) and at equal distances from the center.} 
\item{Melt the plastic around each spoke in the bottle cap so that they are all secure and will not come out of the center cap.} 
\item{Making the generator: Use a pin to make a small hole in the center of a plastic water bottle cap. Glue the top of the cap to the wheel of a motor so that the center of the cap and the center of the motor wheel are perfectly aligned. Note: if you have already made the wind generator, you do not need to complete these steps.} 
\item{Attach connecting wires to each terminal on the motor.} 
\item{Screw the water wheel into the motor as you would screw a bottle cap onto a bottle so that the water wheel is free to rotate about the motor's axis of rotation.} 
\end{enumerate}

\subsubsection*{Activity Procedure}
\begin{enumerate}
\item{Make sure that the water wheel is free to rotate on the motor.} 
\item{Make sure that the galvanometer is working.} 
\item{Connect the wires on the generator to the terminals on the galvonometer.} 
\item{Pour water from a pitcher or spout and place the water wheel under the water so that is turns vertically.} 
\item{Observe any deflection in the galvanometer.} 
\end{enumerate}

\subsubsection*{Results and Conclusion}
Students will see that while the water wheel is turning, a current is created in the wire around the galvanometer. They should understand that the mechanical energy of the rotating wheel (or of the water falling) is being converted to electrical energy which can be read by the galvanometer. The conversion must be taking place in the motor, which they understand involves a magnetic field and a rotating coil.  

\subsubsection*{Clean Up Procedure}
\begin{enumerate}
\item{Disconnect the galvanometer and return it to its proper place.} 
\item{Unscrew the water wheel from the generator and return both to their proper places.} 
\end{enumerate}

\subsubsection*{Discussion Questions}
\begin{enumerate}
\item{What causes the water wheel to turn?}
\item{What is happening inside the motor to convert mechanical energy (rotation) into electrical energy (electric current)?}
\item{If the water wheel is turning but the galvanometer does not deflect, what are some possible causes?}
\end{enumerate}

\subsubsection*{Notes}
We convert electrical energy to mechanical energy using a motor, and we convert mechanical energy to electrical energy using a generator. In fact they are the same thing, except that the energy being put into the system is different in each case.  
For a water generator, the falling water (gravitational mechanical energy) causes the water wheel to rotate (also mechanical energy). The rotating wheel causes the wire coil in the motor to rotate. A rotating coil in a magnetic field produces an electric current in the coil, which can then be detected by the galvanometer.  

\subsection{Wind Energy}

\subsubsection*{Learning Objectives}
\begin{itemize}
\item{To explain the conversion of mechanical energy to electrical energy using a generator coil} 
\item{To construct a simple wind turbine and generator} 
\end{itemize}

\subsubsection*{Background Information}
The mechanism for the wind-powered generator is the same as that of a water-powered generator.  The force of the wind on a turbine causes the turbine to rotate.  This, in turn, causes a coil to rotate in a magnetic field, producing electric current.  These types of generators are used all over the world to produce electricity for use.

\subsubsection*{Materials}
1' x 1' piece of flexible plastic sheet, pin, scissors, super glue, plastic water bottle (any size) with cap, small motor from a car stereo or other device, connecting wires, galvanometer

\subsubsection*{Preparation Procedure}
\begin{enumerate}
\item{Making the propeller: Make 5 small holes in the plastic sheet: one in the middle and one at each corner.} 
\item{Cut curved lines from each side near the right-hand corner inward towards the center hole (see diagram).} 
\item{Fold each corner in towards the center so that the five holes are aligned and glue them in place.} 
\item{Remove the cap from the water bottle and cut the top of the bottle off just below the lip where the cap sits.} 
\item{Glue the water bottle top to the center of the back of the propeller so that the side of the bottle which holds the cap is facing backwards and the propeller is facing forwards.} 
\item{Making the generator: Make a small hole in the center of the bottle cap with the pin.} 
\item{Glue the top of the bottle cap to the motor wheel; use the hole to line up the center of the cap with the center of the wheel. When the motor turns, the cap should turn evenly.} 
\item{Screw the propeller into the generator as you would close the cap on a bottle. Now the propeller should be able to turn freely on the motor.} 
\item{Connect the terminals of the motor to the terminals of your galvanometer.} 
\end{enumerate}

\subsubsection*{Activity Procedure}
\begin{enumerate}
\item{Make sure that the wind generator can turn freely and that it is connected to the galvanometer.} 
\item{Hold the propeller upright into the wind so that it spins; the galvanometer will deflect to show that a current is being created in the wire.} 
\end{enumerate}

\subsubsection*{Results and Conclusion}
When the generator and galvonometer are working and connected properly, it will be seen that as the propeller turns, a current is created in the wire causing the galvanomter to deflect. If the propeller turns just a little, the galvanometer will show a small deflection; if the propeller turns quickly, the galvanometer will show a large deflection.  
Students will observe that mechanical energy (wind) is being converted into electrical energy (electric current) using a generator. The generator uses a magnet and motion to produce an electric current, so students will see that, as with a motor, a Magnetic Field, Electric Field and Motion are related.  

\subsubsection*{Clean Up Procedure}
\begin{enumerate}
\item{Disconnect the galvonomer and connecting wires and return them to their places.} 
\item{Unscrew the propeller from the generator and return them to their places.} 
\end{enumerate}

\subsubsection*{Discussion Questions}
\begin{enumerate}
\item{Why does wind, which moves in one direction, cause the propeller to spin?}
\item{What type of energy is being used, and what type of energy is being created in this system?}
\item{What is happening in the motor to convert between the two types of energy?}
\item{If the propeller is turning but the galvanometer does not show the presence of a current, what are some possible causes?}
\end{enumerate}

\subsubsection*{Notes}
If you have constructed the water-powered generator, this activity will be very simple.  The motor used in the other generator can be used again here; simply unscrew the water wheel and replace it with the wind turbine.
