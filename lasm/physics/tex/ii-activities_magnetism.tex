\section{Magnetism}

\subsection{Magnetic and Non-magnetic Materials}

\subsubsection*{Learning Objectives}
\begin{itemize}
\item{To identify magnetic and non-magnetic materials} 
\item{To observe the effect of a magnet on magnetic and non-magnetic materials} 
\end{itemize}

\subsubsection*{Background Information}
The only naturally occurring magnet is Lodestone, which is a kind of mineral. However, we tend to use iron to make magnets as it is easily magnetized and very strong. Magnets can have a very strong attractive or repulsive force, but the force only acts on some materials.

\subsubsection*{Materials}
magnet*, various objects in the classroom or home like iron nails, plasitc, wood, cloth, metal, etc.  

\subsubsection*{Hazards and Safety}
\begin{itemize}
\item{Be gentle with the magnet as they are sometimes made of soft iron and can break easily.  Speaker magnets tend to be stronger}
\end{itemize}

\subsubsection*{Preparation Procedure}
\begin{enumerate}
\item{Collect a magnet and various magnetic and non-magnetic materials.} 
\end{enumerate}

\subsubsection*{Activity Procedure}
\begin{enumerate}
\item{Bring the magnet close to an object.} 
\item{Observe if the object moves or not, or if it is difficult to remove the magnet.} 
\item{Repeat this for any objects you or the students can find in the classroom.} 
\item{Keep a list of what is attracted by the magnet and what is not.} 
\end{enumerate}

\subsubsection*{Results and Conclusion}
It will be seen that, in general, metals are attracted to magnets while all other materials are not. The difference between magnetic materials and non-magnetic materials is similar to the difference between conductors and insulators.  

\subsubsection*{Clean Up Procedure}
\begin{enumerate}
\item{Return all materials to their proper places.} 
\end{enumerate}

\subsubsection*{Discussion Questions}
\begin{enumerate}
\item{What objects were attracted to the magnet?}
\item{In general, what materials are attracted to magnets?}
\item{What objects were not attracted to the magnet?}
\item{In general, what materials are not attracted to magnets?}
\end{enumerate}

\subsubsection*{Notes}
Almost all metals (which are conductors) are magnetic materials with the exception of copper and aluminium. The strongest magnet material, and therefore the strongest magnet, is iron.  

\subsection{Magnetic Fields}

\subsubsection*{Learning Objectives}
\begin{itemize}
\item{To map magnetic field lines of a given magnet using iron filings}
\item{To explain the shape and direction of magnetic fields on a magnet or between magnets}
\end{itemize}

\subsubsection*{Background Information}
Magnetic fields are invisible lines of force that run between the poles of a magnet (from North to South) or between the poles of multiple magnets.  We cannot see these lines, but we can feel them when we bring magnets close to each other.

\subsubsection*{Materials}
bundle of Steel wool, magnets

\subsubsection*{Hazards and Safety}
\begin{itemize}
\item{Wash your hands after using the steel wool}
\item{Do not put fingers in your eyes after rubing the steel wool}
\item{Do not eat anything before washing your hands}
\end{itemize}

\subsubsection*{Preparation Procedure}
\begin{enumerate}
\item{Collect all necessary materials.}
\item{Place a sheet of paper on the table}
\item{Take the steel wool and rub it over the sheet of paper.  Small pieces (filings) of the steel wool will fall to the paper.}
\end{enumerate}

\subsubsection*{Activity Procedure}
\begin{enumerate}
\item{Place a magnet on the table.}
\item{Place a piece of paper over the magnet so that the paper is flat}
\item{Slowly and gently drop the iron filings on the paper.  Spread them evenly.}
\item{Observe the positions of the filings and the shape they create.}
\item{If another magnet is present, place it near the other magnet under the paper.}
\item{Drop more filings on the paper to observe the shape of the field between the two magnets.}
\item{Repeat this process for the two magnets in various positions: repeling, attracting, etc.}
\end{enumerate}

\subsubsection*{Results and Conclusion}
The iron filings are magnetic and light-weight, so they will move to follow the magnetic lines of force on a magnet.  The filings will clearly show the curved lines around the magnet from one pole to the other.  If the magnet is strong enough, the filings will show the three-dimensional field.
Between magnets, the filings will show a strong concentration of force between opposite poles and a neutral point between like poles.

\subsubsection*{Clean Up Procedure}
\begin{enumerate}
\item{Store iron filings inside a bottle after using them}
\item{Return the magnets to a safe storage place.}
\end{enumerate}

\subsubsection*{Discussion Questions}
\begin{enumerate}
\item{Apart from steel wool, where can you obtain iron filings?}
\item{Describe the shape of the magnetic field lines}
\item{Where is the magnetic force strongest?  Where is it weakest?}
\end{enumerate}

\subsubsection*{Notes}
You can obtain a steel wool from any nearby shop/kiosk. Also you can visit any garage in town where there is a welder or metal saw and collect iron dust and use them as iron filings.
