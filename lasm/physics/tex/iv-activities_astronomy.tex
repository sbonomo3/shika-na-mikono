\section{Astronomy}

\subsection{Solar System Mobile}

\subsubsection*{Learning Objectives}
\begin{itemize}
\item{To understand the arrangement of celestial bodies in our solar system}
\end{itemize}

\subsubsection*{Background Information}
A solar system is simply the group of bodies which surround a star.  In our solar system we have 8 planets, comets, asteroids and asteroid belts, dwarf planets, and many moons.  All of these move in circles or ellipses around our star, the Sun.  There are too many objects in our solar system to count, but we have names for the largest of these: the planets, their moons and some comets.  Outward from the sun, the planets are Mercury, Venus, Earth, Mars, Jupiter, Saturn, Uranus and Neptune.  Pluto was once a planet but has since become a dwarf planet.  Each planet has a unique size, environment and history, though Earth is the only planet that we know of which supports life.

\subsubsection*{Materials} flour, water, balloons, mixing bowl, newspaper or old papers, string, sticks

\subsubsection*{Preparation Procedure}
\begin{enumerate}
\item{Blow up the balloons, one for each of the 8 planets and sun.}
\item{Make the paper mache mixture with flour and water; you want a wateryglue texture.}
\item{Wet the paper in this mixture and apply artistically to the balloons until you have a layer a couple papers-thick on each balloon.}
\item{Leave the balloon slightly exposed at the bottom.}
\item{When the papers are dried, pop the balloons inside.}
\item{Use paint or marker pens to make the paper balls look like planets.}
\item{Attach string to each of the planets.}
\item{Hang the planets on sticks with the sun in the middle and each of the planets at different points moving away from the sun.}
\end{enumerate}

\subsubsection*{Activity Procedure}
\begin{enumerate}
\item{Hang the mobile from a beam or ceiling so that the planets are suspended around the sun at different distances}
\item{Identify each of the planets and discuss them/}
\end{enumerate}

\subsubsection*{Cleanup Procedure}
\begin{enumerate}
\item{Dispose of any remaining liquid before it dries.}
\item{Cap any markers or paint and return all materials to their proper places.}
\end{enumerate}

\subsubsection*{Discussion Questions}
\begin{enumerate}
\item{Name the eight planets.}
\item{Which planet is the largest?  Is it solid or gas?}
\item{Which planet is the smallest?  Is it solid or gas?}
\item{What do the first four planets have in common?}
\item{What do the last five planets have in common?}
\end{enumerate}

\subsubsection*{Notes}
Theory: This activity is helpful to explain to students what is actually
happening outside of the world.  This mobile is helpful to remember that there is
more to the solar system than just the earth.  Remember that the planets are all at different distance from the sun, but they are all in the same plane.  For this reason, you can hang the planets are about the same height from the sticks.

\subsection{Star Gazing}

\subsubsection*{Learning Objectives}
\begin{itemize}
\item{To identify objects in the night sky}
\item{To understand various structures and bodies in the galaxy and universe in relation to the earth}
\end{itemize}

\subsubsection*{Background Information}
Astronomy is one of the oldest sciences.  It has been used for thousands of years in navigation and has provided the proof or evidence for many laws and theories of nature such as gravitation and relativity.

\subsubsection*{Activity Procedure}
\begin{enumerate}
\item{Take the students out at night where there is little light from lamps and fires.}
\item{Look for constellations, stars, planets and satellites. Discuss the reason for having constellations and the motion of the sky over the course of a night and a year.}
\end{enumerate}

\subsubsection*{Results and Conclusion}
Especially in rural areas, the stars and other celestial bodies are very clear.  Depending on the time of year, different planets and constellations will be visible.  The most obvious constellations are Orion, Ursa Major and the Southern Cross.  The brightest star is Sirius.  If the sky is clear our galaxy, the Milky Way, is visible as a bright stripe across the sky.

\subsubsection*{Discussion Questions}
\begin{enumerate}
\item{What objects did you observe in the night sky?}
\item{What are the brightest objects?}
\item{Is it easier to see the stars when the moon is present, or when it is absent?}
\item{What constellation could you use to know which direction is North?}
\end{enumerate}

\subsubsection*{Notes}
Before going outside at night, check a star chart and planet charts to find out which objects should be visible that night.  This will make it easier to identify objects and help students.  If you have binoculars, they can be used to see planets easily or to see the difference between a single star system and a binary star system, or between a star and a distant galaxy.  If possible, tell students the stories behind the constellations and see if they can find their own.
