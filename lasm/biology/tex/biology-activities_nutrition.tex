\section{Nutrition}
Nutrition is the way in which organisms obtain materials they need to live. There are two types of nutrition, autotrophic nutrition and heterotrophic nutrition. Autotrophic nutrition is how plants get their food by taking raw materials and changing them into food.  This process is called photosynthesis. \\Heterotrophic nutrition is how animals get food and it must already be made, like plants or other animals. Food is anything that provides the body with a source of energy, material for growth and repair, or factors for good health. Nutrients are food substances which are needed for healthy growth. Food eaten by humans contains nutrients which include carbohydrates, fats, proteins, vitamins, and mineral elements. Nutrition is important because it allows us to move, grow, keep our bodies warm, repair damaged tissue and fight diseases.

The activities in this section often appear on the NECTA examination because their relevance to students' daily life.  Therefore, it is important that these nutrition activities are practised by the students, either in small groups or individually.

\subsection{Food Test for Lipids}
Lipids are an organic food substance made of carbon, oxygen, and hydrogen. Lipids occur in two forms: fats and oils. Oils are liquid at room temperature whereas fats are solid. Lipids provide the body with energy and create a layer of insulation to help keep the body warm. The main sources of lipids are milk, animal fats, groundnuts, coconuts, and avocado.

\subsubsection*{Learning Objectives}
\begin{itemize}
\item{To carry out a test for lipids in a given food sample.}
\end{itemize}

\subsubsection*{Materials}
Iodine solution*, water, empty plastic bottles, test tubes*, droppers*, and a cooking oil that is liquid at room temperature, e.g. sunflower oil

\subsubsection*{Hazards and Safety}
\begin{itemize}
\item{Iodine solution is harmful to swallow.}
\item{Iodine solution can stain clothing. Remove stains promptly with a solution of crushed vitamin C (ascorbic acid). Iodine will also migrate into wood stain, permanently discolouring tables - prevent spills.}
\end{itemize}

\subsubsection*{Preparation Procedure}
\begin{enumerate}
\item{Mix about 10 mL (one cap full) of cooking oil and about 100 ml of water in a plastic bottle.}
\item{Close the bottle and shake vigorously.}
\end{enumerate}

\subsubsection*{Activity Procedure}
\begin{enumerate}
\item{Pour 2 mL of the food sample solution into a test tube. You should shake the bottle of sample solution each time before pouring it to prevent the oil from separating.}
\item{Add 3 drops of iodine solution to the test tube.}
\item{Shake the test tube and let the mixture settle.}
\item{Record results.}
\end{enumerate}

\subsubsection*{Results and Conclusion}
You should see the formation of a red ring at the top of the sample solution. This indicates the presence of lipids.

\subsubsection*{Clean Up Procedure}
\begin{enumerate}
\item{Unused iodine solution should be stored in a labelled bottles for future use.}
\item{None of the waste from this experiment requires special disposal.}
\end{enumerate}

\subsubsection*{Discussion Questions}
\begin{enumerate}
\item{In which part of the digestive system is the identified food substance digested?}
\item{Name the enzyme responsible for its digestion.}
\item{When the identified food substance is digested, what is the end product?}
\end{enumerate}

\subsubsection*{Notes}
Many biology books call for a chemical called Sudan III to test for lipids. Sudan III is a bright red pigment molecule that is much more soluble in oil than in water. For this reason, Sudan III solution is usually prepared using ethanol to bring the Sudan III pigment into the solution. In mixtures of oil and water, the oil separates and moves to the top. When shaken with Sudan III, this oil absorbs the Sudan III, turns red, and produces a "red ring" at the top of the test tube. However, the ethanol used to make Sudan III causes the water and oil to form an emulsion. The oil is broken into very small particles and it takes a long time for this dispersion to break down and form an oil layer on the top. Hence testing with Sudan III takes a long time to show a clear result.
Iodine is another coloured molecule that is more soluble in oil than in water. When a mixture of oil and water is shaken with iodine solution, the iodine moves to the oil layer, colouring it orange or red. This also gives the result of a "red ring" at the top of the test tube. To prevent an emulsion forming - as happens with Sudan III - it is very important to make iodine solution from pharmacy tincture that is without ethanol. Another benefit of using iodine is that while Sudan III is always red, iodine is uniquely yellow in water and red in oil, make the difference between positive and negative results more clear. Because there is no ethanol in iodine solution, the result also comes much faster, usually within 10-20 seconds.
Note that if the oil and water mixture settles before you transfer it to the test tube, you will receive a large quantity of oil whereas if you do not you will not receive any. Shake the food sample solution before taking each sample.

\subsection{Food Test for Proteins}
Proteins are organic food substances consisting of carbon, hydrogen, oxygen, and nitrogen. Proteins create growth and repairs damaged tissue. The main source of protein are beans and nuts, meat, fish, milk, cheese, and eggs.

\subsubsection*{Learning Objectives}
\begin{itemize}
\item{To carry out test for proteins in a given food sample.}
\end{itemize}


\subsubsection*{Materials}
Copper sulphate solution*, sodium hydroxide solution*, water, food sample containing protein such as egg, beaker*, empty plastic bottles, plastic spoon, test tube*, and citric acid*

\subsubsection*{Hazards and Safety}
\begin{itemize}
\item{Copper sulphate solution is poisonous and should not be swallowed.}
\item{Use a plastic spoon for measuring caustic soda - the hydroxide will corrode a metal one.}
\item{Sodium hydroxide is corrosive - concentrated solutions can burn skin and wood. Even dilute solutions can blind if they get into eyes.}
\item{If sodium hydroxide solution spills, neutralize spills with citric acid solution or vinegar.}
\item{Close the container of sodium hydroxide solution after use to prevent reaction with atmospheric carbon dioxide.}
\end{itemize}

\subsubsection*{Preparation Procedure}
\begin{enumerate}
\item{Make a small hole at the tip of an egg.}
\item{Pour some of the egg white into a beaker.}
\item{Dilute the egg white with 150 mL water.}
\item{Stir until the solution is clear.}
\end{enumerate}

\subsubsection*{Activity Procedure}
\begin{enumerate}
\item{Put 2 mL of sample solution into a test tube.}
\item{Add 1 mL of sodium hydroxide solution to the test tube, then 1 mL of copper sulphate solution to the test tube.}
\item{Record results.}
\end{enumerate}

\subsubsection*{Results and Conclusion}
The colour of the food sample will change from a clear colour to a violet or purple colour. This indicates the presence of protein in the food sample.

\subsubsection*{Clean Up Procedure}
\begin{enumerate}
\item{Unused reagents should be stored in plastic bottles for further use. Do not store sodium hydroxide in glass bottles.}
\item{Dispose chemical waste in a pit latrine.}
\end{enumerate}

\subsubsection*{Discussion Questions}
\begin{enumerate}
\item{List down any three examples of food that contain the nutrient identified in this experiment.}
\item{What is the function of this nutrient in the human body?}
\item{What is the deficiency disease causes by a lack of this nutrient?}
\end{enumerate}

\subsubsection*{Notes}
Some textbooks may recommend using Millon's reagent to test for protein. This reagent contains mercury, which is extremely poisonous and should never be handled by students.

\subsection{Food Test for Starch}
Starch is a carbohydrate, specifically a polymer of glucose. Carbohydrates provide the body with energy. Starch is found in food like potatoes, cassava, maize, and wheat.

\subsubsection*{Learning Objectives}
\begin{itemize}
\item{To carry out test for starch in a given food sample.}
\end{itemize}

\subsubsection*{Materials}
Iodine solution*, water, empty plastic bottle, droppers*, heat source, test tubes*, and a food sample containing starch such as maize flour.

\subsubsection*{Hazards and Safety}
\begin{itemize}
\item{Iodine solution is harmful to swallow.}
\end{itemize}

\subsubsection*{Preparation Procedure}
\begin{enumerate}
\item{Prepare a food sample solution by either saving the water that remains from boiling pasta or potatoes, or by mixing 2 teaspoons of maize flour into about a litre of water, then heat up to dissolve and reduce the white appearance of the liquid, which may give students the answer without having to perform the experiment.}
\end{enumerate}

\subsubsection*{Activity Procedure}
\begin{enumerate}
\item{Place 2 mL of sample solution into a test tube.}
\item{Add 3 drops of iodine solution to the test tube and record what happens.}
\end{enumerate}

\subsubsection*{Results and Conclusion}
When iodine is added, the food sample will change to a blue-black colour. This indicates that the food sample contains starch.

\subsubsection*{Clean Up Procedure}
\begin{enumerate}
\item{Unused iodine solution should be stored in a labelled reagent bottle.}
\item{No special disposal is required for waste from this activity.}
\end{enumerate}

\subsubsection*{Discussion Questions}
\begin{enumerate}
\item{What have you persevered after adding iodine solution to the food sample?}
\item{List down three foods which contain the nutrient identified in the experiment.}
\item{What is the importance of this food nutrient to the human body?}
\end{enumerate}

\subsection{Food Test for Reducing Sugars}
 Reducing sugars are simple sugars with the ability to reduce copper (II) ions to copper (I). All monosaccharides (fructose, glucose, galactose) are reducing sugars as are some disaccharides, such as lactose and maltose. Simple sugars are all carbohydrates, and are used by the body as a source of energy.

\subsubsection*{Learning Objectives}
\begin{itemize}
\item{To carry out food tests for reducing sugar in a given food sample.}
\end{itemize}

\subsubsection*{Materials}
Benedict's solution*, cooking pot, kerosene stove or charcoal burner, plastic spoon, droppers*, empty plastic bottles, test tube*, test tube holders*, and food sample containing a reducing sugar like glucose or onions.

\subsubsection*{Preparation Procedure}
\begin{enumerate}
\item{Make a solution of a food sample containing a reducing sugar. This can be done by adding a spoonful of glucose to a litre of water or cutting an onion into quarters, grinding them in a mortar and pestle, and collecting and diluting the juice. Let it settle and decant the mixture.}
\end{enumerate}

\subsubsection*{Activity Procedure}
\begin{enumerate}
\item{Put 2 mL of the food sample solution into a test tube.}
\item{Add 1 mL of Benedict's solution to the test tube.}
\item{Hold the test tube upright in the water bath and heat the solution to boiling.}
\end{enumerate}

\subsubsection*{Clean Up Procedure}
\begin{enumerate}
\item{Unused Benedict's solution should be stored in a labelled plastic bottle for future use.}
\item{Dispose of chemical waste in a pit latrine.}
\end{enumerate}

\subsubsection*{Hazards and Safety}
\begin{itemize}
\item{Copper is harmful to swallow and in large quantities is harmful to the environment.}
\end{itemize}

\subsubsection*{Discussion Questions}
\begin{enumerate}
\item{What changes did you observe in the food sample during the experiment?}
\item{Name any two sources of the food nutrient identified in the experiment above.}
\item{What is the importance of the identified food nutrient in the human body?}
\end{enumerate}

\subsubsection*{Results and Conclusion}
The colour of the food sample will change to green, yellow, orange, and finally form a brick red precipitate. This indicates the presence of a reducing sugar.

\subsubsection*{Notes}
Benedict's solution contains aqueous copper (II) sulphate, sodium carbonate, and sodium citrate. The citrate ions in Benedict's solution prevent the formation of insoluble copper (II) carbonate. In the presence of a reducing sugar, however, the copper (II) ions are reduced to copper (I) ions which form a brick red precipitate of copper (I) carbonate.
Normally, sugar molecules form five or six member rings and have no reducing properties. In water, however, the rings of some sugar molecules can open to form a linear structure, often with an aldehyde group at one end. These aldehyde groups react with copper (II) to reduce copper (II) to copper (I). Sugars that do not have an aldehyde group in the linear structure are not able to reduce copper (II) ions and are thus called non-reducing sugars.

\subsection{Food Test for Non-Reducing Sugars}
Disaccharides are compound sugars formed when two monosaccharide molecules combine. Disaccharides are found in sugar cane (sucrose), malt (maltose), and milk (lactose). Some disaccharides are reducing sugars (lactose and maltose), while others are non-reducing sugars (sucrose).

\subsubsection*{Learning Objectives}
\begin{itemize}
\item{To carry out food test for non-reducing sugar in a given food sample.}
\end{itemize}

\subsubsection*{Materials}
Benedict's solution*, cooking pot, kerosene stove or charcoal burner, plastic spoon, droppers*, empty plastic bottles, test tube*, test tube holders*, citric acid solution*, sodium hydroxide solution*, food sample containing non-reducing sugar like table sugar or fresh sugar cane

\subsubsection*{Preparation Procedure}
\begin{enumerate}
\item{Make a solution of a food sample containing a non-reducing sugar.}
\end{enumerate}

\subsubsection*{Activity Procedure}
\begin{enumerate}
\item{Put 2 mL of the sample solution in a test tube.}
\item{Add 2 drops of citric acid solution to the food sample.}
\item{Heat the mixture to boiling in a hot water bath.}
\item{Remove the solution as soon as it boils and let the solution cool.}
\item{Add 2 drops of sodium hydroxide solution to the food sample.}
\item{Add 2 drops of Benedict's solution to the food sample.}
\item{Heat the mixture in the water bath again and record your observations.}
\end{enumerate}

\subsubsection*{Clean Up Procedure}
\begin{enumerate}
\item{Unused reagents should be stored in plastic bottles for further use. Do not store sodium hydroxide in glass bottles.}
\item{Dispose of chemical waste in a pit latrine.}
\end{enumerate}

\subsubsection*{Hazards and Safety}
\begin{itemize}
\item{Sodium hydroxide is corrosive - concentrated solutions can burn skin and wood and even dilute solutions can blind if they get into eyes.}
\item{Citric acid is irritating - keep out of eyes.}
\item{If sodium hydroxide solution spills, neutralize spills with citric acid solution or vinegar.}
\item{Close the container of sodium hydroxide solution after use to prevent reaction with atmospheric carbon dioxide.}
\end{itemize}

\subsubsection*{Discussion Questions}
\begin{enumerate}
\item{What have you observed during the experiment?}
\item{Name two examples that contain the identified food nutrient in the experiment above.}
\item{What is the importance of the identified food nutrient in the human body?}
\item{What is the purpose of adding acid to the sample and heating it?}
\item{What is the importance of adding sodium hydroxide to the sample?}
\end{enumerate}

\subsubsection*{Results and Conclusion}
The colour of the food sample will change from green to yellow and finally to a brick red precipitate. This indicates the presence of a non-reducing sugar.

\subsubsection*{Notes}
This experiment will also test positive for all reducing sugars. Therefore it is important to first perform the test for reducing sugars before considering this test. If the test for reducing sugars is positive, there is no reason to perform the test for non-reducing sugars - the conclusion will be invalid.
Non-reducing sugars are a misnomer, that is, their name is incorrect. This test does not test for any sugar that is not reducing. Rather, this is a test for any molecule made of multiple reducing sugars bound together, such as sucrose or starch. When these polysaccharides are heated in the presence of acid, they hydrolyse and release monosaccharides. The presence of these monosaccharides is then identified with Benedict's solution. The purpose of the sodium hydroxide is to neutralize the citric acid added for hydrolysis. If the citric acid is not hydrolysed, it will react with the sodium carbonate in Benedict's solution, possibly making the solution ineffective.
For information about Benedict's solution and reducing sugars, see the explanation with the previous experiment: Food Tests - Reducing Sugars.

\subsection{Investigating the Structures of a Leaf}
Photosynthesis is the process by which green plants make their own food using water, carbon dioxide, and energy from the sun.  Photosynthesis takes place in the leaves.  The green colour, which is caused by chlorophyll, absorbs the sunlight and uses that energy to convert ce\\{CO2} and ce\\{H2O}  into glucose. \\A leaf consists of a broad, flat part called the lamina which is joined to the rest of the plant by a leaf stock or petiole.  Running through the petiole are vascular bundles which then form the veins in the leaf.  These contain tubes which carry substances to and from the leaf.   Each vein contains large, thick walled xylem vessels for carrying water and smaller, thin walled phloem tubes for carrying away food which the leaf has made.    

\subsubsection*{Learning Objectives}
\begin{itemize}
\item{To describe the different structures in a leaf and their roles in photosynthesis.}
\end{itemize}

\subsubsection*{Materials}
Variety of leaves, razor blades, GV stain*, water drop microscope*, plastic slides*, plastic cover slips*, and water

\subsubsection*{Hazards and Safety}
\begin{itemize}
\item{Use caution when cutting with razor blades. Make sure to cut away from your fingers. Have available soap and water for cleaning cuts. Do not use dull razor blades - you have to apply more pressure, increasing the risk of cuts.}
\end{itemize}

\subsubsection*{Preparation Procedure}
\begin{enumerate}
\item{Put collected leaves into a beaker with water and a few drops of GV.}
\end{enumerate}

\subsubsection*{Activity Procedure}
\begin{enumerate}
\item{Collect Materials}
\item{Cut a leaf in half, vertically. Next, cut a very thin transverse section from the centre of the leaf, so that the mid rib is included. The result will be in a thin diamond-like cross section of the leaf.}
\item{Mount the cross section on a slide with a drop of water and cover it with a cover slip.}
\item{Observe the specimen under the water drop microscope. They should be able to see the vascular bundles in the mid rib and differentiate between the upper and lower surface.}
\item{Draw what you see in the microscope.}
\end{enumerate}

\subsubsection*{Results and Conclusion}
The upper and lower epidermis will be seen in the water drop microscope.  You should also be able to view the palisade cells.

\subsubsection*{Clean Up Procedure}
\begin{enumerate}
\item{Collect all the used materials, cleaning and storing items that will be used later.}
\item{Dispose of waste containing GV in a pit latrine.}
\end{enumerate}

\subsubsection*{Discussion Questions}
\begin{enumerate}
\item{Why do you think there is a close package of palisade cells at the upper surface of the leaf?}
\item{What would happen if the stomata were at the upper surface of the leaf?}
\item{What is the function of the cuticle on the upper surface of the leaf?}
\item{Mention the functions of stomata in relation to photosynthesis.}
\end{enumerate}

\subsubsection*{Notes}
The water drop microscope can only show the outlines of cells. The stomata and conducting tissues cannot be seen clearly. The best result can be obtained through the use of succulent leaves like a comelina plant.

\subsection{Test for Starch in Leaves}
Photosynthesis is the process by which green plants and some other organisms use sunlight to synthesize food from carbon dioxide and water. One product of photosynthesis in green plants is starch. The presence of starch can be confirmed by the addition of iodine solution.

\subsubsection*{Learning Objectives}
\begin{itemize}
\item{To show that starch is a product of photosynthesis.}
\end{itemize}

\subsubsection*{Materials}
Green leaf, ethanol*, iodine solution*, heat source, cooking pots, water, test tube*, white tile, dropper*, and cotton wool*

\subsubsection*{Hazards and Safety}
\begin{itemize}
\item{Ethanol is very flammable! Make sure that student cover their test tubes with cotton wool to avoid excess release of ethanol vapour. If a test tube catches on fire, instruct student to cover the tube with a non-flammable object to extinguish the flame.}
\end{itemize}


\subsubsection*{Preparation Procedure}
\begin{enumerate}
\item{Collect green leaves from the environment. Try to find leaves that do not have a very waxy outer coat.}
\item{Heat water to boiling using the heat source.}
\end{enumerate}

\subsubsection*{Activity Procedure}
\begin{enumerate}
\item{Choose one leaf and submerge a piece of it in the boiling water for about 3 minutes. (Note: the piece of leaf used should be no larger than a bottle cap.)}
\item{Remove the leaf from the water and insert it into a test tube containing methylated spirit and plug the test tube with a piece of cotton wool. The test tube should be less than half full of ethanol.}
\item{Submerge the test tube in the boiling water and leave it to boil until the leaf loses all of its colour.}
\item{Once the leaf has lost its colour, Remove it from the ethanol solution and dip it briefly into the boiling water to remove the ethanol and soften it.}
\item{Spread the decolourized leaf on a white tile and add iodine solution until the whole leaf is covered. Record your observations.}
\end{enumerate}

\subsubsection*{Results and Conclusion}
The leaf is dipped in hot water to kill the cells. The leaf is then submerged in boiling ethanol to extract the colour from the leaf. The ethanol will change to a green colour while the leaf should lose all of its colour to become white. When iodine solution is added, it should turn a dark blue/black colour which indicates the presence of starch in the leaf.

\begin{figure}[h]
\begin{center}
\def\svgwidth{6cm}
\input{./img/starch-leaf.pdf_tex}
\caption{When iodine solution is added, the leaf will turn to a black-blue colour if it is positive for starch}
\label{fig:starch}
\end{center}
\end{figure}

\subsubsection*{Clean Up Procedure}
\begin{enumerate}
\item{Collect all the used materials, cleaning and storing items that will be used later. No special waste disposal is required.}
\end{enumerate}

\subsubsection*{Discussion Questions}
\begin{enumerate}
\item{What was the reason for boiling the leaf?}
\item{What was the importance of boiling the leaf in ethanol? What did you observe during this step?}
\item{Why was the test tube containing methylated spirit plugged with cotton wool?}
\item{Why was the leaf dipped in boiling water after it was removed form the ethanol?}
\item{Why was a water bath used to heat the ethanol?}
\item{What did you observe when the iodine was added to the leaf? What does this indicate is present in the leaf?}
\end{enumerate}

\subsubsection*{Notes}
It is important that the leaf does not contain a thick waxy coating. Before doing this experiment with students, test some leaves from the local environment to ensure that they respond well to the experiment. Leaves such as amaranthus, beans and commilina respond fast. Make sure that the leaf has been in sunlight for at least 6 hours prior to the experiment or there may not be enough starch present to detect. This practical should not be done in the morning.
Ethanol boils at a lower temperature than water, thus it can be boiled in a water bath. Ethanol is very flammable and it is possible that the top of test tube catches fire. If this happens a non-flammable material such as glass or metal can be used to cover the flame and deprive it of oxygen.


\subsection{The Importance of Carbon Dioxide in Photosynthesis}
Photosynthesis is the process by which green plants and some other organisms use sunlight to synthesize food from carbon dioxide and water. Carbon dioxide is needed for photosynthesis.

\subsubsection*{Learning Objectives}
\begin{itemize}
\item{To show that carbon dioxide is necessary for photosynthesis.}
\end{itemize}

\subsubsection*{Materials}
Potted plant, sodium hydroxide*, ethanol*, iodine solution*, heat source, cooking pot, water, test tube*, white tile*, dropper*, cotton wool, empty water bottle or clear plastic bag, and rubber bands

\subsubsection*{Hazards and Safety}
\begin{itemize}
\item{Sodium hydroxide is corrosive to skin and wood. Even when dilute it can blind if it gets in the eyes. Neutralise spills with a weak acid.}
\item{Ethanol is very flammable! Make sure that student cover their test tubes to avoid excess release of ethanol vapour. If a test tube catches on fire, instruct student to cover the tube with a non-flammable object to extinguish the flame.}
\end{itemize}

\subsubsection*{Preparation Procedure}
\begin{enumerate}
\item{Put a potted plant in a dark place for 24 hours to de-starch its leaves.}
\item{Enclose one leaf in a clear plastic bag or empty plastic water bottle containing approximately one teaspoon of sodium hydroxide.}
\item{Seal the plastic container so that no air can enter. The aim of this is to prevent the leaf from coming into contact with carbon dioxide.}
\item{Allow the plant to sit in direct sunlight for at least 6 hours.}
\end{enumerate}

\begin{figure}[h]
\begin{center}
\def\svgwidth{3cm}
\input{./img/CO2-photosynth.pdf_tex}
\caption{Experiment to show the importance of \ce{CO2} in photosynthesis}
\label{fig:CO2-photosynth}
\end{center}
\end{figure}
\subsubsection*{Activity Procedure}

\begin{enumerate}
\item{Choose one of the leaves that has been deprived of carbon dioxide.}
\item{Submerge this leaf in boiling water for about 3 minutes.}
\item{Remove the leaf from the water and insert it into a test tube containing ethanol and plug the test tube with a piece of cotton wool. Note: the test tube should be less than half full of ethanol.}
\item{Submerge the test tube in the boiling water and leave it to boil until the leaf loses all of its colour.}
\item{Once the leaf has lost its colour, remove it from the ethanol solution and dip it briefly into the boiling water to remove the ethanol and soften it.}
\item{Spread the decolourized leaf on a white tile and add iodine solution drop wise until the whole leaf is covered. }
\item{Record your observations and draw a picture showing the colour pattern of the leaf. The leaves should test negative for starch.}
\end{enumerate}

\subsubsection*{Results and Conclusion}
After adding iodine solution to the leaf, it retains the colour of iodine. This implies that the leaf has no starch, which means that photosynthesis did not occur. This proves that carbon dioxide is necessary for photosynthesis.

\subsubsection*{Clean Up Procedure}
\begin{enumerate}
\item{Collect all the used materials, cleaning and storing items that will be used later. No special waste disposal is required.}
\end{enumerate}

\subsubsection*{Discussion Questions}
\begin{enumerate}
\item{What was the aim of keeping the plant in darkness before the experiment?}
\item{What was the purpose of attaching the bag with sodium hydroxide to the leaf?}
\item{What did you observe after adding iodine solution to the leaf? Did photosynthesis occur in this leaf? Explain why or why not.}
\item{Explain why carbon dioxide is necessary for photosynthesis.}
\end{enumerate}

\subsubsection*{Notes}
It is important that the leaf does not contain a thick waxy coating. Ethanol boils at a lower temperature than water, thus it can be boiled in a water bath. Sodium hydroxide is used in the bag with the leaf to absorb any \ce{CO2} that might be present.

\subsection{The Importance of Chlorophyll in Photosynthesis}
Photosynthesis is the process by which green plants and some other organisms use sunlight to synthesize food from carbon dioxide and water. Chlorophyll is a green pigment present in all green plants. It is responsible for the absorption of light which provides energy for photosynthesis to occur. A variegated leaf is a leaf that has two different colours (i.e. green and white or green and red). Only the parts of the leaves which are green contain chlorophyll.

\subsubsection*{Learning Objectives}
\begin{itemize}
\item{To demonstrate the importance of chlorophyll in photosynthesis.}
\end{itemize}

\subsubsection*{Materials}
Variegated leaf, ethanol*, iodine solution*, heat source, cooking pot, water, test tube*, white tile*, dropper*, and cotton wool*

\subsubsection*{Hazards and Safety}
\begin{itemize}
\item{Ethanol is very flammable! Make sure that student cover their test tubes to avoid excess release of ethanol vapour. If a test tube catches on fire, instruct student to cover the tube with a non-flammable object to extinguish the flame.}
\end{itemize}

\subsubsection*{Preparation Procedure}
\begin{enumerate}
\item{Collect variegated leaves from the environment.Try to find leaves that do not have a waxy outer coat.}
\item{Heat water to boiling using the heat source.}
\end{enumerate}

\subsubsection*{Activity Procedure}
\begin{enumerate}
\item{Choose a small piece of one leaf (the piece should not be bigger than the lid of a soda bottle) and draw a picture to show the colour pattern. Label which parts of the plants are green and which parts are not.}
\item{Submerge this leaf in boiling water for about 3 minutes.}
\item{Remove the leaf from the water and insert it into a test tube containing ethanol and to plug the test tube with a piece of cotton wool. Note: the test tube should be less than half full of ethanol.}
\item{Submerge the test tube in the boiling water and leave it to boil until the leaf loses all of its colour.}
\item{Once the leaf has lost its colour, remove it from the ethanol solution and dip it briefly into the boiling water to remove the ethanol and soften it.}
\item{Spread the decolourized leaf on a white tile and add iodine solution drop wise until the whole leaf is covered. Record your observations and draw a diagram showing the colour pattern of the leaf.}
\end{enumerate}

\begin{figure}[h]
\begin{center}
\def\svgwidth{6cm}
\input{./img/leaf-stain-1.pdf_tex}
\caption{A diagram of a variegated leaf.}
\label{fig:variegate leaf}
\end{center}
\end{figure}

\subsubsection*{Results and Conclusion}
The leaf is dipped in boiling water to kill the cells. The leaf is then submerged in boiling ethanol to extract the colour from the leaf. The ethanol will change to a green colour while the leaf should lose all of its colour, becoming white. When iodine solution is added, the leaf will turn a dark blue/black colour in all the places that the leaf was green. The non-green parts of the leaf should not turn dark, but should remain the colour of iodine. This indicates that chlorophyll is necessary for photosynthesis because only the parts of the leaf containing chlorophyll were able to photosynthesise and produce starch.

\subsubsection*{Clean Up Procedure}
\begin{enumerate}
\item{Collect all the used materials, cleaning and storing items that will be used later. No special waste disposal is required.}
\end{enumerate}

\subsubsection*{Discussion Questions}
\begin{enumerate}
\item{Why is it important to draw the leaf before starting this experiment?}
\item{What was the reason for boiling the leaf?}
\item{What was the importance of boiling the leaf in ethanol?}
\item{Why was the test tube containing methylated spirit plugged with cotton wool?}
\item{Why was the leave dipped in boiling water after it was removed form the ethanol?}
\item{Why was a water bath used to heat the ethanol rather than an open flame?}
\item{What did you observe when the iodine was added to the leaf? Which part of the leaf showed the presence of starch?}
\item{Why is chlorophyll necessary for photosynthesis?}
\item{Why would it be a bad idea to do this experiment early in the morning or on a rainy day?}
\end{enumerate}

\subsubsection*{Notes}
It is important that the leaf does not contain a thick waxy coating. Variegated leaves can be found in areas that contain decorative plants, such as in front of houses and buildings. Before doing this experiment with students, test some variegated leaves from the local environment to ensure that they respond well to the experiment. Make sure that the leaf has been in sunlight for at least 6 hours prior to the experiment or there may not be enough starch to detect. This practical should NOT be done in the morning.
Ethanol boils at a lower temperature than water, thus it can be boiled in a water bath. Ethanol is very flammable and it is possible that the top of test tube catches fire. If this happens a non-flammable material such as glass or metal can be used to cover the flame and deprive it of oxygen.

\subsection{The Importance of Light in Photosynthesis}
Photosynthesis is the process through which green plants and some other organisms use sunlight to synthesize food from carbon dioxide and water. Light energy is required for photosynthesis to occur.

\subsubsection*{Learning Objectives}
\begin{itemize}
\item{To show that light is needed for photosynthesis.}
\end{itemize}

\subsubsection*{Materials}
Aluminium foil or black carbon paper, clips, a green leaf, ethanol*, iodine solution*, heat source, cooking pot, water, test tube*, white tile*, dropper*, and cotton wool

\subsubsection*{Hazards and Safety}
\begin{itemize}
\item{Ethanol is very flammable! Make sure you cover the test tubes with cotton wool to avoid excess release of ethanol vapour. If the test tube catches on fire, cover the tube with a non-flammable object to extinguish the flame.}
\end{itemize}

\subsubsection*{Preparation Procedure}
\begin{enumerate}
\item{Put a potted plant in a dark place for 24 hours to de-starch it.}
\item{Use aluminium foil or carbon paper to cover a portion of the upper and lower epidermis of a leaf (see diagram).}
\item{Allow the plant to sit in sunlight for at least 6 hours.}
\end{enumerate}

\subsubsection*{Activity Procedure}
\begin{enumerate}
\item{Submerge the leaf in boiling water for about 3 minutes.}
\item{Remove the leaf from the water and insert it into a test tube containing ethanol and plug the test tube with a piece of cotton wool. Note: the test tube should be less than half full of ethanol.}
\item{Submerge the test tube in the boiling water and leave it to boil until the leaf loses all of its colour.}
\item{Once the leaf has lost its colour, remove it from the ethanol solution and dip it briefly into the boiling water to remove the ethanol and soften it.}
\item{Spread the decolourized leaf on a white tile and add iodine solution drop wise until the whole leaf is covered.}
\item{ Record your observations and draw a diagram showing the colour pattern of the leaf after the addition of iodine solution.}
\end{enumerate}

\subsubsection*{Results and Conclusion}
The aluminium foil blocks the sunlight from reaching the leaf, thus preventing photosynthesis from taking place. Any part of the leaf that was covered with foil will test negative for starch while the parts exposed to sun will test positive for starch. This proves that sunlight is necessary for photosynthesis.

\subsubsection*{Clean Up Procedure}
\begin{enumerate}
\item{Collect all the used materials, cleaning and storing items that will be used later. No special waste disposal is required.}
\end{enumerate}

\subsubsection*{Discussion Questions}
\begin{enumerate}
\item{What was the purpose of covering the leaf and allowing it to sit in the sun?}
\item{Which parts of the leaf tested positive for starch?}
\item{What did you observe when the iodine was added to the leaf? Which part of the leaf showed the presence of starch?}
\item{Why is sunlight light necessary for photosynthesis?}
\end{enumerate}

\subsubsection*{Notes}
It is important that the leaf does not contain a thick waxy coating. Do the procedure for testing a plant for starch before setting the aluminium foil to ensure it works.
Because ethanol boils at a lower temperature that water, the ethanol solution can be boiled in a water bath. Ethanol is very flammable and it is possible that the top of test tube will catch fire. If this happens, cover the top of the test tube with a non-flammable material such as glass or metal to cover the flame and deprive it of oxygen.

\subsection{Oxygen as a By-product of Photosynthesis}

Photosynthesis is the process by which green plants and some other organisms use sunlight to synthesize food from carbon dioxide and water. Photosynthesis produces oxygen. This helps to replace the oxygen that is used during burning, respiration, rusting and other processes.

\subsubsection*{Learning Objectives}
\begin{itemize}
\item{To demonstrate that oxygen is a by-product of photosynthesis.}
\end{itemize}

\subsubsection*{Materials}
2 empty plastic bottles (350 mL), straw, potted plant, super glue, sodium hydroxide*, sodium hydrogen carbonate*, dilute weak acid (citric or acetic acid)*

\subsubsection*{Hazards and Safety}
\begin{itemize}
\item{Sodium hydroxide is corrosive to skin and wood and even when dilute can blind if it gets in the eyes. Neutralise spills with a weak acid.}
\end{itemize}

\begin{figure}[h]
\begin{center}
\def\svgwidth{6cm}
\input{./img/plant-resp.pdf_tex}
\caption{An activity showing the production of oxygen.}
\label{fig:oxygen}
\end{center}
\end{figure}

\subsubsection*{Activity Procedure}
\begin{enumerate}
\item{Take two bottles and make a hole in one side of each bottle and connect them using a straw. Make sure there is an airtight seal by sealing leaks with superglue or cellotape.}
\item{Label the bottles A and B.}
\item{In bottle A put potted plant.}
\item{In bottle B put about one teaspoon of sodium hydroxide crystals. This is to absorb any excess carbon dioxide later in the experiment.}
\item{Tie or bend the connecting straw to prevent movement of air between the two bottles.}
\item{Squeeze extra air out of bottle B and cap it tightly.}
\item{In a separate beaker, combine acid and sodium hydrogen carbonate in order to form carbon dioxide gas. Slowly pour the gas (not the liquid) into bottle A. Repeat this until a glowing splint is extinguished in the mouth of test tube A. The aim of this is to fill the bottle with carbon dioxide, thus ensuring that any oxygen found later was produced by the plant.}
\item{Seal bottle A and allow the set up to sit in sunlight for 6 hours.}
\item{After 6 hours, open the straw and squeeze bottle A to force any gas into bottle B.}
\item{Shake bottle B so that any carbon dioxide gas is absorbed by the sodium hydroxide crystals.}
\item{Open bottle B and use a glowing splint to test for oxygen gas in bottle B.}
\end{enumerate}

\subsubsection*{Results and Conclusion}
When a glowing splint was inserted into bottle B it relights. This shows the presence of oxygen gas from the potted plant in bottle A.

\subsubsection*{Clean Up Procedure}
\begin{enumerate}
\item{Collect all the used materials, cleaning and storing items that will be used later. No special waste disposal is required.}
\end{enumerate}

\subsubsection*{Discussion Questions}
\begin{enumerate}
\item{What is the aim of putting sodium hydroxide in bottle B?}
\item{Why was bottle B compressed at the beginning of the experiment?}
\end{enumerate}

\subsubsection*{Notes}
There should be enough ce\\{CO2} in Bottle A and no loss of air between the two bottles during this experiment.

\subsection{Essential Minerals in Plants}
Plants need mineral elements in addition to the food they manufacture. Mineral elements are found in the soil or dissolved in water and they are absorbed by plants in the form of ions. Mineral elements required for normal healthy plant growth include nitrogen, phosphorus, potassium, magnesium, calcium, sulphur and iron. Each mineral element has a specific function in the plant body some are used in the production of building materials while others play an important role in the metabolic activities of the plant.

\subsubsection*{Learning Objectives}
\begin{itemize}
\item{To investigate the roles of essential mineral elements in plant nutrition.}
\item{To identify different mineral deficiencies in plants.}
\end{itemize}

\subsubsection*{Materials}
Inorganic fertilizers: CAN, DAP, NPK and SA; magnesium sulphate*, iron pills*, sodium chloride*, beakers*, cotton wool*, maize plants, and rain water

\subsubsection*{Preparation Procedure}
\begin{enumerate}
\item{Sow maize seeds and wait for about 5-7 days for the seedling to develop.}
\item{Label six beakers A, B, C, D, E and F and put a bundle of cotton wool into each beaker.}
\item{Grind CAN, DAP, NPK, SA, MgSO4, Fe and NaCl so they are in fine powder form.}
\end{enumerate}
\subsubsection*{Activity Procedure}
\begin{enumerate}
\item{Make seven solutions of salts by dissolving the specified amounts in approximately 1 L of rainwater}
\begin{enumerate}
\item{\textbf{Solution 1:} one slightly heaped teaspoon of  sodium chloride}
\item{\textbf{Solution 2:} one flat teaspoon of sodium chloride + a pinch of CAN (a few crystals)}
\item{\textbf{Solution 3:} one flat teaspoon of  sodium chloride + a pinch of DAP (a few crystals)}
\item{\textbf{Solution 4:} one flat teaspoon of sodium chloride + a pinch of NPK (a few crystals)}
\item{\textbf{Solution 5:} one flat teaspoon of sodium chloride + a pinch of SA (a few crystals)}
\item{\textbf{Solution 6:} one flat teaspoon of sodium chloride + a pinch of MgSO4 (a few crystals)}
\item{\textbf{Solution 7:} one flat teaspoon of sodium chloride + a pinch of iron (a few crystals)}
\end{enumerate}
\item{Combine these solutions in beakers as follows. Put 2 mL of each mentioned solution in the beaker:}
\begin{enumerate}
\item{\textbf{Beaker A (all nutrient)} solutions 2, 4, 6 and 7}
\item{\textbf{Beaker B (calcium deficient)} solutions 4, 6 and 7}
\item{\textbf{Beaker C (iron deficient)} solutions 2, 4, and 6}
\item{\textbf{Beaker D (magnesium deficient)} solutions 2, 4, 5, and 7}
\item{\textbf{Beaker E (potassium deficient)} solutions 2, 3, 6, and 7}
\item{\textbf{Beaker E (no nutrients)}  6 mL of solution 1}
\end{enumerate}
\item{Place 3 seedlings in each beaker and place beakers near a window.}
\item{Measure and record the height of each of each seeding every day}
\item{Observe and record the colour of the leaves of the seedlings in each bottle every three days.}
\item{Make sure that plants do not dry out-if the water level gets low, increase by adding 1 mL of each solution added initially.}
\end{enumerate}

\subsubsection*{Results and Conclusion}
When plants lack one of the mineral elements, their growth will be disturbed. The plants will have slow growth and leaves may drop or change colour.

\begin{tabular}{|c|c|c|}
\hline Essential Plant Element & Role of Element in Plant Growth & Deficiency Symptoms \\ 
\hline Nitrogen (N) & Makes proteins, manufactures chlorophyll, and promotes normal plant growth & Leaves become pale green or yellow, the plant has small leaves, thin, weak stem, stunned growth\\ 
\hline Phosphorus (P) & Root and Branch growth, makes proteins, releases energy during respiration & Short or small roots, leaves, and branches; leaves become a reddish purple \\ 
\hline Potassium (K) & Potassium is used during photosynthesis and for protein metabolism in young leaves & Yellow leaves with dead spots especially at the margins and tips \\ 
\hline Magnesium (Mg) & Creates chlorophyll and helps in enzyme activity & Leave become yellow \\ 
\hline Calcium (Ca) & Promotes normal plant growth and the creation of cell walls & Poor root growth and dead growing regions \\ 
\hline Sulphur (S) & Synthesizes or creates proteins & Small growth and yellow patches on leaves \\ 
\hline Iron (I) & Creation of Chlorophyll & Thin and weak stems; leaves become white or pale \\ 
\hline 
\end{tabular} 

\subsubsection*{Clean Up Procedure}
\begin{enumerate}
\item{Collect all the used materials, cleaning and storing items that will be used later.  No special waste disposal is required.}
\end{enumerate}

\subsubsection*{Discussion Questions}
\begin{enumerate}
\item{What minerals are required by plants in large amount?}
\item{What observations did you make about the plant that was} 
\begin{enumerate}
\item{calcium deficient}
\item{iron deficient}
\item{magnesium deficient}
\item{ potassium deficient}
\end{enumerate}
\item{Look in a book to identify the qualities of a plant that is nitrogen deficient.}
\end{enumerate}


\subsubsection*{Notes}
Minerals are required by plants in small quantities. Solutions that are too concentrated may kill the plant because of water loss through osmosis.
