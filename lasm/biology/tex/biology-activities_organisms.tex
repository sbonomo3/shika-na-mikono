\section{Interaction of Living Organisms}
Our environment is made up of two components. First are the biotic (living) components like plants, animals, and other groups of living organisms. There are also abiotic (non-living) components like water, air, soil, rocks, climate, and weather. Biotic components depend on abiotic components for their survival. The type of abiotic components found in different areas determine the types of biotic components found there.

\subsection{Investigation of Abiotic and Biotic Components in the Environment}

\subsubsection*{Learning Objectives}
\begin{itemize}
\item{To describe biotic and abiotic components in the environment.}
\end{itemize}

\subsubsection*{Materials}
Ants, termites, tadpoles, soil, stones, plants, dried fish, beaker*, water, and plastic bags

\subsubsection*{Preparation Procedure}
\begin{enumerate}
\item{Collect live tadpoles from a lake in a plastic bottle with water.}
\end{enumerate}

\subsubsection*{Activity Procedure}
\begin{enumerate}
\item{Bring soil, plants, ants, termites, stones and uprooted plant seedlings inside.}
\item{Blow into a plastic bag and tie it to hold air in.}
\item{Fill pne beaker with only water, a second beaker with a tadpole in water, and a third beaker with a dry fish.}
\item{Arrange all of the components into 2 groups: biotic and abiotic.}
\end{enumerate}

\subsubsection*{Results and Conclusion}
Non-living components are very important to the survival of biotic components. Water, air, and soil are all abiotic components. Light, water, and carbon dioxide are also abiotic but plants cannot manufacture their foods without them. It is important to preserve our environment to maintain the biotic and abiotic components for our survival.

\subsubsection*{Clean Up Procedure}
\begin{enumerate}
\item{Return the biotic and abiotic components to their environment. Clean and store items that will be used later. No special waste disposal is required.}
\end{enumerate}

\subsubsection*{Discussion Questions}
\begin{enumerate}
\item{What will happen to the biotic components if there is no water?}
\item{Suppose the level of oxygen goes down drastically in the environment. What kind of biotic components will survive?}
\item{Is soil biotic or abiotic?}
\end{enumerate}

\subsubsection*{Notes}
This activity will enable the students to realise the importance of abiotic components normally thought to be freely available. The activity will raise the students consciousness in avoiding the activities causing water, air and soil pollution. Vegetation plays a big role in purifying air by increasing the level of oxygen in the atmosphere but plants depend on water and soil for their survival.

\subsection{Construction of Food Webs and Food Chains}
Feeding relationships can be shown in a simple way where organisms feed on the next organisms in a liner sequence. This is called a food chain. However, in reality most organisms have several food sources that interact with one another. These interactions can be represented in a food web.

\subsubsection*{Learning Objectives}
\begin{itemize}
\item{To mention the components of a food chain and food web.}
\end{itemize}

\subsubsection*{Materials}
Manila sheet or flat boxes, maker pens, specimens or pictures of maize seedlings, termite, toad, caterpillar (butter fly/beetle larva), and a small bird

\subsubsection*{Preparation Procedure}
\begin{enumerate}
\item{Germinate maize grains to get seedlings.}
\item{Ask students to collect and bring in termites, toad, a small bird, and a caterpillar in a cage or plastic bottle.}
\end{enumerate}

\subsubsection*{Activity Procedure}
\begin{enumerate}

\item{Arrange the 3 organisms on the manilla paper or flat box in such away that one organism is the food source for another organism. This feeding relationship should make a line, for example maize, caterpillar, small bird.}
\item{Write names of each organism and their tropic level.}
\item{Draw arrows on the manilla paper, pointing away from the organism being eaten.}
\item{Arrange all the organisms randomly on the other manilla sheet.}
\item{Draw arrows away from each organism being eaten towards the organisms that is eating it.}
\item{Write the names of each organism and their tropic level.}

\end{enumerate}

\subsubsection*{Results and Conclusion}
The first diagram indicates a food chain while the second one indicates a food web. Food chain shows a sequence of living things in which each organism is the food of the next one in the sequence. Arrows are used to show the direction of flow of energy. A food chain starts with the producer and ends with the top consumer.

\subsubsection*{Clean Up Procedure}
\begin{enumerate}
\item{Collect all the used materials, cleaning and storing items that will be used later. No special waste disposal is required.}
\end{enumerate}

\subsubsection*{Discussion Questions}
\begin{enumerate}
\item{Write the names of the food relation in the first and second drawings.}
\item{What is the importance of food web in real life?}
\end{enumerate}

\subsubsection*{Notes}
Primary producers occupy the lowest level in trophic levels. In most cases they are plants. The top consumers occupy the top levels. In this activity birds were the top consumer but in most cases decomposers like bacteria or fungi would be at the top.
