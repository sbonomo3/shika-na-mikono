\section{Transport of Materials in Living Things}
Living things need transport systems to supply all their cells with food, oxygen, and other materials in order to carry out life processes such as growth, respiration, and reproduction. Lungs take in oxygen for the combustion of food and they eliminate the carbon dioxide produced. The urinary system disposes of dissolved waste molecules (urea), the intestinal tract removes solid wastes, and the skin and lungs rid the body of heat energy. The circulatory system moves all these substances to and from cells where they are needed or produced, responding to changing demands. The methods of transport are diffusion, osmosis and mass flow.

These activities can be prepared by the teacher and performed easily by students to show the importance of diffusion, osmosis, and mass flow in living organisms.  

\subsection{Demonstration of Diffusion}

Diffusion is the movement of particles from an area of high concentration to an area of low concentration. Diffusion continues until the particles are evenly distributed.

\subsubsection*{Learning Objectives}
\begin{itemize}
\item{To carry out experiment to demonstration the process of diffusion.}
\end{itemize}

\subsubsection*{Materials}
Beakers*, water, soda or water bottle caps, GV stain*, droppers*

\subsubsection*{Activity Procedure}
\begin{enumerate}
\item{Put a very small amount of GV in the bottle cap.}
\item{Fill the beaker about half way with water.}
\item{Draw a drop of GV from their cap using the dropper.}
\item{Put one drop of GV into the beaker.}
\item{Observe what is happening in the container for the first 5 minutes.}
\item{After 20 minutes, observe their beaker again and record their observations.}
\end{enumerate}

\subsubsection*{Clean Up Procedure}
\begin{enumerate}
\item{Collect all used materials, storing items that will be used later. The chemicals used in this experiment require no special disposal.}
\end{enumerate}

\subsubsection*{Hazards and Safety}
\begin{itemize}
\item{GV stains skin and clothes.}
\end{itemize}

\subsubsection*{Discussion Questions}
\begin{enumerate}
\item{Apart from this experiment, where else have you seen diffusion taking place?}
\end{enumerate}

\subsubsection*{Results and Conclusion}
Immediately after putting the drop of GV in water, GV molecules start moving towards areas with lower concentration of GV molecules. The movement continues until the whole solution will have the same concentration of GV molecules.

\subsubsection*{Notes}
Diffusion can take place through any substance. Other examples of diffusion include colour of tea spreading in hot water, the smell of perfume molecules spreading through the air, and the sight of heavy smoke thinning into the air.
Molecules are always moving. Solid particle remain in the same location, however liquid, gas, and solute particles move randomly through space. Because there are more molecules in an area of high concentration than in an area of low concentration, more molecules are available to move from the area of high concentration to the area of low concentration than are available to move from the area of low concentration to the area of high concentration. While particles are always moving in all directions, over time there is a net flow of particles from the area of high concentration to low concentration. Eventually, the concentration in all parts is the same. In this state the molecules continue to move from one place to another, but no net change is observed. This is called equilibrium.

\subsection{Osmosis}

Osmosis is the movement of water molecules through a semi-permeable membrane from an area of low solute concentration to an area of high solute concentration.

\subsubsection*{Learning Objectives}
\begin{itemize}
\item{To demonstrate osmosis.}
\end{itemize}

\subsubsection*{Materials}
Irish potatoes, beakers*, sugar, water, knife, kerosene stove, and a cooking pot

\subsubsection*{Preparation Procedure}
\begin{enumerate}
\item{Dissolve 10 spoons of sugar in about 100 mL of water. This solution should be very concentrated and thick.}
\item{Fill a cooking pot half way with water and heat it to boiling on a stove.}
\end{enumerate}

\subsubsection*{Activity Procedure}
\begin{enumerate}
\item{Boil one potato and leave the other uncooked.}
\item{Peel the potatoes and cut them into two halves.}
\item{Make a shallow hole in the four halves of the potatoes. Each cut potato should look like a bowl.}
\item{Put a small amount of water in each beaker.}
\item{Put one carved potato in each beaker. The water should not spill into the inside of the potato bowl.}
\item{Put sugar solution in the centre of 1 raw potato and 1 boiled potatoes. The other two potatoes will act as controls.}
\item{Set the experiment aside for an hour.}
\item{After one hour examine the potatoes and write down what you observe.}
\end{enumerate}

\subsubsection*{Results and Conclusion}
After an hour the level of water in the raw potato with sugar will rise while in the boiled potato with sugar there will be no change. Boiling kills the cells, therefore the cell membrane loses its permeability. The potatoes without sugar should show no change.

\subsubsection*{Clean Up Procedure}
\begin{enumerate}
\item{Collect all the used materials, cleaning and storing items that will be used later. No special waste disposal is required. The used potatoes should not be eaten.}
\end{enumerate}

\subsubsection*{Discussion Questions}
\begin{enumerate}
\item{What is the function of the controls in this experiment?}
\item{If you put chemical fertilizers on plant seedlings in the dry season, the plants dries up. Why does this happen?}
\end{enumerate}

\subsubsection*{Notes}
Osmosis is a special case of diffusion - the movement of a substance from an area of high concentration to an area of low concentration. The area of low solute concentration has relatively high water concentration whereas the area of high solute concentration has relatively low water concentration. Therefore water moves from the area of high water concentration to the area of low water concentration, or from the area of low solute concentration to the area of high solute concentration. The solute cannot pass through the semi-permeable membrane.

\subsection{Demonstration of Capillarity}
Capillarity is the action that causes water to rise in narrow tubes. Capillarity is made possible by cohesion and adhesion forces. Cohesion is the attraction between molecules of the same substance and adhesion is the attraction between  molecules of different substances. Water molecules in a plant are attracted to each other (cohesion) as well as to the walls of the xylem vessel (adhesion). Xylem vessels have a narrow tube which makes it possible for water to rise in them through capillarity.

\subsubsection*{Learning Objectives}
\begin{itemize}
\item{To conduct an experiment to demonstrate capillarity.}
\end{itemize}

\subsubsection*{Materials}
Plastic tubes of different diametres (the empty ink tube of a pen or a straw can be used), water, GV stain*, clothes pin, and 2 beakers*

\subsubsection*{Hazards and Safety}
\begin{itemize}
\item{GV will stain clothes and skin.}
\end{itemize}

\subsubsection*{Preparation Procedure}
\begin{enumerate}
\item{Fill a beaker half way with water and add a few drops of GV to make a coloured solution.}
\end{enumerate}

\subsubsection*{Activity Procedure}
\begin{enumerate}
\item{Dip the different straws in the coloured solution.}
\item{Attach 1 clothes pin to the edge of the beaker and tie the other clothes pin to it. In the clothes pin there will be 2 holes. Place 2 straws in the 2 holes.}
\item{Observe the changes as time goes by.}
\end{enumerate}

\subsubsection*{Results and Conclusion}
The coloured solution climbed to a higher level in the thinner tube than the wide one. Xylem vessels in a plant works the same ways as a capillary tubes.

\subsubsection*{Clean Up Procedure}
\begin{enumerate}
\item{Collect all the used materials, cleaning and storing items that will be used later. GV stain should be disposed of in the pit latrine.}
\end{enumerate}

\subsubsection*{Discussion Questions}
\begin{enumerate}
\item{What changes did you notice in the experiment?}
\item{Which part of the vascular system of a plant functions like the capillary tubes?}
\end{enumerate}

\subsubsection*{Notes}
The narrower the tubes used in this experiment, the more you will be able to see the capillary effect.

\subsection{Demonstration of Mass Flow}

\subsubsection*{Learning Objectives}
\begin{itemize}
\item{To carry out experiments to demonstrate the process of mass flow.}
\end{itemize}


Mass flow is the movement of fluids within a cell or along a vessel that does not pass through a membrane. This mode of transport is important in large complex organisms where substances are required in a large amounts and also have to be transported over large distances to reach the required area at the right time. Diffusion and osmosis can not perform such large functions. In many animals mass flow is demonstrated in their lymphatic and circulatory systems. In plants, mass flow is responsible for the transport of water and mineral salts. These travel from roots, through the stem and branches to the leaves in xylem vessels. Sugars are also transported from different parts of a plant through phloem vessels by the same process.

\subsubsection*{Materials}
Beaker*, water, GV stain*, water drop microscope*, razor blade, dropper*, plastic slide, plastic cover slip, and an uprooted plant (e.g. commelina plant)

\subsubsection*{Hazards and Safety}
\begin{itemize}
\item{GV stains clothes and skin.}
\end{itemize}

\subsubsection*{Preparation Procedure}
\begin{enumerate}
\item{Put some water into a beaker and add two drops of GV to colour the water.}
\item{Place the uprooted plant upright in a plastic beaker containing coloured water.}
\item{Leave the plant in the sun for one hour.}
\item{After an hour remove the plant from the sun and put it in a special area to be used by students for observation.}
\end{enumerate}

\subsubsection*{Activity Procedure}
\begin{enumerate}
\item{Cut a leaf, stem, and root from the plant in half. Make 3 transverse cross sections by cutting a thin slice from the centre of the root, stem, and leaf.}
\item{Mount the circular cross sections on a slide with a drop of water and cover it with a plastic cover slip.}
\item{Observe the colour of the specimens carefully using the water drop microscope.}
\item{Draw the cross section of the root, stem, and leaf they have just observed, showing the distribution of colour.}
\end{enumerate}

\subsubsection*{Results and Conclusion}
The presence of colour in the leaf, stem, and root cross section indicates the presence of coloured water that was present in the container. This suggests that there was a movement of coloured water molecules from the container to the rest of the plant parts via the xylem tissue. This proves that there is a mass flow of water from the low plant parts like roots to the high plant parts like the stems and leaves.

\subsubsection*{Clean Up Procedure}
\begin{enumerate}
\item{Collect all the used materials, cleaning and storing items that will be used later. No special waste disposal is required.}
\end{enumerate}

\subsubsection*{Discussion Questions}
\begin{enumerate}
\item{What does the presence of colour in the tissue of leaves, stems and roots indicate?}
\item{What can you conclude from the experiment?}
\end{enumerate}

\subsubsection*{Notes}
 The commelina plant species is best to use in this experiment because it is clearly seen by the water drop microscope and brings better results as compared to other plants.

\subsection{Demonstration of Transpiration Pull}

There are three mechanisms which facilitate movement of water from the ground through the stem to the leaves: root pressure, transpiration pull and capillarity. Transpiration pull occurs when water evaporates through the stomata. As water evaporate through the stomata, mesophyll cells draws water from the xylem of the leaves, which in turn draws water from xylem in the stem. This create a tension called transpiration pull.

\subsubsection*{Learning Objectives}
\begin{itemize}
\item{To conduct experiments to demonstrate transpiration pull.}
\end{itemize}


\subsubsection*{Materials}
Plastic bottles, narrow tubes from a used ink pen, cover from syringe needle, GV stain*, super glue, and the stem from a plant with leaves attached.

\subsubsection*{Hazards and Safety}
\begin{itemize}
\item{GV stain skin and clothes.}
\end{itemize}

\subsubsection*{Preparation Procedure}
\begin{enumerate}
\item{Cut plastic bottles in half to get plastic containers.}
\item{Cut the closed end of the syringe cover to fit the ink pen tube. Fit the ink pen tube into the hole in the syringe cover.}
\item{Seal the junction between the syringe cover and the tube with super glue so that air or water cannot pass through.}
\item{Cut a plant to get leafy shoots without roots.}
\item{Prepare coloured water using few drops of GV.}
\end{enumerate}

\subsubsection*{Activity Procedure}
\begin{enumerate}
\item{Fill the plastic container with coloured water.}
\item{Fill the tube with clean water, making sure that one air bubble is created in the tube. Mark the location of the air bubble using pen.}
\item{Fix the stem into the open part of the syringe cover.}
\item{Transfer the ink pen tube-syringe cover-plant set up into the coloured water in the plastic container. Hold the set-up so that the tube does not touch the bottom of the container. Place near a window.}
\item{Observe the set up every 15 minutes and note the upward movement of air bubble through the tube.}
\end{enumerate}

\subsubsection*{Results and Conclusion}
The leaves and stem will draw up water, causing the air bubble to move up. The higher the rate of transpiration, the higher the speed of the moving air bubble.

\subsubsection*{Clean Up Procedure}
\begin{enumerate}
\item{Collect all the used materials, cleaning and storing items that will be used later. No special waste disposal is required.}
\end{enumerate}

\subsubsection*{Discussion Questions}
\begin{enumerate}
\item{How can you define transpiration pull?}
\item{What can you conclude from the experiment?}
\end{enumerate}

\subsubsection*{Notes}
Capillarity and root pressure are not enough to push water to the leaves. Root pressure facilitate movement of water to the leaves. However, the transpiration pull is associated with water loss. Therefore, transpiration pull facilitates drawing of water upwards, but results in a loss of water. The comelina plant has been tested and works very well for this experiment. The stem of the plant should tightly into the syringe cap.


\subsection{Examination of the Vascular System in Plants}

The vascular system in plants is composed of xylem, which carries water up from the roots to the leaves, and phloem, which transports nutrients down from the leaves to the roots. The arrangement of vascular bundles in dicot stems is a ring shape, while in the monocot stems the bundles are scattered. In monocot roots the vascular bundles are also scattered and have a pith, while in dicot roots the xylem ressembles a star. 

\subsubsection*{Learning Objectives}
\begin{itemize}
\item{To describe the components of the vascular system in plants.}
\end{itemize}

\subsubsection*{Materials}
Water drop microscope*, plastic slides*, plastic cover slips*, 4 plastic cups, maize grains, bean seeds, water, cotton wool, GV stain*, razor blades, and 2 droppers*

\subsubsection*{Hazards and Safety}
\begin{itemize}
\item{GV can stain the skin and clothes.}
\end{itemize}

\subsubsection*{Preparation Procedure}
\begin{enumerate}
\item{Soak maize and bean seeds in water overnight.}
\item{Wet cotton wool and place it into 2 plastic cups.}
\item{Remove the seeds from the water and put them into the two separate cups with cotton wool.}
\item{Cover the seeds with wet cotton wool and leave the seeds for four days.}
\item{Transfer the seedlings to a plastic cup that has water and a few drops of GV.}
\end{enumerate}

\subsubsection*{Activity Procedure}
\begin{enumerate}
\item{Cut a stem in half vertically, then cut a very thin transverse section from centre of the stem, resulting in a thin circular cross section of the stem.}
\item{Cut a root in half vertically, then cut a very thin transverse section from centre of the root, resulting in a thin circular cross section of the root.}
\item{Cut a leaf in half vertically, then cut a very thin transverse section from centre of the leaf, so that the mid rib is included. The result will be in a thin diamond-like cross section of the leaf.}
\item{Mount the cross sections on slides with a drop of water and cover with a cover slip.}
\item{Observe the specimens using the water drop microscope.}
\item{Draw what is seen in the microscope.}
\item{Classify the specimens as monocots or dicots.}
\end{enumerate}

\subsubsection*{Results and Conclusion}
The purple colour will be seen in the xylem tissue, which should be similar to the decription mentioned in the introduction to this activity.

\subsubsection*{Clean Up Procedure}
\begin{enumerate}
\item{Collect all the used materials, storing items that will be used later. Waste containing GV should be disposed of in the pit latrine.}
\end{enumerate}

\subsubsection*{Discussion Questions}
\begin{enumerate}
\item{Describe the difference between the arrangement of the vascular bundles in monocot stems and dicot stems.}
\item{Describe the difference between the arrangement of the vascular bundles in monocot stems and dicot roots.}
\item{What are names of the cells coloured by GV?}
\end{enumerate}

\subsubsection*{Notes}
In this experiment, you will not be able to see the phloem because it is not coloured.

\subsection{Examination of Root Hair in Germinated Seeds}

Roots of plants are responsible for absorption of water and mineral salts from the soil. The roots have root hairs which are responsible in the absorption process. These are extensions of epidermal cells of roots. They are long and slender to provide large surface area for absorption.

\subsubsection*{Learning Objectives}
\begin{itemize}
\item{To explain the functions of root hairs in absorption in the movement of water and mineral salts.}
\end{itemize}



\subsubsection*{Materials}
Bean seeds, maize grains, plastic bottles, water, and soil

\subsubsection*{Preparation Procedure}
\begin{enumerate}
\item{Cut the plastic bottles to get a six inches container.}
\item{Soak the maize grains and bean seeds in two separate clear plastic containers overnight.}
\item{Remove the bean seeds and maize grains from the water and place them into two separate plastic containers with wet cotton wool.}
\item{Leave the seeds for two days to allow them to germinate.}
\item{Shift the seeds into another plastic container with soil.}
\item{Make sure that the seeds are grown near the wall of the container to make sure that the roots will be seen from outside.}
\item{Leave the experiment for three days to allow growth of the roots.}
\end{enumerate}

\subsubsection*{Activity Procedure}
\begin{enumerate}
\item{Observe the root hairs of the germinating seeds through the containers.}
\item{Draw what you have seen.}
\end{enumerate}

\subsubsection*{Results and Conclusion}
You will see the root hairs through the plastic containers. The root hairs look like small white threads on the root tips.

\subsubsection*{Clean Up Procedure}
\begin{enumerate}
\item{Remove the germinating seeds from the classroom and place them where there is enough light to be used for other experiments which require seedlings.}
\end{enumerate}

\subsubsection*{Discussion Questions}
\begin{enumerate}
\item{What would happen if the plants did not have root hairs?}
\item{When plants are transplanted they wilt for sometimes and flourish again after a day or so. Why?}
\item{Root hairs are not found at the tip of the roots. Why this is the case?}
\end{enumerate}

\subsubsection*{Notes}
There are no root hairs at the tips of the roots. The root hairs develop after differentiation. There are a large number of root hairs to increase surface areas for absorption of water and mineral salts. The root hairs need to be observed without uprooting the seeding because they can be destroyed in the process of uprooting.


\subsection{Determination of Pulse Rate}

\subsubsection*{Learning Objectives}
\begin{itemize}
\item{SWABT measure human pulse rate.}
\end{itemize}


Pulse is the result of contraction and relaxation of arteries. Pulse rate is the number of pulses per minute. Pulse rate reflects the heart beat. An adult human's heart beats at an average of 72 times a minute. This can increase or decrease depending on the physical activity, emotional state or health factors.

\subsubsection*{Materials}
Wrist watch, notebook and a pen or a pencil.

\subsubsection*{Activity Procedure}
\begin{enumerate}
\item{Use the first two fingers on the right side of your left wrist to feel for pulse. Pulse can also be measured by placing the first two fingers on one side of the neck under the lower jaw.}
\item{Count the number of pulses in one minute. Record your answer.}
\item{Repeat this measurement 3 times and calculate the average.}
\item{Run around outside for a minute, then return and repeat steps number 2 and 3. Record the pulse rate.}
\end{enumerate}

\begin{figure}[h]
\begin{center}
\def\svgwidth{3cm}
\input{./img/pulse-1.pdf_tex}
\caption{Pulse can be taken at the neck.}
\label{fig:pulse-1}
\end{center}
\end{figure}

\begin{figure}[h]
\begin{center}
\def\svgwidth{4cm}
\input{./img/pulse-2.pdf_tex}
\caption{Pulse can also be measured at the wrist.}
\label{fig:pulse-2}
\end{center}
\end{figure}

\subsubsection*{Results and Conclusion}
There should be an increase in the pulse rate after exercise.

\subsubsection*{Discussion Questions}
\begin{enumerate}
\item{What is pulse rate?}
\item{What was the average pulse rate before exercise?}
\item{What was the average pulse rate after exercise}
\item{Why are these pulse rates different?}
\end{enumerate}

\subsubsection*{Notes}
When counting the pulse rate, there should be no distractions for the students.
