\section{Introduction to Biology}

\subsection{Measuring and Recording Mass, Temperature, Pulse Rate, and Volume}
Measurement is when we use a set unit to describe a factor of the thing that we are studying. There are 6 things that we measure. Mass is how heavy something is. It is measured in grams (g) or kilograms (kg). Length is the how long or wide something is. It is measured in centimetres (cm), metres (m) and kilometres (km). Time is measured in seconds, minutes, and hours. Temperature is how hot or cold something is. Temperature is measured in degrees. Rate is how quickly something changes. It is usually measured in kilometres per hour. Volume is the amount of space something takes up. It is usually measured in litres.

\subsubsection*{Learning Objectives}
\begin{itemize}
\item{To take measurements of mass, temperature, pulse rate, and volume.}
\end{itemize}

\subsubsection*{Materials}
Spring balance*, thermometer, plastic bottles, droppers*, digital watch, cold water, warm water, beaker*, volumetric flask*, twine or rope, marker pen, a ruler, and sand

\subsubsection*{Preparation Procedure}
\begin{enumerate}
\item{Collect materials}
\item{Pour a known amount of water into the plastic bottles and write the volume in millilitres on the bottle.}
\end{enumerate}

\subsubsection*{Activity Procedure}
\begin{enumerate}
\item{Mark the rope at 1 foot intervals.}
\item{Record all of the following measurements.}
\item{Measure the temperature of the cold and warm water by using the thermometer.}
\item{Measure the weight of sand using the spring balance.}
\item{Determine your pulse rate by placing your first two fingers on your neck. This should be measured for one minute.}
\item{Measure your height by using the rope which was marked every foot.}
\item{Measure the volume of the water in the bottle using the volumetric flask.}
\end{enumerate}

\subsubsection*{Results and Conclusion}
The data collected should reflect realistic values for the specimens. 

\subsubsection*{Clean Up Procedure}
\begin{enumerate}
\item{Collect all the used materials, cleaning and storing items that will be used later. No special waste disposal is required.}
\end{enumerate}

\subsubsection*{Discussion Questions}
\begin{enumerate}
\item{Write down the metric units used to measure mass, length, and temperature.}
\item{Which one is heavier, a 1000 grams of water or 1000cc of water? Why?}
\end{enumerate}

\subsubsection*{Notes}
There are several units used to measure length, weight and volume. It can be very difficult to convert units if you do not understand what the unit in question measures. Therefore, knowledge on measurements and their corresponding units will minimize the problem.
