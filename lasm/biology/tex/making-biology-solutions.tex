\chapter{Making Biology Solutions}
Activities in the topics of Nutrition and Respiration require specific solutions. In this section you will find materials and instructions on how to prepare common solutions for the Biology laboratory.

For local and low cost sources of the chemicals mentioned in these preparations, see the section on Sources of Chemicals.

\begin{flushleft}
\textbf{Benedict's Solution}\\
Description: Bright blue solution\\
Use: To test for reducing and non-reducing sugars\\
Result: Gives orange precipitate when boiled with reducing sugar\\
Hazard: Copper is poisonous\\
Procedure: Dissolve 5 spoons of sodium carbonate, 3 spoons of citric acid,
and one spoon of copper sulfate in half a litre of water. Shake until everything is fully dissolved.\\Note: The addition of the citric acid and sodium carbonate should be done slowly as they cause effervescence when mixed quickly.
\end{flushleft}

\begin{flushleft}
\textbf{Calcium Hydroxide Solution (Lime Water)}
\end{flushleft}
\vspace{-10pt}
Description: Opaque white liquid\\
Use: To test for \ce{CO2}\\
Result: This liquid will change from clear to cloudy if \ce{CO2} is present.\\
Procedure: Add 3 spoonfuls of white cement into about half a litre of water. Stir the solution and let it settle. Decant the clear solution and transfer it to a reagent bottle.\\

\begin{flushleft}
\textbf{Citric Acid Solution}
\end{flushleft}
\vspace{-10pt}
Description: Colourless solution\\
Use: To hydrolize non-reducing sugars to reducing sugars\\
Procedure: Dissolve 2 1/2 spoonfuls of citric acid in half a litre of water.\\
Note: The acid should be used immediately after preparation, the remaining solution should be disposed.

\begin{flushleft}
\textbf{Copper Sulfate Solution}
\end{flushleft}
\vspace{-10pt}
Description: Light blue solution\\
Use: To test for proteins, to prepare Benedict's Solution\\
Result: Gives a purple colour when combined with \ce{NaOH} in protein solution\\
Procedure: Dissolve 1 spoonful of \ce{CuSO4} crystals in 1/2 litre of water. Dissolve the \ce{CuSO4} completely.\\

\begin{flushleft}
\textbf{Iodine Solution}
\end{flushleft}
\vspace{-10pt}
Description: Light brown solution\\
Use: To test for starch and lipids\\
Result: Gives a red ring with lipids and a black-blue with starch\\
Procedure: Dilute 1 part concentrated iodine tincture with 9 parts water. Keep the solution in a labelled reagent bottle.\\

\begin{figure}[h]
\begin{center}
\def\svgwidth{2cm}
\input{./img/reagent-bottle.pdf_tex}
\caption{Iodine solution can be easily prepared and stored for later use.}
\label{fig:iodine}
\end{center}
\end{figure}

\begin{flushleft}
\textbf{Sodium Hydroxide Solution}
\end{flushleft}
\vspace{-10pt}
Description: Slightly cloudy white solution\\
Use: To test for proteins\\
Result: Gives a purple colour when combined with \ce{CuSO4} in protein solution\\
Hazard: Corrodes metal, burns skin, and can blind if it gets in eyes\\
Procedure: Combine 1 spoon of \ce{NaOH} with 1/2 litre of water.\\
Burn dry grass and collect the ash. Dissolve 3 spoonfuls of ash into a litre of water. Stir the solution and let it settle. Decant the solution, then place the solution in a labelled reagent bottle.
