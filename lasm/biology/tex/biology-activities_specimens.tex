\chapter{Biology Activities with Specimens}
There are a number of activities in O-Level Biology that require students to observe, identify, and classify different organisms. In this section, you will find a number of those activities and how to execute each in small groups with your students. You can refer to the previous chapter for collection and preservation ideas.

\section{Characteristics of Living Things}
There are 7 characteristics of living things; respiration, reproduction, excretion, irritability, movement, nutrition, and growth. The following activity can be done by students in small groups to show the characteristics of living things and enforce observation skills.

\subsubsection*{Learning Objectives}
\begin{itemize}
\item{To outline the characteristics of living things.}
\item{To differentiate between living and non-living things.}
\end{itemize}

\subsubsection*{Materials}
plastic water bottles, traps, plastic cups, non-living things like a pen and a rock, and a cardboard box

\subsubsection*{Specimens}
\begin{itemize}
\item{Grasshopper, lizard, ant, or any other living things}
\end{itemize}

\subsubsection*{Hazards and Safety}
\begin{itemize}
\item{Some organisms may be poisonous and should be avoided.}
\end{itemize}

\subsubsection*{Preparation Procedure}
\begin{enumerate}
\item{Prepare cages and traps from the plastic water bottles. Purchase rat traps from the market.}
\item{Collect living and non-living things by using the traps and cages.}
\item{Put the collected organisms in cages, petri dishes, and plastic cups for students to observe.}
\end{enumerate}

\subsubsection*{Activity Procedure}
\begin{enumerate}
\item{Observe the specimens, draw and label each.}
\item{Record the characteristics of life you have observed in each specimen.}
\item{Categorize the specimens as living or non-living using your observations.}
\end{enumerate}

\subsubsection*{Results and Conclusion}
Using the seven characteristics of life and observation skills, you should be able to determine which specimens are living and which are non-living. 

\subsubsection*{Clean Up Procedure}
\begin{enumerate}
\item{Collect and clean all the used materials, storing items that will be used later. No special waste disposal required.}
\end{enumerate}

\subsubsection*{Discussion Questions}
\begin{enumerate}
\item{How can you determine if a specimen is a living organism?}
\item{What are the differences between plants and animals?}
\end{enumerate}

\subsubsection*{Notes}
The collection of the specimens and thorough observation is very important. You may not see all of the characteristics in one day, but through observation students will have a general understanding of the characteristics of life.


\section{Introduction to Classification}
Because there are so many different living things in the world, biologists put these organisms into groups to make it easier to study and identify them. This process is called classification. Classification enables scientists to make predictions. When we know the characteristics of a group we can predict the features of an organism in that group. For example, an owl and chicken are both birds. If we know what the heart of a chicken looks like we can predict what the heart of an owl will look like even if we have not seen it.

\subsubsection*{Learning Objectives}
\begin{itemize}
\item{To group living things according to their similarities and differences.}
\end{itemize}

\subsubsection*{Materials}
Marker pen, cardboard, bread, and a tomato

\subsubsection*{Specimens}
\begin{itemize}
\item{Rat, ants, hibiscus or another type of flower, beetle, fish, worm}
\end{itemize}

\subsubsection*{Hazards and Safety}
\begin{itemize}
\item{When collecting and observing specimens, avoid dangerous animals like snakes, black ants, wasps, and bees. Stay away from poisonous plants like deadly nightshade and poisonous fungi like Amonita.}
\end{itemize}

\subsubsection*{Preparation Procedure}
\begin{enumerate}
\item{Collect different living things like fungi, plants of different shapes and sizes, and animals.}
\item{ Place a piece of moist bread near a window to culture bread mould.}
\item{ Cut a tomato in half and leave it overnight to prepare mucor.}
\item{Mount the different specimens on a piece of cardboard box and label each specimens with a single letter.}
\end{enumerate}

\subsubsection*{Activity Procedure}
\begin{enumerate}
\item{Display the specimens for observation.}
\item{Group the organisms based on their similarities and differences.}
\item{Classify the organisms, naming their Kingdom, Phylum/Division, and Class. Refer to the previous chapter as a guide.}
\end{enumerate}

\subsubsection*{Results and Conclusion}
Students are expected to observe and group living things according to their similarities and differences.

\subsubsection*{Clean Up Procedure}
\begin{enumerate}
\item{Collect all the used materials, storing items that will be used later. No special waste disposal is required.}
\end{enumerate}

\subsubsection*{Discussion Questions}
\begin{enumerate}
\item{Why do you think it is important to classify living things?}
\item{Draw and label a specimen from each Phylum.}
\end{enumerate}


\section{Classification System}
Organisms are classified basing on two systems of classification which are artificial and natural systems of classification. Artificial system group organisms according to observable features. Eg. presence or absence of wings. Natural system group organisms according to external as well as internal features. Natural system is the best way of classifying living organisms.

\subsubsection*{Learning Objectives}
\begin{itemize}
\item{To carry out practical activity of classifying living things according to natural and artificial system.}
\end{itemize}

\subsubsection*{Materials}
Cardboard box, marker pen

\subsubsection*{Specimens}
\begin{itemize}
\item{Varieties of living things like insects, lizard, preserved snake, dried fish, earthworms, preserved ascaris, preserved beetles, rat, pictures of reptiles, yeast cells, mushrooms, cultured bread mould and mucor from cut tomatoes, Bidens pilosa(black jack plant), maize plant, Commelina spp plants, conifers branch and cones, cactus stem, variegated leaf, hibiscus flower, and moss plant}
\end{itemize}

\subsubsection*{Hazards and Safety}
\begin{itemize}
\item{Take care of the poisonous organisms during collection of specimens.}
\item{Be careful with the preservatives as they can irritate or damage your skin.}
\end{itemize}

\subsubsection*{Preparation Procedure}
\begin{enumerate}
\item{Collect varieties of specimens (live or preserved).}
\item{Collect pictures showing variety of reptiles.}
\end{enumerate}

\subsubsection*{Activity Procedure}
\begin{enumerate}
\item{Display the collected specimens by mounting them in cardboard boxes. Label them with letters using a marker pen .}
\item{Display the pictures showing varieties of reptiles.}
\item{Observe the external features from each specimen collected and group them.}
\item{Discuss on how to group organisms basing on Artificial and Natural system.}
\end{enumerate}

\subsubsection*{Results and Conclusion}
Students are expected to group organisms by using the observable features and the behaviour of the organisms. They also need to understand how the internal features help to give a better way grouping organisms.

\subsubsection*{Clean Up Procedure}
\begin{enumerate}
\item{Remove all unwanted materials left after observation.}
\item{Return specimens on specimen's bottles for preservation.}
\end{enumerate}

\subsubsection*{Discussion Questions}
\begin{enumerate}
\item{What are the differences between artificial and natural systems of classification?}
\item{What are the advantages and disadvantages of the two systems of classification?}
\end{enumerate}

\subsubsection*{Notes}
Classification should be based on features which show evolutionary relationships. Otherwise few features and external features may lead to the wrong grouping of organisms.


\section{Investigation of Kingdom Fungi}
Kingdom Fungi has many effects on other organisms, like humans.  They can cause disease that directly affect humans and also indirectly affect our way of life through the destruction of crops.  This activity can be done in groups of four to six students to teach them about fungus, how to prevent its negative effects, and increase its benefits.

\subsubsection*{Learning Objectives}
\begin{itemize}
\item{To describe the structures of the representative organisms of each phylum of Kingdom Fungi}
\end{itemize}

\subsubsection*{Materials}
Petri dishes*, water drop microscope*

\subsubsection*{Specimens}
\textit{Instructions for collecting and preserving these specimens are in the previous chapter}
\begin{itemize}
\item{One specimen from Phylum Basidiomycota}
\item{One specimen from Phylum Zygomycota}
\item{One specimen from Phylum Ascomycota}
\end{itemize}

\subsubsection*{Hazards and Safety}
\begin{itemize}
\item{Some mushrooms are poisonous. Wash your hands with soap after this activity.}
\end{itemize}

\subsubsection*{Preparation Procedure}
\begin{enumerate}
\item{Assemble the required materials and specimens.}
\end{enumerate}

\subsubsection*{Activity Procedure}
\begin{enumerate}
\item{Collect at least one sample for each phylum}
\item{Using a knife and a drop of water, mount bread moulds and yeast cells on the plastic slides and cover with a cover slip.}
\item{Observe the specimens using the water drop microscope and draw what you see.}
\item{Observe the mushroom with the naked eyes and draw a labelled diagram.}
\end{enumerate}

\subsubsection*{Discussion Questions}
\begin{enumerate}
\item{What features are common to all species in Kingdom Fungi?}
\item{What are the distinctive features of each phylum in Kingdom Fungi?}
\item{What are the advantages and disadvantages of Kingdom Fungi?}
\item{How do organisms in Kingdom Fungi reproduce and why is this advantageous?}
\end{enumerate}

\subsubsection*{Results and Conclusion}
You should be able to observe the common and distinctive features of the phyla in Kingdom Fungi that are described in the proceeding chapter. Advantages of Kingdom Fungi include delicious food, effective bread and alcohol manufacture, and a market for the chemical preservative industry. Disadvantages of Kingdom Fungi include the spoilage of food, damage to crops, and infections like Athlete's foot or Ringworm. The organisms in Kingdom Fungi reproduce asexually by the use of spores.


\section{Investigation of Division Coniferophyta}
Conifers are shrubs and trees with needle shaped leaves found in cool climates like Iringa, Mbeya and Ruvuma. Their reproductive structures are cones. Male cones are smaller in size, closely packed, and produce pollen grains. Female cones are larger in size, openly packed to receive pollen grains, and produce naked seeds. This activity is especially easy for students to conduct and can be conducted in groups or individually depending on the number of specimens you have.

\subsubsection*{Learning Objectives}
\begin{itemize}
\item{To explain the general and distinctive features of Division Coniferophyta.}
\end{itemize}

\subsubsection*{Materials}
Razor blades

\subsubsection*{Specimens}
\textit{Instructions for collecting and preserving these specimens are in the previous chapter}
\begin{itemize}
\item{One specimen from Phylum Coniferophyta}
\end{itemize}

\subsubsection*{Hazards and Safety}
\begin{itemize}
\item{Razor blades are extremely sharp and should be used with care to prevent injury.}
\end{itemize}

\subsubsection*{Preparation Procedure}
\begin{enumerate}
\item{Collect a few branches from a conifer tree with both male and female cones. If no conifer trees are available, use dried specimens. If no specimens are available, skip this activity.}
\end{enumerate}

\subsubsection*{Activity Procedure}
\begin{enumerate}
\item{Observe branches of conifers, their leaves, and their cones. Then draw what is seen and identify the male and female cones.}
\item{Cut a longitudinal section of male and female cones using a razor blade.}
\item{Observe the internal parts of the male and female cones, drawing a diagram of each.}
\end{enumerate}

\subsubsection*{Results and Conclusion}
You should understand where conifers are found and how they reproduce. You should also be able to identify male and female cones and state the differences between them.

\subsubsection*{Clean Up Procedure}
\begin{enumerate}
\item{Place conifer specimens in a cool, dry place to use for future activities. No special waste disposal required.}
\end{enumerate}

\subsubsection*{Discussion Questions}
\begin{enumerate}
\item{What are the features of a conifer plant?}
\item{How are conifers adapted to their environment?}
\item{What are the differences between male and female cones?}
\item{What is the economic importance of Division Coniferophyta?}
\end{enumerate}

\subsubsection*{Notes}
Conifers may be difficult to find in some areas. Therefore, teachers should dry these specimens when they are found so they may be used repeatedly.


\section{Investigation of Division Angiospermophyta}

\subsection{Identification of the Reproductive Parts of the Flowers}
Angiosperms develop specialized structures called flowers used for reproduction. A flower is a modified part of a stem in which the primary sex organs are found. A flower has the following parts; peduncle, receptacle, calyx (sepal), corolla (petals). The male reproductive organ called the stamen consists of filament and anthers which form pollen grains. The female reproductive organs consisting of an ovary, style and stigma. Hibiscus flowers are bisexual because they have both male and female organs in one flower. Some pawpaw flowers have either female or male flowers and are referred to as unisexual flowers. Plants with both male and female flowers on the same plant are called monoecious like maize. Plants with male and female flowers on separate plants are dioecious like a pawpaw plant.

\subsubsection*{Learning Objectives}
\begin{itemize}
\item{To identify the reproductive parts of the flowers.}
\end{itemize}

\subsubsection*{Materials}
A variety of flowers like hibiscus, rose, or morning glory, razor blade, hand lens, cardboard box, and a labelled diagram of a flower

\subsubsection*{Hazards and Safety}
\begin{itemize}
\item{During collection of flowers from plants be careful with bees and you should not destroy other parts of the plant.}
\end{itemize}

\subsubsection*{Preparation Procedure}
\begin{enumerate}
\item{Collect a variety of flowers from a nearby garden.}
\item{Prepare cardboard mounts from empty boxes by cutting the cardboard into small squares using a razor blade.}
\end{enumerate}

\subsubsection*{Activity Procedure}
\begin{enumerate}
\item{Examine the external structures of the flowers collected, look for similarities and differences.}
\item{Draw and label the external structures of a hibiscus flower.}
\item{Cut a hibiscus flower in longitudinal section by starting the cut at the base and cutting along the carpel to the stigma.}
\item{Observe the flower using a hand lens, then draw and label the observed internal structures of the flower.}
\end{enumerate}

\begin{figure}[h]
\begin{center}
\def\svgwidth{9 cm}
\input{./img/hibiscus.pdf_tex}
\caption{The reproductive structures of a bisexual angiosperm.}
\label{fig:hibiscus}
\end{center}
\end{figure}

\subsubsection*{Results and Conclusion}
The female reproductive organs, the stigma, style, and ovary should be visible. You will also see the male reproductive parts which are the anthers and filament.

\subsubsection*{Clean Up Procedure}
\begin{enumerate}
\item{Collect all the used materials, cleaning and storing items that will be used later. No special waste disposal is required.}
\end{enumerate}

\subsubsection*{Discussion Questions}
\begin{enumerate}
\item{Is the flower that you examined male or female or bisexual?}
\item{Mention the reproductive parts of the flower and state their functions.}
\end{enumerate}

\subsubsection*{Notes}
Cutting the longitudinal section of the hibiscus flower should be done carefully to avoid destroying other floral parts. 
 
\subsection{Examination of Structures of Representative Dicotyledons and Monocotyledons}
Monocot and dicot plants are flowering plants and are found in division Angiospermatophyta. They differ in morphological structures from roots, stems, leaves, and flowers. Eg root size, leaf shape, floral parts, arrangement of vascular bundles and number of cotyledons in their seeds.

\subsubsection*{Learning Objectives}
\begin{itemize}
\item{To describe the structures of representative dicotyledons and monocotyledons.}
\end{itemize}

\subsubsection*{Materials}
Razor blade, maize grain, bean seed, petri dishes*, GV stain*, cardboard boxes, water drop microscope*, scalpel, monocot and dicot plants

\subsubsection*{Hazards and Safety}
\begin{itemize}
\item{Care must be taken when cutting the specimens as you may cut yourself.}
\end{itemize}

\subsubsection*{Preparation Procedure}
\begin{enumerate}
\item{From a nearby field or garden, collect dicot plants(hibiscus plant, bean plants, black jack plant)and monocot plants(grasses, maize plants, Commelina spp).}
\item{Place the plants into a beaker with a few drops of GV.}
\end{enumerate}

\subsubsection*{Activity Procedure}
\begin{enumerate}
\item{Observe the dicot and monocot plants from the external appearance by considering roots, leaves and flowers.}
\item{Record the features seen from each plant.}
\item{Cut a transverse section of a stem and roots of monocot and dicot that can be mounted on a slide and observe the arrangement of the vascular bundles on a water drop microscope.}
\item{Draw the vascular bundles as seen under water drop microscope( for monocot and dicot roots and stem)}
\item{Cut maize grain and bean seeds longitudinally to see how many cotyledons are in each specimen and draw them.}
\end{enumerate}

\subsubsection*{Results and Conclusion}
All of the common features of each class should be easily observed. Refer to the previous chapter for the characteristics of monocots and diocots.
\begin{figure}[h]
\begin{center}
\def\svgwidth{6cm}
\input{./img/dicot-root.pdf_tex}
\caption{Cross section of a dicot root}
\label{fig:diroot}
\end{center}
\end{figure}

\begin{figure}[h]
\begin{center}
\def\svgwidth{6cm}
\input{./img/dicot-stem.pdf_tex}
\caption{Cross section of a dicot stem}
\label{fig:fish}
\end{center}
\end{figure}

\begin{figure}[h]
\begin{center}
\def\svgwidth{6cm}
\input{./img/monocot-root.pdf_tex}
\caption{Cross section of a monocot}
\label{fig:monoroot}
\end{center}
\end{figure}

\begin{figure}[h]
\begin{center}
\def\svgwidth{6cm}
\input{./img/monocot-stem.pdf_tex}
\caption{Cross section of a stem}
\label{fig:monostem}
\end{center}
\end{figure}

\subsubsection*{Clean Up Procedure}
\begin{enumerate}
\item{Remove all waste materials from the bench.}
\end{enumerate}

\subsubsection*{Discussion Questions}
\begin{enumerate}
\item{With the aid of diagrams, differentiate monocots from dicots.}
\item{What is the economic importance of monocots and dicots?}
\item{Classify maize and bean plants.}
\end{enumerate}


\section{Investigation of Phylum Arthropoda}
There are five classes of Arthropods: Arachnida, Chilopoda, Crustacea, Diplopoda, and Insecta. All of these organisms have jointed appendages and a hard exoskeleton. This phylum has many different varieties of organisms, which have both positive and negative effects for the human race.  Some Arthropoda act as pollinators or a source of food, while others can cause humans pain and destroy crops.
This activity is a introduction to Phylum Arthropoda and is most effective when students collect the specimens themselves.  Classifying the organisms can also be done in small groups with minimal administration from the teacher.

\subsubsection*{Learning Objectives}
\begin{itemize}
\item{To explain the general features of Phylum Arthropoda.}
\item{To explain the distinctive features of each class in Phylum Arthropoda.}
\end{itemize}

\subsubsection*{Materials}
Variety of live and preserved specimens from Phylum Arthropoda, pictures of organisms in Phylum Arthropoda, bottle cages*, and petri dishes*

\subsubsection*{Specimens}
\textit{Instructions for collecting and preserving these specimens are in the previous chapter}
\begin{itemize}
\item{}
\end{itemize}

\subsubsection*{Hazards and Safety}
\begin{itemize}
\item{Be aware of dangerous organisms when collecting specimens in the field. For example, some arthropods may bite or sting - care should be taken when handling them. Avoid specimens known to be poisonous.}
\item{Preservatives like formalin are poisonous and should only be handled by the teacher in a well-ventilated room.}
\end{itemize}

\subsubsection*{Preparation Procedure}
\begin{enumerate}
\item{Put the specimens in bottle cages and petri dishes. Label each specimen with a marker pen. Also display any available pictures.}
\end{enumerate}

\subsubsection*{Activity Procedure}
\begin{enumerate}
\item{Observe the specimens.}
\item{Identify features common to all specimens.}
\item{Identify the distinctive features of each specimen.}
\item{Group the specimens into the five classes of Phylum Arthropoda.}
\item{Write down the general features of Phylum Arthropoda.}
\item{Draw and label a representative specimen from each class.}
\end{enumerate}

\begin{figure}[h]
\begin{center}
\def\svgwidth{12cm}
\input{./img/grasshopper.pdf_tex}
\caption{The external structures of a Grasshopper.}
\label{fig:fish}
\end{center}
\end{figure}

\subsubsection*{Results and Conclusion}
The organisms can be easily differentiated into the five classes by observing the number of legs and antennae. 

\subsubsection*{Clean Up Procedure}
\begin{enumerate}
\item{Collect and clean all the used materials, storing items that will be used later. No special waste disposal required.}
\end{enumerate}

\subsubsection*{Discussion Questions}
\begin{enumerate}
\item{What are the general features of Phylum Arthropoda?}
\item{Describe the habitat and feeding habits of each specimen.}
\item{Mention the classes of Phylum Arthropoda and the features of each.}
\item{What is the economic importances of Phylum Arthropoda?}
\end{enumerate}


\section{Investigation of Phylum Chordata}
Chordates are organisms in Kingdom Animalia with a vertebral column. There are six classes in this phylum; Reptiles, Amphibia, Mammalia, Aves, Chondrichthyes, and Osteichthyes. Such organisms differ in mode of reproduction, respiration, habitat, and structure.

\subsubsection*{Learning Objectives}
\begin{itemize}
\item{To explain distinctive features of each class of Phylum Chordata.}
\end{itemize}

\subsubsection*{Materials}
Live and preserved chordates, hand drawn pictures of chordates such as reptiles, mammals, aves, amphibians and fish, cardboard boxes, petri dishes*, bottle cages*, and traps.

\subsubsection*{Specimens}
\textit{Instructions for collecting and preserving these specimens are in the previous chapter}

\subsubsection*{Hazards and Safety}
\begin{itemize}
\item{When collecting specimens, take precaution with dangerous animals.}
\item{Preservatives like formalin maybe be poisonous and should only be handled by the teacher in a well-ventilated room.}
\end{itemize}

\subsubsection*{Preparation Procedure}
\begin{enumerate}
\item{Collect at least one specimen from each class of Phylum Chordata, for example a rat, fish, and worm. If you cannot find a specimen, photocopy the pictures from the previous chapter.}
\item{Collect live specimens by using traps and cut bottles from the field. Purchase preserved fish from the market.}
\item{Find or draw pictures showing chordates that can not be found in your area.}
\item{Put the specimens in bottles and petri dishes. Then label them using masking tape and a marker pen.}
\end{enumerate}

\subsubsection*{Activity Procedure}
\begin{enumerate}
\item{Display specimens for observation.}
\item{Identify the general and distinctive features of each specimen.}
\item{Group the specimens into their respective classes according to similarities and differences.}
\item{Draw and label a specimen from each class.}
\end{enumerate}

\subsubsection*{Results and Conclusion}
The tail of each organism should be easily observed, showing one of the main features of Phylum Chordata. Additionally, each organism will exhibit characteristics specific to their class. For more information about classification features, refer to the previous chapter.

\subsubsection*{Clean Up Procedure}
\begin{enumerate}
\item{Collect and clean all the used materials, storing items that will be used later. No special waste disposal required.}
\item{Return the specimens into their respective bottles/containers for preservation and future use.}
\end{enumerate}

\subsubsection*{Discussion Questions}
\begin{enumerate}
\item{What are the general features of Chordates?}
\item{Mention the classes of Kingdom Chordata and distinct features of each.}
\item{Draw and label a specimen from each class.}
\item{Discuss the economic importance of Class Osteichthyes and Aves.}
\end{enumerate}

\subsubsection*{Notes}
Some features may not be seen unless the organism is dissected.  Refer to the next activity for instructions on dissection.


\section{Dissection of a Rat}
The body contain different system which are vital to our daily life. They perform different important functions inside us. One of the systems is the digestive system. Our digestive system breaks down large food particles into pieces that are small enough to pass through the gut wall and dissolve into the blood. In order to see the digestive system inside the body and how it works we need to dissect a mammal.
Dissections may work well as a demonstration first by the teacher and then later performed by the students on their own in small groups.

\subsubsection*{Learning Objectives}
\begin{itemize}
\item{To identify parts of the mammalian digestive system and their adaptive features.}
\end{itemize}

\subsubsection*{Materials}
Rat, dissection tray*, knife, needles or office pins, clothes pin, razor blade, a bucket full of water, trap, bleach, tomato, charts/diagrams of human digestive system

\subsubsection*{Specimens}
\textit{Instructions for collecting and preserving these specimens are in the previous chapter}
\begin{itemize}
\item{rat}
\end{itemize}

\subsubsection*{Hazards and Safety}
\begin{itemize}
\item{Always cut away from yourself to avoid injury.}
\end{itemize}

\subsubsection*{Preparation Procedure}
\begin{enumerate}
\item{Buy a rat trap from market.}
\item{Cut a piece of tomato and put it inside the trap.}
\item{Put the trap in a place where there are rats overnight.}
\item{Prepare a dissecting tray.}
\item{Prepare a dissecting kit with a knife, razor blade, pins( from injection needles from pharmacy and thorns from acacia tree), and forceps from cloth pegs/wood.}
\item{After catching a rat in the trap, kill it by dipping the trap together with the rat in a bucket full of water for 5-10 minutes.}
\item{When the rat is dead, add 3 spoons of bleach to the bucket full of water and wait for 10 minutes so that the bleach kills the micro organisms.}
\end{enumerate}

\subsubsection*{Activity Procedure}
\begin{enumerate}
\item{Remove the rat from the bucket.}
\item{Lay the freshly killed rat on the dissecting board ventral side (abdomen) facing upward and pin it using needles or thorns from acacia plant.}
\item{Using a clothes pin, lift the skin of the abdomen and using a knife or a razor blade, make a longitudinal cut/slit at the centre of the abdomen.}
\item{Extend the cut with a knife up to the thoracic cavity and make an incision towards the limbs.}
\item{Using a piece of wood or clothes pin, separate and stretch the skin from the lower body wall.}
\item{Pin the folds of skin on the dissection tray using acacia thorns or needles.}
\item{Cut the body wall on either side of the mid line in order to observe different internal organs. Do not cut too deep otherwise you will damage the underlying organs.}
\item{Open the thoracic cavity by cutting through intercoastal muscles and rib cage.}
\item{Observe the internal digestive system clearly and use the provided chart/diagram of a human digestive system to compare the structures.}
\end{enumerate}

\begin{figure}[h]
\begin{center}
\def\svgwidth{10cm}
\input{./img/rat-dissection.pdf_tex}
\caption{A labelled diagram of a rat dissection}
\label{fig:rat-dissection}
\end{center}
\end{figure}

\subsubsection*{Results and Conclusion}
The digestive system of a rat is very similar to that of a humans. You will be able to observe the organs of the digestive system as well as other main organs like the lungs and heart.

\subsubsection*{Clean Up Procedure}
\begin{enumerate}
\item{Discard the dissected rat together with used pin and all unwanted materials into a pit latrine.}
\item{Clean the dissection tray with disinfectant.}
\end{enumerate}

\subsubsection*{Discussion Questions}
\begin{enumerate}
\item{Mention 5 different organs found in the digestive system and their function.}
\item{How is the digestive system of a rat similar to that of a human being?}
\end{enumerate}

\subsubsection*{Notes}
Rats may carry disease causing microorganisms. This is why it is important to place bleach into the water when drowning the rat.
