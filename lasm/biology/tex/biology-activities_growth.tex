\section{Growth}

\subsection{Investigating Conditions Necessary for Seed Germination}

\subsubsection*{Learning Objectives}
\begin{itemize}
\item{Students will be able to investigate conditions necessary for seed germination.}
\end{itemize}

\subsubsection*{Background Information}
Seed germination is the development of a seed into a seedling. The changes which occur during seed germination include absorption of water through micropyle, bursting of testa, and emerging and elongation of radicle. Then seed coat is discarded and cotyledons open out and begin to photosynthetize. Between the cotyledons, the plumule emerges and produces leaves. At this point, the young plant is called a seedling. The conditions necessary for seed germination include water, oxygen and optimum temperature.

\subsubsection*{Materials}
Bean seeds, water, cotton wool, rubber stopper*, 4 test tubes*, cooking oil*, ice water, and boiled water

\subsubsection*{Preparation Procedure}
\begin{enumerate}
\item{Prepare 4 test tubes from syringes and label them 1, 2, 3, and 4.}
\item{Buy ice and prepare hot water.}
\end{enumerate}

\subsubsection*{Activity Procedure}
\begin{enumerate}
\item{Place cotton wool at the bottom of each test tube.}
\item{Add a few seeds to each of the test tubes.}
\item{In test tube 1, add enough water to soak the cotton wool.}
\item{In test tube 2, add cool boiled water to flood the seeds; add small amount of oil to form a layer above the water.}
\item{In test tube 3, add ice water.}
\item{In test tube 4, do not add any water.}
\item{Keep the test tubes under these conditions for four days.}
\item{Record your observations every day and add water as needed to each tube, being sure to add only ice water to test tube 3.}
\end{enumerate}

\subsubsection*{Results and Conclusion}
Only test tube 1 will show proper germination. The other test tubes will show little to no growth because they do not have the conditions necessary for germination.

\subsubsection*{Clean Up Procedure}
\begin{enumerate}
\item{Collect all the used materials, cleaning and storing items that will be used later. No special waste disposal is required.}
\end{enumerate}

\subsubsection*{Discussion Questions}
\begin{enumerate}
\item{Why does test tube 2 use boiled water?}
\item{What conditions are missing from test tubes 3 and 4?}
\item{Describe the changes that occur during germination of bean seed.}
\end{enumerate}

\subsubsection*{Notes}
When planting seeds we must make sureo that they recieve oxygen, water, and proper temperature for germination.


\subsection{Demonstration of Epigeal and Hypogeal Germination.}

\subsubsection*{Learning Objectives}
\begin{itemize}
\item{Students will be able to demonstrate epigeal and hypogeal germination.}
\end{itemize}

\subsubsection*{Background Information}
Seed germination is a development of seed into a seedling. There are two types of germination which are epigeal and hypogeal. Different seeds germinate differently depending on their classification groups.  Monocotyledons leaves their cotyledons underground, this is called hypogeal germination. In dicotyledons the cotyledon emerges above the soil, a process called epigeal germination.

\subsubsection*{Materials}
Bean seeds, maize seeds, pots for sowing*, soil, and water

\subsubsection*{Preparation Procedure}
\begin{enumerate}
\item{Prepare 2 pots with soil for sowing the seeds.}
\item{Collect maize and beans seeds.}
\item{Direct students to sow a few maize and bean seeds in two separate pots.}
\end{enumerate}

\subsubsection*{Activity Procedure}
\begin{enumerate}
\item{Observe the seedlings as they emerge from the soil.}
\item{Draw and label diagrams of maize and bean seedlings and classify as epigeal or hypogeal.}
\end{enumerate}

\begin{figure}[h]
\begin{center}
\def\svgwidth{6cm}
\input{./img/seed-germ-1.pdf_tex}
\caption{Hypogeal germination}
\label{fig:hypogeal}
\end{center}
\end{figure}

\begin{figure}[h]
\begin{center}
\def\svgwidth{7cm}
\input{./img/seed-germ-2.pdf_tex}
\caption{Epigeal germination}
\label{fig:epigeal}
\end{center}
\end{figure}

\subsubsection*{Results and Conclusion}
In epigeal germination cotyledons are carried above the soil like the germination of bean seed (dicotyledonous seeds). In hypogeal germination cotyledons remains underground like the germination of maize seed(monocotyledonous seed).

\subsubsection*{Discussion Questions}
\begin{enumerate}
\item{Define hypogeal and epigeal germination.}
\item{Distinguish between the germination of maize and bean seeds.}
\item{Draw and label maize seedling and bean seedling, identify the roots, shoots, and codyledons.}
\end{enumerate}

\subsubsection*{Notes}
