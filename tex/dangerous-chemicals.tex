\chapter{Dangerous Chemicals}
\label{cha:dangerchem}
%==============================================================================
\section{Chemicals that should never be used in a school}

%------------------------------------------------------------------------------
\subsection{Mercury and its compounds 
(e.g. Million's Reagent, Nestler's Reagent)}

\begin{itemize}

\item{Hazard: Toxic}

\item{Route: Ingestion of solutions and salts; 
inhalation of vapors from the liquid metal. 
Mercury has a very low vapor pressure, 
but the vapors that do form are quite poisonous – 
inhalation is therefore a significant hazard.}

\item{Use: Showing off to students, 
Million's reagent for biology (no longer used)}

\item{Alternatives: Use the biuret test to detect proteins 
(1~M \ce{NaOH} followed by 1\% \ce{CuSO4}, a purple color is a positive result)}

\item{Precautions if it needs handling (e.g. broken thermometers): 
Wear gloves or plastic bags on the hands.}

\item{First aid: If ingested, induce vomiting at once. 
Administer activated charcoal. Seek medical attention.}

\item{Disposal: If you find mercury or its compounds, 
keep them in sealed in a bottle and locked away. 
Label the bottle very clearly “POISON, DO NOT OPEN, DO NOT DUMP” 
and also include a strong warming in the local language(s). 
If you spill liquid mercury, 
ask everyone to leave and apply powdered sulfur immediately. 
Put on gloves and tie a cloth on your face. 
Open windows to increase ventilation. 
Then use pieces of cardboard to gather the mercury back together 
so you can seal it in a bottle. 
Apply powdered sulfur to any mercury that cannot be reached -- 
e.g. cracks in the floor.}

\end{itemize}

%------------------------------------------------------------------------------
\subsection{Benzene}

\begin{itemize}

\item{Proven carcinogen, toxic. 
A horrific and generally fatal form of cancer 
is associated with benzene exposure, 
with tumors appearing rapidly throughout body.}

\item{Route: Can be fatal if ingested, 
especially if aspirated into the lungs 
(e.g. if mouth pipetting); also passes through skin(!)}

\item{Use: Multi-purpose non-polar solvent. 
Less dense than water.}

\item{Alternative: kerosene}

\item{Precautions if it needs handling: Thick rubber gloves. 
It will pass rapidly through latex.}

\item{Disposal: If you find a bottle of benzene, 
leave it sealed and in a secure place with a stern warning label. 
If a bottle breaks, 
evacuate the room and return only wearing a cloth over your face 
and thick rubber gloves. 
Absorb the benzene with cardboard, cotton wool, 
saw dust, rice hulls, or flour, 
transfer the mass to a dry place outside, 
add a significant amount of kerosene and burn completely. 
Benzene will combust on its own, 
but you want to make sure it burns hot enough 
that none simply vaporizes without combusting.}

\end{itemize}

%------------------------------------------------------------------------------
\subsection{Tetrachloromethane (carbon tetrachloride)}

\begin{itemize}

\item{Hazard: Proven carcinogen. 
The chemical has killed students in both Tanzania and Kenya.}

\item{Route: Ingestion can be fatal. 
Will pass through skin. 
Inhalation of vapors is quite dangerous.}

\item{Use: Multi-purpose non-polar solvent. 
More dense than water.}

\item{Trichloromethane (chloroform) is 
another non-polar solvent more dense than water, 
though still dangerous (listed in category two, below). 
If the density is not important, use kerosene. 
If the solvent must not be flammable, 
consider a different experiment.}

\item{Precautions if it needs handling: Thick rubber gloves. 
It will pass rapidly through latex.}

\item{First Aid: Seek medical attention immediately. 
Ask a medical expert if you should induce vomiting 
(the chemical can kill if absorbed through the stomach, 
but also if aspirated into the lungs when vomiting)}

\item{Disposal: If you find a bottle of carbon tetrachloride, 
leave it sealed and in a secure place. 
If a bottle breaks, 
absorb the chemical with cotton wool 
and move the cotton to a place where it can off-gas 
away from people and other living things. 
Protect from rain and from leaching into the ground. 
Once the cotton is completely dry, 
douse with kerosene and burn it. 
Do not burn the chemical directly – 
it used to be used in some fire extinguishers. 
On heating, it decomposes to release poisonous gases.}

\end{itemize}

%------------------------------------------------------------------------------
\subsection{Other hazardous organic solvents}
The following chemicals have hazards similar to 
if less severe than benzene and tetrachloromethane. 
None should ever be used in a school. 
Leave them sealed in their bottles. 
If a bottle breaks, follow the instructions 
listed with tetrachloromethane.

\begin{itemize}

\item{Chlorobenzene}

\item{Dichloromethane}

\item{Trichloroethane}

\end{itemize}

%==============================================================================
\section{Dangerous chemicals that you might need to use}

%------------------------------------------------------------------------------
\subsection{Ammonia (ammonium hydroxide solution)}

\begin{itemize}

\item{Hazard: The liquid burns skin, the fumes burn lungs, 
and reaction with bleach or any combination of chloride and oxidizer 
can form toxic chloroamine fumes.}

\item{Use: Common bench reagent.}

\item{Alternative: For a simple weak base, 
use carbonate or hydrogen carbonate.}

\item{Precaution: Strongly prohibit mixing of different bench reagents. 
Neutralize waste completely before disposal. 
When pouring ammonia for distribution, 
wear cloth over your mouth and nose and work outside, upwind. 
To smell, waft carefully – never inhale ammonia directly from a bottle!}

\item{First Aid: In case of skin contact, 
wash with plenty of water followed by a dilute weak acid 
(vinegar or dilute citric acid) and more water. 
In case of eye contact, wash with water for at least ten minutes. 
If ingested, do NOT induce vomiting. 
In case of inhalation, move victim to fresh air. 
Seek medical attention if the victim does not recover quickly.}

\item{Disposal: Save unused solution for another day. 
If you must dispose of it, add to several liters of water 
and leave in an open bucket in the sun, 
away from people and animals. 
The ammonia will evaporate, leaving water behind. 
The process is finished when the bucket no longer smells like ammonia.}

\end{itemize}

%------------------------------------------------------------------------------
\subsection{Barium}

\begin{itemize}

\item{Hazard: Very poisonous if ingested in a soluble form 
(e.g. barium chloride, hydroxide, or nitrate). 
Note that barium carbonate will dissolve very quickly in stomach acid.}

\item{Use: Preparation of hydrogen peroxide, 
test for sulfates, flame tests.}

\item{Alternatives: hydrogen peroxide is often sold in pharmacies, 
sulfates may be confirmed with soluble lead salts (also poisonous!), 
and boron compounds (e.g. boric acid, borax) 
also produce a green flame color.} 

\item{Precautions: Distribute only in small quantities 
in bottles clearly labeled “POISON.” 
Also use the local word for poison, e.g. SUMU in Swahili. 
Collect all barium waste in a special container. 
This will require training students. 
Have a bottle of magnesium sulfate 
or sodium sulfate available for spills on skin or tables. 
Sodium sulfate can be prepared by neutralizing dilute sulfuric acid 
with sodium bicarbonate – err on the side of excess bicarbonate. 
See Sources of Chemicals for magnesium sulfate.}

\item{First Aid: If ingested, induce vomiting 
and administer activated charcoal if available. 
Go to the hospital. 
The material safety data sheet for barium chloride 
recommends use of sodium or magnesium sulfate under a doctor's direction. 
Chemically, this would precipitate barium sulfate, 
preventing absorption of the element. 
Magnesium sulfate is non-toxic, though will probably cause diarrhea.}

\item{Disposal: Collect unused solutions for another day. 
Collect all waste in a large container 
and add dilute sulfuric acid until precipitation stops. 
Pour off most of the liquid and treat it as dilute acid waste. 
Use the remaining liquid to send the precipitate 
to the bottom of a pit latrine.}

\end{itemize}

%------------------------------------------------------------------------------
\subsection{Chloroform (Trichloromethane)}

\begin{itemize}

\item{Hazard: used to render mammalian specimens unconscious; 
it has the same effect on humans. 
Also toxic in ingested. Passes through skin.}

\item{Use: Knocking out dissection specimens, 
sometimes as a specialty non-polar solvent.}

\item{Alternatives: Dissect dead specimens; 
use safer solvents.}

\item{Precautions: Work in a well-ventilated space, like outside. 
NEVER, EVER MOUTH PIPETTE!}

\item{First Aid: If ingested, go to the hospital. 
Do not induce vomiting unless instructed by a medical professional. 
If inhaled, immediately remove the victim to fresh air and sit 
(but not lie) him or her down in case of fainting. 
If the victim loses consciousness, go to the hospital. 
In both cases, monitor breathing and pulse. 
In case of skin contact, wash off immediately, 
and use soap as soon as it is available.}

\item{Disposal: For the small amounts used 
in preparing specimens for dissection, 
allow the chemical to evaporate away from people and animals. 
For large amounts, e.g. if a bottle spills or breaks, 
evacuate the room and keep everyone away for at least one day. 
Return carefully, allowing more time 
if the room still smells like trichloromethane.}

\end{itemize}

%------------------------------------------------------------------------------
\subsection{Concentrated Acids 
(sulfuric, hydrochloric, nitric, ethanoic (acetic))}

\begin{itemize}

\item{Hazard: Serious skin burns, will blind in the eyes.}

\item{Use: Often the starting material 
when preparing dilute acids for titrations or food tests. 
Also used directly in small quantities in chemical qualitative analysis.}

\item{Alternatives: If any acid will suffice, use a safer weak acid, 
e.g. citric acid (best) or ethanoic (acetic) acid. 
If a dilute strong acid is required, 
use battery acid as a starting source of sulfuric acid. 
Note that many experiments calling for dilute hydrochloric acid 
work just as well with dilute sulfuric acid. 
Battery acid will also work for many experiments 
calling for “concentrated” sulfuric acid – 
indeed it is about 5~M – but will not work if one requires 
the dehydrating action of concentrated sulfuric acid. 
For such cases, consider other experiments. 
Note that battery acid is still quite dangerous – 
it will burn holes in clothing and blind in the eyes.}

\item{Precautions: Always have a full bucket of water 
and at least half a liter of sodium bicarbonate 
or other weak base solution available. 
Use thick rubber gloves and wear goggles. 
If you do not have these in your lab, go buy them. 
Whenever handling battery acid, wear goggles. 
Keep other people away when pouring concentrated acids. 
If you are using either concentrated hydrochloric, ethanoic (acetic), 
or nitric acid, work outside and stand upwind -- the fumes corrode the lungs. 
Wear cloth over your mouth and nose. 
If a bottle ever falls and breaks, 
calmly but clearly ask everyone to stop working and leave the room. 
Keep everyone upwind while the fumes blow away. 
Most of the acid will be consumed by reacting with cement. 
If the damage is significant, a building engineer should inspect the structure. 
Always pour acid into water when diluting. 
The heat of solvation of sulfuric acid especially is so exothermic 
that it can cause water to boil. 
If a small quantity of water is added to concentrated acid, 
it can boil so vigorously that it will cause acid to splash 
out of the container, on skin or into eyes. 
Finally, pour slowly from the bottle, 
always allowing air to enter as you pour. 
Otherwise, air will enter in sudden amounts, 
causing acid to exit the same way. 
This can cause it to splash back up at you.}

\item{First Aid: For skin burns, promptly wash the affected area 
with a large amount of water. 
Then liberally apply a sodium bicarbonate 
or other weak base solution to the affected area. 
Then wash again with a large amount of water. 
Repeat until the burning sensation is gone. 
If the chemical ever gets in the eye, 
immediately apply sodium hydrogen carbonate solution 
to neutralize the acid in the eye, but nothing stronger -- 
not carbonates and definitely not hydroxides. 
Then wash continuously with large amounts of water for ten minutes, 
longer if the eye still burns. 
Seek medical immediately. 
If swallowed, do not induce vomiting -- 
the damage is done on the way down.}

\item{Disposal: Add the concentrated acid to twenty or more 
times its volume of water and then add ash or baking soda 
until the mixture stops fizzing. 
The gas produced is carbon dioxide. 
Note that containers used to measure or hold concentrated acids 
often have enough residual acid to be dangerous. 
They should be submerged in a large container of water following use.}

\end{itemize}

%------------------------------------------------------------------------------
\subsection{Diethyl Ether (ethoxyethane)}

\begin{itemize}

\item{Hazard: Can be fatal if aspirated into lungs. 
Also extremely flammable and a significant flash fire hazard. 
May also cause unconsciousness on inhalation.}

\item{Use: Non-polar solvent}

\item{Alternative: for a non-polar solvent, use kerosene. 
For a more volatile solvent, use paint thinner or lighter fluid. 
To demonstrate a rapidly evaporating substance, 
use propanone, ethyl ethanoate, or iso-propanol - 
note that all are also extremely flammable.}

\item{Precautions: Never use alone (in general, do not use the lab alone). 
Distribute in bottles with lids 
or in beakers covered with e.g. cardboard to prevent evaporation. 
Under no circumstances should an open container of diethyl ether 
be in the same room as open flame. 
Only use in well ventilated spaces and encourage students to go outside 
if they feel at all drowsy or unwell.}

\item{Disposal: See instructions on recycling of organic solvents 
to minimize the need for disposal. 
For what cannot be recovered, 
place where it can evaporate without being disturbed 
and without anyone downwind.}

\end{itemize}

%------------------------------------------------------------------------------
\subsection{Ethandioic acid (oxalic acid), 
sodium/ammonium ethandioate (oxalate)}

\begin{itemize}

\item{Hazard: Poison}

\item{Use: Volumetric analysis, 
oxidation-reduction reactions, qualitative analysis}

\item{Alternatives: For its weak acid properties, 
use citric acid (best) or ethanoic (acetic) acid. 
For its reducing properties, use ascorbic acid or sodium thiosulfate.}

\item{First aid: If ingested, induce vomiting 
and administer activated charcoal. Go to the hospital.}

\item{Disposal: Collect unused solutions for another day. 
To dispose, add potassium permanganate solution slowly 
until the ethandioic acid / ethandioate lacks the power to decolorize it. 
At this point the compound should have been fully converted to carbon dioxide. 
If you used far excess oxidizing agent, 
neutralize with ascorbic acid prior to disposal.}

\end{itemize}

%------------------------------------------------------------------------------
\subsection{Lead}

\begin{itemize}

\item{Hazard: Poisonous, toxic to many organs including the brain.}

\item{Use: Unknown salt for qualitative analysis. 
Thus students must treat ALL unknown salts as potential lead compounds.}

\item{Precautions: Unequivocally prohibit taste-testing of unknown salts. 
This seems obvious. 
Unfortunately, to many students it is not. 
Explain the hazard clearly -- 
there are salts in the lab (e.g. barium compounds) 
where even a small taste can kill. 
Also, make sure students wash their hands.}

\item{First Aid: If ingested, 
induce vomiting and administer activated charcoal.}

\item{Disposal: Collect unused solids for another day. 
If the salt is soluble, 
dissolve all waste in a large container 
and add sodium chloride solution until precipitation stops. 
Send the precipitate to the bottom of a pit latrine. 
If the salt is already insoluble, drop it down.}

\end{itemize}

%------------------------------------------------------------------------------
\subsection{Potassium hexacyanoferrate (potassium ferr[i/o]cyanide)}

\begin{itemize}

\item{Hazard: Reaction with concentrated acid releases hydrogen cyanide, 
the agent used in American gas chambers for executions. 
On inhalation, the cyanide enters the blood stream 
and binds cytochrome-c oxidase with a higher affinity than oxygen. 
Cellular respiration halts and tissues slowly die.}

\item{Use: Qualitative analysis bench reagent.}

\item{Precautions: Strongly prohibit mixing of different bench reagents. 
There are plenty of other dangerous combinations.}

\item{First Aid: If a student seems to have trouble breathing, 
bring him/her outside immediately. 
If breathing remains difficult, seek medical attention. 
If the chemical is ingested, induce vomiting.}

\item{Disposal: Dilute with plenty of water and send down the pipe. 
Make sure all acid waste is also diluted and neutralized.}

\end{itemize}

%------------------------------------------------------------------------------
\subsection{Sodium hydroxide (and potassium hydroxide)}

\begin{itemize}

\item{Hazard: Concentrated solutions corrode metal, 
blacken wood, and burn skin. 
Even solutions as dilute as 0.1~M can blind if they get in the eyes. 
Note that this is a common concentration for volumetric analysis. 
Also note that the dissolution of sodium and potassium hydroxide 
are highly exothermic -- rapid addition, especially to acidic solutions, 
can cause boiling and splatter. 
Finally, the salts are highly deliquescent 
and often turn to liquid if containers are not well sealed. 
This liquid is maximally concentrated hydroxide -- 
the most dangerous form; do not dispose without neutralization.}

\item{Use: Volumetric analysis, food tests, 
qualitative analysis bench reagent.}

\item{Precautions: Use weak bases (carbonates and hydrogen carbonates) 
for volumetric analysis, provide students with goggles.}

\item{First Aid: Treat spills and skin burns 
with a dilute solution of a weak base -- ethanoic (acetic) or citric acid. 
If it gets in the eyes, immediately wash with a large amount of water. 
Continue washing for at least five minutes 
and seek medical attention if the eye still hurts.}

\item{Disposal: Save for future use. 
To dispose, neutralize with citric acid or other acid waste and dump.}

\end{itemize}

%==============================================================================
\section{Chemicals that merit warning}

%------------------------------------------------------------------------------
\subsection{Ammonium nitrate}
Can explode (and shatter glassware, sending shards into eyes) if heated. 
Otherwise as innocuous as any other inorganic fertilizer.

%------------------------------------------------------------------------------
\subsection{Ethanol}
The vapors are flammable, 
so ethanol should never be heated directly on a stove. 
If it must be warmed, it should be heated in a hot water bath. 
If the ethanol ignites anyway, do not panic. 
Cover the top of the ethanol container and smother the flame. 
Please note that methylated spirits have chemical additives that are poisonous, 
causing blindness, etc. 
Also, alcohol prepared for laboratory or industrial use 
is sometimes purified by extraction with benzene 
and probably contains traces of this potent carcinogen. 
Do not even think about drinking it.

%------------------------------------------------------------------------------
\subsection{Ethyl acetate/ethyl ethanoate}

\begin{itemize}

\item{Hazard: Extremely flammable}

\item{Use: Solvent}

\item{Precautions: Never open a bottle in the same room as an open flame.}

\item{Disposal: Save for use as a solvent. 
If you must dispose, allow to evaporate away from people and fire.}

\end{itemize}

%------------------------------------------------------------------------------
\subsection{Potassium permanganate}
Powerful oxidizing agent. 
Do not mix with random substances. 
If you are trying something for the first time, use small quantities. 
Concentrated solutions and the crystals themselves will discolor skin, 
though the effect lasts only a few hours. 
This is the same stuff they sell in the pharmacies 
to prevent infections of cuts and surface wounds. 
Do not eat!

%------------------------------------------------------------------------------
\subsection{Propanone (acetone)}

\begin{itemize}

\item{Hazard: Extremely flammable}

\item{Use: Solvent}

\item{Precautions: Never open a bottle in the same room as an open flame.}

\item{Disposal: Save for use as a solvent. 
If you must dispose, allow to evaporate away from people and fire.}

\end{itemize}
