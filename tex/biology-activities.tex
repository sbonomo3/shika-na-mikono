\chapter{Biology Activities}

\section{Anatomy}

\subsection{Animal Dissection}
\begin{itemize}
\item{Preparation time: 30 minutes}
\item{Materials: razor blades or scalpel, animal parts, wood board, pins}
\item{Procedure: Find a local butcher and ask for intact eyes, hearts, lungs, liver, and other parts. If you have an iron stomach, purchase a goat yourself and take it apart. The meat can be eaten, the organs studied, and the bones buried for a month and then exhumed to give a complete skeleton. Use a razor blade or a scalpel to dissect each of the different parts. To make a scalpel, take a small piece wood, stick, plastic, or anything that will not break under a little pressure, and line up a razor blade at the end and tape the blade. Make sure to tape all the way around the bottom part of the razor blade to secure it properly to the handle. Dissect on a wood board and use pins from tailor to hold back different tissues.}
\item{Theory: By using fresh anatomical parts from animals, students can get the full experience of each of these organs. Be sure to have students to look into the differences between organs in terms of size, tissues, textures, colors, and more.}
\end{itemize}

\subsection{Volume of the Lungs}
\begin{itemize}
\item{Preparation time: 20 minutes}
\item{Materials: I.V. line, water, large container, if available a large graduated cylinder}
\item{Procedure: Fill a shallow container with water. Fill a small plastic water bottle to the full with water. Quickly invert the water bottle or cylinder into the shallow dish so that the mouth is in the water of the shallow dish. This prevents the water in the water bottle or the cylinder from coming out. Insert an IV line in the cylinder or water bottle. Have a student blow all the air in their lungs through the IV line into the water bottle or cylinder. Record the volume of air in each student’s lungs.}
\item{Theory: It is easy to talk about a human’s breath. In this activity, the students put a value with their breath. As the lungs fill with air, their volume expands. Through displacement of volume, we can find the volume of each student’s lungs. Make a table for each class listing all the volumes of each student. Find the average volume for each student.}
\end{itemize}

\subsection{Fingerprints, Part A – Identification}
\begin{itemize}
\item{Preparation time: 10 minutes}
\item{Materials: Ink pad, white paper }
\item{Procedure: One a piece of white paper, have students make a table with 10 boxes: one for each finger. Using an inkpad, have students carefully cover their fingers with ink. Press each of their fingers one at a time on the paper without smudging the prints. Roll the fingers without moving them in order to create a good print. Have students compare prints with each other (they should notice differences).}
\item{Theory: Each person’s fingerprints are different. It is one of the unique features that can distinguish between people. This is one of the more common techniques used in finding the culprits in crimes.}
\end{itemize}

\subsection{Fingerprints, Part B – Soot Prints}
\begin{itemize}
\item{Preparation time: 10 minutes}
\item{Materials: Small pieces of glass, candles, white paper}
\item{Procedure: If there are no inkpads on hand, fingerprints can be made from the soot of candles. Take a piece of glass and hold it above the candle flame until it becomes black with soot. Have a student carefully press their finger into the soot. Then, press the print on white paper.}
\item{Theory: The soot from candles also works for making fingerprints. This method is more difficult and easier to smudge than ink.}
\end{itemize}

\subsection{Human Symmetry}
\begin{itemize}
\item{Preparation time: 10 minutes}
\item{Materials: mirrors, students}
\item{Procedure: Give students mirrors and use them to identify symmetry of their faces. Have one student put the mirror down half of their face, and see if the other students can see the symmetry.}
\item{Theory: The human body is interestingly symmetric. The easiest place to see symmetry is on the face. This symmetric line runs down from the middle of the face all the way down the body.}
\end{itemize}

\subsection{Peristalsis, Part A – Upside Down Eating}
\begin{itemize}
\item{Preparation time: 5 minutes}
\item{Materials: bread, students, wall, kanga}
\item{Procedure: Have a student stand on their head or hands with their feet on the wall. Be mindful that this could be a problem for students wearing skirts - have a kanga available for students with skirts to wrap around their skirts.]. With a student upside down, give them a small piece of bread and let them chew and swallow.}
\item{Theory: Food does not move through the human body by gravity alone. While gravity may assist, it is not required in order to move food from the mouth to the stomach. Lining the sides esophagus are muscles. These muscles contract to move food along the throat from the mouth to the stomach. These muscles can even move food to the stomach when it has to go up to arrive!}
\end{itemize}

\subsection{Peristalsis, Part B – The Movement}
\begin{itemize}
\item{Preparation time: 10 minutes}
\item{Materials: 1 sock or tire tube, 1 small ball}
\item{Procedure: Hold the sock or tube vertically. Put the ball in the bottom opening. Use your hands to close just underneath the ball to push it farther up the sock. Slowly repeat this motion to move the ball up through the sock or tube.}
\item{Theory: This follow up activity shows what is happening inside the esophagus. The hand motion illuminates the work the muscles are doing in order to move to the stomach. Note that this is the same motion that works throughout the digestive system.}
\end{itemize}

\subsection{Pulse}
\begin{itemize}
\item{Preparation time: 0 minutes}
\item{Materials: students, stopwatch, rubber snakes}
\item{Procedure: Have a student sit and record their pulse. Have a student walk around the school, and record his or her pulse on return. Have a student run for a few minutes and then record his or her pulse. Lastly, scare a student and record his or her pulse. A rubber snake is highly recommended.}
\item{Theory: The heart pumps blood to distribute oxygen to the body. When the body needs more oxygen, the heart begins to pump faster. This is shown by comparing the pulse rate between the different activities: sitting, walking, and running. Sitting requires little oxygen, while walking needs more, and running need even more.The body also pumps more blood depending upon the possible need of oxygen. When humans get scared, the body enters a fight or flight response. In order to run away or fight, the body will need more oxygen. Therefore, when a person becomes scared, the pulse will rise since the body thinks it will need more oxygen.}
\end{itemize}

\section{Cells}
\subsection{Rotting Food}
\begin{itemize}
\item{Preparation time: 5 minutes}
\item{Materials: 2 tomatoes, 1 plastic container with well fitting lid}
\item{Procedure: Take 1 tomato and seal inside a plastic container. Leave a second tomato next to the container. Wait for both to rot. Observe the smell, the color, and texture of the bacteria and fungi growing on the tomatoes.}
\item{Theory: In the air, there are different bacteria and fungal spores. Differences in air circulation, moisture, and temperature favor different organisms – the two tomatoes should look pretty different. In this activity, the tomato in the sealed container will rot faster and possibly have different colors or odors from the decomposition of the fruit.}
\end{itemize}


\subsection{Yeast Fun, Part A}
\begin{itemize}
\item{Preparation time: 5 minutes}
\item{Materials: 1 sealed syringe shell or test tube, balloon, sugar, water}
\item{Procedure: Pour about 5 mL of water in the syringe shell / test tube. Add 5 grams of sugars and 1 gram of yeast. Cover the tube with the balloon. After a few hours, the balloon should fill with carbon dioxide, and the contents of the tube should smell like alcohol}
\item{Theory: Yeast is actually a fungus. This organism eats sugar and breaks it down into alcohols and carbon dioxide. In this activity, this living organism feed on sugar and produce carbon dioxide and alcohol. This process is the fundamental process behind making beers, wines, and other spirits.}
\end{itemize}


\subsection{Yeast Fun, Part B - Temperature}
\begin{itemize}
\item{Preparation time: 10 minutes}
\item{Materials: 3 sealed syringe shells, 3 balloons, sugar, water}
\item{Procedure: Take 1 syringe shell and follow the procedure from Yeast Fun, Part A. Fill a second and third syringe in the similar way. However, heat the second and the third in a water bath. Remove the second one around 50 C, or until it is too hot to keep your hand in the water. Boil the third syringe. Finally, put balloons on all syringes.}
\item{Theory: Biological organisms tend to be vulnerable to temperature changes. Most organisms cannot live in temperatures where the water boils. This is the reason we need to boil water: to kill all bacteria and fungus in water. It is not enough to heat water, but boil it completely; we are demonstrating this fact in this activity. The syringe left at room temperature slowly fills with gas but the syringe that was heated part way should accelerate the work of the yeast. If the yeast is more active at higher temperature, you will see the he balloon on the second syringe fill with gas faster than the room temperature one. However, the syringe that was boiled should produce no carbon dioxide. This is because the temperature of the boiling water was too high and killed the yeast.}
\end{itemize}


\subsection{Yeast Fun, Part C - Food}
\begin{itemize}
\item{Preparation time: 10 minutes}
\item{Materials: 2 sealed syringe shells or test tubes, water, sugar, salt, balloons, yeast}
\item{Procedure: Place sugar, water, and yeast in a test tube. The second test tube, use only salt, yeast, and water. Cap both with a balloon.}
\item{Theory: The proper food is necessary for organisms to grow. The wrong food prevents, and possibly kills organism. The yeast’s normal food is sugars or starches. Salt and sugar are very different. In fact, while the syringe with sugar will produce carbon dioxide as usual, the syringe with salt produces no carbon dioxide because there is no food for the yeast to eat.}
\end{itemize}


\subsection{Yeast Fun, Part D - Light}
\begin{itemize}
\item{Preparation time: 10 minutes}
\item{Materials: 2 sealed syringe shells, water, sugar, yeast, balloons}
\item{Procedure: Fill both syringes similarly as before in Yeast Fun, Part A. Cap both with balloons. Place 1 syringe shell in the sun and place the other in box or dark cabinet.}
\item{Theory: Some organisms require light in order to thrive. Many green leafy plants are a good example. Yeast however does not require sunlight to eat sugar. In this activity, both balloons will fill with carbon dioxide.}
\end{itemize}


\subsection{Yeast Fun, Part E - Living Environment}
\begin{itemize}
\item{Preparation time: 10 minutes}
\item{Materials:  3 sealed syringe shells, sugar, yeast, balloon, methylated spirits, cooking oil.}
\item{Procedure: Prepare one syringe as before in Yeast Fun, Part A. Follow the procedure two more times with variations: instead of water, use methylated spirits in one and cooking oil in the other. Observe the size of the balloon after a few hours. }
\item{Theory: The environments that microorganisms live in are very important. Some organisms have a small threshold for change in the environment. Yeast is a rather tolerant organism and will endure through many different changes in its environment. In this demonstration, the first syringe acts like a control for use in comparing with other variations. The second syringe changes water for methylated spirits. Methylated spirits is a mixture of 70 \% by volume of ethanol to water. Ethanol is a product of yeast fermentation, however at certain concentrations it actually becomes lethal to yeast. This is around 15\% by volume. This second syringe should have a much smaller balloon. This is not due to the yeast formation, but ethanol evaporating and filling the balloon. The last syringe will also have a small balloon or none at all. This is because the oil is not a good medium for yeast to thrive. The sugar becomes bound together in the oil layer while the yeast does not survive well in it. Therefore, little or no fermentation can occur.}
\end{itemize}

\section{Classification}

\subsection{Scavenger Hunt}
\begin{itemize}
\item{Preparation time: 5 minutes}
\item{Materials: --}
\item{Procedure: Prior lessons about classification, find different animals, plants, fungi, or more that is available around the school. Look especially for mosses (wet places) and fungi (decaying material in the shade). After teaching students about classification, send them to find different specimen around the school. If they do not find something you know is there, give them hints about where to look. They will also probably find specimens you have not seen. You could even ask them to bring specimens from near their homes to class the next day. When the students return with their specimens, have them classify what has been found. Keep the best specimens for the school collection.}
\item{Theory: One of the biggest problems about learning classification is that students rarely use the process, and often do not connect it do their daily lives. A scavenger hunt to find examples to classify is a very effective way to encourage students to utilize and retain information about classification. Further, encourage discussion and debate if there are any doubts as students work on classification. Encourage students to think about their specimen, and then have them defend their rationale.}
\end{itemize}


\subsection{Leaf Game}
\begin{itemize}
\item{Preparation time: 20 minutes}
\item{Materials: leaves from around the school}
\item{Procedure: Send students to collect three leaves from each tree around the school. Place them in a pile and let students observe the differences in size, color, texture, and more. Mix up the pile, and time the time it takes students to separate all the leaves. Repeat this experiment, but separate them on based on certain traits, like size or color or texture.}
\item{Theory: Leaves provide an interesting opportunity for students to experience the act of classification. By collecting a pile of different leaves, let students organize the leaves based on different traits. This process is the basics of classification: taking many different organisms, identifying a specific trait, and grouping from them.}
\end{itemize}

\section{Counting in the Ecosystem}


\subsection{Age of Trees}
\begin{itemize}
\item{Preparation time: 0 minutes}
\item{Materials: 1 branch or cut stump. A clean cut is very important.}
\item{Procedure: Look at the cut of the branch. Find the concentric rings. Count each ring.}
\item{Theory: Trees grow slowly, but their age is easy to find. Each concentric ring of the branch or tree relates to 1 growing year. Therefore, each ring represents one year. Use this activity to determine the age of different trees and branches. However, this will not work on fibrous trees, like the baobab tree. Furthermore, proper cutting is important. If you are cutting, use a saw and not a machete.}
\end{itemize}

\subsection{Age of Fish}
\begin{itemize}
\item{Preparation time: 0 minutes}
\item{Materials: Fish}
\item{Procedure: Certain fish have different colored scales depending on the growing season. Find a fish that has different colored bands. Count each repeating color. }
\item{Theory: The growing seasons for fish can sometimes lead to different colors in the scales of fish. This is helpful to learn how long fish have been living. Identifying the bands in a fish can tell students how long some fish have been living.}
\end{itemize}

\subsection{Crickets to Tell Temperature}
\begin{itemize}
\item{Preparation time: 0 minutes}
\item{Materials: crickets, stopwatch.}
\item{Procedure: Wait until crickets are chirping for extended periods of time. Count the amount of chirps in 1 minute. Take the number of chirps, subtract 40, and then divide the subtotal by four. Add 50. This is the temperature in degrees F. Convert to degrees C}
\item{Theory:  Crickets are more or less active depending on the temperature. This is an approximate calculation to determine the temperature from the crickets.}
\end{itemize}

\section{Ecology}
\subsection{Food Chain, Part A - Interconnectedness}
\begin{itemize}
\item{Preparation time: 5 minutes}
\item{Materials: one long bundle of rope or twine, students}
\item{Procedure: Organize the students into a circle. Holding on to the rope tightly at the end, throw the rope to another student. Pull the rope tight and the student throws the rope to another student. These throws do not need to be immediately adjacent; the more cross-circle passes the better. Pull the rope taut after each throw.}
\item{Theory: The food chain illuminates the interconnected nature of all the different living organisms in an ecosystem. Organisms eat other organisms, or the products of the organisms. Generally, organisms are connected to other organisms in very many different ways. In this activity, each student acts is a member of the food chain.}
\end{itemize}

\subsection{Food Chain, Part B - Extinction}
\begin{itemize}
\item{Preparation time: 0 minutes}
\item{Materials: a complete Food Chain, Part Interconnectedness}
\item{Procedure: Once the food chain is complete, have one student let go of all rope.}
\item{Theory: The food chain in an ecosystem is a fragile thing. If one species becomes extinct or dies, then it impacts the entire food chain. In this activity, when a student lets go of the rope it represents the death of a species. The entire chain may not fall apart, but other species may find themselves with only one connection or no connections to the chain. In that case, those students are in jeopardy of extinction themselves. @Alternatively, you can select students sit down but still holding their rope. This means that they died as a species. This causes strain in the system. The more students sit down, the more difficult it is for everyone to keep standing. This can represent the strain of many species going extinct in the food chain}
\end{itemize}

\subsection{Balance of Nature}
\begin{itemize}
\item{Preparation time: 1 hour}
\item{Materials: 2 plastic water bottles, soil, beans, water from a nearby stream or lake}
\item{Procedure: Poke holes in the bottle of the first water bottle. Fill the bottom third of the second water bottle with soil. Plant the bean sprouts in the soil. Cut the top off the second water bottle. Fill it 1/3 full of nearby stream water or lake water. Close the lid on the top bottle. Push the first water bottle into the second bottle. It will fit, and form an incomplete seal. If available, tape the bottles. It will help form a better seal. Open the top cap and thoroughly water the soil. Then seal it shut and place the system where it will receive sunlight. Observe the system over time. }
\item{Theory: Large ecosystems are difficult to contain and control. Smaller ecosystems are better for observing the growth and interaction. In this activity, a small ecosystem is made from the plant growth in the upper water bottle and the liquid environment in the lower bottle. The plants and soil will interact with the water, algae, and organisms in the lower bottle. This ecosystem should sustain itself for more than a few weeks if properly balanced. It might even rain inside.}
\end{itemize}

\subsection{Organisms in Soil, Part A – The Square}
\begin{itemize}
\item{Preparation time: 10 minutes}
\item{Materials: jembe, soil with plants or grass}
\item{Procedure: Carefully dig a small square of soil with a shovel or jembe. Dig down at least 15 cm and drag up the square completely not to damage the layering in the soil. Observe the soil, noting different colors, textures, moisture, organisms and more.}
\item{Theory: It is easy to forget that the ground that we walk on is a rather complex ecosystem. It is possible to observe this ecosystem in action through properly exhuming the soil. The different colors or layers in the soil suggest different types of soil, from loose top soil or darker, deeper soil. Try to identify as many different layers as possible. As time passes, it is possible to identify insects and other organisms growing in the soil.}
\end{itemize}

\subsection{Organisms in Soil, Part B – Forcing Them Out}
\begin{itemize}
\item{Preparation time: 30 minutes}
\item{Materials: 1 large plastic water bottle, 1 smaller plastic water bottle, some mosquito screen, freshly dug soil, lamp}
\item{Procedure: Cut off the top of the large plastic water bottle. This will act as a funnel to hold the soil. Put some mosquito screen in the funnel to hold back the soil. Again, cut off the top of the smaller plastic water bottle. Put the larger lid into the smaller bottle. Add soil to the top funnel. Place a lamp, or out in the hot sun, to heat up the soil. This may take 30 minutes up to 2 hours depending on the heat of the sun or the lamp. Plan ahead of time so that it this activity is ready for observation in the lesson. Observe the small organisms coming out of the soil to escape the heat.}
\item{Theory: In freshly dug soil, lots of different small creatures and organisms live and thrive. This activity shows the large amount of organisms insects that live in a small piece of freshly dug soil. Normally soil an environment has a much lower temperature than the air above it. The organisms prefer living in temperature that is similar to the soil. In this activity, a lamp, or sunlight is employed to heat the soil. As the soil slowly rises, it becomes a non ideal environment for organisms in the soil. From this rise in temperature, the organisms will escape the soil to find a cooler place to live. Observe the exodus of the organisms from the soil heating up.@You might also put some soil in a metal pot and heat gently over a fire. That should do the trick.}
\end{itemize}

\subsection{Terrarium}
\begin{itemize}
\item{Preparation time: 30 minutes}
\item{Materials: 1 square plastic water bottle, rocks, soil, moss, plants, an extra water bottle lid, board, insects if available}
\item{Procedure: Cut a capped plastic water bottle just below the neck, and cut all the way down the side. This allows the bottle to lie flat on a board and lifts like a lid. This allows it to act as a terrarium. Fill with rocks, sticks, moss, insects, or anything else. Be sure to put in an upside down plastic water bottle cap. Fill this cap with water. Place this terrarium some place sunny. Observe this ecosystem over time.}
\item{Theory: Using a clear plastic water bottle allows us to view inside of the terrarium. This becomes a miniature ecosystem. With plants, moss, and possibly bugs, a small temporary community can thrive. Cut the bottle properly so that it lies flat completely and the ecosystem becomes sealed. It may be necessary to poke a small hole or two in the water bottle to allow for air flow. Now you will have a small scale ecosystem and observe the changes in the system over time. An ecosystem is a natural unit consisting of all plants, animals, and micro-organisms (biotic factors) in an area functioning together with all of the non-living physical (abiotic) factors of the environment. A terrarium can be considered as an ecosystem because it has plants, animals (mice, worms, bugs, insects, frogs), bacteria, and non-living physical factors (soil, rocks, leaf litter).}
\end{itemize}

\subsection{Ant Farm}
\begin{itemize}
\item{Preparation time: 10 minutes}
\item{Materials: 1 large glass jar with lids, 1 small glass jar with lids or replace both with plastic water bottles, dirt, ants, 2 plastic water bottle lids, dirt, kanga}
\item{Procedure: Place a small glass jar with the lid on inside the larger glass jar. Go out and find an ant hill. Dig up some ants look for a queen, larva, and eggs. This ant will be larger than the rest and possibly has wings. Be careful if you start digging up siafu or safari ants. Their bites hurt. Fill the larger jar carefully with dirt with the ants. Fill 2/3 of the way full. On top of the soil, put two plastic water bottle lids: one filled half way with water and the other with a small amount of bread, fruit, crumbs, sugar, or honey. Screw on the lid with many air holes – make sure they are smaller than the ants! Observe the activities of the ants. Cover with a kanga or another piece of cloth when not observing.}
\item{Theory: Ants are remarkably industrious insects. Immediately the ants will start to dig out a new home in the dirt. The smaller jar on the inside forces the ants on the outside where we can view their work. Do not forget to feed and water your pets.}
\end{itemize}

\subsection{Pond / Ocean Aquarium}
\begin{itemize}
\item{Preparation time: 30 minutes}
\item{Materials: Large plastic water bottle or glass container, pond / ocean water (not tap water!)}
\item{Procedure: Use a plastic container to collect a fish or other water animal. Collect some rocks or soil from the bottom and gently place it on container. Observe the fish over a course of a day, and then return to the water. Observe the movements, the color, body, and more features of the organism.}
\item{Theory: Unlike the terrarium, the aquarium is not a long term possibility because aquatic organisms often require more oxygen dissolved in the water than a small volume can hold. If you can balance the ecosystem with enough water plants to oxygenate it during the day, but not so many they consume all the oxygen at night, you can keep fish living for a long time, if they are small fish that require little oxygen and subside on the other smaller (invisible) organisms living in your pond water.}
\end{itemize}

\section{Enzymes}
\subsection{Enzymes in the Mouth, Part A - Sweetness}
\begin{itemize}
\item{Preparation time: 10 minutes}
\item{Materials: crackers, bread, or another starchy substance}
\item{Procedure: Give students the bread and have them chew without swallowing. Chew for a long time. As time passes, the students will start to taste something sweet in their mouths.}
\item{Theory: In human mouths, there is an enzyme called salivary amylase. Amylases are enzymes that break down starches into simpler sugars. Remember, starch is a polysaccharide or a compound that is a long chain of sugars. The enzyme cuts the starches into smaller sugars. The students will taste these sugars – hence the sweet taste from the bread as time goes on.}
\end{itemize}

\subsection{Enzymes in the Mouth, Part B - Testing}
\begin{itemize}
\item{Preparation time: 10 minutes}
\item{Materials: crackers, bread, or another starchy substance, iodine}
\item{Procedure: After the students have let the bread in their mouth become sweet from Enzymes in the Mouth, Part A - Sweetness, have them spit out the remaining bread. Put a few drops of iodine on the slush. Also, drop some iodine on an uneaten piece of bread for comparison.}
\item{Theory: Dropping iodine on starch is the food test for determining presence of starch. Iodine binds to starch to form a blue complex. The iodine test will show starch on the bread, but should not show starch on the bread the students spit out. This is because salivary amylase has cut the starches into simple sugars.}
\end{itemize}

\subsection{Catalase, Part A}
\begin{itemize}
\item{Preparation time: 10 minutes}
\item{Materials: 1 plastic water bottle with a lid, yeast, hydrogen peroxide, water}
\item{Procedure: Place about 10 mL of hydrogen peroxide in a small plastic water bottle. Crush the bottle and add some yeast. Cap the bottle and gently invert to make sure all of the yeast is in contact with the hydrogen peroxide. The bottle should expand. Once the bottle is full, test for oxygen using the splint test; roll a piece of paper into a thin tube, lighting it on fire, and then blowing out the flame so the paper is just glowing orange. Open the bottle and put the glowing paper inside near the top – it should relight.}
\item{Theory: In yeast there is an enzyme called catalase. This enzyme is responsible for the break down of hydrogen peroxide into oxygen. Hydrogen peroxide is a poisonous byproduct of metabolism. This enzyme protects the organism by eating the peroxide.}
\end{itemize}

\subsection{Catalase, Part B - Temperature}
\begin{itemize}
\item{Preparation time: 10 minutes}
\item{Materials: 3 plastic water bottles with lids, yeast, hydrogen peroxide, water, glass jar, heat source}
\item{Procedure: This demonstration is slightly different from Catalase Enzyme, Part A. Prepare one bottle just as in Catalase Enzyme, Part A – this is your control. Then take some yeast, place it in water, and boil it. Let the yeast solution cool, and transfer to a plastic water bottle. Add some hydrogen peroxide, crush the bottle, and observe.}
\item{Theory: Boiling the yeast causes catalase – a protein – to denature (lose its special shape). This is the same reason egg white turn white and solid when heated. This means that after boiling, catalase produces little or no oxygen. By doing both procedures, the normal procedure and the boiled procedure, students can see that the normal one produces lots of oxygen and the boiled produces almost none. This shows that biological enzymes are susceptible to temperature. A good comparison between biological catalysts, enzymes, and inorganic catalysts is to do this demonstration. See Oxygen Production, Part A in the Chemistry Section.}
\end{itemize}

\subsection{Catalase, Part C – Other Sources}
\begin{itemize}
\item{Preparation time: 10 minutes}
\item{Materials: 2 water bottles with cap, liver, potatoes, hydrogen peroxide}
\item{Procedure: Into one bottle, cut some potatoes into small pieces. A fresh potato is necessary. Crush the bottle, pour in hydrogen peroxide, and cap the bottle. Repeat this procedure but instead of potato, place small pieces of freshly cut liver instead. }
\item{Theory: Catalase is present in many different locations. It is present in starchy foods; it is the enzyme that allows organisms to convert stored energy, starches, into usable energy, glucose. It is also present in the livers of different animals. In these demonstrations, the catalase present in both potatoes and liver will slowly produce oxygen by decomposing hydrogen peroxide. The bottle will slowly refill. Prove the gas is indeed oxygen by doing the glowing splint test.@In addition, they could compare with cooked potato or liver}
\end{itemize}

\section{Genetics}
\subsection{DNA Extraction}
\begin{itemize}
\item{Preparation time: 20 minutes}
\item{Materials: salt, soap (liquid hand soap is best, but laundry soap also works), water, methylated spirits}
\item{Procedure: Use drinking water to prepare a saturated salt solution – there should be a little extra salt on the bottom of the container after mixing. Have students swish the solution in their mouths for at least 60 seconds - 90 is better if the students can take it - and then spit the solution into a small container. Add about a third of this spittle to a soap solution. Gently rock the bottle back and forth for two to five minutes. Finally, carefully pour methylated spirits down the inside of the container so it forms a separate layer on top. Add no more than ¼ of the total volume as methylated spirits. Transparent strings of DNA should precipitate at the boundary between the two layers.}
\item{Theory: Salt provides the DNA with a favorable environment; it contributes positively charged atoms that neutralize the normal negative charge of DNA. In the experiment, the enzymes in the soap are breaking down the lipid molecules of the cell and nuclear membranes, releasing the contents of the cell, including the DNA. These enzymes in the soap are what break down grease while washing dishes.@In this experiment, the DNA will slowly rise from the watery lower layer up into the alcohol layer above it. The DNA will look stringy and have small bubbles attached to it. It will be a clear substance and may be hard to see. You may slowly twist this substance onto a toothpick. (Do not scoop up cell scum from the lower layer.)}
\end{itemize}

\subsection{Mendelian Genetics}
\begin{itemize}
\item{Preparation time: 30 minutes}
\item{Materials: lots of beans and maize seeds}
\item{Procedure: Provide an ample but equal amount of beans and maize seeds to each student group. The beans represent the dominant allele (Z) and the maize seeds represent the recessive allele (z). In this activity, students are going to cross two heterozygotes (Zz x Zz). Let students make a mixture of Zz, in this case, 50\% beans and 50\% maize seeds. Label this the Mother. Repeat this to make the Father pile. In order to make the offspring, take one seed from each pile. Repeat this procedure at least 20 times and record each off spring and its genotype.}
\item{Theory: In sexual reproduction, each parent gives the offspring one copy of each gene. Thus for a particular gene “zed” every individual has two copies of the gene, one from the mother and one from the father. In this activity, we assume that there are only two alleles for this trait, Z, a dominant allele, and z, a recessive allele. If an offspring has ZZ, Zz, or zZ, it will have the trait associated with Z. If an offspring has zz, it will have the trait associated with z. Homozygous means having the same alleles, ZZ or zz. Heterozygous means different alleles, Zz or zZ. An example of a dominant trait in humans is polydactyl - an extra finger on the hand. An example of a recessive trait is cycle cell anemia. Another example is eye color – blue eyes are recessive while brown are dominant, but the gene for blue eyes is so rare in many countries that this example is not very helpful.}
\end{itemize}

\section{Germ Growth}
\subsection{Making Petri Dishes}
\begin{itemize}
\item{Preparation time: 1 hour}
\item{Materials: 5 shallow plastic or glass dishes, 1 bouillon cube (any flavor), sugar, water, 1 envelope of gelatin}
\item{Procedure: Boil together 250 mL of water. Add 10 grams of sugar and 1 bouillon cube. Once mixed thoroughly, remove from heat. Let the solution cool for a few minutes and then add 1 envelope of gelatin. Mix completely then divide among different shallow containers. Let the containers cool in a cold place or overnight. If available, use a fridge.  }
\item{Theory: The gelatin mixture forms a nice gel that is a perfect place for germ growth. There are proteins, salts, sugars, and water. This is a prime growing ground for bacteria. Use these Petri dishes as a starting point for other germ activities.}
\end{itemize}

\subsection{Why Wash Your Hands}
\begin{itemize}
\item{Preparation time: 10 minutes}
\item{Materials: Petri dishes, water, soap, dirty hands}
\item{Procedure: For each group, give the students 3 Petri dishes. They should have their lids on until use. One Petri dish should not be contaminated; it is the control. Have one student with dirty hands touch one Petri dish many times. Have the very same student wash their hands vigorously for 5 or 10 minutes, with soap Watch them, and ensure that they do a good job of washing their hands. Rinse their hands and have them touch the third Petri dish. Cover the dishes with clean lids and put aside in a place that will keep them safe. Observe what happens over the next few days or over the weekend.}
\item{Theory: Most people wash their hands quite infrequently. This demonstration is meant to teach people the reason for hand washing. Ideally, students will observe that while the second dish grows many bacteria, the third dish looks like the first – no or very little bacterial growth. The bacteria growing might not be the same ones that cause disease in people, but you can assure students that hand washing has the same effect on pathogenic bacteria. If everyone just washed their hands before eating, touching their faces, and every time after visiting the toilet, there would be much less disease in the world.}
\end{itemize}

\subsection{Bacteria in humans}
\begin{itemize}
\item{Preparation time: 30 minutes}
\item{Materials: Petri dishes, q tips or cotton swabs}
\item{Procedure: In this demonstration, we need many Petri dishes. Have students swab various parts of the body, e.g. nostrils, ears, mouth, hands, armpits, knees, feet, in between toes, hair, with a q tip. Then brush each q tip on a different Petri dish. For each dish, make several strokes in one direction and then several strokes in a perpendicular direction. Observe the dishes over the next few days to see the growth. Be sure to have a control dish.}
\item{Theory: Bacteria grow all over the human body. Different bacteria thrive in different environments on the human body. Utilizing this fact, many different Petri dishes can culture the different bacteria growing all over the human body. Identify the different Petri dishes and compare their growths. Look at different colors, shapes, even number of different colonies. From this information, the students can identify which parts of the body have greater amounts of bacteria growing in them. Further, this should tell the students where they need to wash, or wash more vigorously, on their bodies.@This demonstration shows the utility of skin in protecting the human body from infection. There may be bacterial all over our outsides, but they cannot invade healthy skin. Some of these bacteria cause acne when they grown in a clogged sweat pore. More importantly, many of these bacteria can cause serious infections if the skin is cut. This is why it is always important to wash cuts thoroughly with soap. This is why doctors use sterile needle and use iodine or alcohol to clean the skin before giving injections. @This demonstration can be hybridized with Why Wash Your Hands by taking swabs on one part of the body, then wash it, and swab it a second time. For example, take swaps in between a students toe. Have the student wash their feet and toes quite well and swab a second time. Observe the different bacteria growths. }
\end{itemize}

\subsection{Germs in Other Organisms}
\begin{itemize}
\item{Preparation time: 1 hour}
\item{Materials: Petri dishes, 1 specimen like a frog, q tips}
\item{Procedure: Dissect a frog to open up the different parts of the body: brain, heart, lungs, digestive system, and skin. Swap each part of the frog with a q tip and transfer to a Petri dish. Observe the dishes over time to see the different colonies in each dish from the different parts of the body. Be sure to have a control. A frog is not necessary, in fact it is recommended to do this with a variety of different organisms to see what bacteria are growing in their bodies. }
\item{Theory: Bacteria grow all over every living organism. By using the Petri dishes, the different bacteria cultures can be identified in different organisms. Further, this can be used to analyze which parts of the bodies contain bacteria. Find which parts. }
\end{itemize}

\subsection{Germs All Around}
\begin{itemize}
\item{Preparation time: 10 minutes}
\item{Materials: Petri dishes, q tips}
\item{Procedure: Use the q tips to swab different items in the classroom or the school. Swab desks, the chalkboard, the door handle, the door to the restroom, the choo, and more. Let the students test everything they want. Observe the bacteria growth over the next few days. Be sure to have a control.}
\item{Theory: Not only do bacteria live all over organisms, they also grow over everything in the world. This can be used to create bacteria colonies from all over the school, and to show which places have the most. This should also tell the students where they need to wash their hands after visiting in order to reduce the amount of bacteria that starts to grown on their hands. }
\end{itemize}

\section{Home Microscopes}

\subsection{Water Magnification}
\begin{itemize}
\item{Preparation time: 5 minutes}
\item{Materials: water, clear glass test tube}
\item{Procedure: Fill the test tube with water. Put your thumb over the top of the test tube and turn it horizontal. Look through the water in the test tube to whatever needs magnifying.}
\item{Theory: The light that passes through water refracts and gives rise to a magnifying effect. This magnifying effect is not terribly strong; maybe it magnifies things by a factor of two. This is better than nothing in terms of seeing small things.}
\end{itemize}

\section{Other Educational Resources}

\subsection{Field Trip}
\begin{itemize}
\item{Preparation time: --}
\item{Materials: --}
\item{Procedure: Take students to see the local dispensary or hospital. Ask doctors if students can see slides of malaria infected blood or slides showing worms. Be sure to coordinate ahead of time with the hospital or dispensary. }
\item{Theory: Many schools do not have a microscope available for use. However, dispensaries and hospitals almost certainly have them. If the dispensary is both nearby and open to hosting a field trip, the hospital can provide a real life example of the lessons in your biology class. }
\end{itemize}

\subsection{Guest Speaker}
\begin{itemize}
\item{Preparation time: --}
\item{Materials: --}
\item{Procedure: Invite a doctor or nurse to the school to talk about a variety of different topics that coincide with the lessons. Some possible topics are basic hygiene, disease prevention, malaria prevention, AIDS prevention, and the importance of boiling or otherwise cleaning water.}
\item{Theory: Guest speakers provide an excellent opportunity for the students to learn more about the importance of the lessons being taught. Not only that, guest speaks provide a life path or even a role model for many students to aspire towards. Be careful, some guest speakers are better than others.}
\end{itemize}

\section{Plants}

\input{./tex/activities/chlorophyll-colors.tex}

\subsection{Seed Germination}
\begin{itemize}
\item{Preparation time: 10 minutes}
\item{Materials: seed or beans, small paper cups or bottles of small plastic water bottles or plastic bag containers with holes in the bottom, water}
\item{Procedure: Plant some seeds in a handful of different small cups. Be sure to use some good soil. Water the plants every day. Each day, take out one of the seeds. Have the students identify the different stages of germination.}
\item{Theory: Since seeds sprout underground, it is hard to see what is happening to the seed as it germinates. By uprooting a seed after each day, every stage of the germination can be observed and sketched. Ask students to classify the germination as hypgeal or epigeal . Have students draw pictures of the sprouting seeds and identify each part.@When a bean seedling emerges from the soil it is curved (this is the hypocotyl) and it pushes through the soil. As the seedling continues to grow, the hypocotyls straightens and carries the cotyledons and the plumule above the soil surface. This type of germination, where the cotyledons are carried above the soil, is called EPIGEAL germination. Examples: bean seeds, castor oil seeds, groundnuts, cotton, and bambara nuts. A few monocotyledonous seeds, such as onions and lilies, exhibit epigeal germination.@Germination of a maize seed follows a different pattern from that of a bean seed. The plumule pushes its way out of the soil while the cotyledon remains underground. The plumule does not form a hook as in bean seeds. This type of germination in which the cotyledons remain underground is called HYPOGEAL germination. Examples: maize seeds, wheat, sorghum, and millet. A few dicotyledonous seeds, such as kidney beans and broad beans, exhibit hypogeal germination.}
\end{itemize}

\subsection{Avocado Germination}
\begin{itemize}
\item{Preparation time: 10 minutes}
\item{Materials: Avocado pits, water, small cups or bottoms of plastic water bottles with holes in the bottom, toothpicks}
\item{Procedure: Fill the container mostly full of water. Using toothpicks, keep the avocado seed half under water and half exposed. Each day, examine the avocado seed.}
\item{Theory: This germination is an example of hydroponic germination. Instead of soil, the pit or seed is sprouted by water, air, and sunlight only. In this example, it is easy to see the changes the seed undergoes.}
\end{itemize}

\subsection{Potato Germination}
\begin{itemize}
\item{Preparation time: 10 minutes}
\item{Materials: Potato, water, cups with holes in the bottom, soil}
\item{Procedure: Follow the same procedure as Avocado Germination, except with a potato. After the potato sprouts, bury it in the ground and wait for the potato plant to grow.}
\item{Theory: Avocadoes are very difficult to sprout and grow into a tree. However, potatoes make great substitute to see the plant go from beginning stages until complete plant growth.}
\end{itemize}

\subsection{Discovering Factors Affecting Plant Growth, Part A - Soil}
\begin{itemize}
\item{Preparation time: 30 minutes}
\item{Materials: beans, water, plastic water bottle bottoms or cups with holes in the bottom, top soil, deeper soil, sand}
\item{Procedure: In one water bottle bottom, fill 2/3 full of top soil from the surface of the ground. In a second bottle bottom, repeat but take a darker soil from 1 meter down from the surface. In a third, repeat but fill with sand. In a forth, fill with a 50:50 sand soil mixture. Plant each with 2 bean seeds with at least 5 cm apart. Place in a location with lots of sunlight. Water gently each day. Observe growth each day over the course of 2 weeks, paying particular attention to date of sprouting, size, color, condition and more.}
\item{Theory: Soil has a major effect on seed germination and plant growth. Soil contains a small ecosystem that allows the seeds to flourish; healthy soils lead to healthy plants. Healthy soil has the proper pH balance, water, salinity, nutrients, and microorganisms to promote plant growth. Every soil is different. Sand, for example, has few of the important factors involved in germination. By varying the soil factors, students can learn about how what factors are involved in the germination process. Connect discussion of different kinds of soil to environmental issues, like the relationship between desertification and plant germination}
\end{itemize}

\subsection{Discovering Factors of Plant Growth, Part B - Salinity}
\begin{itemize}
\item{Preparation time: 30 minutes}
\item{Materials: beans, water, plastic water bottle bottoms or cups with holes in the bottom, top soil, deeper, salt}
\item{Procedure: In the first and second plastic bottle bottom, fill 2/3 full with soil. In the third and forth plastic bottle bottom, fill 2/3 with soil mixed with 100 grams of salt each. Plant each with 2 seeds at least 5 cm apart. Water the first and the third plastic bottle bottom with normal water every day. Water the second and the forth plastic bottle bottom with a salt water solution made from 100 g salt in 1000 mL of water. . Observe growth each day over the course of 2 weeks, paying particular attention to date of sprouting, size, color, condition and more}
\item{Theory: Salts and other electrolytes are necessary to live. However, too much salt makes the soil inhospitable for plants to germinate or grow. Ask students why this is so – get them thinking about osmotic pressure, why they get thirsty after eating salty food, and why it hurts to get salt in cuts. In the first variant, the plant should grow well. The second variant has normal soil but is watered with salt water. This plant will suffer from the added salinity from the watering. The third variant is watered with normal water, but is trying to grow in soil with high salinity. This plant will also not thrive. Lastly, the forth variant has a plant trying to grow in saline soil with salty water. Sadly, this plant has no chance at survival.}
\end{itemize}

\subsection{Discovering Factors of Plant Growth, Part C - Water}
\begin{itemize}
\item{Preparation time: 30 minutes}
\item{Materials: beans, water, plastic water bottle bottoms or cups with holes in the bottom, soil}
\item{Procedure: Have each group of students fill 5 plastic bottle bottoms 2/3 full with soil and plant with 2 beans 5 cm apart. In the first cup, do not water at all. In the second cup, water the cup with 25 mL of water once a day. For the third cup, water with 100 mL once a day. In the forth cup, water with 50 mL twice a day. In the last cup, water with 50 mL four times a day. . Observe growth each day over the course of 2 weeks, paying particular attention to date of sprouting, size, color, condition and more.}
\item{Theory: This activity explores different watering amounts for plant growth. Every plant requires varying amounts of water to survive. Some grasses need very little, some plants need almost complete watering. In this demonstration, all the variants will have different growth rates. Some may even receive too much water.}
\end{itemize}

\subsection{Discovering Factors of Plant Growth, Part D - Sunlight}
\begin{itemize}
\item{Preparation time: 10 minutes}
\item{Materials: beans, water, plastic water bottle bottoms or cups with holes in the bottom, soil}
\item{Procedure: Have students fill 2 plastic water bottle bottoms 2/3 full with soil and plant 2 beans 5 cm apart. Place one seedling in a location that receives sunlight. Put the other in a cupboard, under a bucket, or in another place that receives no sunlight. Observe growth each day over the course of 2 weeks, paying particular attention to date of sprouting, size, color, condition and more.}
\item{Theory: Green leafy plants require sunlight to germinate and grow properly. If the seedlings are denied sunlight, their growth will be stunted or even nonexistent. However, not all plants require the same amount of sunlight. Some plants only need a few hours of indirect sunlight at most while some require 6 or more hours of direct sunlight. As a variation on this experiment, keep some plants in the dark for most of the day but receiving various controlled amounts of sunlight – one hour, two hours, etc. Make sure to water all plants throughout the experiment!}
\end{itemize}

\subsection{Discovering Factors of Plant Growth, Part E - Chemical Fertilizers}
\begin{itemize}
\item{Preparation time: 10 minutes}
\item{Materials: beans, water, plastic water bottle bottoms or cups with holes in the bottom, soil, fertilizer}
\item{Procedure: Have students fill 2 plastic water bottle bottoms 2/3 full with soil and plant 2 beans 5 cm apart in each one. In one of the bottoms, mix in a small amount of fertilizer in the soil before planting. Water each day. Observe growth each day over the course of 2 weeks, paying particular attention to date of sprouting, size, color, condition and more.}
\item{Theory: Plants require certain nutrients for growth. Fertilizers provide some of these nutrients in a direct and highly concentrated form. Soils with fertilizer may allow plants to grow much faster than they would otherwise. Have students discuss the implication of this to farming. You might also put a lot of fertilizer in another container – use enough and it will kill the plant (“fertilizer burn”), another good lesson with relevance to farming, and for teaching about moderation in general.}
\end{itemize}

\subsection{Soil Retention}
\begin{itemize}
\item{Preparation time: 20 minutes}
\item{Materials: different examples of soil, packed, loose, sand, clay, and more, one coffee can, water}
\item{Procedure: Remove both ends of a coffee tin. Place the tin on different soils. Fill with water, and observe the time required for the water to drain out of the tin completely. Repeat this with different types of soil.}
\item{Theory: Water moves through different soils at different rates. This ability for soil to let water move or hold on to water is important for plants to grow. If water passes too quickly, like soil, plants have no chance at grabbing water. If water does not pass at all, it may be too packed for plants to grow easily in the soil.}
\end{itemize}

\subsection{Tropic Movements}
\begin{itemize}
\item{Preparation time: 10 minutes}
\item{Materials: bean seedlings in plastic water bottle bottoms or cups with holes in the bottom, box,}
\item{Procedure: In the box, put one medium sized hole in the top. Put two or three seedlings in the box. Be sure that one is directly under the sunlight while the rest around the center plant so they receive directional sunlight. Water each plant every day and observe the growth of the seedlings.}
\item{Theory: Green leafy plants grow in a direction to maximize the exposure to sunlight. The seedling directly under the sunlight will grow upwards like normal. However, each of the plants that surround the center plant will grow towards the center plant so that it will increase their exposure to sunlight. }
\end{itemize}

\subsection{Seed Window}
\begin{itemize}
\item{Preparation time: 30 minutes }
\item{Materials: 1 plastic cup or plastic water bottle bottom, soil, water, bean seeds}
\item{Procedure: Cut small segments around the outside of the water bottle and put a few holes in the bottom. Use tape or paper clips to hold the segments on the plastic bottles. Fill the bottom of the plastic water bottle with soil. Plant bean seeds around the outside of the plastic water bottle container. Water the plants. As the plants grow, open up the flaps and see how the plant is working its way through the soil.}
\item{Theory: The purpose of the flaps on the germination container is to view the bean seeds as they germinate in the soil. Taking the seeds outside the soil is nice, but it does not help to see what it looks like in the soil. In this case, the flaps allow to view the germination in process}
\end{itemize}

\subsection{Making a Greenhouse}
\begin{itemize}
\item{Preparation time: 10 minutes}
\item{Materials: 1 large plastic water bottle, 2 small plastic water bottle, soil, bean seeds, water}
\item{Procedure: Cut off the top from both water bottles. Put a few holes in the smaller water bottle. Fill lower bottle 2/3 full with soil and plant bean seeds. Water and turn the large plastic water bottle upside and cover the smaller plastic bottle. Use the second smaller plastic water bottle}
\item{Theory: A green house works by containing heat and moisture around the plant. In a slightly warmer, moister environment, plants will grow faster. This demonstration is most effective in the cold season and in cooler regions.}
\end{itemize}

\subsection{Leaf Outlines}
\begin{itemize}
\item{Preparation time: 10 minutes}
\item{Materials: white paper, leaves, pencils.}
\item{Procedure: Have every student take a leaf and place a piece of paper over it. They should gently run a pencil over the paper. As the pencil colors the entire page, the border and veins of the leaf become visible. Repeat this process for different leaves, and create a book of all the different leaves around the school.}
\item{Theory: The structure of the leaves of different trees will also be different. This activity is a good way to start to look at the differences in the trees around the school and how the leaves are different. Let the students keep a book of all the different trees from the school.}
\end{itemize}

\section{Transport}

\subsection{Powder Diffusion}
\begin{itemize}
\item{Preparation time: 0 minutes}
\item{Materials: powdered food coloring or kool-aid like product, water, plastic water bottle}
\item{Procedure: Fill the plastic water bottle with water. Quickly add the powdered food color, but do not shake. Observe the color diffuse through the water.}
\item{Theory: Mixing does not occur immediately. Without shaking or stirring, it occurs slowly. By using a colored compound, it is easy to see how the molecules are slowly dissolving into the solution. }
\end{itemize}

\subsection{Orange Diffusion, Part A – Sweet Smells}
\begin{itemize}
\item{Preparation time: 5 minutes}
\item{Materials: one orange or other citrus fruit}
\item{Procedure: Have students sit in their seats. Start to peel the orange. When students begin to smell oranges, have them raise their hands. Be sure the students only raise their hands as they smell the orange and not before.}
\item{Theory: Diffusion happens in not only liquids but also gases. Peeling oranges or other citrus fruits releases small compounds that diffuse through gases. When these compounds come in contact with out noses, we smell oranges. However, we cannot smell oranges immediately on peeling; the compounds must migrate towards our noses. In this case, the compounds will slowly diffuse in the classroom with the students closest to the orange smelling it first. The students in the back of the classroom will smell it last. The effects of wind should be considered.}
\end{itemize}

\subsection{Orange Diffusion, Part B - Trapped}
\begin{itemize}
\item{Preparation time: 5 minutes}
\item{Materials: a box, one orange or other citrus fruit}
\item{Procedure: Turn the box upside down. Without turning the box over, peel the orange inside of the box. When students begin smelling oranges, have them raise their hands.}
\item{Theory: Diffusion can only occur when the molecules can move freely. Some objects will not allow compounds through. In this activity, the cardboard box prevents the compounds in the orange to diffuse out through the classroom. This time, students not smell oranges or it will take a long time for students to start smelling.}
\end{itemize}

\subsection{Osmosis}
\begin{itemize}
\item{Preparation time: 10 minutes}
\item{Materials: 1 potato or carrot, water, salt, two water bottle bottoms}
\item{Procedure: Cut two equal sized pieces of potato. Put one piece is normal water and the other in a salt-water solution. Observe over the next few hours.}
\item{Theory: In all cells and plants, there is a proper balance of different concentrations of salts and sugars. Osmosis is the process where the salts move from a high concentration either to a low concentration or where water moves from a low concentration to a high concentration. In this activity, placing the potato in pure water will cause the potato to swell. Inside the potato, there is a higher concentration of salts and sugars compared to the water surrounding it. The water moves into the potato in order to make the concentrations inside the potato more similar to the water. The potato swelling is visual evidence of this phenomenon. The potato in salt water has exactly the opposite effect. The concentration of salts inside the potato is much lower compared to the concentration of salt in the water surrounding the potato. The water in the potato moves out of the potato to dilute the salt solution. }
\end{itemize}

\subsection{Water Transport in Flowers}
\begin{itemize}
\item{Preparation time: 15 minutes}
\item{Materials: white flowers with stems, food coloring, water}
\item{Procedure: Cut on the bias along the bottom of the stem of a flower. Place this flower in colored water. Observe over the next few days.}
\item{Theory: Plants need to move water into its flowers. In this activity, the white flowers will change color to match the color of the colored water. This activity should work with colored flowers; however, it will be much more difficult to see the color change with the transport.}
\end{itemize}

\section{Trash and Pollution}
\subsection{Pollution Catcher}
\begin{itemize}
\item{Preparation time: 15 minutes}
\item{Materials: coffee filters, paper towels, sticks}
\item{Procedure: Roll the coffee filters or paper towels into a cone. Attach to a stick. Place these around the school or other locations on a dry sunny day, for example near trash burning pits, around the school kitchens, or near roads. Collect the cones at the end of the day. Observe what you find caught in the paper. Identify the most polluted areas around the school.}
\item{Theory: Air pollution is a common problem. This activity uses paper towels to catch whatever is floating in the air. Certain activities produce pollution and this pollution is sometimes harmful. For example, the smoke from cooking fires causes eyes to tear and people to cough; the smoke from burning trash smells bad for you because it is.}
\end{itemize}

\subsection{Trash Journal}
\begin{itemize}
\item{Preparation time: 60 minutes}
\item{Materials: each student needs a notebook, balance.}
\item{Procedure: Have each student create a journal. In this journal, students need to write down all of the trash that they make every day for 2 weeks. If possible, have students collect all of their trash and weigh it every day. }
\item{Theory: Trash is a rather interesting problem facing society today. Too many goods have small plastic wrapping or pieces that are cast aside easily. This means it is very likely people do not understand the volume or weight of trash that each person uses every day. This journal is used to encourage students to be mindful of the waste they make. This activity would be very interesting to compare students of different locals, like the trash generated by students from villages compared to towns. How much of the trash produces is biodegradable? Help students to discuss the difference between different kinds of trash – how they are disposed of, how long they remain in the environment, what the effect is of burning them, etc.}
\end{itemize}

\subsection{Landfill Jar}
\begin{itemize}
\item{Preparation time: 10 minutes}
\item{Materials: trash of different types like plastic or food waste or metal, glass jar with lid, soil}
\item{Procedure: Fill the jar with a mixture of trash and soil. Trash can be paper, food scraps, plastic, metals, etc. Close the jar with the lid. Observe the jar and identify which trash remains after a week, a month, a year.}
\item{Theory: Trash is not just plastic trash. Trash can be food scraps or metals. Therefore, trash is divided into two general categories: biodegradable and non-biodegradable. Biodegradable trash will break down in the environment. Non-biodegradable trash does not. That is a problem with many types of plastic trash; it does not break down. Since it does not break down, it just piles up. Generally, this can be seen on the sides of roads when people just throw plastic water bottles out of the window.}
\end{itemize}

\subsection{Backwards Garden}
\begin{itemize}
\item{Preparation time: 10 minutes}
\item{Materials: area for garden, trash}
\item{Procedure: Follow the same procedure as the Landfill Jar, but instead of planting in a jar, plant outside in a garden bed. }
\item{Theory: This activity is very similar to Landfill Jar. Proper disposal of trash is very important. Without handling of trash properly, it will build up in the environment. That is both harmful to the environment and humans. Use these activities to teach students about proper stewardship of their environment and to deal with trash properly.}
\end{itemize}
