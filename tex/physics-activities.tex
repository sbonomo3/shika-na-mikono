\chapter{Physics Activities}

\section{Astronomy}

\subsection{Solar System Mobile}
\begin{itemize}
\item{Preparation time: a few hours}
\item{Materials: flour, water, balloons, mixing bowl, newspaper or old papers, string, sticks}
\item{Construction: Blow up the balloons, one for each of the 8 planets and sun. Make the paper mache mixture with flour and water; you want a watery-glue texture. Wet the paper in this mixture and apply artistically to the balloons until you have a layer a couple papers-thick on each balloon. Leave the balloon slightly exposed at the bottom. When the papers are dried, pop the balloons within and set to work making them look like planets. Use paint, markers, or colored pencils. Attach string and hang them as a mobile. If you want to get fancy, you can place the string between layers of paper before it dries, thus saving yourself some tape or glue.}
\item{Theory: This activity is helpful to explain to students what is actually happening off the world. This mobile is helpful to remember that there is more to the solar system than just earth.}
\end{itemize}

\subsection{Star Gazing}
\begin{itemize}
\item{Preparation time: 0 minutes}
\item{Materials: none}
\item{Procedure: Take the students out at night. Look for constellations, starts, planets, and even satellites. In addition to planets, look for Orion’s Belt, the Southern Cross, and more constellations. Due to the large amount of information regarding this topic, we cannot include this information. However, a quick internet search will give you what you need.}
\item{Theory: For the longest time, stargazing was some of the most important aspects of navigation and even religion. Recreate this experience by finding stars and constellations. Tell the stories behind them, and encourage students to find their own constellations and give their own stories.}
\end{itemize}

\section{Archimedes’ Principle}
subsection{Water Weight and Upthrust}
\begin{itemize}
\item{Preparation Time: 1 minute}
\item{Materials: spring balance, syringe with the bottom melted shut and no plunger, eureka can (can be made cheaply by a metal craftsman), water, heavy object, thread, small dish}
\item{Procedure: Fill the eureka can up to its spout with water and place the spout over the dish. This can is designed so that when the water is being displaced, it is collected to another container for later measurements. Hang the object by the thread from the spring balance and measure its weight. Now immerse the water completely in the eureka can and measure its Apparent Weight (weight in water).\\
When you immersed the object in water, some water will have overflowed from the can into the small dish. Pour this water into your syringe shell and measure the weight of water. Record this result with the earlier Weight and Apparent Weight.}
\item{Theory: Archimedes’ Principle states that the upthrust of a liquid on an object is equal to the weight of water displaced by the object. The upthrust is equal to the Weight of the object minus its Apparent Weight in the water:@	Upthrust = Weight in air – Apparent Weight in liquid\\
But upthrust is also equal to the water displaced:@	Upthrust = Weight of liquid displaced\\
By calculating the upthrust, you should see that the result is equal to the weight of water in the syringe.}
\end{itemize}

\section{Conservation of Energy}
subsection{Pencil Launcher}
\begin{itemize}
\item{Preparation Time: 5 minutes}
\item{Materials: Clothes clip, thread, two pencils}
\item{Procedure: Open the clip and tie the closed end with thread so that the clip stays open against the tension of the spring. Place the clip flat on a table and place two pencils next to the clip, one on either side, so that the eraser touches the tied end and the tips point out in opposite directions along the table. Cut the thread holding the clip open and stay clear of the flying pencils.}
\item{Theory: The spring inside the clip holds energy when it is forced to contract. When the clip is allowed to close, the potential energy of the spring is transformed into mechanical energy as the clip moves, forcing the pencils away at a decent speed.}
\end{itemize}

\section{Current Electricity}

\subsection{Conductor Switch Test}
\begin{itemize}
\item{Preparation Time: 5 minutes}
\item{Materials: Two or three batteries, wires, bulb, switch, various materials to test (spoon, rubber strip, stick, mchelewaji, etc.)}
\item{Procedure: Set up the circuit so that the wires, switch, and bulb are connected to the batteries in series. Close the switch to show that the bulb lights when current is passing. Open the switch. Place each material in turn across the switch, closing the circuit. Any materials that successfully close the circuit and the bulb lights can be considered conductors, and any materials that fail to close the circuit and do not light the bulb are insulators or poor conductors.}
\item{Theory: Conductors will freely allow electric current to pass, so when a conductor is used to close the switch, the circuit is complete and current will pass, lighting the bulb. Insulators, however, do not permit current to flow, so the circuit will still be broken despite the switch being ‘closed’ with the insulator.}
\end{itemize}

\subsection{Light Bulb in a Jar}
\begin{itemize}
\item{Preparation Time: half hour}
\item{Materials: Glass jar with lid, glue, wires, power source, small length of thin iron wire}
\item{Procedure: Poke two holes in the jar lid and pass a wire through each about half way into the jar. Connect the wire ends inside the jar with the length of iron wire and seal the wires into the lid with glue. Close the lid and connect the wires to the power source. If enough current is passing, the iron wire will light up, creating a ‘light bulb’ for a short time until the wire burns out.}
\end{itemize}

\subsection{Foil Fuse}
\begin{itemize}
\item{Preparation Time: 15 minutes}
\item{Materials: Power source, wires, two small nails, small piece of wood, metal foil (from Blueband container, gum, etc.)}
\item{Procedure: Hammer the nails into the wood about 5 cm apart to act as wire terminals. Connect wires to each of the nails and place a thin strip of foil between the nails, bending it around the nails to secure it. Connect the wires to the power source. If the source is powerful enough, it will cause the foil to heat and eventually burn, breaking the circuit.}
\item{Theory: Foil, having a very small cross-sectional area compared to that of a wire, has a low tolerance for current. If too much current passes through the foil, it will burn away. This is essentially how a fuse works in a radio or other electrical device. To be more scientific in your experiment, use a rheostat in the circuit, gradually lowering the rheostats resistance until the fuse blows.}
\end{itemize}

\section{Density}
See the Density activities in the Chemistry section for more.

\subsection{U-Tube apparatus}
\begin{itemize}
\item{Preparation Time: 1 hour}
\item{Materials: 3 clear plastic pen tubes, cardboard, hot poker or knife, tape, pen, super glue, water, any fluid, which will not readily mix with water.}
\item{Construction: Cut two of the pens at one end at a 45-degree angle, and cut the third pen (shorter than the other two) at both ends at a 45-degree angle. With the shorter pen on the bottom, attach the other two as styles so that the 45-degree angles meet to form right angles. Together the 3 pens should form a U-shaped tube with open ends at the top of each style (vertical tube). Melt the angled ends together with a hot knife, soldering iron, etc. so that the whole apparatus is watertight except for the tops. Glue the apparatus to a cardboard base so that it can stand up straight. Put thin strips of tape up each side of the U-tube and mark each strip with evenly spaced marks. The two scales should be identical. One good way to do this is to put steadily increasing volumes of water (3 ml, 4 ml, 5 ml, etc.) and mark the levels on each scale for each volume. Label these marks from top to bottom as 0, 1, 2, etc.}
\item{Procedure: Place an amount of water into the U-tube such that the water rises about half way on either side of the tube. The actual volume of water is not important as long as you can see the levels clearly. Stand the tube upright and slowly drip about 1 ml of another fluid, kerosene in this case, into one side of the U-tube (if the fluid has a higher density than water, it should go in first, and then the water). The kerosene will displace the water, so you should see the water level on the other side rise slightly.\\
Measure the relative heights of water and the kerosene from the bottom level of the kerosene. The heights are related to the densities by:
\[ \frac{\mathrm{Height of water}}{\mathrm{height of kerosene}} = \frac{\mathrm{density of kerosene}}{\mathrm{density of water}} \]
} % Procedure
\item{Theory: If a fluid’s density is less than that of water, it will float on top (if it is added slowly) of the water, displacing the water on the other side of the tube. From Archimedes’ principle and the Law of Flotation, we know that the relative density of the fluid is equal to the inverse ratio of the heights of the liquid. The scales drawn on the outside of the U-tube allow you to find the ratio of the heights without needing units, and the density of water is known to be 1.0 g/ml, so you can easily calculate the density of the other fluid.\\
If the other fluid has a higher density than water, the experiment can still be done, but you need to add the fluid with higher density first, then displace it with water, performing the same calculation.\\
This apparatus was designed and brought forward by two form 4 students without any prompting. They then proceeded to find the density of kerosene accurate to two decimal places. Never underestimate the curiosity and ability of students, or the power of broken pens.}
\end{itemize}

\section{Electromagnetism}

\subsection{Simple Motor}

\begin{itemize}
\item{Preparation Time: 2 hours}
\item{Materials: Flat piece of wood 6”x12”, four large nails, two small nails, two screws, two pieces of thick wire 4” long, rubber stopper about 1.5” diameter and at least 1” thick, 20ml syringe, two small pieces of sheet copper, lots and lots of speaker wire, hammer, knife, glue, two batteries}
\item{Procedure:

\begin{enumerate}
\item{Arrange the piece of wood and nails/screws as follows:

\begin{enumerate}
\item{Through the center of the board, drive a large nail so that it goes all the way through; turn the board over so that the nail sticks up.}
\item{On either side of that nail, the long way across the board, drive two other large nails into the board just enough so that they stay. The distance between these two nails should be the length of the last large nail plus about 1 cm.}
\item{At a 45-degree angle, about 1” from the center nail and directly across from each other, drive the two small nails into the board just enough so that they stay.}
\item{Along one long side of the board, in each corner, screw the small screws into the board, leaving a few cm between the head and the board.}
\item{All the nails and screws are driven into the top of the board except A, which is driven up through the bottom all the way.}
\end{enumerate}
} % Arrange the piece of wood...

\item{Electromagnets: Connect one end of a wire to the batteries. From there, wind it once around one of the screws (D) and tighten the screw to hold the wire in place. From the screw, extend the wire to the top of the nearest large nail (B) and begin winding it around the nail from the top to the bottom. More turns will produce a stronger magnet. After reaching the bottom of the magnet, extend the wire to the nearest small nail (C) and wind the wire from the bottom of the small nail to the top. At the top, solder the wire to one of the thick copper wires, called brushes. Do this again from the other terminal of the battery, to the other screw, wound down around the other large nail (B) in the same direction as the other nail, wound around the small nail to the second brush.}

\item{Rotor: Remove the plunger from the syringe and cut the finger tabs off. Along the top of the syringe tube, glue the two pieces of sheet copper as shown (upside down) in figure (ii). The sheets should be about 1” wide and should each wrap around half of the syringe’s circumference, leaving about 6 mm between the strips where they meet on each side. These copper plates, together with the brushes from (2) above, form the commutator. Hollow out a small space for the syringe to fit snugly into in the bottom of the stopper. Insert the bottom of the syringe (where the needle attaches) into this space. You will glue this later, but leave it for now. Through the top of the stopper, insert the last large nail horizontally so that exactly half of the nail sticks out on either side. The nail and upside-down syringe should form a T, with the stopper at the intersection (ii). This is the armature of the rotor.}

\item{Glue or solder the end of a wire to one of the copper sheets. From there, wind the wire around one half of the armature nail starting at the stopper and winding outwards. Once the wire reaches the end of the nail, extend it over to the other end of the nail on the other side of the stopper and start winding again towards the stopper, circling the opposite way around the nail (circling in the same direction will cause opposing magnetic fields – use the RHR if you get stuck on this). Once the wire reaches the stopper, extend it down to the other copper plate and glue or solder it there. You can cut the wire at this point. Put the syringe over the center large nail (A) so that it can rotate freely. The nail along the top should be able to turn, passing close to the two large nails (B). Turn the syringe in the stopper until the brushes touch the gaps between the copper plates when the armature magnet is aligned with the two upright electromagnets. The commutator and armature are now complete. Now you can glue the brushes to their respective nails so that they brush lightly against the gaps between the copper plates when the magnet is aligned.\\
*Note: more windings will produce a stronger magnet so you can wind back and forth, but keep the same direction of the wire. Also, a syringe with a larger diameter will produce a more accurate commutator.}
\item{The motor is now complete: allow current to run and give the armature a push to start it spinning. It may take a few tries to get it going, but if your commutator is well placed and your coils are wound in the right direction, the motor should keep going.}
\end{enumerate}
} % Procedure

\item{Theory: The two coils on each side (B) are large electromagnets, which keep a constant polarity, as the current never changes. These coils are also connected to the thick wires, which brush against the commutator of the armature (rotating bit), so the thick wires act as the opposite battery terminals.\\
The armature is also a large electromagnet, but as the commutator is repeatedly switching poles as it rotates and brushes against the thick wires, the direction of current, and therefore the poles of the magnet, is always switching with each half rotation.\\
In one position with the armature magnet facing the two magnets on the side, a strong magnetic force holds it in place. As the rotor rotates (being pushed), the commutator plates switch brushes and the current through the armature reverses, thereby reversing the poles of the magnet. Now the armature magnet is attracted to the opposite upright magnet. As it passes the magnets again, 180-degrees, the current switches again and the cycle continues. As the rotor gains speed, it will become more stable.}
\end{itemize}

\subsection{Creating a Current in a Wire}
\begin{itemize}
\item{Preparation Time: 5 minutes}
\item{Materials: Wire about 50 cm, ammeter or sensitive bulb, strong magnet}
\item{Procedure: Coil the wire to create a solenoid, connecting the free ends to the ammeter or bulb. Use a bar magnet or one pole of a horseshoe magnet and pass it through the solenoid (if you are using a speaker magnet, you will need to adjust the coil to accommodate the odd shape). As the magnet passes through the coil, the ammeter or bulb will show a current. When the magnet stops or leaves the coil, the current will cease.}
\item{Variation: If you have a very strong bar magnet, wrap the wire around a syringe multiple times and connect the ends to an ammeter. Place a small wad of cloth in the bottom of the syringe and insert the magnet. Cover the opening with your thumb and shake. The wad and your thumb will protect the magnet as it bounces back and forth, creating an alternating current in the coil.}
\item{Theory: A magnetic field that moves perpendicular to a conductor will induce a current in that conductor. When the conductor is a coil and a bar magnet is passed through it, a significant current is induced and should be enough to light a sensitive bulb or deflect the needle in an ammeter. The current will be stronger if the number of coils is increased or if a stronger magnet is used.}
\end{itemize}

\subsection{Mapping Induced Magnetic Field from a Coil}
\begin{itemize}
\item{Preparation Time: 15 minutes}
\item{Materials: power source, length of wire about 50 cm, bulb or switch, cardboard, scissors, iron wool}
\item{Procedure: Cut the cardboard so that a single tab about 10 cm long and 3 cm wide sticks out from the larger piece. Notch this tab every 1 cm on either side and coil the wire around the tab, keeping it in place in the notches. Connect the wire to the switch/bulb and power source so that there is a strong current in the wire. Use the iron wool to sprinkle iron filings onto the tab inside the wire coil. The filings will create a single solid line the length of the coil, spreading out at each end.@Theory: A coil of wire creates a single, strong magnetic field inside it in one direction. At the ‘poles’ (for it is indeed an electromagnet) the field spreads out again. You can use the 2nd Right Hand Rule to find the direction of the field. The filings will align themselves with the strong field inside.}
\end{itemize}

\subsection{Mapping Induced Magnetic Field from Wire}
\begin{itemize}
\item{Preparation Time: 10 minutes}
\item{Materials: power source, length of straight wire, switch or bulb, paper or cardstock, iron wool}
\item{Procedure: Cut a hole in the paper or cardstock so that the wire passes vertically through the middle of the paper so it lies flat. Connect the wire, hanging vertically, to the switch/bulb and power source so that there is a strong current in the wire. Using your thumb and forefinger, rub the iron wool to create iron filings, distributing them widely onto the paper. The filings should form concentric circles around the wire.}
\item{Theory: Current in a straight wire produces a magnetic field around the wire (use the Right Hand Rule to find the direction) in concentric circles. At the surface of the paper, the magnetic field is a series of circles and the filings will align themselves with the field.}
\end{itemize}

\subsection{Spinning Compass}
\begin{itemize}
\item{Preparation Time: 1 minute}
\item{Materials: batteries or other power supply (the stronger the better), wire, switch or bulb, compass (the pin in the compass demo will do)}
\item{Procedure: Connect the wire and switch/bulb to the battery or power source so that there is a strong current running through the wire. Run the wire over the compass in a straight line. If the current is DC, the compass will turn to face a new direction. If the current is AC, the compass will spin or waver back and forth quickly.}
\item{Theory: Current in a straight wire creates a magnetic field around the wire in concentric circles. The direction of the magnetic field can be found using the first right-hand-rule. DC current produces a steady magnetic field in one direction (circular), so the magnet of the compass will align itself with the field. AC current produces a constantly shifting magnetic field, so the compass will spin, trying to align itself as the field changes direction.}
\end{itemize}

\section{Fluid Mechanics}

\subsection{Water Pour}
\begin{itemize}
\item{Preparation time: none}
\item{Materials: pitcher, bucket, water}
\item{Procedure: Fill the pitcher with water. Place the bucket on the floor. Stand on a desk or table to increase your height, and pour water from the pitcher into the bucket. Try to pour the water at a constant rate. Point out that the stream of water is thicker at the top, and becomes thinner as it falls. If the rate of flow is small, or if you are pouring from a very high height, the stream will also break up into drops near the end.}
\item{Theory: As the stream of water falls, gravity causes it to accelerate to higher speeds. The continuity principle tells us that when the water is moving faster, it must have a smaller cross-sectional area. Thus, the stream of water is thicker at the top, and thinner at the bottom. If the stream of water becomes thin enough, surface tension will pull it into individual drops, because this will minimize the surface area, and therefore the surface-energy.}
\end{itemize}

\subsection{Cone Blow}
\begin{itemize}
\item{Preparation time: 5 minutes}
\item{Materials: A funnel or top of a 1.5 liter water bottle, one piece of paper, adhesive tape, scissors}
\item{Construction: If you do not have a funnel, cut off the top of an empty Kilimanjaro water bottle. Cut out a circular piece of paper, and cut along one radius. Bend the paper into a cone. Tape the cone so that it fits neatly into the funnel.}
\item{Procedure: Place the cone in the funnel. Ask for a student volunteer. Tell the volunteer to push the cone out of the funnel by blowing upwards into it. After the student fails at this, ask for several other volunteers.}
\item{Theory: When the student begins to blow through the funnel, air passes through the narrow space between the cone and the funnel with some velocity. According to Bernoulli’s Principle, this moving air has a lower pressure than the stationary air inside of the cone. Because the air outside of the cone has a lower pressure than the air inside of the cone, the cone is actually pulled back into the funnel. Thus, no matter how hard the student blows, the cone will remain in the funnel.}
\end{itemize}

\subsection{Paper Blow}
\begin{itemize}
\item{Preparation time: none}
\item{Materials: 2 sheets of paper, not wrinkled}
\item{Procedure: Hold two sheets of paper roughly parallel, several centimeters from each other. Blow into the space between the two sheets of paper. They will be pulled towards each other.}
\item{Theory: According to Bernoulli’s Principle, the moving air between the two sheets of paper has a lower pressure than the stationary air outside of them. Thus, atmospheric pressure will push the two sheets of paper towards each other.}
\end{itemize}

\section{Friction}

\subsection{Spring pull}
\begin{itemize}
\item{Preparation Time: 1 minute}
\item{Materials: various surfaces, pens, oil, wheel bed, spring or spring balance, block of wood, eyehook}
\item{Procedure: Screw the eyehook into the block of wood so that the spring can easily be attached to it. Drag the block along a rough surface with the spring and measure the spring’s extension. Now place a row of pens, side-by-side on the surface and drag the block over the pens, again measuring the spring’s extension. Repeat this experiment using the wheel bed (can be done easily with water bottle caps and nails), using an oiled surface, and various other surfaces. Also turn the block so that alternately a large side and a thin side are in contact with the surface, measuring the relative spring extensions.}
\item{Theory: Friction depends on the nature of the surface in contact and not on the surface areas in contact, so there should be no noticeable difference between the two sides of the block. However, the rollers, in this case pens, wheels, and lubricant, in this case oil, all reduces friction, so the spring’s extension should decrease when these methods are used.}
\end{itemize}

\section{Kinetic Theory of Gases}
See the Gas Laws in the Chemistry section for activities

\section{Light}

\subsection{Pinhole Camera}
\begin{itemize}
\item{Preparation Time: half hour}
\item{Materials: Cardboard box, black paint if necessary, translucent screen (tissue paper, color gel, etc.), pin, tape, scissors, light source, any object}
\item{Procedure: Cut out one side of the cardboard box and paint the inside black. Replace the cutout side of the box with your translucent screen, taping it shut along all four edges. On the opposite side of the box from the screen, poke a small hole with the pin. Your camera is now complete.\\
In a dark room, shine a bright light source on an object and aim the camera at it so that the light from the object passes through the pinhole to the screen. If the source is bright enough, the image should appear, upside down, on the screen. Play around with the object distance until you have a large, clear image on the screen. It is recommended to try this outside on a bright day, but you will need to cover the space between the camera and your head completely so that no light can enter.}
\item{Theory: Light travels in a straight line, and so light from the top of the object will pass at an angle through the pinhole, appearing at the bottom of the screen on the other side. Alternately, light from the bottom of the object will appear at the top of the screen. A strong light source is needed because the aperture pinhole is small and will only admit a small amount of light.}
\end{itemize}

\subsection{Light through a Comb}
\begin{itemize}
\item{Preparation Time: 1 minute}
\item{Materials: comb, light source, optional mirror}
\item{Procedure: In a dark place, shine the light parallel to a table surface through the comb. The apertures in the comb will act as ‘beams’ of light. Reflect the beams off a mirror and observe the straight-line propagation of light.}
\item{Theory: Light travels in a straight line, even when reflected at a surface.}
\end{itemize}

\subsection{Kaleidoscope}
\begin{itemize}
\item{Preparation Time: 5 minutes}
\item{Materials: 3 or more mirrors of equal size OR 3 or more pieces of glass of equal size with metal foil on one side, tape; Optional: colored objects}
\item{Procedure: Tape the three mirrors together so that they form a triangular tube with the reflective sides facing in. Look through the kaleidoscope at any objects, especially colored beads or paper, and turn the scope to watch the pretty colors change!}
\end{itemize}

\subsection{Color wheel}
\begin{itemize}
\item{Preparation time: Half hour}
\item{Materials: white paper, colored pencils, Nido can lid, 1” screw, tape or glue@Construction: Cut the paper into a circle with the same diameter as the Nido cap. Using a pencil and straight-edge, divide the circle into seven pie slices and color each slice a single color from ROYGBIV (Red, orange, yellow, green, blue, indigo, violet). Drawing 14 or 21 slices is more effective, but seven works well.@Using a pencil or something else sharp, balance the Nido lid until you find its center of gravity. Mark it and carefully screw the screw through at that point, creating a kind of top. Tape or glue the colored paper to the top, colored side up, screw point down. Now you have a top with the seven rainbow colors on top.}
\item{Procedure: Review the colors with your students using the colors on the wheel/top/thing. Place the wheel on a desk and give it a good spin. If it is well-balanced, it should spin smoothly and all the colors run together to form white. Bask }
\item{Variation: (1) Wrap string around the screw so that both ends of the string stick out. Pull them quickly in opposite directions to get the top spinning quickly. (2) Poke two holes about 1 cm on either side of the center of the wheel (no screw). Loop a string through the two holes so that two lengths of string of equal length stick out on each side. Tie the loose ends on the one side and then hook your thumbs into each loop. Twist the strings by spinning the wheel. If you pull, the wheel will spin and re-twist itself in the other direction. This is a child’s toy in many villages, so get one of your students to help you.}
\item{Theory: For light, white is the presence of all colors (this is opposite for pigment). Theoretically, you could do this demo using only the three primary colors, but it might be harder to get the top to spin fast enough to lose all resolution of the colors. By using the seven rainbow colors, we make it easier for them to blend together; creating white as far as the eye is concerned.}
\end{itemize}

\subsection{Thin Film Interference}
\begin{itemize}
\item{Preparation time: 5 minutes}
\item{Materials: A small bowl or other dish, water, oil}
\item{Procedure: Pour water into the dish. Touch your finger to the oil, then to the surface of the water. A small amount of oil should be transferred to the surface of the water, where it will form a thin film. A colorful rainbow pattern should be visible in the thin film. Try looking at it from different angles. If you are having trouble seeing the colors try moving the dish into brighter light (direct sunlight works well), or using a dark-colored dish.}
\item{Theory: When light strikes the surface of the water, some is reflected off the top of the film of oil, and some is reflected from the oil-water interface. When the difference in the path length between these two paths is an integer number of wavelengths, light of that wavelength will be strongly seen. This gives rise to the rainbow pattern.}
\end{itemize}

\subsection{Water Prism}
\begin{itemize}
\item{Preparation Time: 1 minute}
\item{Materials: mirror, clear rectangular container, water, white light source}
\item{Procedure: Fill the container with water. On the inside of one of the sides, place the mirror with the reflective side facing in. In a dark place, shine a light through the opposite side of the container at an angle so that the light passes through the water, reflects off the mirror, and exits the container on the original side. The light leaving the container should be dispersed into the color spectrum.}
\item{Theory: Light refracts when entering a dense medium like water. As white light is refracted, each color that makes up the white light (the whole color spectrum) refracts at a different angle depending on its wavelength, and so a refracted ray creates a slight rainbow pattern. Normally this effect can be seen only partially as white light passes through water, but as you are refracting the water twice, once into the water and once back into air, the dispersion effect will have twice the magnitude.}
\end{itemize}

\subsection{Refraction of light through water}
\begin{itemize}
\item{Preparation Time: 5 minutes}
\item{Materials: cardstock or cardboard, jar, water, Nido or powdered soap, light source}
\item{Procedure: Cut a small hole, about half a cm, in the cardstock. Put some Nido or soap into the water in the jar so that it becomes cloudy. Shine the light through the hole in the card so that a thin beam can be seen in the cloudy water on the other side. Change the direction of the beam through the water to see the different refracted angles.}
\item{Theory: The Nido or soap provides particles in the water that will reflect light, clearly showing the path of the light through the water (picture headlights on a foggy day). Light slows down as it enters a denser medium, like water, from a less dense medium, like air. As such, the direction of the light changes in order to reduce the traveling time through the medium. This effect, called refraction, can be seen in the cloudy water.}
\end{itemize}

\subsection{Rectangular Prism}
\begin{itemize}
\item{Preparation Time: 1 minute}
\item{Materials: rectangular prism (available in a lab store for about 6,000/=), paper, cardboard, sewing pins or surgical needles, pencil, protractor}
\item{Procedure: Place the paper on the cardboard and secure it with staples or tape on the edges. Place the prism flat on the middle of the paper and trace its outline with the pencil. Draw an incident ray and its respective normal on the paper and then place two pins – one next to the prism and the other farther away – vertically on the line representing the incident ray.\\
When you look through the prism from the other side, you should see the two pins clearly. Line them up so that they look like one pin; now place two more pins on this side so that they also line up with the pins of the incident ray. Now you should have four pins creating two parallel lines, as shown here:\\
Trace the line connecting the two new pins all the way to the prism. This is the ray as it leaves the prism. Draw its respective normal. Now, inside the prism, you can connect the two rays with a line through the glass prism. Do this and measure the resulting angle of refraction. When this is done, you can calculate the index of refraction of glass.}
\item{Theory: An single ray of light incident on a surface between two media will be subject to Snell’s law:  where n1 is the index of refraction of the first medium (air, in this case), n2 is the index of refraction of the second medium (glass), i is the angle of incidence, and r is the angle of refraction. Since we can measure the two angles easily with a protractor on the paper, and we know the index of refraction in air to be 1.0, we can calculate the index of glass, around 1.52.\\
Since we do not have a point source of light, we use the pins to represent a single ray that would start at one and pass through the other. In a way, the four pins are one ray of light, at least for our purposes.}
\end{itemize}

\subsection{Pouring light}
\begin{itemize}
\item{Preparation Time: 5 minutes}
\item{Materials: Opaque container (Nido can), nail, flashlight, regular container, water}
\item{Procedure: Use the nail to poke a hole at the bottom of one side of the opaque container. Fill the container with water and allow the water to pour out into the other container. In a dark place, shine the flashlight down through the top of the opaque container: you will see the water glow as it is poured out.}
\item{Theory: The light is reflected at the surface of water (total internal reflection), so when it travels through the stream of pouring water, it continues to be reflected inside the stream until it reaches the container below. The light that does escape the pouring water is what we see as the glowing effect.}
\end{itemize}

\section{Lenses}
subsection{Focusing an Image through a Convex Lens}
\begin{itemize}
\item{Preparation Time: 1 minute}
\item{Materials: convex lens (magnifying glass on a Swiss army knife works well), white paper or screen, tissue paper (the paper used to wrap the Rexa toilet paper is perfect), pen, point light source (your headlamp, desk lamp, etc.), optional retort stands}
\item{Procedure: cut a piece of tissue paper to fit over your light source. Flatten this paper in a book overnight if necessary. Draw a thick arrow on this tissue paper and tape it over your light source. With students, set up the light source to shine directly on a white screen or paper about half a meter away. The distance depends on how strong the light is). Move the magnifying glass/convex lens back and forth between the light and screen until the image of the arrow is focused on the screen. Measure the distances from the lens to the screen and lens to the light source. Now you can calculate the focal length of the lens.}
\item{Variation: Fry some bugs with sunlight! If the sun is strong enough, you should be able to get paper to smoke, and no one likes siafu anyway.}
\item{Theory: The lens equation is given as 1/f = 1/u + 1/v where f is the focal length of the lens, u is the distance from the object to the lens, and v is the distance from the focused image to the lens. In our case, the object is out light source and arrow, and the image is on the white screen. By focusing the image, we set u and v, allowing us to calculate f.}
\end{itemize}

\section{Measurement}

\subsection{Measurement Errors}
\begin{itemize}
\item{Preparation time: 1 minute}
\item{Materials: Meter sticks or stopwatches}
\item{Procedure: Ask for several students to volunteer to help. If using stopwatches, tell them to measure the time between two claps that you will give. Clap once, at which time the students should start their stopwatches. After a period of several seconds, clap a second time, at which point the students should stop their stopwatches. Make a simple table of their results, including several intervals. Each student should have a slightly different measurement. As they were all measuring the same event, this shows that their measurements contain errors.\\
Alternately, place a chalk mark on the wall at a height of more than 1 meter above the floor. Give several students a meter stick, and ask them to measure the height of the mark. Again, their answers should all be slightly different, because measurements always contain an error.\\
For this demonstration, it may sometimes happen that the first student will report a certain value, and then all of the following students will agree with the first value, regardless of what they have measured. The students should not all get the same value. To prevent this false agreement, it may be necessary to have each student first write their measurement on a small piece of paper, and then hand all of the papers to you.}
\end{itemize}

\subsection{Beam balance}
\begin{itemize}
\item{Preparation Time: half hour}
\item{Materials: Coat hanger, retort stand or other support, cardboard, pen, two water bottle bottoms, string, tape}
\item{Procedure: Hang the coat hanger from the retort stand or support so that it is free to swing. Cut out a strip of cardboard with a single centerline and tape it to the stand upright. Cut out a cardboard pointer and tape it to the coat hanger so that when the hanger hangs level, the pointer lines up exactly with the centerline on the cardboard strip. Any swinging of the hanger should cause the pointer to drift from the line. Now hang the water bottle bottoms with string from each corner of the hanger (you can bend the hanger a bit so that the string does not slide in; the bottles act as scale pans. Make sure that the hanger still hangs level when nothing is placed in either scale pan; calibrate with extra bending or mass as necessary. With a little bit of fidgeting, you should have a decent beam balance ready for use!}
\item{Theory: Beam balances do not need to be calibrated to specific masses; as long as they indicate clearly when two masses are equal, it is enough. If you have a set of known, graduated masses, you can do specific measurements.}
\end{itemize}

\subsection{Spring balance}
\begin{itemize}
\item{Preparation time: 1 hour}
\item{Materials: pen springs, paper, ruler, known masses, pen, eye-hooks, glue}
\item{Construction: Hang the spring(s) from an eye-hook in whatever frame you choose. From the bottom of the springs, hang the other eye-hook such that any weight can be hung again from it. On the bottom eye-hook place a pointer (paper, needle, etc.) facing sideways, then glue the paper over the frame so that the springs are free to move up and down and the pointer always points to some point on the paper (toothpaste boxes, glycerin boxes, etc. work well). Mark the pointer’s position when the springs hang freely, then when they hold 1 gram, 2 grams, etc.\\
You now have a spring balance, though you will have to do your own work to make it smooth and structural depending on your materials. You can measure mass, acceleration due to gravity, and the spring constant. Every spring has a constant.}
\end{itemize}

\section{Magnetism}

\subsection{Suspended Magnet Compass}
\begin{itemize}
\item{Preparation Time: 1 minute}
\item{Materials: thread, bar magnet, Optional: second magnet}
\item{Procedure: Tie the thread around the bar magnet’s center so that it hangs horizontally and is free to spin. Allow it to settle and you will see that it points north and south. Turn it away and allow it to settle again. Rotate your hand and the magnet will stay facing north and south. If you have a second magnet, pass it by the suspended magnet and watch the suspended magnet try to face the other magnet. Take the magnet away and the suspended magnet will return to its original direction.}
\item{Theory: A magnet will naturally align itself with the Earth’s magnetic field. Usually there is too much friction for this to happen, but a suspended magnet is free to face North and South. Even if you try to confuse it by turning it or by bringing another magnet close, it will eventually align itself with the earth’s field.}
\end{itemize}

\subsection{Magnetic Dip Gauge}
\begin{itemize}
\item{Preparation Time: 15 minutes}
\item{Materials: magnet, sewing needle, cork, two pins, paper, pen, cardboard or metal strip}
\item{Procedure: Push the two pins into the ends of the cork to create an axle. Push the sewing needle through the cork perpendicular to the axle pins so that the needle rolls end-over-end when you roll the cork/pins between your fingers. Adjust the needle so that it rests horizontally when the cork is free to pivot (equilibrium). Use the magnet to magnetize the needle without changing its position in the cork.\\
Bend the metal or cardboard strip into a U-shape to create a stand for the cork and pins. Rest the pins on each style of the U-stand so that the needle is free to rotate vertically. If you like, cut out a semicircular piece of paper and label the angles 0 – 90 degrees on it; tape or glue this to the stand. The needle will rotate, or dip, to point in the vertical direction of the earth’s magnetic field.}
\item{Theory: Earth’s magnetic field is not level across the surface of the earth: it goes into or out of the ground at an angle depending on the latitude. The angle of the field relative to the surface is called Magnetic Dip and is measured with this needle.}
\end{itemize}

\subsection{Mapping Magnetic Fields}
\begin{itemize}
\item{Preparation Time: 1 minute}
\item{Materials: bar or horseshoe magnet, iron wool, piece of white paper}
\item{Procedure: Place the magnet on a table and the paper over the magnet. Using your thumb and forefinger, rub the iron wool above the paper. Small pieces of iron should fall onto the paper, gradually mapping out the field of the magnet below. Move the wool around as you do this to try to show the field in a wide area. If the magnet is too strong, put some space between it and the paper. Try this with two magnets, showing attraction and repulsion between the poles. Note that the field is strongest at the poles.}
\item{Variation: Pour the filings into a container of viscous fluid (play around with glycerin and others). Shake the container so the filings are distributed around the fluid. Hold a magnet next to the container; the filings will arrange themselves into the 3D pattern of the field.}
\item{Theory: Magnetic fields extend from the North Pole of a magnet to any South Pole. Iron filings are small enough that they can form patterns in any magnetic field, showing the shape and the relative strength and various points on the field. If the field is strong enough, the filings will also form a 3D structure.}
\end{itemize}

\subsection{Pin compass}
\begin{itemize}
\item{Preparation Time: 1 minute}
\item{Materials: small pin, magnet, small dish of water}
\item{Procedure: Magnetize the pin by stroking it with one pole of the magnet; use this time also to review methods of magnetization. Place the pin gently on the surface of the water so that it does not sink (you can review surface tension here if you like), watch as it rotates to face north and south.}
\item{Theory: The earth is a magnet and its field lines can be seen using a compass, as a compass itself is a magnet and will align itself with any magnetic field. By magnetizing the pin, you make it into a compass needle, which will naturally align with the earth’s field, and the water allows it to pivot freely.}
\end{itemize}

\subsection{Magnetizing a Nail}
\begin{itemize}
\item{Preparation Time: 1 minute}
\item{Materials: nail, insulated wire (speaker wire), 2 or more D-cell batteries}
\item{Procedure: coil the middle of the wire around the nail to create a solenoid. Connect the two ends of the wire to the battery. The nail and connectors will become hot and the nail will become magnetized. You can use it to pick up staples, paper clips, etc.}
\item{Theory: The moving electric charge in the wire solenoid creates a magnetic field in the nail (use the RHR), aligning the “domains” in north-south. The stronger the current is, the stronger the magnetic field and therefore the stronger the magnet. If you use another material, you will find that the magnet is not as strong as the iron nail.}
\end{itemize}

\section{Newton’s Laws}

\subsection{Tin Can Piñata}
\begin{itemize}
\item{Preparation Time: 5 minutes}
\item{Materials: Two cans or buckets, sand, string or rope, stick}
\item{Procedure: Hang the two cans/buckets from the stick with the string so that they hang at equal lengths. Pour a small amount of sand in one can and a large amount in the other. Support the stick between two desks and start the cans swinging. Have students stop each can, feeling the difference in the force it takes to stop the almost empty can as opposed to the full can.}
\item{Theory: Inertia, or momentum, of an object is directly proportional to its mass. The full can, therefore, has more inertia and will tend to continue its motion more than the empty can. You can also offer to throw a piece of chalk or a desk to a student. They usually choose the chalk.}
\end{itemize}

\subsection{Magic Card Trick}
\begin{itemize}
\item{Preparation Time: none}
\item{Materials: empty soda bottle, card, heavy coin}
\item{Procedure: Place the card over the mouth of the bottle and let the coin rest on top. Invite students to try to remove the card without moving the coin. Most will not be able to do it. Flick the card quickly from the side; it should fly off the bottle, leaving the coin resting neatly on top of the bottle.}
\item{Theory: The coin has inertia, meaning it will resist any changes to its motion. Despite the friction from the card pulling the coin off the bottle, the coin will remain in place. This is also a good demonstration of impulse.}
\end{itemize}

\subsection{Balloon Rocket}
\begin{itemize}
\item{Preparation Time: 0 minutes – easy: 15 minutes – advanced}
\item{Materials: easy: balloon; advanced: also 2 m (or longer) string, nails, paper, tape, 1 large rubber band, paper clip}
\item{Procedure:
\begin{itemize}
\item{Easy – Inflate the balloon by blowing into it. When it is big, release the balloon. It will fly around the room.}
\item{Advanced – Cut paper into a strip about 5 cm by 10 cm. Roll the paper strip into a cylinder 5cm long, with a small diameter, maybe 0.5 cm. Tape the cylinder so it stays, and attach the rubber band with tape. Put the string through the cylinder. Attach the ends of the string to nails in the ceiling, or perhaps stretched between 2 retort stands (or even have students holding the ends) so that the string is horizontal. Put the paper clip on the string. Inflate the balloon and then use the rubber band to hold the big part to the paper, and attach the mouth of the balloon to the paper clip. Release the balloon and it will shoot across the string. This demonstrates the same principles as the “easy” version above, but because the balloon goes in a straight line, it is somewhat easier to see.}
\end{itemize}
} % Procedure
\item{Theory: The balloon pushes the air out, so there is an equal and opposite force of the air pushing the balloon. Momentum is conserved; as the air goes backwards, the balloon goes forwards.}
\end{itemize}

\subsection{Bottle Rocket}
\begin{itemize}
\item{Preparation Time: 1 hour}
\item{Materials: empty 500 ml water bottle, nail, rubber stopper, straight pin, bicycle pump, needle attachment for pump (the type used to fill a football), tape, old pen, rigid straight wire (approx 1 meter), water.}
\item{Construction: Make a small round hole (between 0.5 and 1.0 cm in diameter) in the lid of the water bottle. One easy way to do this is to heat the head of a nail until it is hot, and then use it to melt a hole in the lid. Cut a round piece of the rubber stopper so that it can be used to stop this hole. The stopper should form a good seal in this hole, but it should be possible to push the stopper through the hole by exerting some force on it. Pierce the stopper with a straight pin (if may help to heat the straight pin first) so that you can pass the needle attachment for a bicycle pump through the stopper (see figure 1, page 6). You should be able to put the stopper in the hole inside of the lid, and insert the needle attachment through the stopper so that you can increase the pressure inside of the bottle.@Disassemble a pen and cut the body so that you have two hollow cylindrical pieces approximately 3cm long each. Affix them to the side of the bottle using adhesive tape. They should be in a straight line with each other.}
\item{Procedure: This demonstration should be done outside. Insert the rigid straight wire into the ground. Fill approximately half the bottle with water. Put the stopper on the inside of the lid. Put the needle attachment through the stopper. Put the lid on the water bottle and tighten. Pass the rigid wire through the pen cylinders, and lower the bottle to the ground (see figure 2, page 6) Pump the bicycle pump. Once the pressure in the bottle becomes great enough, the stopper will be forced out of the bottle, and the rocket will fly into the air. It should be possible to reach a height of 10 meters or more.}
\item{Theory: When the stopper leaves the bottle, pressurized air forces water out of the bottom of the bottle at a high speed. Just as with the matchstick rocket and the balloon rocket, this results in a reaction force forwards on the rocket. As with the matchstick rocket and balloon rocket, we can also consider this from the perspective of conservation of momentum.\\
N.B.: After the bottle rocket fires, find the rocket and note the thick white fog that appears inside of the bottle. This is caused because the rapid expansion of the air during firing is adiabatic, causing cooling, lowering the temperature of the gas below its dew point.\\
Figure 1\\
Figure 2}
\end{itemize}

\subsection{Matchstick Rocket}
\begin{itemize}
\item{Preparation time: 5 minutes}
\item{Materials: Matches, straight pin, metal foil, scissors, small rock}
\item{Construction:
\begin{itemize}
\item{For this demonstration, I have found that the metal foil from underneath the lid of a Blue Band container works best. If using this foil, make sure you have cleaned off any Blue Band that may have adhered to it. Normal aluminum foil works as well.}
\item{Different brands of matches work better or worse. Best results have been found with Lucky brand matches, although others have also worked well (the current record for Kasuku matches is 3m). You should attempt this demonstration once or twice yourself with a certain type of matches before doing it in from of the class.}
\item{Cut out a rectangular piece of foil approximately 2cm x 4cm. Place the straight pin along the length of the match, with the point of the pin touching the match head (see figure 1). Wrap the foil tightly around the match and pin, with about half of the foil extending past the tip of the match (see figure 2). Fold the extra foil securely down over the tip of the match (see figure 3). Remove the straight pin. You have now finished constructing a matchstick rocket.}
\end{itemize}
} % Construction
\item{Procedure: Prop the matchstick rocket on two small, smooth rocks so that it is at approximately a 45° angle. Light another match and hold it underneath for several seconds (see figure 4). The heat will cause the matchstick rocket to ignite. It should fire for a distance of between one and five meters.\\
Once you have practiced on several matchstick rockets, you should be able to make several of them per minute. It is therefore easy and worthwhile to bring several of them to class, so that you can repeat the demonstration several times. This also gives you an opportunity to invite several students to launch a rocket themselves. If time permits, have a contest to see who can get the best distance on a rocket.}
\item{Theory: When the matchstick rocket ignites, rapidly expanding hot gases are produced. These are only able to escape by following the pathway left behind by the straight pin. The hot gases are forced backwards from the rocket at a high speed. Newton’s Third Law of Motion tells us that for every force there is an equal and opposite counter-force. Because the hot gases are being forced backwards, there must be a counter-force pushing the rocket forwards.\\
Alternately, we can consider Conservation of Momentum. Initially, the matchstick rocket is at rest. Once it ignites, hot gases develop a backwards velocity. Because momentum must be conserved, some other part of the system must develop a forward velocity. Thus, the rocket will fly forwards.}
\end{itemize}

\section{Projectile Motion}

\subsection{Object Toss}
\begin{itemize}
\item{Preparation time: none}
\item{Materials: Any object(s)}
\item{Procedure: When teaching projectile motion, it is productive to throw objects in the classroom. This is useful, and extremely simple. Almost any object may be used. In the past, I have used the keys from my pocket, lemons from the lemon tree next to our classroom, small pieces of chalk, and my coffee cup.\\
One good demonstration consists of repeatedly throwing an object vertically up in the air and then catching it when it returns to your hand. Point out that when you first throw it up, it has an upward velocity. As it moves up, the velocity becomes less. At the top of its trajectory, it momentarily has a zero velocity. After that, it gains a downward velocity, at first a small one and then increasing in magnitude.\\
If you are walking across the classroom at a constant rate while performing this demonstration, you can additionally show that the projectile continues to move horizontally at the same rate, matching your motion. This shows that the horizontal velocity of a projectile is a constant.}
\end{itemize}

\section{Pressure}

\subsection{Hydraulic Press}
\begin{itemize}
\item{Preparation Time: 15 minutes}
\item{Materials: Two syringes of different sizes (50 ml and 20 ml work well), thin rubber tubing, water}
\item{Procedure: Fill the larger syringe with water and attach one end of the rubber tubing to its end. Attach the other end of the tubing to the smaller syringe (the plunger should be inserted all the way in the smaller syringe). Pushing the plunger of the larger syringe will cause the plunger of the smaller syringe to go out, and vice-versa. You will notice that it is easier to push the plunger of the small syringe than that of the larger syringe.}
\item{Theory: Pressure is equal to force per area, and the pressure is distributed equally throughout a liquid. As such, the pressure at one plunger must be equal to the pressure at the other plunger. Setting the two ratios equal, we can see that a small force over a small area can overcome a large force over a large area.}
\end{itemize}

\subsection{Holey Bottle 1}
\begin{itemize}
\item{Preparation time: 5 minutes}
\item{Materials: empty 1.5 liter water bottle, water}
\item{Construction: Pierce three or more small (less than 0.5cm) neat holes in the water bottle, at different vertical heights. An easy way to do this is to use firm but gently pressure with a metal needle from a syringe. Make sure to pierce the bottle in parts that are vertical, not parts that slope in or out.}
\item{Procedure: Fill the bottle with water and place it on the ground or on a table. Water will pour out of the bottle through the small holes. Note that the streams of water strike the ground at different distances from the bottle. At any given time, the hole that is closest to one-half of the height of the water level should strike the ground or table at the greatest distance from the bottle.}
\item{Theory: To understand this we must consider both Bernoulli’s Principle and Projectile Motion. Bernoulli’s Principle tells us that the horizontal speed of each stream of water varies as the square root of the depth from the surface of the water. Projectile Motion tells us that the distance reached by a stream of water is proportional to the velocity of the water and to the time before the water strikes the ground. The time before the water strikes the ground is proportional to the square root of the height above the ground. Combining these two factors, we find that the maximum distance will be reached for a stream of water that is halfway between the ground and the surface.}
\end{itemize}

\subsection{Holey Bottle 2}
\begin{itemize}
\item{Preparation Time: 10 minutes}
\item{Materials: Water bottle, pin or small nail, water, bucket (for catching water)}
\item{Procedure: In even intervals around the base of the water bottle, poke small holes with the pin or nail. Try to get an even distribution and the same size hole all around. Fill the bottle with water and watch the water leave each hole with the same force. Blowing into the bottle will help illustrate the equality of the pressure in all directions.}
\item{Theory: Pressure in a fluid acts equally in all directions, therefore the water being forced out the bottom should feel the same amount of pressure and shoot the same distance.}
\end{itemize}

\subsection{Straw Fountain}
\begin{itemize}
\item{Preparation Time: 10 minutes}
\item{Materials: 0.5 liter water bottle with cap, water, straw, glue}
\item{Procedure: Poke a hole with the diameter of the straw in the cap of the water bottle with a hot nail or pin. Insert the straw so that it extends almost to the bottom of the water bottle and leaves enough sticking out for your mouth. Secure it with glue so that it is airtight. When the glue is dry, fill the bottle about half way with water and close the cap with the straw inside. Have a student blow as hard as they can through the straw into the water. When they run out of air and stop blowing, they will get a nice spray in the face. }
\item{Theory: By blowing into the bottle, you greatly increase the pressure inside. When you finish blowing, the pressure will try to equilibrate by forcing the pressure back out through the straw. There is nowhere for the water to go but out.}
\end{itemize}

\subsection{Siphon}
\begin{itemize}
\item{Preparation Time: 1 minute}
\item{Materials: two containers, half meter of rubber tubing/IV line/feeding tube, water}
\item{Procedure: Place one jar with water on a table and the other empty jar on a chair just below the table. Place one end of the tubing into the water and the other in your mouth. Suck on the tube until the water starts coming out and place the end of the tube into the empty beaker, holding the middle of the tube at the level of your mouth. The water will continue to flow from the water jar to the empty jar, despite the water’s initial uphill climb.}
\item{Theory: By sucking on the tubing, you create low pressure on that side. The slightly higher pressure (atmospheric) at the water will cause the water to continue to travel as long as the pressure difference is enough to overcome gravity. If you raise the middle of the tube too high, the water will stop flowing.}
\end{itemize}

\subsection{Balloon Pop}
\begin{itemize}
\item{Preparation Time: 20 minutes}
\item{Materials: piece of wood, nails, water balloons, water}
\item{Procedure: Put one nail through the board in one place and a large cluster of closely spaced nails in another place, all pointing up. Fill a balloon with water. As you lower the balloon onto the single or few nails, the balloon eventually pops. Fill another balloon with water and slowly lower it onto the cluster of nails. It should not pop.}
\item{Theory: As area of a force increases, pressure decreases. Therefore, as more nails are added and the area of the force (the weight of the balloon) increases, the pressure decreases and the balloon does not pop. Or, it takes more force to pop.}
\item{Alternative: hang the balloon from a spring balance as you lower it (by holding the spring balance) onto the nails. The difference in weight will allow you to calculate the force needed to pop the balloon.}
\end{itemize}

\subsection{Potato Poke}
\begin{itemize}
\item{Preparation Time: none}
\item{Materials: some straws, potato}
\item{Procedure: Take a straw and jab it into the potato. The straw should bend easily leaving the potato unharmed. Now place your thumb firmly over one end of a straw and jab the other end into the potato. This time the straw should enter the potato quite easily.}
\item{Theory: The straw is weaker than the potato and so will bend rather than break the potato’s skin. But, with your thumb plugging the back of the straw, the air inside the straw cannot leave and instead pushes out against the sides of the straw. As the straw strikes the potato, it cannot bend with the air pressure inside and so instead can poke through the skin into the potato.}
\end{itemize}

\subsection{Straw Elevator}
\begin{itemize}
\item{Preparation Time: none}
\item{Materials: two straws, container, water}
\item{Procedure: Fill the container with water and insert one straw so that it stands vertically in the water. Using the other straw, blow across the opening of the vertical straw; the water level in the straw will rise.}
\item{Theory: Bernoulli’s Principle states that moving air causes low pressure; the air passing in a stream over the vertical straw creates low pressure and therefore a pressure differential between the bottom of the straw (the water) and the top. The water will move towards the lower pressure, moving up the straw.}
\end{itemize}

\subsection{Reverse Air Pump}
\begin{itemize}
\item{Preparation Time: varies, about 1 hour}
\item{Materials: Bicycle pump (the tall, metal kind), short piece of rubber tubing fitted to pump valves, utility knife, tightening sleeves, extra valve}
\item{Procedure: There are two parts of the pump that control the direction of airflow: the first is a diaphragm inside the pump and the second is a ball valve at the base of the pump in the hose.
\begin{enumerate}
\item{You need to open the pump and pull out the ‘dipstick’ with the diaphragm attached. At the bottom, there should be a diaphragm with holes around the top, a metal disc the same diameter as the diaphragm, and a few nuts and washers to keep it all together. In its normal configuration, the diaphragm is pulled down by friction away from the disc when the pump handle is pulled up, allowing air to enter the pump freely. When the pump handle is pushed in, the diaphragm is forced against the disc, restricting any back airflow, and forcing all the air forward through the hose. Switch the position and direction of the diaphragm and disc so that it has the opposite effect when the pump handle is pulled in or out.}
\item{Next, you need to cut open the hose at the base of the pump and find the valve with the small bead inside. Normally, when air is forced forward through the valve, the bead does not restrict any airflow. When air tries to go back through the pump, the bead blocks the valve and stops any airflow. Switch the direction of the valve.}
\item{From here, you need to reattach the hose to the pump. You may need to get another nozzle to attach to the pump, attaching the hose with reversed valve with the extra bit of rubber tubing. It depends on your pump, but if you have made it this far, you will find a way to make it work. Tightening sleeves will come in handy here to make sure no air is lost after all this cutting and jury-rigging.}
\end{enumerate}
} % Procedure
\item{Applications: This suction pump is great for showing the gas laws and boiling points: suck the air out of a jar of water and watch the water boil, you could also kill stuff in the jar this way, but that is just morbid, and possibly cool, or that sound travels through a medium.}
\end{itemize}

\subsection{Atmospheric Pressure}
\begin{itemize}
\item{Preparation Time: 5 minutes}
\item{Materials: Water bottle, pin and/or nail, water}
\item{Procedure: Using an empty water bottle (bigger is better), poke four or five small holes (0.5 cm) in the bottom with the pin and then the nail. Fill the bottle about half way with water, allowing it to spill out through the holes in the bottom. While the students are watching, seal the cap on the bottle. The water will cease to pour out of the bottom despite the holes and rather predictable effect of gravity. When the gasps of wonder die down, discuss the following:}
\item{Theory: The pressure of the water combined with the pull of gravity is enough to cause the water to pour through the holes in the bottle when the cap is not sealed. When the cap is on tight, however, the combined high air pressure outside the bottle and low air pressure inside the bottle creates enough of an upward force on the water to counter the downward force of gravity.}
\end{itemize}

\section{Properties of Matter}

\subsection{Water drops}
\begin{itemize}
\item{Preparation time: none}
\item{Materials: Water dropper or syringe}
\item{Procedure: Slowly drip water from the water dropper or syringe and point out that before a drop falls; it will hang suspended by its surface tension. Explain that as the drop becomes larger, its weight increases until surface tension is insufficient to support it, at which point it falls.}
\end{itemize}

\subsection{Blowing bubbles}
\begin{itemize}
\item{Preparation time: 5 minutes}
\item{Materials: Thin piece of wire (approximately 30cm), water, detergent, glycerin (optional)}
\item{Procedure: Bend the wire into a loop 2 to 3 cm in diameter. Continue to bend the wire so that it circles around the circumference of this circle several times. Leave a straight piece several centimeters long to use as a handle. This is the “bubble blower”. Dip the circular part of the bubble blower into a strong solution of detergent (regular powdered laundry detergent works well) mixed with glycerin. When you remove the bubble blower from the solution, a thin film should remain across the circle. Gently blow through the center of the circle. With a little practice, you should be able to cause a spherical bubble to separate from the blower and float away.}
\item{Theory: The detergent causes the surface tension in the solution to be slightly variable. In areas of higher concentration of detergent, the surface tension is lower. In order for the films and bubbles to be stable, the surface tension near the top must be slightly higher than at the bottom. As the detergent molecules are heavier than water, they tend to sink towards the bottom of the film, accomplishing this.\\
When you blow through the bubble blower, we can see that then tension is pulling it back towards a flat surface. Once an independent bubble is formed, we see that it forms a nearly perfect sphere. This is because the surface is under tension. This tension forces the bubble to form the shape with the minimum surface area, a sphere. It is also worth noting that both the film that stays on the bubble blower and the bubbles themselves appear to have small rainbows of colors in them. This is caused by thin-film interference.}
\end{itemize}

\subsection{Pin Float}
\begin{itemize}
\item{Preparation time: none}
\item{Materials: A cup or small dish, a straight pin, water, detergent}
\item{Procedure: Make sure the cup or dish is clean, and has no soap or detergent residue. Fill the cup or dish with clean water. Carefully place the straight pin on the surface of the water, being careful not to break the surface. If done properly, it should be possible to get the straight pin to remain suspended on the surface (see also floating compass). Next, sprinkle a small amount of detergent onto the water. The pin should sink to the bottom.}
\item{Theory: When the straight pin is placed on the surface, it causes the surface of the water to bend downwards. This means that the surface tension of the water is pulling the pin upwards. Although the pin is denser than water, and would normally sink, this surface tension is enough to support the weight of the pin. When detergent is sprinkled onto the surface of the water, it lowers the surface tension of the water. The surface tension is no longer strong enough to hold up the pin, so the pin sinks.}
\end{itemize}

\subsection{Pepper Float}
\begin{itemize}
\item{Preparation time: none}
\item{Materials: A cup or dish, water, ground black pepper, soap or detergent}
\item{Procedure: Make sure that the cup or dish is clean, and has no soap or detergent residue. Fill the cup or dish with clean water. Sprinkle ground black pepper over the surface of the water in a way that the pepper is distributed evenly and covers the whole surface. Next, apply a small amount of soap or detergent to one finger. Touch this finger to the surface of the water in the center of the cup or dish. The pepper will flee your finger, and all run to the sides of the cup or dish.}
\item{Theory: When you touch your finger to the surface, you introduce a small amount of soap or detergent, lowering the surface tension at that point. The surface of the water is now unbalanced – the surface tension near the edge is pulling the surface outwards more strongly than the surface tension at the center is pulling the surface inwards. As there is a net force on the surface outwards towards the edge, the surface moves, pulling the pepper along with it to the edges of the cup or dish.}
\end{itemize}

\subsection{Water Dome}
\begin{itemize}
\item{Materials: Coin, water, syringe or eyedropper}
\item{Preparation Time: none}
\item{Procedure: Place a coin flat on the table. Using the syringe or eyedropper, carefully drop individual water drops onto the coin. With some practice, you should be able to get quite a few drops onto the coin before the water spills over, creating a dome of water.}
\item{Theory: The surface tension of the water holds it together against the force of gravity, which is trying to pull the water off the coin.}
\end{itemize}

\subsection{Pinching Water}
\begin{itemize}
\item{Preparation Time: 10 minutes}
\item{Materials: 0.5 liter bottle, water, pin or small nail}
\item{Procedure: At the bottom of the side of the can or bottle, poke five small holes close together with the syringe needle or nail. Be careful not to let the holes overlap or be too far apart. Pour water into the bottle and allow the water to start flowing out of the holes at the bottom. Using your thumb and forefinger, pinch the streams of water together so that they form a single stream (this takes some practice). To undo your great work of creation, pass your hand in front of the holes and five streams will appear again.}
\item{Theory: Water has a tendency to “cling” to itself due to its surface tension and cohesion. As you bring the streams together, you allow the water to stick to itself forming a single stream. Passing your hand in front again stops the flow of water at the holes and allows it to start again, which it will do in five streams.}
\end{itemize}

\subsection{Lemonade}
\begin{itemize}
\item{Preparation Time: 5 minutes}
\item{Materials: lemon, drinking water, pitcher}
\item{Procedure: Make lemonade by putting lemon wedges (oranges also work) into the pitcher and adding about a liter of water. Let it sit for a couple hours, then drink and enjoy! Adding sugar or honey is recommended.}
\item{Theory: The citrus flavor of the lemons will gradually spread throughout the water, though no force is apparent. This process is called diffusion. See the Transport topic in the Biology section for more activities involving diffusion.}
\end{itemize}

\section{Rotation}
subsection{Object Spin}
\begin{itemize}
\item{Preparation Time: 1 minute}
\item{Materials: toilet paper tube, string, or rope 2 m, two objects of any kind (masses)}
\item{Procedure: Pass the half rope through the cardboard tube and tie each end to a mass of some kind. Holding the tube vertically in your hand, get the rope end on top spinning over your head. As you increase the speed of the spinning end, the rope should slide up through the tube, creating a longer length of rope spinning and shorter length hanging below the tube. Decrease the speed and allow the rope to settle back through the tube.}
\item{Theory: Centripetal force is related to the rotational speed of the object, so as the speed increases, the force with which it pulls outward increases. As this happens, the centripetal force of the spinning object overcomes the gravitational force on the hanging object, pulling it up.}
\end{itemize}

\subsection{Twenty Shillings of Equilibrium}
\begin{itemize}
\item{Preparation Time: none}
\item{Materials: Meter rule, handful of 20/= coins}
\item{Procedure: Balance the meter rule on your forefinger at the 50 cm mark. Discuss the idea of equilibrium and equal moments. Place one coin at the 1 m mark and watch the rule rotate and the coin fall, demonstrating a lack of equilibrium. Now replace the coin and add another coin to the 0 m mark. The meter rule should stay steady. For more numerical fun, add two coins at the 25 cm mark, four coins at the 12.5 cm mark, and even eight coins at 6.25 cm if you dare! Have students calculate the necessary lengths or masses (number of coins – we do not need actual units here since the masses are equal) while you perform the daring feats in front.}
\item{Theory: The moment of a force is equal to the force times the distance from the pivot. The pivot is your finger and the forces are all at even intervals. If the total moment in the clockwise direction is equal to the moment in the anticlockwise direction, you have equilibrium. You can add or remove coins, change distances, etc. while still keeping the moments equal.}
\end{itemize}

\subsection{Windmills}
\begin{itemize}
\item{Preparation Time: 15 minutes}
\item{Materials: thin cardboard or cardstock, scissors, pen, colored pencils/markers if desired, glue, paper fastener or thumb tack, straw or stick}
\item{Procedure: Use the following illustration (enlarge it); copy it onto a piece of thin cardboard or cardstock.}
\begin{enumerate}
\item{Cut along the lines and make holes with a pencil or pen.}
\item{Bend the four corners together into the center and glue them in place.}
\item{Push the fastener or tack through the center hole into a straw or stick.}
\end{enumerate}
\item{Reference: This demo was published in the Science Lab Kit by Silver Dolphin Books in 1997, compiled by Brenda Walpole}
\end{itemize}

\subsection{Helicopters}
\begin{itemize}
\item{Preparation Time: 15 minutes}
\item{Materials: paper, scissors, paper clip}
\item{Procedure: Copy the following design onto a piece of paper. Cut along the solid lines and fold along the dotted lines, attaching the paper clip to the bottom. Drop the helicopter with the paperclip down and watch it spin!
*This demo was published in the Science Lab Kit by Silver Dolphin Books in 1997, compiled by Brenda Walpole}
\end{itemize}

\subsection{Gyroscope}
\begin{itemize}
\item{Preparation time: 30 minutes}
\item{Materials: rubber strip (at least 4 pieces, about 40cm long), weights (at least 4, ½ kg is good), bicycle wheel, rope}
\item{Construction: The weights are used to increase the moment of inertia of the bicycle wheel. Use the rubber strip to tie firmly the weights to the bicycle wheel. Space them evenly so that the weight is equally distributed. Using four ½ kg weights works well, six works even better. Then, tie the rope in a loop to something strong enough to support the bicycle wheel. Arrange it so that the loop is about chest-high.}
\item{Procedure: Hold the axle of the bicycle wheel and have a student spin the wheel very fast. Then place one end of the axle in the loop of rope. The wheel will remain vertical, but will turn in a circle around the rope. In addition, you can have students hold the spinning bicycle wheel. If they move it, they should be able to feel that if the axis stays in the same direction, it is easy to move it, but it is difficult to change the direction of the axis of rotation. You can also balance the axle of the wheel on the end of one finger as the wheel spins around your arm.}
\item{Theory: The bicycle wheel is a gyroscope. By adding the weights to the rim and not the center, you increase the moment of inertia greatly, which makes the angular velocity nice and slow. It also increases the angular momentum of the spinning wheel so that it takes more torque to change the direction of the axis of rotation when the students try to move it.}
\end{itemize}

\subsection{Door Tug-o-War}
\begin{itemize}
\item{Preparation time: none}
\item{Materials: a strong door}
\item{Procedure: Get 2 students. One is going to push against the door near the hinge; one will push the other way near the other side (handle) of the door. The one pushing near the edge of the door will find it easy to push the door her way.}
\item{Theory: The student that pushes farther from the axis of rotation can exert less force, but still produce a greater torque, or moment of force, than the one pushing close to the hinge, because.}
\end{itemize}

\subsection{Spinning Eggs}
\begin{itemize}
\item{Preparation time: none}
\item{Materials: 2 eggs (1 boiled, 1 fresh)}
\item{Procedure: Spin the eggs on their sides. If you briefly stop the fresh egg, then release it, it will begin to spin again by itself. However, if you briefly stop the boiled egg, it will remain stopped.}
\item{Theory: The boiled egg is a rigid body, so when you stop its shell, you stop the whole egg. The fresh egg is not a rigid body. It has liquid inside, so when you stop the shell, the inside continues spinning. When you release it, the inside has enough angular momentum to start the whole thing spinning again.}
\end{itemize}

\subsection{Race of Rigid Bodies}
\begin{itemize}
\item{Preparation time: 5 minutes}
\item{Materials: Wide ramp, several different rolling objects (empty can or bottle, small ball, coin, etc)}
\item{Procedure: Roll the several objects down the ramp. Observe that objects with a larger ratio of radius of gyration to actual radius will roll more slowly. Point out which objects roll slower and which faster. Use this to demonstrate radii of gyration.}
\item{Theory: Rotational momentum depends on the distribution of mass along the radius of an object. A solid object will therefore roll differently than a hollow object.}
\end{itemize}

\section{Simple Harmonic Motion}

\subsection{Simple Oscillator}
\begin{itemize}
\item{Preparation time: 1 minute}
\item{Materials: Spring, thread or piece of rubber strip, several weights}
\item{Procedure: Attach the spring or rubber strip to your weight. The weight could be a laboratory weight, a set of keys, or a small padlock. Start the weight oscillating, while explaining to the students how simple harmonic motion works. Add more weight (more keys, another padlock) and observe that there is no change to the period. Now increase or decrease the length of the pendulum and observe any changes to the period. You can tabulate the results for different masses and lengths (keeping one thing constant each time) so that students can see experimentally the dependence of period on length or mass.}
\item{Theory: The period of a pendulum depends on the length of the pendulum (neglecting air resistance), so no change should be noticed with more mass.}
\end{itemize}

\subsection{Bottle Sine Graph}
\begin{itemize}
\item{Preparation time: 5 minutes}
\item{Materials: Empty water bottle, string (approximately 0.5m), water}
\item{Construction: Remove the cap of a water bottle and make a small hole in the center of the bottom with a syringe needle or a nail. Tie a string around the top of the water bottle.}
\item{Procedure: Fill the water bottle. Swing it as a pendulum from left to right while walking forwards. Water will pour out from the bottom of the water bottle in a thin stream, leaving a wet mark on the floor, which creates a graph of its position.}
\item{Theory: As you walk forward, you cause the forward direction to be the time axis. The bottle swings left to right, leaving a watery record of where it has been. Because a swinging pendulum executes simple harmonic motion, this demonstration allows us to see that simple harmonic motion has the shape of a sine curve.}
\end{itemize}

\subsection{Barton’s Pendulums}
\begin{itemize}
\item{Preparation time: 5 minutes}
\item{Materials: Several pieces of string, one large weight (approximately 0.5kg), several small weights}
\item{Construction: Suspend a piece of string horizontally between two fixed objects. Hang the various weights from different points along the string. Each of the small weights should hang from a string of different length. The large weight should hang from a string of similar length to one of the small weights.}
\item{Procedure: Start the large weight swinging. Tell the students to take note of how this affects the behavior of the smaller weights. You should find that the small weight hanging from a string of the same length as the large one exhibits the largest oscillation.}
\item{Theory: The large weight acts as a driving force. Each small weight can swing as a simple harmonic oscillator. We know that a driving force will have the largest effect on a simple harmonic oscillator if the driving force is operating at the natural frequency of the oscillator. When the lengths of the two pendulums are the same, their frequencies are the same. You should be able to get “harmonics” going if you measure the lengths accurately (see string instruments).}
\end{itemize}

\section{Sound}

\subsection{Sound in a Medium}
\begin{itemize}
\item{Preparation Time: half hour}
\item{Materials: Large jar with lid, glue, bicycle pump needle, string, cell phone, vacuum pump (see Reverse Pump)}
\item{Procedure: Poke a small hole in the jar lid and insert the pump needle with at least 1 cm above the lid. Secure the needle with glue, rubber, whatever you need to ensure that it is airtight. Program your cell phone to play something repeatedly at full volume. Hang the phone by the string in the jar so that it is not touching the sides; close the lid on the jar (if the glue is dry) and listen for the phone. You should still be able to hear the phone. Attach the vacuum pump from Reverse Air Pump to the needle on top of the jar and start pumping out the air. You should hear the sound of the phone decrease until it is not heard at all.}
\item{Theory: Sound requires a medium to travel. The denser the medium, the faster sound will travel. Without a medium, there is nothing to vibrate and therefore no sound. By removing the air in the jar, you are removing any material medium and the sound will not be able to travel beyond the cell phone speaker itself.}
\end{itemize}

\subsection{Spoon bell}
\begin{itemize}
\item{Preparation Time: 1 minute}
\item{Materials: spoon, string 1 m}
\item{Procedure: Tie the spoon into the middle of the length of string so that it will hang freely when you hold the string ends. Have a student hold the string ends to his or her temples or the bone just under his or her ear as you strike the spoon with a pen or other object. The student should hear a clear, loud sound.}
\item{Theory: The vibrations of the spoon propagate up the string and into the student’s head. Bone, especially around the temples and outer ear, resonates readily in response to sound.}
\end{itemize}

\subsection{One-String Guitar}
\begin{itemize}
\item{Preparation Time: 5 minutes}
\item{Materials: String or thin steel wire, two clothes clips, any mass, tape}
\item{Procedure: Secure the two clips to the table so that one is close to the edge. Stretch the string or wire across the two clips, securing it at one end by tying or clamping, and hanging the other end over the second clip and over the edge of the table. Attach some mass to this free end so that it pulls the string taught and produces a clear pitch when the string is plucked. Play around by changing the length or tension of the string and hear the different pitches. See if you can play a song!}
\item{Variation: Tape some paper just under the string and mark the ‘frets’ as you find them. The 1st harmonic should be at half the length, and so on.}
\item{Theory: The frequency of a standing wave on a string depends on the tension of the string, its length, and it’s mass per unit length. As you are not changing the string material, you do not need to worry about the last one. Increasing mass (tension) on the string will raise the pitch, as will shortening the length. Most students have seen guitars played and will notice what is going on.}
\end{itemize}

\subsection{Bottle Amplifier}
\begin{itemize}
\item{Preparation Time: 10 minutes}
\item{Materials: Plastic water bottle, string or thread, match or small stick}
\item{Procedure: Poke a small hole in the bottom of the bottle and string one end of the thread through the hole. Tie the end on the inside to the match or small stick so that it cannot be pulled back through the hole. Pull the string taught and have a student hold the top of the bottle. Pluck the string and hear the nice loud sound! Play around with plucking just the string vs. the string and bottle together. Try it with the cap on or off.}
\item{Theory: The vibration of the string causes the bottle itself to vibrate. Rather than hearing just the sound of the string vibrating, we hear the sound of the bottle, which produces noticeably greater amplitude.}
\end{itemize}

\subsection{Transverse Waves on a String}
\begin{itemize}
\item{Preparation time: depends, but in any case a long time}
\item{Materials: Show a design to a fabricator/welder and let them decide this. You can supply thin string, a small pulley, and a weight.}
\item{Construction: Using whatever driving device available. (I used a bicycle, like the men who pedal a bike wheel to drive a grinder), drive a piston with a very small amplitude (1 mm is fine). Whomever you find to do this will have their own way of doing this, but the easiest thing to do is just an offset axle, where the axle being driven jogs to one side a small amount. When you have a piston which can be driven at a very small amplitude by a bike wheel, car motor, etc., attach a string to the top of the piston and hang the other end of the string over a pulley about two meters (varies) away, suspended by a weight. Now you have a string that is driven at whatever frequency you choose.}
\item{Procedure: Pedal the bicycle or turn on the motor and increase the speed (frequency) until you see the fundamental on the string, a standing wave with one antinode and two nodes – the ends of the string. Chat about that for a minute, then increase the frequency until you get the first harmonic, then the second harmonic, etc., until you run out of juice in one way or another.}
\item{Variation: Drive the string with a speaker connected to a single-tone generator. This could be a simple circuit, in fact, allowing you to combine two of the biggest physics topics ever! Use a rheostat to vary the frequency of the circuit, ergo the speaker.}
\item{Theory: Every string has a natural frequency at which it will vibrate with ease, meaning with the greatest possible amplitude. This is called the fundamental (and is directly related to the fundamental as known in music theory, since all harmonics which follow are the octave, 5th, 4th, 3rd, etc.) and is the simplest standing wave. Doubling the frequency will give you the 1st harmonic (octave), which is the next simplest standing wave. All harmonics which follow are closer in frequency and become gradually more complex, but might be difficult to do on this machine unless you have a super-high gear ratio on the bike wheel or a speedy car motor.}
\end{itemize}

\subsection{Musical rubber strip}
\begin{itemize}
\item{Preparation time: none}
\item{Materials: a length of rubber strip}
\item{Procedure: Stretch the rubber strip taught and pluck it. It should produce a musical note. Demonstrate that increasing the tension but keeping the length the same gives a higher note. Demonstrate that keeping the tension the same but increasing the length gives a lower note. Allude to tuning a guitar, which many students will have seen in church.}
\end{itemize}

\subsection{Musical Soda Bottle}
\begin{itemize}
\item{Preparation time: none}
\item{Materials: 2 soda bottles of the same type and size, water}
\item{Procedure: By blowing over the top of a soda bottle it is possible to create a musical note. Add water and blow again several times to demonstrate that the higher the water level in the bottle, the higher the pitch of the note produced. Empty one bottle entirely, and in the other add enough water to achieve a depth of approximately one millimeter. Ask for a volunteer to help at this point. Blow over one bottle to produce a note. Ask the volunteer to blow over the other bottle to produce a note. Point out that the two notes sound almost identical. Now blow over your bottle at the same time as the volunteer. A beat frequency should be heard.}
\item{Theory: Because the soda bottle is open at the top and closed at the bottom, it acts as a half-open pipe, and produces notes with a wavelength of four times the height of the column of air in the bottle. Thus, by adding water, we shorten the height of the column of air, shortening the wavelength and increasing the frequency. When two soda bottles with slightly different heights of water are blown, they produce slightly different frequency notes, and so a beat frequency can be heard.}
\end{itemize}

\subsection{Doppler Whirl}
\begin{itemize}
\item{Preparation time: 5 minutes}
\item{Materials: string of length 1 or 2 meters, mobile phone, sock}
\item{Procedure: You will need a mobile phone that can be programmed with user generated ring tones. Program a ring tone that consists of one note repeated for a period of at least 20 seconds. Demonstrate to the class that the ring tone consists of just the one note. Now place the phone in the sock, tie it to the string, and swing the string rapidly around your head so that the phone moves in a large circle around you. As the phone moves towards the students, they will hear the pitch increased, and as it moves away, they will hear the pitch decreased, because of the Doppler Effect. Note that for the person swinging the phone, their phone neither approaches nor moves away from their ears, but circles around them. For them, there will be little or no discernable Doppler Effect.}
\item{Theory: Sound waves are pressure waves, so they depend on the medium through which they travel as well as the motion of the source. If the source of sound is moving, the sound waves in front of the source become compressed (much like they are being pushed), which translates as higher frequency or shorter wavelength. The sound waves behind the source are extended (much like they are being stretched behind), so the frequency is lower or wavelength longer. A higher or lower frequency is heard as higher or lower pitch.}
\end{itemize}

\section{Simple circuits}

\subsection{Circuit Board}
\begin{itemize}
\item{Preparation Time: 1 hour}
\item{Materials: Piece of flat wood, staples or small nails, hammer, broken radio case, glue, any circuit components}
\item{Procedure: Draw out a grid on the wood with squares about 5-6 cm on a side. At each grid intersection, gently tap in a staple or small nail. From the radio, take the battery casing with its terminals and clips and glue it to one side of the wood; this will be your power supply. Using any configuration you like, set up a circuit on the board using the pins as wire holders. This makes the circuit easier to handle and see.}
\end{itemize}

\subsection{Breadboards}
\begin{itemize}
\item{Preparation Time: 3-4 hours}
\item{Materials: shower sandal (new), knife, glue, sewing needle, metal strips or aluminum foil (0.5 - 1 cm wide, 5 - 10 cm long), sharpie or marker, simple circuit components}
\item{Construction: Remove the sole of the sandal and cut the bottom layer (about 0.5 cm; keep this piece whole for later) off the sandal. Inlay the metal strips at angles into the other section of the sandal in the arrangement shown below. These are the wires of your breadboard. The two long, thin sections are your power strips; the larger section is the board itself where circuit components will be placed. The angle of the metal strips allows the components to remain in contact with the strips under the constriction of the rubber sandal, but if another configuration works better, do that.\\
The diagram shows the layout of a typical breadboard; change this as necessary. Replace the cut-off section of sandal; you will need to cut slots for the metal strips to fit snugly. Using the sewing needle, punch thin holes into the bottom of the sandal along the lines of the metal strips, about 1 cm apart. Use a sharpie or marker to indicate the positions of the holes and the outlines of the different sections on the breadboard, as shown above. On the power strips, label one line as positive and the other as negative. Use glue to keep the pieces of sandal together. Now your breadboard is done; some modification may be necessary depending on the resources available.\\
As for electrical components, broken radios, cell phone chargers, old computers, etc., can be stripped for parts. Resistors, transistors, capacitors (parallel plate or cylindrical), diodes, variable resistors (rheostats), switches, fans, LEDs, heat syncs, speakers, and wires can be found easily, even in villages without electricity. If you are stuck, drop a few shillings at the Broken Stuff shop in town.}
\item{Procedure: Using your new broken radios, pull out the various components and place them in the breadboard as the circuit you desire. Connect the negative and positive ends of this circuit into the power strips, and the appropriate terminals of the power strips to some batteries, or an accumulator. If you smell burning resistors, that is another lesson. If not, then you have a circuit to play with.}
\item{Theory: Students spend plenty of time staring at circuit drawings on the board and sometimes become fairly adept at analyzing them, but when shown a real circuit they cannot tell parallel from series. When teaching simple circuits, accompany any real circuit with a drawing for students to follow.}
\end{itemize}

\section{Simple Machines}

\subsection{Bottle Cap Gearworks}
\begin{itemize}
\item{Preparation Time: 30 minutes}
\item{Materials: handful of bottle caps, pliers, nails, small piece of flat wood, hammer}
\item{Procedure: Find the exact center of each bottle cap and poke a hole through it for the nail. Bend the edges of the bottle caps in so that the ridges along the sides will act as gear teeth when the cap rotates. Nail the caps into the wood at even intervals so that they can freely rotate and in turn cause others to rotate. Make different configurations and note the direction of rotation from one gear to another.}
\end{itemize}

\subsection{Pulleys and Inclined Planes}
\begin{itemize}
\item{Preparation Time: 15 minutes}
\item{Materials: thread spool or water bottle, cardboard, thread: meter rule, spring or spring balance, various masses, stiff wire}
\item{Procedure – Pulleys: A thread spool works well as a pulley, but you can also cut out the ridged section of a water bottle and insert a circle of cardboard into the center as a support. Bend the stiff wire through the hole of the pulley so it can rotate easily without sliding off. Tie the masses to the thread and drape it over the pulley. You can make any fixed or moveable pulley with these resources.}
\item{Procedure – Inclined Planes: Prop up a meter rule at an angle. Hang a mass from a spring or spring balance and drag it up the slope. Measure the extension of the spring (or weight) for the mass when it hangs freely, and again as it moves up the slope. If friction is low, there will be a noticeable decrease in spring extension from the free-hanging mass to the mass on the slope.}
\end{itemize}

\section{Static Electricity}

\subsection{Salt and Pepper Trick}
\begin{itemize}
\item{Preparation Time: 1 minute}
\item{Materials: salt, pepper flakes, pen, dish}
\item{Procedure: Mix a spoonful of salt with a spoonful of pepper and place it on a piece of paper or dish. Charge the pen by rubbing it on your hair or a piece of cloth and hold it over the salt and pepper. Which flakes jump to the pen?}
\item{Theory: Both salt and pepper will be attracted to the pen, but the salt is too heavy to move so only the pepper will make the jump.}
\end{itemize}

\subsection{Electrostatics}
\begin{itemize}
\item{Preparation Time: 5 minutes}
\item{Materials: Plastic ruler and piece of nylon cloth, a glass object and silk cloth, or a latex balloon and piece of fur (or hair), small pieces of metal foil, thread}
\item{Procedure: Rub the plastic ruler against the piece of nylon cloth. This transfers electrons between the two items, producing an electrostatic charge. If the piece of nylon cloth is small, try suspending it from a thread near the ruler. As the two items have opposite charges, they attract each other, causing the nylon to lean towards the ruler.\\
Crumple a piece of foil into a small ball, and suspend it from a thread. Bring the charged ruler near to the foil ball. The charge on the ruler should cause an induced dipole in the foil, which is in turn attracted to the charge on the ruler, causing the foil to lean towards the ruler.\\
If you rub the ruler on two different small pieces of cloth, try suspending the two cloths near each other. As they have the same charge, they will repel and lean away from each other.\\
N.B.: The above can be performed by rubbing a plastic ruler on nylon cloth, or by rubbing glass on silk, or by rubbing latex on fur. Some clothing is made out of nylon. Silk is commonly found in the liner to suit jackets. Other combinations of items can also produce static electric charges. It is best to try these on your own before showing them in front of class.\\
This demonstration is best performed in a room with no wind or air currents, which will make it difficult to see the objects leaning towards each other. The static charges will last for a longer time if there is low humidity and a low amount of dust. On humid or dusty days, the static charges will discharge faster. This is a good alternative to the Gold Leaf Electroscope, which is rather expensive and unnecessary.}
\end{itemize}

\section{Thermodynamics}

\subsection{Copper Coil Candle Snuffer}
\begin{itemize}
\item{Preparation Time: 5 minutes}
\item{Materials: thick copper wire about 40 cm, candle, match}
\item{Procedure: Bend the copper wire into a spiral coil, leaving a length enough for a handle. It should be in the shape of a candle snuffer but clearly open. Light a candle and then put out the flame with your new snuffer.}
\item{Theory: Metal, especially copper, conducts heat readily. By putting the copper coil over a flame, you allow the copper to conduct all of the heat away from the flame, careful not to hurt your hand, depriving the flame of its own heat source.}
\end{itemize}

\subsection{Conduction of heat by different materials}
\begin{itemize}
\item{Preparation Time: 2 minutes}
\item{Materials: wooden stick, metal rod, candle, match}
\item{Procedure: Light the candle and set the stick and rod to rest with one end in the flame (the stick should not light on fire if you just grabbed it from outside, but you can dampen it to be sure). Teach your lesson for a couple minutes and then check to see if it is ready. Have students touch the end (the end NOT in the flame) of each and determine which is hotter. The metal should be significantly warmer than the stick.}
\item{Variation: Drip candle wax at even intervals along both the wooden stick and the metal rod. As heat is conducted along each, the wax will melt and drop off. You should see a significant difference between rate of melting of the stick and metal. You can also stick beans into the wax before it dries, to get a more dramatic effect when the wax melts.}
\item{Theory: Heat is conducted through metal much more efficiently than through wood; therefore the end of the metal rod will become hotter faster than the stick. You should be able to feel it easily, and if the candle is hot enough the metal will be almost too hot to hold.}
\end{itemize}

\subsection{Hot Water Hold}
\begin{itemize}
\item{Preparation time: 10 minutes}
\item{Materials: 3 beakers or drinking glasses, a thermos of very hot water 3 metal coins, a piece of cloth, a piece of rubber strip}
\item{Procedure: Pour hot water into the three beakers. Ask for three students to help with the demonstration. One student will hold his glass using rubber strip to protect against the heat. One will use fabric. One will use metal coins. Tell the students they can put down their beaker if it becomes too hot.}
\item{Theory: Metal is a good conductor of heat, and so we expect that the student using metal coins will only be able to last a short time. rubber strip is a poor conductor of heat (a good insulator), and so will protect its student’s hands for a longer time. Thus, we expect the student using rubber strip to last a long time.}
\end{itemize}

\subsection{Thermal Windmills}
\begin{itemize}
\item{Preparation Time: 1 minute}
\item{Materials: Windmills from another demo “Windmills,” candle, match}
\item{Procedure: If you have already created the windmill, set it up so that it spins horizontally about 10 cm above a candle. Light the candle and watch the windmill rotate.}
\item{Theory: As air is heated, it rises, displacing the cooler air above it. The heated air from the candle will force the windmill to rotate horizontally just as wind would cause it to rotate vertically.}
\end{itemize}

\subsection{Sawdust Water Currents}
\begin{itemize}
\item{Preparation Time: 1 minute}
\item{Materials: sawdust, water, beaker, heat source}
\item{Procedure: Fill the beaker with water and pour in a handful of sawdust so that the sawdust spreads out through the water. Heat the water over a candle or jiko and watch the sawdust cycle through the water from top to bottom.}
\item{Theory: As water is heated, it moves up, displacing the cooler water above it. In this way, heated water continually cycles. As water heats, it moves up and cools, whereby it is later displaced by newly heated water moving up. The sawdust follows the currents in the water so you can clearly see the cycle.}
\end{itemize}

\subsection{Specific Heats}
\begin{itemize}
\item{Preparation Time: 5 minutes}
\item{Materials: thermometer, water, any liquid, measuring cylinder, small pot, glass container or jar, heat source}
\item{Procedure: Measure a known volume of the liquid into a glass container. Heat the water in the pot over a jiko or stove until it is significantly warmer than the other liquid. Measure the volume of the water in the measuring cylinder. Before mixing the liquids, measure each temperature and record it. Now pour the hot water into the other liquid and wait for the temperature of the mixture to equalize. When the temperature levels off, measure and record it. With this data, you can calculate the specific heat capacity of the liquid.}
\item{Theory: Specific heat capacity is given by where H is the heat needed to raise a mass m a temperature T. Since the liquid and water are being mixed, the same amount of energy used to raise the liquid’s temperature is lost by the water to cool it down. We can set the heat of water Hw equal to the heat of the liquid Hl. The masses of the substances are known (by using the mole equations), and you measured the changes in temperature with the thermometer. The specific heat capacity of water is 4200 J/kgK, so you can solve for the specific heat capacity of the liquid.}
\end{itemize}

\subsection{Expansion of Liquids: Moving Colors}
\begin{itemize}
\item{Preparation Time: 15 minutes}
\item{Materials: 0.5 liter water bottle, water, food coloring, metal pot, heat source, straw, knife, glue}
\item{Procedure: Cut a small hole in the bottle cap for the straw to fit through. Insert the straw most of the way and glue it so that it is airtight. Fill the bottle most of the way with water and add some food coloring, then close the cap. Place the bottle into a metal pot with more water and heat the metal pot. As the temperature increases, the level of colored water in the straw will increase. When the level is high enough and the students can see clearly what is happening, take the bottle out of the metal pot and dip it in cold water. Watch the level in the straw drop quickly!}
\item{Theory: As with solids and gases, liquids expand when heated. In a sealed bottle with a straw, the liquid must expand as the temperature increases and can only move up the straw.}
\end{itemize}

\subsection{Expansion of Gases: Oil Elevator}
\begin{itemize}
\item{Preparation Time: 15 minutes}
\item{Materials: bottle with cap or flask with stopper, cooking oil, straw, glue}
\item{Procedure: Create the same bottle-straw configuration as in the Expansion of Liquids demo, or use a flask with a stopper with a glass tube if you have it. Close the cap tightly and carefully pour a drop of oil into the top of the straw. The oil should stop in the straw before reaching the bottom. Heat the bottle a little and watch the oil drop move back up the straw. If you have a glass flask, you should be able to heat it with just your hands.}
\item{Theory: Gases respond much more to heat than solids or liquids, and will expand noticeably with even a small amount of heat. By slightly heating the bottle, you raise the temperature, and therefore the pressure, of the air inside. As the pressure increases, the air pushes the oil up the straw.}
\end{itemize}

\subsection{Expansion of Solids: Screw and Loop}
\begin{itemize}
\item{Preparation Time: 5 minutes}
\item{Materials: screw, length of thick-ish metal wire about 10 cm, heat source, tongs}
\item{Procedure: Bend the wire into a loop such that the head of the screw will just fit through. Demonstrate that the screw fits, then heat the screw in a candle or jiko for a minute or so. Try to pass the screw through the loop again; demonstrate that it no longer fits! Try alternately heating the loop and screw, then let them cool and see if the screw fits again.}
\item{Theory: Metals expand when heated. When you heat the screw, the diameter of the head increases slightly. It is not noticeable to the naked eye, but it becomes obvious when the screw becomes too large to fit into the loop.}
\end{itemize}

\section{Uniform Circular Motion}

\subsection{Conical Pendulum}
\begin{itemize}
\item{Preparation time: 1 minute}
\item{Materials: string (about 0.5 m), small weight (e.g. pendulum bob, rock)}
\item{Procedure: Tie the weight to one end of the string. Hold the other end of the string and gently swing the weight in a horizontal circle.}
\item{Theory: This can help students to see what is meant by the angle beta, to show that the length of the string is different from the radius of the circle, and to help them think about the forces involved—that there is only tension and gravity, but that the tension has a vertical component equal and opposite to gravity, and a horizontal component that provides the centripetal acceleration.}
\end{itemize}

\subsection{Bicycle Turn}
\begin{itemize}
\item{Preparation time: none}
\item{Materials: bicycle}
\item{Procedure: Ride the bicycle in a circle by leaning in on direction rather than turning the handlebars.}
\item{Theory: It is possible to turn a bicycle just by leaning in one direction. When leaning, the normal force has two components: the vertical component is equal and opposite of the gravitational force and the horizontal component provides a centripetal acceleration, causing the change in direction of velocity: you go in a circle.}
\end{itemize}

\section{Waves}

\subsection{Ripple tank}
\begin{itemize}
\item{Preparation time: 1-2 hours}
\item{Materials: pane of glass (40 cm x 60 cm is average), wood frame, caulk or some sealant, straight-edge, pen, lamp or torch, white paper.}
\item{Construction: Make the ripple tank itself using an old window in its frame, or have a craftsman make a sealed window to your specifications (bigger is better, but use what is available). Seal the glass and frame with caulk or some equivalent so that water will not leak through it. Prop this shallow tank up on stands or books about a foot off the table; place the lamp above the tank facing straight down and the paper about 30 cm directly below the tank.}
\item{Procedure: Level the tank and fill it with water 5 mm deep. Turn on the lamp so that any variation in the water’s surface shows up as a shadow on the white paper underneath. Create plane waves using a ruler or circular waves using a pen. You can also attach a thin strip of metal so that half hangs over the tank and half outside. Attach a pen to the inside half and flick the metal strip. If the masses on either side of the tank frame are equal, the strip should oscillate fairly easily. Observe diffraction, reflection, etc. by placing objects in the tank. All phenomena will be visible on the paper and can be measured there.}
\item{Theory: Water waves, while different from sound or light waves, show the same properties of propagation, reflection, diffraction, etc. as any other wave. The varying depth of the water due to oscillating crests and troughs creates areas of bright and dark spots from the lamp above. Any behavior of the waves can be seen clearly and even measured: wavelength, frequency, and wave velocity.}
\end{itemize}

\subsection{String Waves}
\begin{itemize}
\item{Preparation time: none}
\item{Materials: at least 2 meters of string}
\item{Procedure: We can show several different types of waves using just a string. First, lay the string out along a table. By snapping your hand, you can create a transverse pulse wave that will travel along the length of the string. Now tie one end of the string to a rigid object and repeat. You should see the pulse reflect from the fixed end and come back towards you. Now wave the string up and down so as to create a standing wave. By changing the frequency with which you move your hand, you should be able to create at least two different modes.}
\end{itemize}

\section{Work}

\subsection{Work as Heat, Part A}
\begin{itemize}
\item{Preparation time: 5 minutes}
\item{Materials: thin strip of metal, pliers}
\item{Procedure: Take a piece of metal. Use a set of pliers to bend the metal back and forth. Feel the temperature of the metal.}
\item{Theory: Work can manifest itself in a variety of ways. One of the most common ways is the rise in temperature. By moving the metal back and forth, you are doing work on the metal. This work is converted into heat. This heat is evidenced by the rise in temperature in the metal. }
\end{itemize}

\subsection{Work as Heat, Part B}
\begin{itemize}
\item{Preparation time: 0 minutes}
\item{Materials: radio antennas, old or new }
\item{Procedure: The radio antennas operate in a telescopic motion. Pull the radio antenna in and out for one full minute. Do not break the antenna in this movement. Observe the temperature of the antenna after the work is over.}
\item{Theory: Again, you are doing work on the radio antenna by moving it in an out quickly. Through this action, the antenna heats up. This is the evidence of the work you have been doing. Work is defined as force times distance or. In this case, the force is the effort required to move the antenna in and out while the distance is the length of the antenna.}
\end{itemize}

\subsection{Work as Light}
\begin{itemize}
\item{Preparation time: 0 minutes}
\item{Materials: duct tape, or other tape that holds together tightly.}
\item{Procedure: Cut two pieces of duct tape. Press the ends of the bottom pieces of tape together but allow the top pieces of tape to be apart. Hold tightly to both pieces of tape at the top, and quickly rip them apart. Observe the blue light when the tape comes apart.}
\item{Theory: Pulling the tape apart quickly creates a faint blue light. It is best to observe this light at night since it is so faint. In this activity, this is the work being done to pull the tape apart. Unlike the previous activities, this work is released as light. This phenomenon as where work manifests itself as light is called triboluminescence. This is the same phenomenon that causes the green light when snapping wintergreen mints.}
\end{itemize}
