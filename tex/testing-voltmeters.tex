\chapter{Checking Voltmeters and Ammeters/Galvanometers}

Needed: Meters to check, 
a couple wires, 
some resistors and a fresh battery.

Important note: There is a wrong way to hook up the meter. 
The needle will try to deflect down 
because negative and positive are swapped. 
If the reading is zero, 
make sure that you try the opposite connection to be sure.

\section{Voltmeters}
Hook up the voltmeter across the battery. 
The battery is probably 1.5 V, 
but do not worry if you see 1.1, 
1.2, 
even if using a brand new battery. 
Try not to use a battery that reads much below 1 V 
on several different meters.

\subsection{Unuseable Voltmeters}
\begin{itemize}
\item{Totally dead, no deflection of the needle}
\item{Voltage reading jumps excessively. 
Ensure that the connections are solid and test again.}
\item{Measured voltage is totally wrong, not close to 1.5 V}
\end{itemize}

\subsection{Useable Voltmeters}:
Read a voltage close to 1.5. 
If the voltage if not 1.5 exactly, 
the voltmeter is probably working fine, 
and the battery is just off a bit.

\section{Ammeters}
Hook up the ammeter in series with a resistor. 
Because you do not necessarily know the condition of the ammeter before testing, 
be sure to have several different resistors on hand. 
An ammeter may appear not to work if resistance is too high or too low. 
Start testing different ammeters.

\subsection{Unuseable Ammeters} 	
\begin{itemize}
\item{Totally dead, 
no deflection of the needle}
\item{Current reading jumps excessively (but check connections)}
\item{Totally wrong, 
reads much different from other ammeters}
\end{itemize}

\subsection{Useable Ammeters}
Read a current similar to other ammeters. 
Hard to say exactly what current, 
but feel free to calculate based on your resistor using V=IR, 
although do not forget that there is 
some internal resistance r of battery, 
so $V=I(R+r)$. 
The resistance of the resistor is usually coded 
on the resistor in a series in stripes - 
see the instructions under Resistors in the Sources of Equipment section.

Tip: You can hold the wires onto the battery with your fingers; 
the current is far too low to shock you.

Other: Now that you have tested to see 
if your voltmeters and ammeters work, 
you can feel free to check all of them for accuracy, 
by calculating expected values and comparing between meters. 
Most practicals will still work alright with “somewhat” accurate meters, 
and most meters are either fine, 
or broken.
