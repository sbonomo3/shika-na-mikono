\chapter{Waste Disposal}
\label{cha:wastedisp}
\section{Introduction to waste management}
Practical work produces chemical waste. 
Some of these wastes may be harmful to people, property or the environment 
if not properly treated before disposal. 
Regardless of where the waste will go -- 
a sink, a flower bed, a pit latrine -- 
the following procedures should always be followed.

Note, often there are unused reagents at the end of a practical. 
These are valuable and should be stored for use on another day. 
When storing left over reagents, label the container with:
\begin{enumerate}
\item{The name of the compound, e.g. "sodium hydroxide solution"}
\item{The concentration, e.g. 0.1~M}
\item{The date of preparation, e.g. 15 June 2010}
\item{Important hazard information, 
e.g. "CORROSIVE, neutralize spills with weak acid."}
\end{enumerate}

Sometimes, there are used reagents that may be recycled. 
Recycling of chemicals reduces harm to the environment and saves money. 
Examples of chemical recycling are:
\begin{itemize}
\item{Regenerating silver nitrate solution from qualitative analysis waste.}
\item{Purification for reuse of organic solvents from 
distribution/partition law waste}
\end{itemize}

In order to recycle these compounds, 
students must put their waste in designated containers. 
Specific instructions for chemical recycling follow in another section.

Some wastes may be discarded without worry. 
These solutions may be poured down a sink or into a pit latrine. 
These include:
\begin{itemize}
\item{The final mixture in the flask after a titration. 
This is neutral salt water.}
\item{All of the wastes from food tests in biology. 
Note that unused reagents are not waste!}
\end{itemize}

Finally, some wastes require special treatment. 
These wastes and their treatments follow.

\section{Special instructions for certain wastes}
\subsection{Organic wastes}
These are any substance that does not mix with water, 
for example kerosene, isobutanol, ether, chloroform, etc. 
These substances should be placed in an open container 
and left to evaporate down-wind from people and animals. 
Setting these wastes on fire is usually unnecessary and may be dangerous.

\subsection{Strong acids}
Sulfuric, hydrochloric, and nitric acid solutions 
will corrode sinks and pipes if not neutralized before disposal. 
These wastes should be collected in a special bucket during a practical. 
After the practical, bicarbonate of soda should be added 
until further addition no longer causes effervescence. 
The gas produced is carbon dioxide.

\subsection{Strong bases}
Sodium and potassium hydroxide solutions 
as well as concentrated ammonia solutions are also corrosive. 
These wastes should be collected 
in a different special bucket during a practical. 
After the practical, the waste should be colored with POP 
or a local indicator and acid waste should be added until the color changes. 
If there is more base waste than acid waste available to neutralize it, 
citric acid may be added until the color finally changes.

\subsection{Heavy metals}
Barium, lead, silver and mercury solution 
are highly damaging to the environment 
and may poison human or animal drinking water if disposed without treatment. 
Waste containing barium and lead, generally from qualitative analysis, 
should be collected in a special container during a practical. 
After the practical, dilute sulfuric acid should be added drop-wise 
until further addition no longer causes precipitation. 
At this point, soluble lead and barium will have been converted 
to insoluble lead sulfate and barium sulfate. 
These salts may then be disposed in a pit latrine. 
The waste should of course first be neutralized with bicarbonate of soda.

Waste containing silver should be collected in a different special container. 
Ideally, this waste will be treated to regenerate silver nitrate solution 
according to the instructions in the next section. 
If such recycling is infeasible, 
sodium chloride solution should be added drop-wise 
until further addition no longer causes precipitation. 
At this point, soluble silver will have been converted 
to insoluble silver chloride and may be disposed in a pit latrine.

There is no treatment for mercury solutions 
that may be safely performed in a secondary school. 
This fact combined with the extreme danger 
of using mercury compound in schools 
supports the recommendation that mercury compounds never be used. 
If mercury waste is ever discovered at the school, 
it should be placed in a well-sealed bottle labelled: 
MERCURY WASTE. 
TOXIC. 
DO NOT USE. 
MUST NOT ENTER THE ENVIRONMENT. 
SUMU KALI. 
USITUMIE NA USIMWAGE.

\subsection{Strong oxidizers}
Concentrated solutions of potassium permanganate, 
chromate, dichromate, hypochlorites (bleach), and chlorates 
should be reduced prior to disposal. 
Grind ascorbic acid (vitamin C) tablets to powder 
and add until the permanganate decolorizes, 
chromate and dichromate turn green or blue, and hypochlorites lose their smell. 
The resulting solutions may be safely disposed in a sink or pit latrine.

\subsection{Solid waste}
Solids clog pipes and should never be put into sinks. 
If the solid is soluble, dissolve it in excess waste 
and treat as solution waste. 
If the solid is insoluble, dispose into a pit latrine.

\subsection{Unknown compounds}
If you do not know what a compound is, 
you do not know what kind of treatment it requires prior to disposal. 
That solution that looks like water could be nitric acid, 
or mercury chloride solution. 
Before disposing of unknown compounds, 
please use the Guide to Identifying Unknown Chemicals in the appendix. 
Even if you cannot identify the compound with these instructions, 
you can use them to ensure that it is not dangerous to dispose.
