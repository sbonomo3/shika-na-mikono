% SHIKA NA MIKONO

% This code is published with the GNU General Public License v2.

% The remaining comments explain the code.
% -------------------------------------------

% This file - shika_int.tex - is the backbone file for the manual.
% It's job is to set the global parameters and order the chapters.
% Each chapter lives in a different file (all in the tex folder)
% This keeps everything more organized.

% The code begins...
\ifx\pdfoutput\undefined
	% using latex
	\documentclass[10pt,a4paper]{report}
	\usepackage[dvips]{graphicx}
\else
	% using pdflatex
	\documentclass[pdftex,10pt,a4paper]{report}
	\usepackage[pdftex]{graphicx}
	\DeclareGraphicsExtensions{.pdf,.png,.jpg}
	\usepackage{hyperref}
	\hypersetup{
  		colorlinks,%
  		linkcolor=blue,%
  		urlcolor=cyan
	}
\fi

% We want a variety of style packages.
% We have them in a separate file - shika.sty - to keep this file tidy.
% This command calls all of them at once.
\usepackage{shika}

% Now the document can begin:
\begin{document}

% The input command inserts the content of another file into this master document.
% Here it calls for the title page, title.tex
\begin{titlepage}
% title section
\begin{center}
	\textsc{{\Huge Shika na Mikono}}\\[0.4cm]
	\textbf{{\huge Version 3.0}}\\[1.5cm]
	\HRule\\[0.4cm]
	\textsc{{\Large Hands-On Science Resource Manual}}\\[0.4cm]
	\HRule\\[0.5cm]
\end{center}

\vfill
% date at bottom
\begin{center}
	{\large \today}
\end{center}
\end{titlepage}


% To make page numbers consistent w/ the numbering in the PDF viewer.
% (e.g. page 35 is actually page 35 in the .pdf)
\setcounter{page}{2} 

% Some front matter...
\chapter{About This Book}

Just before the rains returned at the end of 2009, 
Peace Corps Tanzania recruited Leigh Carroll and me 
to conduct a training on laboratory methods 
for the new cohort of education Volunteers. 
We decided to write a hand-out for our presentation, 
summarizing “everything that Volunteers needed to know about the lab.” 
We quickly realized that this was a much larger undertaking 
than we could manage on our own 
and began recruiting other Peace Corps Volunteers. 
Thus the Shika Na Mikono project was born.

As the first edition came together, several themes emerged. 
We believe that hands-on activities are essential for learning science. 
We believe that the use of local and low cost materials 
can enable any school to do these activities. 
Finally, we believe that for hands-on science to be successful, 
it must be safe. 
These three ideas – interactive learning, equity, and safety – 
remain the core of the Shika vision.

The first Shika na Mikono spread far beyond 
its original use in Peace Corps. 
Within six months of its publication, 
copies were used in several teacher trainings in Tanzania, 
hundreds of additional copies were locally printed in various regions, 
and the book was presented to the Tanzanian Minister of Education. 
Shika na Mikono became a project, a set of values, 
and a solid cadre of Peace Corps Volunteers 
joined in furthering these ideas in the field. 
Peace Corps gave me a third year extension to 
coordinate this work and to train both Peace Corps Volunteers 
and Tanzanians in these methods.

Given the enthusiastic response to the first edition 
and the significant work the project had done in the past year, 
we decided to produce a second edition, 
this one for the 2010 Peace Corps pre-service training. 
Once again, many people contributed to this effort, 
most especially PCVs Michael Rush, Kristen Grauer-Gray, and Peter McDonough 
– without their excellent ideas and passionate hard work, 
this book would not exist.

The revised edition proved even more popular, 
and we began receiving requests for information 
from other Peace Corps posts, beyond Tanzania. 
Much of the value of the Tanzanian version of this book 
is its specificity – 
chemicals sourced to an address in the capital, 
Swahili vocabulary for specific items and actions, 
explicit instructions for performing practical exercises 
for the national examinations. 
Nevertheless, we thought that some of the general information 
in the text would be of use abroad, 
and have prepared this version – 
the first “internationalized” edition of Shika na Mikono. 
Our hope is that this book will equip teachers in other countries 
with some core ideas and inspire them to make 
a book of specific utility to their environment. 
PCVs Jessi Bond, Carolyn Rhodebeck, and Dylan Masters 
created additional content for this version. 
PCV Dave Berg provided the inspiration, instruction, 
and much of the labor to import this version into LaTeX.

Many of the ideas for locally available materials 
come from or were inspired by the Source Books 
published by the Mzumbe Book Project, Morogoro, Tanzania. 
Several other ideas for locally available materials 
were developed at Bihawana Secondary by Mwl. Mohamed Mwijuma. 
PCVs Peter Finin and Gregor Passolt wrote 
a book on physics demonstrations in 2008 
that has been incorporated wholesale 
into the Hands-On Activities section of this book. 
My own knowledge of the laboratory was greatly increased 
by a brilliant if ancient book found on the shelves of Bihawana Secondary – 
the cover and title pages with the title have long sense been lost, 
but the preface sites G.P.Rendle, M.D.W. Vokins and P.M.H. Davis as authors, 
and 1967 as the date of publication.

We are all grateful to our schools 
for giving us the opportunity to work in such supportive environments, 
the freedom to explore these ideas, 
and the time to document them. 
We have certainly benefited from the wisdom 
and creativity of many other teachers, 
both in this country and in America. 
Many of us working without reliable electricity 
or internet connections benefited enormously 
from the hospitality of the numerous expatriate families 
who sheltered us in town. 
We are all grateful to Peace Corps Tanzania for supporting our work, 
especially James Ogondiek and the now retired and much beloved Thomas Msuka, 
both of whom recognized early on the value of this project 
and advocated for us to undertake the work required to develop 
and spread the ideas in this book.

Most of all, we are grateful to our students, 
for it is their curiosity and enthusiasm that has motivated everything.\\[24pt]
Aron Walker\\
Bihawana Secondary, Dodoma\\
aronwalk@alum.mit.edu

\chapter{The Shika na Mikono Teaching Philosophy}

Science is the study of the natural world. 
To learn science, students must interact with the world around them. 
They must ask their own questions and seek their own answers. 
They must see things and they must grasp them in their hands; 
hence the name of this book: 
\textit{shika na mikono} is Swahili 
for ``grasping in hands.''

It is our fervent belief that every student in the world 
should perform science practical exercises. 
For too long we have heard complaints that schools lack the materials 
necessary for these exercises. 
This book attempts to make clear that students may perform 
science practicals at any school, 
most especially at those without traditional laboratories, starting today. 
Everything teachers need to create these hands-on learning experiences 
is available locally and/or at low cost.

Many national syllabi require practical exercises, 
often on their national examinations. 
This is good. 
Critically, however, we urge teachers to expand the scope of 
students' hands-on work beyond the practicals for the national exams. 
Every topic, every lesson may be a ``practical,'' 
not just a demonstration on the front bench 
but an opportunity for students to touch 
and manipulate and discover on their own.

In this vision, the science teacher becomes a guide, 
someone who can assemble parts of the natural world 
into a compelling lesson and ask the questions that help students see how things work. 
In this capacity, the science teacher remains a resource 
irreplaceable by the march of technology. 
Photocopy machines produce student editions of notes 
much more efficiently than teachers copying them to the board 
for students to copy again. 
Instructional films shown on low cost solar-powered projectors 
offer students articulate explanations and demonstrations. 
But no technology can replace the essential role of 
the modern science teacher: she is an architect, 
one who builds a space in which students can learn for themselves, 
and a shepherd, who tends to their learning through that discovery.

The aim of this book is to inspire and empower this sort of teaching. 
For years, many educators have bantered about the phrase 
``student-centered teaching.'' 
This sounds rather like patient-centered medicine -- 
anything else is simply absurd. 
The focus of a lesson must always be the experience of the student. 
To prepare such a lesson, the teacher should answer the following questions. 
What will the student do in class? 
How will she use her hands to interact with the world? 
What will the student observe with her own senses? 
Given these experiences, 
what questions will arise from the student's observations? 
Given these questions, 
how might the teacher respond to provoke further inquiry and 
critical thinking? 
How might the student's peers respond to build a common understanding? 
How might the student, through further observation and experimentation, 
arrive at the answer herself? 
Given these goals, 
what experiences will put her on the journey to that answer? 
What series of activities should be offered to her 
to facilitate that discovery? 
This is student-centered teaching -- 
a lesson plan crafted around the experience of the student, 
the internal, cognitive, 
and emotional experience of being in class that day.

The process of answering these questions involves several steps. 
The teacher must first organize the material that 
the student is to understand into a well-structured framework: 
logical, sequential, and hierarchical. 
Then, using this framework, the teacher should design activities 
for the student to discover each aspect of the material. 
These activities should be sequenced to expand understanding, 
moving from simple phenomena to the more complex, 
from the specific to the general. 
Discussion questions should seek first to uncover core phenomena 
and then to link each new insight with 
what the students already understand about their world. 
Targeted questions catalyze introspection, group discussions, 
and the realizations necessary for the students themselves 
to start articulating scientific theories. 
Once the students have discovered phenomena, 
linked them to pre-existing understanding, 
and begun articulating general theory, 
the teacher can help focus and form these articulations 
into the accepted vocabulary and nomenclature of modern science.

In this vision, 
the students learn material from their experience and 
their reflection on that experience. 
They believe in theories because they have demonstrated and 
articulated them for themselves. 
In this vision, 
science becomes the study of reality, 
an ever-growing understanding combined with a powerful set of mental tools 
to bear on all parts of life. 
What students learn in the classroom connects with and 
illuminates aspects of life at home, 
in the village, in town, and on the farm. 
The capacity that students gain to ask critical questions and 
seek their own answers empowers them well beyond 
high scores on formal assessments; 
the scientific mindset allows students to seek truth in all matters, 
and to invent solutions to the many challenges before them -- 
not just for test questions in school, 
but for the ones that really matter in life.

This ultimate achievement, 
that students gain something in the classroom with 
value beyond the limits of the school, 
is further incentive to embrace the style 
of teaching through hands-on activities using local materials 
as we advocate. 
Few students will encounter professional scientific instruments 
later in their lives; 
an understanding founded on exotic apparatus and 
imported high-end chemicals has little applicability to life after school. 
When students explore the world with readily available materials, 
however -- 
when they see parts of their own world appear in the classroom 
for focused experimentation and analysis -- 
they gain an understanding that bridges scientific theory and 
daily reality, 
that sheds light on the world beyond the laboratory, 
and that lets them wield scientific thinking anywhere.

Hands-on science education is possible anywhere. 
The materials we need are available in our villages and in our towns. 
The key ingredients in science education are not precision glassware, 
imported reagents, nor massive loans. 
The key ingredients are curiosity, creativity, 
and the ability of teachers to use what they already have 
to provide students with experiences 
that broaden their understanding of the world.

Finally, to realize this vision, 
we teachers must embrace questions; 
we must encourage students to ask about what they do not understand. 
Rather than answer these questions directly, 
whenever possible we should design experiments 
or ask questions in return that allow students 
to find answers for themselves. 
As role models, we also must embrace the limits of our own understanding. 
Often students ask questions to which we do not know the answer. 
This is a fundamental aspect of science education. 
Our job is to help students to understand the world better, 
to guide them in that discovery. 
Our job is not to know everything; this is neither necessary, 
nor is it possible, nor even desirable. 
When our students observe us confronting the unknown, 
when they see how we ask questions and 
perform experiments ourselves to seek out the truth, 
then they become more comfortable asking questions 
and seeking answers themselves. 
This experience helps them to understand the true power of science, 
that a person anywhere may always find the answer.

Let us gather the world around us and 
put it in the hands of our students, 
so they might understand how it works. 
Let us let them grasp it in their hands -- 
\textit{walishike na mikono yao.}


% Now a table of contents:
\tableofcontents

% ...but we do not want a page number to apprear on the table of contents
\thispagestyle{empty}

% The rest of the book follows, in four parts

% Part 1 - Lab Development
\part{Laboratory Development}
 % This is a page dedicated to annoucing Part 1
\chapter{Starting School Laboratories}

A science laboratory is any place 
where students learn science with their hands. 
It might be a room, 
or just a box. 
The goal is to develop a space that facilitates hands-on learning.

\section{Benefits of a School Laboratory}
There are many benefits of having a laboratory:
\begin{itemize}
\item{Students learn more and better science}
\item{Students get more excited about science class}
\item{Students have to go to the lab for class, 
thus eliminating those too lazy to walk over}
\item{Practical exams are easier than the alternative-to-practical exams}
\item{Everyone thinks practicals are important, 
and that science without practicals is silly.}
\end{itemize}

\section{Challenges of a School Laboratory}
There are some challenges with having a laboratory:
\begin{itemize}
\item{They are places where people can get hurt\\
\textit{This is true. 
Please see the sections on managing a laboratory and laboratory safety 
to mitigate this risk.}}
\item{Many teachers do not know how to use a laboratory\\
\textit{Then use the lab to teach them how to use it, thus spreading skills.}}
\item{Laboratories are far too expensive for poor schools to build and stock\\
\textit{This is simply incorrect. 
Any room will work for a lab, 
and any school can afford the materials required to stock it. 
The rest of this book is dedicated to this point.}}
\end{itemize}

So you want to build a laboratory?

\section{Step one: Location}
A permanent location is obviously preferable. 
If your school has an extra classroom, 
great. 
The only requirements of a potential room are 
that it be well ventilated 
(have windows that either open or lack glass altogether) 
and be secure: bars in the windows, 
a sturdy door, 
and a lock. 
If you plan to put fancy equipment in your lab, 
remember that hack saw blades are cheap 
and that the latch through which many pad locks pass 
can be cut quickly regardless of the lock it holds. 
But if you are just starting, 
there will probably not be any fancy equipment; 
a simple lock is enough to keep overly excited students 
from conducting unsupervised experiments.

If there is no extra space at all, 
the lab can live in a few buckets and be deployed in a classroom 
during class time. 
``There is no lab room,'' is no excuse for not having a lab.

\section{Step two: Funding}
Yes, 
some is required. 
But the amount is surprisingly little -- 
in most countries a single month of a teacher's salary 
is enough to furnish a basic laboratory. 
Almost every school can find the amount required to get started, 
and if not the community certainly can. 
A single cow in most countries would pay 
for a basic laboratory many times over. 
A cow is valuable. 
So is science education.

We encourage you to resist the temptation 
to ask people outside of the school community or school system 
to pay for the lab. 
There is simply no need to encourage that sort of dependence; 
this can be done locally, and it should be.

\chapter{Specific Technical Needs of a School Laboratory}

Laboratories facilitate hands-on investigation of various phenomena. 
Every syllabus requires different topics for study, 
but a core of topics provide a good foundation for each subject.

\section{Basic Biology Laboratory}

A basic biology laboratory should allow the following investigations:
\begin{itemize}
\item{Collection, shelter, and observation of living specimens 
(plant, insect, fish, reptile, mammal)}
\item{Bacterial and fungal cultures}
\item{Preservation and dissection of dead specimens 
(plant, insect, fish, reptile, mammal; both whole and parts thereof)}
\item{Assembly and observation of miniature ecosystems}
\item{Low power microscopy}
\item{Diffusion and osmosis}
\item{Chemical tests of basic biological molecules 
(``biochemical tests'' / ``food tests'')}
\item{Chemical analysis of the products of animal and plant respiration}
\item{Non-invasive investigation of human systems 
(nervous, sensory, circulatory, muscular, parts of the digestive)}
\end{itemize}

Key materials are:
\begin{itemize}
\item{Containers, bottles, tubes, super glue}
\item{Plants, insects, fish, (safe) reptiles, and small mammals}
\item{Sugar, starch, protein source, fertilizer, salt, food coloring}
\item{Chemicals for preservation of specimens}
\item{Scalpels and pins}
\item{Low power microscopes (water drop microscopes, locally assembled)}
\item{Reagents for biochemical tests}
\item{Reagents for gas identification}
\item{Stopwatches}
\item{Heat sources}
\end{itemize}

\section{Basic Chemistry Laboratory}

A basic chemistry laboratory should allow for the following investigations:
\begin{itemize}
\item{Distinguishing compounds from mixtures, 
preparing chemical compounds, separating mixtures}
\item{Changes in the state of matter (melting/freezing, 
evaporation/condensation, sublimation/deposition, 
dissolution/crystallization)}
\item{Comparison of metals and non-metals}
\item{Comparison of covalent and ionic (electrovalent) compounds}
\item{Observing various elements and compounds 
and their reactivity with air, water, acids and bases}
\item{Acid/base, oxidation/reduction, and precipitation reactions}
\item{Energy changes from chemical reactions (thermochemistry, energetics)}
\item{Factors affecting the rates of chemical reactions (chemical kinetics)}
\item{Properties of gases (gas laws)}
\item{Preparation of gases (hydrogen, oxygen, carbon dioxide)}
\item{Electrochemical experiments 
(conductivity, electrolysis, electroplating, voltage generation)}
\item{Volumetric analysis (titration)}
\item{Identification of unknown salts (``qualitative analysis'')}
\item{Very basic organic reactions 
(e.g. preparation of ethanol by fermentation, 
oxidation of ethanol to ethanal)}
\end{itemize}

Key materials are:
\begin{itemize}
\item{Containers, bottles, tubes, balloons}
\item{Tools for measuring volume 
(calibrated plastic water bottles, plastic syringes)}
\item{Low cost balance (digital)}
\item{Heat sources and open non-luminous flames}
\item{Stopwatches}
\item{Power supplies (e.g. batteries) and wires}
\item{Wide variety of chemicals including metallic elements, 
non-metallic elements, solid covalent compounds, salts, 
acids, bases, redox reagents, indicators, 
and many chemicals for specific kinds of reactions}
\end{itemize}

\section{Basic Physics Laboratory}

A basic physics laboratory should allow for the following investigations:
\begin{itemize}
\item{Measuring volume, mass, and density of liquids and solid objects}
\item{Measuring time, velocity, acceleration}
\item{Gravitational acceleration, force, and friction}
\item{Mechanical tools (levers, pulleys, etc.)}
\item{Simple harmonic motion (pendulum, spring)}
\item{Temperature, heat capacity, and heat transfer 
(conduction, convection, radiation)}
\item{Waves (including water and sound)}
\item{Optical experiments (reflection, refraction, diffraction)}
\item{Electromagnetic experiments 
(conductivity, magnetic field lines, 
induction, motors, electrical generation)}
\item{Simple circuits (including resistors, capacitors, and switches)}
\end{itemize}

Key materials are:
\begin{itemize}
\item{A low cost balance (digital)}
\item{Tools for measuring volume}
\item{Containers, misc. objects, bottles, etc.}
\item{Stopwatches}
\item{Heat sources}
\item{Thermometers}
\item{String, springs, wire}
\item{Water, oil, sand, rocks}
\item{Mirrors, lenses, glass blocks, diffracting surfaces}
\item{Magnets}
\item{Power supply (e.g. batteries)}
\item{Inexpensive multimeters or locally made galvanometers}
\item{Electrical components}
\end{itemize}

\chapter{Sources of Laboratory Equipment}

Below are common apparatus you might order from a laboratory supply company, 
and comments about which are really necessary 
and which have good if not superior alternatives 
available in villages and towns. 
Given equal quality, 
it is generally better to use local materials, 
because these help connect classroom learning to students' lives.

\section{Alligator clips}
Are generally available. 
You can also glue aluminum foil to clothespins.

\section{Balance}
These are expensive. 
Look many places, 
especially port cities and capitals, 
to get a better price. 
A digital balance might be less expensive 
and is probably more accurate. 
If you know anyone going to the USA, 
digital balances for jewelers (and drug dealers) are cheap – 
between \$20-30 on eBay. 
Search for 0.01~g precision.

\section{Beakers}
Beakers have many uses, 
so it is good to know which use you are trying to replace.

A jam jar, 
disposable plastic cup, 
or a cut off water bottle works well for holding solutions 
that you will transfer out via pipette or syringe.

Having the “beak” is nice when filling burettes or measuring cylinders, 
but the little plastic funnels that come with kerosene stoves work well too. 
You could also varnish a small metal funnel from the market. 
You can also fill measuring cylinders or burettes crudely 
from a jar or any other bottle 
and then use a syringe to add the final few milliliters.

A big borosilicate (e.g. 
Pyrex brand) beaker is useful for water baths, 
but an aluminum pot is superior if you have many things to heat. 
For warming a test tube or two only, 
consider using the bottom of a small metal can. 
You can cut the bottom from a beverage can 
by repeatedly scoring it with a razor blade, 
or scissors, 
and then use it to hold a water bat. 
HIf you use a cut can, 
fold down the cut edge to prevent cut fingers.

If you do purchase beakers, 
buy plastic ones. 
Plastic lab beakers withstand concentrated acids 
and most other forms of chemical attack 
and they do not break when dropped. 
The only exception to buying plastic is 
if you need beakers for heating on an open flame. 
Then they must be borosilicate (Pyrex) glass, 
but again, 
an aluminum pot will heat the water faster, 
be easier to handle, 
and again does not break if dropped.

\section{Bunsen Burner}
\textit{See Heat Sources}

\section{Burettes}
Ideally, 
your school has enough of these for every student 
to use one if they are required for national exams, 
as well as extra for those that malfunction. 
For example, 
if you plan two sessions of forty students each, 
you want at least 45 burettes. 
First, 
note that broken burettes can often be repaired – 
see the instruction in Repairing Broken Burettes. 
Second, 
if you buy burettes, 
buy plastic ones so they do not need to be repaired. 
They ARE available, 
especially if you order them in advance.

If burettes are not available, 
use 10 mL disposable plastic syringes with 0.2 mL gradations (e.g. 
NeoJect brand). 
Students can estimate between the lines to at least 0.05 mL. 
This is sufficiently precise. 
If more than 10 mL are required, 
the student can simply refill the syringe.

\section{Condenser}
Pass clear plastic tubing through a water bottle filled 
with cold water and prevent leaks with super glue. 
If your condenser is under-performing (i.e. 
steam comes out), 
coil the plastic tubing so a greater length is in the water. 
If you condenser is still under-performing 
or you plan to use it for longer period of time, 
devise a way to keep changing water inside the water bottle 
to keep it from getting too hot. 
Or, 
submerge this condenser in a trough of water. 
You could even run the plastic tubing through the sides of a bucket.

\section{Deflagrating Spoon}
For heating chemicals to observe melting, 
decomposition, 
or other changes on heating, 
metal spoons work well. 
They can usually be cleaned by scrubbing with steel wool, 
although for the national exam you might buy new spoons. 
Bigger spoons may be less expensive than the smallest ones.

\section{Delivery Tube}
You can buy clear plastic tubing at many hardware shops. 
Even better are intravenous ‘giving sets’ available at pharmacies. 
If you are trying to prepare a gas that would corrode this tubing, 
think about what it would do to your students' lungs 
and consider a different experiment.

\section{Droppers}
There is no need to buy these. 
A 2 mL syringe works better and costs very little. 
They are available at almost any pharmacy.

\section{Electrodes}
\textit{See Carbon (graphite), 
Copper, 
Iron, 
and Zinc in the section on Sources of Chemicals}

\section{Electrolytic cell}
Remove the plungers from two 10 mL syringes 
and bore out the needle port with something sharp (knife, 
thin pliers, 
etc). 
Remove material gradually, 
rotating the piece to ensure a circular cut. 
When the hole is just big enough, 
force through a graphite battery electrode 
so 0.5-1cm remains on the outside. 
Twist the electrode during insertion to prevent it from snapping. 
The gap should be air-tight, 
but if the cut was too big or irregular 
you can seal the holes with super glue. 
Attach wires to the exposed part of the electrodes.

To use the cell, 
fill the tubes with your electrolyte solution 
and place them wire end up in the cut off bottom of a large water bottle, 
also filled with electrolyte solution. 
Attach the wires to a power supply 
(three or more 1.5V dry cell batteries in series, 
a 6V motorcycle battery, 
or a 12V car battery) to start electrolysis. 
The volume of gas produced at each electrode 
may be measured by the gradations on the syringes, 
and other products (copper metal plating, 
iodine in solution) may be clearly observed.

\section{Flasks}
Flasks can generally be replaced with clean, 
used glass liquor bottles, 
available in most markets. 
You can also make arrangements with local bars 
to reserve empty cans and bottles for your school. 
When using these flasks for titrations, 
students must practice swirling enough that the solution remains well mixed. 
Small water bottles may also be used.

Sometimes flasks are needed for dissolving salts 
to make solutions as their shape is particularly well suited 
for thorough mixing. 
But a half full plastic water bottle with a good cap 
can be shaken much more vigorously 
and will work as well if not better for most solutions. 
Plus, 
the solution is then already in a storage container.

If you need to prepare a solution that requires heating, 
be creative. 
Starch solution, 
for example, 
can be prepared in an aluminum pot without trouble. 
Liquor bottles also have caps for shaking and heat well 
(with the cap off!) in a water bath, 
especially if heated slowly.

\section{Funnel}
Plastic funnels are available at the market. 
Metal funnels are usually less expensive 
but need to be varnished for use with more reactive chemicals like acids. 
Glass funnels are entirely unnecessary. 
If you order funnels from a lab supply company 
you should buy plastic – 
plastic funnels both from the supply company and the markets 
are suitable for concentrated acids.

\section{Glass blocks}
Glass blocks from a lab supply company are 
generally 15 mm thick rectangular pieces of glass with beveled edges, 
so students do not cut themselves. 
They can be expensive, 
especially if you need many. 
Fortunately, 
it is possible to buy your own glass 
and find a craftsman to make blocks for you, 
especially if you insist on the importance of clean, 
parallel cuts. 
 
8mm glass is relatively common in towns 
and 10mm glass can be found in industrial areas of the most major cities. 
12mm and thicker glass exists though is even more difficult to find. 
However, 
for most optics practicals, 
several pieces of thinner glass can simply be stacked together and 
turned on their edge. 
This is a powerful way of showing refraction, 
and the necessary material (ordinary glass) is cheap and widely available.

\section{Gloves}

\subsection{Latex gloves}
These are worthless to the chemist, 
detrimental in fact because they make the hands less agile 
and give the user a false sense of security. 
Concentrated acids will burn through latex. 
Organic chemicals will pass straight through (and then through your skin) – 
without any obvious signs. 
One Shika author learned this when the skin on his hand started peeling off, 
under the latex glove. 
The only reasons to wear latex gloves is if one has open cuts on the hands 
and has no choice but to perform the practical (e.g. 
national exams), 
or if one needs to perform first aid. 
If for these reasons you want some of these gloves, 
pharmacies sell boxes of one hundred. 
Do not waste money on the individually wrapped sterile gloves.

The biology teacher may want to wear gloves for handling specimens. 
Latex are appropriate for this. 
Human skin is also relatively impervious if it is free of cuts. 
Just wash well with soap and water after handling specimens.

\subsection{Thick gloves}
Thick rubber gloves that withstand exposure more corrosive chemicals 
are sold by village industry supply companies, 
and some hardware stores. 
These gloves will withstand concentrated acids 
for long enough to protect your hands. 
They will, 
however, 
inevitably make you more clumsy, 
and more likely to splash acid or drop the bottle. 
Given that splashed acid and especially a broken bottle 
are much worse than some burned skin on the hand, 
using these gloves for concentrated acids is not recommended. 
Have weak base solution available for treating burns immediately 
and work carefully.

Thick gloves are recommended for when working with organic solvents. 
Remember that the most dangerous organic solvents (benzene, 
carbon tetrachloride) should never be used in a school, 
with or without gloves. 
Also remember that students will probably not have these gloves, 
so do not give them any chemicals 
that you would not use without the gloves. 
If you need to measure out one hundred samples of ether, 
for example, 
wear gloves because this task presents 
a much more significant exposure risk 
than an individual student handling her sample. 
If you are demonstrating technique for the students, 
however, 
do not wear gloves, 
unless you expect all of your students to wear them for the same process.

In general, 
avoid using chemicals that would make you want to wear gloves.

\section{Goggles}
These are essential. 
There are all sorts of ways to make goggles. 
For example, 
students can tie a strip of clean plastic around their heads 
so that it protects their eyes. 
You can make goggles from clear plastic and stiff paper and cardboard. 
Sunglasses work great. 
Be creative! For science labs, 
goggles do not need to be impact resistant – 
they just need to stand between hazardous chemicals and your eyes. 
If you think there is any risk of students getting chemicals in their eyes, 
they should wear goggles. 
Anyone handling concentrated acid (or battery acid) should wear goggles. 
We only have two eyes, 
and they are very vulnerable to permanent damage. 
Skin heals after an acid burn – eyes may not.

\section{Heat Sources}

There are many different fuels that may be burned: gas in Bunsen burners, 
butane in gas lighters, 
alcohol in spirit burners, 
kerosene in common cook stoves, 
refined heavy oil (used by caterers) in metal cups, 
wax in candles, 
and charcoal in clay or metal stoves. 
Fire has three uses in the school laboratory – heating solutions, 
heating solid samples, 
and flame tests – so it is good to know which you are trying to do.

\subsection{Heating solutions}
The ideal heat source has a high heat rate (Joules transferred per second), 
little smoke, 
and cheap fuel. 
A charcoal stove satisfies all of these 
but takes time to light and requires relatively frequent re-fueling. 
Kerosene stoves have excellent heat rates but are smoky. 
Alcohol infused refined heavy oil burns smokeless 
and the heat rate scales with the size of the container – 
filling the bottom half of a cut off aluminum can 
will produce significant heat per unit time. 
In some countries this fuel is advertised for home and commercial use (e.g. 
Motopoa brand in Tanzania) and use is easy to find. 
In other countries, 
it might be much less common.

\subsection{Heating solids}
The ideal heat source has a high temperature and no smoke. 
Ideal would be be a Bunsen burner. 
For heating small objects for a short time (no more than 10-20 seconds), 
a butane lighter provides a very high temperature. 
Refined heavy oil will provide a flame of satisfactory temperature 
for as long as necessary.

\subsection{Flame tests}
The ideal heat source has a high temperature 
and produces a non-luminous flame. 
The Bunsen burner is ideal. 
The next best flame is again refined heavy oil – hot and non-luminous. 
Spirit burners produce a non-luminous flame at much greater cost, 
unless methylated spirits are used as fuel 
in which case the flame is much cooler. 
A butane lighter produces a very hot flame of sufficient size 
and time for flame tests although the non-luminous region is small. 
Kerosene stoves will work for some salts, 
especially if you pull the wicks longer 
or remove the outer protective shell (usually green) 
to give students access to the hotter blue flame in between the inner shells.

As can be seen by the above discussion, 
alcohol infused heavy oil burners provide the best compromise heat source. 
They are also the easiest of these heat sources to use – 
pour the fluid into an open-topped metal container and set it on fire. 
They are also the safest heat source – 
they produce no smoke (unlike kerosene and candles), 
do not have fuel that spreads when it spills (kerosene and ethanol), 
nor can explode like gas. 
Students do not have to hold the burning apparatus (as with a lighter) 
and the flame may be extinguished by simply blowing it out 
or smothering it with a lid. 
Finally, 
the burners themselves are free – soda bottle tops, 
medicine and liquor bottle caps, 
cut aluminum cans, 
and metal tins; different sizes for different size flames. 
If you have access to this fuel, 
we strongly recommend using it.

No matter which heat sources you use, 
always have available firefighting equipment that you know how to use. 
See the section on Fire Fighting in the Laboratory Safety section. 
Remember that to put out a Bunsen burner safely, 
you need to turn off the gas.

\section{Lightbulbs}
Small shops, 
hardware stores, 
LEDs from broken phone chargers or flashlights, 
electronic shops. 
The external phone battery chargers that clamp the battery 
under a plastic jaw have a row of four LEDs with unusually long leads, 
perfect for wiring into circuits.

\section{Meter Rule}
Buy one, 
take it and a permanent pen to a carpenter, 
and leave with twenty. 
Measure each new one to the original rule to prevent compounding errors. 
Ceiling board is a cheap source of flat wood, 
although it is not very stiff.

\section{Microscope}
\textit{See the instructions in Low Cost Microscopy}

\section{Mirrors}
Large sheets of mirror glass is available in most towns, 
and scrap is usually available for sale. 
Glass vendors generally have the tools 
to make small squares of mirror glass 
and you can superglue these to small wooden blocks so they stand upright. 
These will also work for the hand mirrors sometimes required in biology.

\section {Mortar and Pestle}
To powder chemicals, 
place them between two nested metal spoons and grind down. 
Alternatively, 
you can crush chemicals on a sheet of paper on a table 
by pressing on them with the bottom of a glass bottle.

\section{Nichrome Wire}
For flame tests in chemistry, 
you can use a steel wire thoroughly scraped clean with iron or steel wool. 
For physics experiments, 
\textit{see Wire}

\section{Optical Pins}
Ordinary pins are cheap and perfectly effective. 
To make them easier to see, 
buy some brightly colored nail polish and paint the head. 
If you happen to purchase many plastic syringes, 
the needles make excellent optical pins. 
Pinch a point in the shaft with pliers so no one can take the needle 
and use it for injecting anything.

\section{Pipettes}
Do not buy and do not use. 
Use disposable plastic syringes. 
They come in 1, 
2, 
5, 
10, 
20, 
25, 
30, 
and 50 mL sizes and are available at both pharmacies and veterinary shops. 
These are easier to use, 
much more safe (no danger of mouth pipetting), 
do not break and are much less expensive. 
They are also often more accurate 
as glass pipettes are often incorrectly calibrated. 
To use them, 
suck first 1  mL of air and then put the syringe into the solution to suck up the liquid. 
There should be a flat meniscus under the layer of air.

\section{Resistors}
You can buy these from electronic stores in town or from a repairman. 
You can also find old radios or other trash circuit boards 
and take the resistors off them. 
This requires melting the solder, 
easy with a soldering iron, 
but also possible with a stiff wire thrust into a charcoal stove. 
If you need to know the ohms, 
the resistors tell you. 
Each has four strips (five if there is a quality band) 
and should be read with the silver or gold strip for tolerance on the right. 
Each color corresponds to a number: black = 0, 
brown = 1, 
red = 2, 
orange = 3, 
yellow = 4, 
green = 5, 
blue = 6, 
violet = 7, 
gray = 8, 
white = 9, 
and additionally for the third stripe, 
gold = -1 and silver = -2. 
The first two numbers should be taken as a two digit number, 
so green violet would be 57, 
red black 20, 
etc. 
The third number should be taken as the power of ten (a 10n term), 
so red orange yellow would be 23 x104 or 230000, 
red brown black would be 21 x100 or just 21 
and blue gray silver would be 68x10-2 or 0.68. 
The unit is always ohms. 
The fourth and possibly fifth bands may be ignored.

Capacitors, 
diodes, 
transistors, 
inductors and other useful circuit parts can also be bought 
at shops in town or liberated from old circuit boards, 
radios, 
phone chargers, 
etc. 
Capacitors tend to state their capacitance in microFarads on their bodies.

\section{Retort stand}

\subsection{To hold burettes}
Satisfactory retort stands may be produced 
by cutting a piece of cement reinforcing rod (re-rod, 
about 1 cm in diameter) and placing it in a metal tin full of wet cement. 
Once the cement hardens, 
you may attach a boss head and a clamp to have an equivalent stand. 
Stiff wire may be used in place of boss heads and clamps 
but they do not hold burettes as well. 
Especially if you have the misfortune of owning fragile glass burettes, 
the investment in good clamps is worthwhile.

Of course, 
if you use syringes as burettes, 
there is no need for a retort stand.

\subsection{To hold pendulums}
Hang the pendulum from any elevated point. 
You can place a chair on a desk and hang a pendulum form the legs, 
or for a smaller diameter rod anchor a welding stick under a large rock.

\subsection{To hold other apparatus}
Improvise! 
Both wire and strings of bicycle inner tube are versatile 
and effective binding agents.

\section{Scalpels}
Razor blades. 
You can find ways to modify them 
to remove one of the edges and add a handle, 
though students are pretty skilled with these blades. 
Dull blades should be discarded – 
because students need to apply more pressure when using them, 
there is a greater risk of slipping and thus of cuts. 
Sharp tools are much safer. 
For dissection, 
find a way to attach a handle to the blade 
to increase the pressure 
the student is able to safely apply to the cutting point.

\section{Spatula}
Get stainless steel spoons from the market or a small shop. 
These work better than traditional spatula. 
For removing salts from containers, 
you can use the other end of the spoon. 
Make sure to clean all metal tools promptly after using with hydroxide, 
potassium manganate (VII), 
or manganese (IV) oxide. 
If you forget and the spoon corrodes, 
you can remove most of the corrosion 
by scraping with another spoon or steel wool.

\section{Springs}
Ask around. 
Springs may be found at hardware stores, 
bike stores, 
and especially junk merchants in markets. 
If you tell some of these people what you are looking for, 
they will probably find it for you by the next time you come to town. 
There are also much thinner springs 
readily available at stores that sell window blinds. 
There is a small white long (1 meter to 2 meters) spring 
used to hold up window shades in homes. 
Use a knife to remove the plastic coating. 
Cut the spring into small segments 
and you have 10 to 20 5cm long springs with a useful spring constant. 

\section{Stoppers}
These can be made by the people who cut up old tires 
or you can make them yourself from old sandals. 
However, 
stoppers are rarely required. 
If you are using the stopper because you want to shake a flask, 
consider just using a water bottle with a screw cap. 
If you want a stopper with a hole for passing out a gas you are producing, 
again use a water bottle 
and super glue the tip of a clear plastic pen body into the cap. 
You can then mount rubber tubing onto the pen tip 
for a reliable connection.

\section{Stopwatches}
Stop watches with the look of athletic and laboratory stopwatches 
are often available in big city goods markets 
for much less than at laboratory supply stores. 
Many digital wristwatches also have a stopwatch feature 
and these are widely available.

\section{Test tubes}

\subsection{Plastic test tubes}
These will work for everything in biology, 
everything in physics, 
and everything in chemistry except thermal decomposition of salts. 
They have the obvious advantage of not breaking. 
To make these, 
remove the needle and plunger from 10 mL syringes. 
Heat the end of the shell where the needle joined in a flame until it melts. 
Press the molten end against a flat surface (like the end of the plunger) 
to fuse it closed. 
If the tube leaks, 
fuse it again. 
Test tubes made this way may be heated in a water bath up to boiling, 
hot enough for most experiments.

\subsection{For thermal decomposition}
\textit{see Deflagrating spoon}

\subsection{Glass test tubes}
You can purchase glass vacuum sample tubes 
in bulk from medical and veterinary supply shops. 
These may generally be heated in open flame, 
although they are not labeled as borosilicate (Pyrex), 
and will probably break sooner. 
We have not tested them in Bunsen burners.

\section{Test tube holder / tongs}
For prolonged heating, 
you can wrap stiff wire tightly around the lip of the test tube. 
For shorter heating, 
you can do the same with a strip of paper or clot. 
HYou can also find a carpenter to make large wooden holders. 
Clothespins work well if you can find them large enough, 
or if you use smaller tubes, 
or it you use tubes made from syringes with useful flanges at the top.

\section{Test tube racks}
If you have test tubes, 
it is nice to have something to keep them from falling over and breaking. 
You can get some styrofoam and punch holes in it, 
or make one from a plastic water bottle - 
just put sand in the bottom to increase stability 
and prevent hot tubes from melting the bottle. 
You can make a fancier rack by cutting up a bottle: 
slice it in half along the vertical axis 
and rest the two cut edges on a flat surface 
so the bottle half bows up towards you. 
Cut holes into it for the test tubes. 
Or just use a rectangular bottle. 
The possibilities are endless. 
Local carpenters can also make them from wood 
and this is a good if small way 
to get the village more involved with the school. 
Ordering these from a supply company should only be a last resort. 

\section{Tripod stands}
A welder or metal worker in town can make these. 
Bring a sample to make sure the stand is not too short or too tall. 
You can also make your own from stiff wire.

\section{Volumetric “Glass”ware}
It is often necessary to measure large volumes (100 mL – 2 L) 
of solution rather accurately. 
The ultimate equipment for this job are flat bottomed volumetric flasks, 
the spherical globes with the long vertical neck. 
Such precision is usually not required in secondary school, 
especially because titration solutions should be standardized prior to use. 
Relatively accurate measurements may be made using devices 
that are found in every village: plastic water bottles.

The trick is that because water bottles are made in a factory 
by injection molding, 
they are essentially all identical in size. 
Most bottles have various markings molded into the bottle, 
and since the engineers that design these bottles tend 
to prefer round numbers, 
many bottles have very convenient marks.

Borrow a graduated cylinder and find samples of various water bottles. 
Identify the volume of every useful mark on the bottles 
and then share the information widely.

Volumes not immediately measurable with bottles 
may sometimes be measured by addition or subtraction of bottle measures. 
Remember those egg timer problems?

\section{Wash bottle}
Put a hole in the cap of a water bottle. 
Perhaps the best method is to use a syringe needle. 
Gently and firmly apply pressure and push the needle through the lid. 
This is gives a very small hole 
that responds well to the application of pressure. 
If more movement of liquid is needed, 
puncture more holes with the syringe needle. 
You can also heat a stiff wire in a flame and burn the hole, 
hammer a nail through the cap, 
or use a small knife like an awl.

\section{Water bath}
Take an aluminum pot and fill with water. 
Put this on top of any heat source and let the water heat up. 
Place the test tubes in the hot water to heat. 
If heating the liquids in the test tubes to a specific temperature, 
make sure students put the thermometer in the test tube, 
not the water. 
For smaller scale work, 
use the bottom half of an aluminum can.

Many times, 
the water bath will be much larger than the test tubes 
and they might fall over, 
into the water. 
Devise methods to prevent this. 
You might clamp the tubes to the side with clothespins, 
attach parallel wires to the container to rest the tubes in between, 
or punch holes in a flat piece of plastic to put over the top of the water.

\section{Weights}

\subsection{Crude weights}
Batteries, 
coins, 
glass marbles from town, 
etc. 
You do NOT need to know the mass of these objects; just make new units. 
For example, 
if using marbles, 
measure force as 2 marble-meters-per-second-squared. 
This is an excellent way to teach the meaning of units. 
Note that coins often have surprising variation depending on age, 
wear, 
etc.

\subsection{Adding weight in known intervals}
For practicals where specific weights must be added 
to a system of unknown mass, 
e.g. 
when weights are added to a weigh pan in spring practicals, 
water may be used. 
As the weight of the pan is both unknown and irrelevant, 
consider “zero added mass” the displacement of the pan with an empty water bottle. 
Then, 
added 50 g, 
100 g, 
etc masses with water bottles with 50 mL, 
100  mL, 
etc of water.

/subsection{Precise weights}
Find small (250 mL) water bottles. 
Get as many as you need weights. 
They must be all the same type.
Remove the labels and make sure the bottles are completely dry. 
This is readily accomplished 
by leaving them uncapped outside on a warm day.
Use an accurate balance to find the mass of one bottle, 
cap included. 
If you do not have an accurate balance, 
visit a school that does. 
You should only have to do this once.
Subtract the mass of the empty bottle (say, 
1.24 g) from the mass you want for your weight (say, 
50 g). 
This mass in grams will be provided 
by this volume of water in milliliters (so, 
for our example, 
50 – 1.24 = 48.76 mL water). 
Use a plastic syringe to add exactly this mass of water to the dry bottle.
Cap the bottle firmly and label it with permanent pen: “50 g weight” 
If you want your masses to have hooks, 
attach some wire around the neck of the bottle 
and bend one end to make a hook. 
Of course, 
do this before step 3 so you add that much less water.

You could also make weights by using a balance 
to fill small plastic bags with sand. 
This makes smaller weights (good!) 
but requires a balance for making each one, 
and a balance to replace any one that rips open.

\section{White tiles}
White paper works just as well. 
If your students are using syringes as burettes, 
they can also hold their flask up against a white wall.

\section{Wire}

\subsection{All-purpose wire}
Use speaker wire, 
the pairs of colored wires brained together. 
The wire is easy to strip using a wide variety of tools, 
or just your teeth, 
specifically the space between your incisors and front molars.

\subsection{Specific gauge wire}
Copper wire is imported and sold in large quantities in port cities 
for use in industry. 
These wires, 
often used for motor winding and other electrical applications, 
are generally coating with an insulating varnish 
and come in a variety of diameters (gauges). 
A useful chart for converting diameter to gauge may be found at: 
http://www.dave-cushman.net/elect/wiregauge.html. 
If the wire is sold by weight, 
you can find the length if you know the diameter - 
the density of copper metal at room temperature is 8.94 g/cm3. 
For example, 
with 0.375 mm wire, 
250 g is about 63 meters.


\chapter{Sources of Chemicals}
\label{cha:sourcesofchemicals}
The following is a list of most of the chemicals 
used in science laboratories. 
For each we note local sources of these chemicals, 
low cost industrial sources of these chemicals, 
methods to manufacture these chemicals at your school, 
and/or functional alternatives to these chemicals. 
We also list information like other names, 
common uses, 
and hazards. 
Finally, 
we include descriptions of many of the compounds 
and confirmatory tests for some to assist 
with identification of unlabelled chemicals. 
For more information on this, 
see \nameref{cha:unknownchemicals}.

Chemicals are generally listed alphabetically by IUPAC name, 
although many compounds are also cross listed by their common name (e.g. 
acetone (common) / propanone (IUPAC)).

\section{2-methylpropanol}
\label{sec:methylpropanol}
Formula: (CH$_{3}$)$_{2}$CHCH$_{2}$OH\\
Other names: isobutanol\\
Description: clear liquid less dense than water, 
alcohol smell similar to isopropanol (American rubbing alcohol)\\
Use: organic solvent for distribution (partition) experiments\\
Alternative: paint thinner or kerosene\\
Note: if ordering this chemical for the national exam, 
make sure that you get this chemical exactly. 
Other compounds, 
e.g. 
CH$_{3}$CH$_{2}$CH$_{2}$(OH)CH$_{3}$ (butan-2-ol) are sometimes sold 
as ‘isobutanol’ but do not work the same way.

\section{Acetaldehyde}
See \nameref{sec:ethanal}.

\section{Acetic acid}
See \nameref{sec:ethanoic}.

\section{Acetone}
See \nameref{sec:propanone}.

\section{Alum}
See \nameref{sec:potalsulf}.

\section{Ammonia solution}
\label{sec:ammoniasol}
Formula: NH$_{3(aq)}$\\
Other names: ammonium hydroxide, 
ammonium hydroxide solution\\
Description: clear liquid less dense than water, 
completely miscible in water, 
strong biting smell similar to old urine\\
Use: qualitative analysis, various experiments\\
Source: released from an aqueous mixture of ammonium salt and hydroxide, 
for example calcium ammonium nitrate and sodium hydroxide. 
The gas can be trapped and dissolved in water.\\
Alternative: to distinguish between zinc and lead cations, 
add dilute sulfuric acid dropwise. 
The formation of a white precipitate -- lead sulfate -- confirms lead.
Note: ammonia solution also is called ammonium hydroxide 
because ammonia undergoes autoionization to form ammonium and hydroxide ions. 
Just like water, 
there is an equilibrium concentration of the ions in an ammonia solution.

\section{Ammonium dichromate}
Formula: (NH$_{3}$)$_{2}$Cr$_{2}$O$_{7}$\\
Description: orange crystals soluble in water\\
Use: qualitative analysis (identification of sulfur dioxide gas)\\
Hazard: toxic, 
water pollutant\\
Alternative: make ammonium/potassium dichromate paper tests. 
Many can be made from a single gram of ammonium/potassium dichromate.

\section{Ammonium hydroxide solution}
See \nameref{sec:ammoniasol}.

\section{Ammonium carbonate, chloride, nitrate, and sulfate}
Use: qualitative analysis, 
preparation of ammonia\\
Alternative: to teach the identification 
and confirmation of ammonium salts and to prepare ammonia, 
use calcium ammonium nitrate.

\section{Ammonium thiocyanate}
Formula: NH$_{4}$SCN\\
Use: confirmation of iron III in qualitative analysis\\
Alternative: addition of sodium ethanoate 
should also produce a blood red solution; 
additionally, 
the test is unnecessary, 
as iron III is also the only chemical 
that will produce a red/brown precipitate with sodium hydroxide solution 
or sodium carbonate solution.

\section{Ascorbic acid}
Other names: vitamin C\\
Formula: C$_{6}$H$_{7}$O$_{7}$\\
Description: white powder, 
but pharmacy tablets often colored\\
Confirm: aqueous solution turns blue litmus red 
AND decolorizes dilute iodine or potassium permanganate solution\\
Use: all-purpose reducing agent, 
may substitute for sodium thiosulfate in redox titrations, 
removes iodine and permanganate stains from clothing\\
Source: pharmacies

\section{Barium chloride and barium nitrate}
Use: confirmatory test for sulfate in qualitative analysis\\
Description: white crystals\\
Hazard: toxic, 
water pollutant\\
Alternative: lead nitrate will precipitate lead sulfate – 
results identical to when using barium

\section{Boric acid}
Formula: H$_{3}$BO$_{3}$\\
Description: white powder\\
Confirm: deep green flame color\\
Use: flame test demonstrations, preparation of sodium borate\\
Source: village industry supply shops, industrial chemical

\section{Benedict's solution}
\label{sec:benedict}
Description: bright blue solution\\
Confirm: gives orange precipitate when boiled with glucose\\
Use: food tests (test for reducing and non reducing sugars)\\
Hazard: copper is poisonous\\
Manufacture: combine 5 spoons of sodium carbonate, 
3 spoons of citric acid, 
and one spoon of copper sulfate in half a liter of water. 
Shake until everything is fully dissolved.

\section{Benzene}
Formula: C$_{6}$H$_{6}$\\
Description: colorless liquid insoluble in water\\
Use: all purpose organic solvent\\
Hazard: toxic, 
highly carcinogenic – see section on Dangerous Chemicals\\
Alternative: toluene is safer but for most solvent applications 
kerosene is equally effective and far less expensive.

\section{Butane}
Formula: C$_{4}$H$_{10}$\\
Source: the fluid in gas lighters is butane under pressure; 
liquid butane may be obtained at normal pressure with the help of a freezer

\section{Calcium ammonium nitrate}
Other names: CAN\\
Description: small pellets, 
often with brown coating; 
endothermic heat of solvation\\
Use: low cost ammonium salt for teaching qualitative analysis; 
not as useful for teaching about nitrates 
as no red/brown gas released when heated. 
May be used for the preparation of ammonia and sodium nitrate.\\
Source: agricultural shops (fertilizer)\\

\section{Calcium carbonate}
Formula: CaCO$_{3}$\\
Description: white powder, 
insoluble in water
Confirm: brick red flame test and acid causes effervescence\\
Use: demonstration of reactivity of carbonates, 
rates of reaction, 
qualitative analysis\\
Source: coral rock, 
sea shells, 
egg shells, 
limestone, 
marble, 
white residue from boiling water\\
Local manufacture: prepare a solution of aqueous calcium 
from either calcium ammonium nitrate or calcium hydroxide 
and add a solution of sodium carbonate.\\ 
Calcium carbonate will precipitate and may be filtered and dried.

\section{Calcium chloride and calcium nitrate}
Description: highly deliquescent colorless crystals 
(poorly sealed containers often become thick liquid)\\
Use: qualitative analysis salts, 
drying agents\\
Alternatives (qualitative analysis): 
to practice identification of the calcium cation, 
use calcium sulfate; 
to practice identification of the chloride anion, 
use sodium chloride\\
Alternative (drying agent): sodium sulfate

\section{Calcium hydroxide}
Formula: Ca(OH)$_{2}$\\
Other names: quicklime\\
Description: white to off white powder, 
sparingly soluble in water\\
Use: dissolve in carbonate-free water to make limewater\\
Source: building supply shops\\
Alternative: add a small amount of cement to water, 
let settle, 
and decant the clear solution; 
this is limewater.

\section{Calcium oxide}
Formula: CaO\\
Other names: lime\\
Use: reacts with water to form calcium hydroxide, 
thus forming limewater\\
Source: cement is mostly calcium oxide

\section{Calcium sulfate}
Formula: CaSO$_{4} \cdot$2H$_{2}$O\\
Other names: gypsum, 
plaster of Paris\\
Description: white powder, 
insoluble in cold water but soluble in hot water\\
Use: qualitative analysis\\
Source: building supply companies (as gypsum powder)

\section{Carbon (amorphous)}
Source: soot, 
charcoal (impure)

\section{Carbon (graphite)}
\label{sec:carbongraphite}
Use: element, \\
inert electrodes for chemistry and physics
Source: dry cell battery electrodes, 
pencil cores (impure)

\section{Carbon dioxide}
Preparation: react an aqueous weak acid 
(citric acid or ethanoic acid) with a soluble carbonate 
(sodium carbonate or sodium hydrogen carbonate)

\section{Carbon tetrachloride}
See \nameref{sec:tetrachloromethane}.

\section{Chloroform}
See \nameref{sec:trichloromethane}.

\section{Citric acid}
Formula: C$_{6}$H$_{8}$O$_{7}$ =  CH$_{2}$(COOH)COH(CHOOH)CH$_{2}$COOH\\
Description: white crystals soluble in water, 
endothermic heat of solvation\\
Use: all purpose weak acid, 
volumetric analysis, 
melting demonstration, 
preparation of carbon dioxide, 
manufacture of Benedict's solution\\
Hazard: acid – keep out of eyes!\\
Source: markets (sold as a spice often with a local name), 
supermarkets

\section{Cobalt chloride}
Use: test for water (hydrated cobalt chloride is pink)\\
Hazard: cobalt is poisonous\\
Alternative: white anhydrous copper sulfate turns blue when hydrated

\section{Copper}
\label{sec:copper}
Use: element, 
preparation of copper sulfate, 
electrochemical reactions\\
Description: dull red/orange metal\\
Source: electrical wire -- e.g. 
2.5~mm gray insulated wire has 50~g of high purity copper per meter.\\
Note: modern earthing rods are only copper plated, 
and thus no longer a good source of copper

\section{Copper carbonate}
Formula: CuCO$_{3}$\\
Description: light blue powder\\
Confirm: blue/green flame test and dilute acid causes effervescence\\
Use: qualitative analysis, 
preparation is a demonstration of double decomposition\\
Hazard: powder may be inhaled; 
copper is poisonous\\
Local manufacture: prepare solutions of copper sulfate 
and sodium carbonate and mix them. 
Copper carbonate will precipitate 
and may be purified by filtration and drying.

\section{Copper chloride and copper nitrate}
Description: blue-green (copper chloride) 
and deep blue (copper nitrate) salts \\
Use: qualitative analysis\\
Alternatives: for practice identifying the copper cation, 
use copper sulfate; 
for practice identifying the chloride anion, 
use sodium chloride

\section{Copper oxygen chloride}
Formula: Cu$_{2}$OCl\\
Other names: copper oxychloride, 
blue copper\\
Description: light blue powder\\
Hazard: powder may be inhaled; 
copper is poisonous\\
Source: agricultural shops (fungicide)

\section{Copper sulfate}
Formula: CuSO$_{4}$ (anhydrous), 
CuSO$_{4} \cdot$5H$_{2}$O (pentahydrate)\\
Description: white (anhydrous) or blue (pentahydrate) crystals\\
Confirm: blue/green flame test 
and aqueous solution gives a white precipitate 
when mixed with lead or barium solution\\
Use: qualitative analysis, 
demonstration of the reactivity series, 
manufacture of Benedict's solution, 
test for water\\
Source: imported ``local'' medicine (manufactured in India).\\ 
Local manufacture: Electrolyze dilute (1-2~M) sulfuric acid 
with a copper anode and inert (e.g. 
graphite) cathode. 
Evaporate final solution until 
blue crystals of copper sulfate pentahydrate precipitate. 
To prepare anhydrous copper sulfate from copper sulfate pentahydrate, 
gently heat until the blue color has faded. 
Strong heating will irreversibly form black copper oxide. 
Store anhydrous copper sulfate in an air-tight container – 
otherwise atmospheric moisture will reform the pentahydrate.

\section{Dichloromethane}
Formula: CH$_{2}$Cl$_{2}$\\
Use: organic solvent for distribution (partition) experiments\\
Hazard: toxic by inhalation and ingestion (mouth pipetting) 
and by absorption though skin\\
Alternative: paint thinner or kerosene, 
although these are less dense than water

\section{Diethyl ether}
Formula: (CH$_{3}$CH$_{2}$)$_{2}$O\\
Description: colorless liquid with smell similar to nail polish remover, 
evaporates quickly at room temperature\\
Use: organic solvent for distribution (partition) experiments, 
demonstration of low boiling point\\
Hazard: extremely flammable (boils near room temperature) 
and dangerous to inhale (unfortunate as it is very volatile!). 
It is of the utmost importance not to mouth pipette this chemical. 
Breathing ether was the first anesthesia, 
discontinued because it can be lethal.\\
Alternatives (distribution/partition): paint thinner or kerosene\\
Alternative (low boiling point): propanone

\section{Distilled water}
Formula: H$_{2}$O and nothing else!\\
Use: qualitative analysis\\
Source: rain water.\\
Allow the first 15 minutes of rain to clean off the roof 
and then start collecting water. 
In schools in dry climates, 
collect as much rain water as possible during the rainy season. 
Use it only for qualitative analysis, 
preparation of qualitative analysis reagents, 
and manufacture of qualitative analysis salts.\\ 
Distilled water may also be purchased at most petrol stations 
and automotive shops.\\
Local manufacture: Heat water in a kettle 
and use a rubber hose to bring the steam through a container of cold water. 
Collect the condensate -- pure water.\\
Alternative: river or tap water is almost always sufficient. 
Volumetric analysis never needs distilled water 
if you follow the instructions in Relative Standardization. 
Also, 
the tap water in many places is sufficient for even qualitative analysis.

\section{Ethanal}
\label{sec:ethanal}
Formula: CH$_{3}$CHO\\
Other names: acetaldehyde\\
Description: clear liquid with a foul smell\\
Local manufacture: oxidize ethanol with potassium permanganate\\
Note: the product is truly bad smelling and probably unhealthy to inhale. 
Include this entry only to show that rather than useful ethanoic acid, 
one can only get useless ethanal by chemical oxidation of ethanol; 
manufacture of ethanoic acid requires elevated temperature 
and high pressure vessels (or biology, 
as in the traditional manufacture of vinegar). 
The reaction at small scale 
(1~mL of ethanol used to decolorize dilute potassium permanganate) 
is useful when teaching oxidation of alcohols in organic chemistry.

\section{Ethandioic acid}
Formula: C$_{2}$H$_{2}$O$_{4} \cdot$2H$_{2}$O\\
Other names: oxalic acid\\
Description: clear crystals\\
Use: volumetric analysis, 
primary standard for absolute standardization, 
reducing agent (oxidized to carbon dioxide)\\
Hazard: poisonous (also acidic)\\
Alternative: substitute citric acid or ethanoic acid 
for weak acid solutions and use ascorbic acid as a reducing agent.

\section{Ethanoic acid}
\label{sec:ethanoic}
Formula: CH$_{3}$COOH\\
Other names: acetic acid\\
Description: clear liquid, 
completely miscible with water, 
strong vinegar smell\\
Use: all purpose weak acid, 
volumetric analysis\\
Source: 96\% solution available from village industry supply shops, 
vinegar (5\% solution) available in small shops and supermarkets\\
Safety for 96\% ethanoic acid: HARMFUL VAPORS. 
Use outside or in a well ventilated space. 
CORROSIVE ACID. 
Always have dilute weak base solution (e.g. 
sodium hydrogen carbonate) available to neutralize spills. 
Wear gloves and goggles when handling. 
Do not induce vomiting if ingested.\\
Alternative: for a weak acid, 
citric acid. 

\section{Ethanol}
Formula: CH$_{3}$CH$_{2}$OH\\
Description: clear liquid, 
completely miscible with water, 
strong and sweet alcohol smell\\
Use: solvent, 
extraction of chlorophyll, 
removes permanent marker, 
preparation of POP solution, 
distillation, 
preservation of biological specimens\\
Hazard: ethanol itself is a mild poison, 
and methylated spirits and other industrial alcohol contain 
additional poisonous impurities (methanol) 
specifically so that no one drinks it\\
Sources: methylated spirits are 70\% ethanol, 
hard liquor is often 30-40\%, 
village-brewed concentrated alcohol varies 
and may contain toxic quantities of methanol\\
Local manufacture: fermentation of sugar by yeast will produce 
up to a 15\% solution -- at that point, 
the yeast dies; 
distillation can in theory concentrate this to up to 95\%, 
but this is hard with simple materials. 
Nevertheless, 
preparing ethanol of sufficient concentration to dissolve POP (50-60\%) 
is quite possible.\\
Note: the color of most methylated spirits makes them undesirable 
for preparation of POP; 
hard liquor will suffice, 
but poorly because of its relatively low ethanol content. 
Colored methylated spirits can be run 
through a simple distillation apparatus to produce colorless spirits, 
as the pigment is less volatile than the ethanol. 
Of course, 
methanol and other poisons remain, 
but the clear solution works beautifully for dissolving POP.\\ 
Beware that ethanol vapors are flammable -- 
a poorly constructed distillation setup may explode.

\section{Ethyl acetate}
See \nameref{sec:ethylethanoate}.

\section{Ethyl ethanoate}
\label{sec:ethylethanoate}
Formula: CH$_{3}$COOCH$_{2}$CH$_{3}$\\
Other names: ethyl acetate\\
Description: clear liquid, 
immiscible with water, 
smells like nail polish remover\\
Use: solvent\\
Source: nail polish remover (mixture with propanone)\\
Alternative: paint remover, 
paint thinner, 
or methylated spirits\\
Preparation (demonstration of esterification): 
mix ethanol and ethanoic acid 
with a catalytic amount of strong acid or base; 
the decrease in ethanoic acid can be detected 
by titration and the ethyl ethanoate can be detected by smell.

\section{Gelatin}
Source: may be extracted from chicken bones. 
This process is lengthy compared 
to purchasing gelatin powder from supermarkets. 
Be sure to purchase the non flavored varieties, 
usually in white boxes.

\section{Glucose}
Formula: C$_{6}$H$_{12}$O$_{6}$\\
Description: white powder\\
Use: food tests (biology), 
reducing agent\\
Sources: small shops, 
pharmacies\\
Note: for food tests, 
the vitamins added to most glucose products will not cause a problem

\section{Gold}
Source: a very thin coat of gold is plated 
onto the electrical contacts of cell phone batteries 
and mobile phone SIM cards.

\section{Graphite}
See \nameref{sec:carbongraphite}.

\section{Hydrochloric acid}
\label{sec:hydroacid}
Formula: HCl, 
36.5~g/mol, 
density 1.18~g/cm$^{3}$ when concentrated ($\sim$12~M)\\
Other names: muriatic acid, 
pH decreasing compound for swimming pools\\
Description: clear liquid, 
may be discolored by contamination, 
distinct smell similar to chlorine 
although sometimes smells strongly of vinegar\\
Confirm: decolorizes weak solutions of potassium permanganate; 
white precipitate in silver nitrate solution 
and effervescence with (hydrogen) carbonates\\
Use: volumetric analysis, 
qualitative analysis\\
Source: swimming pool chemical suppliers, industrial chemical\\ 
Safety: HARMFUL VAPORS. 
Use outside or in a well ventilated space. 
CORROSIVE ACID. 
Always have dilute weak base solution (e.g. 
sodium hydrogen carbonate) available to neutralize spills. 
Wear gloves and goggles when handling. 
Extremely toxic hydrogen cyanide gas formed 
on mixing with cyanides or hexacyanoferrate compounds. 
Toxic chlorine gas formed on reaction with oxidizing agents, 
especially bleach. 
Do not induce vomiting if ingested.\\
Alternative (strong acid): sulfuric acid\\
Alternative (acid): citric acid\\
Alternative (qualitative analysis): for the test for carbonates, 
use dilute sulfuric acid; 
to dissolve insoluble carbonates, 
nitric acid may be used instead

\section{Hydrogen}
Formula: H$_{2}$\\
Confirm: ``pop sound,'' i.e. 
ignites with a bang; 
in an inverted test tube the rapid movement of air 
near the mouth creates a rapid, 
high pitch ``whoosh'' that gives the ``pop'' name\\
Preparation: combine dilute acid (e.g. 
battery acid) and a reactive metal (steel wool or zinc) 
in a plastic water bottle. 
Attach a balloon to the top of the water bottle; 
being less dense than air, 
hydrogen will migrate up and slowly fill the balloon. 
Specific instructions for various alternatives are available 
in the Hands-On activities section. 
Before ignition, 
always move the balloon away from the container of acid.

\section{Hydrogen peroxide}
Formula: H$_{2}$O$_{2}$\\
Description: solutions are colorless liquids 
appearing exactly like water\\
Confirm: decolorizes potassium manganate (VII) solution 
in the absence of acid, 
neutral pH\\
Use: preparation of oxygen, 
general oxidizer and also may act as a reducing agent (e.g. 
with potassium permanganate)\\
Source: pharmacies sell 3\% (10 volume) and 6\% (20 volume) solutions 
as medicine for cleaning sores\\
Note: `20 volume' means it will produce 20 times its liquid volume in oxygen gas.

\section{Hydrogen sulfide}
Formula: H$_{2}$S\\
Description: colorless gas with the smell of rotting eggs, 
ocean mud, 
and other places of anaerobic respiration\\
Safety: the gas is quite poisonous, 
although the body can detect extremely small amounts\\
Preparation: a sufficient quantity to smell 
may be prepared by igniting sulfur in a spoon 
and then quenching it in water.

\section{Indicator}
\label{sec:indicator}
Source: red flowers\\
Preparation: Crush flower petals in water. 
Some effective flowers include rosella, 
bougainvillea, 
and hibiscus. 
Test other flowers near your school.\\
Note: For bougainvillea and some other flowers, 
extract the pigment with ethanol 
or hard alcohol to get a better color. 
Color will change from pink (acidic) to colorless (basic). 
Rosella will change from red (acidic) to green (basic).
For an indicator in redox titrations involving iodine, 
see starch solution.

\section{Iodine}
Formula: I$_{2(s)}$\\
Description: purple/black crystals\\
Local manufacture: add a little dilute sulfuric acid 
to iodine solution from a pharmacy. 
Then add sodium hypochlorite solution (bleach) dropwise 
until the solution turns colorless with solid iodine resting on the bottom. 
The solid iodine can be removed by filtration or decantation. 
If pure iodine is necessary, 
the solid may be purified by sublimation.\\
Note: this reaction produces poisonous chlorine gas. 
Therefore, 
produce iodine in a well ventilated area and stand upwind.

\section{Iodine solution}
\label{sec:iodinesol}
Composition: I$_{2}$ + KI dissolved in water and sometimes ethanol\\
Description: light brown solution\\
Confirm: turns starch blue or black\\
Use: food tests for detection of starch and fats\\
Source: pharmacies sell a ‘weak iodine solution’ 
or ‘tincture of iodine’ that is really about 50\% by mass iodine. 
To prepare a useful solution for food tests, 
dilute this 10:1 in ordinary water.\\
Note: to use this solution for detection of fats, 
it must be made without ethanol, 
spirits, 
alcohol and the like. 
Either kind works for detection of starch.

\section{Iron}
\label{sec:iron}
Use: element, 
demonstration of reactivity series, 
preparation of hydrogen, 
preparation of iron sulfide, 
preparation of iron sulfate\\
Source: for samples of the element 
and for use in electrochemical experiments, 
buy non-galvanized nails at a hardware store, 
or find them on the ground. 
You can tell they are not galvanized because they are starting to rust. 
Clean off the rust with steel wool prior to use. 
For samples of the element for preparation of other compounds, 
buy steel wool from small shops or supermarkets. 
This has a very high surface area / mass ratio, 
allowing for faster reactions.

\section{Iron sulfate}
Description: iron (II) sulfate is light green. 
If exposed to air and especially water, 
iron (II) sulfate oxidizes to form yellow/red/brown iron (III) sulfate.\\
Use: oxidation-reduction experiments, 
qualitative analysis\\
Local manufacture: add excess steel wool to battery acid 
and leave overnight or until the acid is completely consumed. 
Beware! This reaction produces poisonous sulfur dioxide gas! 
Decant the solution of iron sulfate and leave to evaporate. 
Gentle heating is useful to speed up evaporation, 
but be careful to not heat too strongly once crystals form.\\
Note: the product may contain both iron II sulfate and iron III sulfate – 
you can guess based on the color. 
Such a mixture may be used to demonstrate confirmation of iron 
with potassium hexacyanoferrate (II/III), 
though not the specificity of one versus the other. 
To see if any iron II sulfate is present, 
add a solution of the product 
to a very dilute solution of potassium permanganate. 
If the permanganate is decolorized, 
iron (II) is present. 
If the solid has any yellow or red color, 
iron (III) is present.

\section{Iron sulfide}
Use: preparation is a demonstration of chemical changes\\
Preparation: grind steel wool into a fine powder 
and mix with a similar quantity of sulfur. 
This is a mixture that may be physically sorted (e.g. 
with a magnet). 
Now, 
heat the mixture in a spoon over a flame. 
Iron sulfide will form. 
This is a chemical compound; 
the iron and sulfur can no longer be separated by physical means.

\section{Isobutanol}
See \nameref{sec:methylpropanol}.

\section{Lead}
Description: soft, 
dull gray metal\\
Hazard: toxic, 
especially its soluble compounds (e.g. 
lead acetate, 
chloride, 
and nitrate) and in powder form (e.g. 
lead carbonate)\\
Use: element\\
Source: electrodes from old car batteries; 
the old batteries themselves may be purchases from scrap dealers. 
Remember that the electrolyte may still be 5~M sulfuric acid 
and thus great care is required 
when opening these batteries to extract the electrodes. 
If you pay someone else to extract them, 
make sure they understand the hazards and use protective gear (gloves, 
goggles, 
etc.).

\section{Lead nitrate}
Formula: Pb(NO$_{3}$)$_{2}$\\
Use: qualitative analysis salt, 
alternative to barium chloride/nitrate when confirming sulfates\\
Hazard: toxic, 
water pollutant\\
Note: Yes, 
you could prepare this from lead metal and dilute nitric acid, 
and yes, 
this would be less expensive than buying lead nitrate. 
However, 
the process of dissolving a reactive metal in a highly corrosive acid 
to produce a toxic salt is anything but safe. 
Lead nitrate is a good chemical to purchase. 
Note that lead does not react with concentrated nitric acid.

\section{Lead shot}
Use: very dense material for building hydrometers, 
etc.\\
Source: shotgun shells from a firearm shop -- 
ask them to open them for you\\
Note: most lead shot these days is actually a bismuth compound 
to reduce the environmental pollution of spraying lead everywhere. 
To test the lead shot, 
put in a ceramic or metal container 
and heat over a charcoal or kerosene stove. 
If the metal is lead, 
it will melt. 
Bismuth melts at a much higher temperature.\\
Alternative: If you just need a dense material for physics experiments, 
use iron and adjust the calibration. 
This is both safer and less expensive. 
If you need lead as a chemical reagent 
1) see the entry for lead but 
2) consider another demonstration with a less poisonous material.

\section{Lithium}
Use: flame test demonstrations\\
Source: broken cell phone batteries from a phone repair shop\\
Extraction: Open the metal battery case by chipping 
or smashing it and then prying it open with pliers. 
There should be sealed packets inside. 
Stand upwind and cut these open; 
leave the contents to evaporate the noxious solvent for a few minutes. 
Do not breathe the fumes. 
After waiting ten minutes, 
remove the contents of the packets with pliers 
and unroll a strip of black covered silvery metal foil. 
Somewhere in here is some lithium ion. 
We used to think the silvery metal was lithium. 
That seems to be incorrect. 
Regardless, put some of the metal and the black coating
into a really hot flame (Bunsen burner, gas lighter)
and you should get the crimson flame color
characteristic of lithium.

\section{Magnesium carbonate}
Use: preparation is a demonstration of double displacement reactions 
as well as a qualitative analysis test\\
Local manufacture: Mix a solution of magnesium sulfate 
with a solution of sodium carbonate. 
Manganese carbonate will precipitate and may be filtered and dried.

\section{Magnesium sulfate}
\label{sec:magsulfate}
Formula: MgSO$_{4} \cdot$7H$_{2}$O\\
Other names: epsom salts\\
Description: white or clear crystals\\
Use: crystallization experiments, 
qualitative analysis test reagent 
(confirmation of hydrogen carbonate and carbonate), 
precipitation reactions\\
Source: livestock and veterinary supply shops sell Epsom salts 
to treat constipation in cattle

\section{Manganese (IV) oxide}
Formula: MnO$_{2}$\\
Other names: manganese dioxide\\
Description: black powder\\
Confirm: liberates oxygen from hydrogen peroxide\\
Use: preparation of oxygen, 
qualitative analysis (confirmation of chlorides)\\
Source: old dry cell batteries (radio batteries)\\
Extraction: smash a dry cell battery with a rock 
and scrape out the black powder. 
This is a mixture of manganese dioxide, 
zinc chloride, 
and ammonium chloride. 
This impure mixture is suitable for the preparation of oxygen. 
To purify manganese dioxide for use in qualitative analysis, 
boil the powder in water to dissolve away the chlorides. 
Filter the solution after boiling 
and repeat if the test gives false positives (e.g. 
confirms chlorides in samples that lack chlorides)\\
Note: Wash your hands with soap if you accidentally touch the powder. 
Do not get it on your clothes or into cuts on your hands. 
MnO$_{2}$ causes metal to corrode; 
if you use a metal tool to scrap out the powder, 
be sure to clean it off afterwards. 
Better: use non-metal tools. 

\section{Methane}
Formula: CH$_{4}$\\
Other names: natural gas\\
Use: optimal Bunsen burner fuel\\
Local manufacture: biogas systems --
a school could in theory build one of these to supply gas for Bunsen burners\\
Alternative: compressed gas, 
propane, 
may be purchased in most towns; 
this is generally how schools operate Bunsen burners

\section{Million's reagent}
Composition: mercury metal dissolved in nitric acid\\
Description: clear liquid, 
very low pH, 
addition of excess sodium hydroxide to a small sample 
produces a yellow precipitate (of toxic mercury hydroxide)\\
Use: identification of proteins in food tests\\
Hazard: highly toxic and very corrosive -- never use\\
Alternative: sodium hydroxide solution 
and copper sulfate solution in the Biuret test 
(1~M NaOH followed by 1\% CuSO$_{4}$)

\section{Naphthalene}
Formula: C$_{10}$H$_{8}$\\
Description: solid at room temperature but melts in boiling water, 
distinct smell of moth balls\\
Use: melting point and heat of fusion experiments\\
Source: moth balls are just solid naphthalene\\
Hazard: poison, 
possible carcinogen\\
Alternative: vaseline from small shops is 
another solid at room temperature that melts in boiling water

\section{Nestler's reagent}
Description: colorless liquid, 
sometimes with a precipitate at the bottom; 
addition of excess sodium hydroxide to a small sample 
produces a yellow precipitate (toxic mercury hydroxide)\\
Use: detection of ammonia\\
Hazard: contains dissolved mercury -- very toxic\\
Alternative: ammonia is readily detected by smell; 
a possible ammonia solution can be confirmed by adding it drop-wise 
to a solution of copper sulfate -- 
a blue precipitate should form 
which then dissolves in excess ammonia to form a deep blue / purple solution.

\section{Nitric acid}
Formula: HNO$_{3}$\\
Description: clear liquid though may turn yellow over time, 
especially if left in the light\\
Use: various experiments, 
qualitative analysis, 
cleaning stubborn residues\\
Hazard: highly corrosive acid; 
dissolves essentially everything in the laboratory except glass, 
ceramics, 
and many kinds of plastic; 
may convert organic material into explosives\\
Alternative (strong acid): battery acid\\
Alternative (qualitative analysis): 
have students practice dealing with insoluble carbonates by using copper, 
iron, 
or zinc carbonates that will dissolve in dilute sulfuric acid\\
Alternative (cleaning glassware): 
make residues in metal spoons that can be cleaned easily by abrasion

\section{Organic solvents}
Sources: kerosene, 
petrol, 
paint remover, 
paint thinner and the safest: cooking oil

\section{Oxygen}
Confirm: oxygen gas relights a glowing splint, 
i.e. 
a piece of wood or paper glowing red / orange 
will flame when put in a container 
containing much more oxygen than the typical 20\% in air\\
Preparation: combine hydrogen peroxide 
and manganese (IV) oxide in a plastic water bottle. 
Immediately crush the bottle to remove all other air and then cap the top. 
The bottle will re-inflate with oxygen gas.

\section{Phosphorus}
Use: element\\
Source: the strike pads for matches contain impure red phosphorus

\section{Potassium aluminum sulfate}
\label{sec:potalsulf}
Formula: KFe(SO$_{4}$)$_{2}$\\
Other names: potassium alum\\
Description: colorless to white crystals, 
sometimes very large, 
quite soluble in water\\
Use: coagulant useful in water treatment -- 
a small amount will precipitate all of the dirt in a bucket of dirty water\\
Source: various shops, 
especially those specializing in tradition ``Arab'' of ``Indian'' products

\section{Potassium carbonate}
Formula: K$_{2}$CO$_{3}$\\
Other names: potash\\
Description: white powder\\
Use: volumetric analysis\\
Safety: rather caustic, keep off of hands and definitely out of eyes!\\
Alternative: sodium carbonate -- 
see \nameref{cha:subchemvolana}.

\section{Potassium chromate}
Formula: K$_{2}$CrO$_{4}$\\
Description: yellow crystals soluble in water\\
Hazard: poison, 
water pollutant\\
Use: demonstration of reversible reactions, 
qualitative analysis (confirmation of lead)\\
Alternative (reversible reactions):
Dehydrate hydrated copper (II) sulfate by heating 
and then rehydrate it by adding drops of water\\
Alternative (confirmation of lead): 
Confirm lead by the addition of dilute sulfuric acid -- 
white lead sulfate precipitates

\section{Potassium dichromate}
Formula: K$_{2}$Cr$_{2}$O$_{7}$\\
Description: orange crystals soluble in water\\
Use: demonstration of chemical equilibrium, 
qualitative analysis (identification of sulfur dioxide gas)\\
Hazard: toxic, 
water pollutant\\
Alternative: make ammonium / potassium dichromate paper tests. 
Many can be made from a single gram of ammonium/potassium dichromate.

\section{Potassium hexacyanoferrate (II)}
Formula: K$_{4}$Fe(CN)$_{6}$\\
Other name: potassium ferrocyanide\\
Description: pale yellow salt\\
Use: confirmatory tests in qualitative analysis 
(forms an intensely blue precipitate with iron (III) ions, 
a red-brown precipitate with copper, 
and a blue-white precipitate with zinc\\
Alternative (confirmation of iron (III) ions): 
see possibilities listed with ammonium thiocyanate\\
Alternative (confirmation of copper): blue/green flame test, 
blue precipitate on addition of sodium hydroxide 
or sodium carbonate solution

\section{Potassium hexacyanoferrate (III)}
Formula: K$_{3}$Fe(CN)$_{6}$\\
Other name: Potassium ferricyanide\\
Description: yellow / orange salt\\
Use: confirmatory tests in qualitative analysis 
(makes an intense blue precipitate in the presence of iron (II) ions\\
Alternative: iron (II) ions will also instantly decolorize a weak, 
acidic solution of potassium manganate (VII)

\section{Potassium hydroxide}
Formula: KOH\\
Description: white crystals, 
deliquescent (poorly sealed containers may be just viscous water)\\
Use: volumetric analysis\\
Hazard: corrodes metal, 
burns skin, 
and can blind if it gets in eyes\\
Alternative: sodium hydroxide -- 
see \nameref{sec:commonsubs}.

\section{Potassium iodide}
\label{sec:potiodide}
Formula: KI\\
Description: white crystals very similar in appearance to common salt, 
endothermic heat of solvation\\
Confirm: addition of weak potassium permanganate 
or bleach solution causes a clear KI solution to turn yellow/brown 
due to the formation of I$_{2}$ (which then reacts with I$^{-}$ to form soluble I$^{-}_{3}$)\\
Use: preparation of iodine solution for food tests in biology, 
preparation of iodine solutions for redox titrations, 
confirmatory test for lead in qualitative analysis\\
Local manufacture: Heat a pharmacy iodine tincture strongly until 
only clear crystals remain. 
In this process, 
the I$_{2}$ will sublimate -- 
placing a cold dish above the iodine solution should cause must of the iodine 
to deposit as solid purple crystals. 
Note that the iodine vapors are harmful to inhale.
If you need KI for a solution that may contain impurities, 
add ascorbic acid solution to dilute iodine tincture 
until the solution exactly decolorized.\\
Alternative (food tests): see \nameref{sec:iodinesol}\\
Alternative (redox titrations): 
often you can also use iodine solution for this; 
just calibrate the dilution of pharmacy tincture 
and the other reagents to create a useful titration\\
Alternative (qualitative analysis): 
confirm lead by the addition of dilute sulfuric acid -- 
white lead sulfate precipitates

\section{Potassium manganate (VII)}
Formula: KMnO$_{4}$\\
Other names: potassium permanganate, 
permanganate\\
Description: purple/black crystals, 
sometimes with a yellow/brown glint, 
very soluble in water -- 
a few crystals will create a strongly purple colored solution\\
Hazard: powerful oxidizing agent -- 
may react violently with various compounds; 
solutions stain clothing (remove stains with ascorbic acid solution); 
crystals and concentrated solution discolor skin 
(the effect subsides after a few hours, 
but it is better to not touch the chemical!)\\
Use: strong oxidizer, 
self-indicating redox titrations, 
identification of various unknown compounds, 
diffusion experiments\\
Source: imported ``local'' medicine. 
Also sold in very small quantities in many pharmacies. 
May be available in larger quantities from hospitals.\\
Alternative (oxidizer): bleach (sodium hypochlorite), 
hydrogen peroxide\\
Alternative (diffusion experiments): solid or liquid food coloring, 
available in markets and small shops

\section{Potassium thiocyanate}
Formula: KSCN\\
Use: confirmation of iron (III) ions in qualitative analysis\\
Alternative: addition of sodium ethanoate 
should also produce a blood red solution; 
additionally, 
the test is unnecessary, 
as iron (III) ions is also the only chemical 
that will produce a red/brown precipitate 
with sodium hydroxide solution or sodium carbonate solution

\section{Propanone}
\label{sec:propanone}
Formula: H$_{3}$CCOCH$_{3}$\\
Other names: acetone\\
Description: clear liquid miscible in water, 
smells like nail polish remover, 
quickly evaporates\\
Use: all-purpose lab solvent, 
iodoform reaction (kinetics, organic chemistry)\\
Hazard: highly flammable\\
Source: nail polish remover (mixture with ethyl ethanoate)\\
Alternative (volatile polar solvent): ethanol, 
including methylated spirits

\section{Silicon}
Use: element\\
Source: fragments of broken solar panels; 
the cells are in part doped silicon

\section{Silicon dioxide}
Description: clear solid\\
Source: quartz rock, 
quartz sand, 
glass

\section{Silver nitrate}
Formula: AgNO$_{3}$\\
Description: white crystals, 
turn black if exposed to light (hence, 
the use of silver halides in photography)\\
Confirm: silvery-white precipitate formed with chlorides\\
Use: confirmatory test for chlorides in qualitative analysis\\
Hazard: poison, 
water pollutant\\
Alternative: heat sample together 
with a dilute solution of acidified potassium manganate (VII) -- 
decolorization confirms chlorides -- see \nameref{cha:qualana}

\section{Sodium}
Description: very soft metal (cuts with a knife) 
with a silvery color usually obscured by a dull oxide; 
always stored under oil\\
Use: demonstration of reactive metals (add to water)\\
Hazard: reacts with air and violently with water. 
May cause fire.

\section{Sodium acetate}
See \nameref{sec:sodiumeth}.

\section{Sodium carbonate}
Formula: Na$_{2}$CO$_{3} \cdot$10H$_{2}$O (hydrated), 
Na$_{2}$CO$_{3}$ (anhydrous)\\
Other names: soda ash, washing soda\\
Description: white powder completely soluble in water\\
Use: all-purpose cheap base, 
volumetric analysis, 
qualitative analysis, 
manufacture of other carbonates\\
Safety: rather caustic, keep off of hands and definitely out of eyes!\\
Source: commercial and industrial chemical supply -- 
should be very inexpensive\\
Local manufacture: dissolve sodium hydrogen carbonate in distilled water 
and boil for five or ten minutes 
to convert the hydrogen carbonate to carbonate. 
Let evaporate until crystals form. 
For volumetric analysis, 
the hydrated salt may always substitute 
for the anhydrous with a correction to the concentration -- 
see Chemical Substitutions for Volumetric Analysis

\section{Sodium chloride}
Formula: NaCl\\
Other names: common salt\\
Use: all-purpose cheap salt, 
qualitative analysis\\
Source: the highest quality salt in markets (white, 
finely powdered) is best. 
The iodine salts added to prevent goiter 
do not generally affect experimental results.

\section{Sodium citrate}
Use: buffer solutions, 
preparation of Benedict's solution\\
Local manufacture: react sodium hydroxide 
and citric acid in a 3:1 ratio by mole\\
Alternative: to prepare Benedict's solution, 
see \nameref{sec:benedict}.

\section{Sodium ethanoate}
\label{sec:sodiumeth}
Formula: CH$_{3}$CHOONa\\
Other names: sodium acetate\\
Use: confirmation of iron (III) ions\\
Local manufacture: react sodium hydrogen carbonate 
and ethanoic acid in a 1:1 ratio by mole -- 
one 70~g box of baking soda to one liter of white vinegar labelled 5\%; 
if you need to err add excess sodium hydrogen carbonate. 
If the solid is required, 
leave to evaporate, 
but mostly likely you want the solution.

\section{Sodium hydrogen carbonate}
Formula: NaHCO$_{3}$\\
Description: white powder, 
in theory completely soluble in cold water 
in practice often dissolves poorly\\
Other names: sodium bicarbonate, 
bicarbonate of soda\\
Use: all-purpose weak base, 
preparation of carbon dioxide, 
qualitative analysis\\
Source: small shops \\
Note: may contain ammonium hydrogen carbonate

\section{Sodium hydroxide}
Formula: NaOH\\
Other names: caustic soda\\
Description: white deliquescent crystals -- 
will look wet after a minute in contact with air 
and will fully dissolve after some time, 
depending on humidity and particle size\\ 
Use: all-purpose strong base, 
volumetric analysis, 
food tests in biology, 
qualitative analysis, 
preparation of sodium salts of weak acids\\
Hazard: corrodes metal, 
burns skin, 
and can blind if it gets in eyes\\
Source: industrial supply shops, 
supermarkets, 
hardware stores (drain cleaner)\\
Local manufacture: mix wood ashes in water, 
let settle, 
and decant; 
the resulting solution is mixed sodium and potassium hydroxides 
and carbonates and will work for practicing volumetric analysis\\
Note: ash extracts are about 0.1~M base and may be concentrated by boiling; 
this is dangerous, 
however, 
and industrial caustic soda is so inexpensive 
and so pure that there is little reason to use ash extract 
other than to show that ashes are alkaline 
and that sodium hydroxide is not exotic.

\section{Sodium hypochlorite solution}
Formula: NaOCl$_{(aq)}$\\
Other names: bleach\\
Use: oxidizing agent\\
Source: small shops, 
supermarkets\\
Local manufacture: electrolysis of concentrated salt water solution 
with inert (e.g. 
graphite) electrodes; 
4-5~V (three regular batteries) is best for maximum yield\\
Note: commercial bleach is usually 3.5\% sodium hypochlorite by weight

\section{Sodium nitrate}
Formula: NaNO$_{3}$\\
Description: colorless crystals\\
Use: qualitative analysis\\
Hazard: oxidizer, 
used in the manufacture of explosives e.g. 
gunpowder\\
Alternative: to practice identification of the sodium cation, 
use sodium chloride\\
Local manufacture: Mix solutions of calcium ammonium nitrate 
and sodium carbonate and decant the clear solution 
once the precipitate (calcium carbonate) settles. 
Add a stoichiometric quantity of sodium hydroxide 
and let the reaction happen either outside 
or with under a condenser to trap the ammonia produced. 
The clear solution that remains should have no residual ammonia smell 
and should be neutral pH. 
Allow the solution to evaporate until sodium nitrate crystallizes.

\section{Sodium oxalate}
Formula: Na$_{2}$C$_{2}$O$_{4}$\\
Use: demonstration of buffer solutions\\
Hazard: poisonous\\
Alternative: rather than oxalic acid / sodium oxalate, 
use citric acid / sodium citrate

\section{Sodium sulfate}
Formula: Na$_{2}$SO$_{4}$\\
Use: qualitative analysis\\
Local manufacture: combine precisely stoichiometric amounts 
of copper sulfate and sodium carbonate in distilled water. 
A balance is required to measure exactly the right amounts. 
Copper carbonate will precipitate and the resulting solution 
should contain only sodium sulfate. 
Filter out the copper carbonate and evaporate the clear solution to dryness. 
Sodium sulfate is thermally stable, 
so strong heating may be used to speed up evaporation.

\section{Sodium thiosulfate}
Formula: Na$_{2}$S$_{2}$O$_{3} \cdot$5H$_{2}$O\\
Description: clear, 
hexagonal crystals\\
Use: reducing agent for redox titrations, 
sulfur precipitation kinetics experiments\\
Alternative (reducing agent): ascorbic acid\\
Alternative (kinetics): reaction between sodium hydrogen carbonate solution 
and dilute weak acid (citric acid or ethanoic acid), 
iodoform reaction (iodine solution and propanone)\\

\section{Succinic acid}
Formula: HOOCCH$_{2}$CH$_{2}$COOH\\
Description: white solid\\
Use: solute for partitioning in distribution (partition) experiments\\
Alternative: iodine also partitions well between aqueous and organic solvents; 
titrate iodine with ascorbic acid (or sodium thiosulfate) 
rather than sodium hydroxide as you would with succinic acid; 
ethanoic acid also partitions between some solvent combinations.

\section{Sucrose}
Formula: C$_{12}$H$_{22}$O$_{11}$\\
Use: non-reducing sugar for food tests\\
Source: common sugar; 
the brown granular sugar at the market and in small shops is more common; 
the more refined white sugar is available in supermarkets\\
Note: sometimes impure sucrose causes Benedict's solution to turn green, 
even yellow. Try using more refined sugar.
Alternatively, insist to students than only a red/orange precipitate 
is a positive test for a reducing sugar during exams.

\section{Sudan III solution}
Use: testing for fats in food tests\\
Alternative: ethanol-free iodine solution

\section{Sulfur}
Description: light yellow powder with distinct sulfurous smell\\
Use: element, 
preparation of iron sulfide\\
Source: large agricultural shops (fungicide, 
e.g. 
for dusting crops), 
imported ``local'' medicine

\section{Sulfuric acid}
Formula: H$_{2}$SO$_{4}$\\
Other names: battery acid\\
Description: clear liquid with increasing viscosity at higher concentrations; 
fully concentrated sulfuric acid ($\sim$18~M) is almost twice as dense as water 
and may take on a yellow, 
brown, 
or even black color from contamination\\
Use: all-purpose strong acid, 
volumetric analysis, 
qualitative analysis, 
preparation of hydrogen and various salts\\
Source: battery acid from petrol stations 
is about 4.5~M sulfuric acid and one of the least expensive sources of acid\\
Hazard: battery acid is dangerous; 
it will blind if it gets in eyes and will put holes in clothing. 
Fully concentrated sulfuric acid is monstrous, 
but fortunately never required. 
For qualitative analysis, 
``concentrated'' sulfuric acid means $\sim$5~M -- battery acid will suffice.\\
Note: ``dilute'' sulfuric acid should be about 1~M. 
To prepare this from battery acid, 
add one volume of battery acid to four volumes of water (e.g. 
100~mL battery acid + 400~mL water)

\section{Starch}
Description: light weight, 
fine, 
white powder, 
not readily soluble in cold water\\
Confirm: makes a blue to black color with iodine solution\\
Use: preparation of starch solution\\
Source: supermarkets

\section{Starch solution}
Use: sample for food tests, 
indicator for redox titrations involving iodine\\
Source: dilute the water left from boiling pasta or potatoes\\
Note: prepare freshly -- after a day or two it will start to rot!

\section{Tetrachloromethane}
\label{sec:tetrachloromethane}
Formula: CCl$_{4}$\\
Other names: carbon tetrachloride\\
Description: clean liquid, 
insoluble in and more dense than water\\
Use: organic solvent for distribution (partition) experiments\\
Hazard: toxic, 
probably carcinogen -- never use\\
Alternative: other organic solvents -- 
paint thinner and kerosene are the least expensive

\section{Trichloromethane}
\label{sec:trichloromethane}
Foruma: CHCl$_{3}$\\
Other names: chloroform\\
Description: clear liquid, 
insoluble in and more dense than water, 
noxious smell\\
Use: rendering biological specimens unconscious prior to dissection, 
as an organic solvent for the distribution (partition) experiments\\
Alternative (biology): the specimen will die regardless 
so unless you are investigating the circulatory system 
you might as well kill it in advance; 
this also avoids the problem of specimens regaining consciousness 
before they bleed to death. 
See instructions in Dissections.\\
Alternative (chemistry): lower cost and safer organic solvents like kerosene can be used to practice distribution (partitioning), 
but unlike chloroform they are less dense than water.

\section{Tungsten}
Symbol: W\\
Use: element\\
Source: incandescent light bulb filaments\\
Extraction: wrap a light bulb in a rag and break it with a blunt object. 
The filament is the thin coiled wire. 
Dispose of the broken glass in a safe place, 
like a pit latrine.\\
Note: in a dead bulb, the cause of failure is probably the filament, 
so there might not be much left.

\section{Zinc}
\label{sec:zinc}
Description: firm silvery metal, 
usually coated with a dull oxide\\
Use: element, 
preparation of hydrogen, 
preparation of zinc carbonate and zinc sulfate\\
Source: dry cell batteries; 
under the outer steel shell is an inner cylinder of zinc. 
In new batteries, 
this whole shell may be extracted. 
In used batteries, 
the battery has consumed most of the zinc during the reaction, 
but there is generally an unused ring of zinc around the top 
that easily breaks off. 
Note that alkaline batteries, 
unlike dry cells, 
are unsafe to open -- and much more difficult besides.

\section{Zinc carbonate}
Formula: ZnCO$_{3}$\\
Description: white powder\\
Use: qualitative analysis\\
Local manufacture: dissolve excess zinc metal 
in dilute sulfuric acid and leave overnight 
or until the acid is completely consumed. 
Decant the resulting zinc sulfate solution and 
mix with a sodium carbonate solution. 
Zinc carbonate will precipitate 
and may be purified by filtration and gentle drying.

\section{Zinc chloride and zinc nitrate}
Description: clear, 
deliquescent crystals\\
Use: qualitative analysis\\
Alternative: to practice identification of zinc, 
use zinc sulfate or zinc carbonate; 
to practice identification of chloride use sodium chloride

\section{Zinc sulfate}
Formula: ZnSO$_{4}$\\
Use: qualitative analysis\\
Local manufacture: dissolve excess zinc metal in dilute sulfuric acid 
and leave overnight or until the acid is completely consumed. 
Decant the resulting zinc sulfate solution and evaporate until crystals form.

\chapter{Improving an Existing School Laboratory}

If there is already a laboratory at your school, 
the immediate tasks are to see what it has, 
make it safe, 
get it organized, 
make repairs, 
and ensure smart use with sound management.

\section{Inventory}

Making a list of what and how much of everything is in your lab is easy, 
if time consuming. 
Difficulties arise when you find apparatus you have never seen before, 
or containers of chemicals without labels.

There is no harm in unknown apparatus, 
they just are not useful until you know what they do. 
Ask around.

Unknown chemicals, 
however, 
pose a hazard, 
because it is unclear how to properly store them or how to clean up spills. 
If a chemical is unknown, 
there is no safe way to responsibly dispose of it. 
Therefore, 
it is best to attempt to identify unknown chemicals. 
For assistance in identifying unknown chemicals, 
please see the Guide to Identifying Unknown Chemicals.

Burettes and apparatus concerning electricity, 
for example voltmeters and ammeters, 
should be testing to ensure that they work. 
Please consult Volumetric Analysis Technique 
to learn how to use burettes 
and Testing Voltmeters and Ammeters to do just that.

\section{Organize}

\subsection{Have enough space}
The key to organization is having enough space. 
Usually, 
this means building shelves. 
In the long term, 
find a carpenter to build good shelves. 
In the short term, 
boards and bricks, 
scrap materials, 
chairs, 
anything to provide sturdy and horizontal storage space. 
It should be possible to read the label of every chemical, 
and to see each piece of equipment

\subsection{Apparatus}
\begin{itemize}
\item{Arrange apparatus neatly so it is easy to find each piece.}
\item{Put similar things together.}
\item{Beakers can be nested like Russian dolls.}
\end{itemize}

\subsection{Chemicals}
\begin{itemize}
\item{Organize chemicals alphabetically. 
There are more complicated schemes involving the function 
or the properties of the chemical but what is most important 
is a scheme that everyone working in the lab can follow. 
ABC is the easiest, 
and has the best chance of being used.}
\item{Glass bottles of liquid chemicals should be kept on the floor, 
unless the laboratory is prone to flooding, 
in which case they should be on a sufficiently elevated, 
broad and stable surface. 
What you do not want are these bottles falling and breaking open.}
\item{Million's Reagent, 
benzene, 
and other chemicals that should never be used should be kept in a special place, 
ideally locked away, 
and labeled to prevent use. 
See Dangerous Chemicals for a list of chemicals that should never be used.}
\item{Label plastic containers directly with a permanent pen, 
especially if the printed label is starting to come off.} 
\item{Replace broken or cracked containers with new ones.}
\end{itemize}

\subsection{Make a map and ledger}
Once you have labeled and organized everything in a lab, 
draw a map. 
Sketch the layout of your laboratory 
and label the benches and shelves. 
In a ledger or notebook, 
write down what you have and the quantity. 
For example, 
Bench 6 contains 20 test tubes, 
3 test tube holders, 
and 4 aluminum pots. 
This way, 
when you need something specific, 
you can find it easily. 
Further, 
this helps other teachers – especially new ones – better use the lab. 
Finally, 
having a continuously updated inventory will let you know what 
materials need to be replaced or are in short supply. 
Proper inventories are a critical part of maintaining a laboratory, 
and they really simplify things around exam time.

\section{Repair / Improve}

Once the lab is organized, 
it is easy to find small improvements. 
Here are some ideas:

\subsection{Build more shelves}
You really cannot have too many.

\subsection{Fix broken burettes}
Burettes are useful, 
expensive and – if glass – fragile. 
Broken burettes can often be made functional again. 
If you have broken burettes, 
see the Repairing Burettes in the appendix.

\subsection{Identify key apparatus needs}
Sometimes a few pieces of apparatus can be very enabling, 
like enough measuring cylinders, 
for example. 
Buy plastic!

\section{What next?}

Once the lab is safe and organized, 
develop a system for keeping it that way. 
Consider the advice in Routine Cleanup and Upkeep. 
Make sure students and other teachers in involved.

Then, 
start using the lab! Every class can be a lab class. 
That is the whole point.

\chapter{Checking Voltmeters and Ammeters/Galvanometers}
\label{cha:voltamm}
Needed: Meters to check, 
a couple wires, 
some resistors and a fresh battery.

Important note: There is a wrong way to hook up the meter. 
The needle will try to deflect down 
because negative and positive are swapped. 
If the reading is zero, 
make sure that you try the opposite connection to be sure.

\section{Voltmeters}
Hook up the voltmeter across the battery. 
The battery is probably 1.5 V, 
but do not worry if you see 1.1, 
1.2, 
even if using a brand new battery. 
Try not to use a battery that reads much below 1 V 
on several different meters.

\subsection{Unuseable Voltmeters}
\begin{itemize}
\item{Totally dead, no deflection of the needle}
\item{Voltage reading jumps excessively. 
Ensure that the connections are solid and test again.}
\item{Measured voltage is totally wrong, not close to 1.5 V}
\end{itemize}

\subsection{Useable Voltmeters}:
Read a voltage close to 1.5. 
If the voltage if not 1.5 exactly, 
the voltmeter is probably working fine, 
and the battery is just off a bit.

\section{Ammeters}
Hook up the ammeter in series with a resistor. 
Because you do not necessarily know the condition of the ammeter before testing, 
be sure to have several different resistors on hand. 
An ammeter may appear not to work if resistance is too high or too low. 
Start testing different ammeters.

\subsection{Unuseable Ammeters} 	
\begin{itemize}
\item{Totally dead, 
no deflection of the needle}
\item{Current reading jumps excessively (but check connections)}
\item{Totally wrong, 
reads much different from other ammeters}
\end{itemize}

\subsection{Useable Ammeters}
Read a current similar to other ammeters. 
Hard to say exactly what current, 
but feel free to calculate based on your resistor using $ V=IR $, 
although do not forget that there is 
some internal resistance r of battery, 
so $ V=I(R+r) $. 
The resistance of the resistor is usually coded 
on the resistor in a series in stripes -- 
see the instructions under \nameref{sec:resistors} in \nameref{cha:labequip}.

Tip: You can hold the wires onto the battery with your fingers; 
the current is far too low to shock you.

Other: Now that you have tested to see 
if your voltmeters and ammeters work, 
you can feel free to check all of them for accuracy, 
by calculating expected values and comparing between meters. 
Most practicals will still work alright with ``somewhat'' accurate meters, 
and most meters are either fine, 
or broken.

\chapter{Repairing Burettes}
\label{cha:burettes}

First, if you need burettes, consider buying plastic burettes. They are widely available if you ask persistently and they tend not to break. This may be hard as many suppliers prefer to sell glass burettes. Why? As one supplier told us, ''Because when people buy plastic burettes, they don’t return.''

The good news for every school with glass burettes is than often broken burettes can be repaired.

\section{The top of the burette is broken, 
above the 0~mL line.}

This burette is still fully functional. 
A student will probably need a beaker for filling the burette, 
but she should be using one anyway. 
Use a metal file (best!), 
stone, 
or piece of cement to gently grind the broken edge smooth to prevent cuts.

\section{The burette is broken in the graduated section, 
that is, 
between 0~ml and 50~ml.}
This burette is still slightly useful for titrations 
if it has most of its length. 
Students will just have an initial volume of 7~ml, 
perhaps. 
If it has broken around the 45~ml mark, 
no such luck. 
The burette tube however, 
is still quite useful as a glass pipe. 
Keep it around for other kinds of experiments. 
At the very least you have a glass rod for mixing solutions. 
Regardless, 
grind the edges smooth as in case one.

\section{The burette is broken below the 50~ml but above the valve.}
To fix this, 
you need a Biafa (fake Bic) pen and about 8cm of rubber tubing. 
Orange gas supply tubing is best, 
but hard to find. 
The black rubber of the inside of bicycle pump hoses also works. 
Large bike supply shops often have broken pumps 
with which they are willing to part for free. 
First, 
cut off the tip of the pen, 
the first 2~cm of so, 
and attach the non-tapered end it to the tubing. 
Cutting is easiest done by scoring all the way around 
with a razor blade and then cleanly snapping the shaft. 
Remove any plastic burrs from the cut edge 
and then insert the wider end of the severed tip 
into the plastic tubing so the narrow end hangs out. 
Second, 
remove from the pen the little plastic end cap 
(the one that tells you what color ink you have) 
and insert it into the tubing, 
curved side first. 
Push it about half way down the tube using your 
fingers like esophageal peristalsis and make sure that 
the axis of symmetry of the pen cap stays aligned 
with that of the rubber tubing. 
That is, 
if the now discarded pen were still there, 
it would be surrounded by the tube. 
Finally, 
attach the other end of the tubing to the broken burette. 
Again, 
grind the sharp glass end to smooth it. 
What you should end up with is a burette that does not pass solution 
except when you press on the tubing around the pen end cap, 
deforming the tube to allow liquid to pass. 
With practice this can be easier than using a valve, 
and just as accurate.

Steel ball bearings are available for cheap at bicycle supply shops. 
These might be an alternative to the end caps of Biafa pens 
if you can get them in the right size. 
Experiment!

\section{The valve is jammed}
No problem! Soak it in dilute acid (not nitric) until it is free.

\section{Case Five: The valve is hopelessly broken.}
Break the burette just above the valve and follow the instructions above. 
Soak a string in something flammable -- kerosene, 
nail polish remover -- and gently squeeze out the excess. 
Tie the string around the shaft where 
you want to ''cut'' the glass and remove the excess string. 
Dry up any liquid that spilled on other parts of the glass. 
Light the string on fire and rotate to make sure it burns evenly. 
After five or so seconds of burning, 
plunge the piece into a beaker or bucket of water. 
The contraction of the rapidly cooling glass 
should break the burette along where you tied the string. 
Grind the edge to smooth it.

\section{The burette is broken below the valve.}
This problem is mostly aesthetic, 
but to fix it you only need about 3~cm of rubber tubing 
and a clear plastic pen. 
Cut the tip from the pen as above and insert it into the tubing. 
Then stick the other end of the tubing onto the broken burette, 
grinding down the glass edge before you do.

\section{The rubber tubing is cracking.}
This usually comes from leaving clamps on the tubing during storage. 
To fix this, 
replace the rubber tubing. 
But while you are at it, 
insert a pen cap as in case three and do away with the clamps. 
They are more difficult to use and not as sensitive.

\chapter{Identifying Unknown Chemicals}
\label{cha:unknownchemicals}
Unlabelled chemicals are dangerous. 
If you do not know what the chemical is, 
then you do not know what to do if it spills, 
or how to safely get it out of your school.

%==============================================================================
\section{Identifying Bottles of Unknown Liquids}

Usually, 
these are:
Concentrated acids (sulfuric, 
hydrochloric, 
nitric, 
ethanoic)
Concentrated ammonia solution
Organic solvents including methanol, 
ethanol, 
isobutanol, 
propanone (acetone), 
diethyl ether, 
ethyl ethanoate (ethyl acetate), 
dichloromethane, 
trichloromethane (chloroform), 
tetrachloromethane (carbon tetrachloride), 
trichloroethene, 
benzene, 
chlorobenzene, 
toluene, 
xylene, 
and petroleum spirits

Distinguishing these chemicals is important, 
and relatively possible. 
Here is a procedure:

First, 
protect yourself against whatever it might be. 
Concentrated acids burn skin on contact and blind if they get in the eyes. 
Concentrated hydrochloric acid and concentrated ammonia solution 
release fumes that corrode the throat and lungs. 
Diethyl ether and propanone rapidly evaporate at room temperature 
and pose a significant flash fire hazard if opened near flame. 
Ingesting even a small amount of toxic carbon tetrachloride can be fatal, 
and benzene is a proven and serious carcinogen.

Why, 
you might say, 
should I even attempt this? Because sooner or later, 
someone will, 
and better it be someone with these instructions than without. 
But if you do not feel comfortable, 
call a friend who is more excited about this process.

Many precautions are available. 
Tie a cloth over your mouth and nose to mitigate inhalation. 
Find a pair of goggles or sunglasses to protect you eyes 
from any splash when opening the stopper in the bottle. 
Wear gloves or at least plastic bags on your hands. 
Neither will protect your hands for more than a second 
or a few against concentrated acids or some organics, 
but that second can be useful in this case. 
Thick rubber gloves are available (see \nameref{cha:labequip}) 
and offer greater protection. 
Regardless, 
have at the ready a bucket of water and a box of baking soda 
(bicarbonate of soda) to neutralize acid burns. 
Move the container outside and remain upwind. 
Have a small, 
dry, 
clean beaker ready to hold a sample.

Open the bottle. 
This may be as simple as unscrewing the top 
or there may be an internal stopper that requires prying off. 
Find a suitable tool, 
one that can pry under the cap but cut neither the cap nor you. 
A butter knife works well. 
Do not use your fingers.

When the bottle opens, 
look at the top. 
Are there white fumes? 
Is there an obvious smell that you can perceive 
from where you are standing? 
White fumes suggest hydrochloric acid 
and an intense smell could be ammonia (smells like stale urine), 
hydrochloric or ethanoic acid (both smell like vinegar), 
or an organic solvent (various odors).

If the contents smell obviously like ammonia, 
there is no need to further experimentation. 
Nothing else in schools smells even remotely like ammonia. 
Stopper that bottle and give it a good label.

Otherwise, 
carefully, 
pour a few cubic centimeters 
of the liquid into your sample beaker. 
As you pour the liquid, 
observe the viscosity. 
Concentrated acids are all noticeably more viscous than water, 
especially concentrated sulfuric acid. 
Propanone, 
on the other hand, 
is noticeably more fluid than water. 
Close the bottle and take the beaker 
to a safe place for experimentation.

Color is surprisingly useless in identifying unknown liquids 
because most readily take on color 
from even small amounts of contamination.

Rest the beaker on a sturdy surface. 
If you have already noticed an intense smell, 
leave the cloth on your face. 
If you have not yet noticed a smell, 
remove it.

%==============================================================================
\section{Test one: Add to water}

Fill a large, 
clean test tube half way with ordinary water. 
Alternatively, 
find the smallest beaker you have (probably 50~mL), 
and fill it about a quarter of the way with water. 
Carefully pour in a few drops of your unknown 
and observe what happens. 


If it does not mix with the water, 
instead forming a new (possibly quite small) layer on top, 
you have an organic solvent less dense than water, 
probably one of: isobutanol, 
diethyl ether, 
ethyl ethanoate, 
benzene, 
chlorobenzene, 
toluene, 
xylene, 
or petroleum spirits. 
If it does not mix with the water, 
instead sinking to form a distinct layer on the bottom, 
you have an organic solvent more dense than water, 
probably dichloromethane, 
chloroform, 
or carbon tetrachloride.

If your unknown does not mix with water, 
jump down to \nameref{sec:testorganic} on what to do with organics.

If the unknown seems to sink into the water but not mix completely, 
you probably have a concentrated acid. 
The test tube might even get a little warmer. 
You might also have a very concentrated solution of some other solute, 
left over from a previous experiment.

If the unknown seems to mix into the water like, 
well, 
water, 
you probably have an aqueous solution that is not very concentrated. 
It might be dilute acid, 
dilute hydroxide, 
hydrogen peroxide solution, 
etc. -- more work lies ahead.

%==============================================================================
\section{Test two: Is it an acid?}

This only applies to solutions that mix completely into water.

This is easy with a piece of blue litmus paper. 
Dip a corner down into the test tube or beaker. 
If it turns bright red, 
you probably have an acid, 
and if your liquid was noticeably viscous, 
a concentrated acid. 
If there is no change, 
move on to \nameref{sec:whatelse}.

Another option is universal indicator or universal pH paper. 
Prepare a 100-fold dilution of the original acid 
and test with the indicator. 
If the color is bright red, 
you must have a strong acid, 
like hydrochloric, 
sulfuric, 
or nitric acid. 
If the color is instead orange or yellow, 
you must have a weak acid, 
like ethanoic acid. 
If there is no universal indicator, 
you can show that something definitely is an acid 
if it causes methyl orange to turn from orange to red. 
However, 
if there is no color change, 
you might still have a weak acid, 
so you cannot use methyl orange to eliminate the possibility of an acid. 
You also cannot use POP to show that there is an acid, 
as both concentrated acid and tap water have the same effect on POP: 
none whatsoever.

If you do not have any litmus paper or other indicator, 
find another beaker and add 10-20~mL of ordinary water 
and dissolve a bit of baking soda (bicarbonate of soda). 
Carefully, 
with eye protection, 
add a few drops of your DILUTED unknown (from test one). 
If there are bubbles, 
you have an acid. 
Adding a concentrated acid directly to baking powder 
can cause such vigorous effervescence as to eject acid from the test tube.

%==============================================================================
\section{Test three: What kind of acid?}

%------------------------------------------------------------------------------
\subsection{Sulfuric acid}

\begin{itemize}

\item{Hints: obviously viscous, 
significantly denser than water, 
noticeable heat released on dilution, 
no smell}

\item{Confirmatory test: dip the wooden end of a match into the original solution. 
If the end appears to char, 
you have concentrated sulfuric acid. 
Another variant of this test is take some concentrated sulfuric acid 
and pour over some sugar in a beaker. 
After some time, 
a black color from carbon produced 
from the dehydration of sugar confirms sulfuric acid. 
Yes, 
the same thing happens to skin. 
The downside of this second test is that the beaker 
is almost impossible to clean.}

\item{Alternative test: Find or prepare a 0.1~M barium nitrate, 
barium chloride, 
or lead nitrate solution. 
In a test tube, 
add about one centimeter of your diluted sample 
and then a few drops of one of the above solutions. 
An instant, 
white, 
cloudy precipitate demonstrates that sulfate is present. 
To confirm that this is from sulfuric acid and not, 
say, 
your tap water, 
test in the same way the water you used for the dilution. 
Not much should happen. 
If your tap water contains sulfates, 
find some distilled (e.g. 
rain) water and remake the dilution.}
\end{itemize}

%------------------------------------------------------------------------------
\subsection{Hydrochloric acid}

\begin{itemize}

\item{Hints: white fumes, 
intense acidic smell similar to vinegar, 
more dense than water}

\item{Confirmatory test: 
prepare a dilute potassium permanganate solution. 
This should be pink in color, 
which might require significant dilution. 
Fill a test tube with a couple centimeters of your dilute solution 
and add the potassium permanganate solution drop wise. 
If the pink color is rendered colorless 
after mixing with your diluted sample, 
you probably have hydrochloric acid. 
This reaction makes small amounts of chlorine gas, 
but that poses much less risk than the hydrochloric acid fumes.}

\item{Alternative test: 
Find or prepare a 0.05~M or 0.1~M silver nitrate solution. 
Remember that this chemical is very expensive, 
so only make a small quantity. 
In a test tube, 
add about one centimeter of the water you used 
for diluting your sample 
and then a few drops of one of silver nitrate solution. 
An instant, 
white to gray, 
cloudy precipitate demonstrates that chloride is present. 
If this happens, 
your tap water contains chlorine 
and you will have to prepare another dilution 
using rain or distilled water. 
If the water you used for dilution lacks chlorine, 
add a centimeter of the diluted sample 
to a clean test tube and add a few drops of silver nitrate solution. 
The precipitate confirms that you have hydrochloric acid. 
Note that for this test to be effective, 
the hydrochloric acid must be diluted. 
Concentrated hydrochloride acid reacts with aqueous silver 
to form the [AgCl$_{2}$]-complex, 
which is soluble.}

\end{itemize}

%------------------------------------------------------------------------------
\subsection{Ethanoic (acetic) acid}

\begin{itemize}

\item{Confirmatory Test: This acid smells strongly of vinegar. 
If you have a definite vingar smell, 
it is probably ethanoic acid, 
but beware that concentrated HCl can have a similar smell. 
To confirm ethanoic acid, 
use some diluted acid from test one 
and add a small amount of baking soda until it is just neutralized. 
Do not add excess baking soda - neutralization is the goal. 
After neutralizing, 
add a small amount of iron (III) chloride or nitrate. 
A blood red solution of iron (III) acetate 
proves that the acid is ethanoic. 
Boiling the solution should form a red brown precipitate. 
If you do not have iron (III) salts but do have universal indicator, 
use the indicator method above 
for confirming that your unknown is a weak acid -- 
ethanoic is the only common weak acid that smells like vinegar.} 
\end{itemize}

%------------------------------------------------------------------------------
\subsection{Nitric acid}
A concentrated acid in a school that does not smell like vinegar 
and is not hydrochloric or sulfuric acid is very likely to be nitric acid. 

\begin{itemize}

\item{Confirmatory Tests: Take a wooden splinter 
or match stick and dip it in the concentrated acid. 
If the splinter turns yellow, 
the acid is nitric. 
A second confirmatory test is adding copper wire 
or turnings to the concentrated acid. 
A brown gas of nitrogen dioxide is formed. 
Do this confirmatory test in a well ventilated area.}

\item{Special note: if you suspect nitric acid, 
dip a piece of copper wire into the solution. 
If it comes back with a silvery coating, 
you have Million’s Reagent, 
mercury metal dissolved in nitric acid. 
This is highly toxic, 
very dangerous, 
and should never be used in a school. 
Label the bottle “Million’s Reagent, 
Contains Hg$_{2}^{+}$, 
TOXIC, 
CORROSIVE, 
do not use, 
do not dump” along with similar warnings 
in any local language(s) and find a safe place to store it.}

\end{itemize}

%==============================================================================
\section{Test four: What kind of organic?}
\label{sec:testorganic}
Let us be honest. 
Distinguishing between different kinds of organic solvents 
is hard with the resources that are probably available. 
If the chemical is more dense than water 
and no one at the school claims that it is chloroform 
(trichloromethane) for the biology lab, 
there is no way to show that it is not carbon tetrachloride 
(tetrachloromethane), 
a toxic organic solvent responsible 
for the death students in several countries. 
Label the bottle ``Unknown organic solvent more dense than water, 
possibly carbon tetrachloride, 
TOXIC, 
never use, 
never dump,'' with similar warnings in any local language(s) 
and find a safe place to store it.

If the chemical is less dense than water 
and you are familiar with organic solvents, 
you might try a careful smell test.

If the unknown smells like strong booze and is soluble in water, 
it is probably ethanol or methanol. 
Do not drink it! -- methanol blinds. 
If it is bright red, 
it is probably Sudan III solution, 
for biology. 
Label and use it. 
If it is yellow or brown it might be iodine solution, 
see below in test five. 
If it is light purple or green or whatever 
the popular color in your country, 
it is probably methylated spirits, 
a mixture of about 70\% ethanol and 30\% water
with some impurities to make it undrinkable. 
Confirm this by showing that paper soaked in the chemical 
will burn with a blue flame but that paper soaked in a 50/50 
mixture of the chemical and water will not burn. 
If it is clear and someone at the school can assure 
that the contents are ethanol and not methanol, 
label the bottle ``ethanol'' and use it. 
If the bottle might be methanol, 
a poison, 
pour the contents into a large bucket 
and leave it in a place where no one will disturb it 
and where the fumes will not accumulate. 
Let it evaporate.

If the unknown smells like nail polish remover 
and is soluble in water, 
it is probably propanone (acetone). 
If you put a drop in a spoon it should evaporate relatively quickly. 
Label it ``Propanone, 
EXTREMELY FLAMMABLE'' and keep it around. 
If it is not soluble in water and smells like magic markers, 
it is probably diethyl ether or ethyl ethanoate (ethyl acetate). 
If you are familiar with organics, 
perhaps you can pick between these. 
Otherwise just label the bottle ``volatile organic solvent, 
insoluble in water, 
EXTREMELY FLAMMABLE'' and keep it around.

If the unknown has a sweet sickly smell it might be toluene. 
It also might be benzene. 
If you cannot further identify it and no one else can, 
label the bottle ``unknown non-volatile organic solvent 
less dense that water, 
possibly benzene, 
TOXIC, 
never use, 
never dump'' with similar warnings in any local language(s), 
and find a safe place to store it.

%==============================================================================
\section{Test five: What else?}
\label{sec:whatelse}
If you unknown is not an acid, 
not ammonia, 
and soluble in water, 
see what it smells like. 
If it smells like booze or nail polish remover, 
it could be methanol, 
ethanol, 
or acetone. 
See the above section on organics. 
If it does not have a smell, 
it is probably a solution left over from an earlier lab. 
These are not nearly as dangerous as concentrated acids 
or some volatile organics. 
However, 
be sure to use proper handing methods. 
Here are some possibilities:

%------------------------------------------------------------------------------
\subsection{Sodium hydroxide solution}

\begin{itemize}

\item{Hints: cloudy and a jammed stopper, 
but not always}

\item{Test: red litmus turns blue or POP pink.}

\item{What to do: sodium hydroxide is cheap when bought as caustic soda. 
Keep it around just for neutralizing acid wastes. 
If its presence disturbs you, 
add some indicator and then cheap acid until neutralization. 
After complete neutralization, 
dump.}

\end{itemize}

%------------------------------------------------------------------------------
\subsection{Hydrogen peroxide}

\begin{itemize}

\item{Hints: colorless liquid, 
in an opaque or dark bottle}

\item{Test: add a bit of acidified potassium permanganate solution. 
The potassium permanganate should turn colorless 
on mixing and bubbles of gas should be observed.}

\item{What to do: label and use. 
If you want to dispose of it for some reason, 
leave it in a bucket in the sun.}

\end{itemize}

%------------------------------------------------------------------------------
\subsection{Potassium permanganate solution}

\begin{itemize}

\item{Hints: intensely purple, 
pink after significant dilution}

\item{Test: to a very dilute solution, 
add crushed vitamin C (ascorbic acid) from a pharmacy. 
The solution should turn colorless.}

\item{What to do: test the pH with litmus paper 
or methyl orange to see if acid has been added. 
Then label ``(acidified) potassium permanganate'' and use. 
If you want to dispose of it, 
add crushed vitamin C until the color disappears 
and then pour into a pit latrine.}

\end{itemize}

%------------------------------------------------------------------------------
\subsection{Iodine solution}

\begin{itemize}

\item{Hints: brown color, 
smells like iodine tincture, 
and possibly also like ethanol}

\item{Test: to a dilute solution, 
add crushed vitamin C (ascorbic acid) from a pharmacy. 
The solution should turn colorless.}

\item{What to do: Put a centimeter of water in a test tube 
followed by a smaller quantity of cooking oil. 
Add a few drops of the solution, 
cap with your thumb and mix thoroughly for one minute. 
If two layers quickly separate, 
the iodine solution has been prepared without ethanol. 
If a cloudy mixture (an emulsion) forms, 
the iodine solution has been prepared with ethanol. 
Label the solution ``iodine solution (with ethanol)'' and use it.}

\end{itemize}

%------------------------------------------------------------------------------
\subsection{Potassium ferrocyanide solution}

\begin{itemize}

\item{Hints: light neon green or yellow color}

\item{Test: make a dilute solution of copper sulfate and add a few drops of the unknown. 
An instant, 
massive brown precipitate confirms potassium ferrocyanide solution.}

\item{What to do: Label and use. 
Do not dump while it remains useful.}

\end{itemize}

%==============================================================================
\section{Unidentifiable Liquids}

…are worthless. 
In order to safely dump a liquid chemical, 
ensure the following are true:
The liquid is water soluble (otherwise see the organic section above)
The liquid is neutral pH (if acid, 
neutralize with bicarbonate of soda, 
if base neutralize with acid waste, 
citric acid, 
or, 
carefully, 
battery acid)
The liquid does not contain heavy metals (to a small sample, 
add dilute sulfuric acid drop-wise. 
A precipitate indicated lead or barium. 
Continue adding until additional precipitation stops. 
Then neutralize with bicarbonate of soda. 
The solids are safe for disposal in a pit latrine, 
but may clog sink pipes).
The liquid does not contain mercury (to a small sample, 
add sodium hydroxide solution until POP turns the solution pink. 
A yellow precipitate indicates mercury. 
Label the solution ``Contains Hg$_{2}^{+}$, 
TOXIC, 
do not use, 
do not dump'' and store it in a safe place.)
Then, 
dilute the chemical in a large amount of water 
and dispose of it in a lab sink or pit latrine.

%==============================================================================
\section{Deliquescent Salts}

If you have a chemical in a container that seems meant 
for holding solids but the chemical looks like a thick liquid, 
you probably have a deliquescent salt that fully deliquesced. 
These solutions can be quite dangerous 
because they are maximally concentrated. 
Make sure that no unknown chemicals touch your skin, 
and wear goggles for this work. 
Then, 
do the following:

%------------------------------------------------------------------------------
\subsection{Test for a base}
The most common deliquescent salt is sodium hydroxide. 
Fill a test tube half way with water 
and a few drops of the unknown syrup 
followed by a few drops of POP indicator. 
If the solution turns pink, 
you almost certainly have either 
sodium hydroxide or potassium hydroxide. 
Dilute the liquid in at least 10 times its 
rough volume of water and titrate a sample against 1M acid. 
Find a plastic water bottle with a screw cap for your dilution and 
label it ``sodium or potassium hydroxide, 
$n$~M'', 
where $n$ is the molarity you measured in your titration.

%------------------------------------------------------------------------------
\subsection{Color}
If the liquid is not a base, 
it is probably a chloride or nitrate salt of one element or another. 
If it is colorless, 
the cation is probably in Group IIA (Ca, 
Sr, 
or Ba) or Group IIB (Zn, 
etc). 
Group IIA compounds have distinct flame test colors: Ca = orange red, 
Sr = bright red, 
Ba = apple green) while Zn has no flame test color. 
If it is red or brown, 
it is probably iron (III) nitrate or iron (III) chloride. 
If it is intensely pink, 
it might be cobalt. 
To identify the compound completely, 
you will have to perform qualitative inorganic analysis. 
An introduction to the art is 
in the Qualitative Analysis section of this manual, 
and more advanced methods are available on the internet 
and in some advanced chemistry books.

%------------------------------------------------------------------------------
\subsection{Check for mercury}
If the liquid is not a base, 
dilute a small sample in water 
and add sodium hydroxide solution until POP turns pink. 
If a bright yellow precipitate forms, 
you probably have a mercury salt. 
Transfer all of the compound to a sturdy container 
with a well-sealing lid, 
wash the original container with minimal water once 
and add the washings to the storage container. 
Then wash the original container 
and anything the liquid touched thoroughly. 
Label the new storage container ``Solution of unknown mercury salt, 
CONTAINS Hg!!, 
TOXIC! Do not use, 
Do not dump,'' along with appropriate warnings in any local language(s), 
and find a safe and secure place for long term storage.

%==============================================================================
\section{Identifying Unknown Solid Chemicals}

This is not nearly is important as identifying unknown liquids 
for two reasons. 
First, 
these chemicals are generally (though not always!) less dangerous, 
and second, 
accidental spills are less dramatic. 
The smallest containers are the most likely to hold dangerous chemicals, 
like mercury salts. 
It is best to just leave these ones alone.

What you can do is look at the solid 
and see if it matches any of the descriptions below.
Color is much more useful for identifying solids.

\begin{itemize}

\item{Bright orange crystals are likely a chromate 
or dichromate salt (toxic) or a ferricyanide salt (much less toxic). 
The later will form an intensely blue precipitate 
with a small amount of Fe$_{2}^{+}$, 
perhaps from iron (II) sulfate. 
Chromates form a yellow solution that turns orange 
on addition of acid while dichromates for an orange solutions 
that turns yellow on addition of base.}

\item{Bright yellow, 
orange, 
or red powders might be lead or mercury compounds. 
These are poisonous, 
the latter very. 
It also might be methyl orange powder. 
Try to dissolve a small amount in water. 
Methyl orange will dissolve readily to give a bright orange solution, 
one that turns red in acid and yellow in base. 
Label the powder and keep it around. 
If the salt dissolves but does not seem to be methyl orange, 
add sodium hydroxide until POP changes color. 
A yellow precipitate suggests mercury. 
Label as with mercury compounds encountered above. 
Most lead compounds are not soluble, 
and will not form a color in solution. 
Other mercury compounds are also insoluble. 
Label a container that might be lead or mercury as 
``possible lead or mercury compound, 
POISON,'' and store it for the long haul.}

\item{A yellow powder insoluble in water may also be sulfur. 
It should smell like sulfur. 
A small amount will dissolve in kerosene, 
and the dry powder will melt when heated in a spoon 
over a flame and then burn with a blue flame -- 
producing sulfur dioxide, 
a poisonous gas. 
Do not heat an unknown yellow compound 
unless you are fairly sure it is sulfur.}

\item{Blue compounds are often copper salts. 
These should have a green flame test.}

\item{Purple crystals or flakes insoluble in water are probably iodine. 
Iodine will dissolve in kerosene to form a red solution.}

\item{One of the few green powders is nickel carbonate.}

\item{Pink wet looking crystals might be a cobalt compound. 
Heat them gently in a spoon and they should dehydrate to turn blue. 
The blue crystals should turn pink when dissolved in water. 
Cobalt is poisonous.}

\item{Crystals so purple they look brown or yellow 
are probably potassium permanganate. 
They should form an intensely purple solution in water. 
Confirm as with potassium permanganate solution above.}

\item{White crystals and powders are really hard to identify. 
Label them ``unknown white powder/crystals'' 
and move them to a safe and secure place.}

\item{Flat dull gray metallic ribbon about 5~mm wide 
and 1~mm thick is probably magnesium metal. 
It should turn shiny if polished with steel wool. 
It will also burn with a very intense white light 
if lit in either a Bunsen burner or gas cigarette lighter. 
Hold it with tongs, 
and do not stare at the light.}

\item{A metal stored under oil is probably sodium or potassium. 
If you are feeling adventurous, 
remove a sample and cut off a VERY small piece, 
perhaps 5~mm on a side. 
Both metals may be easily cut with a knife. 
Return the rest to the original container and seal it again. 
Then, 
add the piece of metal to an open container of water and stand back. 
Both react violently and generally send the piece of metal 
spinning around on a cushion of hydrogen gas. 
Potassium generally gets hot enough to ignite this gas 
which then burns with a lilac flame. 
If the hydrogen under sodium burns, 
it will be yellow. 
The water will become a solution of sodium or potassium hydroxide.}

\end{itemize}

\chapter{Qualitative Analysis}
\label{cha:qualana}
Of all branches of chemistry studied in secondary schools, 
qualitative analysis bears the most resemblance to alchemy. 
Students receive a salt, 
a source of heat, 
a variety of dangerous chemicals, 
and sometimes a few pages of paper listing 
qualitative analysis procedures. 
Students then follow the procedures to identify their salt. 
There is great potential for both harm and mess. 
At the same time, 
students have an opportunity to independently perform 
a large variety of chemical reactions, 
and do observe some remarkable transformations.

For the chemistry teacher or lab manager, 
qualitative analysis is a useful process for identifying unknown salts, 
thus permitting their use. 
The process is time consuming and requires a wide variety of reagents. 
Many, 
however, 
are common in chemistry labs, 
and many can be found from local sources.

This section contains the following:
\begin{itemize}
\item{Overview of qualitative analysis}
\item{Description of each step}
\item{List of bench reagents with recommended concentrations}
\item{Hazards and cleanliness}
\item{Qualitative analysis with limited resources}
\end{itemize}

%==============================================================================
\section{Overview of Qualitative Analysis}
The goal of qualitative analysis is to identify an unknown salt, 
both the cation and the anion. 
Generally, 
these are identified separately 
although often knowing one helps interpret 
the results of tests for the other. 
This guide is written for the requirements of 
ordinary level qualitative analysis in Tanzania -- 
the chemistry applies everywhere 
although the process may have to be adapted 
to different expectations and requirements. 
In Tanzania, 
students are confronted with binary salts made from the following ions:
\begin{itemize}
\item{Cations: NH$_{4}^{+}$, 
Ca$_{2}^{+}$, 
Fe$_{2}^{+}$, 
Fe$_{3}^{+}$, 
Cu$_{2}^{+}$, 
Zn$_{2}^{+}$, 
Pb$_{2}^{+}$, 
Na$^{+}$}
\item{Anions: CO$_{3}^{2-}$, 
HCO$_{3}^{-}$, 
NO$_{3}^{-}$, 
SO$_{4}^{2-}$, 
Cl$^{-}$}
\end{itemize}
At present, 
ordinary level students receive only one salt at a time. 
Advanced level students may receive mixtures.

The ions are identified by following a series of ten steps, 
divided into three stages. 
These are:
\begin{itemize}
\item{Preliminary tests:
These tests use the solid salt. 
They are: appearance, 
flame test, 
action of heat, 
action of dilute H$_{2}$SO$_{4}$, 
action of concentrated H$_{2}$SO$_{4}$, 
and solubility.}
\item{Tests in solution:
The compound should be dissolved in water before carrying out these tests. 
If it is not soluble in water, 
use dilute HNO$_{3}$ to dissolve the compound. 
The tests in solution involve addition of NaOH and NH$_{3}$.}
\item{Confirmatory tests:
These tests confirm that the conclusions students draw from the previous steps. 
By the time your students start the confirmatory tests, 
they should have a good idea of what cation and anion are present. 
Have students do one confirmatory test for the cation they believe is present, 
and one for the anion you believe is present. 
Even if several confirmatory tests are listed, 
students only need to do one. 
When identifying an unlabelled container, 
however, 
you might be moved to try several, 
especially if you are new to this process.}
\end{itemize}

%==============================================================================
\section{Description of Each Step}
\subsection{Appearance}
Students observe three properties: the color of the salt, 
the physical form of the salt (whether it is powdery or crystalline), 
and the salt’s smell. 
The color shows whether a transition metal such as copper 
or iron is present. 
A powdery salt contains carbonate or hydrogen carbonate; 
a crystalline salt does not. 
Students should not touch the salt; 
they can tell if it is powder or crystalline with their eyes. 
A salt with a sharp smell like stale urine contains ammonium ions. 
Not all ammonium salts have a strong smell. 
Students should not inhale the salt or put it near their nose. 
If an ammonia smell is present, 
the smell will be obvious.

\subsection{Flame test}
A solid sample of the salt is placed in a non-luminous flame. 
The best flames are, 
in order, 
a bunsen burner, 
an alcohol infused heavy oil burner 
(see \nameref{sec:heatsources} in \nameref{cha:labequip}), 
a spirit burner, 
a butane lighter, 
and a kerosene stove without the outer cover 
to allow access to the blue flame. 
Some salts produce a characteristic flame color: 
copper produces blue or green flames and calcium produces brick red flames. 
Sodium and lead can be problematic: sodium produces a golden yellow flame, 
which can be hard to distinguish 
from the ordinary flame of some kerosene stoves, 
and the bluish-white flame of lead is a rare find at lower temperatures, 
though is obvious in a Bunsen burner.

\subsection{Action of heat}
The purpose of this test is to decompose the salt. 
Students should observe two things: the formation of gas or a residue. 
Damp pieces of blue and red litmus paper should be held 
over the test tube to test the acidity or basicity of any gas formed.

Gases can be problematic. 
Gases that are easy to identify are nitrogen dioxide 
(which is brown) and ammonia (which is basic). 
Most sulfates do not decompose to sulfur dioxide when heated, 
so students should not expect a gas if sulfate is present. 
Carbonates decompose, 
but carbon dioxide does not usually change the color of litmus paper. 
Chlorides do not decompose when heated. 
A test that produces no identifiable gas is inconclusive: a carbonate, 
sulfate, 
or chloride are possible anions. 
Do not teach the students to leave it blank and fill it in later – 
this is lying, 
and terrible science. 
Tell them to write what they see and what it means – no gas observed, 
so a sulfate of chloride may be present..

If the salt makes a crackling sound while heating, 
then either sodium chloride or lead (II) nitrate is present. 
None of the other salts used in qualitative analysis have this property.

\subsection{Action of dilute H$_{2}$SO$_{4}$}
This step tests for carbonates. 
The student should add a few drops of dilute sulfuric acid 
(0.5 -- 1~M) to the solid sample. 
If effervescence (bubbles) is observed, 
then carbonates or hydrogen carbonates are present. 
If there is no effervescence, 
then they are absent. 
To confirm that the gas is carbon dioxide, 
students can pass it through lime water 
(hard) or show that it will extinguish a glowing splint (easy).

\subsection{Action of concentrated H$_{2}$SO$_{4}$}
If no identifiable gas was formed due to the action of heat, 
the addition of concentrated H$_{2}$SO$_{4}$ will help students 
distinguish whether a chloride or sulfate is present. 
The student should just cover a small sample 
with concentrated sulfuric acid (5~M, 
or battery acid -- not the full 18~M sulfuric acid from a stock bottle. 
This is unnecessarily dangerous) and heat gently. 
Hold a piece of blue litmus above the mouth of the test tube. 
If no gas forms, 
then sulfate anion is present. 
If a colorless, 
acidic gas forms, 
the blue litmus will turn red. 
This means that HCl gas formed from the addition of concentrated sulfur acid. 
This means that the anion is chloride. 
Unfortunately, 
students want to see a gas, 
and will often boil the acid solution 
until the litmus paper changes color due to 
acidic fumes of the sulfuric acid rather than the decomposition of the salt. 
Advise them to remove the test tube from the heat before the acid boils -- 
if no gas has formed by this point, 
than it is unlikely that a gas will ever form. 
Further, 
boiling concentrated acids efficiently courts injury.

SAFETY: if the addition of dilute acid identified 
a carbonate or hydrogen carbonate, 
students should add very little concentrated acid -- 
just a drop or two. 
The addition of more concentrated sulfuric acid 
to a powder carbonate can cause such violent effervescence 
that concentrated acid is ejected from the test tube. 
This can blind.

\subsection{Solubility}
Students test whether the salt is soluble in water, 
and observe the color of the solution. 
If the salt is not soluble in cold water, 
they test its solubility in warm water by heating the mixture. 
Only two commonly used salts 
(lead chloride and calcium sulfate) are insoluble in cold water 
but soluble in warm water. 

Most qualitative analysis sheets have a table of solubility on the last page. 
Students can use this table to cross check their final results. 
For example, 
if students identified a soluble salt as calcium carbonate, 
then they made a mistake and should repeat some of their tests.

Better than tables, 
teach students basic rules of solubility:
All nitrates and hydrogen carbonates are soluble
All Group I (sodium, 
potassium, 
etc) and ammonium salts are soluble (sodium borate is an exception)
Most chlorides are soluble (silver and lead chlorides are exceptions, 
though the latter is soluble in hot water)
Carbonates of metals outside of Group I are generally insoluble 
(note that aluminum and iron carbonate do not exist)
Lead sulfate is insoluble and calcium sulfate is soluble only in hot water. 
Magnesium sulfate is completely soluble 
while sulfates of the Group II metals heavier than calcium 
(strontium and barium) are insoluble.

\subsection{Addition of NaOH solution}
NaOH solution is added to a solution of the salt. 
The purpose of this test is to identify the cation present. 
A small amount of NaOH solution should be added first 
(to identify the color of the precipitate), 
then a large amount should be added (to test if the precipitate 
forms a hydroxide complex soluble in excess sodium hydroxide solution). 
Blue precipitates indicate copper cations. 
Green precipitates suggest iron (II) 
and red/brown precipitates suggest iron (III). 
White precipitates insoluble in excess hydroxide are from calcium – 
this is often observed when hydroxide is added to ordinary hard water. 
White precipitates soluble in excess hydroxide are either zinc or lead. 
If no precipitate forms, 
the cation is probably sodium or ammonium.

\subsection{Addition of NH$_{3}$ solution}
NH$_{3}$ solution is added to a solution of the salt. 
The purpose of this test is to identify the cation. 
As with NaOH, 
a small amount of solution should be added first, 
followed by a large amount. 
The results similarly divide the cations into groups 
based on the presence and nature of the precipitate – 
importantly zinc produces a precipitate that dissolves in excess ammonia 
while the precipitate from lead remains. 
Note that NH$_{3}$ solution degrades over time 
due to evaporation of the ammonia. 
Store this reagent in plastic water bottles with screw-on caps 
and test the solution yourself before using it with students.

\subsection{Confirmatory tests}
Emphasize to students that they need to carry out 
only one confirmatory test for the cation, 
and one for the anion. 
If the test gives the expected result, 
then they can be sure that the ion they have identified is present. 
If the test does not give the expected result, 
they have probably made a mistake, 
and they should revisit the results of their previous tests 
and choose a different possibility to test.

%==============================================================================
\section{List of Bench Reagents}

The reagents required for most qualitative analysis practicals are:
\begin{itemize}
\item{dilute (about 0.5~M) sulfuric acid, 
H$_{2}$SO$_{4}$}
\item{concentrated (about 5~M) sulfuric acid, 
H$_{2}$SO$_{4}$}
\item{sodium hydroxide solution, 
NaOH}
\item{ammonia solution, 
ammonium hydroxide solution, 
NH$_{3}$}
\item{dilute nitric acid (about 1~M), 
HNO$_{3}$}
\end{itemize}

The reagents used for confirmatory tests are:
\begin{itemize}
\item{Ammonium ethandioate, 
ammonium oxalate, 
COO(NH$_{4}$)$_{2} \cdot$2H$_{2}$O}
\item{Ammonium thiocyanate, 
NH$_{4}$SCN, 
or potassium thiocyanate, 
KSCN}
\item{Barium chloride, 
BaCl$_{2}$, 
or barium nitrate, 
BaNO$_{3}$}
\item{Copper turnings, 
small pieces of copper metal}
\item{Iron (II) sulfate, 
FeSO$_{4}$}
\item{Magnesium sulfate, 
MgSO$_{4}$}
\item{Manganese (IV) oxide, 
manganese dioxide, 
MnO$_{2}$}
\item{Potassium chromate, 
KCrO$_{4}$}
\item{Potassium hexacyanoferrate (II), 
potassium ferrocyanide, 
K$_{4}$Fe(CN)$_{6} \cdot$3H$_{2}$O}
\item{Potassium hexacyanoferrate (III), 
potassium ferricyanide, 
K$_{3}$Fe(CN)$_{6}$}
\item{Potassium iodide, 
KI}
\item{Silver nitrate, 
AgNO$_{3}$}
\end{itemize}

Other reagents which may be required in qualitative analysis:
\begin{itemize}
\item{Ethanoic (acetic) acid (CH$_{3}$COOH).}\\
\textit{Glacial (concentrated) acetic acid 
is sometimes used in a confirmatory test for calcium.}
\item{Lead ethanoate (acetate) (Pb(CH$_{3}$COO)$_{2} \cdot$3H$_{2}$O)}\\
\textit{Sometimes used as a confirmatory test for sulfates.}
\end{itemize}
You do not need all the confirmatory test reagents for a single practical. 
At minimum, 
you only need the reagents required to confirm the ions 
that are present in that practical. 
However, 
it is a good idea to have the other solutions around -- 
students will often ask for them, 
and it is better to let the students realize their errors 
by carrying out these confirmatory tests 
than to simply tell them that they are wrong. 
In addition, 
the more clever students may infer the presence of a certain ion 
simply by looking at the solutions you have placed on their bench.

%==============================================================================	
\section{Hazards and Cleanliness}
Qualitative analysis practicals are full of hazards, 
from open flames to concentrated acids. 
To reduce the risk of accidents, 
teach students how to use their flame source 
before the day of the practical, 
especially if you are using Bunsen burners. 
Most students have never used gas before, 
and do not know the basic safety precautions involved in using gas. 
If you have choice about what salts are offered, 
do away with those requiring concentrated acid.

Teach students to hold their test tubes at an angle when they heat them. 
Test tubes should be pointed away from the student holding them 
and from other students. 
This will prevent injuries due to splashing chemicals, 
and will also minimize inhalation of any gases produced.

Teach students never to fill test tubes 
or any other container more than half. 
That way, 
they minimize spills and boiling over of chemicals during heating. 
In addition, 
this also prevents bumping in the test tubes 
(when a gas bubble forms suddenly), 
which can cause dangerous spray.

Teach students that if they get chemicals on their hands, 
they should wash them off immediately, 
without asking for permission first. 
Some students have been taught to wait for a teacher's permission 
before doing anything in the lab, 
even if concentrated acid is burning their hands. 
On the first day, 
give them permission to wash their hands 
if they ever spill chemicals on them. 
Also, 
teach students to tell you immediately when chemicals are spilled. 
Sometimes they hide chemical spills for fear of punishment. 
Do not punish them for spills -- legitimate accidents happen. 
Do punish them form unsafe behavior of any kind, 
even if it does not result in an accident. 
See the \nameref{sec:basiclabrules} in \nameref{cha:labsafety}.

Practicals involving nitrates, 
chlorides, 
ammonium compounds, 
and some sulfates produce harmful gases. 
Open the lab windows to maximize airflow. 
Kerosene stoves also produce noxious fumes -- 
all the more reason to consider an alternative flame source. 
If students feel dizzy or sick from the fumes, 
let them go outside to recover.

Make absolutely sure that students clean their tables 
and glassware before they leave. 
Walking into a lab sprinkled with mystery salts 
and test tubes of unlabelled liquids is annoying and possibly dangerous. 
Qualitative analysis experiments can leave residues in test tubes 
that are difficult to clean with brushes alone. 
To remove stubborn residues, 
pour a little dilute nitric acid into the test tube. 
The acid should dissolve the precipitate 
and leave a clean test tube behind. 
Remember that the utility of nitric acid -- 
that it will dissolve almost anything -- is also a serious hazard.

For additional advice, 
refer to the chapters on \nameref{cha:labsafety} and \nameref{cha:dangerchem}.

%==============================================================================
\section{Qualitative Analysis with Limited Resources}

Qualitative analysis can seem impossible for schools 
without a traditional laboratory. 
There are so many specific chemicals 
and so many complicated procedures. 
To the contrary, 
students can be well prepared for the national exam 
using very basic resources. 
This section offers some general suggestions as well as 
a specific procedure for teaching qualitative analysis.

\subsection{General Suggestions}
\begin{itemize}
\item{Heat sources – Alcohol infused heavy oil burners 
cost nothing to make and consume only a small amount of fuel. 
They give a non-luminous flame ideal for flame tests 
and still produce enough heat for the other tests. 
See \nameref{sec:heatsources} in \nameref{cha:labequip}.}
\item{Test tubes – Most of the tests do not involve heating, 
so students may perform these experiments in plastic tubes 
made from disposable plastic syringes. 
Many of the tests requiring heating use the salt in solution 
and thus can be performed 
by holding the plastic test tube in a hot water bath. 
For more information, 
read the entries for \nameref{sec:testtubes} 
and \nameref{sec:hotwaterbathes} in \nameref{cha:labequip}.
For the action of heat test, 
salts may be heating in metal spoons to observe residue products, 
although it is difficult to test the gases produced. 
Do not wait for test tubes to start teaching qualitative analysis. 
Just try to find at least one borosilicate (Pyrex, 
Borosil) test tube per student before the national exams.}
\item{Litmus paper – Make your own. 
See \nameref{sec:indicator} in \nameref{cha:sourcesofchemicals}.}
\item{Low cost sources of chemicals. 
Many chemicals have low cost alternatives. 
For any specific chemical, 
read its entry in \nameref{cha:sourcesofchemicals}.}
\item{Share expensive chemicals among many schools. 
A single container of potassium ferrocyanide, 
for example, 
can supply ten or even twenty schools for several years. 
Schools might consider bartering 10~g of one chemical for 10~g of another. 
Schools without any expensive chemicals could produce distilled water (see \nameref{sub:distillwater}) and exchange this for 10~g samples of expensive salts.}
\end{itemize} 
 
%==============================================================================
\section{Performing Qualitative Analysis in the Basic Chemistry Laboratory}

For each step in the qualitative analysis procedure, 
we recommend a procedure using low cost materials 
and offer low cost compounds for students to use to learn this step.

\subsection{Appearance}
\begin{itemize}
\item{Procedure: place salt on a piece of plastic cut from a water bottle}
\item{Example salts:}
\begin{itemize}
\item{White salts = sodium chloride, 
sodium carbonate, 
sodium hydrogen carbonate}
\item{Blue salts = copper sulfate}
\item{Yellow/Green salts = iron sulfate}
\item{Powder salts = sodium carbonate, 
sodium hydrogen carbonate}
\item{Crystal salts = sodium chloride, 
copper sulfate}
\item{Ammonia smell = calcium ammonium nitrate}
\end{itemize}
\item{Reagents: none}
\end{itemize}

\subsection{Flame test}
\begin{itemize}
\item{Procedure: clean an iron wire or metal spoon handle 
with steel wool and use it to hold a small amount 
over a non-luminous flame.}
\item{Example salts:}
\begin{itemize}
\item{blue/green flame = copper sulfate}
\item{golden yellow flame = sodium chloride, 
sodium carbonate, 
sodium hydrogen carbonate}
\item{brick red flame = calcium sulfate}
\item{no flame color = iron sulfate}
\item{crackling sound = sodium chloride}
\end{itemize}
\item{Reagents: none}
\end{itemize}

\subsection{Action of heat}
\begin{itemize}
\item{Procedure: place a small quantity of the salt 
in a metal spoon and heat over a flame}
\item{Example salts:} 
\begin{itemize}
\item{blue to white dehydration = copper sulfate}
\item{black residue = copper sulfate}
\item{yellow to white residue = zinc carbonate or zinc sulfate} 
\item{red to dark residue = iron sulfate}
\end{itemize}
\item{Reagents: none}
\end{itemize}

\subsection{Action of Dilute Acid} 
\begin{itemize}
\item{Procedure: place a small quantity of the salt in a test tube, 
plastic or glass, 
and add dilute sulfuric acid drop-wise}
\item{Example salts:}
\begin{itemize}
\item{effervescence = sodium carbonate, 
sodium hydrogen carbonate}
\item{no effect = sodium chloride, 
calcium sulfate}
\end{itemize}
\item{Reagents: prepare dilute sulfuric acid 
by adding one part battery acid to nine parts water (e.g. 
fill a 1.5~L bottle half full with ordinary water, 
add 150~mL of battery acid, 
and then fill the bottle the rest of the way 
with water.in 900~mL of water)}
\end{itemize}

\subsection{Action of Concentrated Acid}
\begin{itemize}
\item{Procedure: place a small quantity of the salt in a (plastic) test tube, 
add concentrated sulfuric until the sample is submerged, 
and heat in a water bath. 
Hold blue litmus over the tube while heating.}
\item{Example salts:}
\begin{itemize}
\item{acidic gas produced: sodium chloride}
\item{no gas produced: calcium sulfate}
\end{itemize}
\item{Reagents: use battery acid for concentrated sulfuric acid}
\end{itemize}

\subsection{Solubility}
\begin{itemize}
\item{Procedure: add a small quantity of the salt in a test tube 
and then fill it half way with distilled water. 
If the salt does not dissolve in cold water, 
heat in a water bath.}
\item{Example salts:}
\begin{itemize}
\item{Soluble in cold water to form a clear solution = sodium chloride}
\item{Soluble in cold water to form a blue solution = copper sulfate}
\item{Soluble in cold water to form a solution of another color = iron sulfate}
\item{Soluble only in hot water = calcium sulfate} 
\item{Insoluble in water = copper carbonate}
\end{itemize}
\item{Reagents: rain water works as distilled water. 
If no rain water is available, 
tap water may be distilled with a tea kettle, 
rubber tubing, 
and some water bottles or a bucket.}
\end{itemize}

\subsection{Making a Solution}

To make a solution, 
dilute nitric acid is required for lead carbonate, 
lead sulfate and lead chloride 
while dilute hydrochloric acid will suffice 
for calcium carbonate and calcium sulfate. 
See the entry for \nameref{sec:hydroacid} in \nameref{cha:sourcesofchemicals}. 
Dilute battery acid will suffice for all other insoluble salts. 

\subsection{Action of sodium hydroxide solution}
\begin{itemize}
\item{Procedure: add about 1~mL of a solution of the salt to a test tube, 
plastic or glass, 
and then add sodium hydroxide solution drop wise.}
\item{Example salts:}
\begin{itemize}
\item{no effect = sodium chloride, 
sodium carbonate}
\item{blue precipitate, 
insoluble in excess = copper sulfate}
\item{red precipitate, 
insoluble in excess = iron sulfate}
\item{white precipitate insoluble in excess = calcium sulfate}
\item{white precipitate, 
soluble in excess NaOH = zinc sulfate, 
zinc carbonate}
\end{itemize}
\item{Reagents: to prepare 1~M sodium hydroxide solution, 
dissolve 4~g of caustic soda in 100~mL of distilled water.}
\end{itemize}

\subsection{Action of Ammonia Solution}
In theory, 
ammonia may be prepared, 
with difficulty, 
by mixing sodium hydroxide (caustic soda) 
and calcium ammonium nitrate (a fertilizer) in solution 
and distilling the mixture. 
A liquid rich in dissolved ammonia will be produced 
as the distillate (ammonia liquor). 
This is potentially dangerous, 
especially if the connections are not well sealed. 
Pass exit gases through water to prevent filling the room 
with ammonia fumes. 
Bear in mind also that ammonia is corrosive, 
so things that start well sealed might not remain so. 

Fortunately, 
the procedure and results of this step are similar 
to that of adding sodium hydroxide. 
Therefore, 
students can learn about this technique 
and how to recognize results without the use of ammonia. 
Then the school can find a small quantity for mock and the national exam.
Note: a small quantity can be used for many tests 
if students learn to use a small quantity for each test!

\subsection{Confirmatory Tests for the Cation}

\subsubsection{Ammonium}
\begin{itemize}
\item{Example salt: calcium ammonium nitrate}
\item{Procedure: add sodium hydroxide solution and heat in a water bath}
\item{Confirming result: smell of ammonia} 
\item{Reagents: NaOH solution as used in Step 7 above}
\end{itemize}

\subsubsection{Calcium}
\begin{itemize}
\item{Example salt: calcium sulfate}

\item{Procedure: Two options
\begin{enumerate}
\item{flame test}
\item{addition of NaOH solution}
\end{enumerate}
} % Procedure

\item{Confirming results:
\begin{enumerate}
\item{brick red flame}
\item{white precipitate insoluble in excess}
\end{enumerate}
} % Confirming results

\item{Reagents:
\begin{enumerate}
\item{none}
\item{NaOH solution from Step 7}
\end{enumerate}
} % Reagents

\end{itemize} % Calcium

\subsubsection{Copper}
\begin{itemize}
\item{Example salt: copper sulfate}
\item{Procedure: flame test}
\item{Confirming result: blue/green flame}
\item{Reagents: none}
\end{itemize}

\subsubsection{Iron (II)}
\begin{itemize}
\item{Example salt: locally manufactured iron sulfate 
(kept away from water and air)}
\item{Procedure: addition of sodium hydroxide solution 
and then transfer of precipitate to the table surface}
\item{Confirming result: green precipitate 
that oxidizes to brown when exposed to air}
\item{Reagent: sodium hydroxide solution from Step 7}
\end{itemize}

\subsubsection{Iron (III)}
\begin{itemize}
\item{Example salt: locally manufactured iron sulfate 
(oxidized by water and air)}
\item{Procedure: addition of sodium ethanoate solution}
\item{Confirming result: blood red solution}
\item{Reagent: see \nameref{sec:sodiumeth} in \nameref{cha:sourcesofchemicals}}
\end{itemize}

\subsubsection{Lead}
\begin{itemize}
\item{Example salt: no local sources for safe manufacture, 
consider purchasing lead nitrate}

\item{Procedure: Three options
\begin{enumerate}
\item{flame test} 
\item{addition of dilute sulfuric acid}
\item{addition of potassium iodide solution}
\end{enumerate}
} % Procedure

\item{Confirming results:
\begin{enumerate}
\item{blue/white flame}
\item{white precipitate}
\item{yellow precipite that dissolves when heated and reforms when cold}
\end{enumerate}
} % Confirming results

\item{Reagents:
\begin{enumerate}
\item{none but a very hot flame, e.g. Bunsen burner, is required} 
\item{dilute sulfuric acid used in Step 5 above}
\item{see \nameref{sec:potiodide} in \nameref{cha:sourcesofchemicals}}
\end{enumerate}
} % Reagents

\end{itemize} % Lead

\subsubsection{Sodium}
\begin{itemize}
\item{Example salt: sodium chloride, 
sodium carbonate, 
sodium hydrogen carbonate}
\item{Procedure: flame test}
\item{Confirming result: golden yellow flame}
\item{Reagents: none}
\end{itemize}

\subsubsection{Zinc}
\begin{itemize}
\item{Example salt: locally manugactured zinc carbonate 
or zinc sulfate}
\item{Procedure: addition of potassium ferrocyanide solution}
\item{Confirming result: gelatinous gray precipitate}
\item{Reagents: no local source of potassium ferrocyanide -- 
consider collaborating with many schools to share a container; 
only a very small quantity is required}
\end{itemize}

\subsection{Confirmatory Tests for the Anion} 

\subsubsection{Hydrogen carbonate}
\begin{itemize}
\item{Example salt: sodium hydrogen carbonate}
\item{Procedure: add magnesium sulfate solution 
and then boil in a water bath}
\item{Confirming result: white precipitate forms only after boiling}
\item{Reagent: dissolve Epsom salts in distilled water 
(see \nameref{sec:magsulfate} in \nameref{cha:sourcesofchemicals})}
\end{itemize}

\subsubsection{Carbonate}
\begin{itemize}
\item{Example salt: sodium carbonate}
\item{Procedure for soluble salts: addition of magnesium sulfate solution}
\item{Confirming result: white precipitate forming in cold solution}
\item{Reagent: dissolve Epsom salts in distilled water 
(see \nameref{sec:magsulfate} in \nameref{cha:sourcesofchemicals})}
Note that insoluble salts that effervesce 
with dilute acid are likely carbonates. 
None of the other anions described here produce gas with dilute acid 
(all hydrogen carbonates are soluble).
\end{itemize}

\subsubsection{Chloride}
\begin{itemize}
\item{Example salt: sodium chloride}
\item{Procedure: Three Options
\begin{enumerate}
\item{addition of silver nitrate solution}
\item{addition of manganese IV oxide and concentrated sulfuric acid 
followed by heating in a water bath}
\item{addition of weak acidified potassium permangante solution 
followed by heating in a water bath}
\end{enumerate}
} % Procedure
\item{Confirming results:
\begin{enumerate}
\item{white precipitate of silver chloride} 
\item{production of chlorine gas that bleaches litmus} 
\item{decolorization of permanganate}
\end{enumerate}
} % Confirming results
\item{Reagents:
\begin{enumerate}
\item{silver nitrate has no local source 
but may be shared among many schools as only a small amount is required; 
also see \nameref{cha:recyclesilver}.}
\item{Manganese dioxide may be purified from used batteries 
(see \nameref{cha:sourcesofchemicals}) and battery acid is concentrated sulfuric acid. 
This method is useful because of its low cost, 
but remember that chlorine gas is poisonous! 
Students should use very little sample salt in this test.}
\item{Prepare a solution of potassium permanganate, 
dilute with distilled water until the color is light pink, 
and then add about 1 percent of the solution's volume in battery acid. 
Note that this solution will cause lead to precipitate, 
and will also be decolorized by iron II, 
so it is not a perfect substitute for silver nitrate. 
This final option may not be recognized by examination boards.}
\end{enumerate}
} % Reagents
\end{itemize}

\subsubsection{Sulfate}
\begin{itemize}
\item{Example salt: copper sulfate, 
calcium sulfate, 
iron sulfate}
\item{Procedure: addition of a few drops of a solution of lead nitrate, 
barium nitrate, 
or barium chloride}
\item{Confirming result: white precipitate}
\item{Reagents: none of these chemicals have local sources. 
Because lead nitrate is also an example salt, 
it is the most useful and the best to buy. 
The ideal strategy is to share one of these chemicals among many schools. 
Remember that all are toxic.}
\end{itemize}


% Part 2 - Lab Management
\part{Laboratory Management}

\chapter{Laboratory Safety}
\label{cha:labsafety}
There is no excuse for laboratory accidents. 
Students and teachers get hurt when they do something dangerous 
or when they are careless. 
If you do not know how to use a substance or a tool safely, do not use it. 
If your students do not how to use a chemical or a tool safely, 
do not let them use it until they do. 
Adopt a zero tolerance policy towards truly unsafe behavior 
(running, fighting, throwing objects, etc) -- 
first infraction gets to students kicked out of class for the day. 
Explain the error to everyone to make sure that it is never repeated. 
If the same student errs again, expel him for longer. 
Make it clear that you will not tolerate unsafe behavior.

Remember, the teacher is responsible for everything that happens in the lab. 
If a student is hurt the teacher is to blame. 
Either the teacher did not understand the danger present, 
did not adequately prepare the laboratory or the lesson, 
did not adequately train the student in safe behavior, 
or did not offer adequate supervision. 
As a teacher, you must know exactly the hazards of your chemicals, 
tools, and apparatus. 
Explain these hazards clearly and concisely to your students 
before they touch anything.

The following rules are for everyone in the lab to follow -- 
students, teachers, and visitors alike. 
We recommend painting them directly on the wall 
as most paper signs eventually fall down.

\section{Basic Lab Rules}
\label{sec:basiclabrules}
\begin{enumerate}
\item{Wear proper clothes. For every practical, wear shoes. 
Sandals are not acceptable lab ware. 
If you are pouring concentrated chemicals, you need to wear safety goggles.}
\item{Nothing enters the mouth in the lab. 
This means no eating, no drinking, and no mouth pipetting.}
\item{Follow the instructions from the teacher. 
Obey commands immediately. 
Only mix chemicals as instructed.}
\item{If you do not know how to do something or what to do, ask the teacher.}
\end{enumerate}
In addition to these rules, 
we recommend a variety of guidelines for teachers and lab managers 
to keep the lab a safe place.
\renewcommand{\theenumii}
{\arabic{enumi}.\arabic{enumii}.}
\renewcommand{\labelenumii}{\theenumii}
\renewcommand{\theenumiii}
{\arabic{enumi}.\arabic{enumii}.\arabic{enumiii}.}
\renewcommand{\labelenumiii}{\theenumiii}
\renewcommand{\theenumiv}
{\arabic{enumi}.\arabic{enumii}.\arabic{enumiii}.\arabic{enumiv}.}
\renewcommand{\labelenumiv}{\theenumiv}

%******************************************************************************
\subsection{Specific Guidelines to Reduce Risk}
\label{sub:specguide}
\begin{enumerate}

%==============================================================================
\item{Never use the following chemicals:}
\begin{enumerate}
\item{Organic liquids, including:}
\begin{enumerate}
\item{Benzene ($\mbox{C}_{6}\mbox{H}_{6}$)}
\item{Chlorobenzene ($\mbox{C}_{6}\mbox{H}_{5}\mbox{Cl}$)}
\item{Dichloromethane ($\mbox{CH}_{2}\mbox{Cl}_{2}$)}
\item{Tetrachloromethane/carbon tetrachloride ($\mbox{CCl}_{4}$)}
\item{Trichloroethane ($\mbox{CH}_{3}\mbox{CCl}_{3}$)}
\item{Trichloromethane/chloroform ($\mbox{CHCl}_{3}$)}
\end{enumerate}
\item{Anything containing mercury:}
\begin{enumerate}
\item{Mercury metal (Hg)}
\item{Mercurous/mercuric chloride (HgCl/$\mbox{HgCl}_{2}$)}
\item{Million's Reagent ($\mbox{Hg}+\mbox{HNO}_{3}$)}
\item{Nestler's Reagent ($\mbox{HgCl}_{2}+\mbox{others}$)}
\item{For more information about these chemicals, their risks, 
and what to do if you find them in your lab, 
see Laboratory Management: Dangerous Chemicals}
\end{enumerate}
\end{enumerate}

%==============================================================================
\item{Do not make hazardous substances}
\begin{enumerate}
\item{Chlorine gas - electrolysis of chloride salts, 
oxidation of chloride salts or hydrochloric acid 
by oxidizing agents such as bleach or potassium permanganate}
\item{Chloroamines - ammonia with bleach. People have died 
mixing ammonia and bleach together when mixing cleaning agents.}
\item{Hydrogen cyanide - cyanide salts, 
including ferro- and ferri-cyanide, with acids.}
\end{enumerate}

%==============================================================================
\item{Avoid hazardous substances}
\begin{enumerate}
\item{If you have a choice, use non-poisonous substances. 
To be a good teacher, the only poisons that you have to use 
are those required by the national exams. 
For all other activities, use less dangerous substances.}
\item{Only give students small quantities of required poisons.}
\item{For advice on handling the various required poisons, 
see Laboratory Management: Dangerous Chemicals.}
\end{enumerate}

%==============================================================================
\item{Avoid explosions}
\begin{enumerate}
\item{Never heat ammonium nitrate.}
\item{Never heat nitrates in the presence of anything that burns.}
\item{Never heat a closed container.}
\item{If performing a distillation 
or other experiment with boiling or hot gases, 
make sure that there is always an unobstructed path for gases to escape.}
\end{enumerate}

%==============================================================================
\item{Avoid fires \label{list:fire}}
\begin{enumerate}
\item{Be careful!}
\item{Keep all flammable materials away from flames. 
Never have the following very flammable chemicals in the same room as fire: 
propanone (acetone), ethyl ethanoate (ethyl acetate), diethyl ether.}
\item{Keep stoves clean and in good working order. 
Do not douse stoves with water to extinguish them 
because the metal will corrode much faster (think kinetics). 
There is never a need for this. 
If the stove does not extinguish on its own, you should repair it so it does.}
\item{Only use the appropriate fuel for a given stove. 
For example, never put petrol in a kerosene stove.}
\end{enumerate}

%==============================================================================
\item{Avoid cuts}
\begin{enumerate}
\item{Only use sharp tools when required, 
and design activities to minimize use of sharp tools.}
\item{Keep sharp tools sharp. 
The only thing more dangerous than cutting with a sharp knife 
is cutting with a dull one.}
\item{Use the right tool for cutting.}
\item{Use as little glass as possible.}
\item{Do not use broken glass apparatus. 
The last thing you want to deal with during a practical is serious bleeding. 
It is tempting to keep using that flask with the jagged top. 
Do not. 
Do not let anyone else use it either -- break it the rest of the way.}
\item{Dispose of sharp trash (glass shards, syringe needles) 
in a safe place, like a deep pit latrine.}
\end{enumerate}

%==============================================================================
\item{Avoid eye injuries}
\begin{enumerate}
\item{Students should wear goggles during any activity 
with a risk of eye injury. 
See the Materials: Apparatus section for suggestions on goggles. 
If you do not have the goggles necessary to make an experiment safe, 
do not do the experiment.}
\item{Keep test tubes pointed away from people during heating or reactions. 
Never look down a test tube while using it.}
\item{Never wear contact lenses in the laboratory. 
They have this way of trapping harmful chemicals behind them, 
magnifying the damage. 
Besides, glasses offer decent (though incomplete) protection on their own.}
\end{enumerate}

%==============================================================================
\item{Use adequate protection with hazardous chemicals.}
\begin{enumerate}
\item{Wear eye protection (see above). 
Find goggles or things that will substitute.}
\item{Tie a cloth over your face when using concentrated ammonia or HCl. 
For the latter chemical, see below.}
\item{Sulfuric Acid, $\mbox{H}_{2}\mbox{SO}_{4}$}
\begin{enumerate}
\item{There is never any reason to ever use 
fully concentrated (18~M) sulfuric acid.}
\item{For qualitative analysis, 5~M $\mbox{H}_{2}\mbox{SO}_{4}$ 
is sufficient for "concentrated sulfuric acid."}
\item{Do not buy 18~M sulfuric acid. 
Battery acid will suffice for qualitative analysis 
and is a much safer (if still quite dangerous) source of sulfuric acid.}
\item{If you already have 18~M sulfuric acid in your lab, just leave it. 
Battery acid is so cheap you can afford to get as much as you need.}
\end{enumerate}
\item{Hydrochloric acid, HCl}
\begin{enumerate}
\item{Hydrochloric acid is never required.}
\item{Do not buy concentrated hydrochloric acid. 
Use battery acid for all of its strong acid applications.}
\item{When you need the reducing properties of HCl, 
for the precipitation of sulfur from thiosulfate 
in kinetics experiments for example, 
make a solution with the proper molarity of chloride and $\mbox{H}^{+}$ 
by dissolving sodium chloride in battery acid and diluting with water.}
\end{enumerate}
\item{Nitric acid, $\mbox{HNO}_{3}$}
\begin{enumerate}
\item{The only time nitric acid is required 
is to dissolve certain carbonates in qualitative analysis. 
The first time you need nitric acid, 
prepare a large volume of dilute acid (e.g. 2.5~L) 
so that you do not need to handle the concentrated acid again.}
\item{If many schools share a single bottle of concentrated acid, 
they should dilute it at a central location and transport only the dilute acid.}
\item{Teach qualitative analysis of insoluble carbonates 
using copper, iron, or zinc carbonate -- 
these will dissolve in dilute sulfuric acid.}
\end{enumerate}
\end{enumerate}

%==============================================================================
\item{First Aid}
\begin{enumerate}

\item{Cuts}

\begin{enumerate}
\item{Immediately wash cuts with lots of water 
to minimize chemicals entering the blood stream.}
\item{Then wash with soap to kill any bacteria that may have entered the wound.}
\item{To stop bleeding, apply pressure to the cut and raise it above the heart. 
If the victim is unable to apply pressure him/herself, 
remember to put something (gloves, a plastic bag, etc.) 
between your skin and their blood.}
\item{If the cut is deep (might require stitches) seek medical attention. 
Make sure that the doctor sees how deep the wound really is -- 
you might do such a good job cleaning the cut 
that the doctor will not understand how serious it is.}
\end{enumerate}

\item{Eyes}

\begin{enumerate}
\item{If chemicals get in the eye, immediately wash with lots of water.}
\item{Keep washing for fifteen minutes.}
\item{Remind the victim that fifteen minutes is a short time 
compared to blindness for the rest of life. 
Even in the middle of a national exam.}
\end{enumerate}

\item{First and Second Degree Burns}

\begin{enumerate}
\item{Skin red or blistered but no black char.}
\item{Immediately apply water.}
\item{Continue to keep the damaged skin in contact with water for 5-15 minutes, 
depending on the severity of the burn.}
\end{enumerate}

\item{Third Degree Burns}

\begin{enumerate}
\item{Skin is charred; there may be no pain.}
\item{Do not apply water.}
\item{Do not apply oil.}
\item{Do not removed fused clothing.}
\item{Cover the burn with a clean cloth and go to a hospital.}
\item{Ensure that the victim drinks plenty of water (one or more liters) 
to prevent dehydration.}
\end{enumerate}

\item{Chemical Burns}

\begin{enumerate}
\item{Treat chemical burns by neutralizing the chemical.}
\item{For acid burns, immediately apply a dilute solution of a weak base 
(e.g. sodium hydrogen carbonate).}
\item{For base burns, immediately apply a dilute solution of a weak acid 
(e.g. citric acid, ethanoic acid). 
Have these solutions prepared and waiting in bottles in the lab.}
\end{enumerate}

\item{Ingestion}
\begin{enumerate}

\item{If a student ingests (eats or drinks) the following, induce vomiting.}
\begin{enumerate}
\item{Barium (chloride, hydroxide, or nitrate)}
\item{Lead (carbonate, chloride, nitrate, oxide)}
\item{Silver (nitrate)}
\item{Potassium hexacyanoferrate (ferr[i/o]cyanide)}
\item{Ammonium ethandioate (oxylate)}
\item{Anything with mercury (see list above), 
but mercury compounds should just never be used.}
\end{enumerate}

\item{To induce vomiting:}
\begin{enumerate}
\item{Have the student put fingers into his/her throat}
\item{Have the student drink a strong solution of salt water 
(use food salt, not lab chemicals)}
\end{enumerate}

\item{Do not induce vomiting if a student ingests any organic chemical, 
acid, base, or strong oxidizing agent.}
\begin{enumerate}
\item{These chemicals do most of their damage to the esophagus 
and the only thing worse than passing once is passing twice.}
\item{Organic chemicals may be aspirated into the lungs if vomited, 
causing a sometimes fatal pneumonia-like condition.}
\end{enumerate}

\end{enumerate}

\item{Fainting}
\begin{enumerate}
\item{If a student passes out (faints), feels dizzy, has a headache, etc., 
move him/her outside until fully recovered.}
\item{Check unconscious students for breath and a pulse.}
\item{Perform CPR if necessary and you know how.}
\item{Generally, these ailments suggest 
that harmful gases are present in the lab -- 
find out what is producing them and stop it. 
Kerosene stoves, for example, may emit enough fumes to have this effect.}
\item{See Sources of Heat in the Materials section for alternatives.}
\item{Chemicals reacting in drain pipes can also emit harmful gases. 
See Waste Disposal.}
\end{enumerate}
\item{Electrocution -- If someone is being electrocuted 
(their body is in contact with a live wire)}
\begin{enumerate}
\item{First disconnect the power source. 
Turn off the switch or disconnect the batteries.}
\item{If that is not possible, use a non-conducting object, 
like a wood stick or branch, to move them away from the source of electricity.}
\item{Unless there is a lot of water around, 
the sole of your shoe is non-conducting.}
\end{enumerate}
\item{Seizure}
\begin{enumerate}
\item{If a student experiences a seizure, 
move everything away from him/her 
and then let the body finish moving on its own.}
\end{enumerate}
\end{enumerate}
\item{Mouth pipetting}
\begin{enumerate}
\item{Never do it!}
\item{This is a dangerous activity 
prohibited in every modern science laboratory.}
\item{Use rubber pipette filling bulbs or plastic syringes.}
\item{For more explanation, see The Danger of Mouth Pipetting below.}
\end{enumerate}
\item{Be prepared}
\begin{enumerate}
\item{Set aside a bucket of water for first aid.}
\item{It should not be used for anything else.}
\item{Have materials to fight fires and know how to use them.}
\item{A bucket of sand will work for any lab fire, 
is available to every school, and can be used by anyone.}
\end{enumerate}
\item{Good habits}
\begin{enumerate}
\item{Hand Washing}
\begin{enumerate}
\item{Students should wash their hands every time they leave the lab.}
\item{Always have water and soap available, 
ideally in buckets on a desk near the door.}
\item{Even if students do not touch any chemicals when they are in the lab, 
they should still wash their hands. }
\end{enumerate}
\item{Clean all benches and chemicals}
\begin{enumerate}
\item{Stray chemicals and contaminated apparatus has the potential for danger.}
\item{Make sure students do not leave stray pieces of paper.}
\item{Ensure all students clean the apparatus they use immediately after use.}
\item{Have students to clean apparatus prior to use. 
It is not always possible to trust the students washed the apparatus 
after their last use.}
\end{enumerate}
\item{Tasting Chemicals}
\begin{enumerate}
\item{Students should never eat anything in the lab. Ever.}
\item{Barium nitrate looks just like sodium chloride. 
Lead carbonate looks like starch.}
\item{Do not bring food into the lab.}
\item{If you use domestic reagents 
(vinegar, salt, baking soda, etc.) in the lab, 
label them and leave them in the lab.}
\end{enumerate}
\item{Smelling Chemicals}
\begin{enumerate}
\item{If there is a reason to smell something, 
teach students how to waft the fumes towards their nose, 
carefully getting closer.}
\item{Many lab reagents -- 
ammonia, hydrochloric acid, nitric acid, ethanoic (acetic) acid -- 
can cause serious damage if inhaled directly.}
\end{enumerate}
\item{Keep bottles and other apparatus away from the edge of the table. 
Twenty centimeters is a good rule.}
\item{Cap reagent botles when they are not in use.}
\item{Do not do things you do not want your students to do. 
They are always watching, always learning.}
\end{enumerate}
\item{Dispose of wastes properly}
\begin{enumerate}
\item{See Lab Management: Waste Disposal}
\end{enumerate}
\end{enumerate}

\chapter{Dangerous Chemicals}
\label{cha:dangerchem}
%==============================================================================
\section{Chemicals that should never be used in a school}

%------------------------------------------------------------------------------
\subsection{Mercury and its compounds 
(e.g. Million's Reagent, Nestler's Reagent)}

\begin{itemize}

\item{Hazard: Toxic}

\item{Route: Ingestion of solutions and salts; 
inhalation of vapors from the liquid metal. 
Mercury has a very low vapor pressure, 
but the vapors that do form are quite poisonous – 
inhalation is therefore a significant hazard.}

\item{Use: Showing off to students, 
Million's reagent for biology (no longer used)}

\item{Alternatives: Use the biuret test to detect proteins 
(1M NaOH followed by 1\% CuSO4, a purple color is a positive result)}

\item{Precautions if it needs handling (e.g. broken thermometers): 
Wear gloves or plastic bags on the hands.}

\item{First aid: If ingested, induce vomiting at once. 
Administer activated charcoal. Seek medical attention.}

\item{Disposal: If you find mercury or its compounds, 
keep them in sealed in a bottle and locked away. 
Label the bottle very clearly “POISON, DO NOT OPEN, DO NOT DUMP” 
and also include a strong warming in the local language(s). 
If you spill liquid mercury, 
ask everyone to leave and apply powdered sulfur immediately. 
Put on gloves and tie a cloth on your face. 
Open windows to increase ventilation. 
Then use pieces of cardboard to gather the mercury back together 
so you can seal it in a bottle. 
Apply powdered sulfur to any mercury that cannot be reached – 
e.g. cracks in the floor.}

\end{itemize}

%------------------------------------------------------------------------------
\subsection{Benzene}

\begin{itemize}

\item{Proven carcinogen, toxic. 
A horrific and generally fatal form of cancer 
is associated with benzene exposure, 
with tumors appearing rapidly throughout body.}

\item{Route: Can be fatal if ingested, 
especially if aspirated into the lungs 
(e.g. if mouth pipetting); also passes through skin(!)}

\item{Use: Multi-purpose non-polar solvent. 
Less dense than water.}

\item{Alternative: kerosene}

\item{Precautions if it needs handling: Thick rubber gloves. 
It will pass rapidly through latex.}

\item{Disposal: If you find a bottle of benzene, 
leave it sealed and in a secure place with a stern warning label. 
If a bottle breaks, 
evacuate the room and return only wearing a cloth over your face 
and thick rubber gloves. 
Absorb the benzene with cardboard, cotton wool, 
saw dust, rice hulls, or flour, 
transfer the mass to a dry place outside, 
add a significant amount of kerosene and burn completely. 
Benzene will combust on its own, 
but you want to make sure it burns hot enough 
that none simply vaporizes without combusting.}

\end{itemize}

%------------------------------------------------------------------------------
\subsection{Tetrachloromethane (carbon tetrachloride)}

\begin{itemize}

\item{Hazard: Proven carcinogen. 
The chemical has killed students in both Tanzania and Kenya.}

\item{Route: Ingestion can be fatal. 
Will pass through skin. 
Inhalation of vapors is quite dangerous.}

\item{Use: Multi-purpose non-polar solvent. 
More dense than water.}

\item{Trichloromethane (chloroform) is 
another non-polar solvent more dense than water, 
though still dangerous (listed in category two, below). 
If the density is not important, use kerosene. 
If the solvent must not be flammable, 
consider a different experiment.}

\item{Precautions if it needs handling: Thick rubber gloves. 
It will pass rapidly through latex.}

\item{First Aid: Seek medical attention immediately. 
Ask a medical expert if you should induce vomiting 
(the chemical can kill if absorbed through the stomach, 
but also if aspirated into the lungs when vomiting)}

\item{Disposal: If you find a bottle of carbon tetrachloride, 
leave it sealed and in a secure place. 
If a bottle breaks, 
absorb the chemical with cotton wool 
and move the cotton to a place where it can off-gas 
away from people and other living things. 
Protect from rain and from leaching into the ground. 
Once the cotton is completely dry, 
douse with kerosene and burn it. 
Do not burn the chemical directly – 
it used to be used in some fire extinguishers. 
On heating, it decomposes to release poisonous gases.}

\end{itemize}

%------------------------------------------------------------------------------
\subsection{Other hazardous organic solvents}
The following chemicals have hazards similar to 
if less severe than benzene and tetrachloromethane. 
None should ever be used in a school. 
Leave them sealed in their bottles. 
If a bottle breaks, follow the instructions 
listed with tetrachloromethane.

\begin{itemize}

\item{Chlorobenzene}

\item{Dichloromethane}

\item{Trichloroethane}

\end{itemize}

%==============================================================================
\section{Dangerous chemicals that you might need to use}

%------------------------------------------------------------------------------
\subsection{Ammonia (ammonium hydroxide solution)}

\begin{itemize}

\item{Hazard: The liquid burns skin, the fumes burn lungs, 
and reaction with bleach or any combination of chloride and oxidizer 
can form toxic chloroamine fumes.}

\item{Use: Common bench reagent.}

\item{Alternative: For a simple weak base, 
use carbonate or hydrogen carbonate.}

\item{Precaution: Strongly prohibit mixing of different bench reagents. 
Neutralize waste completely before disposal. 
When pouring ammonia for distribution, 
wear cloth over your mouth and nose and work outside, upwind. 
To smell, waft carefully – never inhale ammonia directly from a bottle!}

\item{First Aid: In case of skin contact, 
wash with plenty of water followed by a dilute weak acid 
(vinegar or dilute citric acid) and more water. 
In case of eye contact, wash with water for at least ten minutes. 
If ingested, do NOT induce vomiting. 
In case of inhalation, move victim to fresh air. 
Seek medical attention if the victim does not recover quickly.}

\item{Disposal: Save unused solution for another day. 
If you must dispose of it, add to several liters of water 
and leave in an open bucket in the sun, 
away from people and animals. 
The ammonia will evaporate, leaving water behind. 
The process is finished when the bucket no longer smells like ammonia.}

\end{itemize}

%------------------------------------------------------------------------------
\subsection{Barium}

\begin{itemize}

\item{Hazard: Very poisonous if ingested in a soluble form 
(e.g. barium chloride, hydroxide, or nitrate). 
Note that barium carbonate will dissolve very quickly in stomach acid.}

\item{Use: Preparation of hydrogen peroxide, 
test for sulfates, flame tests.}

\item{Alternatives: hydrogen peroxide is often sold in pharmacies, 
sulfates may be confirmed with soluble lead salts (also poisonous!), 
and boron compounds (e.g. boric acid, borax) 
also produce a green flame color.} 

\item{Precautions: Distribute only in small quantities 
in bottles clearly labeled “POISON.” 
Also use the local word for poison, e.g. SUMU in Swahili. 
Collect all barium waste in a special container. 
This will require training students. 
Have a bottle of magnesium sulfate 
or sodium sulfate available for spills on skin or tables. 
Sodium sulfate can be prepared by neutralizing dilute sulfuric acid 
with sodium bicarbonate – err on the side of excess bicarbonate. 
See Sources of Chemicals for magnesium sulfate.}

\item{First Aid: If ingested, induce vomiting 
and administer activated charcoal if available. 
Go to the hospital. 
The material safety data sheet for barium chloride 
recommends use of sodium or magnesium sulfate under a doctor's direction. 
Chemically, this would precipitate barium sulfate, 
preventing absorption of the element. 
Magnesium sulfate is non-toxic, though will probably cause diarrhea.}

\item{Disposal: Collect unused solutions for another day. 
Collect all waste in a large container 
and add dilute sulfuric acid until precipitation stops. 
Pour off most of the liquid and treat it as dilute acid waste. 
Use the remaining liquid to send the precipitate 
to the bottom of a pit latrine.}

\end{itemize}

%------------------------------------------------------------------------------
\subsection{Chloroform (Trichloromethane)}

\begin{itemize}

\item{Hazard: used to render mammalian specimens unconscious; 
it has the same effect on humans. 
Also toxic in ingested. Passes through skin.}

\item{Use: Knocking out dissection specimens, 
sometimes as a specialty non-polar solvent.}

\item{Alternatives: Dissect dead specimens; 
use safer solvents.}

\item{Precautions: Work in a well-ventilated space, like outside. 
NEVER, EVER MOUTH PIPETTE!}

\item{First Aid: If ingested, go to the hospital. 
Do not induce vomiting unless instructed by a medical professional. 
If inhaled, immediately remove the victim to fresh air and sit 
(but not lie) him or her down in case of fainting. 
If the victim loses consciousness, go to the hospital. 
In both cases, monitor breathing and pulse. 
In case of skin contact, wash off immediately, 
and use soap as soon as it is available.}

\item{Disposal: For the small amounts used 
in preparing specimens for dissection, 
allow the chemical to evaporate away from people and animals. 
For large amounts, e.g. if a bottle spills or breaks, 
evacuate the room and keep everyone away for at least one day. 
Return carefully, allowing more time 
if the room still smells like trichloromethane.}

\end{itemize}

%------------------------------------------------------------------------------
\subsection{Concentrated Acids 
(sulfuric, hydrochloric, nitric, ethanoic (acetic))}

\begin{itemize}

\item{Hazard: Serious skin burns, will blind in the eyes.}

\item{Use: Often the starting material 
when preparing dilute acids for titrations or food tests. 
Also used directly in small quantities in chemical qualitative analysis.}

\item{Alternatives: If any acid will suffice, use a safer weak acid, 
e.g. citric acid (best) or ethanoic (acetic) acid. 
If a dilute strong acid is required, 
use battery acid as a starting source of sulfuric acid. 
Note that many experiments calling for dilute hydrochloric acid 
work just as well with dilute sulfuric acid. 
Battery acid will also work for many experiments 
calling for “concentrated” sulfuric acid – 
indeed it is about 5M – but will not work if one requires 
the dehydrating action of concentrated sulfuric acid. 
For such cases, consider other experiments. 
Note that battery acid is still quite dangerous – 
it will burn holes in clothing and blind in the eyes.}

\item{Precautions: Always have a full bucket of water 
and at least half a liter of sodium bicarbonate 
or other weak base solution available. 
Use thick rubber gloves and wear goggles. 
If you do not have these in your lab, go buy them. 
Whenever handling battery acid, wear goggles. 
Keep other people away when pouring concentrated acids. 
If you are using either concentrated hydrochloric, ethanoic (acetic), 
or nitric acid, work outside and stand upwind -- the fumes corrode the lungs. 
Wear cloth over your mouth and nose. 
If a bottle ever falls and breaks, 
calmly but clearly ask everyone to stop working and leave the room. 
Keep everyone upwind while the fumes blow away. 
Most of the acid will be consumed by reacting with cement. 
If the damage is significant, a building engineer should inspect the structure. 
Always pour acid into water when diluting. 
The heat of solvation of sulfuric acid especially is so exothermic 
that it can cause water to boil. 
If a small quantity of water is added to concentrated acid, 
it can boil so vigorously that it will cause acid to splash 
out of the container, on skin or into eyes. 
Finally, pour slowly from the bottle, 
always allowing air to enter as you pour. 
Otherwise, air will enter in sudden amounts, 
causing acid to exit the same way. 
This can cause it to splash back up at you.}

\item{First Aid: For skin burns, promptly wash the affected area 
with a large amount of water. 
Then liberally apply a sodium bicarbonate 
or other weak base solution to the affected area. 
Then wash again with a large amount of water. 
Repeat until the burning sensation is gone. 
If the chemical ever gets in the eye, 
immediately apply sodium hydrogen carbonate solution 
to neutralize the acid in the eye, but nothing stronger -- 
not carbonates and definitely not hydroxides. 
Then wash continuously with large amounts of water for ten minutes, 
longer if the eye still burns. 
Seek medical immediately. 
If swallowed, do not induce vomiting - 
the damage is done on the way down.}

\item{Disposal: Add the concentrated acid to twenty or more 
times its volume of water and then add ash or baking soda 
until the mixture stops fizzing. 
The gas produced is carbon dioxide. 
Note that containers used to measure or hold concentrated acids 
often have enough residual acid to be dangerous. 
They should be submerged in a large container of water following use.}

\end{itemize}

%------------------------------------------------------------------------------
\subsection{Diethyl Ether (ethoxyethane)}

\begin{itemize}

\item{Hazard: Can be fatal if aspirated into lungs. 
Also extremely flammable and a significant flash fire hazard. 
May also cause unconsciousness on inhalation.}

\item{Use: Non-polar solvent}

\item{Alternative: for a non-polar solvent, use kerosene. 
For a more volatile solvent, use paint thinner or lighter fluid. 
To demonstrate a rapidly evaporating substance, 
use propanone, ethyl ethanoate, or iso-propanol - 
note that all are also extremely flammable.}

\item{Precautions: Never use alone (in general, do not use the lab alone). 
Distribute in bottles with lids 
or in beakers covered with eg.g cardboard to prevent evaporation. 
Under no circumstances should an open container of diethyl ether 
be in the same room as open flame. 
Only use in well ventilated spaces and encourage students to go outside 
if they feel at all drowsy or unwell.}

\item{Disposal: See instructions on recycling of organic solvents 
to minimize the need for disposal. 
For what cannot be recovered, 
place where it can evaporate without being disturbed 
and without anyone downwind.}

\end{itemize}

%------------------------------------------------------------------------------
\subsection{Ethandioic acid (oxalic acid), 
sodium/ammonium ethandioate (oxalate)}

\begin{itemize}

\item{Hazard: Poison}

\item{Use: Volumetric analysis, 
oxidation-reduction reactions, qualitative analysis}

\item{Alternatives: For its weak acid properties, 
use citric acid (best) or ethanoic (acetic) acid. 
For its reducing properties, use ascorbic acid or sodium thiosulfate.}

\item{First aid: If ingested, induce vomiting 
and administer activated charcoal. Go to the hospital.}

\item{Disposal: Collect unused solutions for another day. 
To dispose, add potassium permanganate solution slowly 
until the ethandioic acid / ethandioate lacks the power to decolorize it. 
At this point the compound should have been fully converted to carbon dioxide. 
If you used far excess oxidizing agent, 
neutralize with ascorbic acid prior to disposal.}

\end{itemize}

%------------------------------------------------------------------------------
\subsection{Lead}

\begin{itemize}

\item{Hazard: Poisonous, toxic to many organs including the brain.}

\item{Use: Unknown salt for qualitative analysis. 
Thus students must treat ALL unknown salts as potential lead compounds.}

\item{Precautions: Unequivocally prohibit taste-testing of unknown salts. 
This seems obvious. 
Unfortunately, to many students it is not. 
Explain the hazard clearly -- 
there are salts in the lab (e.g. barium compounds) 
where even a small taste can kill. 
Also, make sure students wash their hands.}

\item{First Aid: If ingested, 
induce vomiting and administer activated charcoal.}

\item{Disposal: Collect unused solids for another day. 
If the salt is soluble, 
dissolve all waste in a large container 
and add sodium chloride solution until precipitation stops. 
Send the precipitate to the bottom of a pit latrine. 
If the salt is already insoluble, drop it down.}

\end{itemize}

%------------------------------------------------------------------------------
\subsection{Potassium hexacyanoferrate (potassium ferr[i/o]cyanide)}

\begin{itemize}

\item{Hazard: Reaction with concentrated acid releases hydrogen cyanide, 
the agent used in American gas chambers for executions. 
On inhalation, the cyanide enters the blood stream 
and binds cytochrome-c oxidase with a higher affinity than oxygen. 
Cellular respiration halts and tissues slowly die.}

\item{Use: Qualitative analysis bench reagent.}

\item{Precautions: Strongly prohibit mixing of different bench reagents. 
There are plenty of other dangerous combinations.}

\item{First Aid: If a student seems to have trouble breathing, 
bring him/her outside immediately. 
If breathing remains difficult, seek medical attention. 
If the chemical is ingested, induce vomiting.}

\item{Disposal: Dilute with plenty of water and send down the pipe. 
Make sure all acid waste is also diluted and neutralized.}

\end{itemize}

%------------------------------------------------------------------------------
\subsection{Sodium hydroxide (and potassium hydroxide)}

\begin{itemize}

\item{Hazard: Concentrated solutions corrode metal, 
blacken wood, and burn skin. 
Even solutions as dilute as 0.1~M can blind if they get in the eyes. 
Note that this is a common concentration for volumetric analysis. 
Also note that the dissolution of sodium and potassium hydroxide 
are highly exothermic -- rapid addition, especially to acidic solutions, 
can cause boiling and splatter. 
Finally, the salts are highly deliquescent 
and often turn to liquid if containers are not well sealed. 
This liquid is maximally concentrated hydroxide -- 
the most dangerous form; do not dispose without neutralization.}

\item{Use: Volumetric analysis, food tests, 
qualitative analysis bench reagent.}

\item{Precautions: Use weak bases (carbonates and hydrogen carbonates) 
for volumetric analysis, provide students with goggles.}

\item{First Aid: Treat spills and skin burns 
with a dilute solution of a weak base -- ethanoic (acetic) or citric acid. 
If it gets in the eyes, immediately wash with a large amount of water. 
Continue washing for at least five minutes 
and seek medical attention if the eye still hurts.}

\item{Disposal: Save for future use. 
To dispose, neutralize with citric acid or other acid waste and dump.}

\end{itemize}

%==============================================================================
\section{Chemicals that merit warning}

%------------------------------------------------------------------------------
\subsection{Ammonium nitrate}
Can explode (and shatter glassware, sending shards into eyes) if heated. 
Otherwise as innocuous as any other inorganic fertilizer.

%------------------------------------------------------------------------------
\subsection{Ethanol}
The vapors are flammable, 
so ethanol should never be heated directly on a stove. 
If it must be warmed, it should be heated in a hot water bath. 
If the ethanol ignites anyway, do not panic. 
Cover the top of the ethanol container and smother the flame. 
Please note that methylated spirits have chemical additives that are poisonous, 
causing blindness, etc. 
Also, alcohol prepared for laboratory or industrial use 
is sometimes purified by extraction with benzene 
and probably contains traces of this potent carcinogen. 
Do not even think about drinking it.

%------------------------------------------------------------------------------
\subsection{Ethyl acetate/ethyl ethanoate}

\begin{itemize}

\item{Hazard: Extremely flammable}

\item{Use: Solvent}

\item{Precautions: Never open a bottle in the same room as an open flame.}

\item{Disposal: Save for use as a solvent. 
If you must dispose, allow to evaporate away from people and fire.}

\end{itemize}

%------------------------------------------------------------------------------
\subsection{Potassium permanganate}
Powerful oxidizing agent. 
Do not mix with random substances. 
If you are trying something for the first time, use small quantities. 
Concentrated solutions and the crystals themselves will discolor skin, 
though the effect lasts only a few hours. 
This is the same stuff they sell in the pharmacies 
to prevent infections of cuts and surface wounds. 
Do not eat!

%------------------------------------------------------------------------------
\subsection{Propanone (acetone)}

\begin{itemize}

\item{Hazard: Extremely flammable}

\item{Use: Solvent}

\item{Precautions: Never open a bottle in the same room as an open flame.}

\item{Disposal: Save for use as a solvent. 
If you must dispose, allow to evaporate away from people and fire.}

\end{itemize}

\chapter{Dangerous Techniques}

Some common laboratory techniques are actually quite dangerous. 
Identify practices in your school 
that seem likely to cause harm and devise safer alternatives. 
Below are two examples of techniques often performed in the laboratory 
that can easily bring harm 
and alternative methods to do the same thing more safely.

\section{Mouth pipetting}

Many schools use pipettes for titrations. 
Many students use their mouths to fill these pipettes. 
We strongly discourage this practice.

The solutions used in ordinary acid-base titrations 
are not particularly dangerous. 
A little 0.1M NaOH in the mouth 
does not merit a trip to the hospital. 
Nevertheless, 
there are two pressing safety issues. 
First, 
there are often other solutions present on the same benches – 
the qualitative analysis test reagents for example – 
that can kill if consumed. 
It seems like it would be a rare event 
for a student to mix up the bottles, 
but in the panic of the exam anything is possible.

The second safety issue applies to the best students, 
those that continue on to more advanced levels. 
High level secondary and university students 
must measure volumes of the size fit for pipettes 
for chemicals that under no circumstances should be mouth pipetted. 
If a student is trained in mouth pipetting, 
she will continue with this habit in advanced level, 
especially in a moment of frustration 
when a pipette filling bulb seems defective, 
or if the school has not taught her how to use them, 
or if they are not supplied. 
Students have died in many countries from mouth pipetting toxins. 
Pipetting is another instance 
when doing something without the rubber is a bad plan.

Fortunately, 
there is no reason to ever use a pipette in secondary school, 
even if rubber-filling bulbs are present. 
Disposable plastic syringes are in every way superior 
to pipettes for the needs of students. 
First, 
they have no risk of chemical ingestion. 
Second, 
they are more accurate – 
plastic is much easier to make standard size than glass; 
the pipettes available general vary from their true volume, 
but all the syringes of the same model 
and maker are exactly the same volume. 
A third advantage is that plastic syringes are easier to use. 
Fourth, 
they are faster to use. 
Fifth, 
they are much more durable and, 
sixth, 
when they do break they make no dangerous shards. 
Last, 
and truly least, 
they are much less expensive, 
by about an order of magnitude. 
Schools all over are already substituting plastic syringes for glass pipettes.
For information on how to use these plastic syringes, 
please see How to Use a Plastic Syringe.

\section{Shaking separatory funnels}

Separatory funnels are useful for separating immiscible liquids. 
They are also made of glass, 
very smooth, 
and prone to slipping out of students' hands. 
The liquids often used in these funnels 
can be quite harmful and no one wants them 
splashed along with glass shards on the floor.

Much better is to add the mixture to a plastic water bottle, 
cap it tightly, 
and shake. 
After shaking, 
transfer the contents of the bottle into a narrow beaker. 
Either layer can be efficiently removed with a plastic syringe.

There are some cases where a separatory funnel remains essential. 
For secondary school, 
however, 
simply design experiments that use other equipment - 
and less harmful chemicals.

\section{Looking down into test tubes}
May blind.

\chapter{Classroom Management in the Laboratory}

In addition to the guidelines recommended 
in the Laboratory Safety section, 
we recommend the following strategies to keep lab work safe, 
productive, 
and efficient.

\section{Set lab rules}
Before the first practical of the year, 
hold a short session to teach lab rules and lab first aid. 
Try to set a few clear, 
basic rules – 
like the four proposed in the Laboratory Safety section – 
instead of a long list of rules. 
Post these rules in the lab, 
and be consistent and strict 
in enforcing them with students and teachers.

\section{Train students in basic techniques}
For students just beginning laboratory-based education, 
you can probably teach each specific skill 
one at a time as they come up in experiments. 
For more advanced students, 
especially when they have different backgrounds 
in terms of laboratory experience, 
it is wise to spend several sessions practicing basic techniques (e.g. 
titrations for chemistry, 
using the galvanometer for physics, 
etc).

\section{Have students copy the lab instructions before entering the lab}
Do not let them into the lab unless 
they can show you their copy of the procedure, 
etc. 
Have a class dedicated to explaining 
the practical activity before the actual session. 
Bring a demo apparatus into the classroom.

\section{Demonstrate procedures at the beginning}
Do not assume that students know how to use a syringe 
or measure an object with calipers. 
If there are many new procedures, 
hold a special session before the practical to teach them the procedures. 
For titration, for example, 
hold a practice session in using burettes 
and syringes with water and food coloring. 
For food tests, 
explain and demonstrate each step to the students 
before holding a practical. 
It will save you a lot of trouble during the actual practical.

\section{Have enough materials available}
Always prepare 25-50 percent more reagent 
than you think you will need. 
Also have spare apparatus in case they fail in use. 
For example with physics, 
have extra springs, 
resistors, 
weights, 
etc. 
That said; do not make all of what you prepare 
immediately available to the students. 
As with sugar and salt, 
an obvious surplus increases consumption. 
If there is a definite scarcity of resources, 
it may be necessary to distribute the exact volumes necessary 
to each student. 
If you are doing this, 
make sure students understand that there is no more. 
In an exam, 
you might take unique objects, 
such as ID cards, 
to ensure each student receives her/his allotment only once.

\section{Have enough bottles of reagent available}
Even if only a small quantity of a reagent is needed, 
divide it into several bottles and put a bottle on each bench. 
If the volume is sufficiently small, 
distribute the chemical in plastic syringes. 
Do not use syringes for concentrated acids or bases – 
because these chemicals can degrade the rubber in the syringe, 
there is a risk of the syringe jamming 
and the student chemicals squirting into eyes. 
The waiting caused by shared bottles leads 
to frustration and quarrels between groups. 
The last thing you want are students wandering around the lab 
and crowding to get chemicals. 

\section{Designate fetchers}
If students must share a single material source, 
designate students to fetch materials
If a reagent needs to be shared among many students, 
explain this at the beginning, 
and have them come to the front of the room to get it 
rather than carrying it to their benches. 
This will help to avoid arguments 
and confusion over where the reagent is. 
If the students are in groups, 
have each group appoint one student 
to be in charge of fetching that chemical. 
However, 
it is much better to have the reagent available 
for each group at their workplace.

\section{Teach students to clean up before they leave}
This will save you a lot of time in preparing 
and cleaning the lab—and it is just a good habit. 
Do not let students leave the lab until their glassware is 
clean and the bench is free of mystery salts and scraps of paper. 
If they do, 
consider not letting them in for the next practical. 
This might take assigned seats if you have many students. 
When they perform this clean up, 
make sure they follow whatever guidelines 
you have set for proper waste disposal.

\section{Allow more time than you think you will need}
What seems like a half hour experiment to you 
may take an hour for your students. 
Add fifteen minutes to a half hour more 
than you think will be necessary. 
If you finish early, 
you can have them clean up and then do a bonus demonstration.

\section{Know the laboratory policies at the school}
What is the policy on replacing broken equipment at the school? 
As a teacher, 
you need to know what you are going to do 
when the student drops an expensive piece of glassware. 
It is no fun to make up procedure while a student is in tears. 
What criteria will you use to determine if the student is “at fault?” 
Of course, 
this is less of an issue if you do not use glass apparatus.


\chapter{Routine Cleanup and Upkeep}
\label{cha:routineup}
Like gardens and children, 
laboratories require constant attention. 
The Second Law of Thermodynamics does not sleep. 
The following advice should keep you on the winning side 
of the struggle against entropy.

\section{Things to do immediately}
\begin{itemize}
\item{Remove broken glass from the floor. 
Use tools, 
like pieces of cardboard, 
not fingers!}
\item{Neutralize and wash up chemical spills}
\item{Replace chemical labels that have fallen off}
\end{itemize}
The person who made the mess should clean it up. 
Make sure they know how before they are in a position to make a mess. 
If they are unable (e.g. 
hurt), 
have someone else do it. 
Review the incident with everyone present focusing on 
how to prevent similar accidents in the future. 
Avoid blaming other people -- 
as the supervisor the accident is your fault; 
either you did not train someone well enough 
or your supervision of their behavior/technique was inadequate.

\section{Things to do right after every lab use}
\begin{itemize}
\item{Return stock containers of chemicals to the store area. 
Only teachers should move glass bottles of corrosive or toxic chemicals. 
Remember to carry these with two hands!}
\item{Transfer waste, 
including chemicals to be reused, 
into suitable storage containers}
\item{Return apparatus to their proper places}
\item{Put broken apparatus in a special place}
\item{Wash off all benches / tables}
\end{itemize}
The people who used the lab should do these things. 
If it is a lab class, 
the students should clean up the lab in that class period. 
If it is a group of teachers preparing experiments, 
the teachers should clean up their mess. 
Mess tends to grow with time, 
and no one wants to clean up someone else's mess.

\section{Things to do either right after lab use or later that same day}
\begin{itemize}
\item{Transfer chemicals to be reused into more permanent 
and well labeled storage containers.}
\item{Process all waste for disposal. 
See the instructions in \nameref{cha:wastedisp}}
\item{Remove all trash from the laboratory}
\end{itemize}
If done right after lab use, 
those who used the lab should do this work. 
If the work is done later 
anyone can take out the trash 
but waste should only be processed 
by someone who knows what (s)he is doing, 
and never working alone.

\section{Things to do every week}
\begin{itemize}
\item{Sweep and mop the floor. 
Note that this should be done with brooms and buckets of water, 
or long handled mops, 
not by pushing cloth on the floor directly with hands.}
\item{Wipe down the chemical storage area. 
Check for broken and leaking bottles.}
\item{Ensure that sinks (if present) are not clogged. 
If a sink is clogged, 
either unclog it immediately or prevent use of the sink 
by physically obstructing the basin and also writing a sign. 
Signs by themselves are often insufficient. 
Barriers with signs tend to get moved.}
\end{itemize}
You can do this work or you can train students to do it. 
Supervise their work while they are learning 
to make sure they use safe techniques. 
Ensure that students never work alone -- 
even for mopping at least two students must be present at all times. 
Students should not work in the chemical storage area 
without a teacher present.

\section{Things to do when you have time}
\begin{itemize}
\item{Unclog sinks}
\item{Repair broken apparatus}
\item{Rearrange materials -- 
make sure you plan enough time to finish the job!}
\end{itemize}
These are good projects to do together with students or other teachers.

\chapter{Waste Disposal}
\label{cha:wastedisp}
\section{Introduction to waste management}
Practical work produces chemical waste. 
Some of these wastes may be harmful to people, property or the environment 
if not properly treated before disposal. 
Regardless of where the waste will go -- 
a sink, a flower bed, a pit latrine -- 
the following procedures should always be followed.

Note, often there are unused reagents at the end of a practical. 
These are valuable and should be stored for use on another day. 
When storing left over reagents, label the container with:
\begin{enumerate}
\item{The name of the compound, e.g. "sodium hydroxide solution"}
\item{The concentration, e.g. 0.1~M}
\item{The date of preparation, e.g. 15 June 2010}
\item{Important hazard information, 
e.g. "CORROSIVE, neutralize spills with weak acid."}
\end{enumerate}

Sometimes, there are used reagents that may be recycled. 
Recycling of chemicals reduces harm to the environment and saves money. 
Examples of chemical recycling are:
\begin{itemize}
\item{Regenerating silver nitrate solution from qualitative analysis waste.}
\item{Purification for reuse of organic solvents from 
distribution/partition law waste}
\end{itemize}

In order to recycle these compounds, 
students must put their waste in designated containers. 
Specific instructions for chemical recycling follow in another section.

Some wastes may be discarded without worry. 
These solutions may be poured down a sink or into a pit latrine. 
These include:
\begin{itemize}
\item{The final mixture in the flask after a titration. 
This is neutral salt water.}
\item{All of the wastes from food tests in biology. 
Note that unused reagents are not waste!}
\end{itemize}

Finally, some wastes require special treatment. 
These wastes and their treatments follow.

\section{Special instructions for certain wastes}
\subsection{Organic wastes}
These are any substance that does not mix with water, 
for example kerosene, isobutanol, ether, chloroform, etc. 
These substances should be placed in an open container 
and left to evaporate down-wind from people and animals. 
Setting these wastes on fire is usually unnecessary and may be dangerous.

\subsection{Strong acids}
Sulfuric, hydrochloric, and nitric acid solutions 
will corrode sinks and pipes if not neutralized before disposal. 
These wastes should be collected in a special bucket during a practical. 
After the practical, bicarbonate of soda should be added 
until further addition no longer causes effervescence. 
The gas produced is carbon dioxide.

\subsection{Strong bases}
Sodium and potassium hydroxide solutions 
as well as concentrated ammonia solutions are also corrosive. 
These wastes should be collected 
in a different special bucket during a practical. 
After the practical, the waste should be colored with POP 
or a local indicator and acid waste should be added until the color changes. 
If there is more base waste than acid waste available to neutralize it, 
citric acid may be added until the color finally changes.

\subsection{Heavy metals}
Barium, lead, silver and mercury solution 
are highly damaging to the environment 
and may poison human or animal drinking water if disposed without treatment. 
Waste containing barium and lead, generally from qualitative analysis, 
should be collected in a special container during a practical. 
After the practical, dilute sulfuric acid should be added drop-wise 
until further addition no longer causes precipitation. 
At this point, soluble lead and barium will have been converted 
to insoluble lead sulfate and barium sulfate. 
These salts may then be disposed in a pit latrine. 
The waste should of course first be neutralized with bicarbonate of soda.

Waste containing silver should be collected in a different special container. 
Ideally, this waste will be treated to regenerate silver nitrate solution 
according to the instructions in the next section. 
If such recycling is infeasible, 
sodium chloride solution should be added drop-wise 
until further addition no longer causes precipitation. 
At this point, soluble silver will have been converted 
to insoluble silver chloride and may be disposed in a pit latrine.

There is no treatment for mercury solutions 
that may be safely performed in a secondary school. 
This fact combined with the extreme danger 
of using mercury compound in schools 
supports the recommendation that mercury compounds never be used. 
If mercury waste is ever discovered at the school, 
it should be placed in a well-sealed bottle labelled: 
MERCURY WASTE. 
TOXIC. 
DO NOT USE. 
MUST NOT ENTER THE ENVIRONMENT. 
SUMU KALI. 
USITUMIE NA USIMWAGE.

\subsection{Strong oxidizers}
Concentrated solutions of potassium permanganate, 
chromate, dichromate, hypochlorites (bleach), and chlorates 
should be reduced prior to disposal. 
Grind ascorbic acid (vitamin C) tablets to powder 
and add until the permanganate decolorizes, 
chromate and dichromate turn green or blue, and hypochlorites lose their smell. 
The resulting solutions may be safely disposed in a sink or pit latrine.

\subsection{Solid waste}
Solids clog pipes and should never be put into sinks. 
If the solid is soluble, dissolve it in excess waste 
and treat as solution waste. 
If the solid is insoluble, dispose into a pit latrine.

\subsection{Unknown compounds}
If you do not know what a compound is, 
you do not know what kind of treatment it requires prior to disposal. 
That solution that looks like water could be nitric acid, 
or mercury chloride solution. 
Before disposing of unknown compounds, 
please use the Guide to Identifying Unknown Chemicals in the appendix. 
Even if you cannot identify the compound with these instructions, 
you can use them to ensure that it is not dangerous to dispose.

\chapter{Recycling Silver Nitrate}
\label{cha:recyclesilver}
In many places, 
silver nitrate is the most expensive chemical 
used in a school laboratory. 
Silver nitrate is often used to confirm the presence of halide ions, 
which form insoluble precipitates with silver cations. 
The result of such tests are silver halide precipitates, 
themselves of little value.

To regenerate the silver nitrate from these silver halides 
you must first reduce the silver halides to silver metal 
and then dissolve the metal in nitric acid. 
This process is easiest and most efficient 
with a large amount of material, 
so consider accumulating silver waste in a central location 
for many terms and perhaps from many schools.

To reduce the silver halides, 
they must be in solution. 
Add aqueous ammonia solution to the silver halides until they dissolve. 
You have formed a soluble silver - ammonia complex. 
Add to the mixture a reducing agent. 
We have used both glucose and steel wool. 
Ascorbic acid, 
zinc metal, 
and sodium thiosulfate should in theory also work. 
Heat the mixture until a metallic silver precipitate forms. 
It is OK if the solution boils.

Once you believe all of the silver has precipitated as metal, 
decant the liquid, 
ideally filtering to separate all of the silver metal. 
Wash the silver metal in distilled (rain) water and filter again.

Before adding nitric acid, 
make sure that the silver is dry. 
Then, 
add concentrated nitric acid slowly. 
The goal is to dissolve most but not all of the silver metal. 
If you dissolve all of the metal, 
you may have residual nitric acid in your silver nitrate solution 
that will make it ineffective for many uses. 
Decant the solution into a dark bottle - 
silver nitrate decomposes in light - 
and save the residual silver metal for the next time you do this.

\chapter{Recycling organic waste}

Organic chemicals are often expensive 
to purchase and difficult to dispose. 
Every effort should be made to collect organic wastes and recycle them. 
For the purpose of this discussion, 
organic chemicals are liquids insoluble in water, 
e.g. 
kerosene, 
ether, 
ethyl ethanoate (ethyl acetate), 
etc.

Mixtures of multiple organic wastes 
require fractional distillation to separate. 
This is difficult and dangerous without the right equipment. 
Generally, 
if none of the organic chemicals in the mixture are particularly dangerous – 
see the section on Dangerous Chemical – 
it is best to label the mixture “mixed organic solvents, 
does not contain benzene or chlorinated hydrocarbons” 
and keep it for future use as a generic solvent or for solubility activities.

Mixtures of a single organic waste and water are inherently unstable, 
and given enough time will separate out into two layers. 
If the organic layer is on the bottom, 
it is probably di-, 
tri-, 
or tetrachloromethane, 
all dangerous chemicals. 
Follow the instructions in Dangerous Chemicals. 
If the organic layer is on the top, 
simply decant it off. 
You might do this in two steps – 
the first to separate only water from organic mixed with some water, 
and the second to separate from the latter fraction pure organic 
from a small volume that remains a mixture. 
Then the water can be discarded, 
the organic saved, 
and the small residual mixture left open to the air to evaporate.

Often, 
mixtures of organic and aqueous waste 
contain a solute dissolved in both solvents. 
The solvent is said to be distributed 
or partitioned between these two layers. 
Examples of compounds that partition between an organic 
and an aqueous layer are organic acids, 
like ethanoic acid (acetic acid) and succinic acid, 
and iodine when the aqueous layer is rich in iodide 
(usually potassium iodide). 
To reuse the organic layer it is necessary to first remove the solute.

If the solute is an organic acid, 
add a small amount of indicator to the mixture 
and then sodium hydroxide solution, 
shaking vigorously from time to time. 
The sodium hydroxide will react with the organic acid 
in the aqueous layer, 
converting it to the salt. 
As the concentration of the acid in the aqueous layer decreases, 
the distribution equilibrium will ``push'' acid dissolved 
in the organic layer into the aqueous layer, 
where it too is converted to salt. 
Eventually, 
all the organic acid will be converted to its conjugate base salt, 
which is only soluble in the aqueous layer, 
and the indicator will show that the aqueous layer 
is alkaline even after much shaking. 
Now the organic layer may be run off as above.

If the solute is iodine, 
the organic layer should have a color due to the iodine, 
and thus it will be straightforward 
to know when the iodine is fully removed. 
If there is no color, 
add starch to give a black color to the aqueous layer. 
Then add ascorbic acid (crushed vitamin C tablets) 
to the mixture and shake vigorously 
until either the organic layer returns to its normal color 
or the starch-blackened aqueous layer turns colorless. 
At this point all of the iodine will have been reduced to iodide, 
soluble only in the aqueous layer. 
The clean organic layer may be run off as above. 
Sodium thiosulfate may be used instead of ascorbic acid.

If you require the final organic to be of quite high purity, 
repeat the treatment. 
A small amount of residual water 
may also be removed with use of a drying agent, 
such as anhydrous sodium sulfate or calcium chloride.


\chapter{Industrial Ecology in the Laboratory}

\textit{Industrial Ecology} is a manufacturing design philosophy 
where the byproducts of one industrial operation 
are used as input material for another. 
The philosophy may be applied to a school laboratory 
with similar economic and environmental benefits.

The science teacher generally plans the term in advance, 
and thus has a good understanding of the experiments students will perform. 
Each experiment has input reagents and output products. 
Normally, 
each of these inputs has to be purchased, 
sometimes at great expense, 
and each of these outputs has to be disposed of properly. 
When the term is analyzed in aggregate, 
however, 
there should be many occasions where the outputs of one experiment 
may serve as the inputs for another.

For example, 
students learning about exothermic reactions might 
dissolve sodium hydroxide in water and measure the temperature increase. 
The students might then use this solution 
of sodium hydroxide for a titration against a solution of ethanoic acid. 
The product of this titration will be 
perfectly balanced sodium ethanoate solution, 
which may be used in qualitative analysis for detecting iron (III) salts.

The maximize the opportunities for such pairings of inputs and outputs, 
the teacher should identify the reagents and byproducts of 
all activities planned for the term. 
Teachers may even coordinate between subjects - 
the reaction between citric acid and sodium carbonate 
to make carbon dioxide in chemistry class 
produces a sodium citrate solution that may be used 
to prepare Benedict's solution for biology class.


% Part 3 - Lab Techniques
\part{Laboratory Techniques}

\chapter{Use of the Beam Balance}

\section{Measuring Mass}

A common tool for measuring mass is the triple beam balance. The name comes from the three parallel beams holding sliding weights, labeled in the diagram below. On one side of pivot point there is either a flat metal surface or a boom suspending a weighing tray. On the other side of the fulcrum, one the three parallel beams, are weights that the user slides closer or further to the pivot point. At the far end of the three beams is some kind of level indicator showing when the balance is in equilibrium or, if not, which side is too heavy.

\section{Calibration}

Calibrate the balance prior to use. Move all the sliding masses as far as they go towards the pivot point – the zero mass mark. There are usually small groves that the sliders will fit snugly in. Make sure they are in those groves – each slider except for the smallest should “click” into place. Take off any weight on the weighing tray and clean it completely. Look at the level indicator. There are two pieces. The right side not moving, but the left side of the level will move on addition of mass. The level shows the balance is calibrated when the level forms an unbroken horizontal line. If the balance is not level, there usually is a massive screw or a dial under the weigh pan. Turn it until the balance becomes level.

\section{Weighing Samples}

Triple beam balances are very accurate at measuring masses if used properly. Do not measure the chemicals directly on the metal weighing tray; use a piece of paper or glass. Many samples will react with the metal, permanently altering its mass and ruining the balance. Because the paper of glass you put the chemical on has mass, before adding any chemical you must weigh the paper or the glass first by itself. To weigh properly, move the sliders slowly until the balance becomes level or makes a horizontal line. Start with the smallest. If you reach the end before the balance equalizes, return the mass to its zero and start moving the next larger mass, one stop at a time. When the balance “tips,” move back one notch and again move the smallest slider until the balance is level. Record this mass by adding each of the slides together. The mass should be recorded to one decimal place beyond the units of the smallest lines on the balance. For example, if the lines each represent 0.1g, estimate the position of the slider to the nearest 0.01~g.

Sum the desired chemical mass with mass of the paper or glass you just measured. Move the sliders to this total mass. Now, slowly add the chemical onto the paper or glass until the beam balance becomes level. After weighing, transfer the chemical from the glass or paper into whatever will actually hold it. If you use a glass and plan for the sample to be dissolved, rinse the glass into your solution container to get every last bit of chemical into your solution. If you spill any chemical on the balance, clean it up immediately.

\section{Simplified Procedure}

\begin{itemize}

\item{Clean and calibrate balance}
\item{Use some paper or glass and move the sliders till level}
\item{Sum the mass of desired chemicals to the mass of the paper or watch glass.}
\item{Add the chemical until balance is level.}
\item{Transfer chemical to receiving container.}
\item{Clean up any spills}

\end{itemize}

\section{Other Important Tips}

Many times, you will need to measure small masses, less than 5 grams. Unfortunately, the beam balance is not as accurate when measuring such small masses, as movements in the air can cause the balance to err. To overcome this problem, place an additional mass on the weighing tray along with the paper so that the effective mass is much larger. If you are using a glass container, this step is probably unnecessary. If you add another object to the tray, make sure that there is enough space still for your chemical!

Wind is another difficulty – find a place to weigh where the air is still, perhaps in a closed room or behind some sort of obstacle or screen.

If you need many samples each the same weight, use papers of identical size and therefore mass. This allows you to keep the sliding masses in the same place for each weighing.

If you are measuring a deliquescent chemical – one that takes in water from the air, e.g. sodium hydroxide, iron (III) chloride, etc – work efficiently, but remain careful not to spill. Close the stock chemical bottle as soon as possible after use. Measure the chemical on glass rather than paper if possible as the paper often absorbs the solution that forms as the chemical deliquesces.

Finally, make sure that the volume of substance you are measuring will physically fit on your paper or glass. For volumes greater than 20g of most substances, consider using a beaker or plastic container. For volumes 100g or greater, you almost certainly need a wide mouthed and high walled vessel to hold it all. Look at the volume of substance in a contain of known mass to have an idea of how much space your sample will occupy.

\chapter{Use of a Plastic Syringe}

\section{Safety First}

\paragraph{Syringes are probably the best means of transferring specific volumes. They are also very safe – if used correctly. First, many syringes come with sterile needles in the same package. If this is the case, open the packages yourself and collect the needles. Never provide students with both syringes and needles. Syringe needles are designed to inject compounds into the bloodstream. Many laboratory chemicals can be very toxic if injected into the blood, and any injection done improperly carries significant risk of serious infection. Laboratory syringes should be used without the needles. If you decide to keep the syringe needles for tools (e.g. optical pins, dissection pins), store them in a well labeled container. If you decide to not keep the syringe needles, dispose of them in a sharps bin at a health center (best) or in a pit latrine.}

\paragraph{Laboratory syringes should never be used for anything other than work in the laboratory. They should never leave the laboratory. Do not let students play with the syringes like squirt guns or point them at students’ eyes even when empty. The mantra for all gun users – treat every gun like a loaded gun – should apply to syringes. They should be held with the nozzle pointed down.}

\paragraph{Anyone working with organic solvents or concentrated acids/bases should wear goggles, whether or not syringes are involved.}

\section{Measuring Volume}

There are two methods ways to use a syringe. The second is superior.

\subsection{Direct Measure}
Place the syringe in the solution you want to measure. Push the plunger completely in to remove all air. Draw the plunger back past the proper volume is measured. Use the front of the rubber plunger to read the volume measured. Slowly push the plunger in until the rubber reaches the desired volume. Remove the syringe from the liquid being measure and transfer the liquid to the desired receptacle.

This method is a poor way to use the syringe. First of all, it is difficult to remove all the air bubbles from syringes. You will push the plunger in and out many times and still not be free of the bubbles. Often students turn the syringe upside-down and try pushing the bubbles up and out. While this effectively removes air, this method is likely to eject chemicals out into a student’s eye. 

In addition, using the rubber stopper to measure is surprisingly difficult. It is hard to see the volume markings, and the curvature of the rubber can cause confusion. Also, the refractive index of water is different than air, introducing additional error. Finally, if this method is used to measure organic solvents or concentrated solvents, these chemicals will react with the rubber in the syringe. This will make the rubber sticky and difficult to draw in and out. This makes the likelihood of an accident even higher. Therefore, we do not recommend this method for measuring volume with a syringe.

\subsection{Air Bubble Method}
Before putting the syringe into the solution you want to measure, draw back the plunger so it hold about 1 mL of air. Now put the syringe in the solution. Draw the plunger back beyond the desired volume. This time, there will be a large air bubble between the rubber and the top of the solution. Hold the syringe about the liquid being measure and push down the plunger until the top of the liquid inside is at the desired volume. Make sure that the top level of the liquid is level with your eye to prevent parallax error. Hold the container of liquid up so liquid exiting the syringe does not fall a long distance and splatter. Transfer the measured volume to its receptacle. 

This method is the preferred manner of using a syringe. The air bubble allows for easier and more exact volume measurements. In addition, this method can be used with concentrated chemicals and organic solvents. The air bubble does not allow these chemicals to come in contact with the rubber, at least on the initial measure. The rubber will start to react with the residue, and without prompt cleaning this can destroy the syringe.
 
\section{Cleaning Syringes After Use}

Like all lab equipment, syringes need to be cleaned after use. Fill a beaker or other open mouth container with water. Draw water into the syringe and push it out. Repeat 2 or 3 times. If you used the syringe to measure an organic solvent, wash the syringe thoroughly in soapy water and then rinse in ordinary water until all the soap is removed.

\chapter{Measures of Concentration}

\section{Molarity (M)}

Molarity is the number of moles of substance per liter of SOLUTION. Note that molarity is not the number of moles of substance per liter of solvent (e.g. water), although practically these are very similar. A molar solution has a concentration of 1~M.

\section{Density and percent purity}

These measurements are used to find the concentration of stock acid solutions. The acid bottle should list two pieces of information: the density of the acid in $^g/_{cm^3}$ or $^kg/_{dm^3}$, and the percent purity of the acid. The percent purity tells you what portion of the density is due to the acid itself, and what portion is due to water or impurities. See the chapter on Calculating the Molarity of Bottled Reagents to see how this information is used to find molarity.

\section{Percent by mass}

The percent by mass of a solute (\% or $^w/_w$ or $^m/_m$) is the grams of the solute in 100~g of solution. Now, for most practicals, solutions do not need to be very precise. Thus it is acceptable to let the percent by mass just be the grams of solute in 100~ml of water. This makes these solutions much faster to prepare.

Such approximation may not suffice for more advanced work. Consider a 1\% by mass solution of copper (II) sulfate. This solution should contain 1~g of $\mathrm{CuSO}_4$ in 100~g of solution. This means that the mass of water is $100 g - 1 g = 99 g$. By assuming that the density of water before adding the solute is $1 ^g/_{mL}$, we find that 99~mL of water must be combined with 1~g of CuSO4 to make the solution. This difference matters if you are making, say, a solution of iron sulfate on which students will perform a redox titration. 

\section{Percent by volume (\% or $^v/_v$)}

Percent by volume is used to measure concentration for a mixture of a liquid chemical and water. It is equal to the volume of the chemical divided by the volume of the solution.

Example: What volume of pure ethanol must be used to make 500~mL of a 70\% ethanol solution?

70\% ethanol means 70~mL ethanol per 100~mL of solution. Thus, the required volume is:

\[ \mathrm{volume of pure ethanol} = \mathrm{total solution volume} \times \mathrm{desired fraction ethanol} \]
\[ V = 500ml \times 0.70 \]
\[ V = 350 mL \] 

\section{Normality (N)}

The normality of the solution is closely related to the molarity. For many solutions, the normality IS the molarity. Normality is generally used in older books to refer to acid and base solutions. Technically, it is the "moles of equivalent" per liter. So for an acid solution, it is the moles of $\mathrm{H}^{+}$ per liter of solution. For a base solution, it is the moles of $\mathrm{H}^{+}$ that may be neutralized per liter of solution. For example, 1~M HCl has one mole of $\mathrm{H}^{+}$ per liter of solution. Thus 1~M HCl is also 1~N. However, 1~M H2SO4 provides TWO moles of $\mathrm{H}^{+}$ per liter of solution, so 1~M $\mathrm{H}_2\mathrm{SO}_4$ is 2~N. In a similar vein, 1~M NaOH is 1~N, but 1~M $\mathrm{Na}_2\mathrm{CO}_3$ is 2~N.

\section{Molality}

MolaRity is the number of moles of solute per liter of solution. MolaLity is the number of moles of solute per kiLogram of SOLVENT. In dilute aqueous solutions, the molarity and the molality are almost the same.

\section{Some Notes on Calculations}

Many textbooks and student notebooks transcribed from them feature equations that range between novel and obtuse to the American eye. Here are two very common equations that you should be aware of, mostly because the teachers that mark exams expect students to use them.

First off, the equation that often defined molarity as $M = \frac{concentration}{molecular mass}$. That is, molarity is equal to the concentration in grams per liter divided by the molecular mass of the solute (in grams per mole).

Second, the central equation for titration calculations: 

\[ \frac{(M_A)(V_A)}{(M_B)(V_B)} = \frac{n_A}{n_B} \]

A refers to the acid, B to the base, M to molarity, V to volume, and N to the stoichiometric coefficient of the acid/base in the reaction equation.

This said, there is a strong case to be made for teaching students equations that rely on an understanding of moles rather than encouraging them to memorize antiquated methods. The above equations essentially try to circumvent the need to think about moles. If you are teaching ordinary level, teach your students moles, and then show how the molarity and titrations equations come about from this unifying concept. If students can reduce every quantitative problem to moles, they will have a better understanding of the manipulations they are performing.

\chapter{Calculating the Molarity of Bottled Liquids}

You need three pieces of information to perform this calculation:

\begin{enumerate}

\item{The molecular mass of the acid. This is usually written on the bottle and can be easily calculated if it is not. For concentrated acids: sulfuric acid is 98 g/mol, hydrochloric acid is 36.5 g/mol, ethanoic acid is 60 g/mol, and nitric acid is 63 g/mol}

\item{The percent purity of the compound. This might be expressed as a percent (e.g. 31\% HCl), with the symbol $ ^m/_m $ (e.g. $ ^m/_m = 68\% $), or with the word purity ("98\% pure"). If you cannot find this information, see the note at the end.}

\item{The density ($\rho$) or specific gravity (s.g.) of the acid. This should be in grams per cubic centimeter (cc or $\mathrm{cm}^3$).}

\end{enumerate}

Then, you can calculate the molarity of your concentrated acid with this formula:

\[ \mathrm{molarity} = M = \frac{(\mathrm{percent purity})(\mathrm{density})(1000 \frac{cm^3}{L})}{\mathrm{molecular mass}} \]

For example, the molarity of an acid bottle labeled “H2SO4, 98\%, s.g. 1.84” we would calculate:

\[ \mathrm{molarity} = M = \frac{(0.98)(1.84 \frac{g}{cm^3})(1000 \frac{cm^3}{L})}{98 \frac{g}{mol} }\]

Note that we used 0.98 for 98\%. Convert all percents to decimals.

Once you do this work, take out a permanent pen and label your stock bottle with its molarity. Then no one needs to do this calculation again.

Note: since you will correct the concentration of your solutions with relative standardization, you really just need to know the approximate molarity of your liquid stock reagent. For new bottles of concentrated acid, you may assume that sulfuric acid is about 18~M, hydrochloric acid is about 12~M, and that both nitric acid and ethanoic (acetic) acid are about 16~M. Battery acid should be 4.5~M $ \mathrm{H}_2 \mathrm{SO}_4 $.

\chapter{Preparation of Solutions}

For many exercises, solutions do not need to be prepared accurately. Event a 50\% error in the preparation will still allow an effective experiment. For other activities, the solutions should be prepared with a great deal of accuracy. This is especially true for volumetric analysis and conductivity experiments. This section deals with the preparation of solutions when accuracy counts.

%==============================================================================
\section{Measure the Water}

\begin{itemize}

\item{Calculate the total volume of solution you need to prepare. For example, if you are doing a practical with 100 students and each requires 150~mL of solution you should make at least 15~L of solution. Making 20~L is probably wise, to have some extra.}

\item{Find a container large enough for the total volume. Plan ahead to ensure you have a large enough container.}

\item{Add the required volume of ordinary water.}

\item{If your syllabus encourages you to often practice acid-base titrations, designate a pair of suitably large buckets as your permanent ACID and BASE buckets and label them as such with a permanent pen. Then, use a 1 liter container to add water to these buckets, one liter at a time. Use the permanent pen to mark the water height after each liter. Use these marks when adding water to make solutions. Round up the volume you need to the nearest liter (e.g. 71 students * 200 mL per student = 14.2 L, so make 15 L). As long as you use relative standardization when you finish preparing the solutions, any errors you make when measuring the volume will not affect your students' results.}

\item{Distilled water is rarely necessary. If you are preparing solutions for volumetric analysis, read the section on Relative Standardization to learn how to correct small errors caused by the composition of the tap or river water. If the water forms a precipitate when making solutions of hydroxide or carbonate, allow the precipitate to settle and decant the solution. If you are making a dilute solution, you might add hydroxide or carbonate gradually with mixing until precipitation stops and then add the amount you actually need to the liquid after decantation. If the only water supply if muddy, let the dirt settle and decant or use a cloth filter. If the particles are very fine, add a chemical like potassium aluminum sulfate (alum) or iron sulfate to precipitate the dirt.//
//
If you think that you do need distilled water, rain water is almost always sufficient.}
\end{itemize}

What comes next depends on the nature of your stock chemical. In general, there are two kinds of solutions:
\begin{itemize}
\item{Solutions prepared from solid stock chemicals, e.g. sodium hydroxide, citric acid}
\item{Solutions prepared from liquid stock chemicals, e.g. sulfuric acid}
\end{itemize}

%==============================================================================
\section{Preparing solutions from solid stock chemicals}

\begin{itemize}

\item{Calculate the amount of solid chemical required. If the instructions give the required concentration in grams per liter (e.g. $ 4 ^g/_L $ NaOH solution), multiple the total volume by the required concentration (e.g. $ 4 ^g/_L \times 10 L = 40 g $). If the instructions give the required concentration in molarity or moles per liter (e.g. 0.1 M~NaOH solution), multiple the required molarity by the molecular mass of the compound to find the required concentration in grams per liter (e.g. $ 0.1 ^{mol}/_L \times 40 ^g/_{mol} = 4 ^g/_L $). Then, multiple the required concentration by the total volume ($ 4 ^g/_L \times 10 L = 40g $).}

\item{Use a balance to weigh the solid chemical. Remember to weigh the chemical in a plastic container or on a sheet of paper and not on the scale pan directly. Some chemicals (e.g. sodium hydroxide) react with the metal pan. If you are unfamiliar with how to use a balance, read How to Use a Beam Balance. If you do not have a balance, read the section on Preparation Solutions without a Balance.}

\item{Carefully add the solid chemical to the water and stir with something unreactive (e.g. glass rod, broken burette, thick copper wire) until it has completely dissolved.}

\end{itemize}

%==============================================================================
\section{Preparing solutions from liquid stock solutions}

\begin{itemize}

\item{Calculate the amount of liquid chemical required. To do this, you need to know the molarity of your stock chemical. See the section on Calculating the Molarity of Bottled Liquids. If the instructions give the required concentration in molarity or moles per liter, use the dilution equation to calculate the amount of concentrated required:

\[ (M_{concentrated})(V_{concentrated}) = (M_{dilute})(V_{dilute}) \]

rearranging

\[ V_{dilute} = \frac{M_{concentrated})(V_{concentrated}}{M_{dilute}} \]

For example, if you need 10~L of 0.1~M HCl and you have 12~M stock solution, the required volume of concentrated acid is

\[ V_{dilute} = \frac{(12 M)(10 L}{0.1 M} \]

}

\item{If the instructions give the required concentration in grams per liter, divide this concentration by the molecular mass to get molarity (e.g. $ \frac {3.65 ^g/_L}{36.5 ^g/_{mol}} = 0.1 ^{mol}/_L $) and then use the dilution equation as above.}

\item{Use a DRY measuring cylinder the measure the required amount of liquid chemical. Concentrated acids may be measured in standard lab grade plastic measuring cylinders – there is no need for glass. If you do not have a measuring cylinder, you can use a plastic syringe. Be sure to use the Air Cushion Method for measuring volumes with syringes (see the section on How to Use a Plastic Syringe) – concentrated acids will rapidly corrode the rubber in the syringe on contact, causing the syringe to jam and become dangerous. Also, please read the description of Concentrated Acids in Dangerous Chemicals.}

\item{Carefully pour the liquid chemical into the container of water. Stir with something non reactive (glass rod, broken burette, thick copper wire) for about one minute.}

\end{itemize}

Then, for all volumetric analysis solutions, use the instructions in the Relative Standardization section to perfect the mole ratio of your solutions.

\chapter{Volumetric Analysis Theory}

Most examples of volumetric analysis involve acid-base reactions, so first is a bit of acid-base theory.

\section{Acids, Bases, and pH}

The Bronsted-Lowery definition of an acid is a substance that provides $\mathrm{H}^{+}$ to a solution while a base is a substance that removes $\mathrm{H}^{+}$ from a solution.

It is important to remember that in a water solution, $\mathrm{H}^{+}$ does not exist. Rather, $\mathrm{H}^{+}$ binds with water to form the hydronium ion, $ \mathrm{H}_3 \mathrm{O}^{+} $ .

\[ \mathrm{H}^{+} + \mathrm{H}_2 \mathrm{O} \longrightarrow \mathrm{H}_3 \mathrm{O}^{+} \]

pH is defined as the power of the hydronium ion concentration. To find the pH of a solution:

\[ \mathrm{pH} = \log{[\mathrm{H}^{+}]_{aq}} \]

Pure water has $ 10^{7} $ moles of $\mathrm{H}_3 \mathrm{O}^{+}$ per liter, or $ \mathrm{pH} = 7 $. This is because some water molecules are always reversibly reacting with each other to form hydronium and hydroxide:

\[ 2\mathrm{H}_2\mathrm{O} \longleftrightarrow \mathrm{H}_3 \mathrm{O}^{+} + \mathrm{OH}^{-} \]

Acids increase the amount of $\mathrm{H}_3 \mathrm{O}^{+}$. By increasing the concentration of hydronium ion, the power of the concentration increases to a less negative number, and thus the solution will have a smaller pH. Bases decrease the amount of H3O+ and thus basic (alkaline) solutions have pH greater than 7.

\section{Types of Acids and Bases}

\subsection{Strong Acids}

Strong acids are acids that dissociate completely to provide $\mathrm{H}^{+}$. One can approximate the molarity of $\mathrm{H}^{+}$ (or $\mathrm{H}_3 \mathrm{O}^{+}$) as the molarity of the acid. For example, a solution of 1~M HCl has one mole of $\mathrm{H}_3 \mathrm{O}^{+}$ per liter of solution (pH 0); most of the molecules of HCl have dissociated and the $\mathrm{H}^{+}$ has reacted with water to form $\mathrm{H}_3\mathrm{O}^{+}$.

\[ \mathrm{HCl} + \mathrm{H}_2\mathrm{O} \longrightarrow \mathrm{H}_3 \mathrm{O}^{+} + \mathrm{Cl}^{-} \]

The most common strong acids are sulfuric acid ($\mathrm{H}_2\mathrm{SO}_4$), hydrochloric acid (HCl), and nitic acid ($\mathrm{HNO}_3$).

\subsection{Weak Acids}

Weak acids, however, are reticent to contribute $\mathrm{H}^{+}$ to solution. For example, in a solution of ethanoic acid, an equilibrium forms where only one in 250 ethanoic acid molecules dissociates to form $\mathrm{H}_3 \mathrm{O}^{+}$.

\[ \mathrm{CH}_3\mathrm{COOH} + \mathrm{H}_2\mathrm{O} \longleftrightarrow \mathrm{H}_3 \mathrm{O}^{+} + \mathrm{CH}_3\mathrm{COO}^{-} \]

The most common weak acids are ethanoic acid or acetic acid ($\mathrm{CH}_3\mathrm{COOH}$), ethandioic acid or oxalic acid ($\mathrm{C}_2\mathrm{H}_2\mathrm{O}_4$), and citric acid ($\mathrm{COOHCH}_2\mathrm{COH(COOH)CH}_2\mathrm{COOH}$).

One mole of hydrochloric acid and one mole of ethanoic acid both require the same amount of base for neutralization. The difference is how the pH of the solution changes during the titration. When hydrochloric acid is titrated, the pH remains very low until right before the endpoint when it jumps to alkaline. When ethanoic acid is titrated, the pH gradually rises through a range of acidic pH's and then jumps at the endpoint. This is why methyl orange cannot be used for titrations with weak acids – see Properties and Preparation of Indicators.

\subsection{Strong Bases}

Strong bases are bases that either dissociate completely in solution to form $\mathrm{OH}^{-}$ which reacts to remove $\mathrm{H}_3\mathrm{O}^{+}$. The common strong are sodium hydroxide, NaOH, and potassium hydroxide, KOH.

\subsection{Weak Bases}

Weak bases form an equilibrium with water where only a few of the molecules react to remove $\mathrm{H}_3\mathrm{O}^{+}$. Common weak bases include ammonia (ammonium hydroxide), soluble carbonates, $\mathrm{CO}_3^{2-}$ and all hydrogen carbonates, $\mathrm{HCO}_3^{-}$.

Much like strong and weak acids, both strong and weak bases readily react with acids to neutralize them. As with acids, weak bases will form a buffered solution that changes pH gradually whereas strong bases will change pH abruptly when the base is neutralized fully.

\section{Volumetric Analysis}

Volumetric Analysis is a method to find the concentration (molarity) of a solution of a known chemical by comparing it with the known concentration of a solution of another chemical known to react with the first.

For example, to find the concentration of a solution of citric acid, one might use a 0.1~M solution of sodium hydroxide because sodium hydroxide is known to react with citric acid.

The most common kinds of volumetric analysis are for acid-base reactions and oxidation-reduction reactions. Acid-base reactions require use of an indicator, a chemical that changes color at a known pH. Some oxidation-reduction reactions require an indicator, often starch solution, although many are self-indicating, that is one of the chemicals itself has a color. For more about indicators, read Properties and Preparation of Indicators. For more on the specific technique of volumetric analysis, read Traditional Volumetric Analysis Technique if you have burettes and Volumetric Analysis Without Burettes if you do not.

The process of volumetric analysis is often called \textit{titration.}

\chapter{Properties of Indicators}

\section{Acid-base indicators}
These indicators are chemicals that change colors in a specific pH range, which makes them suited to use in acid-base reactions. When the pH of changes from low pH to high pH or from high to low, the color of the solution changes. 

Four common acid-base indicators are methyl orange (MO), phenolphthalein (POP), bromothymol blue (BB), and universal indicator (U)

\begin{itemize}

\item{Methyl Orange, MO, is always used when titrating a strong acid against a weak base. The pH range of MO is 4.0-6.0 and thus no color change is observed until the base is completely neutralized. If you use MO with a weak acid, the color might start to change before completely neutralizing the acid.}

\item{Phenolphthalein, POP, is always used when titrating a weak acid against a strong base. The pH range of POP is 8.3-10.0, and thus no color change is observed until the weak acid is completely neutralized. If you use POP with a weak base, the color might start to change before completely neutralizing the base.}

\item{Bromothymol Blue, BB, is used in the same manner as methyl orange.}

\item{Universal indicator, U, is not suitable for volumetric analysis involving either weak acids or bases as it changes color continuously rather than in a limited pH range. It is very useful for tracking the pH continuously over a titration, perhaps by performing two titrations side by side, one with a standard indicator and another with universal indicator.}

\end{itemize}

Any indicator can be used when titrating a strong acid against a strong base. Universal indicator, however, will not produce very accurate results.

No indicator is suitable for titrating a weak acid against a weak base.

In some experiments, more than one indicator may be used in the same flask, for example when titrating a mixture of strong and weak acids or bases.

\subsection{Colors of Indicators}
The colors of the above indicators in acid and base are:

\begin{center}
\begin{tabular}{ c | c | c | c |}
\hline
Indicator & Acid & Neutral & Base \\ \hline
Methyl Orange & Red & Orange & Yellow \\ \hline
Phenolphthalein & Colorless & Colorless & Pink \\ \hline
Bromothymol Blue & Yellow & Blue & Blue \\ \hline
Universal Indicator & Red, Orange, Yellow & Yellow/Green & Green, Blue, Indigo  \\
\hline
\end{tabular}
\end{center}

Titration is finished when the indicator starts a permanent color change. For example, when methyl orange turns orange, the titration is finished. If students wait until methyl orange turns pink (or yellow) they have overshot the endpoint of the titration, and their volume will be incorrect. Likewise, POP indicates that the titration is finished when it turns light pink. If students wait until they have an intensely pink solution, they will use too much base and get the wrong answer. 

Note that light pink POP solutions may turn colorless if left for a few minutes. This is due to carbon dioxide in the air reacting to neutralize bases in solution.

\subsection{Note on technique}
Students should use as little acid-base indicator as possible. This is because some acid or base is required to react with the indicator so that it changes color. If a lot of indicator is used, students will add more acid or base than they need.

\section{Other indicators}
Starch indicator is used in oxidation-reduction titrations involving iodine. This is because iodine forms an intense blue to black colored complex in the presence of starch. Thus starch allows a very sensitive assessment of the presence of iodine in a solution.

It is important to add the starch indicator close to the end point when there is an acid present. The acid will cleave the starch and that will prevent the starch from working properly. Students using starch should use a pilot run to get an idea when to add the starch indicator.

\section{Preparation of Indicators}
\begin{itemize}

\item{Methyl orange (MO): if you have a balance, weigh out about 1 g of methyl orange powder and dissolve it in about 1 L of water. Store the solution in a plastic water bottle with a screw on cap and it will keep for years. If it gets thick and cloudy, add a bit more water and shake. If you do not have a balance, add half of a small tea spoon to a liter of water.}

\item{Phenolphthalein (POP): Dissolve about 0.2 g of phenolphthalein powder in 100 mL of pure ethanol; then add 100 mL water with constant stirring. If you use much more water than ethanol, solid phenolphthalein will precipitate. Store POP in a plastic water bottle with a screw on cap. We recommend making POP in smaller quantities than MO as it does not keep as well, mostly due to the evaporation of ethanol. If the solution develops a precipitate, add a bit of ethanol and shake. We do not recommend using purple methylated spirits as a source of ethanol for making POP. You can distill purple spirits to make clear spirits. For clear methylated spirits, use 140ml of spirit and 60ml of water, as spirits generally are already 30\% water.}

\item{Starch: place about 1 g of starch in 10 mL of water in a test tube. Mix well. Pour this suspension into 100 mL of boiling water and continue to boil for one minute or so. Alternatively, use the water leftover after boiling pasta or potatoes. If this is too concentrated, dilute it with regular water.}

\item{The authors have never prepared bromothymol blue or universal indicator from powder, but suspect their preparation is similar to methyl orange.}

\end{itemize}

Note that the exact mass of indicator used is not very important. You just need to use enough so that the color is clearly visible. Students use very little indicator in each titration, and a liter of indicator solution should last you a long time.

\chapter{Traditional Volumetric Analysis Technique}
\label{cha:volanatech}
\section{Burettes}

In most acid-base titrations, the acid comes from the burette, although sometimes the burette holds the base. Prior to use, the student should thoroughly wash the burette to remove any residue from previous use. Then, the student should close the stopcock and add about 5ml of the solution that they will use in the burette. With their thumb over the open end of the burette they should make sure the solution covers every surface of the burette. They should then run this solution out into a waste container. This step is to replace the residue of water from the first washing with a layer of the titration solution. If students do not perform this step, the water reside will dilute their titration solution.

Most burettes have a volume of 50 mL. The 0 mL mark is at the top, and the 50 mL mark is at the bottom. This is because the burette tells you the volume of solution used, not the volume of solution present. If you start at 0 mL, and finish at 20 mL, then you have used 20 mL of acid.

Many burettes do not have stopcocks. Instead, they have a piece of rubber tubing at the bottom, which has a glass tip inserted into it. Either a metal clip is used to hold the rubber tubing closed or there is a small bead in the tubing around which fluid may pass when the tube is squeezed at that point. Broken burettes can often be repaired; see the section on Repairing Burettes.

\section{Reading measurements}

\begin{itemize}

\item{Always read burettes at eye-level. If the burette is clamped to a stand, remove it from the stand so you can hold it at eye-level. Or move the stand.}

\item{Always read from the bottom of the meniscus. Students often forget this; it helps to remind them at the beginning of a practical. In plastic apparatus, there is often no meniscus.}

\item{Burettes are accurate to 2 decimal places. Many times, students are taught that the last number should be either 5 or 0, like 15.55 or 15.50. This is incorrect – students should estimate the fluid level in the burette to the nearest 0.01 mL.}

\end{itemize}

\section{Titration Procedure}

\begin{itemize}

\item{Clean the burette with water. Then rinse it with the solution you will be using for titration.}

\item{Fill the burette with the solution. Allow a little solution to run out of the tip until the top of the fluid is at either 0.00 mL exactly or any value below. An initial volume of 1.32 mL is completely acceptable, at least from a scientific point of view. Your country may have specific expectations for marking exams.}

\item{Record the initial burette reading.}

\item{Use a syringe to transfer the other solution into a conical flask. Record the volume moved by the syringe.}

\item{If you are using indicator, add a few drops to the conical flask. For acid-base indicators, the less indicator used the better. In order to change color the indicator itself must react with some of the fluid from the burette. This consumes more chemical than is technically needed for neutralization; the additional chemicals required for titrating the indicator is called indicator error. One or two drops is really all you need. For starch indicator, use about 1 mL. The starch is not titrated, unlike acid-base indicators, so you can use more and often must to get a good color.}

\item{Slowly add solution from the burette to the conical flask. As you titrate, swirl the flask to mix. Do not shake it back and forth, because the solution in the flask will splatter onto the sides of the flask and thus will not be part of the neutralization reaction. Much the same, be careful to add the drops from the burette so they fall into the solution and are not stuck on the side of the flask. Stop titration when the indicator starts a permanent, slight color change. This is the endpoint. Again, the slightest change in color to the appropriate color indicates the endpoint, as long as the color remains after a few swirls.}

\item{Record the final burette reading.}

\end{itemize}

Titration is often done four times: a pilot followed by three trials. The purpose of the pilot is to find the approximate volume from the burette. The pilot is done quickly, and often overshoots the endpoint. In subsequent titration, use the results of the pilot to avoid overshooting while speeding up the work. For example, if the pilot gave an endpoint of 26 mL, add your volume rapidly from the burette until about 20 mL. Then add drop by drop until you find the endpoint.

The result from the pilot is not considered in calculations, as it is not expected to be accurate. Do not include it when finding the average volume or the variance.

\chapter{Volumetric Analysis Without Burettes}

\section{Theory}

Burettes are not necessary to perform volumetric analysis with reasonable precision. Students may use plastic syringes in place of burettes. These should be the most precise syringes available, which as of late 2010 were the 10~mL NeoJect brand plastic syringes. These syringes are more accurate than the low cost glass pipettes that many school purchase. As the accuracy of the titration is no better than its least accurate instrument, a titration with two plastic syringes is more accurate than a titration with a burette and a cheap glass pipette.

If use of these syringes is new to you, please read Use of Plastic Syringes before proceeding.

To get maximum precision from plastic syringes, students should learn how to estimate values between the lines on the syringe body. The NeoJect syringes are marked with lines every 0.2~mL. Students should observe the top of the fluid and decide if it is on the line exactly, half way in between, or in between half way and one of the lines. This allows them to divide the space between lines into four parts, giving them a precision of 0.05~mL. Estimation between gradations is standard practice with scientific instruments; even students using burettes should estimate the fluid height between the lines to at least 0.05~mL. Syringes have the capacity to deliver the precision required by most if not all  national exams.

If students are using syringes in place of burettes, they require two syringes for the practical, one as a burette and a different one as the pipette. We recommend that you label the syringes, for example, on one syringe writing ‘Burette’ with a permanent pen to help students remember which is which.

\section{Titration Procedure without Burettes}

\begin{enumerate}

\item{Clean the ‘pipette’ syringe with water. Then rinse it with the acid or base solution you will be putting in the flask.}

\item{Use a syringe to transfer the required amount of acid or base to the flask. To do this transfer accurately, add first 1 mL of air to the syringe and then suck up the fluid to beyond the desired amount. Push back the plunger until the top of the fluid is exactly the volume required. Delivering the required volume to the flask may take multiple transfers with the single syringe. Record the total volume transferred to the flask as the ‘volume of pipette used’}

\item{Add one or two drops of indicator to the flask.}

\item{Clean the ‘burette’ syringe with water. Then rinse it with the acid or base solution you will be using to titrate.}

\item{Add 1~mL of air to the syringe and then suck up the acid or base solution to beyond the 10~mL mark. Slowly push back the plunger until the top of the fluid is exactly at the 10 mL line.}

\item{Slowly add the solution from the syringe to the flask. As you titrate, swirl the flask to mix. As described above, swirl instead of shaking to keep all of the liquid together. Make sure that each drop from the syringe hits the liquid rather than getting suck on the edge of the container. Stop titration when the indicator starts a permanent color change. Just as with a burette, this is the endpoint.}

\item{Often the volume required from the ‘burette’ is greater than 10~mL. This is no problem – after finishing the syringe students should simply fill it again as they did the first time and continue. On their rough paper (scratch paper), they should note that they have already consumed 10~mL.}

\end{enumerate}

\section{Table of Results when using syringes in place of burettes}

At present, many national exam marking boards expect students to use burettes. The obvious problem is that while the top line on a burette is 0~mL, the top of the syringe reads 10~mL. For students to get the marks their careful technique deserves, they must record their results in a manner consistent with traditional reporting. On rough paper, students should calculate the volume of solution used in their titration. This is easy – if the syringe started at 10.00 mL and ended at 2.55~mL, the student used $10.00 \mathrm{mL} – 2.55 \mathrm{mL} = 7.45 \mathrm{mL}$ of solution. If the student used two full syringes and the third finished at 4.65~mL, then the student used $10.00 \mathrm{mL} – 4.65 \mathrm{mL} = 5.35 \mathrm{mL}$ in the last syringe plus 10~mL in each of the first two syringes, so $5.35 \mathrm{mL} + 10 \mathrm{mL} + 10 \mathrm{mL} = 25.35 \mathrm{mL}$ total.

In the Table of Results, the student should then write 25.35~mL for the Volume Used. If this volume had been used in a burette, the student would have found an initial volume of 0.00~mL and a final volume of 25.35~mL. The rest of the table should be filled in this manner. When using a syringe as a burette, the student should always write 0.00~mL for the Initial Volume and then for Final Volume they should write the total number they calculated for Volume Used. This method will ensure that the students gets the marks he or she deserves for careful titration – and likewise ensure that he or she loses the appropriate marks for mistakes.

\chapter{Relative Standardization}

Preparing large volumes of solution is difficult with great accuracy. Relative standardization is a technique to correct the concentration of solutions so that they give the correct results for practical exercises. Note that this technique is only useful in educational situations where the purpose is to prepare a pair of solutions for titration that give an answer known by the teacher. In scientific research, the aforementioned technique – absolute standardization – is used because the concentration of one of the solutions is truly unknown.

All schools should use relative standardization to check the concentration of the solutions they prepare for the national examinations. This ensures that the tests measure the ability of the students to perform the practical, and not the quality of the school's balance, water supply, glassware, etc. While useful for all schools, relative standardization is particularly helpful for schools with few resources, as it allows the preparation of high quality solutions with extremely low cost apparatus and chemicals.

\section{General Theory}

The principle of a titration is that the chemical in the burette is added until it exactly neutralizes the chemical in the flask. If the two chemicals react 1:1, e.g. 

\[ \mathrm{HCl}_{(aq)} + \mathrm{NaOH}_{(aq)} \longleftarrow \mathrm{NaCl}_{(aq)} + \mathrm{H}_2\mathrm{O}_{(l)} \]

then exactly one mole of the burette chemical is required to neutralize one mole of the chemical in the flask. If the two chemicals react 2:1, e.g. 

\[ 2\mathrm{HCl}_{(aq)} + \mathrm{Na}_2\mathrm{CO}_{3}_{(aq)} \longleftarrow 2\mathrm{NaCl}_{(aq}) + \mathrm{H}_2\mathrm{O}_{(l)} + \mathrm{CO}_2_{(g)} \]

then exactly two moles of the burette chemical is required to neutralize one mole of the chemical in the flask. Let us think of this reaction as a mole ratio.

\[ \frac{moles of A}{moles of B} = \frac{n_{A}}{n_{B}} \]

Where $ n_A $ and $ n_B $ are the stoichiometric coefficients of A and B respectively.

\[ \mathrm{moles} = \mathrm{molarity} \times \mathrm{volume} = M \times V \mathrm{(so long as V is measured in liters)} \]

By substitution,

\[ \frac{(M_A)(V_A)}{(M_B)(V_B)} = \frac{n_A}{n_B} \]

A student performing a titration might rearrange this equation to get

\[ M_A = \frac{(n_a)(M_B)(V_B)}{(n_B)(V_A)} \]

or

\[ M_B = \frac{(n_B)(M_A)(V_A)}{(n_A)(V_B)} \]

As teachers, however, we care with something else: making sure that our students find the required volume in the burette. Solving the equation for $ V_{A} $ we find that

\[ V_A = \frac{(n_a)(M_B)(V_B)}{(n_B)(M_A)} \]

As $ n_A $ and $ n_B $ are both set by the reaction, as long as we use the correct chemicals there is no problem here.

$ V_B $ is measured by the students – it is the volume they transfer into the flask. As long as the students know how to use plastic syringes accurately, they should get this value almost perfectly correct.

The remaining term, $ \frac{M_B}{M_A} $ is for the teacher, not the student, to make correct. If we prepare the solutions poorly, our students can do everything right but still get the wrong value for $ V_A $. It is very important that we ensure that our solutions have the correct ratio of $ \frac{M_B}{M_A} $ so that the exercise properly assesses the ability of our students.

Many people look at this ratio and decide that they therefore need to prepare both solutions perfectly, so that $ M_B $ and $ M_A $ are exactly what is required. This not true. The actual values for $ M_B $ and $ M_A $ are not important; what matters is the ratio $ M_B $ to $ M_A $!

For example, if the titration requires 0.10~M HCl and 0.10~M NaOH, our expected mole ratio is:

\[ \frac{M_{HCl}}{M_{NaOH}} = \frac{0.10}{0.10} = 1 \]

Preparing 0.11 M HCl and 0.09 M NaOH will cause the students to get the wrong answer:

\[ \frac{M_{HCl}}{M_{NaOH}} = \frac{0.11}{0.09} = 1.22 \]

However, preparing exactly 0.05 M HCl and 0.05 M NaOH results in the same molar ratio:

\[ \frac{M_{HCl}}{M_{NaOH}} = \frac{0.05}{0.05} = 1 \]

Thus the students can get exactly the right answer if they use the right technique even though neither solution was actually the correct concentration.

How can we ensure that we have the correct molar ratio between our solutions? Titrate your solutions against each other. If the volume is not the expected value, one of your solutions is too concentrated relative to the other. You can calculate exactly how much too concentrated and add the exact amount of water necessary to perfect the ratio. This process is called relative standardization, because you are standardizing one solution relative to the other.

\section{Procedure for Relative Standardization}

In some titrations the acid is in the burette and in some it is the base is in the burette. So let us not use “acid” and “base” to refer to the solutions, but rather “solution 1” and “solution 2” where solution 1 is the solution measured in the burette and solution 2 is measured by pipette (syringe).

You should have prepared a bucket or so of each. The volume you have prepared is $ V_1 $ liters of solution 1 and $ V_2 $ liters of solution 2.

Titrate the solutions against each other. Call the volume you measure in the burette “actual titration volume” You know the desired molarity of each solution, so from the above student equations you can calculate the burette volume you expect, which you might call “theoretical titration volume.”

After the titration, there are three possibilities. If the actual titration volume equals the theoretical titration volume, your solutions are perfect. Well done.

If the actual titration volume is smaller than the theoretical titration volume, solution 1 is too concentrated and must be diluted. Use the ratio:

\[ \frac{V_1 \mathrm{(before dilution)}}{V_1 (\mathrm{after dilution)}} = \frac{\mathrm{actual titration volume}}{\mathrm{theoretical titration volume}} \]

If the actual titration volume is larger than the theoretical titration volume, solution 2 is too concentrated and must be diluted. Use the ratio:

\[ \frac{V_2 \mathrm{(before dilution)}}{V_2 (\mathrm{after dilution)}} = \frac{\mathrm{theoretical titration volume}}{\mathrm{actual titration volume}} \]

After diluting one of your solutions, repeat the process. After a few cycles, the solutions should be perfect. Remember that the volume “before dilution” is the volume actually in the bucket, so the amount you made less the amount used for these test titrations.

\chapter{Preparation of Solutions without a Balance}

The procedure in the section on Relative Standardization allows us to do something seemingly impossible – prepare solutions for volumetric analysis that allow students to get perfect results without using either a balance or volumetric glassware in the preparation. All that you have to do is make two solutions that are close, and then use several cycles of relative standardization to prefect the molarity ratio. 

To measure volume, we can use marks on plastic water bottles as described in the entry for volumetric glassware in the Sources of Equipment section. What follows is an example of how rough solutions can be prepared in Tanzania based on the water bottles available in that country. We encourage people in other country to calibrate their water bottles and then to customize these instructions for the resources available to them.

\section{To make 0.05~M sulfuric acid (equivalent to 0.1~M HCl) for fifty students}
\begin{enumerate}
\item{Put 9.9 liters of water into a bucket. On the new 1.5~L Kilimanjaro water bottle, the bottom points of the crown embossed on the side correspond to 300~ml and the top of the mountain corresponds to 1.5~L. Therefore one can measure 9.9 liters by filling the bottle to the mountain top six times and then to the bottom points of the crown three times.}
\item{Add 110~mL of battery acid. This may be accomplished easily by filling a 10~mL plastic syringe eleven times. Please read the safety note in Dangerous Chemicals.}
\end{enumerate}

\section{To make 0.033~M citric acid (equivalent to 0.1~M HCl) for fifty students}
\begin{enumerate}
\item{Put 10 liters of water into a bucket. One the new 500~mL Kilimanjaro water bottle, the second straight line corresponds to 300~mL and the highest straight line corresponds to 400~mL. Therefore one can use the 1.5~L bottle six times to add nine liters and then use the 500~mL bottle to add one more liter, 400~mL + 300~mL + 300~mL.}
\item{Add 64~g of citric acid. In the absence of a balance, one can often have $^1/_8$ of a kilogram (125~g) measured in the market. Dissolved this in 20~L of water to produce a 0.033~M solution. Alternately, use a plastic syringe to find the volume of a plastic spoon. Fill the spoon with citric acid and push off any extra acid until there is a flat surface (like the water). Then use that spoon to add a total $38 \mathrm{cm}^3$ or mL of citric acid soda knowing the volume of each spoonful.}
\end{enumerate}

\section{To make 0.1~M sodium hydroxide for fifty students}
\begin{enumerate}
\item{Put 10 liters of water into a bucket. See the instructions above.}
\item{Add 40~g of caustic soda. In the absence of a balance, measure the volume of a spoon as above and add $19 \mathrm{cm}^3 \mathrm{or mL}$ of caustic soda. Please read the safety note in Dangerous Chemicals.}
\end{enumerate}

\section{To make 0.1~M sodium hydrogen carbonate for fifty students}
\begin{enumerate}
\item{Put 10 liters of water into a bucket. See the instructions above.}
\item{Add 84~g of bicarbonate of soda. In the absence of a balance, find the volume of a spoon as above and add $39 \mathrm{cm}^3 \mathrm{or mL}$ of bicarbonate of soda. Alternately, if 8.33 liters of solution is sufficient, measure this volume of water and then add one whole box of bicarbonate of soda. A box is 70~g.}
\end{enumerate}

\chapter{Substituting Chemicals in Volumetric Analysis}
\label{cha:subchemvolana}
\section{Theory}

The volumetric analysis practical exercises sometimes call for expensive chemicals, for example potassium hydroxide or oxalic acid. As the purpose of exercises and exams is to train or test the ability of the students and not the resources of the school, it is possible to use different chemicals as long as the solutions are calibrated to give equivalent results. For example, if the instructions call for a potassium hydroxide solution, you can use sodium hydroxide to prepare this solution. It will not affect the results of the practical -- if you make the correct calibration. How to calibrate solutions when substituting chemicals is the subject of this section.

Technically, only two chemicals are required to perform any volumetric analysis practical: one strong acid and one strong base. The least expensive options are sulfuric acid, as battery acid, and sodium hydroxide, as caustic soda. To substitute one chemical for another in volumetric analysis, the resulting solution must have the same normality (N).

\begin{itemize}

\item{For all monoprotic acids (HCl, ethanoic acid), the normality is the molarity.\\
\textit{Example: 0.1~M ethanoic acid = 0.1~N ethanoic acid}}
\item{For diprotic acids (sulfuric acid, ethandiotic acid), the normality is twice the molarity, because each molecule of diprotic acid brings two molecules of $\mathrm{H}^{+}$.\\
\textit{Example: 0.5~M sulfuric acid = 1.0~N sulfuric acid}}
\item{For the hydroxides and hydrogen carbonates used in ordinary level (NaOH, KOH, NaHCO$_{3}$), the normality is the molarity.\\
\textit{Example: 0.08~M KOH = 0.08~N KOH}}
\item{For the carbonates most commonly used ($\mathrm{Na}_2\mathrm{CO}_3$, $\mathrm{Na}_2\mathrm{CO}_3 /dot 10\mathrm{H}_2\mathrm{O}$, $\mathrm{K}_2\mathrm{CO}_3$), the normality is twice the molarity.\\
\textit{Example: $0.4 M \mathrm{Na}_2\mathrm{CO}_3 = 0.8 N \mathrm{Na}_2\mathrm{CO}_3$}}

\end{itemize}

\section{Substitution Calculations}

When instructions describe solutions in terms of molarity, calculating the molarity of the substitution is relatively simple. For example, suppose we want to use sulfuric acid to make a 0.2~M solution of ethanoic acid. 0.2~M ethanoic acid is 0.2~N ethanoic acid which will titrate the same as 0.2~N sulfuric acid. 0.2 N sulfuric acid is 0.1~M sulfuric acid, and thus we need to prepare 0.1~M sulfuric acid.

When instructions describe solutions in terms of concentration ($^g/_L$), we just need to add an extra conversion step. For example, suppose we want to use sodium hydroxide to make a $14.3 ^g/_L$ solution of sodium carbonate decahydrate. $14.3 ^g/_L$ sodium carboante decahydrate is 0.05~M sodium carbonate decahydrate which is 0.1~N sodium carbonate decahydrate. This will titrate the same as 0.1~N sodium hydroxide, which is 0.1~M sodium hydroxide or $4 ^g/_L$ sodium hydroxide, and thus we need to prepare $4 ^g/_L$ sodium hydroxide to have a solution that will titrate identically to $14.3 ^g/_L$ sodium carbonate decahydrate.

\section{Common Substitutions}
\label{sec:commonsubs}
To simplify future calculations, we have prepared general conversions for the most common chemicals used in volumetric analysis. Remember to check all final solutions with relative standardization to ensure that they indeed give the correct results.

\begin{center}
\begin{tabular}{| p{2cm} | p{2cm} | p{5cm} | p{5cm} | p{5cm} |}
\hline

\textbf{Required Chemical} & 
\textbf{Low Cost Alternative} & 
\textbf{Substiution Method} & 
\textbf{Molarity Example} & 
\textbf{Concentration Example} \\ \hline

Hydrochloric Acid & 
Sulfuric Acid (Battery Acid) & 
If you are required to prepare an X molarity solution of HCl, prepane a $X \times 0.5$ molarity solution of battery acid & 
The instructions call for 0.12~M HCl. Instead, prepare 0.06~M sulfuric acid & 
 \\ \hline

Ethanoic (Acetic) Acid & 
Sulfuric Acid (Battery Acid) & 
If you are required to prepare an M molarity solution of ethanoic acid, prepare a $M \times 0.5$ molarity solution of sulfuric acid & 
The instructions call for 0.10~M ethanoic acid. Prepare 0.05~M sulfuric acid. & 
 \\ \hline

Ethandioic (Oxalic) Acid dihydrate (C$_{2}$H$_{2}$O$_{4} \cdot$2H$_{2}$O) & 
Sulfuric Acid (Battery Acid) & 
If you are required to prepare an M molarity solution of ethandioic acid, prepare an M molarity solution of sulfuric acid. If you are required to prepare a C concentration solution of ethandioic acid, prepare a $^C/_{126}$ molarity solution of sulfuric acid. & 
The instructions call for 0.075~M ethandioic acid. Prepare 0.075~M sulfuric acid. & 
The instructions call for $6.3 ^g/_L$ ethandioic acid. Prepare 0.05~M sulfuric acid. \\ \hline

Potassium Hydroxide & 
Sodium Hydroxide (Caustic Soda) & 
For M molarity potassium hydroxide, make M molarity sodium hydroxide. For C concentration potassium hydroxide, make $C \times ^{40}/_{56}$ concentration sodium hydroxide. & 
The instructions call for 0.1~M potassium hydroxide. Just prepare 0.1~M sodium hydroxide. &
The instructions call for $2.8 ^g/_L$ potassium hydroxide. Prepare $2 ^g/_L$ sodium hydroxide. \\ \hline

Anhydrous Sodium Carbonate & 
Sodium Carbonate Decahydrate (Soda Ash) &  
For M molarity anhydrous sodium carbonate, make M molarity sodium carbonate decahydrate. For C concentration anhydrous sodium carbonate, make $C \times ^{286}/_{106}$ sodium carbonate decahydrate. & 
The instructions call for 0.09~M anhydrous sodium carbonate. Make 0.09~M sodium carbonate decahyrate. & 
The instructions call for $5.3 ^g/_L$ anhydrous sodium carbonate. Make 14.3 g/L sodium carbonate decahydrate. \\ \hline

Anhydrous Sodium Carbonate & 
Sodium Hydroxide (caustic soda) & 
For M molarity anhydrous sodium carbonate, make $M \times 2$ molarity sodium hydroxide. For C concentration anhydrous sodium carbonate, make $C \times 2 \times ^{40}/_{106}$ sodium hydroxide. & 
The instructions call for 0.09~M anhydrous sodium carbonate. Make 0.18~M sodium hydroxide. & 
The instructions call for $5.3 ^g/_L$ anhydrous sodium carbonate. 4.0 g/L sodium hydroxide. \\ \hline

Sodium Carbonate Decahydrate (Na2CO3∙10H2O) &
sodium hydroxide (caustic soda) &
For M molarity sodium carbonate ecahydrate, make $M \times 2$ molarity sodium hydroxide. For C concentration sodium carbonate decahydrate, make $C \times 2 \times ^{40}/{286}$ sodium hydroxide. &
The instructions call for 0.09~M sodium carbonate decahydrate. Make 0.18~M sodium hydroxide. &
The instructions call for 14.3 g/L sodium carbonate decahydrate. Make 4.0 g/L sodium hydroxide. \\ \hline

Anhydrous Potassium Carbonate & 
Sodium Carbonate decahydrate (Soda Ash) & 
For M molarity potassium carbonate, make M molarity sodium carbonate decahydrate. For C concentration potassium carbonate, make $C \times ^{286}/_{122}$ concentration sodium carbonate. & 
The instructions call for 0.08~M anhydrous potassium carbonate. Prepare 0.08~M sodium carbonate decahydrate. & 
The instructions call for $6.1 ^g/_L$ anhydrous potassium carbonate. Prepare $14.3 ^g/_L$ sodium carbonate decahydrate. \\ \hline

Anhydrous Potassium Carbonate & 
Sodium Hydroxide (caustic soda) & 
For M molarity potassium carbonate, make $M \times 2$ molarity sodium hydroxide. For C concentration potassium carbonate, make $C \times 2 \times ^{40}/{122}$ concentration sodium hydroxide. & 
The instructions call for 0.08~M anhydrous potassium carbonate. Prepare 0.16~M sodium hydroxide. & 
The instructions call for $6.1 ^g/_L$ anhydrous potassium carbonate. Prepare $4.0 ^g/_L$ sodium hydroxide. \\ \hline

\end{tabular}
\end{center}

\section{Additional Notes}

\begin{itemize}

\item{In volumetric analysis experiments with two indicators, it is not possible to substitute one chemical for another as the acid/base dissociation constant is critical and specific for each chemical. It is still possible to substitute sodium carbonate decahydrate for anhydrous sodium carbonate with the above conversion.}

\item{These substitutions only work for volumetric analysis. In qualitative analysis, the nature of the chemical matters. If the instructions call for sodium carbonate, you cannot provide sodium hydroxide and expect the students to get the right answer!}

\end{itemize}

\chapter{Preservation of Specimens}

\section{Dead Specimens}
\begin{itemize}

\item{Mosses and lichens: 
Wrap in paper or keep in a closed container.}

\item{Plants and parts thereof: 
hang in the sun until dry. 
Alternatively, press the plants using absorbent material and a stack of books.}

\item{Insects: 
Leave exposed to air but out of reach by other insects 
until bacteria eat everything except the exoskeleton. 
If you want to preserve the soft tissue, 
store under methylated spirits.}

\item{Fish, worms, amphibians, and reptiles: 
Store in methylated spirits (will makes specimens brittle) 
or a 10\% formaldehyde solution (more poisonous and more expensive).}

\item{Parts of mammals (e.g. pig eyes, bovine reproductive organs): 
store in 10\% formaldehyde solution.}

\end{itemize}

\section{Skeletons}
Skin the animal and remove as much meat as possible. 
Bury the bones for several months. 
Exhume and assemble with wire and superglue.

\section{Living Specimens}
Be creative! 
Figure out what the animal will eat, 
who will feed it, what it will drink, where it can hide, 
how it can be observed, etc.

\chapter{Dissection}

%==============================================================================
\section{Preparation of Specimens}

Unless you want students to observe a beating heart, 
dead specimens are much easier to work with than unconscious ones. 
This also removes the problem of stunned animals waking up 
in the middle of their dissection.

\begin{itemize}

\item{Flowers and other plant parts: 
No preparation required as long as the samples are relatively fresh. 
Store samples in closed plastic bags to minimize drying. 
If you intend to keep them for more than a day or two, 
submerge the bags in cold water to slow the rate of molding.}

\item{Insects: Kill with household aerosol insecticide. 
Use specimens within one day of collection, 
unless you have refrigeration or freezer.}

\item{Fish: Keep living until the day of the dissection. 
Then remove from water until they suffocate. 
Use immediately after death.}

\item{Frogs: Able to breathe above and below water, 
frogs are hard to starve of oxygen. 
One option is to seal them in a container of methylated spirits 
and then rinse the dead bodies with water prior to dissection.}

\item{Reptiles, birds, and mammals: For most organ systems, 
you can kill the animal by blunt trauma without ruining the lesson. 
Students can even bring animals caught and killed in homes. 
Snakes should be decapitated along with enough of the body 
to remove the fangs and venom sacks. 
Bury these deeply. 
Do not use animals killed by poison, 
or those that were found dead. 
For completely undamaged specimens, 
enclose the live animal in a cage (or a tin with adequate holes) 
and submerge in a bucket of water until drowned.}

\item{Living specimens: 
If you really want to see that heart beating, use chloroform. 
This can be transferred from bottle to specimen jar via cotton ball, 
or perhaps made in situ by the reaction between propanone (acteone) and bleach. 
We have not yet attempted the latter – 
if you do, remember that the products are poisonous gases; 
indeed, that is the point. Note that if you use too little chloroform, 
the animal will feel the blade opening it up. 
If you use way too little, it may start squirming. 
If you use too much chloroform, however, you will simply kill the animal – 
you might as well have drowned it.}

\end{itemize}

%==============================================================================
\section{Tools}

\begin{itemize}

\item{Scalpels are the best tool for this job. 
They are very sharp and deliver a great deal of force to their point. 
This allows students to made clean cuts with minimal pressure. 
The next best things are homemade scalpels, 
razor blades attached to a handle to ensure a firm command of the blade. 
If the blade is dull or floppy, the students will probably push too hard, 
and may cut themselves when the skin finally gives and the blade slips.}

\item{Pins should be sharp and strong. 
Unused needles from new disposable syringes are an easy option.}

\item{Dissection trays can be prepared 
by making a 1cm thick layer of wax on the bottom of a shallow tray or bowl. 
This surface will readily accept pins and is easy to clean.}

\end{itemize}

%==============================================================================
\section{Procedure}

This varies by species. The internet has many resources and there are many good books with very detailed instructions – alas, this manual is not yet one of them. A crude method follows:

Position the specimen on its back and make a clean, symmetric, and shallow incision down the full length of the underside. Make additional perpendicular cuts at the top and bottom of the torso for an overall “I” shape. These cuts should only just penetrate the body cavity. Open up skin “doors” you have created, pinning them back onto the dissection tray. Pick an organ system – circulation, digestion, nervous, etc – and, with the aid perhaps of a good drawing, remove other material to focus on the target anatomy. You can teach many systems from one specimen – start with the most ventral (front) and move to the most dorsal (back).

Encourage students to sketch at various steps in the process. Also encourage them to identify anatomy for themselves, perhaps with the aid of thought provoking questions and discussion in groups.

%==============================================================================
\section{Cleanup and Carcass Disposal}

Wash all blades, pins, and trays with soapy water. 
Rinse all tools to remove the soap 
and then soak for about fifteen minutes in bleach water. 
When finished, rinse again in ordinary water.

Bury all carcasses in a deep pit, below the reach of dogs. 
You may also add kerosene and burn, 
but this smells bad and costs money.

\chapter{Preparation of Culture Media}

\section{Introduction}

In microbiology, there are two basic types of media: solid agar media and a liquid broth media. From these, many types of media can be made. Generally, exact amounts of ingredients are not needed so if you want to make some agar plates or liquid cultures try with the resources you have. The recipes listed are a guideline to help you get started.

%==============================================================================
\section{Media Recipes}

%------------------------------------------------------------------------------
\subsection{Basic Agar (1.5\%)}

\begin{itemize}

\item{15 g/L agar\\
it is like gelatin or if you can find seaweed you can grind it up}

\item{10 g/L nutrient source\\
e.g.sugar, starch (potatoes), beans fruits like mango and papaya}

\item{1-2 g/L salts and phosphates\\
this varies with what you want to grow — experiment! (table salt is usually fine)}

\item{1 L water}

\end{itemize}

Add and mix all the ingredients together and heat until boiling. Boil for \~15 minutes and make sure all the gelatin/agar is dissolved. Pour liquid into Petri plates (15-20~mL each). The plates should solidify \~45°C. Cover and keep agar side up in a cool place if possible. If the plates do not solidify, try adding more gelatin or corn starch to thicken it up. You can also pour agar into test tubes/syringes to do oxygen tests (aerobic vs. anaerobic)

%------------------------------------------------------------------------------
\subsection{Blood Agar}

\begin{itemize}

\item{15 g/L agar/gelatin/ground sea weed}

\item{10 g/L nutrient source}

\item{15 mL sheep’s blood (other organizisms also work)}

\item{1 L water}

\end{itemize}

Heat and boil agar, nutrient source and water for 15 minutes. After liquid has ceded (\~45°C (when you can leave your hand on the flask for a few seconds) add in blood until the mixture is blood red. Swirl in and pour into plates.

%------------------------------------------------------------------------------
\subsection{Liquid Broths}

\begin{itemize}

\item{10~g nutrient source}

\item{1~L water}

\item{1-2~g salts/phosphates}

Mix together, heat, and boil. Distribute in test tubes.

\end{itemize}

%==============================================================================
\section{Things you can do after media preparation}

\begin{itemize}

\item{Agar-streaked plates! Swab something (back of throat, nose, belly button, door handle, etc) and gently rub onto the agar. Try not to gouge the agar.}

\item{You can also do experiments to test the effects of salt concentrations, temperature, and nutrient concentrations.}

\item{After all the plates solidify, incubate them at around 25-30C. Ideally the temperature remains constant. Check the plates after 24 hours for growth.}

\item{For liquid broths you can inoculate test tubes with a sample from the environment. Incubate and check like agar plates. If there is growth the liquid will be turbid instead of clear like a control tube with only broth.}

\item{You can use liquid cultures for wet mounts under microscopes as samples for agar plates or to allow students to see the difference between growth and no growth.}

\end{itemize}

%==============================================================================
\section{What to use if you do not have plates or test tubes}

\begin{itemize}

\item{Use old water bottles or old plastic packaging for plates}

\item{Use anything rigid and heavy for covered, e.g. building tiles}

\item{Sealed/closed plastic syringes for test tubes}

\item{Try to keep materials as sterile as possible but do not worry if there is contamination. Use contamination as a learning experience. Penicillin was contamination and it became a wonder drug.}

\end{itemize}

\section{Things to do once you have cultures}

\begin{itemize}

\item{Take a sample from agar plate and drop hydrogen peroxide on it. Does it bubble? (Yes, it has catalase)}

\item{Extract DNA from \textit{E. coli.}}

\item{Fermentation = use a liquid broth with peptone, acid-base indicator like phenol red, and inverted tube to trap gas and 0.5 – 1.0\% of carbohydrate you want to test. If fermentation occurs (phenol red), the broth will turn yellow and gas should be collected in the tube. If the tube remains red, you can test for glucose production by adding a few drops of methyl orange. If the pH is below 4.4, it will remain red. If the pH is above 6.0, it will turn yellow.}

\end{itemize}

%==============================================================================
\section{Guide to Identifying Common Microorganisms}

\begin{itemize}

\item{\textit{Pseudomonas aeroginosa}: is green and smells like grape jelly (can grow in disinfectant)}

\item{\textit{Serratia marcescens}: grows pink-red between 25-32°C (will be white otherwise)}

\item{\textit{Escherichia coli}: pale white/yellow, smells like inole}

\item{\textit{Proteus spp}: swarm on plates and smell like urine and brownies}

\item{\textit{Bacillus subtillis}: pale beige, smells a bit sweet}

\item{\textit{Vibrio cholera}: smells like buttery popcorn}

\item{Staph vs Strep: Staph is catalase (+), strep is (-)}

\end{itemize}

\chapter{Using a Microscope}

\section{Parts of a Microscope}

\begin{itemize}

\item{Eyepiece: or ocular lens is what you look through at the top of the microscope. Typically, the eyepiece has a magnification of 10x.}

\item{Body Tube: tube that connects the eyepiece to the objectives}

\item{Objective Lenses: primary lenses on the microscope (low, medium, high, oil immersion) which are used to greater magnify the object being observed. A low power lens for scanning the sample, a medium power lens for normal observation and a high power lens for detailed observation. Normal groups of lens magnifications may be [4×, 10×, 20×] for low magnification work and [10×, 40×, 100×] for high magnification work. Some microscopes also use oil immersion lenses and these must be used with immersion oil between the lens and the cover slip on the slide. Oil immersion allows for a much greater magnification than air and typically ranges from 40x-100x.}

\item{Revolving Nosepiece: houses the objectives and can be rotated to select the desired magnification.}

\item{Coarse Adjustment Knob: a large knob used for focusing the specimen}

\item{Fine Adjustment Knob: small knob used to fine-tune the focus of the specimen after using the coarse adjustment knob.}

\item{Stage: where the specimen to be viewed is placed}

\item{Stage Clips: used hold the slide in place}

\item{Aperture: hole in the stage that allows light through to reach the specimen}

\item{Diaphragm: controls the amount of light reaching the specimen}

\item{Light Source: is either a mirror used to reflect light onto the specimen or a controllable light source such as a halogen lamp}

\end{itemize}

\section{How to Use a Microscope}

\begin{itemize}

\item{Always carry a microscope with two hands! One on the arm and one on the base!}

\item{Plug the microscope into an electrical source and turn on}

\item{Make sure the stage is lowered and the lowest power objective lens is in place}

\item{Place the slide under the stage clips with specimen above the aperture}

\item{Look through the eyepiece and use the coarse adjustment knob to bring the specimen into focus}

\item{If the microscope uses a mirror as the light source, adjust the mirror so enough light is reflected through the aperture onto the specimen}

\item{You can adjust the amount of light reaching the specimen by opening and closing the diaphragm}

\item{Once the object is visible, use the fine adjustment knob for a more precise focus}

\item{At this point you can increase the magnification by switching to a higher power objective lens}

\item{Once you switch from the low power objective lens, you should no longer be using the coarse adjustment knob for focusing because it is possible to break the slide and scratch the lenses}

\item{If you switch objectives, use the fine adjustment to fine-tune the focus of the object
If the high powered objective lenses on the microscope say oil then you can place a small drop of immersion oil on the cover slip then switch to the oil immersion lens. \textit{Only use the oil immersion lens with immersion oil and don’t use oil with any other objective that does not say oil.}}

\item{Once you have finished observing the specimen, lower the stage, remove the slide, and return to the lowest objective}

\item{Clean the lenses with lens cleaner and lens paper (only use lens paper as other tissues will scratch the lenses)}

\item{Wrap the cord around the base and cover the microscope for storage}

\end{itemize}

\section{Making a Wet Mount}

\begin{itemize}

\item{Collect a thin slice (one cell layer thick is optimal) of specimen and place on the slide}

\item{Place a drop of water directly over the specimen}

\item{Place a cover slip at a 45 degree angle over the specimen with one edge touching the drop of water then drop the cover slip over the specimen. If done correctly, the cover slip will completely cover the specimen and there will be no air bubbles present.}

\end{itemize}

\section{Staining a Slide}

\begin{itemize}

\item{Once you have completed the above process place one small drop of stain (ex. Iodine, methylene blue) on the outside edge of the cover slip}

\item{Place the flat edge of a paper towel on the other side of the cover slip. The paper towel will draw the water out from under the cover slip and pull in the stain}

\end{itemize}

\section{Magnification}
The actual power of magnification is a product of the ocular lens (usually 10x) times the objective lens.

\begin{center}
\begin{tabular}{ c | c | c | }
\hline
Ocular lens (eyepiece) & Objective Lens & Total magnification \\ \hline
10x & 4x & 40x \\ \hline
10x & 40x & 400x \\ \hline
10x & 100x & 1000X \\
\hline
\end{tabular}
\end{center}

\section{Troubleshooting}

\begin{enumerate}

\item{The Image is too dark!\\
\textit{Adjust the diaphragm and make sure your light is on.}}

\item{There's a spot in my viewing field, even when I move the slide the spot stays in the same place!\\
\textit{Your lens is dirty. Use lens paper, and only lens paper to carefully clean the objective and ocular lens. The ocular lens can be removed to clean the inside.}}

\item{I can't see anything under high power!\\
\textit{Remember the steps, if you can't focus under scanning and then low power, you won't be able to focus anything under high power.}}

\item{Only half of my viewing field is lit, it looks like there's a half-moon in there!\\
\textit{You probably don't have your objective fully clicked into place.}}

\end{enumerate}

\chapter{Low Tech Microscopy}

Microscopes are powerful tools for teaching biology, and many of their benefits are hard to replace with local fabrications. However, simple materials can be used to achieve sufficient magnification to greatly expands students' understanding of the very small. They may view up close the anatomy of insects and even see cells.

\section{Water as a lens}
Water refracts light much the way glass does; a water drop with perfect curvature can make a powerful lens. A simple magnifier can be made by twisting a piece of wire around a nail and dipping the loop briefly into some water. Students can observe the optical properties of the trapped drop of water.

\section{Perfect circles}
Better imaging can be had if the drop is more perfect in shape - the asymmetric of the wire twisting distorts the image. Search for a piece of thin but stiff plastic - the firm, transparent packaging around new cell phone batteries works well. Cut a small piece of this plastic, perhaps 1 x 2 centimeters. Near one end, make a hole, the more perfect the better. The best hole-cutting tool is a paper hole punch, available in many schools. With care, fine scissors or a pen knife will suffice; remove all burrs.

\section{Slides}
A slide and even coverslip may be made from the same plastic, although being hydrophobic they will not have the same properties of glass when making wet mounts. Improvise a method for securing the punctured plastic over the slide; ideally the vertical spacing can be closely adjusted to focus.

\section{Backlighting}
On a bright day, there may not be any need for additional lighting, but in most classrooms the image will be too dim to be easily seen. The sun is a powerful light source, though not always convenient. Flashlights are generally inexpensive and available; many cell phones have one built in the end. To angle the light into the slide, find either a piece of mirror glass, wrinkle-free aluminum foil, the metalized side of a biscuit wrapper, etc.

Experiment with a variety of designs to see what works best given the materials available to your school. If you use a slide of onion cells stained with iodine solution (see Sources of Laboratory Chemicals), your students should be able to see cell walls and nuclei.


% Part 4 - Hands-On Activities
\part{Hands-On Activities}

\chapter{Biology Activities}

\section{Anatomy}

\subsection{Animal Dissection}
\begin{itemize}
\item{Preparation time: 30 minutes}
\item{Materials: razor blades or scalpel, animal parts, wood board, pins}
\item{Procedure: Find a local butcher and ask for intact eyes, hearts, lungs, liver, and other parts. If you have an iron stomach, purchase a goat yourself and take it apart. The meat can be eaten, the organs studied, and the bones buried for a month and then exhumed to give a complete skeleton. Use a razor blade or a scalpel to dissect each of the different parts. To make a scalpel, take a small piece wood, stick, plastic, or anything that will not break under a little pressure, and line up a razor blade at the end and tape the blade. Make sure to tape all the way around the bottom part of the razor blade to secure it properly to the handle. Dissect on a wood board and use pins from tailor to hold back different tissues.}
\item{Theory: By using fresh anatomical parts from animals, students can get the full experience of each of these organs. Be sure to have students to look into the differences between organs in terms of size, tissues, textures, colors, and more.}
\end{itemize}

\subsection{Volume of the Lungs}
\begin{itemize}
\item{Preparation time: 20 minutes}
\item{Materials: I.V. line, water, large container, if available a large graduated cylinder}
\item{Procedure: Fill a shallow container with water. Fill a small plastic water bottle to the full with water. Quickly invert the water bottle or cylinder into the shallow dish so that the mouth is in the water of the shallow dish. This prevents the water in the water bottle or the cylinder from coming out. Insert an IV line in the cylinder or water bottle. Have a student blow all the air in their lungs through the IV line into the water bottle or cylinder. Record the volume of air in each student’s lungs.}
\item{Theory: It is easy to talk about a human’s breath. In this activity, the students put a value with their breath. As the lungs fill with air, their volume expands. Through displacement of volume, we can find the volume of each student’s lungs. Make a table for each class listing all the volumes of each student. Find the average volume for each student.}
\end{itemize}

\subsection{Fingerprints, Part A – Identification}
\begin{itemize}
\item{Preparation time: 10 minutes}
\item{Materials: Ink pad, white paper }
\item{Procedure: One a piece of white paper, have students make a table with 10 boxes: one for each finger. Using an inkpad, have students carefully cover their fingers with ink. Press each of their fingers one at a time on the paper without smudging the prints. Roll the fingers without moving them in order to create a good print. Have students compare prints with each other (they should notice differences).}
\item{Theory: Each person’s fingerprints are different. It is one of the unique features that can distinguish between people. This is one of the more common techniques used in finding the culprits in crimes.}
\end{itemize}

\subsection{Fingerprints, Part B – Soot Prints}
\begin{itemize}
\item{Preparation time: 10 minutes}
\item{Materials: Small pieces of glass, candles, white paper}
\item{Procedure: If there are no inkpads on hand, fingerprints can be made from the soot of candles. Take a piece of glass and hold it above the candle flame until it becomes black with soot. Have a student carefully press their finger into the soot. Then, press the print on white paper.}
\item{Theory: The soot from candles also works for making fingerprints. This method is more difficult and easier to smudge than ink.}
\end{itemize}

\subsection{Human Symmetry}
\begin{itemize}
\item{Preparation time: 10 minutes}
\item{Materials: mirrors, students}
\item{Procedure: Give students mirrors and use them to identify symmetry of their faces. Have one student put the mirror down half of their face, and see if the other students can see the symmetry.}
\item{Theory: The human body is interestingly symmetric. The easiest place to see symmetry is on the face. This symmetric line runs down from the middle of the face all the way down the body.}
\end{itemize}

\subsection{Peristalsis, Part A – Upside Down Eating}
\begin{itemize}
\item{Preparation time: 5 minutes}
\item{Materials: bread, students, wall, kanga}
\item{Procedure: Have a student stand on their head or hands with their feet on the wall. Be mindful that this could be a problem for students wearing skirts - have a kanga available for students with skirts to wrap around their skirts.]. With a student upside down, give them a small piece of bread and let them chew and swallow.}
\item{Theory: Food does not move through the human body by gravity alone. While gravity may assist, it is not required in order to move food from the mouth to the stomach. Lining the sides esophagus are muscles. These muscles contract to move food along the throat from the mouth to the stomach. These muscles can even move food to the stomach when it has to go up to arrive!}
\end{itemize}

\subsection{Peristalsis, Part B – The Movement}
\begin{itemize}
\item{Preparation time: 10 minutes}
\item{Materials: 1 sock or tire tube, 1 small ball}
\item{Procedure: Hold the sock or tube vertically. Put the ball in the bottom opening. Use your hands to close just underneath the ball to push it farther up the sock. Slowly repeat this motion to move the ball up through the sock or tube.}
\item{Theory: This follow up activity shows what is happening inside the esophagus. The hand motion illuminates the work the muscles are doing in order to move to the stomach. Note that this is the same motion that works throughout the digestive system.}
\end{itemize}

\subsection{Pulse}
\begin{itemize}
\item{Preparation time: 0 minutes}
\item{Materials: students, stopwatch, rubber snakes}
\item{Procedure: Have a student sit and record their pulse. Have a student walk around the school, and record his or her pulse on return. Have a student run for a few minutes and then record his or her pulse. Lastly, scare a student and record his or her pulse. A rubber snake is highly recommended.}
\item{Theory: The heart pumps blood to distribute oxygen to the body. When the body needs more oxygen, the heart begins to pump faster. This is shown by comparing the pulse rate between the different activities: sitting, walking, and running. Sitting requires little oxygen, while walking needs more, and running need even more.The body also pumps more blood depending upon the possible need of oxygen. When humans get scared, the body enters a fight or flight response. In order to run away or fight, the body will need more oxygen. Therefore, when a person becomes scared, the pulse will rise since the body thinks it will need more oxygen.}
\end{itemize}

\section{Cells}
\subsection{Rotting Food}
\begin{itemize}
\item{Preparation time: 5 minutes}
\item{Materials: 2 tomatoes, 1 plastic container with well fitting lid}
\item{Procedure: Take 1 tomato and seal inside a plastic container. Leave a second tomato next to the container. Wait for both to rot. Observe the smell, the color, and texture of the bacteria and fungi growing on the tomatoes.}
\item{Theory: In the air, there are different bacteria and fungal spores. Differences in air circulation, moisture, and temperature favor different organisms – the two tomatoes should look pretty different. In this activity, the tomato in the sealed container will rot faster and possibly have different colors or odors from the decomposition of the fruit.}
\end{itemize}


\subsection{Yeast Fun, Part A}
\begin{itemize}
\item{Preparation time: 5 minutes}
\item{Materials: 1 sealed syringe shell or test tube, balloon, sugar, water}
\item{Procedure: Pour about 5 mL of water in the syringe shell / test tube. Add 5 grams of sugars and 1 gram of yeast. Cover the tube with the balloon. After a few hours, the balloon should fill with carbon dioxide, and the contents of the tube should smell like alcohol}
\item{Theory: Yeast is actually a fungus. This organism eats sugar and breaks it down into alcohols and carbon dioxide. In this activity, this living organism feed on sugar and produce carbon dioxide and alcohol. This process is the fundamental process behind making beers, wines, and other spirits.}
\end{itemize}


\subsection{Yeast Fun, Part B - Temperature}
\begin{itemize}
\item{Preparation time: 10 minutes}
\item{Materials: 3 sealed syringe shells, 3 balloons, sugar, water}
\item{Procedure: Take 1 syringe shell and follow the procedure from Yeast Fun, Part A. Fill a second and third syringe in the similar way. However, heat the second and the third in a water bath. Remove the second one around 50 C, or until it is too hot to keep your hand in the water. Boil the third syringe. Finally, put balloons on all syringes.}
\item{Theory: Biological organisms tend to be vulnerable to temperature changes. Most organisms cannot live in temperatures where the water boils. This is the reason we need to boil water: to kill all bacteria and fungus in water. It is not enough to heat water, but boil it completely; we are demonstrating this fact in this activity. The syringe left at room temperature slowly fills with gas but the syringe that was heated part way should accelerate the work of the yeast. If the yeast is more active at higher temperature, you will see the he balloon on the second syringe fill with gas faster than the room temperature one. However, the syringe that was boiled should produce no carbon dioxide. This is because the temperature of the boiling water was too high and killed the yeast.}
\end{itemize}


\subsection{Yeast Fun, Part C - Food}
\begin{itemize}
\item{Preparation time: 10 minutes}
\item{Materials: 2 sealed syringe shells or test tubes, water, sugar, salt, balloons, yeast}
\item{Procedure: Place sugar, water, and yeast in a test tube. The second test tube, use only salt, yeast, and water. Cap both with a balloon.}
\item{Theory: The proper food is necessary for organisms to grow. The wrong food prevents, and possibly kills organism. The yeast’s normal food is sugars or starches. Salt and sugar are very different. In fact, while the syringe with sugar will produce carbon dioxide as usual, the syringe with salt produces no carbon dioxide because there is no food for the yeast to eat.}
\end{itemize}


\subsection{Yeast Fun, Part D - Light}
\begin{itemize}
\item{Preparation time: 10 minutes}
\item{Materials: 2 sealed syringe shells, water, sugar, yeast, balloons}
\item{Procedure: Fill both syringes similarly as before in Yeast Fun, Part A. Cap both with balloons. Place 1 syringe shell in the sun and place the other in box or dark cabinet.}
\item{Theory: Some organisms require light in order to thrive. Many green leafy plants are a good example. Yeast however does not require sunlight to eat sugar. In this activity, both balloons will fill with carbon dioxide.}
\end{itemize}


\subsection{Yeast Fun, Part E - Living Environment}
\begin{itemize}
\item{Preparation time: 10 minutes}
\item{Materials:  3 sealed syringe shells, sugar, yeast, balloon, methylated spirits, cooking oil.}
\item{Procedure: Prepare one syringe as before in Yeast Fun, Part A. Follow the procedure two more times with variations: instead of water, use methylated spirits in one and cooking oil in the other. Observe the size of the balloon after a few hours. }
\item{Theory: The environments that microorganisms live in are very important. Some organisms have a small threshold for change in the environment. Yeast is a rather tolerant organism and will endure through many different changes in its environment. In this demonstration, the first syringe acts like a control for use in comparing with other variations. The second syringe changes water for methylated spirits. Methylated spirits is a mixture of 70 \% by volume of ethanol to water. Ethanol is a product of yeast fermentation, however at certain concentrations it actually becomes lethal to yeast. This is around 15\% by volume. This second syringe should have a much smaller balloon. This is not due to the yeast formation, but ethanol evaporating and filling the balloon. The last syringe will also have a small balloon or none at all. This is because the oil is not a good medium for yeast to thrive. The sugar becomes bound together in the oil layer while the yeast does not survive well in it. Therefore, little or no fermentation can occur.}
\end{itemize}

\section{Classification}

\subsection{Scavenger Hunt}
\begin{itemize}
\item{Preparation time: 5 minutes}
\item{Materials: --}
\item{Procedure: Prior lessons about classification, find different animals, plants, fungi, or more that is available around the school. Look especially for mosses (wet places) and fungi (decaying material in the shade). After teaching students about classification, send them to find different specimen around the school. If they do not find something you know is there, give them hints about where to look. They will also probably find specimens you have not seen. You could even ask them to bring specimens from near their homes to class the next day. When the students return with their specimens, have them classify what has been found. Keep the best specimens for the school collection.}
\item{Theory: One of the biggest problems about learning classification is that students rarely use the process, and often do not connect it do their daily lives. A scavenger hunt to find examples to classify is a very effective way to encourage students to utilize and retain information about classification. Further, encourage discussion and debate if there are any doubts as students work on classification. Encourage students to think about their specimen, and then have them defend their rationale.}
\end{itemize}


\subsection{Leaf Game}
\begin{itemize}
\item{Preparation time: 20 minutes}
\item{Materials: leaves from around the school}
\item{Procedure: Send students to collect three leaves from each tree around the school. Place them in a pile and let students observe the differences in size, color, texture, and more. Mix up the pile, and time the time it takes students to separate all the leaves. Repeat this experiment, but separate them on based on certain traits, like size or color or texture.}
\item{Theory: Leaves provide an interesting opportunity for students to experience the act of classification. By collecting a pile of different leaves, let students organize the leaves based on different traits. This process is the basics of classification: taking many different organisms, identifying a specific trait, and grouping from them.}
\end{itemize}

\section{Counting in the Ecosystem}


\subsection{Age of Trees}
\begin{itemize}
\item{Preparation time: 0 minutes}
\item{Materials: 1 branch or cut stump. A clean cut is very important.}
\item{Procedure: Look at the cut of the branch. Find the concentric rings. Count each ring.}
\item{Theory: Trees grow slowly, but their age is easy to find. Each concentric ring of the branch or tree relates to 1 growing year. Therefore, each ring represents one year. Use this activity to determine the age of different trees and branches. However, this will not work on fibrous trees, like the baobab tree. Furthermore, proper cutting is important. If you are cutting, use a saw and not a machete.}
\end{itemize}

\subsection{Age of Fish}
\begin{itemize}
\item{Preparation time: 0 minutes}
\item{Materials: Fish}
\item{Procedure: Certain fish have different colored scales depending on the growing season. Find a fish that has different colored bands. Count each repeating color. }
\item{Theory: The growing seasons for fish can sometimes lead to different colors in the scales of fish. This is helpful to learn how long fish have been living. Identifying the bands in a fish can tell students how long some fish have been living.}
\end{itemize}

\subsection{Crickets to Tell Temperature}
\begin{itemize}
\item{Preparation time: 0 minutes}
\item{Materials: crickets, stopwatch.}
\item{Procedure: Wait until crickets are chirping for extended periods of time. Count the amount of chirps in 1 minute. Take the number of chirps, subtract 40, and then divide the subtotal by four. Add 50. This is the temperature in degrees F. Convert to degrees C}
\item{Theory:  Crickets are more or less active depending on the temperature. This is an approximate calculation to determine the temperature from the crickets.}
\end{itemize}

\section{Ecology}
\subsection{Food Chain, Part A - Interconnectedness}
\begin{itemize}
\item{Preparation time: 5 minutes}
\item{Materials: one long bundle of rope or twine, students}
\item{Procedure: Organize the students into a circle. Holding on to the rope tightly at the end, throw the rope to another student. Pull the rope tight and the student throws the rope to another student. These throws do not need to be immediately adjacent; the more cross-circle passes the better. Pull the rope taut after each throw.}
\item{Theory: The food chain illuminates the interconnected nature of all the different living organisms in an ecosystem. Organisms eat other organisms, or the products of the organisms. Generally, organisms are connected to other organisms in very many different ways. In this activity, each student acts is a member of the food chain.}
\end{itemize}

\subsection{Food Chain, Part B - Extinction}
\begin{itemize}
\item{Preparation time: 0 minutes}
\item{Materials: a complete Food Chain, Part Interconnectedness}
\item{Procedure: Once the food chain is complete, have one student let go of all rope.}
\item{Theory: The food chain in an ecosystem is a fragile thing. If one species becomes extinct or dies, then it impacts the entire food chain. In this activity, when a student lets go of the rope it represents the death of a species. The entire chain may not fall apart, but other species may find themselves with only one connection or no connections to the chain. In that case, those students are in jeopardy of extinction themselves. @Alternatively, you can select students sit down but still holding their rope. This means that they died as a species. This causes strain in the system. The more students sit down, the more difficult it is for everyone to keep standing. This can represent the strain of many species going extinct in the food chain}
\end{itemize}

\subsection{Balance of Nature}
\begin{itemize}
\item{Preparation time: 1 hour}
\item{Materials: 2 plastic water bottles, soil, beans, water from a nearby stream or lake}
\item{Procedure: Poke holes in the bottle of the first water bottle. Fill the bottom third of the second water bottle with soil. Plant the bean sprouts in the soil. Cut the top off the second water bottle. Fill it 1/3 full of nearby stream water or lake water. Close the lid on the top bottle. Push the first water bottle into the second bottle. It will fit, and form an incomplete seal. If available, tape the bottles. It will help form a better seal. Open the top cap and thoroughly water the soil. Then seal it shut and place the system where it will receive sunlight. Observe the system over time. }
\item{Theory: Large ecosystems are difficult to contain and control. Smaller ecosystems are better for observing the growth and interaction. In this activity, a small ecosystem is made from the plant growth in the upper water bottle and the liquid environment in the lower bottle. The plants and soil will interact with the water, algae, and organisms in the lower bottle. This ecosystem should sustain itself for more than a few weeks if properly balanced. It might even rain inside.}
\end{itemize}

\subsection{Organisms in Soil, Part A – The Square}
\begin{itemize}
\item{Preparation time: 10 minutes}
\item{Materials: jembe, soil with plants or grass}
\item{Procedure: Carefully dig a small square of soil with a shovel or jembe. Dig down at least 15 cm and drag up the square completely not to damage the layering in the soil. Observe the soil, noting different colors, textures, moisture, organisms and more.}
\item{Theory: It is easy to forget that the ground that we walk on is a rather complex ecosystem. It is possible to observe this ecosystem in action through properly exhuming the soil. The different colors or layers in the soil suggest different types of soil, from loose top soil or darker, deeper soil. Try to identify as many different layers as possible. As time passes, it is possible to identify insects and other organisms growing in the soil.}
\end{itemize}

\subsection{Organisms in Soil, Part B – Forcing Them Out}
\begin{itemize}
\item{Preparation time: 30 minutes}
\item{Materials: 1 large plastic water bottle, 1 smaller plastic water bottle, some mosquito screen, freshly dug soil, lamp}
\item{Procedure: Cut off the top of the large plastic water bottle. This will act as a funnel to hold the soil. Put some mosquito screen in the funnel to hold back the soil. Again, cut off the top of the smaller plastic water bottle. Put the larger lid into the smaller bottle. Add soil to the top funnel. Place a lamp, or out in the hot sun, to heat up the soil. This may take 30 minutes up to 2 hours depending on the heat of the sun or the lamp. Plan ahead of time so that it this activity is ready for observation in the lesson. Observe the small organisms coming out of the soil to escape the heat.}
\item{Theory: In freshly dug soil, lots of different small creatures and organisms live and thrive. This activity shows the large amount of organisms insects that live in a small piece of freshly dug soil. Normally soil an environment has a much lower temperature than the air above it. The organisms prefer living in temperature that is similar to the soil. In this activity, a lamp, or sunlight is employed to heat the soil. As the soil slowly rises, it becomes a non ideal environment for organisms in the soil. From this rise in temperature, the organisms will escape the soil to find a cooler place to live. Observe the exodus of the organisms from the soil heating up.@You might also put some soil in a metal pot and heat gently over a fire. That should do the trick.}
\end{itemize}

\subsection{Terrarium}
\begin{itemize}
\item{Preparation time: 30 minutes}
\item{Materials: 1 square plastic water bottle, rocks, soil, moss, plants, an extra water bottle lid, board, insects if available}
\item{Procedure: Cut a capped plastic water bottle just below the neck, and cut all the way down the side. This allows the bottle to lie flat on a board and lifts like a lid. This allows it to act as a terrarium. Fill with rocks, sticks, moss, insects, or anything else. Be sure to put in an upside down plastic water bottle cap. Fill this cap with water. Place this terrarium some place sunny. Observe this ecosystem over time.}
\item{Theory: Using a clear plastic water bottle allows us to view inside of the terrarium. This becomes a miniature ecosystem. With plants, moss, and possibly bugs, a small temporary community can thrive. Cut the bottle properly so that it lies flat completely and the ecosystem becomes sealed. It may be necessary to poke a small hole or two in the water bottle to allow for air flow. Now you will have a small scale ecosystem and observe the changes in the system over time. An ecosystem is a natural unit consisting of all plants, animals, and micro-organisms (biotic factors) in an area functioning together with all of the non-living physical (abiotic) factors of the environment. A terrarium can be considered as an ecosystem because it has plants, animals (mice, worms, bugs, insects, frogs), bacteria, and non-living physical factors (soil, rocks, leaf litter).}
\end{itemize}

\subsection{Ant Farm}
\begin{itemize}
\item{Preparation time: 10 minutes}
\item{Materials: 1 large glass jar with lids, 1 small glass jar with lids or replace both with plastic water bottles, dirt, ants, 2 plastic water bottle lids, dirt, kanga}
\item{Procedure: Place a small glass jar with the lid on inside the larger glass jar. Go out and find an ant hill. Dig up some ants look for a queen, larva, and eggs. This ant will be larger than the rest and possibly has wings. Be careful if you start digging up siafu or safari ants. Their bites hurt. Fill the larger jar carefully with dirt with the ants. Fill 2/3 of the way full. On top of the soil, put two plastic water bottle lids: one filled half way with water and the other with a small amount of bread, fruit, crumbs, sugar, or honey. Screw on the lid with many air holes – make sure they are smaller than the ants! Observe the activities of the ants. Cover with a kanga or another piece of cloth when not observing.}
\item{Theory: Ants are remarkably industrious insects. Immediately the ants will start to dig out a new home in the dirt. The smaller jar on the inside forces the ants on the outside where we can view their work. Do not forget to feed and water your pets.}
\end{itemize}

\subsection{Pond / Ocean Aquarium}
\begin{itemize}
\item{Preparation time: 30 minutes}
\item{Materials: Large plastic water bottle or glass container, pond / ocean water (not tap water!)}
\item{Procedure: Use a plastic container to collect a fish or other water animal. Collect some rocks or soil from the bottom and gently place it on container. Observe the fish over a course of a day, and then return to the water. Observe the movements, the color, body, and more features of the organism.}
\item{Theory: Unlike the terrarium, the aquarium is not a long term possibility because aquatic organisms often require more oxygen dissolved in the water than a small volume can hold. If you can balance the ecosystem with enough water plants to oxygenate it during the day, but not so many they consume all the oxygen at night, you can keep fish living for a long time, if they are small fish that require little oxygen and subside on the other smaller (invisible) organisms living in your pond water.}
\end{itemize}

\section{Enzymes}
\subsection{Enzymes in the Mouth, Part A - Sweetness}
\begin{itemize}
\item{Preparation time: 10 minutes}
\item{Materials: crackers, bread, or another starchy substance}
\item{Procedure: Give students the bread and have them chew without swallowing. Chew for a long time. As time passes, the students will start to taste something sweet in their mouths.}
\item{Theory: In human mouths, there is an enzyme called salivary amylase. Amylases are enzymes that break down starches into simpler sugars. Remember, starch is a polysaccharide or a compound that is a long chain of sugars. The enzyme cuts the starches into smaller sugars. The students will taste these sugars – hence the sweet taste from the bread as time goes on.}
\end{itemize}

\subsection{Enzymes in the Mouth, Part B - Testing}
\begin{itemize}
\item{Preparation time: 10 minutes}
\item{Materials: crackers, bread, or another starchy substance, iodine}
\item{Procedure: After the students have let the bread in their mouth become sweet from Enzymes in the Mouth, Part A - Sweetness, have them spit out the remaining bread. Put a few drops of iodine on the slush. Also, drop some iodine on an uneaten piece of bread for comparison.}
\item{Theory: Dropping iodine on starch is the food test for determining presence of starch. Iodine binds to starch to form a blue complex. The iodine test will show starch on the bread, but should not show starch on the bread the students spit out. This is because salivary amylase has cut the starches into simple sugars.}
\end{itemize}

\subsection{Catalase, Part A}
\begin{itemize}
\item{Preparation time: 10 minutes}
\item{Materials: 1 plastic water bottle with a lid, yeast, hydrogen peroxide, water}
\item{Procedure: Place about 10 mL of hydrogen peroxide in a small plastic water bottle. Crush the bottle and add some yeast. Cap the bottle and gently invert to make sure all of the yeast is in contact with the hydrogen peroxide. The bottle should expand. Once the bottle is full, test for oxygen using the splint test; roll a piece of paper into a thin tube, lighting it on fire, and then blowing out the flame so the paper is just glowing orange. Open the bottle and put the glowing paper inside near the top – it should relight.}
\item{Theory: In yeast there is an enzyme called catalase. This enzyme is responsible for the break down of hydrogen peroxide into oxygen. Hydrogen peroxide is a poisonous byproduct of metabolism. This enzyme protects the organism by eating the peroxide.}
\end{itemize}

\subsection{Catalase, Part B - Temperature}
\begin{itemize}
\item{Preparation time: 10 minutes}
\item{Materials: 3 plastic water bottles with lids, yeast, hydrogen peroxide, water, glass jar, heat source}
\item{Procedure: This demonstration is slightly different from Catalase Enzyme, Part A. Prepare one bottle just as in Catalase Enzyme, Part A – this is your control. Then take some yeast, place it in water, and boil it. Let the yeast solution cool, and transfer to a plastic water bottle. Add some hydrogen peroxide, crush the bottle, and observe.}
\item{Theory: Boiling the yeast causes catalase – a protein – to denature (lose its special shape). This is the same reason egg white turn white and solid when heated. This means that after boiling, catalase produces little or no oxygen. By doing both procedures, the normal procedure and the boiled procedure, students can see that the normal one produces lots of oxygen and the boiled produces almost none. This shows that biological enzymes are susceptible to temperature. A good comparison between biological catalysts, enzymes, and inorganic catalysts is to do this demonstration. See Oxygen Production, Part A in the Chemistry Section.}
\end{itemize}

\subsection{Catalase, Part C – Other Sources}
\begin{itemize}
\item{Preparation time: 10 minutes}
\item{Materials: 2 water bottles with cap, liver, potatoes, hydrogen peroxide}
\item{Procedure: Into one bottle, cut some potatoes into small pieces. A fresh potato is necessary. Crush the bottle, pour in hydrogen peroxide, and cap the bottle. Repeat this procedure but instead of potato, place small pieces of freshly cut liver instead. }
\item{Theory: Catalase is present in many different locations. It is present in starchy foods; it is the enzyme that allows organisms to convert stored energy, starches, into usable energy, glucose. It is also present in the livers of different animals. In these demonstrations, the catalase present in both potatoes and liver will slowly produce oxygen by decomposing hydrogen peroxide. The bottle will slowly refill. Prove the gas is indeed oxygen by doing the glowing splint test.@In addition, they could compare with cooked potato or liver}
\end{itemize}

\section{Genetics}
\subsection{DNA Extraction}
\begin{itemize}
\item{Preparation time: 20 minutes}
\item{Materials: salt, soap (liquid hand soap is best, but laundry soap also works), water, methylated spirits}
\item{Procedure: Use drinking water to prepare a saturated salt solution – there should be a little extra salt on the bottom of the container after mixing. Have students swish the solution in their mouths for at least 60 seconds - 90 is better if the students can take it - and then spit the solution into a small container. Add about a third of this spittle to a soap solution. Gently rock the bottle back and forth for two to five minutes. Finally, carefully pour methylated spirits down the inside of the container so it forms a separate layer on top. Add no more than ¼ of the total volume as methylated spirits. Transparent strings of DNA should precipitate at the boundary between the two layers.}
\item{Theory: Salt provides the DNA with a favorable environment; it contributes positively charged atoms that neutralize the normal negative charge of DNA. In the experiment, the enzymes in the soap are breaking down the lipid molecules of the cell and nuclear membranes, releasing the contents of the cell, including the DNA. These enzymes in the soap are what break down grease while washing dishes.@In this experiment, the DNA will slowly rise from the watery lower layer up into the alcohol layer above it. The DNA will look stringy and have small bubbles attached to it. It will be a clear substance and may be hard to see. You may slowly twist this substance onto a toothpick. (Do not scoop up cell scum from the lower layer.)}
\end{itemize}

\subsection{Mendelian Genetics}
\begin{itemize}
\item{Preparation time: 30 minutes}
\item{Materials: lots of beans and maize seeds}
\item{Procedure: Provide an ample but equal amount of beans and maize seeds to each student group. The beans represent the dominant allele (Z) and the maize seeds represent the recessive allele (z). In this activity, students are going to cross two heterozygotes (Zz x Zz). Let students make a mixture of Zz, in this case, 50\% beans and 50\% maize seeds. Label this the Mother. Repeat this to make the Father pile. In order to make the offspring, take one seed from each pile. Repeat this procedure at least 20 times and record each off spring and its genotype.}
\item{Theory: In sexual reproduction, each parent gives the offspring one copy of each gene. Thus for a particular gene “zed” every individual has two copies of the gene, one from the mother and one from the father. In this activity, we assume that there are only two alleles for this trait, Z, a dominant allele, and z, a recessive allele. If an offspring has ZZ, Zz, or zZ, it will have the trait associated with Z. If an offspring has zz, it will have the trait associated with z. Homozygous means having the same alleles, ZZ or zz. Heterozygous means different alleles, Zz or zZ. An example of a dominant trait in humans is polydactyl - an extra finger on the hand. An example of a recessive trait is cycle cell anemia. Another example is eye color – blue eyes are recessive while brown are dominant, but the gene for blue eyes is so rare in many countries that this example is not very helpful.}
\end{itemize}

\section{Germ Growth}
\subsection{Making Petri Dishes}
\begin{itemize}
\item{Preparation time: 1 hour}
\item{Materials: 5 shallow plastic or glass dishes, 1 bouillon cube (any flavor), sugar, water, 1 envelope of gelatin}
\item{Procedure: Boil together 250 mL of water. Add 10 grams of sugar and 1 bouillon cube. Once mixed thoroughly, remove from heat. Let the solution cool for a few minutes and then add 1 envelope of gelatin. Mix completely then divide among different shallow containers. Let the containers cool in a cold place or overnight. If available, use a fridge.  }
\item{Theory: The gelatin mixture forms a nice gel that is a perfect place for germ growth. There are proteins, salts, sugars, and water. This is a prime growing ground for bacteria. Use these Petri dishes as a starting point for other germ activities.}
\end{itemize}

\subsection{Why Wash Your Hands}
\begin{itemize}
\item{Preparation time: 10 minutes}
\item{Materials: Petri dishes, water, soap, dirty hands}
\item{Procedure: For each group, give the students 3 Petri dishes. They should have their lids on until use. One Petri dish should not be contaminated; it is the control. Have one student with dirty hands touch one Petri dish many times. Have the very same student wash their hands vigorously for 5 or 10 minutes, with soap Watch them, and ensure that they do a good job of washing their hands. Rinse their hands and have them touch the third Petri dish. Cover the dishes with clean lids and put aside in a place that will keep them safe. Observe what happens over the next few days or over the weekend.}
\item{Theory: Most people wash their hands quite infrequently. This demonstration is meant to teach people the reason for hand washing. Ideally, students will observe that while the second dish grows many bacteria, the third dish looks like the first – no or very little bacterial growth. The bacteria growing might not be the same ones that cause disease in people, but you can assure students that hand washing has the same effect on pathogenic bacteria. If everyone just washed their hands before eating, touching their faces, and every time after visiting the toilet, there would be much less disease in the world.}
\end{itemize}

\subsection{Bacteria in humans}
\begin{itemize}
\item{Preparation time: 30 minutes}
\item{Materials: Petri dishes, q tips or cotton swabs}
\item{Procedure: In this demonstration, we need many Petri dishes. Have students swab various parts of the body, e.g. nostrils, ears, mouth, hands, armpits, knees, feet, in between toes, hair, with a q tip. Then brush each q tip on a different Petri dish. For each dish, make several strokes in one direction and then several strokes in a perpendicular direction. Observe the dishes over the next few days to see the growth. Be sure to have a control dish.}
\item{Theory: Bacteria grow all over the human body. Different bacteria thrive in different environments on the human body. Utilizing this fact, many different Petri dishes can culture the different bacteria growing all over the human body. Identify the different Petri dishes and compare their growths. Look at different colors, shapes, even number of different colonies. From this information, the students can identify which parts of the body have greater amounts of bacteria growing in them. Further, this should tell the students where they need to wash, or wash more vigorously, on their bodies.@This demonstration shows the utility of skin in protecting the human body from infection. There may be bacterial all over our outsides, but they cannot invade healthy skin. Some of these bacteria cause acne when they grown in a clogged sweat pore. More importantly, many of these bacteria can cause serious infections if the skin is cut. This is why it is always important to wash cuts thoroughly with soap. This is why doctors use sterile needle and use iodine or alcohol to clean the skin before giving injections. @This demonstration can be hybridized with Why Wash Your Hands by taking swabs on one part of the body, then wash it, and swab it a second time. For example, take swaps in between a students toe. Have the student wash their feet and toes quite well and swab a second time. Observe the different bacteria growths. }
\end{itemize}

\subsection{Germs in Other Organisms}
\begin{itemize}
\item{Preparation time: 1 hour}
\item{Materials: Petri dishes, 1 specimen like a frog, q tips}
\item{Procedure: Dissect a frog to open up the different parts of the body: brain, heart, lungs, digestive system, and skin. Swap each part of the frog with a q tip and transfer to a Petri dish. Observe the dishes over time to see the different colonies in each dish from the different parts of the body. Be sure to have a control. A frog is not necessary, in fact it is recommended to do this with a variety of different organisms to see what bacteria are growing in their bodies. }
\item{Theory: Bacteria grow all over every living organism. By using the Petri dishes, the different bacteria cultures can be identified in different organisms. Further, this can be used to analyze which parts of the bodies contain bacteria. Find which parts. }
\end{itemize}

\subsection{Germs All Around}
\begin{itemize}
\item{Preparation time: 10 minutes}
\item{Materials: Petri dishes, q tips}
\item{Procedure: Use the q tips to swab different items in the classroom or the school. Swab desks, the chalkboard, the door handle, the door to the restroom, the choo, and more. Let the students test everything they want. Observe the bacteria growth over the next few days. Be sure to have a control.}
\item{Theory: Not only do bacteria live all over organisms, they also grow over everything in the world. This can be used to create bacteria colonies from all over the school, and to show which places have the most. This should also tell the students where they need to wash their hands after visiting in order to reduce the amount of bacteria that starts to grown on their hands. }
\end{itemize}

\section{Home Microscopes}

\subsection{Water Magnification}
\begin{itemize}
\item{Preparation time: 5 minutes}
\item{Materials: water, clear glass test tube}
\item{Procedure: Fill the test tube with water. Put your thumb over the top of the test tube and turn it horizontal. Look through the water in the test tube to whatever needs magnifying.}
\item{Theory: The light that passes through water refracts and gives rise to a magnifying effect. This magnifying effect is not terribly strong; maybe it magnifies things by a factor of two. This is better than nothing in terms of seeing small things.}
\end{itemize}

\section{Other Educational Resources}

\subsection{Field Trip}
\begin{itemize}
\item{Preparation time: --}
\item{Materials: --}
\item{Procedure: Take students to see the local dispensary or hospital. Ask doctors if students can see slides of malaria infected blood or slides showing worms. Be sure to coordinate ahead of time with the hospital or dispensary. }
\item{Theory: Many schools do not have a microscope available for use. However, dispensaries and hospitals almost certainly have them. If the dispensary is both nearby and open to hosting a field trip, the hospital can provide a real life example of the lessons in your biology class. }
\end{itemize}

\subsection{Guest Speaker}
\begin{itemize}
\item{Preparation time: --}
\item{Materials: --}
\item{Procedure: Invite a doctor or nurse to the school to talk about a variety of different topics that coincide with the lessons. Some possible topics are basic hygiene, disease prevention, malaria prevention, AIDS prevention, and the importance of boiling or otherwise cleaning water.}
\item{Theory: Guest speakers provide an excellent opportunity for the students to learn more about the importance of the lessons being taught. Not only that, guest speaks provide a life path or even a role model for many students to aspire towards. Be careful, some guest speakers are better than others.}
\end{itemize}

\section{Plants}

\input{./tex/activities/chlorophyll-colors.tex}

\subsection{Seed Germination}
\begin{itemize}
\item{Preparation time: 10 minutes}
\item{Materials: seed or beans, small paper cups or bottles of small plastic water bottles or plastic bag containers with holes in the bottom, water}
\item{Procedure: Plant some seeds in a handful of different small cups. Be sure to use some good soil. Water the plants every day. Each day, take out one of the seeds. Have the students identify the different stages of germination.}
\item{Theory: Since seeds sprout underground, it is hard to see what is happening to the seed as it germinates. By uprooting a seed after each day, every stage of the germination can be observed and sketched. Ask students to classify the germination as hypgeal or epigeal . Have students draw pictures of the sprouting seeds and identify each part.@When a bean seedling emerges from the soil it is curved (this is the hypocotyl) and it pushes through the soil. As the seedling continues to grow, the hypocotyls straightens and carries the cotyledons and the plumule above the soil surface. This type of germination, where the cotyledons are carried above the soil, is called EPIGEAL germination. Examples: bean seeds, castor oil seeds, groundnuts, cotton, and bambara nuts. A few monocotyledonous seeds, such as onions and lilies, exhibit epigeal germination.@Germination of a maize seed follows a different pattern from that of a bean seed. The plumule pushes its way out of the soil while the cotyledon remains underground. The plumule does not form a hook as in bean seeds. This type of germination in which the cotyledons remain underground is called HYPOGEAL germination. Examples: maize seeds, wheat, sorghum, and millet. A few dicotyledonous seeds, such as kidney beans and broad beans, exhibit hypogeal germination.}
\end{itemize}

\subsection{Avocado Germination}
\begin{itemize}
\item{Preparation time: 10 minutes}
\item{Materials: Avocado pits, water, small cups or bottoms of plastic water bottles with holes in the bottom, toothpicks}
\item{Procedure: Fill the container mostly full of water. Using toothpicks, keep the avocado seed half under water and half exposed. Each day, examine the avocado seed.}
\item{Theory: This germination is an example of hydroponic germination. Instead of soil, the pit or seed is sprouted by water, air, and sunlight only. In this example, it is easy to see the changes the seed undergoes.}
\end{itemize}

\subsection{Potato Germination}
\begin{itemize}
\item{Preparation time: 10 minutes}
\item{Materials: Potato, water, cups with holes in the bottom, soil}
\item{Procedure: Follow the same procedure as Avocado Germination, except with a potato. After the potato sprouts, bury it in the ground and wait for the potato plant to grow.}
\item{Theory: Avocadoes are very difficult to sprout and grow into a tree. However, potatoes make great substitute to see the plant go from beginning stages until complete plant growth.}
\end{itemize}

\subsection{Discovering Factors Affecting Plant Growth, Part A - Soil}
\begin{itemize}
\item{Preparation time: 30 minutes}
\item{Materials: beans, water, plastic water bottle bottoms or cups with holes in the bottom, top soil, deeper soil, sand}
\item{Procedure: In one water bottle bottom, fill 2/3 full of top soil from the surface of the ground. In a second bottle bottom, repeat but take a darker soil from 1 meter down from the surface. In a third, repeat but fill with sand. In a forth, fill with a 50:50 sand soil mixture. Plant each with 2 bean seeds with at least 5 cm apart. Place in a location with lots of sunlight. Water gently each day. Observe growth each day over the course of 2 weeks, paying particular attention to date of sprouting, size, color, condition and more.}
\item{Theory: Soil has a major effect on seed germination and plant growth. Soil contains a small ecosystem that allows the seeds to flourish; healthy soils lead to healthy plants. Healthy soil has the proper pH balance, water, salinity, nutrients, and microorganisms to promote plant growth. Every soil is different. Sand, for example, has few of the important factors involved in germination. By varying the soil factors, students can learn about how what factors are involved in the germination process. Connect discussion of different kinds of soil to environmental issues, like the relationship between desertification and plant germination}
\end{itemize}

\subsection{Discovering Factors of Plant Growth, Part B - Salinity}
\begin{itemize}
\item{Preparation time: 30 minutes}
\item{Materials: beans, water, plastic water bottle bottoms or cups with holes in the bottom, top soil, deeper, salt}
\item{Procedure: In the first and second plastic bottle bottom, fill 2/3 full with soil. In the third and forth plastic bottle bottom, fill 2/3 with soil mixed with 100 grams of salt each. Plant each with 2 seeds at least 5 cm apart. Water the first and the third plastic bottle bottom with normal water every day. Water the second and the forth plastic bottle bottom with a salt water solution made from 100 g salt in 1000 mL of water. . Observe growth each day over the course of 2 weeks, paying particular attention to date of sprouting, size, color, condition and more}
\item{Theory: Salts and other electrolytes are necessary to live. However, too much salt makes the soil inhospitable for plants to germinate or grow. Ask students why this is so – get them thinking about osmotic pressure, why they get thirsty after eating salty food, and why it hurts to get salt in cuts. In the first variant, the plant should grow well. The second variant has normal soil but is watered with salt water. This plant will suffer from the added salinity from the watering. The third variant is watered with normal water, but is trying to grow in soil with high salinity. This plant will also not thrive. Lastly, the forth variant has a plant trying to grow in saline soil with salty water. Sadly, this plant has no chance at survival.}
\end{itemize}

\subsection{Discovering Factors of Plant Growth, Part C - Water}
\begin{itemize}
\item{Preparation time: 30 minutes}
\item{Materials: beans, water, plastic water bottle bottoms or cups with holes in the bottom, soil}
\item{Procedure: Have each group of students fill 5 plastic bottle bottoms 2/3 full with soil and plant with 2 beans 5 cm apart. In the first cup, do not water at all. In the second cup, water the cup with 25 mL of water once a day. For the third cup, water with 100 mL once a day. In the forth cup, water with 50 mL twice a day. In the last cup, water with 50 mL four times a day. . Observe growth each day over the course of 2 weeks, paying particular attention to date of sprouting, size, color, condition and more.}
\item{Theory: This activity explores different watering amounts for plant growth. Every plant requires varying amounts of water to survive. Some grasses need very little, some plants need almost complete watering. In this demonstration, all the variants will have different growth rates. Some may even receive too much water.}
\end{itemize}

\subsection{Discovering Factors of Plant Growth, Part D - Sunlight}
\begin{itemize}
\item{Preparation time: 10 minutes}
\item{Materials: beans, water, plastic water bottle bottoms or cups with holes in the bottom, soil}
\item{Procedure: Have students fill 2 plastic water bottle bottoms 2/3 full with soil and plant 2 beans 5 cm apart. Place one seedling in a location that receives sunlight. Put the other in a cupboard, under a bucket, or in another place that receives no sunlight. Observe growth each day over the course of 2 weeks, paying particular attention to date of sprouting, size, color, condition and more.}
\item{Theory: Green leafy plants require sunlight to germinate and grow properly. If the seedlings are denied sunlight, their growth will be stunted or even nonexistent. However, not all plants require the same amount of sunlight. Some plants only need a few hours of indirect sunlight at most while some require 6 or more hours of direct sunlight. As a variation on this experiment, keep some plants in the dark for most of the day but receiving various controlled amounts of sunlight – one hour, two hours, etc. Make sure to water all plants throughout the experiment!}
\end{itemize}

\subsection{Discovering Factors of Plant Growth, Part E - Chemical Fertilizers}
\begin{itemize}
\item{Preparation time: 10 minutes}
\item{Materials: beans, water, plastic water bottle bottoms or cups with holes in the bottom, soil, fertilizer}
\item{Procedure: Have students fill 2 plastic water bottle bottoms 2/3 full with soil and plant 2 beans 5 cm apart in each one. In one of the bottoms, mix in a small amount of fertilizer in the soil before planting. Water each day. Observe growth each day over the course of 2 weeks, paying particular attention to date of sprouting, size, color, condition and more.}
\item{Theory: Plants require certain nutrients for growth. Fertilizers provide some of these nutrients in a direct and highly concentrated form. Soils with fertilizer may allow plants to grow much faster than they would otherwise. Have students discuss the implication of this to farming. You might also put a lot of fertilizer in another container – use enough and it will kill the plant (“fertilizer burn”), another good lesson with relevance to farming, and for teaching about moderation in general.}
\end{itemize}

\subsection{Soil Retention}
\begin{itemize}
\item{Preparation time: 20 minutes}
\item{Materials: different examples of soil, packed, loose, sand, clay, and more, one coffee can, water}
\item{Procedure: Remove both ends of a coffee tin. Place the tin on different soils. Fill with water, and observe the time required for the water to drain out of the tin completely. Repeat this with different types of soil.}
\item{Theory: Water moves through different soils at different rates. This ability for soil to let water move or hold on to water is important for plants to grow. If water passes too quickly, like soil, plants have no chance at grabbing water. If water does not pass at all, it may be too packed for plants to grow easily in the soil.}
\end{itemize}

\subsection{Tropic Movements}
\begin{itemize}
\item{Preparation time: 10 minutes}
\item{Materials: bean seedlings in plastic water bottle bottoms or cups with holes in the bottom, box,}
\item{Procedure: In the box, put one medium sized hole in the top. Put two or three seedlings in the box. Be sure that one is directly under the sunlight while the rest around the center plant so they receive directional sunlight. Water each plant every day and observe the growth of the seedlings.}
\item{Theory: Green leafy plants grow in a direction to maximize the exposure to sunlight. The seedling directly under the sunlight will grow upwards like normal. However, each of the plants that surround the center plant will grow towards the center plant so that it will increase their exposure to sunlight. }
\end{itemize}

\subsection{Seed Window}
\begin{itemize}
\item{Preparation time: 30 minutes }
\item{Materials: 1 plastic cup or plastic water bottle bottom, soil, water, bean seeds}
\item{Procedure: Cut small segments around the outside of the water bottle and put a few holes in the bottom. Use tape or paper clips to hold the segments on the plastic bottles. Fill the bottom of the plastic water bottle with soil. Plant bean seeds around the outside of the plastic water bottle container. Water the plants. As the plants grow, open up the flaps and see how the plant is working its way through the soil.}
\item{Theory: The purpose of the flaps on the germination container is to view the bean seeds as they germinate in the soil. Taking the seeds outside the soil is nice, but it does not help to see what it looks like in the soil. In this case, the flaps allow to view the germination in process}
\end{itemize}

\subsection{Making a Greenhouse}
\begin{itemize}
\item{Preparation time: 10 minutes}
\item{Materials: 1 large plastic water bottle, 2 small plastic water bottle, soil, bean seeds, water}
\item{Procedure: Cut off the top from both water bottles. Put a few holes in the smaller water bottle. Fill lower bottle 2/3 full with soil and plant bean seeds. Water and turn the large plastic water bottle upside and cover the smaller plastic bottle. Use the second smaller plastic water bottle}
\item{Theory: A green house works by containing heat and moisture around the plant. In a slightly warmer, moister environment, plants will grow faster. This demonstration is most effective in the cold season and in cooler regions.}
\end{itemize}

\subsection{Leaf Outlines}
\begin{itemize}
\item{Preparation time: 10 minutes}
\item{Materials: white paper, leaves, pencils.}
\item{Procedure: Have every student take a leaf and place a piece of paper over it. They should gently run a pencil over the paper. As the pencil colors the entire page, the border and veins of the leaf become visible. Repeat this process for different leaves, and create a book of all the different leaves around the school.}
\item{Theory: The structure of the leaves of different trees will also be different. This activity is a good way to start to look at the differences in the trees around the school and how the leaves are different. Let the students keep a book of all the different trees from the school.}
\end{itemize}

\section{Transport}

\subsection{Powder Diffusion}
\begin{itemize}
\item{Preparation time: 0 minutes}
\item{Materials: powdered food coloring or kool-aid like product, water, plastic water bottle}
\item{Procedure: Fill the plastic water bottle with water. Quickly add the powdered food color, but do not shake. Observe the color diffuse through the water.}
\item{Theory: Mixing does not occur immediately. Without shaking or stirring, it occurs slowly. By using a colored compound, it is easy to see how the molecules are slowly dissolving into the solution. }
\end{itemize}

\subsection{Orange Diffusion, Part A – Sweet Smells}
\begin{itemize}
\item{Preparation time: 5 minutes}
\item{Materials: one orange or other citrus fruit}
\item{Procedure: Have students sit in their seats. Start to peel the orange. When students begin to smell oranges, have them raise their hands. Be sure the students only raise their hands as they smell the orange and not before.}
\item{Theory: Diffusion happens in not only liquids but also gases. Peeling oranges or other citrus fruits releases small compounds that diffuse through gases. When these compounds come in contact with out noses, we smell oranges. However, we cannot smell oranges immediately on peeling; the compounds must migrate towards our noses. In this case, the compounds will slowly diffuse in the classroom with the students closest to the orange smelling it first. The students in the back of the classroom will smell it last. The effects of wind should be considered.}
\end{itemize}

\subsection{Orange Diffusion, Part B - Trapped}
\begin{itemize}
\item{Preparation time: 5 minutes}
\item{Materials: a box, one orange or other citrus fruit}
\item{Procedure: Turn the box upside down. Without turning the box over, peel the orange inside of the box. When students begin smelling oranges, have them raise their hands.}
\item{Theory: Diffusion can only occur when the molecules can move freely. Some objects will not allow compounds through. In this activity, the cardboard box prevents the compounds in the orange to diffuse out through the classroom. This time, students not smell oranges or it will take a long time for students to start smelling.}
\end{itemize}

\subsection{Osmosis}
\begin{itemize}
\item{Preparation time: 10 minutes}
\item{Materials: 1 potato or carrot, water, salt, two water bottle bottoms}
\item{Procedure: Cut two equal sized pieces of potato. Put one piece is normal water and the other in a salt-water solution. Observe over the next few hours.}
\item{Theory: In all cells and plants, there is a proper balance of different concentrations of salts and sugars. Osmosis is the process where the salts move from a high concentration either to a low concentration or where water moves from a low concentration to a high concentration. In this activity, placing the potato in pure water will cause the potato to swell. Inside the potato, there is a higher concentration of salts and sugars compared to the water surrounding it. The water moves into the potato in order to make the concentrations inside the potato more similar to the water. The potato swelling is visual evidence of this phenomenon. The potato in salt water has exactly the opposite effect. The concentration of salts inside the potato is much lower compared to the concentration of salt in the water surrounding the potato. The water in the potato moves out of the potato to dilute the salt solution. }
\end{itemize}

\subsection{Water Transport in Flowers}
\begin{itemize}
\item{Preparation time: 15 minutes}
\item{Materials: white flowers with stems, food coloring, water}
\item{Procedure: Cut on the bias along the bottom of the stem of a flower. Place this flower in colored water. Observe over the next few days.}
\item{Theory: Plants need to move water into its flowers. In this activity, the white flowers will change color to match the color of the colored water. This activity should work with colored flowers; however, it will be much more difficult to see the color change with the transport.}
\end{itemize}

\section{Trash and Pollution}
\subsection{Pollution Catcher}
\begin{itemize}
\item{Preparation time: 15 minutes}
\item{Materials: coffee filters, paper towels, sticks}
\item{Procedure: Roll the coffee filters or paper towels into a cone. Attach to a stick. Place these around the school or other locations on a dry sunny day, for example near trash burning pits, around the school kitchens, or near roads. Collect the cones at the end of the day. Observe what you find caught in the paper. Identify the most polluted areas around the school.}
\item{Theory: Air pollution is a common problem. This activity uses paper towels to catch whatever is floating in the air. Certain activities produce pollution and this pollution is sometimes harmful. For example, the smoke from cooking fires causes eyes to tear and people to cough; the smoke from burning trash smells bad for you because it is.}
\end{itemize}

\subsection{Trash Journal}
\begin{itemize}
\item{Preparation time: 60 minutes}
\item{Materials: each student needs a notebook, balance.}
\item{Procedure: Have each student create a journal. In this journal, students need to write down all of the trash that they make every day for 2 weeks. If possible, have students collect all of their trash and weigh it every day. }
\item{Theory: Trash is a rather interesting problem facing society today. Too many goods have small plastic wrapping or pieces that are cast aside easily. This means it is very likely people do not understand the volume or weight of trash that each person uses every day. This journal is used to encourage students to be mindful of the waste they make. This activity would be very interesting to compare students of different locals, like the trash generated by students from villages compared to towns. How much of the trash produces is biodegradable? Help students to discuss the difference between different kinds of trash – how they are disposed of, how long they remain in the environment, what the effect is of burning them, etc.}
\end{itemize}

\subsection{Landfill Jar}
\begin{itemize}
\item{Preparation time: 10 minutes}
\item{Materials: trash of different types like plastic or food waste or metal, glass jar with lid, soil}
\item{Procedure: Fill the jar with a mixture of trash and soil. Trash can be paper, food scraps, plastic, metals, etc. Close the jar with the lid. Observe the jar and identify which trash remains after a week, a month, a year.}
\item{Theory: Trash is not just plastic trash. Trash can be food scraps or metals. Therefore, trash is divided into two general categories: biodegradable and non-biodegradable. Biodegradable trash will break down in the environment. Non-biodegradable trash does not. That is a problem with many types of plastic trash; it does not break down. Since it does not break down, it just piles up. Generally, this can be seen on the sides of roads when people just throw plastic water bottles out of the window.}
\end{itemize}

\subsection{Backwards Garden}
\begin{itemize}
\item{Preparation time: 10 minutes}
\item{Materials: area for garden, trash}
\item{Procedure: Follow the same procedure as the Landfill Jar, but instead of planting in a jar, plant outside in a garden bed. }
\item{Theory: This activity is very similar to Landfill Jar. Proper disposal of trash is very important. Without handling of trash properly, it will build up in the environment. That is both harmful to the environment and humans. Use these activities to teach students about proper stewardship of their environment and to deal with trash properly.}
\end{itemize}

\chapter{Chemistry Activities}

\section{Black Light}

A black light is a lamp whose light has wavelengths between 400 and 315 nm. They have some mercury atoms surrounded by helium so that when a high voltage is applied, the lamp produces photons in the ultraviolet spectrum. This system will release UVA and UVB light. UVB light is cancer causing, because camphors painted on the inside of the bulb absorb the UVB light and fluoresce visible purple, hence the color of these lights.@Black lamps are useful because some chemical compounds fluoresce when hit with UV light. Fortunately, in Tanzania black lights are very common – just about any purple tube light in town. If some white object looks unnaturally white, the light hitting it is probably from a UV lamp. Ask the store or bar owner where the light was purchased, and track one down. A small lamp should not be very expensive. @Black lights are particularly useful for helping students understand and explore the concept of light waves, especially as lasers and other optical devices are more difficult to obtain. The properties of fluorescence helps to explain that there are indeed different types of lights. Under normal light, the objects behave normally; under UV light, things change. This section is devoted to helping students experience different types of light and to talk about electronic transitions.

\subsection{Secret Messages}
\begin{itemize}
\item{Preparation time: 0 minutes}
\item{Materials: black light, yellow highlighters}
\item{Procedure: Let students write on their arms with a yellow highlighter. The writing should be difficult to read over dark skin pigments. In a dark place, turn on a black light and bring it close to the highlighter writing. The writing will be very easy and clear to read}
\item{Theory: When some compounds have light shown on them, they can fluoresce. Fluorescence is a series of electron transitions. Much like Balmer series in the hydrogen spectrum, fluorescence is electrons moving from high energy levels to low energy levels and releasing the extra energy as light. In the Balmer series with a hydrogen atom, the photons are released from a single electron movement, from the fourth energy level to the second, for example. Fluorescence has intermediate steps that release photons in the visible spectrum. The electrons absorb light in from photons in the UV spectrum. If the electrons move from one energy level and then drop to the initial energy level, then the photons released would have the same wavelength as the UV light. In other words, it would not be possible to see the light released from the electron transition. That is not the case since some compounds show pretty intense light. What actually occur are intermediate energy transitions; the electrons fall to a nearby energy level before falling down to the initial energy level. Each of these transitions last for a short time and the photons released from these shorter transitions may have wavelengths in the visible spectrum. This is the reason there are some interesting colors in different objects.}
\end{itemize}

\subsection{Chlorophyll Colors}
\begin{itemize}
\item{Preparation time: 10 minutes}
\item{Materials: green leaf, konyagi, 2 jam jars, pen, black light}
\item{Procedure: We need to extract the chlorophyll from a green leaf. Tear the green leaf into small pieces and place into a jam jar. Add enough konyagi to cover the leaves. Using the bottom end of a pen, mash the leaves in the alcohol. The idea is to break the plant cellular structure so that the chlorophyll comes out. After mashing for five minutes, let the leaves sit in the alcohol for another five minutes. Pour off the liquid but remove any green leafy pieces. Take the liquid and place near a black light. The solution will glow red.}
\item{Theory: Chlorophyll is the compound that lets plant absorb visible light for as energy in a cell. This energy is stored as glucose. This is due to a large modified conjugated heme group with a magnesium atom. A conjugated system is a system of bonding orbitals that have similar energies. This allows the electrons to travel not just in its own bond, but also into other adjacent bonds of similar energies. The electrons are free to travel and this allows them to absorb different wavelengths of light and even fluoresce. This modified heme group not only absorbs normal visible light, but it also fluoresces. Under UV light, the green chlorophyll produces a red color. The ability for the electrons to move so easily under UV light is the exact reason plants can convert solar energy into sugars. By moving electrons around, the plant can store electrons for future use in the form of starches.}
\end{itemize}

\subsection{Quinine Blue, Part A – The Tonic}
\begin{itemize}
\item{Preparation time: 5 minutes}
\item{Materials: bottle of tonic water, black light}
\item{Procedure: Under visible light, the tonic water is clear. Place the bottle of tonic water near the black light. The tonic water will glow a slight blue color.}
\item{Theory: One of the flavorings of tonic water is quinine. Quinine is a common medicine against malaria. Quinine has a conjugated system of pi-orbitals that give the molecule molecular orbitals with electron transitions able to fluoresce when hit by UV. This fluorescence is in the blue spectrum. Note that there are no electron transitions that are forced by visible light – hence quinine has no visible color under normal light.}
\end{itemize}

\subsection{Quinine Blue, Part B – Blue Water}
\begin{itemize}
\item{Preparation time: 20 minutes}
\item{Materials: water, clear plastic water bottle, tablet of quinine, black light}
\item{Procedure: Dissolve one or two tablets of quinine in some clean water. Shake well and let the solution sit to allow the tablets to dissolve completely. Place the bottle near a black light. It will glow blue.}
\item{Theory: The quinine from the medicine tablets dissolve readily in solution. The solution glows a bright blue color. This is the same color as the tonic water. The water is has no color in visible light, but when the UV light is applied, the quinine fluoresces brightly.}
\end{itemize}

\subsection{Petroleum Jelly Glow}
\begin{itemize}
\item{Preparation time: 0 minutes}
\item{Materials: petroleum jelly, black light}
\item{Procedure: Let the students put some petroleum jelly on their hands and then have them walk over to the black light. Their hands will glow slightly purple.}
\item{Theory: Petroleum jelly consists of long hydrocarbon chains that are one of the last products of crude oil distillation. Its molecular mass is very high and it has some conjugation in its structure. For reasons similar to those of quinine, these conjugated systems absorb UV light and fluoresce a purple light. }
\end{itemize}

\subsection{Alum Crystals Fluorescence }
\begin{itemize}
\item{Preparation time: 3 hours}
\item{Materials: same supplies from Growing Giant Crystals, fluorescent highlighters, black light}
\item{Procedure: Follow the same procedure as Growing Giant Crystals. When adding the alum to the hot water, add the ink of a water-soluble fluorescent highlighter. When the crystal is formed, let it dry completely. Once dry, place near a black light. The crystal should glow in the UV light.}
\item{Theory: As the crystal cools, some impurities will be trapped inside the crystal. This time, we want impurities trapped in the crystals. The fluorescing agent from the highlighters will be trapped inside the alum crystals. Even though it is inside of the crystals, it will still fluoresce when placed near a black light. The alum crystal will glow the color of the highlighter when it fluoresces. It is a good comparison to have two highlighters of the same color: one for use in making the crystal and one to show that the fluorescing light of the crystal is the same as the highlighter.}
\end{itemize}

\section{Chromatography}

\subsection{Paper Chromatography}
\begin{itemize}
\item{Preparation time: 5 minutes}
\item{Materials: Jam jar with lid, white paper, colored markers, alcohol either methylated spirits or isopropyl}
\item{Procedure: Cut a thin piece of paper so it forms a 6 cm by 2 cm rectangle. One cm from end of a paper, put a mark with the marker. Put 10 mL of alcohol in the bottom of the jam jar. We want to add just enough to have a small layer of solvent at the bottom, which does not reach the marker spot. Add the paper, and cover the lid. After a few minutes, the solvent will climb the paper and separate out the inks in the marker.}
\item{Theory: Chromatography is the process where compounds are separated by different solvents. In this case, the alcohol acts as the solvent. The solutes, the different pigments that combine to give the marker its color, dissolve in the alcohol. However, only some of the compounds in the marker dissolve in the alcohol. As the alcohol climbs up the paper, it will carry the dissolved compounds. Depending on how well the compounds in the marker ink dissolve in the alcohol the colored compounds will move up to different heights on the paper. The better the compound dissolves in the alcohol, the farther the ink will move. With markers, at least four different colors can be separated out from the marker ink. }
\end{itemize}

\subsection{Chalk Chromatography}
\begin{itemize}
\item{Preparation time:  5 minutes}
\item{Materials: Jam Jar with lid, white chalk, colored markers, alcohol either methylated spirits or isopropyl}
\item{Procedure: Follow the same procedure as Paper Chromatography, except instead of using paper, use a piece of chalk. Take the piece of chalk out when the solvent has reached the top of the chalk and let it dry. The chalk remains usable for writing on blackboards.}
\item{Theory: This style of chromatography has two distinct phases: a stationary solid phase and a mobile liquid phase. The solid phase is chalk, which is calcium sulfate, an ionic solid. The mobile phase is the alcohol, a non-polar liquid. The solute, the marker ink, has both polar and non polar compounds. The polar compounds are attracted to the ionic chalk while the non-polar compounds dissolve in the alcohol. As the alcohol moves up the piece of the chalk, the different colored compounds will move up the piece of chalk. These colored compounds will not make the chalk colored, but the outside will have a rainbow appearance.}
\end{itemize}

\section{Clouds}

\subsection{Cloud in a Jar}
\begin{itemize}
\item{Preparation time: 10 minutes}
\item{Materials: 1 wide mouth glass jar, 1 latex glove, water, matches}
\item{Procedure: In the bottom of a glass jar, fill with a small amount of water. Take the glove and put it so the fingers are inside of the jar. Seal the jar with the glove. Put your hand into the glove and pull the glove out. Be sure not to break the seal. Quickly open the seal on the glove and drop in a lit match. Quickly cover the seal with the glove just like before, fingers pointing in the jar. Pull the glove out of the jar once again and observe the cloud inside the jar.}
\item{Theory: The clouds in the sky are formed when water vapor is cooled enough to form tiny water droplets. When moist, cool air rises to a higher altitude, it cools water droplets form and aggregate to form clouds. In this activity, we duplicate this same process by causing air in a bottle to rapidly cool. By putting the glove on the jar and pulling it out, the volume of the gas inside the container increases. Some of the water on the bottom of the jar turns into a gas and the temperature drops. This primes the jar. Dropping a match in the jar creates smoke and other particles that act as nucleation sites for the rain to form. Pulling out the glove a second time lowers the temperature by decreasing the volume enough to start forming clouds.}
\end{itemize}

\section{Colligative properties}

\subsection{Boiling Point Elevation, Part A – Salt Water}
\begin{itemize}
\item{Preparation time: 20 minutes}
\item{Materials: water, salt, battery acid, heat source, thermometers, 3 metal pots}
\item{Procedure: In one metal pot, add water only. In the second pot, make a salt solution. The more concentrated the better. Heat both pots and record the temperature when they start to boil. }
\item{Theory: Pure solutions boil at a lower temperature than solutions that have dissolved salts. The first pot acts as the pure solution of water. It should boil around 100 C. Depending on the amount of salt added, the second pot will boil slightly higher, around 102-105 C. The impurities of salt prevent the water from boiling at its normal temperature. It increases the temperature required to boil. }
\end{itemize}

\subsection{Boiling Point Elevation, Part B - Electrolytes}
\begin{itemize}
\item{Preparation time: 20 minutes}
\item{Materials: glucose, salt, water, balance, jam jars, heat source}
\item{Procedure: In two jam jars, fill each with 150 mL of water. To one, add 30 grams of salt. The other, add 90 grams of glucose. It may look like there is too much glucose, but it will dissolve in the water upon heating. Heat both solutions slowly and record the temperature at which they boil.}
\item{Theory: Colligative properties depend not on the actual makeup of the impurities, but the number of impurity particles. In this variant, we are adding about 0.5 moles of salt to one jar and 0.5 moles sugar to the second. Upon heating, the salt solution will boil at 102 C while the sugar solution will boil at 101 C. This discrepancy is due to the fact that salt is a strong electrolyte and glucose is a non electrolyte. This means that salt will dissolve completely in water to form two ions: a sodium and a chlorine. Glucose does not dissociate. Comparing the number of particles in both the salt and sugar solution, we can see that there is twice the number of impurity particles in the salt solution compared to the sugar solution. This means that the salt solution boils at a higher temperature than the sugar solution.}
\end{itemize}

\section{Combustion}

\subsection{Products of Combustion, Part A – H2O}
\begin{itemize}
\item{Preparation time: 5 minutes}
\item{Materials: A glass jam jar or the bottom half of a plastic water bottle, a candle, water}
\item{Procedure: Find a flame or ignition source. A candle works best for this activity. Take the beaker fill it with water. Place this over the candle not to touch the flame just above the tip of the flame. The gases from the candle should collect bottom on the container. After a minute or two, condensation will form on the outside of the glass. This shows one of the products of organic combustion: water.}
\item{Theory: When organic compounds burn in excess air, they form two main compounds: carbon dioxide and water according to the formula:. If we collect the gases from the candle, we are collecting carbon dioxide gas and water vapor. As the water cools on contact with the cold container, it will undergo a phase transition from gaseous to liquid. In this activity, the water vapor is captured on the jar. It cools and then condenses on the side of the container.}
\end{itemize}

\subsection{Products of Combustion, Part B – CO2}
\begin{itemize}
\item{Preparation time: 5 minutes}
\item{Materials: A glass jam jar or the bottom half of a plastic water bottle, a larger container like a plastic bucket or metal bowl, a candle}
\item{Procedure: Now we are going to collect the other product of combustion, carbon dioxide. For this, we will place our candle on an upside down container. Then place these two inside of a larger container. After ten to thirty minutes, we have collected enough carbon dioxide. Remove the candle from the container and place it next to the large container. Now pour the gas products gently from the large beaker on top of the candle so that there is no rush of air. The candle should gently go out.}
\item{Theory: From the reaction, carbon dioxide is produced from combustion. When compared to air, carbon dioxide has a greater density. By placing the candle on in a container, the carbon dioxide produced falls down to the bottom of the container instead of dispersing away. Now, if carbon dioxide displaces the oxygen that is normally used for combustion, the organic compounds can no longer combust. By pouring the carbon dioxide on the candle, we prevent any oxygen participating in combustion and the candle goes out, since the flame is the sign of combustion of a candle.}
\end{itemize}

\subsection{Reactants of Combustion, Part A – O2}
\begin{itemize}
\item{Preparation time: 5 minutes}
\item{Materials: a clear glass jam jar or the bottom half of a plastic water bottle, a candle}
\item{Procedure: Place the candle on the table so it burns freely. Turn the container upside and place over the candle on the table. A transparent container works best for this activity. This traps the gases inside the container. After a minute, the candle should go out.}
\item{Theory:  Combustion requires oxygen,. If there is no more oxygen for use in the combustion reaction, the combustion ends. When we trap the candle under our beaker, no more oxygen can come in from the surrounding air. That means we have a set amount of oxygen inside the upside down container. As the candle burns, it will consume all the oxygen trapped in the candle. When there is no more oxygen to burn, the reaction stops and the candle go out. Note that before the candle goes out, a lot of smoke is produced. This is because as oxygen becomes scarce the combustion becomes incomplete and unburned hydrocarbons are formed.}
\end{itemize}

\subsection{Reactants of Combustion, Part B}
\begin{itemize}
\item{Preparation time: 5 minutes}
\item{Materials: radio antenna tube, kerosene single wick burner (kibatari)}
\item{Procedure: Light the burner on the table. Take a radio antenna tube and place it at an angle inside the bottom part of the flame. The gases from the flame will travel up the pen. Take a match and ignite the flame at the end of the ratio antenna tube. }
\item{Theory: The purpose of this activity is to show that the hydrocarbons in combustion reaction are in the gaseous state, not the liquid state. We can show this with a kerosene flame. When kerosene burns, the liquid changes to the gaseous state. Through the entire length of the flame, the gases are burning, not the liquid kerosene. By providing an alternate path for the gases to travel, up the radio antenna, these gases do not burn immediately. Now at the end of the tube, we can light a match and ignite the gases as they reach the end of the tube. There is no liquid kerosene traveling in the tube to burn in the second flame, only the gas from the first flame. This shows that it is the hydrocarbons, or kerosene in the case of a burner, are gases not liquids. }
\end{itemize}

\subsection{Burning Money}
\begin{itemize}
\item{Preparation time: 10 minutes}
\item{Materials: jam jars, methylated spirits, water, matches, paper money or ordinary paper, tongs}
\item{Procedure: Make a mixture of 3 parts methylated spirits to 2 parts water. Soak the money in the mixture. Remove the note with the tongs and take a match to it. The bill should flame up and after 5 seconds drop the bill into the extra water. If the blue flames are difficult to see, repeat the activity in a darker space.}
\item{Theory: Methylated spirits are a mixture of ethanol and water. Ethanol is a very volatile compound while water is much less volatile. This means that the flash point of ethanol is low compared to paper. The water regulates the temperature of the flame such that it is higher than the flash point of ethanol and lower than the flash point of paper. This is why we can take a flame to the bill and let it burn for a short period of time without any damage to the bill. The ethanol is burning at a low temperature while the water protects the bill from combusting. However if there is a lot of ethanol and it burns for a long time, the water will evaporate away and then the bill will start burning. Do not be a criminal; just like in the United States, destruction of currency is a crime. Do not let the note to start burning.}
\end{itemize}

\subsection{Rusting of Steel Wool, Part A}
\begin{itemize}
\item{Preparation time: 15 minutes}
\item{Materials: Iron wool, plastic syringe with the top fused shut or a graduated cylinder, a dish, water}
\item{Procedure: A plastic 10 mL syringe is highly recommended for this procedure. Remove the metal needle and the plunger part of the syringe leaving the graduated shell. Then use a flame to fuse closed the narrow opening where the needle joined. If the steel wool has a detergent coating, wash it off before use with soapy water. Then take a piece of steel wool, wet it in water, and stuff it up the shell so it stays in place at the top. Hold the shell upright and place the syringe shell onto of a dish of water. Wait for three days, checking each day to replace the water lost due to evaporation. The water should move up the syringe.}
\item{Theory: Rusting is a form of combustion. When iron is in contact with both water and oxygen, it will rust in a complicated set of reactions. This is the formation of two products, iron (III) oxide, Fe2O3, and iron (III) hydroxide, Fe(OH)3. Inside the syringe, there is a trapped volume of oxygen. The reaction consumes the oxygen as time goes on. This will lower the pressure inside of the syringe and atmospheric pressure pushes the water up the syringe shell.@	For the more advanced, this activity can be used to show the percent by volume, and then percent by mole, of oxygen in the atmosphere. Prior to the reaction, measure the volume of the steel wool. Find this by water displacement in a fused 10 mL syringe shell. Then place the steel wool in the syringe and ensure the water level is in the gradations of the syringe volume. Record the initial volume. The reaction finishes when the water layer stops rising. Record the final volume. By comparing the initial volume minus the volume of the steel wool and the volume of oxygen used indicated by the rise in the water level, we can see the composition of oxygen in air as a ratio of volumes. Further, since Avogadro’s law states volume is proportional to the number of moles, the percent of oxygen in the atmosphere by volume is also the percent of oxygen in the atmosphere by moles. }
\end{itemize}

\subsection{Rusting of Steel Wool Part B – Hard Water}
\begin{itemize}
\item{Preparation time: 30 minutes}
\item{Materials: Iron wool, 2 plastic syringes with top fused shut, 2 dishes, hard water, soft water}
\item{Procedure: This procedure works well if the local water is hard water. Hard water may be prepared by dissolving magnesium sulfate in ordinary water. Use the procedure from Rusting of Steel Wool, Part A with hard water. Repeat the same procedure with one small change: use soft water at the same time. Soft water can be found from collecting the water when distilling water in a teapot or rainwater. After sufficient time passes, the rusting with hard water will have the water at a higher level than the rusting with soft water.}
\item{Theory: Just like Rusting of Steel Wool, Part A, the reaction consumes the air inside of the syringe. The difference between the two rusting variations is the hard water and soft water. Hard water contains dissolved Ca2+ (and Mg2+) ions; soft water has none. These ions will react with the iron (III) hydroxides to form calcium hydroxides (and Mg) and iron calcium oxide hydroxides. Due to these side reactions, the presence of Ca2+ ions speeds up the rusting reactions. As it speeds up the reactions, the oxygen is consumed faster and the atmospheric pressure compensates by pushing more water up the syringe shell. This is reason that the hard water rusting will have a higher water level than the soft water.}
\end{itemize}

\subsection{Rusting of Steel Wool, Part C – Salt Water}
\begin{itemize}
\item{Preparation time: 10 minutes}
\item{Materials: table salt, iron wool, two syringe shells, distilled or soft water}
\item{Procedure: Follow the same procedure to from Rusting of Steel Wool, Part A for one of the syringe shells. For the other, make a concentrated salt solution. Use this salt solution in place of the normal water. Observe which iron will rust first.}
\item{Theory: This activity tests how iron rusts in the presence of other ions. Rusting is not a completely understood reaction. This variant of the activity tests the effect of electrochemical forces, the presence of salt, as it affects the rate of rusting.}
\end{itemize}

\subsection{Rusting of Steel Wool, Part D – Heat of Rusting}
\begin{itemize}
\item{Preparation time: 30 minutes}
\item{Materials: a thermometer, iron wool, vinegar, a container with cap large enough to hold the thermometer like a 1 or 1.5 L empty plastic water bottle}
\item{Procedure: Place the thermometer in the empty plastic water bottle. Wait 10 minutes for the temperature to stabilize. Take a small piece of steel wool soak it in vinegar while the temperature stabilizes. Record the temperature and remove the thermometer. Squeeze the iron wool to release any vinegar. Wrap the bulb of the thermometer with the iron wool that was soaked vinegar, replace them in the plastic water bottle, and cap. Record the temperature every 10 minutes for the next 2 hours. As time proceeds, the temperature will rise.}
\item{Theory: The rusting of iron is an exothermic reaction. An exothermic reaction is a chemical change that is accompanied with the release of heat. In this particular activity, the rusting of the iron releases heat that is recorded by the thermometer. The rise in temperature is a representation of the exothermic nature of this particular reaction. In order to have the reaction proceed quickly and without problems, vinegar is used to clean the surface allowing the iron metal to react easily, especially if there is any detergent coating the iron as wells as any surface iron oxides.}
\end{itemize}

\subsection{Rust prevention}
\begin{itemize}
\item{Preparation time: 30 minutes}
\item{Materials: oil, water, syringes shells fused at the top, iron (non galvanized) nails, paint (if desired), syringe plungers}
\item{Procedure: In this activity, different aspects of the rusting process will be examined. Each part of this activity requires a syringe shell with the top fused. These are all recommended activities but any combination of the aspects maybe conducted. First, make a control syringe. Place the nail in the syringe shell and cover with water. If the nail has some rust on it to begin with, use some iron wool and rub the nail to remove the rust. Stand the syringe shell upright so to leave the nail is submerged in the water but open to the air. Once we have our control, other aspects can be tested of the rusting reaction. Examine each experiment each day for a week for rust on the nail. Here are six different variations to test:

\begin{enumerate}
\item{Place a nail in a syringe and fill with water. Leave open to air. This is the control.}
\item{Place a nail in a syringe and fill with water. Place the plunger back to seal the syringe to prevent oxygen from entering the shell.}
\item{Place a nail in a syringe and cover with oil. Leave open to air.}
\item{Place a nail in a syringe and cover with oil. Replace plunger to seal the syringe.}
\item{Place a nail in a syringe cover with water. Make an oil layer with a thickness of at least 2 cm.}
\item{Take a nail and paint the nail to cover the metal completely. Now place in a syringe and fill with water.}
\end{enumerate}
} % Procedure

\item{Theory: The reaction for the rusting of iron consists of a reaction between iron, oxygen, and water. In this activity, each variety tests a different aspect of the rusting reaction. Experiment 1 is the control experiment. This will be the normal rusting of the iron nail. Experiment 2 now prevents any oxygen from participating in the rusting reaction. This nail should have little or no rust. Experiment 3 allows oxygen to encounter the oil for rusting; however, there is no water since it is filled with oil. In addition, oxygen migrates through oil very poorly that also limits the available oxygen for rusting on the surface of the nail. There should be little or no rusting on the nail. Experiment 4 has no oxygen or water available for rusting. This nail should have no rusting. Experiment 5 allows water available to react, but now there is an oil layer. This oil layer prevents oxygen from entering the water layer and rusting the iron nail. This nail should have little or no rust. Experiment 6 is much like the control, but since we have painted the nail, there is no iron available to react since it is protected by the paint. In each of these experiments, the availability of the iron, oxygen, and water change and the effects of the lack of availability of each component slow down or even prevent the rusting reaction.}
\end{itemize}

\subsection{Oxidation of Iron, Part A – Burning in Air}
\begin{itemize}
\item{Preparation time: 0 minutes}
\item{Materials: lighter, fresh iron wool, iron oxide or rusted iron wool from a previous activity}
\item{Procedure: With a lighter, place the rusted steel wool in the flame from the lighter. There no be no reaction or change other than a rise in temperature. Now, do this activity again with fresh iron wool. The iron wool should burn brightly and oxidize in the air readily, producing a bright light, much like fireworks. Lastly, let the iron wool cool for 2 minutes. Place the iron wool back into a lighter flame. There should be no reaction this time.}
\item{Theory: Iron, when heated in the presence oxygen, oxidizes to iron (III) oxide. The reaction is as follows:. With the rusted iron wool, the iron metal has undergone oxidation to form the iron (III) oxide and iron (III) hydroxides. If the rusted iron wool is heated, there is no reaction since the iron is already been oxidized. The iron metal has all been consumed. The entire iron wool will burn and produce a bright light from the oxidization. Now as the iron wool burns and then cools, it should turn a darker color that shows the presence of iron (III) oxide. This means that there is no more iron available to oxidize. This activity is a good example of the different pathways of oxidation. Oxidization can occur through rusting or it can be direct oxidization in air. }
\end{itemize}

\subsection{Oxidation of Iron, Part B – New Mass}
\begin{itemize}
\item{Preparation time: 5 minutes}
\item{Materials: fresh iron wool, lighter, balance}
\item{Procedure: Take a piece of iron wool and find its mass. Record the mass. Place the iron wool in the lighter flame as to oxidize it according to the procedure in Oxidation of Iron, Part A. Let the iron wool cool to room temperature. Record the mass a second time. The mass of the oxidized iron will be larger than the fresh iron wool}
\item{Theory: From the reaction, the iron wool is just iron metal. As it undergoes oxidation, the iron reacts to form iron (III) oxide. The molecular mass of iron (III) oxide is larger than iron metal due to the additional oxygen. Since the number of moles of iron is not changing, the mass of the entire solid is increasing from the addition of the oxygen.}
\end{itemize}

\subsection{Oxidation of Iron, Part C – In the World}
\begin{itemize}
\item{Preparation time: 30 minutes}
\item{Materials: Nearby metalworker}
\item{Procedure: Arrange a field trip to a nearby metalworker who grinds metal. Observe the technician grind the metal and watch the sparks fly from the metal being grinded. }
\item{Theory: The sparks caused by the metal grinder is the same reaction from Oxidation of Iron, Part A. Bits of iron metal, heated from the mechanical work from the grinder, oxidize as they fly through the oxygen-rich air. The piece being ground does not glow because the rest of the metal conducts the heat away from the grinding site fast enough to keep the working surface below the temperature of combustion. Nevertheless, evidence of the heat is often visible on the finished piece.}
\end{itemize}

\subsection{Combustion of Meats}
\begin{itemize}
\item{Preparation time: 30 minutes}
\item{Materials: 1/4 kg of meat, two burners (electric, charcoal, or kerosene), two metal bowls or pans, one for frying, one for boiling water}
\item{Procedure: Cut the meat into small cubes. Cut enough for 3 pieces for each participant or student. Take one piece for each student and boil it in water for 30 minutes. Take a second piece of meat for each student and sear it in oil. This means to fry, or grill, the meat until the surface of the meat browns and dark flakes appear in the oil, but not until it burns. Take the third piece of meat and fry, or grill, it in oil until it blackens or burns and turns into a black color. Have each student eat a piece of meat, one boiled, one fried (or grilled), one blackened. Ask them to identify the difference in flavors and textures. Of course, this experiment should be done in an ordinary classroom, not the school lab. You would not want to encourage eating in the lab.}
\item{Theory: Meats and other foods are a large system of organic compounds. Specifically, meats have amines on the surface. These amines react with reducing sugars, like glucose and fructose, in the meat and form a large variety of different compounds that all accompany with a good taste and smell. This is why fried meats, and even grilled meats, taste so good. This is due to the way they are cooked. Now if we continue to fry or grill the meat past the point where the meat is browned, the amines and other surface compounds of the meat combust until they are carbon solid, or like charcoal. If the students eat the fried meat, it should taste good. The blackened meat should taste bad or like charcoal. The difference between the two is the extent that the compounds of the meat combust; browned meat is partial combustion or reactions while blackened meat is approaching complete combustion. This particular chemical reaction of the browning of meat is called the Maillard reaction.//
The boiled meat however, does not have the same Maillard reaction. The Maillard reaction happens between the surface chemicals of the meat under heat. The boiled meat allows for these compounds to dissolve in the boiling water. This means that the Maillard reaction does not occur. This can be seen by first looking at the surface of the boiled meat. It is not the same as the fried or grilled meat; in fact, it should look dark and maybe grey but not brown. If eaten, the boiled meat does not have the same flavor as the fried or grilled meat. This is due to the absence of the Maillard reaction.}
\end{itemize}

\section{Crystal Growing}

State of matter transitions are fun and interesting to show. One of the more fun, intriguing and hands on example of changing states of matter is dissolving solids and forming them again. Different crystals will form different shapes. The smooth sides with sharp edges are something that is just too neat not to make and touch. Plus, the students get a big kick out of growing their own crystals to take home. There is a variety of different solids to try to make crystals from cheap and locally available materials.

\subsection{Salt Crystals}
\begin{itemize}
\item{Preparation Time: 0 minutes}
\item{Materials: Table salt, water, container/beaker}
\item{Procedure: Add 19g of table salt (NaCl) to 50 mL water. The actual amount of salt or water does not really matter; just add salt until the solution is saturated. Transfer the salt solution into a wide mouth container or a pan: the larger the surface area that the solution has, the faster it will evaporate. As the salt evaporates, crystals of sodium chloride will form. Normally, table salt is a very fine grain. These crystals will make slightly larger crystals where the cubic habit can easily be seen with a simple water drop magnifier.}

\item{Theory: Solutions are what happens when we mix two or more components dissolve into another. One does not disappear into the other; both components are still there although not seen. An aqueous solution cannot be separated by normal physical means. The two can be separated by letting the water evaporate. The water leaves, but the NaCl does evaporate. It is left behind and forms a solid. Ionic solids tend to form nice crystal lattices, and table salt is no exception. This activity can be used to explain a variety of different topics:

\begin{itemize}
\item{Crystallization: This is a perfect example at the difference between freezing and crystallizing. Freezing occurs when a temperature of a liquid drops until the molecules have so little energy that the intermolecular forces begin to hold them together. We can think of it as the molecules are free to move as a liquid but as the temperature drops, the molecules slow down until they are stuck in place. Crystallization occurs when we have ions that have a greater stability by surrounding themselves in more ions. Instead of one sodium atom bonding to one chlorine atom, the crystal finds its stability by allowing the sodium ion to bond to many chlorine atoms (six in this case). The same happens for the chlorine atom. This extra stability helps the molecules to hold together and form the cubic shape.}
\item{Difference between covalent and ionic compounds: Covalent compounds tend to have low intermolecular forces and are more volatile. This means they easily turn into a gas or evaporate. Ionic compounds on the other hand do not turn into gases easily. Relatively large electrostatic forces, the forces between positive and negative ions, hold the individual ions together so they cannot easily break away from each other. Water is an example of a covalent compound. It is somewhat volatile: if you leave some water in the sun, it will evaporate. Salt or NaCl, is an example of an ionic compound. It is not volatile and stays behind when the water evaporates.}
\end{itemize}
}
\end{itemize}

\subsection{Epsom Salt Crystals}
\begin{itemize}
\item{Preparation Time: 10 minutes}
\item{Materials: Epsom salt, water, glass jam jar, 1 small square of glass or plastic}
\item{Procedure: Add 1g of Epsom salt (MgSO4) to 5 mL water. Heat the solution until all the crystals dissolve. Pour the Epsom salt solution onto a small square piece of glass or plastic: the larger area the solutions occupy, the thinner a layer will form and the faster it will evaporate. As the solution evaporates, it leaves behind crystals. Use a water magnifier, from the activity in the biology section Water Magnifier, to look at the crystals. MgSO4 forms a lattice of needlelike crystals. Another method to see the crystals is to form them on a piece of clear glass. Shine a light over the crystals to project their image onto a piece of paper. The shape of the crystals can be easily seen. }
\item{Theory: The same processes are involved in the formation of MgSO4 crystals. However, we can use this slightly altered procedure to talk about different things.
\begin{itemize}
\item{Temperature dependent solubility: For many solutes, the temperature of the solvent plays a big role in solubility. In the case of these ionic compounds, a slight increase in the temperature of water dramatically increases the amount salt that will dissolve. By heating the water, we are able to dissolve more Epsom salt in the water so that when it evaporates it will form crystals large enough for us to see.}
\item{Crystal Lattice Shapes: Due the difference size of the ions combined with their charges, they will form different shapes as a crystal. NaCl forms a normal cube. MgSO4 forms Monoclinic shape. In fact, we can follow the same procedure to make different crystal shapes from different compounds. 
\begin{itemize}
\item{Sodium thiosulfate: Hexagonal shape}
\item{Magnesium Sulfate heptahydrate: Monoclinic}
\item{Sodium Nitrate or Saltpeter: Tetragonal}
\item{NaCl: Cubic}
\item{FeSO4: Tetragonal}
\item{CuSO4: Triclinic}
\end{itemize}
} %crystal lattice shapes
\end{itemize}
} % theory
\end{itemize}

\subsection{Watching Rapid Crystal Growth}
\begin{itemize}
\item{Preparation Time: 15 minutes}
\item{Materials: Sodium Thiosulfate Na2S2O3∙10H20, water, test tube or thin glass container}
\item{Procedure: In a test tube add the thiosulfate until the bottom 3 or 4 cm of the tube are filled with the salt. Gently heat the test tube until all the crystals have dissolved. All the crystals need to be dissolved and there should be no stray crystals on the side of the container. Put aside to cool. Once cool, add a seed crystal of thiosulfate. A crystal structure should form very quickly and spread throughout the test tube. It might be possible to have this rapid crystal growth a plastic syringe test tube.}
\item{Theory: In this activity, students make crystals rapidly from a super saturated solution. This means that we have more solute dissolved in the solvent than normally allowed, usually by increasing the solubility with heat and then cooling. Here we have extra thiosulfate dissolved in the water and it is ready to form crystals. However, crystals cannot be formed unless there is a nucleation point. This is just a spot for the crystals to start growing from. It can be a piece of dust, a rough surface in the test tube, or another crystal. When we add our seed crystal, the supersaturated solution will grow on the crystal outwards until enough thiosulfate precipitates so that the solution changes from supersaturated to just saturated. Beyond just a visually stunning activity, this activity is a good way to talk about saturated and super saturated solutions. }
\end{itemize}

\subsection{Growing Giant Crystals}
\begin{itemize}
\item{Preparation Time: 30 minutes}
\item{Materials: Alum, water, container of decent volume like a large plastic water bottle or bucket}
\item{Procedure: Making a giant crystal is easy using the ideas we have talked about so far. We are going to increase the temperature to increase the solubility and we are going to make a supersaturated solution crystallize out alum with a seed crystal. A note on alum; There are many times of alum. Commonly available in Tanzania is a white alum, KFe(SO4)2. First, find a container that you want to grow your crystal in. Start with a beaker or a jam jar. Tie a seed crystal into some string. Nylon dental floss is perfect but really any will work. Tie this string to a stick or a pencil such that when placed over the jar, the crystal will hang in the solution but will not touch the bottom. Remove the string. Use enough water such that the jar will be 2/3 filled. Heat this water and add alum slowly and stirring until completely dissolved. Add until no more will dissolve. Pour this solution into the jar. Place the seed crystal on the string into the solution so that it is hanging in the solution. After the solution cools, 1 to 2 hours, or overnight, you will have a much larger crystal than before. }
\item{Theory: This crystallization, we want to cause the crystals to grow on the seed crystal so that we can have a large crystal. The more alum you have dissolved we can precipitate more on our crystal. As expected, the more alum and the larger the container we have, the larger the crystal we can make. With this heat treatment procedure, some crystal deposits on the bottom of the container are expected. Remember that crystals precipitate on a nucleation point. The rougher the surface, crystals can form. Therefore, some containers make poor large crystal forming vessels. Metal or ceramic containers have a quite rough surface, which allows nucleation. A better container would be either glass or a plastic container. A tried and true vessel with the least amount of precipitate on the bottom of the container is plastic water bottles. These bottles come in different volumes, some as large as 12 L. Using a 6 L water bottle, a 4 kg giant alum crystal has been made.//
There are a few tricks to grow really large crystals:
\begin{itemize}
\item{Make sure the alum is clean. The dirt and impurities will provide sites for the alum to crystallize. We want it to crystallize where we want on the seed crystal not on the dirt or impurities that can provide nucleation sites.}
\item{Ensure the solution is saturated at the heated temperature or some of the alum will stay dissolved in the solution}
\item{Grow the crystal in a plastic container}
\item{Cover the container with a piece of cloth to prevent dust and other contaminants to find their way into the crystal solution}
\end{itemize}
If there are impurities found in alum, this is a good activity of the different ways to purify compounds. Crystallization of the alum crystals will purify itself by pushing out many of the contaminants. Especially when a seed crystal is used, a large crystal can be formed while leaving the impurities at the bottom of the container. However, not all the impurities can be removed. A pure alum crystal should be transparent but crystals made with a heated solution rather than evaporated are formed at a faster rate. This means the crystal is purifying by forming the crystal, but the rate is too fast prevent impurities. A cloudy alum crystal signifies the presence of water trapped inside the crystal lattice. }
\end{itemize}

\subsection{Hot Ice, Part A - Production}
\begin{itemize}
\item{Preparation time: 2 hours}
\item{Materials: baking soda, vinegar, heat source}
\item{Procedure: In an aluminum bowl, mix 1 liter of vinegar with 1 boxes of baking soda. Do not mix them quickly or you will have a volcano on your hands. After it has been completely mixed, boil the mixture. Boil until the total volume is reduced to 100 or 150 mL. Or, boil until a crystal skin covers the surface of the mixture. This may take upwards of an hour. It is not a problem is discoloration occurs. Once finished, cover the bowl to prevent evaporation. Ensure there are no stray crystals in the liquid. Move this mixture into a refrigerator to chill. Save any stray crystals.}
\item{Theory: Mixing baking soda and vinegar is a fairly common reaction and much repeated in this text. The reaction is. This solid can do some neat things, most notably that we can supercool it as a liquid in a normal refrigerator.}
\end{itemize}

\subsection{Hot Ice, Part B – Super Cool}
\begin{itemize}
\item{Preparation time: 10 minutes}
\item{Materials: sodium acetate supercooled solution from Hot Ice, Part A - Production, a small sodium acetate crystal}
\item{Procedure: Carefully remove the cooled sodium acetate liquid from the refrigerator. Be careful not to disturb it greatly or let anything fall into the liquid. Once ready, drop a sodium acetate crystal into the liquid. Crystallization will being immediately branching out from the crystal that we dropped in. Feel the container and the newly formed sodium acetate crystal structure.}
\item{Theory: Heat of crystallization is usually very hard to explain. In this activity, we get both an example of supercooling and heat of crystallization. Supercooling is where a liquid is cooled past the point it normally turns into a solid but is still a liquid. Any agitation or seed crystal will immediately cause the crystallization to begin. Heat of crystallization is the heat change when crystals form. Usually, this is an exothermic process. Once the crystallization begins, the heat of crystallization starts to heat up the container.}
\end{itemize}

\subsection{Hot Ice, Part C – Climbing Crystals}
\begin{itemize}
\item{Preparation time: 5 minutes}
\item{Materials: sodium acetate supercooled solution from Hot Ice, Part A - Production, a shallow container}
\item{Procedure: Carefully remove the cooled sodium acetate liquid from the refrigerator. Be careful not to disturb it greatly or let anything fall into the liquid. Once ready, gently pour the supercool sodium acetate solution into the second container. The crystals will start forming on contact, and if you pour slowly enough, will start to grow up the stream towards the main container of sodium acetate.}
\item{Theory: Supercooled solutions are ready to start crystallization at any moment. Any of the slightest disturbance or change in temperature can make crystals fall out of solution. We are using this fact to start the crystallization: through the combined factors of the change in temperature and the violent act of pouring this liquid into another container, crystallization occurs immediately. It is so ready for crystallization that it starts to grow up the pouring stream.}
\end{itemize}

\subsection{Hot Ice, Part D - Recycling}
\begin{itemize}
\item{Preparation time: 0 minutes}
\item{Materials: crystal solutions or crystals from the earlier Hot Ice, Part B – Supercool or Hot Ice, Part C – Climbing Crystals activities.}
\item{Procedure: Collect the remains of the crystal structures or solutions and add just enough water to dissolve the crystals. Heat the solution to boil away excess water until the consistency is much like in Hot Ice, Part A and refrigerate once again.}
\item{Theory: Crystallization is a reversible reaction. This fact can be demonstrated by the sheer fact that the chemicals used in these activities is recyclable is good to show. A reminder for safety: this chemical does not pose a danger for students to handle. It is not poisonous or harmful. Many cases it is a food additive. Further, the heat generation is not enough to burn students. In many reusable heat packs, this is actually the reaction heats the packs.}
\end{itemize}

\section{Density and Polarity of Liquids}

\subsection{Exploring Density, Part A - Temperature Dependent Density}
\begin{itemize}
\item{Preparation time: 20 minutes}
\item{Materials: 2 jam jars, 1 narrow clear glass container, water, two different food colorings, ice or refrigerator, electric or kerosene burner, 1 syringe with metal needle}
\item{Procedure: Place 200 mL of water in each jam jar. Place a few drops of the two different food colors in each jar, for example make one blue and one red. Take one jar and add ice or place it in the refrigerator. Heat the second jar for 20 minutes. Take a second clear, glass, tube or jar and add 50 mL of the cold solution. Using a syringe, take a few mL of the hot solution and add gently to the cold solution. This takes some practice to ensure the water does not mix. Use the metal needle to add one drop at a time so that the liquid runs down the side of the container. Two layers will form. }
\item{Theory: The density of all liquids is temperature dependent. Generally, they expand on heating and contract on cooling. Since density is defined as mass divided by volume, as a liquid is heated the volume increases, decreasing the density. As a liquid cools, the volume contracts, increasing the density. Therefore, in this case the hot water has a smaller density than the cold water. By adding the water carefully to prevent mixing, two layers can be seen in the water: an upper, hot water layer and a lower, cold-water layer. Keep in mind that these two layers are temporary. Water mixes perfectly with other water so as time goes on, the two liquids will mix and even out the temperatures.}
\end{itemize}

\subsection{Exploring Density, Part B - Densities of Different Water Solutions}
\begin{itemize}
\item{Preparation time: 15 minutes}
\item{Materials: 3 jam jars, 1 narrow clear glass container, water, table salt, sugar, three different food colorings, syringe with metal needle}
\item{Procedure: Place 200 mL of water into each of the jam jars. Place a few drops different food colorings in each jam jar. In one jar, place 1 spoon of table salt. In a second jar, place 1 spoon of sugar. In the third jar, add nothing. Mix to ensure all solutes are dissolved completely. Take a syringe and to the narrow container add ten mL of the salt-water solution. Carefully, add drop-by-drop ten mL of the sugar solution to the container in such a way to prevent mixing. Use a syringe needle to aid this process. Now, carefully add drop-by-drop ten mL of pure water. Adding these together requires practice to ensure the layers do not mix. Three layers should form. }
\item{Theory: Dissolving solutes in water will change the density of the liquid. Density is defined as mass divided by volume. If we use the same volume of each solution, the density of each depends on the mass of each solution. The mass of each solution depends both on the mass of the water and the mass of the solute dissolved in it. The pure water solution has no solute dissolved while the other two solutions have extra components dissolved: salt and sugar. One spoon of salt has a different mass than 1 spoon of sugar. In fact, the spoon of salt weighs more than the sugar. Therefore the salt solution will be the heaviest, then sugar, and then pure water will be on top.}
\end{itemize}

\subsection{Floating Eggs}
\begin{itemize}
\item{Preparation time: 5 minutes}
\item{Materials: 1 uncooked egg, 1 jam jar, water, salt}
\item{Procedure: Fill jam jar with water. Place egg in the water. The egg will sink. Add salt until the egg floats. }
\item{Theory: The density of an egg is greater than water. This is why the egg will sink. Since density is defined as mass divided by volume, the density of water can be changed by dissolving extra mass, salt in this case. As more and more salt dissolves in the water, the densities increases until the density of the water is greater than the egg and the egg floats to the surface.//
This is the same reason why it is more difficult to swim in fresh water than salt or ocean water. The extra salt or ions in the ocean water increase its density and making the body more buoyant. Since those ions are much fewer in fresh water, the density of fresh water is greater than salt water. The Dead Sea in the Middle East has so much salt in the water that people do not swim in the water; they just float. Lake Natron in Tanzania probably has similar properties.}
\end{itemize}

\subsection{Density Tower, Part A - Production}
\begin{itemize}
\item{Preparation time: 15 minutes}
\item{Materials: liquid soap, water, honey, cooking oil, propanone (acetone), kerosene, glycerin, methylated spirits, motor oil, 1 tall clear container. Volumes of each depend on size of the container.}
\item{Procedure: Carefully pour one the liquids down the side of the container. Pour it in a way that a small volume slowly goes down the side of the container in a gentle, nonviolent manner. Pour a second liquid in the same manner so that it gently rests on top of the other liquid. Some of the liquids will rise and some will fall. Be very careful so as to prevent mixing of the layers when pouring. This may take some practice. Further, some compounds will mix if poured on top of each other. It may be necessary to have intermediate layers to see which ones will mix even if poured very carefully. }
\item{Theory: Each component has its own density depending upon the composition of the liquid. Some sugar, oil, or other component will give a different density. These liquids will float or rise on top of each other depending on the other liquids density. In other words, the least dense sample will rise to the top of the tower and the compound with the least density will sink to the bottom. In this example, the glycerin is the densest sample while kerosene is the least dense. Experiment with the order of adding the different liquids; the order is important. If you add a layer that dissolves in another layer right on top of each other, like water and methylated spirits, they will combine to form one layer. To get a dramatic ordering seen in this diagram, start first with water. Add kerosene and then add cooking oil. If added properly, the oil will settle in between the water and the kerosene layer. This step is difficult, but possible. If it is not working, add oil first followed by kerosene. Add methylated spirits, and it will sink in between the oil and kerosene layer. Lastly, add glycerin and it will fall through both layers.}
\end{itemize}

\subsection{Density Tower, Part B – Adding More}
\begin{itemize}
\item{Preparation time: 5 minutes}
\item{Materials: The column of liquids of different densities from Density Tower, Part A - Production, small pieces of different materials like metal, wax, plastic, wood, cork, orange peel, sand, or anything that is locally available.}
\item{Procedure: Use the density tower from Density Tower, Part A - Production. Now, it is possible to compare the relative densities of solid materials. Drop in the materials piece by piece, carefully to prevent mixing of the layers. It is recommended to use food coloring with the water to see the layer better. The materials will float or sink to the layers that are close to their density.  }
\item{Theory: The solids will float or sink above or below the liquids. The flotation of each component is a direct example of differing densities. The less dense compound will rise to the surface while the denser will sink. In other words, a solid will float above layers with greater densities and sink below layers with smaller densities. This tower can show us the relative density of different materials we can find locally. For example, the matchstick floats on top of the methylated spirits level. This shows us that the density of the matchstick is greater than kerosene but is less dense than methylated spirits. In fact, since it floats above the methylated spirit layer, this tells us that the matchstick is less dense than all the layers below it, like water. Alternatively, all the layers below it are denser than the matchstick. It is possible to test for this by taking some water by itself and drop in a matchstick. The matchstick will float. Wax floats in between the methylated spirits layer and the cooking oil layer. This shows us that the density of the wax is greater than methylated spirits, but less than cooking oil. Try this with different plastics, as they have differing densities. In fact, this is how many plastic recyclers separate out different plastics from waste.}
\end{itemize}

\subsection{Density Tower, Part C- Mixing}
\begin{itemize}
\item{Preparation time: 0 minutes}
\item{Materials: column of liquids different densities from Density Tower, Part A - Production}
\item{Procedure: Place a lid, a stopper, or even a thumb over to the top of the tower so that it is leak proof. Shake the tower. All the layers will come together in one big emulsion. After some time, the layers will separate into two layers.}
\item{Theory: All the liquids are subject to intermolecular forces. The nature of those forces is related to the nature of the molecules. Components with charged groups are polar compounds, like water. Components without charged groups are called non-polar compounds. Liquids will mix together if the polarities are similar. This gives rise to the common rule of thumb ‘like dissolves like’, polar compounds dissolve polar compounds, and non-polar compounds dissolve non-polar. In the density tower from Density Tower, Part A, there are 3 non-polar layers: kerosene, glycerin, cooking oil. There are 2 polar layers: methylated spirits and water. When the layers are all mixed, they form an emulsion. An emulsion is a mixture of non-polar and polar compounds. The non-polar compounds form micelles to dissolve in the polar water layers. This happens when there is a molecule that is part way between being polar and non-polar. However, this is not the most stable organization. Slowly, the non-polar and polar compounds will separate yielding only two layers after a few minutes. The less dense, or upper layer, is the non-polar layer. The lower layer is the aqueous layer. Now, if there are the extra materials from Density Tower, Part B, there is no problem. A good set of density activities is to do each part of the Density Tower activities in consecutive order.}
\end{itemize}

\section{Dilution}

\subsection{Dilution, Part A – Salt Dilution}
\begin{itemize}
\item{Preparation time: 10 minutes}
\item{Materials: Salt, syringe, salt, water, 5 jam jars}
\item{Procedure: In one jam jar, place 100 mL of water and dissolve enough salt to make a saturated solution. With a syringe remove a 10 mL aliquot of the salt-water solution and put it into another jam jar. Add 90 mL of clean water. Now, take 10 mL of this diluted solution and put it into another jam jar with 90 mL of water. Repeat this procedure 3 more times. Taste each solution. In this activity, students will be eating so do not conduct this experiment in the laboratory, but in a classroom. Remember, no eating or drinking in the lab. For extra safety, utilize fresh syringes. Open them from their plastic wrapper and remove their needles.}
\item{Theory: When we are diluting, we have two things we need to think about: the solute and the solvent. The solute is the compound that dissolves into another compound, the solvent. For this activity, salt is the solute and water is the solvent. Dilution is the process where we start with a specific ratio of solute to solvent, and we add solvent to make that ratio of solute to solvent smaller. When we take 10 mL of our solution and add 90 mL of water, we have a specific amount of saltwater in the syringe and we add water. This changes the ratio of salt and water that we have in the jam jar. This is dilution. We can test this by tasting each of the jam jars. The first jar will be super salty. The second jar, third, forth, and fifth jar will have an increasing smaller ratio of salt to water. By tasting each jar, we can see that the salt is being diluted because they will be less and less salty than the previous jar.}
\end{itemize}

\subsection{Dilution, Part B – Color Dilution}
\begin{itemize}
\item{Preparation time: 10 minutes}
\item{Materials: food coloring, water, syringe, 5 jam jars}
\item{Procedure: Follow the same procedure from Dilution, Part A with one exception. Instead of using salt, use food coloring. As the dilution happens, the color will dilute little by little. This activity, unlike Dilution, Part A, involves no eating or drinking. This activity can be done in the classroom or the laboratory.}
\item{Theory: The explanation is the same as Dilution, Part A. The only difference is that the observed property is color, not taste.}
\end{itemize}

\subsection{Cement Making, Part A - Dilution}
\begin{itemize}
\item{Preparation time: 3 hours}
\item{Materials: cement, sand, water, plastic water bottles, large plastic container}
\item{Procedure: Place 1 volume of cement is a large plastic container or wheelbarrow. The actual volume of cement used is not too important, but use the same volume when adding sand. To that, add 1 volume of sand. Add water to make a paste and pour off into a plastic water bottle with the top cut off. This bottle acts as a mold for the cement. Then, add a second volume of sand to the large plastic container. Pour into a plastic water bottle. Add a third volume of sand, and then pour into a water bottle. Repeat this procedure until you have added 12 volumes of sand. Let the cement dry overnight and cut off the plastic water bottle. Label and keep each different piece of cement.}
\item{Theory: Making cement is an exercise in creating a proper ratio of cement to sand. This is very similar to dilution. In fact, by adding sand 1 volume at a time, we are making cement of different concentrations. Each concentration of cement is a little more diluted from the previous one. This gives the cement different textures and properties. The ratio of waterproof cement is a 1:4 of cement to sand. 1:7 is the standard ratio for normal cement. As the ratio becomes 1:10 and higher, the cement does not have enough binder to effectively hold the cement to the sand in one block. In fact, this ratio tends to flake very easily. This activity highlights this dilution of the cement. Progressively adding more and more sand will make the ratio of the cement to salt grower much larger.}
\end{itemize}

\subsection{Cement Making, Part B - Erosion}
\begin{itemize}
\item{Preparation time: 10 minutes}
\item{Materials: cement made from different ratios from Cement Making, Part A - Dilution}
\item{Procedure: Take each of the cement pieces and place them outside to bear the elements. Record each week the status of each piece of cement.}
\item{Theory: The ratio of the binder, cement, to sand decreases through each dilution. This means that the strength that holds the cement together decreases as the ratio of cement to sand increases. We can see this by leaving all the different pieces of cement outside to erode. The strongest pieces of cement will resist erosion the most. The pieces of cement that have a 1:10 ratio of cement to sand will erode very easily. This is why most cement blocks look like they are melting when it rains. The cement is too diluted to resist erosion effectively. Over the course of a year, the cement that has a 1:10 or 1:12 ratio will erode while the other pieces of cement will not erode.}
\end{itemize}

\section{Electrochemistry}

\subsection{Displacement Reactions, Part A - Metal Reactivity}
\begin{itemize}
\item{Preparation time: 10 minutes}
\item{Materials: jam jars or transparent reaction vessels, variety of chemicals, magnesium metal (optional), magnesium sulfate (Epsom salt), zinc metal from an unused D cell battery, zinc sulfate, iron nail, iron (II) or (III) sulfate, copper wire, copper (II) sulfate, silver nitrate (optional)}
\item{Procedure: Place a piece of metal from the list of metals at the bottom of a jam jar. Then pour solution of a different chemical from the list of solutions in the jam jar to cover the metal. In five or ten minutes, a precipitate will form on the metal if the proper combinations of chemicals are used. For each of these solutions, a 1\% by mass solution is more than enough to see the precipitates. There is a variety of combinations to try:
\begin{itemize}
\item{Metals: magnesium, zinc, iron, copper}
\item{Solutions: magnesium sulfate, zinc sulfate, iron (II) or (III) sulfate, copper (II) sulfate, silver nitrate}
\end{itemize}
A sample activity: take an iron nail. Place at the bottom of the jam jar. Cover the nail with copper (II) sulfate. In 5 minutes a reddish brown precipitate of copper will form. Repeat this activity but instead of copper (II) sulfate, use magnesium sulfate. There will be no precipitate this time.}
\item{Theory: This activity is a direct example of displacement reactions. The metal ion in the solution will reduce and the metal solid will oxidize if organized the proper combination is used. Oxidation means that the compound loses electrons and reduction means that the compound gains electrons. In the first sample activity, the iron metal will oxidize to form iron (II) ions in solution and the copper (II) ions will reduce to form copper metal, precipitating out on the nail. This is not rust, but copper precipitate. It is easily rubbed off to show that it is copper. When the activity is repeated with magnesium sulfate, nothing will happen. This is an example table of electrochemical potentials: ReactionPotential Table//
In this table, E refers to electrode potential. Electrode potential is the ability of metal ions to reduce when compared to the reduction of hydrogen. Another way to think about this is how strongly metals grab electrons in electrochemical reactions. The stronger metal always wins. The values are sometimes negative because zero was assigned to the electrode potential of hydrogen. Remember that metal ions are in the solution, not on the piece of metal.//
We can use this table by the thinking about the reduction potentials. A negative number means that the reaction proceeds in the negative direction. They can be calculated by the formula . By looking at the electrode potential for the cells, it shows which reaction will occur and which will not. To simplify this table, look for the ion and the metal. If the metal is higher on the table than the ion, the ion will reduce and the metal will oxidize.//
This is effect is important because it is the science behind the reactivity series. The reactivity series describes how the metals will interact. However, the reactivity series does not explain the science behind the reactions. This electrochemical table both describes the series and explains the science behind it.}
\end{itemize}

\subsection{Displacement Reactions, Part B – Reactivity Series}
\begin{itemize}
\item{Preparation time: 30 minutes}
\item{Materials: Magnesium ribbon, zinc sulfate solution, iron (II) sulfate solution, copper (II) sulfate solution, silver nitrate solution}
\item{Procedure: The same procedure as Displacement Reactions, Part A – Metal Reactivity with a small change: instead of doing just one displacement in the jam jar, we are going to do all the displacements one by one. In a jam jar, place a strip of magnesium ribbon. Cover with zinc sulfate solution. Wait one day to allow all the zinc to reduce. Carefully remove the ribbon, now coated with zinc, and transfer to a clean jam jar. Cover the ribbon coated with zinc with a solution of iron (II) sulfate. The zinc will oxidize and the iron reduces as it plates out on the ribbon. Now add some copper (II) sulfate and wait a day. The iron will now oxidize while the copper reduces to copper metal. Finally, add some silver nitrate and wait a day. The copper will oxidize back to copper (II) ions and the silver will precipitate out.}
\item{Theory: This specific procedure is very interesting one to follow. This is a good series of reactions to show the reversibility of the electrochemical reactions. In one reaction, the metal will reduce. However, when we add another metal, that very same metal will oxidize.}
\end{itemize}

\subsection{Electrolysis of Water, Part A – The H and the O}
\begin{itemize}
\item{Preparation time: 30 minutes}
\item{Materials: 3 or 4 live d cell batteries, 2 dead d cell batteries, water, Epsom salt or baking soda, wires, 1 LED light for testing the circuit}
\item{Procedure: Make a salt solution with Epsom salt or baking powder. The actual concentration does not matter, but if it is too dilute, it will cause some difficulties in the electrolysis. Take apart the two dead d cell batteries and collect the two graphite electrodes. Place the live batteries in series and connect two wires: one for the positive end and one for the negative. Run the positive line to one electrode and the negative one to the other electrode. Place both electrodes in the Epson salt or baking soda solution without touching each other. Bubbles should form at both electrodes. If no bubbles form, this may be caused from a few different problems. First, check if the batteries are indeed live. Second, use the LED light to check each connection. Third, ensure the solution is concentrated enough to allow the electrons to move between electrodes. }
\item{Theory: In this particular cell, the reaction follows the hydrolysis of water. In cells like this one, the cathode is the electrode where oxidation occurs and the anode is the electrode that reduction occurs. At the cathode, hydrogen reduces according the half cell reaction:. This is the production of hydrogen gas in the hydrolysis of the water. At the anode, the oxidation of the hydroxide ion occurs. The half-cell reaction is. The hydroxide oxidizes to from water and oxygen gas. This can be seen at the anode. The purpose of the magnesium sulfate is to provide a salt in order to allow electrons to flow through the solution and complete the circuit. Since water has poor electrical conductivity, a strong electrolyte needs to be added in order to raise the conductivity. In this activity, baking soda or Epsom salt act as that electrolyte. Further, these electrolytes will not influence the reaction.}
\end{itemize}

\subsection{Electrolysis of Water, Part B – Capturing H and O}
\begin{itemize}
\item{Preparation time: 60 to 90 minutes. }
\item{Materials: Same as Electrolysis of Water, Part A – the H and the O, two syringe shells.}
\item{Procedure: The procedure is the same as Hydrolysis of Water, Part A with one exception. In this activity the gases, both hydrogen and oxygen, can be captured. The needle and the plunger are not needed. Run a wire through the top of the syringe shell to the batteries. Seal the top of the shell to prevent any gas from leaking. Repeat for the second syringe shell. Ensure the electrode is held in place by either gluing or using the wire for the cell. Place both syringe shells and electrodes in the Epsom salt solution. The gas collection is very visible by looking at the displacement of the solution in the syringes. }
\item{Theory: Hydrolysis of water is now captured in this activity. Hydrogen gas is collected at the cathode and oxygen is collected at the anode. If the volumes of both the hydrogen and the oxygen are measured, the hydrogen gas will have twice the volume of the oxygen gas. This is due to the complete reaction for the hydrolysis of water. The reaction for the hydrolysis of water is found by adding the half-cell reaction at the cathode with the half-cell reaction at the anode. Adding the two half-cell reactions yields the reaction. In fact, this reaction can be demonstrated visually by looking at the production of the two gases. Since the reaction states that for 1 mole of oxygen gas, 2 moles of hydrogen gas are produced, expect that the volume of hydrogen will be twice that of oxygen. Historically, this was one of the proofs that water consists of both hydrogen and oxygen. Further, it also confirmed that the ratio of hydrogen to oxygen in water molecules is 2 to 1.//
The purpose of the syringes is to collect the gas to be tested at a later time. Carefully remove the syringes keeping in mind that hydrogen is less dense than air and oxygen has about the same density as air. The test for hydrogen gas is the pop test. Take a match or a piece of burning paper and put it in the hydrogen gas. The hydrogen combusts in air, causing a pop sound. To test for oxygen through the glowing splint test; lower a glowing piece of paper in and see if the paper relights. It is possible the syringes might melt. }
\end{itemize}

\subsection{Electrolysis of Salt Water; Production of Chlorine}
\begin{itemize}
\item{Preparation time: 30 minutes}
\item{Materials: Same as Electrolysis of Water, Part A – the H and the O, table salt  }
\item{Procedure:  it is same as Electrolysis of Water, Part A – the H and the O, with one change: table salt instead of Epsom salt or baking soda. On the cathode, hydrogen will form. On the anode, chlorine gas is produced. Chlorine is a green gas that is poisonous with a pungent smell.}
\item{Theory: The half-cell reaction for the cathode is. The half cell reaction is. The overall electrochemical reaction is. In the Electrolysis of Water, Part A, the reducing agent was hydroxide ion but in this activity, chlorine gas is the reducing agent. It is possible to capture the chlorine gas by using the same apparatus as Electrolysis of Water, Part B. Keep in mind safety. Do not produce chlorine in a small, poorly ventilated room. Further, do not breathe chlorine gas directly.//
As the chlorine gas reacts with the hydroxide ion in the water, it will form hypochlorite (ClO-). Together with the sodium cations from the salt, this is sodium hypochlorite, the active ingredient in bleach. Industrial bleach production is the same process at much larger scale.}
\end{itemize}

\subsection{Electroplating Copper}
\begin{itemize}
\item{Preparation time: 10 minutes}
\item{Materials: 3 to 6 live batteries, metal object to be plated like a spoon, copper metal, wires, baking soda or Epsom salt, LED light for testing connections}
\item{Procedure: Connect the metal object to be copper plated to a wire. Run this wire to batteries. Run a wire to the other side of the batteries and connect this to a piece of copper. If using copper wire as the source of copper, remove the insulating coating to allow the copper metal to have contact with the solution. Place a salt solution, Epson salt or baking soda, in a container. Submerge the object to be copper plated and the copper metal in the solution. In a short time, copper should appear on the metal object. If run for a long time, this reaction will completely plate the metal object and the copper metal will disappear. If there is no reaction occurring in the cell, use a LED light to test each individual connection to ensure a close circuit.}
\item{Theory: In the previous activities of electrolytic cells, a chemical in the solution either oxidizes or reduces to form the product. In this particular cell, the copper metal at the anode will first undergo oxidation to form copper ions, Cu2+. The copper ions will migrate towards the cathode where reduction takes place. The copper ions gain electrons to form copper metal once again. The copper will form on the cathode object, and the reddish brown color of copper metal becomes quite visible. With enough time, the entire object will be covered with copper metal. Given enough time and battery power, the entire copper metal will react, disappear, and plate out on the object. In the sample set up, a spoon will be copper plated. For this activity to work, any electrically conducting object will work: a graphite electrode, a piece of metal, a spoon, etc. A normal household utensil is recommended. To clean the object, the copper metal will just rub off. This is the process for all different types of metal plating. Chrome plating uses a cell like the one used in this activity except chromium is used instead of copper. A process called galvanization can counter rusting. Galvanization is the electroplating of a metal, which does not rust, on top of another metal that is prone to rusting. For example, galvanized nails are iron nails with zinc electroplated onto the surface of the iron.}
\end{itemize}

\section{Flame Tests}

\subsection{Flame Test, Part A – Colored Flames}
\begin{itemize}
\item{Preparation time: 5 minutes}
\item{Materials: lighter, salts of different metals, like CuSO4}
\item{Procedure: Take a pinch of salt and sprinkle it into the flame of the lighter. Watch the flame; its color will change depending on metal in the salt.}
\item{Theory: The electrons in metal atoms move to different orbitals (energy levels) when heated. As the metal cools, the electrons fall to a lower orbital. When the electrons fall to a lower level, they release the difference in energy as a photon. Sometimes these photons are in the visible spectrum. This gives a color to the flame. Different metals give different colors because the wavelength of the photons depends on the energy between orbitals in the atom. Electronic configurations and atomic charges lead to the differing energies between the orbitals giving rise to the different colors.
\begin{itemize}
\item{Lithium compounds give a red color when placed into a flame. See the instructions for extraction lithium from old cell phone batteries in Chemicals in Shika na Mikono Volume 1.}
\item{Calcium hypochlorite is bleaching powder. When placed into a flame, it gives an orange color. Calcium hydroxide should also produce an orange color. Gypsum is calcium sulfate.}
\item{Sodium compounds, like salt or sodium carbonate, give a yellow color to a flame. Even though a lighter has a yellow part of the flame, there is a greater yellow color.}
\item{Boric Acid is composed of boron, which gives a green color when it burns.}
\item{Borax is sodium borate, so when in the flame test it gives a yellowish green color. The yellow is from sodium and green from the boron.}
\item{Copper sulfate gives a green color in the flame while copper chloride gives a blue color.}
\item{A mixture of 3 parts potassium sulfate and 1 part potassium nitrate gives a purple color. The purple flame from the potassium. This mixture of potassium sulfate and potassium nitrate is necessary because potassium sulfate does not burn by itself and potassium nitrate is saltpeter, which is explosive by itself. However, Potassium chloride gives a purple flame by itself.}
\item{Magnesium ribbon, solid magnesium, also burns with a white flame.}
\item{Aluminum also burns white. For aluminum, it will not burn unless aluminum solid is used. Aluminum oxide does not give a color due to the protective oxide on the outside of the metal.}
item{Zinc metal gives a green color in a flame.}
\end{itemize}
} % Theory
\end{itemize}

\subsection{Flame Test, Part B – Colored Fireballs}
\begin{itemize}
\item{Preparation time: 10 minutes}
\item{Materials: iron wool, balloon, battery acid, salt for flame test. Matches}
\item{Procedure: Put some salt in a balloon. This is the salt for the flame test. Use the balloon and fill it with hydrogen gas; use a water bottle with iron wool and battery acid to make hydrogen and let it flow through an IV tube into the balloon with the salt. Once the balloon has some hydrogen, bring it to a flame. See Hydrogen Production, Part B.}
\item{Theory: Instead of using a flame from a lighter, we can burn hydrogen gas to make the flame test occur. This variation on the activity is fun because when the hydrogen balloon combusts, it will give a colored flame. }
\end{itemize}

\subsection{Flame Test, Part C - Spritzer}
\begin{itemize}
\item{Preparation time: 20 minutes}
\item{Materials: hard alcohol of at least 35\% alcohol by volume (methylated spirits will work well), salts for flame test, spray bottle, any high temperature open flame, like motopoa, Bunsen, or kerosene burner.}
\item{Procedure: Put some salt inside a spray bottle. Dissolve it with some hard alcohol. Spray this solution into a burner. There should be a flame test.}
\item{Theory: This variation of the flame test uses alcohol as a medium for the metal ions. As the alcohol burns, it leaves behind the metal ions that produce the color in the flames. }
\end{itemize}

\section{Fun Liquids}

\subsection{Making Borax}
\begin{itemize}
\item{Preparation time: 20 minutes}
\item{Materials: boric acid, either in powder form or as a liquid found in bottles (see the procedure for more explanation), baking soda}
\item{Procedure: Dissolve some powdered boric acid the smallest amount of water possible. Carefully add baking soda without adding excess. Let the white precipitate settle. Decant off the liquid leaving behind the powder. Let it dry. This is sodium borate or borax. If there is no powdered boric acid available, it is possible to buy antifungal boric acid solutions. These are also good sources of boric acid. Add, without adding to excess, baking soda to the liquid. There will be no effervescence. Look for precipitate formation. Adding baking soda slowly so it dissolves and white precipitate forms. When excess baking soda is added, the soda will dissolve but no new white precipitate will form. Let the precipitate settle, decant, and let the solid dry. }
\item{Theory: Borax is sodium borate. This is formed easily by neutralizing the boric acid with the baking soda since the carbonate will form carbon dioxide. Use this solid whenever one of these activities calls for borax.}
\end{itemize}

\subsection{Slime}
\begin{itemize}
\item{Preparation time: 20 minutes}
\item{Materials: warm water, hot water, glue, food coloring, metal spoon, borax from above}
\item{Procedure: Dissolve a small spoonful of borax into 250 mL of hot water. Stir to dissolve. In another container, mix 500 mL of office or clear glue with 500 mL of warm water with a few drops of food coloring. With a metal spoon beat the borax solution into the glue solution. It will be difficult at first, but will become easier with more work.}
\item{Theory: This is a fun polymer to make since it has such a unique texture. Enjoy playing around with this substance; it is safe to touch. This is an example of a polymer, one that may be formed from everyday materials. When you are done playing, store the slime in an airtight container. Do not eat, and do no throw down a drain. To dispose, put it in the trash.}
\end{itemize}

\subsection{Flubber}
\begin{itemize}
\item{Preparation time: 20 minutes}
\item{Materials: borax, warm water, glue, food coloring}
\item{Procedure: In one container, mix 300 mL of water with 500 mL of glue and some food coloring. In a second container, mix 300 mL of water with 3 spoonfuls of borax. Mix each container thoroughly and then gently mix together. The flubber will be sticky at first. Let the excess liquid drain, and begin to play.}
\item{Theory: This is a variant of the Slime activity. The difference between these two is the ratio of borax to glue. By changing this ratio, the texture and consistency of the flubber will also change. When you are done playing, store in an airtight container. Do not eat, and do no throw down a drain. Put it in the trash.}
\end{itemize}

\subsection{Making Oobleck}
\begin{itemize}
\item{Preparation time: 10 minutes}
\item{Materials: cornstarch, water, newspaper, food coloring, shallow mixing bowls, hammer}
\item{Procedure: This activity is difficult to have a defined procedure, since it relies so much on the quality of cornstarch. Begin by putting 200 g of cornstarch in a mixing bowl. Slowly add water mixing the entire time. It is better to use your hands for this process. Add water until the consistency becomes something like honey and that it gives a little resistance and tears on the surface of the mixture. Add food coloring here if you want it to be colored. Use newspaper on all the surfaces to prevent a mess.}
\item{Theory: Oobleck is a non-Newtonian fluid. It is a suspension of starch molecules with water. It is not like a sugar water solution where the sugar dissolves in the water, the starch particles are suspended in the water. If you do not stir long enough, the cornstarch will settle to the bottom.@	Use this procedure for making oobleck for the following activities. It is very important to remember not to pour this suspension down the drain. It will clog drains. It also tastes bad, so do not eat it, and throw it away when you are done with it.}
\end{itemize}

\subsection{Exploring oobleck}
\begin{itemize}
\item{Preparation time: 0 minutes}
\item{Materials: Oobleck from Making Oobleck, bowl, hammer}
\item{Procedure: In the mixing bowl, gently move your fingers through the oobleck. Observe the texture. Grab a handful of oobleck and quickly squeeze it. It will become like a rock. Release the pressure and the oobleck will flow out of the hands like a liquid. Slap the oobleck and observe the texture. Remove all hands from the oobleck and hit it with a hammer. The hammer will bounce on the surface. Do it again with the hammer, but instead of swinging it, let it gently rest on the oobleck and watch it sink.}
\item{Theory: All fluids have a property called a viscosity constant. This describes the fluidity of the liquid. When this viscosity constant remains constant, the liquid is described as a Newtonian liquid. Oobleck, on the other hand, changes viscosity when different amounts of pressure are applied. If a student pokes the oobleck or hits it with a hammer or applies a large force in a small amount of time, the oobleck increases its viscosity greatly and stays in place. If a student applies no pressure or pours it gently, the viscosity becomes similar to water or honey. When a fluid acts like this, it is called a dilatant material. \\
Oobleck is believed to behave this way due to kinetics between the starches and the water. When sitting still the granules of starch are surrounded by water. The surface tension of the water keeps it from completely flowing out of the spaces between the granules. The cushion of water provides quite a bit of lubrication and allows the granules to move freely. But, if the movement is abrupt, the water is squeezed out from between the granules and the friction between them increases rather dramatically. This makes the oobleck becomes almost solid. }
\end{itemize}

\subsection{Walking on Water}
\begin{itemize}
\item{Preparation time: 30 minutes}
\item{Materials: lots of cornstarch, water, big shallow pool or basin, kangas}
\item{Procedure: Make enough oobleck to fill a bottom of a basin or pool. Place some kangas around the basin. When it is ready, take off your shoes and walk on the oobleck. If you walk quickly, you will walk on the top easily. If you walk slowly, you will sink in the oobleck.}
\item{Theory: This is basically the oobleck activity from earlier, but it is on a larger scale. The students will be amazed when you can walk on water.}
\end{itemize}

\subsection{Dancing Oobleck}
\begin{itemize}
\item{Preparation time: 10 minutes}
\item{Materials: oobleck from Making Oobleck, a speaker, plastic wrap or a plastic bag}
\item{Procedure: Turn a speaker so that its speaker is facing straight up. Cover with plastic wrap or a plastic bag. Be absolutely sure nothing can leak or fall into the speaker. Pour on some oobleck. Turn on the speaker and play some music that has some deep base notes. The oobleck will bounce to the beat.}
\item{Theory: As the music plays, the sound from the speakers transfers its kinetic energy into the oobleck. Due to the nature of its composition, the oobleck will respond to the music by bouncing to the beat.}
\end{itemize}

\section{Gas Laws}

\subsection{Boyle’s Law, Part A – Syringe}
\begin{itemize}
\item{Preparation time: 0 minutes}
\item{Materials: One syringe of any size minus metal needle}
\item{Procedure: Fill the syringe with air until the end of the graduations. Place a finger at the tip of the syringe to create a seal. Press the plunger as far as possible. Make a competition with the students to see which person can decrease the volume the greatest. It should be easy to decrease the volume most of the way but impossible by human means to completely squeeze out the air.}
\item{Theory: Boyle’s law states that the pressure of a gas at a constant temperature is inversely proportional to the volume. As the pressure increases, the volume decreases. As the pressure decreases, the volume increases. As the plunger pushes down on the gas, the volume decreases. As the progressively decreases, the pressure to push the plunger progressively increases. The pressure becomes so great that it is hard to puss the plunger in all the way. }
\end{itemize}

\subsection{Boyle’s Law, Part B – Cartesian Diver}
\begin{itemize}
\item{Preparation time: 5 minutes}
\item{Materials: 1 clear plastic water bottle with cap that forms a good seal, syringe, some weights like small nuts or nails, water}
\item{Procedure: Fill the water bottle completely with water. The water should be at the brim. Place the syringe, with the inside loaded with some weights and some air, carefully in the top of the bottle. The wings on the syringe may need to be cut in order to make the syringe fit through the bottleneck. It might bob out of the top of the bottle a little. Seal the bottle with the cap. Squeeze the bottle. The syringe should sink. Release the pressure and the syringe rises again. If the force required to squeeze the bottle in order to make the syringe sink is too great, there are two things to be done. First, ensure that the water is completely to the brim of the bottle. Second, adjust the size of the bulb on the syringe to minimize the volume of air. Find a new syringe if the syringe leaks water on the inside.}
\item{Theory: An object will float or sink depending on its density relative to the liquid it is in. In this case, if the syringe is less dense than water, it floats. If the density becomes greater than the density of water, it will sink. This is exactly what is happening inside the syringe. Inside, there is some air trapped on the inside. This gas is subject to Boyles’ law. Applying pressure to the bottle increases the pressure pushing on the syringe. This pressure makes the volume of the syringe contract. Since density is defined as the mass over the volume, by squeezing the bottle the density changes. The mass does not change, but the volume of the syringe decreases because the volume of air is compressed. As the volume decreases, the density of the syringe increases. If the density increases sufficiently, the syringe sinks. This is also known as a Cartesian Diver.}
\end{itemize}

\subsection{Boyle’s Law, Part C – Filling a Balloon}
\begin{itemize}
\item{Preparation time: 0 minutes}
\item{Materials: 1 bottle, 1 balloon}
\item{Procedure: First, blow up the balloon to stretch out the balloon and show that there are no holes. Release all air. Stretch the balloon such that it hangs in the bottle. Have students try blow up the balloon inside the bottle. It is impossible for a normal person to fill this bottle.}
\item{Theory: This is another good example of Boyle’s Law. Usually when balloons are used, we think of the gas inside the balloon. However, this time we are concerned with the air inside of the bottle. By filling the balloon, the air of the balloon increases. This means that the volume of the air inside the bottle decreases. In order to decrease the volume of the air inside the bottle, Boyle’s Law says that the pressure needs to increase. The normal human’s lungs cannot blow enough air at a high enough pressure to fill the balloon inside the bottles. }
\end{itemize}

\subsection{Boyle’s Law, Part D – Sucking a Balloon}
\begin{itemize}
\item{Preparation time: 10 minutes}
\item{Materials: balloon, plastic water bottle, straw}
\item{Procedure: In the plastic water bottle, put a straw through part of the wall. Seal it up so that it does not leak air. Place a balloon over the mouth of the bottle so it falls into the bottle. Use the straw to suck air out of the bottle to have the balloon fill with air.}
\item{Theory: As we suck the air out of the bottle, the volume of the air inside of the bottle gets smaller due to Boyle’s Law. The atmospheric pressure compensates by pushing the balloon into the bottle, which fills up with air.}
\end{itemize}

\subsection{Charles’ Law, Part A – Coin Cap}
\begin{itemize}
\item{Preparation time: 30 minutes}
\item{Materials: 2 coins, 2 bottles, some way to cool the air in one bottle either through refrigerator or ice.}
\item{Procedure: Take a coin and a bottle. Place the coin over the mouth of the bottle so it covers the entire opening. This is the control bottle. In a second bottle, place the bottle coin a refrigerator with the coin next to it. After 25 minutes, the air inside of the bottle cools down. Remove the bottle from the refrigerator and immediately cover with a coin as before. If no refrigeration is available, take some ice water or cold water and pour into the bottle. Swirl and mix the cold water to ensure the bottle is cold. Pour out the water and cover with the coin. Let the two bottles sit next to each other. After a short time, the coin on the refrigerator bottle will be blown off the top. }
\item{Theory: Charles’ law states that temperature is directly proportional to volume. As the temperature increases, the volume increases. As temperature decreases, the volume decreases. In this activity, the temperature of the first bottle remains constant so nothing happens. However, the air in the second bottle is at a lower temperature so it has less volume. When the temperature increases, the volume of the air expands in volume. This is shown by the coin being pushed off the lid of the container. The air expands but the coin stands in the way. The air pushes the coin so that it is possible to expand further in volume.}
\end{itemize}

\subsection{Charles’ Law, Part B – Spray Time}
\begin{itemize}
\item{Preparation time: 15 minutes}
\item{Materials: 1 can of non-CFC aerosol spray (e.g. Rungu insect repellent), 1 balloon.}
\item{Procedure: Place the plastic bag or balloon to act as a container over the mouth of the spray container. Use the container and spray it into a balloon. If the balloon is too small, use a funnel. The spray will liquefy and be cold inside the balloon. Tie the balloon. As the liquid warms up to room temperature, it will change from a liquid to a gas. Students should be able to hear and feel it boiling. Further, as the gas heats up, the balloon will expand in size.}
\item{Theory: Charles’ Law states that temperature of a gas at constant pressure is directly proportional to volume. Inside of the spray cans, there is a chemical held under high pressure. Phase diagrams show that gases under high pressure become liquids. The pressure is released and the temperature cools. This is called Joule-Thompson effect. It is an adiabatic expansion. However, by spraying long enough the temperature will cool down to the point that the chemicals will change back to a liquid. This liquid can be transferred to the balloon where it changes back into a gas quickly. This is where Charles’ Law comes into play. As the gas comes to room temperature, the volume of the trapped gas will increase.\\
This activity also works quite well for showing phase transitions. As liquids change to a gas, they do not disappear. They still exist even though they may not be seen. Here, the liquid changes to a gaseous state, which accompanies the expansion of the balloon. The size of the balloon is the direct representation of the liquid molecules that have evaporated. The mass of the balloon will also be different than that of one filled with normal air.}
\end{itemize}

\subsection{Charles’ Law, Part C – Bottle Crush}
\begin{itemize}
\item{Preparation time: 10 minutes}
\item{Materials: water bottle, boiling water}
\item{Procedure: pour some boiling water into the water bottle. Cap the bottle and shake to make sure all the air in the bottle is heated from the hot water. Open the bottle and pour out the liquid. Recap the bottle. After a short time, the bottle will contract.}
\item{Theory: Charles’ law states that volume is proportional to temperature. By capping the hot air inside of the water bottle, the volume of the air inside the bottle will decrease as the temperature of the gas cools off. As the volume of the air reduces, the atmospheric pressure crushes the plastic water bottle.}
\end{itemize}

\subsection{Egg Suck}
\begin{itemize}
\item{Preparation time: 0 minutes}
\item{Materials: 1 pealed egg, 1 glass bottle with a narrow mouth, like a konyagi bottle, matches, small piece of paper}
\item{Procedure: With the egg ready to cap the alcohol bottle, use a match to light the piece of paper on fire. Drop the paper into the alcohol container. Let it burn for a second, and then cap the bottle with the egg. The egg should be sucked slowly into the bottle if the egg is not too large. If it does not pull into the bottle, you can try again but use petroleum jelly on the mouth. Even if the egg is not sucked into the bottle completely, there will be a suction holding the egg to the bottle. It is possible to lift the bottle by the egg.}
\item{Theory: The burning match and paper heat the air inside the bottle. When we cap the bottle with the egg, we seal the air inside of the bottle. This air sealed inside is at a higher temperature than the surroundings. As the bottle cools down, the pressure of the air inside the bottle decreases. This is a direct example of Charles’s law. As the pressure inside drops, the atmospheric pressure still pushes down onto the egg. If pressure difference is sufficient, the egg will be pushed slowly into the bottle. }
\end{itemize}

\subsection{Avogadro’s Law, Part A – The Breath}
\begin{itemize}
\item{Preparation time: 0 minutes}
\item{Materials: 2 balloons}
\item{Procedure: Take one balloon and blow one breaths worth of air into it. Tie it off. Take the second balloon. This time, blow two breaths worth of air into the second balloon. Tie it off. The second balloon will be larger than the first. }
\item{Theory: Avogadro’s Law states that for a gas at constant temperature and pressure, the volume is directly proportional to the number of moles. Moles are a reference for the number of particles, or it can be thought of the amount of stuff. In other words, this law basically says the volume of the gas is related to the number of gas molecules. Here, one breath of balloon has fewer air molecules than the balloon of two breaths. Therefore, the volume of the one-breath balloon is smaller than the volume of the two-breath balloon.}
\end{itemize}

\subsection{Avogadro’s Law, Part B – The Spray}
\begin{itemize}
\item{Preparation time: 10 minutes}
\item{Materials: 2 balloons, 1 can of non CFC aerosol spray.}
\item{Procedure: This is the same procedure as Charles’ Law, Part B – Spray Time, except that the amount of spray is controlled. Once the spray starts to become a liquid, spray it for 5 seconds longer. Tie off the balloon. Repeat the same except extend the spray time to 10 or 15 seconds. Tie off the balloon. The second balloon will be larger.}
\item{Theory: The longer the spray is collected, the more of the liquid is collected. As more of the liquids collects, it means that more of the molecules are placed inside of the balloon. When the liquid evaporates and expands, the balloon with more molecules will have the larger volume. This is expected from Avogadro’s Law. The larger balloon should be heavier.}
\end{itemize}

\section{Gas Production}

\subsection{Hydrogen Production, Part A – In the Bottle}
\begin{itemize}
\item{Preparation time: 10 minutes}
\item{Materials: steel wool, citric acid, plastic water bottle, matches}
\item{Procedure: Put some steel wool in the bottom of a plastic water bottle. Pour in enough concentrated citric acid solution to cover the iron. Crush the bottle and cap. The bottle will refill. If using citric acid, you can simply invert the bottle, drain the acid into a container, and bring a lit match to the open bottom. There should be a pop sound. If using battery acid, or any strong acid, do not do this experiment in this way! The acid can burn hands, clothing, eyes, etc. Instead, do Part B below.}
\item{Theory: Hydrogen is an easy and fun gas to make. Steel wool, iron, reacts with acid based on the following reaction (sulfuric acid is used as a model): . By crushing the bottle, it is easy to see that the bottle is filling with a gas. The most common test for hydrogen is the ‘pop’ test. This test involves taking a lit splint, match, or burning piece of paper and taking it to the hydrogen. In this activity, open the lid, and have a student bring the match to the mouth of the bottle. Do this quickly since hydrogen is lighter than air. Every can hear a definite pop sound. This pop sound comes from the combustion of hydrogen in air,. Do not just throw out the waste solution; we need to neutralize any excess acid with baking soda. Also, this activity is a little more dangerous so take caution in deciding which students should be doing this activity.}
\end{itemize}

\subsection{Hydrogen Production, Part B – In a Balloon}
\begin{itemize}
\item{Preparation time: 10 minutes}
\item{Materials: steel wool, battery acid or citric acid, plastic water bottle, IV line, balloon. matches}
\item{Procedure: Run an I.V. line through the cap of the water bottle. On the other end of the line, connect a balloon. Follow the instructions from Hydrogen Production, Part A but instead of crushing the bottle, cap the bottle with the cap with the IV line. In this activity, the balloon will fill with hydrogen. Once the balloon fills with hydrogen, bring a match near the surface of the balloon. This balloon will combust in a small, but flashy flame. This also can be quite loud, depending on the size of the balloon. Note that concentrated citric acid is much safer than battery acid, but the reaction happens more slowly. In some balloons, the hydrogen will leaks faster than it is produced.}
\item{Theory: This activity is very similar to the previous one, except the hydrogen is moved from the bottle to the balloon, making it portable, and also removing the risk of pouring it out. If you tie the end of the balloon, the hydrogen can be captured for a short period of time. It may be possible to create a balloon that floats, or at least sinks less quickly than a normal balloon, by first crushing the bottle and filling the balloon with enough hydrogen. This is not an effective way to store hydrogen for another day as the molecules diffuse out of the balloon over time.}
\end{itemize}

\subsection{Oxygen Production, Part A – Manganese Catalysis}
\begin{itemize}
\item{Preparation time: 10 minutes}
\item{Materials: D cell battery, old or new, hammer (or rock), hydrogen peroxide, match and paper, plastic water bottle}
\item{Procedure: Smash open the battery. Inside there is a black powder. This black powder is manganese (IV) oxide, MnO2. It may be necessary to peal back a metal casing with a pair of pliers. Be sure to clean them afterwards since manganese dioxide is corrosive to metal. Scoop out the powder and put in the bottom of the plastic water bottle. Pour in some hydrogen peroxide, maybe 20 mL at most. Crush the bottle and cap. The bottle will inflate with oxygen gas. Test for oxygen using a glowing splint.}
\item{Theory: Hydrogen peroxide is a rather reactive compound. It can decompose into water and oxygen rather easily. The reaction is. The manganese acts as an inorganic catalyst. This means that the MnO2 speeds up the production of oxygen. As the reaction proceeds, the bottle will fill with a gas. To test for oxygen, there is a test called the glowing splint. Take a piece of paper, roll it like a cigarette, and ignite the end. Once it is burning for a few seconds, blow it out; we are looking for the glowing red color in the paper. Once you have this, slowly lower this glowing red part of the paper into the bottle. The paper will reignite. Remove the paper, blow out the flame, and lower the paper back into the bottle. It should reignite a second time.}
\end{itemize}

\subsection{Oxygen Production, Part B – Yeast Catalysis}
\begin{itemize}
\item{Preparation time: 10 minutes}
\item{Materials: --}
\item{Procedure: --}
\item{Theory: See Catalase Enzyme, Part A in the biology section for procedure for making oxygen using yeast and hydrogen peroxide. }
\end{itemize}

\subsection{Oxygen Production, Part C - Elephant Tooth Paste}
\begin{itemize}
\item{Preparation time: 0 minutes}
\item{Materials: fused syringe shell, powdered soap, yeast hydrogen peroxide, food coloring}
\item{Procedure: In the bottom of a syringe shell, mix some powdered soap and yeast together. If desired, let a drop of food coloring down the inside of the syringe shell. Pour into some 5 mL of hydrogen peroxide. Bubbles will start coming out of the syringe}
\item{Theory: Catalase decomposes hydrogen peroxide to form water and oxygen gas. The soap (dissolved in the water from the peroxide solution) traps the gas in bubbles. These bubbles build up upon each other slowly forcing them out of the syringe shell. If you use food coloring in the process, the bubbles will have a color stripe in the side of the bubbles. This will make the stream of bubbles coming out of the syringe look much like tooth paste. Of course, this is not toothpaste, and hazardous to put in your mouth.}
\end{itemize}

\subsection{Carbon Dioxide Production, Part A – Baking Soda}
\begin{itemize}
\item{Preparation time: 10 minutes}
\item{Materials: 1 jam jar, baking soda, candle, citric acid solution, plastic bottle lid.}
\item{Procedure: In a jam jar or a beaker, pour in 50 mL of water with a spoonful or two of baking soda. Put the candle sitting on the bottle lid, so that the candle is not sitting in the liquid. Light the candle, and add citric acid solution to the water. Bubbles will form and with enough time, the carbon dioxide from the reaction will put out the candle.}
\item{Theory: Baking soda is a chemical called bicarbonate of soda or sodium bicarbonate. This compound, NaHCO3, reacts with an acid to form carbon dioxide. The reaction is a 2 part reaction. First, the carbonate reacts with the acid:. H2CO3 is called carbonic acid. This acid quickly decomposes in water to form carbon dioxide and water:. This is the source of the carbon dioxide that will put out the candle. The carbon dioxide puts out the candle because in order for combustion to occur, oxygen must be present. Carbon dioxide is heavier than air and oxygen, and so as carbon dioxide accumulates, oxygen and other gases are displaced up, eventually out of the reach of the candle.}
\end{itemize}

\subsection{Carbon Dioxide Production, Part B – CO2 Balloon}
\begin{itemize}
\item{Preparation time: 10 minutes}
\item{Materials: 2 plastic water bottles, baking soda, citric acid, IV line, cement or lime water, water}
\item{Procedure: Follow the same procedure from Hydrogen Production, Part B – In a Balloon to put an IV line through a plastic water bottle lid. In a water bottle, put some cement in some water. This makes limewater. Decant to harvest the liquid to leave the solid behind. In a second water bottle, mix some water and baking soda. Add some citric acid and cap the bottle with the lid that has the IV line through it. Run the IV line into the limewater. The limewater will turn cloudy.}
\item{Theory: Another common test for carbon dioxide is called the ‘limewater test’. This is where carbon dioxide is bubbled through a limewater solution. Lime water is calcium oxide dissolved in water. As the carbon dioxide bubbles through the water, it reacts with the calcium to form a white, insoluble precipitate, calcium carbonate according the reaction:. This insoluble precipitate makes the solution appear cloudy. }
\end{itemize}

\section{Iron Chemistry}

\subsection{Activitying Ferrous Acetate}
\begin{itemize}
\item{Preparation time: 10 minutes}
\item{Materials: steel wool, vinegar, heat source}
\item{Procedure: Boil some steel wool in vinegar. The steel wool will dissolve in the vinegar forming ferrous acetate.}
\item{Theory: As the iron dissolves, it is giving up two electrons to for iron (II) acetate. Save this solution for later activities. Label as Ferrous Acetate.}
\end{itemize}

\subsection{Making Ferric Acetate}
\begin{itemize}
\item{Preparation time: 10 minutes}
\item{Materials: Ferrous Acetate, hydrogen peroxide}
\item{Procedure: Take a solution of ferrous acetate and add hydrogen peroxide. Save this solution for later activities. Label as Ferric Acetate.}
\item{Theory: Hydrogen peroxide is an oxidizing agent. In this case, the peroxide oxidizes iron (II) acetate to form iron (III) acetate. This is a brown solution.}
\end{itemize}

\subsection{Iron and Tea}
\begin{itemize}
\item{Preparation time: 5 minutes}
\item{Materials: ferrous acetate, ferric acetate, tea, jam jars}
\item{Procedure: Make some light tea. Do not make the tea too concentrated or it will be too hard to see the reaction. Separate the tea into two jars. Add ferrous acetate and ferric oxide to the tea.}
\item{Theory: The ferrous oxide will not react with the tea; however, the ferric oxide will react with the tannins in the tea. In tea, there are tannates, and these will displace the acetates. Iron (III) tannate produces a black precipitate.}
\end{itemize}

\subsection{Iron and Tartar}
\begin{itemize}
\item{Preparation time: 5 minutes}
\item{Materials: ferrous acetate, ferric oxide, cream of tartar or raisins, jam jars}
\item{Procedure: Add cream of tartar to two jam jars. Add ferrous acetate and ferric acetate to the jam jars.}
\item{Theory: Cream of tartar is the weak organic acid, tartaric acid. Ferrous acetate, again, does not react with the tartaric acid. However, the ferric acetate will react. The tartarate ion displaces the acetate ion. This forms a green iron (III) tartarate compound, ferric tartarate. If there is no cream of tartar available, raisins can be used. Add the raisins to barely enough water and let the acid seep out over night. Remove the raisins and continue the process. A common source of tartaric acid is grapes. It is possible that baobab fruit may contain tartaric acid. Experiment with these fruits; however, we are unsure if they will work.}
\end{itemize}

\subsection{Iron and Blood}
\begin{itemize}
\item{Preparation time: 20 minutes}
\item{Materials: Ferrous acetate, ferric acetate, dilute ammonia, hydrogen peroxide, jam jars}
\item{Procedure: In one jam jar, mix ferrous acetate and dilute ammonia together. Then add a small amount of hydrogen peroxide. Repeat this procedure for ferric acetate.}
\item{Theory: In this activity, the ammonia displaces acetate on the ferrous acetate. However, it does not react with the iron; the ammonia raises the pH to the point forming iron (II) hydroxide. Adding the peroxide forms iron (II) oxide and the solution should turn red. For the ferric acetate, the ammonia raises the pH to form insoluble iron (III) hydroxide. Adding the peroxide creates iron (III) hydroxide, a second red solution.}
\end{itemize}

\section{Miscible and Immiscible Liquids}

\subsection{Mixing Compounds, Part A – Solids and Liquids}
\begin{itemize}
\item{Preparation time: 15 minutes}
\item{Materials: 2 jam jars, sugar, water}
\item{Procedure: Add 50 mL of water to one jar and mark the level of the water. Repeat and make the same mark on the second jam jar. Then add another 50 mL to the second jar and make the second mark, thus showing 100 mL. Empty the water and wipe the jars dry. The actual volumes do not matter, just make sure one jar has a mark at 1 volume and the other jar has 2 marks at 1 volume and 2 volumes. In the first jar, add water until the mark. In the second jar, add sugar until the first mark. Then add the water in the first jar to the second jar with the sugar. }
\item{Theory: It seems logical that adding 50 mL of water and 50 mL of sugar yields 100 mL of total solution. However, this is not the case. Adding 50 mL water and 50 mL of sugar gives us a volume less than 100 mL. This is because there are intermolecular forces between sugar and water that draw the molecules closer together. These forces make the total volume of the solution contract just a little. This activity shows that 50 mL plus 50 mL does not always equal 100 mL. }
\end{itemize}

\subsection{Mixing Compounds, Part B - Liquids}
\begin{itemize}
\item{Preparation time: 15 minutes}
\item{Materials: 2 jam jars, methylated spirits, water}
\item{Procedure: Same as Mixing Compounds, Part A – Solids and Liquids  with one change. Instead of using sugar, use methylated spirits. }
\item{Theory: Intermolecular forces occur between many different molecules. Water and sugar is just one example. Methylated spirits have the same type of intermolecular forces involved in the contraction of the volume like the sugar. This is a useful activity to follow up if you have some clever students that recognize that 50 mL of sugar is not exactly 50 mL since there is sometimes space between sugar grains. In this activity, both components are liquids so there should be no question to the overall volume. However, the contraction will be smaller so thinner containers will show the change in volume much easier.}
\end{itemize}

\subsection{Lava Lamp}
\begin{itemize}
\item{Preparation time: 10 minutes}
\item{Materials: Clear plastic water bottle, water, food coloring, oil, effervescing antacid tablets, flashlight,}
\item{Procedure: Fill the bottom 10 cm of a water bottle with water. Add a few drops of food coloring. Fill rest of the bottle with oil. Drop in an effervescing antacid tablet. Cap and put a flashlight underneath the bottle. Observe the colors and the movement of the liquids.}
\item{Theory: Oil is a compound that is hydrophobic. This means that it repels water due to its nature. Oil is a long non-polar hydrocarbon, while water is a small polar compound. This means that the water cannot mix with the oil layer. This is why there are two layers on mixing oil and water. Adding the effervescing antacid tablets dissolve and release carbon dioxide in the water layer. The carbon dioxide dissolves in the water and forms small bubbles of carbon dioxide. These bubbles trap small amounts of food coloring. These bubbles rise since they have a much lower density than water. When the bubble reaches the surface, the carbon dioxide escapes and the colored water bubble falls down through the oil layer.}
\end{itemize}

\subsection{Magic Milk}
\begin{itemize}
\item{Preparation time: 5 minutes}
\item{Materials: Milk, different food coloring, powdered soap, cotton ball or q tip, shallow dish or a plate}
\item{Procedure: Pour in just enough milk to cover the plate or the bowl. Use food coloring and place a few drops around the plate of the milk. Soak the cotton ball or q tip in some detergent water. Touch the center of the milk plate with the cotton ball or q tip. The colors will start to move and swirl towards the center.}
\item{Theory: Milk is made up of fats and different proteins. These are non polar molecules. Food coloring is a water solution with some coloring compounds. The water solution and the non polar milk barely mix. When the q tip soaked in detergent touches the milk, it leaves behind soap. It is a compound that is both polar on one end and non polar on the other end. When this compound mixes with the milk, some interesting things occur. The milk and the soap intermingle forming micelles. In addition, the surface tension of the water in the milk also breaks. This allows the food coloring to move around in the milk. In other words, the coloring will start to move in the milk mixing with the other colors. It forms a milk color wheel. Do not drink this milk mixture.}
\end{itemize}

\section{Organic Chemistry}

\subsection{Showing the presence of C in sugar}
\begin{itemize}
\item{Preparation time: 10 minutes}
\item{Materials: metal wire with an end coiled so as to hold a metal soda cap or a spoon, soda cap with plastic removed leaving only the metal or a steel spoon, sugar, candle}
\item{Procedure: Bend some metal wire so as to make a cap holder so the handle is a distance from the flame source. Wrap the end of the metal wire into a circle so it can hold a metal soda bottle cap. Place the cap in the metal holder and fill it with a little sugar, or just use a spoon. Using the metal holder, place the metal cap above the flame of the candle. As the cap heats up it will heat the sugar. Heating the sugar will partially burn it before turning completely to carbon and carbon dioxide. This partial combustion is visible as the sugar burns it turns brown and then black.}
\item{Theory:  As sugar burns in air, it partially combusts leaving behind carbon solid and other carbon compounds before the changing into carbon dioxide. These carbon compounds have a brown color and the carbon solid compounds have a black color. As the sugar heats, it will combust until we have a brown color, then to a black color. This black color is solid carbon. This is the same color as charcoal, which is another example of solid carbon. @	The browning of the sugar is called caramallization. As the sugar breaks down into smaller and different saccarides, they bring a very delicious taste. This is the process that is used very much in making different candies, especially toffees, brittles, and caramels. In fact, this is how many candies get their brown or dark color.}
\end{itemize}

\subsection{Emulsification, Part A – Two Layers}
\begin{itemize}
\item{Preparation time: 5 minute}
\item{Materials: 1 jam jar, kerosene, water, powder soap}
\item{Procedure: In the jam jar, mix 20 mL of water and 20 mL of kerosene. Observe the effects when mixing these two liquids.}
\item{Theory: kerosene is a hydrophobic compound. This means that kerosene repels water. This happens because long carbon chains compose kerosene which makes it non polar. Water is a polar compound. Polar compounds dissolve in polar compounds. Non polar compounds dissolve in non polar compounds. Non polar compounds will not dissolve in polar compounds.}
\end{itemize}

\subsection{Emulsification, Part B – Now One Layer}
\begin{itemize}
\item{Preparation time: 5 minutes}
\item{Materials: 1 jam jar, kerosene, water, powder soap}
\item{Procedure: Measure 5 mL of water and 5 mL of kerosene into a jam jar. Two layers are visible from the separation between kerosene and water. Add a half or 1 gram of powdered soap. Shake to mix. The two layers will form one cloudy layer. After a long time, the layers will separate.}
\item{Theory: When a non-polar and a polar liquid come in contact, they will form two layers. The intermolecular forces push each other away. Some compounds, like soap, can be in both layers. In the case of soaps, these compounds have both a charged and a non-charged end. This allows part of the compound to exist in a polar layer, part to exist in the non-polar layer. In fact, they form micelles; small balls of hydrophobic pockets with the polar ends on the outside of the ball. This allows the polar and non-polar to mix. This arrangement is unstable. Over time, it will separate out into two layers. The addition of the soap allows the molecules to mix giving plenty of time before the layers separate. Salad dressing and mayonnaise are examples of emulsions.}
\end{itemize}

\subsection{Converting Soaps Into Lipids or Fats}
\begin{itemize}
\item{Preparation time: 5 minutes}
\item{Materials: powdered soap, battery acid, water, jam jar}
\item{Procedure: take ½ of a spoon of powdered soap and dilute it to about 50 mL in a jam jar. Add about 2 mL of battery acid. Bubbles of oil will form on the surface of the water. }
\item{Theory: Soaps are actually long chains of hydrocarbons with a deprotonated carboxylic acid group on the end. Many from living sources oils, like cooking oil, are long chains of hydrocarbons with a protonated carboxylic acid group on the end. This protonated group allows soaps to cause oil and water to mix: it is both hydrophobic on one end and hydrophilic on the other. One end can dissolve in water while the other ends dissolves in the oil layer. This allows soap to be soluble in both water and oil solutions. In fact, this is why soaps are used to clean off grease, oil, and other organic solvents. The organic solvents are hydrophobic and the soap can dissolve them. The acid protonates the carboxylic acid to revert back to a fat structure. The soap no longer has a polar end, and is completely hydrophobic. It collects on the top and looks like oil. It is oil, and not suitable for cooking. Oil is a triglyceride: a molecule with three long hydrocarbon chains. The soap does not have the same structure. In fact, what were made are palmitic acids and stearic acids. }
\end{itemize}

\subsection{Difference Between Ionic and Covalent Compounds}
\begin{itemize}
\item{Preparation time: 0 minutes}
\item{Materials: 1 lighter, 1 spoon, citric acid, salt}
\item{Procedure: Place a small amount of citric acid the spoon. Heat the spoon. It should bubble first, and then later turn brown then black as the compound thermally decomposes and then pyrolyzes. Repeat with salt. Heating has no effect on salt.}
\item{Theory: Carboxylic acids decompose to carbon dioxide on heating. Heating citric acid will cause the three carboxylic acid groups to decompose quickly to carbon dioxide. This is shown by the bubbles produced. Further heating will decompose the compound to solid carbon, which may itself burn off. When the experiment is repeated with salt, heating will cause no change in the salt. This is because salt is an ionic solid. This activity shows the difference between ionic and covalent compounds and the different interactions. Covalent compounds tend to have covalent bonds and weak intermolecular forces. In order to burn a compound, the bonds that compose the chemical must be break. These covalent compounds have their bonds decompose at relatively low temperatures; the bonds are broken by heating with a lighter. Ionic compounds have strong intermolecular forces that hold the molecules together. Heating salt with a lighter has no effect. In order to break the strong intermolecular forces in ionic compounds requires a great deal of heat, almost 1000 C. By showing that citric acids burns with a lighter and salt does not, students discover the difference between covalent and ionic compounds.}
\end{itemize}

\subsection{Cracking Household Oil}
\begin{itemize}
\item{Preparation time: 20 minutes}
\item{Materials: test tube, steel wool, two burners, oil}
\item{Procedure: Take a narrow a test tube and place it at an angle. At the bottom, place some normal household oil. At the top of the jar or test tube, stuff in some steel wool so it does not move. Place the tube through the stopper or lid and then put it firmly on the test tube so the only way for gas to exit is through the metal tube. Heat the steel wool until very hot. Then place a second burner under the oil so both the oil and the wool is being heated. After a minute or two, take a flame to the end of the tube. The gases escaping the tube can sustain a flame.}
\item{Theory: Cracking is a process where long hydrocarbon chains are heated so they are broken into smaller chains of molecules. Household oil is a triglyceride: a compound that has a glycerin backbone that has 3 long hydrocarbon arms. These long hydrocarbons range between 15 and 30 carbons long. These hydrocarbon arms can be broken easily by heating them in the presence of a catalyst, iron wool in this case. As they break down into smaller hydrocarbons, they become more volatile and combust easily. The long hydrocarbons will break down into hydrocarbon chains of 3 to 8 carbons. These will burn in air supporting a flame at the end of the tube.}
\end{itemize}

\subsection{Different Petroleum Products}
\begin{itemize}
\item{Preparation time: 10 minutes}
\item{Materials: syringe with metal needle, 3 jam jars, and as many as possible of: petrol, diesel, car lubricants, greases, petroleum jelly, kerosene, asphalt, tar, butane from a lighter}
\item{Procedure: Place some petrol, kerosene, diesel, and any other petroleum product in different jam jars. Compare the different properties of each of these compounds; look at density, viscosity, volatility, flammability, and more. Then use a syringe to remove the butane from a lighter. There is usually a small hole where a needle can be inserted to add more butane gas. However, it is possible to open this hole with a syringe needle and let the butane gas escape. It is a liquid in the lighter since it is under high pressure. As you use a syringe to release the butane, the pressure drops and it becomes a gas.}
\item{Theory: We use many petroleum products everyday. Crude oil is a black thick mixture of different hydrocarbons. To get different petroleum products, the crude oil is cracked and distilled to separate compounds depending on the number of carbon atoms. Butane is an early distillate since it has 4 carbons it distills easily. Petrol is an 8 carbon distillate. Kerosene has 12 to 15 carbons. Diesel has 15 to 25 carbons. Petroleum jelly is not a small hydrocarbon, but rather very long chains of varying length that does not distill easily. It is one of the last products from distilling crude oil. If you want to make some mock crude oil to show students, mix road tar with kerosene until you have a viscous liquid.}
\end{itemize}

\section{Plastics and Polymers}

\subsection{Differences in Plastics, Part A – Different Densities}
\begin{itemize}
\item{Preparation time: 5 minutes}
\item{Materials: different plastic objects, water, clear container}
\item{Procedure: Fill container halfway with water. Place different plastic pieces in the water. Depending on the composition of the plastic, some will float some will sink. }
\item{Theory: Plastics are made by polymerizing different monomers and therefore have difference densities. Some of these plastics have densities that are denser than water and will sink. There are a handful of common plastics and are identified by the number on the recycling label. Number one is polyetheneteraphthalate (PETE), two is high density polyethene (HDPE, crinkly plastic bags), three is polyvinylcholoride (PCV, plastic pipes), four is low density polyethene (LDPE), five is polypropene (PP), six is polystyrene (PS, brittle hard plastic cups, Styrofoam), and seven is anything else. }
\end{itemize}

\subsection{Differences in Plastics, Part B – Plastic and the Tower}
\begin{itemize}
\item{Preparation time: 10 minutes}
\item{Materials: density tower from Density Tower, Part A – Different Densities, different plastic objects}
\item{Procedure: Using the density tower, place different pieces of plastic and watch to see what layer the plastic sits in between.}
\item{Theory: The density tower can tell us a little more information about the different densities of the plastics. By looking at which layer the plastic settles at, we can deduce information about the density of each plastic.}
\end{itemize}

\subsection{Strength of Polymers, Part A – Floss Pull}
\begin{itemize}
\item{Preparation time: 5 minutes}
\item{Materials: dental floss}
\item{Procedure: Give each student a 1 meter long length of dental floss. Without using their teeth or a knife, let them try to break the floss.}
\item{Theory: Dental floss is made from a few possible polymers: nylon, Teflon, or polyethylene. Polymers tend to be very strong compounds. Unless a student has help in terms of a sharp edge, they will not be able to break the dental floss unless the floss is of the very cheap variety.}
\end{itemize}

\subsection{Strength of Polymers, Part B – Tearing Bottles}
\begin{itemize}
\item{Preparation time: 5 minutes}
\item{Materials: plastic water bottles}
\item{Procedure: Give each student a plastic water bottle. Without using a sharp edge, have the students try to tear apart the plastic bottles.}
\item{Theory: This is another example of the strength of plastics. This is one of the reasons that we use plastic for many different containers. It is light and very strong. If the same container were made out of glass or metal, it would be too heavy to use. Also, glass breaks too easy and metal reacts with many compounds. Plastic, on the other hand, does not react. These three properties make plastic perfect containers for different compounds. }
\end{itemize}

\subsection{Flammability of Plastic Compounds}
\begin{itemize}
\item{Preparation time: 10 minutes}
\item{Materials: samples of different plastic compounds}
\item{Procedure: Take small samples of plastic compounds and put them in a flame. They should rapidly ignite. Do not breathe in the fumes. Do not use PCV (plastic pipes) or any rubber compounds.}
\item{Theory: While plastic is very strong, light, and nonreactive, it does have one downside. Since it is a polymer made from different carbon compounds, plastics will combust when exposed to a flame. Plastics will contract and change shape under heat until they catch fire. Further, many polymers form toxic compounds when they burn. Therefore it is important to not to breathe the plastic fumes.}
\end{itemize}

\subsection{Non-Reactivity of Plastic Compounds}
\begin{itemize}
\item{Preparation time: 15 minutes}
\item{Materials: 5 different plastic containers, battery acid, caustic soda, iodine, water}
\item{Procedure: Put some battery acid in a plastic container. Do the same with iodine, caustic soda, and normal water. Further, bury one plastic bottle in the ground. After 1 month, check each bottle to see if they have reacted with anything.}
\item{Theory: Plastics are very nonreactive. After 1 month, no plastic bottle will have reacted. They will not react with an acid, a base, or an oxidant. Bacteria and other compounds that decompose different organic material cannot react with plastic either. Since they cannot react with the plastic, the bacteria do not break down the plastic bottle.\\
This is one of the reasons it is bad for the environment to through plastic bottles out of the bus window. Since the bottles do not break down in the environment, they will just remain there. This is why there are big piles of plastic water bottles on the side of the road. They will not break down or decompose so proper disposal of plastics is important.}
\end{itemize}

\section{Reactions and Kinetics}

\subsection{Kinetics of Baking Soda and Battery Acid}
\begin{itemize}
\item{Preparation time: 30 minutes}
\item{Materials: citric acid solution, sodium bicarbonate or baking soda, one plastic bottle, one syringe, water, stop watch}
\item{Procedure: Prepare the reaction container prior to the experiment. Take a lid and place a syringe needle through the lid. Seal the needle-cap connection with glue to prevent any leakage. Do not glue the cap to the bottle. Inside the bottle, place 50 mL of water and 1 g of baking soda. Mix well. Use the syringe to measure out 1 mL of citric acid solution. Connect the syringe to the needle. Start the stopwatch when adding the acid. Let go of the syringe; the plunger will rise as the reaction proceeds. Record the time required until the plunger stops moving or until it reaches the top. Repeat the experiment with different amounts of baking soda and battery acid. Repeat the experiment by changing one variable at a time: diluting the battery acid, increasing mass of baking soda, or increasing the amount of initial water. }
\item{Theory: Given the reaction, the reaction rate can be monitored by the production of a product. In this case, the formation of carbon dioxide is a way to monitor this reaction. Carbon dioxide gas forms and fills the container, as more and more gas is produced, the syringe will be pushed outwards according to Avogadro’s Law. By recording the time required and volume displaced in the syringe, we are studying the kinetics of the reaction. This analysis is much more qualitative than quantitative. Compare the rate of the bubble formation and the syringe rise speed in all the variations. }
\end{itemize}

\subsection{Exothermic and Endothermic Reactions}
\begin{itemize}
\item{Preparation time: 10 minutes}
\item{Materials: Supplies for Making Hydrogen, Part A, Kinetics of Baking Soda and Battery Acid}
\item{Procedure: Make both of these activities side by side. Instead of testing for hydrogen or observing the kinetics of the reaction, feel both bottles as the reaction proceeds. One bottle will feel warm while the other will get cold. }
\item{Theory: This activity provides concrete examples of both thermodynamic types of reactions: exothermic and endothermic. The formation for hydrogen from iron and sulfuric acid is a very exothermic reaction. This means that as the reaction proceeds, it releases energy into its surroundings, and the temperature of the reacting mixture increases. Students often get confused. Exothermic is the conversion of chemical potential energy in the chemicals to thermal energy that transfers this energy to the surrounding environment. In these activities, the reaction takes place in a water environment. This means as the energy transfers from the reaction to the water, the water will heat up. In contrast to the reaction of battery acid and iron, the formation of carbon dioxide from baking soda is endothermic. This reaction absorbs the thermal energy from surrounding environment, the water, and converts it to chemical potential energy necessary to drive this particular reaction. As thermal energy is being consumed and converted to chemical potential energy, the temperature of the reacting mixture decreases.}
\end{itemize}

\subsection{Inorganic and Organic Catalysts}
\begin{itemize}
\item{Preparation time: 20 minutes}
\item{Materials: D cell battery, yeast, water, heat source, 4 20 mL syringe shells, water, balloons, hydrogen peroxide}
\item{Procedure: In all syringe shells, put in 5 mL of water. Add yeast to two syringe shells and Add manganese dioxide from the inside of the D cell batteries. See Producing Oxygen, Part A for more information. Then, place 1 syringe with manganese dioxide and 1 syringe with yeast into a boiling water bath until they boil. In this activity, we want to have 4 variants: 1 syringe with yeast, 1 syringe with MnO2, 1 syringe with boiled yeast and 1 syringe with boiled MnO2. Add 5 mL of hydrogen peroxide to one and cap it. Time how long it takes for the balloon to fill with oxygen until it gets to its maximum size. Repeat for this procedure with the remaining three balloons. }
\item{Theory: Catalysts speed up the rate of reactions. Catalysts can be split into two categories: biological catalysts and inorganic catalysts. Biological catalysts are also known as enzymes. In the decomposition of hydrogen peroxide to oxygen, both manganese (IV) oxide from batteries and catalase from yeast catalyze this reaction. In this activity, it is possible to identify the difference between the two different types of catalysts. The two syringes that are not boiled act just like normal; they both quickly form oxygen in the balloons. However, the two syringes that are boiled do not act the same. Boiling the yeast destroys its catalase. This slows down the production of oxygen since the catalase does not catalyze the reaction. The balloon will not fill or slowly fill at best, and most of that filling will be from thermal decomposition of the hydrogen peroxide caused by the hot syringe. In contrast, manganese dioxide is not degraded. If anything, the higher temperature will speed up the reaction. This balloon will fill normally if not faster than normal. This activity shows one of the differences between biological and inorganic catalysts. Biological catalysts are susceptible to variations in the environment that kill or destroy biological activity.}
\end{itemize}

\section{Soda Science}

\subsection{Pressure Dependent Solubility of Sodas}
\begin{itemize}
\item{Preparation time: 0 minutes}
\item{Materials: unopened soda bottle}
\item{Procedure: Quickly and carefully, open soda bottle. You should see some vapor escape.}
\item{Theory: When sodas are bottled, the carbon dioxide is added by pumping CO2 at a pressure of 5 atmospheres. Henry’s law tells us that as we increase the pressure, the amount of gas dissolved in the water also increases. As we release the pressure by opening the bottle, the CO2 and trapped water vapor dissolved in the liquid come out and become a gas again.}
\end{itemize}

\subsection{Dissolved Carbon Dioxide in Soda, Part A – The Opening}
\begin{itemize}
\item{Preparation time: 0 minutes}
\item{Materials: unopened soda bottle}
\item{Procedure: Open the soda bottle. Watch how long it takes for the soda to stop releases carbon dioxide (when bubbles stop forming).}
\item{Theory: To make a soda, the soda solution has carbon dioxide gas pumped in at 4 or 5 atmospheres of pressure. Henry’s Law states that when we have a solution that is under pressure from a gas, the solution will dissolve some of the gas. In this case, the soda dissolves some of the carbon dioxide from the pressure. Removing the top of a soda allows this pressure to be relieved. This means that the carbon dioxide in the solution will slowly migrate to the surface of the soda and escape to the atmosphere. However, this process is not always quick. There is some small intermolecular attraction between carbon dioxide and the water. The carbon dioxide slowly breaks this attraction between water and carbon dioxide and escapes in the air. This is why sodas will have dissolved carbonation for some time after they are opened. However, this dissolved carbonation is not in an equilibrium state: equilibrium is achieved when most if not all of the dissolved carbon dioxide escapes. The equilibrium state for sodas occurs when the soda becomes flat.}
\end{itemize}

\subsection{Dissolved Carbon Dioxide in Soda, Part B – Shaking It}
\begin{itemize}
\item{Preparation time: 0 minutes}
\item{Materials: unopened soda bottle}
\item{Procedure: Shake the soda bottle vigorously prior to opening. Be careful opening this bottle since it will make a big mess.}
\item{Theory: The process for carbon dioxide to escape from the soda is happening immediately after production. The air pocket in between the lid and soda is filled with carbon dioxide from the production. By shaking the soda, some of the carbon dioxide that previously escaped redissolves in the liquid temporarily. The carbon dioxide is not stable in the states and forms larger bubbles. On opening the lid, these bubbles rush to the surface to escape the solution. The soda will come exploding out the mouth of the soda from these redissolved bubbles. }
\end{itemize}

\subsection{Dissolved Carbon Dioxide in Soda, Part C – Salt and Soda}
\begin{itemize}
\item{Preparation time: 0 minutes}
\item{Materials: unopened soda bottle, salt}
\item{Procedure: Open the soda bottle. Let the carbon dioxide escape. Add salt to remove the rest of the carbon dioxide.}
\item{Theory: The carbon dioxide trapped in the soda is held weakly by the intermolecular attraction between water and carbon dioxide. This weak attraction slowly lets the carbon dioxide escape. Adding salt changes the dynamics of this attraction. Salt is a strong electrolyte and water has greater attraction to the salt molecules than the carbon dioxide molecules. On adding salt, the water will find itself moving towards the salt and pushing the carbon dioxide together. As the carbon dioxide combines and forms larger bubbles, these bubbles rise to the surface of the soda. It is possible to push all the carbon dioxide out of a soda through the addition of salt.}
\end{itemize}

\section{Solubility}

\subsection{Unsaturated and Saturated Solutions, Part A – More, More, More}
\begin{itemize}
\item{Preparation time: 0 minutes}
\item{Materials: sugar or salt, jam jar, water}
\item{Procedure: Fill the jam jar halfway with water. Add a few grains of salt and mix until it is all dissolved. Let the students taste the solution. It should taste faintly like salt.}
\item{Theory: There are three different types of solutions when concerned with solubility: unsaturated solutions, saturated solutions, and supersaturated solutions. An unsaturated solution is a solution that has a little solute dissolved but it is able to dissolve more. In this case, the solution is unsaturated because there are only a few grains of salt in the water; the water can easily dissolve more.}
\end{itemize}

\subsection{Unsaturated and Saturated Solutions, Part B – Just Enough}
\begin{itemize}
\item{Preparation time: 0 minutes}
\item{Materials: same as Unsaturated and Saturated Solutions, Part A – More, More, More}
\item{Procedure: Using a second jam jar filled with the same amount of water, slowly add salt until no more salt dissolves in the water. Let the students taste the water. It will strongly taste like salt.}
\item{Theory: In this part, we are making a saturated solution. This is a solution that has dissolved all the salt that is possible. No more salt will dissolve in the solution. The taste of this solution is a much more salty because more salt has dissolved in the solution.}
\end{itemize}

\subsection{Unsaturated and Saturated Solutions, Part C – Too Much}
\begin{itemize}
\item{Preparation time: 0 minutes}
\item{Materials: same as Unsaturated and Saturated Solutions, Part A – More, More, More}
\item{Procedure: Using a third jam jar filled with the same amount of water, add salt until no more salt dissolves. Then add more salt so that a small mound sits at the bottom. Let the students taste the salt solution.}
\item{Theory:  Once a solution is saturated, it will dissolve no more solute. In this case, we have made a saturated solution of salt. By adding more and more salt, we cannot make more salt dissolve. The salt just sits on the bottom of the jar. By tasting the salt solution, it will taste exactly like the saturated salt solution from Unsaturated and Saturated Solutions, Part B. This is because we taste only the salt dissolved in the solution. The extra salt does not make the water taste more salty; it stays undissolved. }
\end{itemize}

\subsection{Solubility of Salt and Sugar, Part A – Waiting Game}
\begin{itemize}
\item{Preparation time: 5 minutes}
\item{Materials: jam jars, salt, sugar, water}
\item{Procedure: Fill each jar half full with water. Add ½ spoon of salt in the first container and ½ spoon of sugar in the second container. Carefully watch how long it takes, without stirring, to dissolve the salt and sugar.}
\item{Theory: Salt and sugar both dissolve in water, but at different rates. This is due to the strength of the differing intermolecular forces. Salt is a strong electrolyte and the polar water molecules easily respond to the addition of ions from the salt. The intermolecular forces between the polar water and ions from salt allow salt to dissolve easily in the water. Sugar on the other hand is a non-electrolyte. It does not form ions in solution. However, there are some intermolecular forces between the atoms in sugar and water. It will dissolve, but it will take a much longer time to dissolve.}
\end{itemize}

\subsection{Solubility of Salt and Sugar, Part B – There is No Spoon}
\begin{itemize}
\item{Preparation time: 5 minutes}
\item{Materials: same as Solubility of Salt and Sugar, Part A, spoons}
\item{Procedure: Follow the same procedure as Solubility of Salt and Sugar, Part A – Waiting Game. Instead of watching the salt and sugar dissolve without stirring, now stir each solution at the same rate. Watch how long it takes for the salt and sugar to dissolve.}
\item{Theory: Water can only dissolve a certain amount of salt and sugar in one specific area. If the water is not stirred, the water surrounding the salt and sugar will dissolve all that it can. Once it has dissolved all the salt or sugar that is possible, no more can be dissolved. When no more salt or sugar can dissolve, the water is full of the chemicals or it can be thought as locally saturated. As time goes on, the dissolved salt and sugar will migrate to the rest of the container allowing the water around the salt or sugar to dissolve more. This process will continue, albeit slowly, until all the salt and sugar has dissolved. The purpose of the stirring is to speed up the migration of dissolved salt and sugar as well as to move the locally saturated water away from the salt and sugar allowing unsaturated water to dissolve more salt or sugar.}
\end{itemize}

\subsection{Snow Globes}
\begin{itemize}
\item{Preparation time: 1 day}
\item{Materials: borax from Making Borax, clear glass jam jar with lid, water}
\item{Procedure: Into a clean dry jam jar, place a small amount of borax. The actual amount does not matter. Fill the jam jar with water until the brim. Seal tightly. Shake and watch the borax float like snow.}
\item{Theory: Borax is a sparingly soluble salt. In water, it has similar texture to actual snow. Shaking the jar causes the liquid to start moving around. Stopping the jar does not stop the inertia of the water inside the jam jar. This inertia will make sure that borax will keep on moving even when the jar stops. To make the snow globe a little more interesting, glue a toy or a figure inside the jar. Dying the water a color will make it even more interesting. }
\end{itemize}

\section{Visualizing Chemical Reactions}

\subsection{Dancing Reactions}
\begin{itemize}
\item{Preparation time: 10 minutes}
\item{Materials: bottle caps, matches}
\item{Procedure: For a given reaction, use different soda caps as atoms for both the reactants and products. Have students practice balancing the reaction based on having a proper number of bottle caps for each atom. Use the matches for bonds. It may be helpful to use a marker to put letters on each bottle cap for ease in remembering which bottle cap is which atom and/or to code them by color.}
\item{Theory: Balancing reactions is sometimes hard for students to understand. Using bottle caps and matches, it makes the process for balancing reactions much easier. By making it a visual experience, the balancing act becomes much more accessible for students.}
\end{itemize}

\subsection{Modeling Kit}
\begin{itemize}
\item{Preparation time: 1 hour}
\item{Materials: different colored sandals, toothpicks, marker}
\item{Procedure: Cut out small circles of the different sandals. Use a marker to label each color to a specific atom. Use a toothpick to connect different atoms to form molecules. Remember to make these models geometrically correct; carbon, for example, usually makes four bonds that radiate with a tetrahedral structure.}
\item{Theory: VSEPR Theory is can be very hard to students to visualize, especially the three dimensional aspects. Therefore, using cut out sandals allows students to make the different structures of the chemicals. Further, this is helpful to learn about crystal structure. For example, making the crystal structure of graphite is easy to make with these.}
\end{itemize}

\section{Water Chemistry}

\subsection{Distilling Water}
\begin{itemize}
\item{Preparation time: 30 minutes}
\item{Materials: 1 tea kettle, tea leaves or food coloring, plastic tubing, 1 plastic water bottle, 1 container for collecting distilled water}
\item{Procedure: In the teakettle, fill it with water and place some tealeaves. Make the tea so it has a dark color. Connect the pouring spout of a teakettle with some I.V line. If the tubing is too small for the spout, use a syringe shell to connect the two. Do not cut the I.V. line short; just let it be the entire length. Place a container at the end of the tube to collect the condensed water. Boil the teakettle after some time, start collecting water. }
\item{Theory: Distillation is a purification method of miscible liquids or solutions. In this situation, there is tea leaves dissolved in the water. The coloration of the water indicates this. However, the water can be purified since the leaves will not undergo a phase transition to the gaseous state. The water vapor is pure water; the tea is left behind. As the vapor goes down the line, it cools down to below the vaporization point and becomes a liquid again. By collecting this water, it is easy to see that it is pure water and has left behind all the tea in the kettle. @	This is the basic principle behind the distillation of crude oil to make common organic solvents like petrol, airplane fuel, kerosene, lubricants, motor oils, even petroleum jelly. Crude oil is a mixture of a variety of different components that all have different vapor points. Careful heating allows the separation and collection of all the components of oil. These separate out and have different physical and chemical properties based on their composition or molecular weight. The lightest components separate first, while the heaviest components are last.\\
Distilling on a larger scale is a good way to purify water for laboratory use when normal tap water is contaminated with many salts. This is also the principles behind hard alcohols. They first begin as beer or wine or other fermented liquid and then they are distilled to collect and concentrate the alcohol. You may use this apparatus to produce clear spirits from methylated spirits. These are very useful for making POP solution for chemistry practicals. Note that this method if anything concentrates the poisonous methanol in the spirits, so even the clear spirit is toxic to drink.}
\end{itemize}

\subsection{Laboratory Distillation of Water}
\begin{itemize}
\item{Preparation time: 2 hours}
\item{Materials: 1 Large tea kettle, plastic tubing, a 20 liter bucket, water, super glue, container to collect water.}
\item{Procedure: Find a large kettle. There are many large kettles available. Many times people will carry a large kettle and sell coffee in towns. Find one of those large, 5 liter kettles. Connect thick plastic tubing to the kettle and run this tubing into the side of a large 20 liter bucket. This bucket acts as the condenser. Coil the tubing in the bottom of the bucket and let the tubing exit the bottom of the bucket. Put a container underneath the tubing to collect distilled water. Use super glue to seal the tubing connections with the bucket on the inside and outside of the container. When the glue dries, fill the kettle, fill the bucket condenser with normal water, and apply a heat source and start collecting distilled water.}
\item{Theory: In Distilling Water, the IV line is so narrow that the air surrounding the tube can cool the gaseous water inside of it. There is no reason to have a condenser. However, scaling up this project will involve a condenser. The purpose of a condenser is to transfer the heat from the gaseous vapors from inside the tube into something outside the tube. In this case, we use water surrounding the tubing to cool the water inside the tubing. This way, the only thing that comes out from the plastic tubing is condensed water and not water vapor. It is important to have the tubing coil around in the bucket and exit at the bottom so that the distilled water, once condensed, can use gravity to pull itself out of the bucket into the collection. In extended distilling processes, the water in the bucket condenser can get hot. It may be necessary to remove the hot water and add new cold water.\\
The water collected from this activity will be distilled water. This water is perfect for use in creating bench reagents, specifically the ones in the qualitative analysis practicals. Remember, this process is unnecessary for volumetric analysis.}
\end{itemize}

\subsection{Hard Water, Part A - Foaming}
\begin{itemize}
\item{Preparation time: 10 minutes}
\item{Materials: powdered soap, 2 small jam jars, tap water, pure water from Distilling Water}
\item{Procedure: Fill one jar with some tap water and another with some distilled water. Add a big pinch of powder soap. Try to make bubbles in the water by capping and shaking. The amount of foam depends on the water: hard water forms almost no foam, soft water produces a little foam, and distilled water will foam greatly.}
\item{Theory: Water has different compounds dissolved in it. Water from the ground has lots of minerals and ions that dissolved from the earth. These ions and minerals are not necessary bad for a person’s health; rather distilled water has no taste since there is a lack of those minerals and ions. There are three general classifications of water: hard, soft, and distilled. Hard water is water that has lots ions dissolved in it, especially calcium and magnesium. Soft water has a much smaller concentration of ions, but it still has ions nonetheless. Distilled water has no dissolved ions or minerals. The purpose of adding soap is to see the formation of bubbles. Bubbles are an arrangement of the long hydrocarbon chains from the soap. The chains need to organize themselves in a very particular manner and the ions prevent these chains from forming properly. Hard water has many of these ions dissolved in water and they prevent the formation of the bubbles. Soft water has fewer ions, but enough to inhibit some bubble formation. Distilled water has no ions dissolved, and the bubbles form unimpeded.\\
This affects people in regions with hard water. Since people add soap until there is a good foamy mixture to wash dishes, people with hard water have to add more soap. If there is soft water, less soap is required to make a foamy mixture. If the water available is soft water, make hard water by dissolving Epsom salt in the water – this adds magnesium ions, which artificially makes the water hard.}
\end{itemize}

\subsection{Hard Water, Part B - Removal}
\begin{itemize}
\item{Preparation time: 10 minutes}
\item{Materials: 1 object with hard water remains to be removed, citric, water, bucket}
\item{Procedure: Place the object with lime scale or hard water remains in a bucket. Add a one to one ratio of vinegar and water. Add enough to cover the object. Let the object sit overnight. The hard water remains will dissolve. Calcium and magnesium carbonates dissolve easily in vinegar. If there remains some precipitate, soak the object in a more concentrated citric acid solution over night. It may require soaking in diluted battery acid to remove all the remains.}
\item{Theory: Carbonates are sparingly soluble salts. This means that they tend to form solids or precipitates. However, a little does dissolve. When they precipitate out, it is difficult to dissolve them back into water. Their solubility does depend on pH however. Adding an acid, vinegar, citric, or battery acid, decreases the pH that increases the solubility of lime scale or hard water remains. In other words, the acid helps to dissolve the hard water deposit.}
\end{itemize}

\subsection{Polar Water, Part A – Comb Attraction}
\begin{itemize}
\item{Preparation time: 5 minutes}
\item{Materials: water, plastic water bottle with cap, comb, hair}
\item{Procedure: Put a small hole in the cap of a water bottle. Applying firm, constant pressure with a syringe needle on the lid is the easiest method; however take care to be safe. After the hole is in the cap, run the comb through someone’s hair. Turn the plastic water bottle over and squeeze gently to force a thin but continuous water stream at a 45 degree angle. Bring the comb towards the water stream and the water will bend.}
\item{Theory: Water is a polar molecule. It is made up of 2 hydrogens and 1 oxygen. The hydrogens form a positive end and the oxygen forms the negative end. When a comb that has been statically charged comes near water, there are some electrostatic forces. The static forces on the comb pull on the water. The effect of this force is visible by observing the bending of the water stream when the comb comes near, and the water returning to its normal stream when the comb moves away. The bending occurs only because water is a polar molecule and is attracted to the static charge on the comb.}
\end{itemize}

\subsection{Polar Water, Part B – The Dribble}
\begin{itemize}
\item{Preparation time: 10 minutes}
\item{Materials: water, glass container, plastic container}
\item{Procedure: Fill both containers with water. Carefully, and barely pour from one container to the next container. Look to see if the water dribbles down the side of the container.}
\item{Theory: Glass and plastic are made of two different materials and have themselves two different chemical properties. Glass is a polar compound while plastic is a non-polar compound. Between compounds, intermolecular forces play an important role in how compounds hold together. Intermolecular forces between polar and polar compounds are strong, and they will hold onto each other. On the other hand, intermolecular forces between polar and non-polar compounds are very weak, and they will not hold onto each other. Since water is a polar molecule, it will adhere to the glass. This is the reason that water tends to dribble down the sides of glass containers making a mess when pouring. However, since water is polar and plastic non-polar, there is little to no attraction between water and plastic. This means that there will be no water sticking to the plastic dribbling down the side of the plastic. This is the same reason kerosene dribbles down the side of plastic containers but pours freely from glass.}
\end{itemize}

\subsection{Expansion of Ice, Part A – Breaking the Bottle}
\begin{itemize}
\item{Preparation time: 5 minutes}
\item{Materials: 1 small water bottle, water, freezer}
\item{Procedure: Fill a water bottle full of water. Cap the bottle. Place this bottle in the freezer. Check on the bottle the next day.}
\item{Theory: Water is an interesting chemical since its density actually increases as it changes from a liquid phase to the solid phase. Normally, that is not the case with other compounds. In this activity, the water freezes into ice and increases in volume. Since the bottle is capped, there is no extra volume for the water to expand and will actually break the plastic water bottle. }
\end{itemize}

\subsection{Expansion of Ice, Part B - Floatation}
\begin{itemize}
\item{Preparation time: 0 minutes}
\item{Materials: ice, water, jam jar}
\item{Procedure: Put some water in the jam jar and add a piece of ice. Observe if the ice floats or sinks.}
\item{Theory: The density of water changes when it freezes to make ice. Density is defined as mass divided by volume. In ice, the volume increases as the water freezes but the mass does not change. This makes the density of ice smaller than normal water. Since the density is less when water freezes, adding ice to water means that the ice will float on the surface of the water. This indicates that the density of ice is less than that of normal water.}
\end{itemize}

\subsection{Pressure Melting Ice}
\begin{itemize}
\item{Preparation time: 10 minutes}
\item{Materials: Ice, thin plastic string like fishing line or dental floss, two weights like rocks or cans}
\item{Procedure: Take a meter long piece of thin plastic line and tie the two cans or rocks on the end. Drape the string over the ice and let the weights pull down on the ice. The line will move through the ice without the ice obviously melting.}
\item{Theory: Another interesting property of water is that ice melts on the application of pressure. In this activity, the plastic line applies pressure to the ice from the weights. This in turn, applies specific pressure on the ice, melting it only directly under the line. The string moves through the newly melted water, and then the water refreezes above the string. This is a fun activity because it appears that the line is moving through the ice magically, but really, the ice is melting under pressure, and then refreezing when the pressure is alleviated by the string moving on down through the ice.}
\end{itemize}

\chapter{Physics Activities}

\section{Astronomy}

\subsection{Solar System Mobile}
\begin{itemize}
\item{Preparation time: a few hours}
\item{Materials: flour, water, balloons, mixing bowl, newspaper or old papers, string, sticks}
\item{Construction: Blow up the balloons, one for each of the 8 planets and sun. Make the paper mache mixture with flour and water; you want a watery-glue texture. Wet the paper in this mixture and apply artistically to the balloons until you have a layer a couple papers-thick on each balloon. Leave the balloon slightly exposed at the bottom. When the papers are dried, pop the balloons within and set to work making them look like planets. Use paint, markers, or colored pencils. Attach string and hang them as a mobile. If you want to get fancy, you can place the string between layers of paper before it dries, thus saving yourself some tape or glue.}
\item{Theory: This activity is helpful to explain to students what is actually happening off the world. This mobile is helpful to remember that there is more to the solar system than just earth.}
\end{itemize}

\subsection{Star Gazing}
\begin{itemize}
\item{Preparation time: 0 minutes}
\item{Materials: none}
\item{Procedure: Take the students out at night. Look for constellations, starts, planets, and even satellites. In addition to planets, look for Orion’s Belt, the Southern Cross, and more constellations. Due to the large amount of information regarding this topic, we cannot include this information. However, a quick internet search will give you what you need.}
\item{Theory: For the longest time, stargazing was some of the most important aspects of navigation and even religion. Recreate this experience by finding stars and constellations. Tell the stories behind them, and encourage students to find their own constellations and give their own stories.}
\end{itemize}

\section{Archimedes’ Principle}
subsection{Water Weight and Upthrust}
\begin{itemize}
\item{Preparation Time: 1 minute}
\item{Materials: spring balance, syringe with the bottom melted shut and no plunger, eureka can (can be made cheaply by a metal craftsman), water, heavy object, thread, small dish}
\item{Procedure: Fill the eureka can up to its spout with water and place the spout over the dish. This can is designed so that when the water is being displaced, it is collected to another container for later measurements. Hang the object by the thread from the spring balance and measure its weight. Now immerse the water completely in the eureka can and measure its Apparent Weight (weight in water).\\
When you immersed the object in water, some water will have overflowed from the can into the small dish. Pour this water into your syringe shell and measure the weight of water. Record this result with the earlier Weight and Apparent Weight.}
\item{Theory: Archimedes’ Principle states that the upthrust of a liquid on an object is equal to the weight of water displaced by the object. The upthrust is equal to the Weight of the object minus its Apparent Weight in the water:@	Upthrust = Weight in air – Apparent Weight in liquid\\
But upthrust is also equal to the water displaced:@	Upthrust = Weight of liquid displaced\\
By calculating the upthrust, you should see that the result is equal to the weight of water in the syringe.}
\end{itemize}

\section{Conservation of Energy}
subsection{Pencil Launcher}
\begin{itemize}
\item{Preparation Time: 5 minutes}
\item{Materials: Clothes clip, thread, two pencils}
\item{Procedure: Open the clip and tie the closed end with thread so that the clip stays open against the tension of the spring. Place the clip flat on a table and place two pencils next to the clip, one on either side, so that the eraser touches the tied end and the tips point out in opposite directions along the table. Cut the thread holding the clip open and stay clear of the flying pencils.}
\item{Theory: The spring inside the clip holds energy when it is forced to contract. When the clip is allowed to close, the potential energy of the spring is transformed into mechanical energy as the clip moves, forcing the pencils away at a decent speed.}
\end{itemize}

\section{Current Electricity}

\subsection{Conductor Switch Test}
\begin{itemize}
\item{Preparation Time: 5 minutes}
\item{Materials: Two or three batteries, wires, bulb, switch, various materials to test (spoon, rubber strip, stick, mchelewaji, etc.)}
\item{Procedure: Set up the circuit so that the wires, switch, and bulb are connected to the batteries in series. Close the switch to show that the bulb lights when current is passing. Open the switch. Place each material in turn across the switch, closing the circuit. Any materials that successfully close the circuit and the bulb lights can be considered conductors, and any materials that fail to close the circuit and do not light the bulb are insulators or poor conductors.}
\item{Theory: Conductors will freely allow electric current to pass, so when a conductor is used to close the switch, the circuit is complete and current will pass, lighting the bulb. Insulators, however, do not permit current to flow, so the circuit will still be broken despite the switch being ‘closed’ with the insulator.}
\end{itemize}

\subsection{Light Bulb in a Jar}
\begin{itemize}
\item{Preparation Time: half hour}
\item{Materials: Glass jar with lid, glue, wires, power source, small length of thin iron wire}
\item{Procedure: Poke two holes in the jar lid and pass a wire through each about half way into the jar. Connect the wire ends inside the jar with the length of iron wire and seal the wires into the lid with glue. Close the lid and connect the wires to the power source. If enough current is passing, the iron wire will light up, creating a ‘light bulb’ for a short time until the wire burns out.}
\end{itemize}

\subsection{Foil Fuse}
\begin{itemize}
\item{Preparation Time: 15 minutes}
\item{Materials: Power source, wires, two small nails, small piece of wood, metal foil (from Blueband container, gum, etc.)}
\item{Procedure: Hammer the nails into the wood about 5 cm apart to act as wire terminals. Connect wires to each of the nails and place a thin strip of foil between the nails, bending it around the nails to secure it. Connect the wires to the power source. If the source is powerful enough, it will cause the foil to heat and eventually burn, breaking the circuit.}
\item{Theory: Foil, having a very small cross-sectional area compared to that of a wire, has a low tolerance for current. If too much current passes through the foil, it will burn away. This is essentially how a fuse works in a radio or other electrical device. To be more scientific in your experiment, use a rheostat in the circuit, gradually lowering the rheostats resistance until the fuse blows.}
\end{itemize}

\section{Density}
See the Density activities in the Chemistry section for more.

\subsection{U-Tube apparatus}
\begin{itemize}
\item{Preparation Time: 1 hour}
\item{Materials: 3 clear plastic pen tubes, cardboard, hot poker or knife, tape, pen, super glue, water, any fluid, which will not readily mix with water.}
\item{Construction: Cut two of the pens at one end at a 45-degree angle, and cut the third pen (shorter than the other two) at both ends at a 45-degree angle. With the shorter pen on the bottom, attach the other two as styles so that the 45-degree angles meet to form right angles. Together the 3 pens should form a U-shaped tube with open ends at the top of each style (vertical tube). Melt the angled ends together with a hot knife, soldering iron, etc. so that the whole apparatus is watertight except for the tops. Glue the apparatus to a cardboard base so that it can stand up straight. Put thin strips of tape up each side of the U-tube and mark each strip with evenly spaced marks. The two scales should be identical. One good way to do this is to put steadily increasing volumes of water (3 ml, 4 ml, 5 ml, etc.) and mark the levels on each scale for each volume. Label these marks from top to bottom as 0, 1, 2, etc.}
\item{Procedure: Place an amount of water into the U-tube such that the water rises about half way on either side of the tube. The actual volume of water is not important as long as you can see the levels clearly. Stand the tube upright and slowly drip about 1 ml of another fluid, kerosene in this case, into one side of the U-tube (if the fluid has a higher density than water, it should go in first, and then the water). The kerosene will displace the water, so you should see the water level on the other side rise slightly.\\
Measure the relative heights of water and the kerosene from the bottom level of the kerosene. The heights are related to the densities by:
\[ \frac{\mathrm{Height of water}}{\mathrm{height of kerosene}} = \frac{\mathrm{density of kerosene}}{\mathrm{density of water}} \]
} % Procedure
\item{Theory: If a fluid’s density is less than that of water, it will float on top (if it is added slowly) of the water, displacing the water on the other side of the tube. From Archimedes’ principle and the Law of Flotation, we know that the relative density of the fluid is equal to the inverse ratio of the heights of the liquid. The scales drawn on the outside of the U-tube allow you to find the ratio of the heights without needing units, and the density of water is known to be 1.0 g/ml, so you can easily calculate the density of the other fluid.\\
If the other fluid has a higher density than water, the experiment can still be done, but you need to add the fluid with higher density first, then displace it with water, performing the same calculation.\\
This apparatus was designed and brought forward by two form 4 students without any prompting. They then proceeded to find the density of kerosene accurate to two decimal places. Never underestimate the curiosity and ability of students, or the power of broken pens.}
\end{itemize}

\section{Electromagnetism}

\subsection{Simple Motor}

\begin{itemize}
\item{Preparation Time: 2 hours}
\item{Materials: Flat piece of wood 6”x12”, four large nails, two small nails, two screws, two pieces of thick wire 4” long, rubber stopper about 1.5” diameter and at least 1” thick, 20ml syringe, two small pieces of sheet copper, lots and lots of speaker wire, hammer, knife, glue, two batteries}
\item{Procedure:

\begin{enumerate}
\item{Arrange the piece of wood and nails/screws as follows:

\begin{enumerate}
\item{Through the center of the board, drive a large nail so that it goes all the way through; turn the board over so that the nail sticks up.}
\item{On either side of that nail, the long way across the board, drive two other large nails into the board just enough so that they stay. The distance between these two nails should be the length of the last large nail plus about 1 cm.}
\item{At a 45-degree angle, about 1” from the center nail and directly across from each other, drive the two small nails into the board just enough so that they stay.}
\item{Along one long side of the board, in each corner, screw the small screws into the board, leaving a few cm between the head and the board.}
\item{All the nails and screws are driven into the top of the board except A, which is driven up through the bottom all the way.}
\end{enumerate}
} % Arrange the piece of wood...

\item{Electromagnets: Connect one end of a wire to the batteries. From there, wind it once around one of the screws (D) and tighten the screw to hold the wire in place. From the screw, extend the wire to the top of the nearest large nail (B) and begin winding it around the nail from the top to the bottom. More turns will produce a stronger magnet. After reaching the bottom of the magnet, extend the wire to the nearest small nail (C) and wind the wire from the bottom of the small nail to the top. At the top, solder the wire to one of the thick copper wires, called brushes. Do this again from the other terminal of the battery, to the other screw, wound down around the other large nail (B) in the same direction as the other nail, wound around the small nail to the second brush.}

\item{Rotor: Remove the plunger from the syringe and cut the finger tabs off. Along the top of the syringe tube, glue the two pieces of sheet copper as shown (upside down) in figure (ii). The sheets should be about 1” wide and should each wrap around half of the syringe’s circumference, leaving about 6 mm between the strips where they meet on each side. These copper plates, together with the brushes from (2) above, form the commutator. Hollow out a small space for the syringe to fit snugly into in the bottom of the stopper. Insert the bottom of the syringe (where the needle attaches) into this space. You will glue this later, but leave it for now. Through the top of the stopper, insert the last large nail horizontally so that exactly half of the nail sticks out on either side. The nail and upside-down syringe should form a T, with the stopper at the intersection (ii). This is the armature of the rotor.}

\item{Glue or solder the end of a wire to one of the copper sheets. From there, wind the wire around one half of the armature nail starting at the stopper and winding outwards. Once the wire reaches the end of the nail, extend it over to the other end of the nail on the other side of the stopper and start winding again towards the stopper, circling the opposite way around the nail (circling in the same direction will cause opposing magnetic fields – use the RHR if you get stuck on this). Once the wire reaches the stopper, extend it down to the other copper plate and glue or solder it there. You can cut the wire at this point. Put the syringe over the center large nail (A) so that it can rotate freely. The nail along the top should be able to turn, passing close to the two large nails (B). Turn the syringe in the stopper until the brushes touch the gaps between the copper plates when the armature magnet is aligned with the two upright electromagnets. The commutator and armature are now complete. Now you can glue the brushes to their respective nails so that they brush lightly against the gaps between the copper plates when the magnet is aligned.\\
*Note: more windings will produce a stronger magnet so you can wind back and forth, but keep the same direction of the wire. Also, a syringe with a larger diameter will produce a more accurate commutator.}
\item{The motor is now complete: allow current to run and give the armature a push to start it spinning. It may take a few tries to get it going, but if your commutator is well placed and your coils are wound in the right direction, the motor should keep going.}
\end{enumerate}
} % Procedure

\item{Theory: The two coils on each side (B) are large electromagnets, which keep a constant polarity, as the current never changes. These coils are also connected to the thick wires, which brush against the commutator of the armature (rotating bit), so the thick wires act as the opposite battery terminals.\\
The armature is also a large electromagnet, but as the commutator is repeatedly switching poles as it rotates and brushes against the thick wires, the direction of current, and therefore the poles of the magnet, is always switching with each half rotation.\\
In one position with the armature magnet facing the two magnets on the side, a strong magnetic force holds it in place. As the rotor rotates (being pushed), the commutator plates switch brushes and the current through the armature reverses, thereby reversing the poles of the magnet. Now the armature magnet is attracted to the opposite upright magnet. As it passes the magnets again, 180-degrees, the current switches again and the cycle continues. As the rotor gains speed, it will become more stable.}
\end{itemize}

\subsection{Creating a Current in a Wire}
\begin{itemize}
\item{Preparation Time: 5 minutes}
\item{Materials: Wire about 50 cm, ammeter or sensitive bulb, strong magnet}
\item{Procedure: Coil the wire to create a solenoid, connecting the free ends to the ammeter or bulb. Use a bar magnet or one pole of a horseshoe magnet and pass it through the solenoid (if you are using a speaker magnet, you will need to adjust the coil to accommodate the odd shape). As the magnet passes through the coil, the ammeter or bulb will show a current. When the magnet stops or leaves the coil, the current will cease.}
\item{Variation: If you have a very strong bar magnet, wrap the wire around a syringe multiple times and connect the ends to an ammeter. Place a small wad of cloth in the bottom of the syringe and insert the magnet. Cover the opening with your thumb and shake. The wad and your thumb will protect the magnet as it bounces back and forth, creating an alternating current in the coil.}
\item{Theory: A magnetic field that moves perpendicular to a conductor will induce a current in that conductor. When the conductor is a coil and a bar magnet is passed through it, a significant current is induced and should be enough to light a sensitive bulb or deflect the needle in an ammeter. The current will be stronger if the number of coils is increased or if a stronger magnet is used.}
\end{itemize}

\subsection{Mapping Induced Magnetic Field from a Coil}
\begin{itemize}
\item{Preparation Time: 15 minutes}
\item{Materials: power source, length of wire about 50 cm, bulb or switch, cardboard, scissors, iron wool}
\item{Procedure: Cut the cardboard so that a single tab about 10 cm long and 3 cm wide sticks out from the larger piece. Notch this tab every 1 cm on either side and coil the wire around the tab, keeping it in place in the notches. Connect the wire to the switch/bulb and power source so that there is a strong current in the wire. Use the iron wool to sprinkle iron filings onto the tab inside the wire coil. The filings will create a single solid line the length of the coil, spreading out at each end.@Theory: A coil of wire creates a single, strong magnetic field inside it in one direction. At the ‘poles’ (for it is indeed an electromagnet) the field spreads out again. You can use the 2nd Right Hand Rule to find the direction of the field. The filings will align themselves with the strong field inside.}
\end{itemize}

\subsection{Mapping Induced Magnetic Field from Wire}
\begin{itemize}
\item{Preparation Time: 10 minutes}
\item{Materials: power source, length of straight wire, switch or bulb, paper or cardstock, iron wool}
\item{Procedure: Cut a hole in the paper or cardstock so that the wire passes vertically through the middle of the paper so it lies flat. Connect the wire, hanging vertically, to the switch/bulb and power source so that there is a strong current in the wire. Using your thumb and forefinger, rub the iron wool to create iron filings, distributing them widely onto the paper. The filings should form concentric circles around the wire.}
\item{Theory: Current in a straight wire produces a magnetic field around the wire (use the Right Hand Rule to find the direction) in concentric circles. At the surface of the paper, the magnetic field is a series of circles and the filings will align themselves with the field.}
\end{itemize}

\subsection{Spinning Compass}
\begin{itemize}
\item{Preparation Time: 1 minute}
\item{Materials: batteries or other power supply (the stronger the better), wire, switch or bulb, compass (the pin in the compass demo will do)}
\item{Procedure: Connect the wire and switch/bulb to the battery or power source so that there is a strong current running through the wire. Run the wire over the compass in a straight line. If the current is DC, the compass will turn to face a new direction. If the current is AC, the compass will spin or waver back and forth quickly.}
\item{Theory: Current in a straight wire creates a magnetic field around the wire in concentric circles. The direction of the magnetic field can be found using the first right-hand-rule. DC current produces a steady magnetic field in one direction (circular), so the magnet of the compass will align itself with the field. AC current produces a constantly shifting magnetic field, so the compass will spin, trying to align itself as the field changes direction.}
\end{itemize}

\section{Fluid Mechanics}

\subsection{Water Pour}
\begin{itemize}
\item{Preparation time: none}
\item{Materials: pitcher, bucket, water}
\item{Procedure: Fill the pitcher with water. Place the bucket on the floor. Stand on a desk or table to increase your height, and pour water from the pitcher into the bucket. Try to pour the water at a constant rate. Point out that the stream of water is thicker at the top, and becomes thinner as it falls. If the rate of flow is small, or if you are pouring from a very high height, the stream will also break up into drops near the end.}
\item{Theory: As the stream of water falls, gravity causes it to accelerate to higher speeds. The continuity principle tells us that when the water is moving faster, it must have a smaller cross-sectional area. Thus, the stream of water is thicker at the top, and thinner at the bottom. If the stream of water becomes thin enough, surface tension will pull it into individual drops, because this will minimize the surface area, and therefore the surface-energy.}
\end{itemize}

\subsection{Cone Blow}
\begin{itemize}
\item{Preparation time: 5 minutes}
\item{Materials: A funnel or top of a 1.5 liter water bottle, one piece of paper, adhesive tape, scissors}
\item{Construction: If you do not have a funnel, cut off the top of an empty Kilimanjaro water bottle. Cut out a circular piece of paper, and cut along one radius. Bend the paper into a cone. Tape the cone so that it fits neatly into the funnel.}
\item{Procedure: Place the cone in the funnel. Ask for a student volunteer. Tell the volunteer to push the cone out of the funnel by blowing upwards into it. After the student fails at this, ask for several other volunteers.}
\item{Theory: When the student begins to blow through the funnel, air passes through the narrow space between the cone and the funnel with some velocity. According to Bernoulli’s Principle, this moving air has a lower pressure than the stationary air inside of the cone. Because the air outside of the cone has a lower pressure than the air inside of the cone, the cone is actually pulled back into the funnel. Thus, no matter how hard the student blows, the cone will remain in the funnel.}
\end{itemize}

\subsection{Paper Blow}
\begin{itemize}
\item{Preparation time: none}
\item{Materials: 2 sheets of paper, not wrinkled}
\item{Procedure: Hold two sheets of paper roughly parallel, several centimeters from each other. Blow into the space between the two sheets of paper. They will be pulled towards each other.}
\item{Theory: According to Bernoulli’s Principle, the moving air between the two sheets of paper has a lower pressure than the stationary air outside of them. Thus, atmospheric pressure will push the two sheets of paper towards each other.}
\end{itemize}

\section{Friction}

\subsection{Spring pull}
\begin{itemize}
\item{Preparation Time: 1 minute}
\item{Materials: various surfaces, pens, oil, wheel bed, spring or spring balance, block of wood, eyehook}
\item{Procedure: Screw the eyehook into the block of wood so that the spring can easily be attached to it. Drag the block along a rough surface with the spring and measure the spring’s extension. Now place a row of pens, side-by-side on the surface and drag the block over the pens, again measuring the spring’s extension. Repeat this experiment using the wheel bed (can be done easily with water bottle caps and nails), using an oiled surface, and various other surfaces. Also turn the block so that alternately a large side and a thin side are in contact with the surface, measuring the relative spring extensions.}
\item{Theory: Friction depends on the nature of the surface in contact and not on the surface areas in contact, so there should be no noticeable difference between the two sides of the block. However, the rollers, in this case pens, wheels, and lubricant, in this case oil, all reduces friction, so the spring’s extension should decrease when these methods are used.}
\end{itemize}

\section{Kinetic Theory of Gases}
See the Gas Laws in the Chemistry section for activities

\section{Light}

\subsection{Pinhole Camera}
\begin{itemize}
\item{Preparation Time: half hour}
\item{Materials: Cardboard box, black paint if necessary, translucent screen (tissue paper, color gel, etc.), pin, tape, scissors, light source, any object}
\item{Procedure: Cut out one side of the cardboard box and paint the inside black. Replace the cutout side of the box with your translucent screen, taping it shut along all four edges. On the opposite side of the box from the screen, poke a small hole with the pin. Your camera is now complete.\\
In a dark room, shine a bright light source on an object and aim the camera at it so that the light from the object passes through the pinhole to the screen. If the source is bright enough, the image should appear, upside down, on the screen. Play around with the object distance until you have a large, clear image on the screen. It is recommended to try this outside on a bright day, but you will need to cover the space between the camera and your head completely so that no light can enter.}
\item{Theory: Light travels in a straight line, and so light from the top of the object will pass at an angle through the pinhole, appearing at the bottom of the screen on the other side. Alternately, light from the bottom of the object will appear at the top of the screen. A strong light source is needed because the aperture pinhole is small and will only admit a small amount of light.}
\end{itemize}

\subsection{Light through a Comb}
\begin{itemize}
\item{Preparation Time: 1 minute}
\item{Materials: comb, light source, optional mirror}
\item{Procedure: In a dark place, shine the light parallel to a table surface through the comb. The apertures in the comb will act as ‘beams’ of light. Reflect the beams off a mirror and observe the straight-line propagation of light.}
\item{Theory: Light travels in a straight line, even when reflected at a surface.}
\end{itemize}

\subsection{Kaleidoscope}
\begin{itemize}
\item{Preparation Time: 5 minutes}
\item{Materials: 3 or more mirrors of equal size OR 3 or more pieces of glass of equal size with metal foil on one side, tape; Optional: colored objects}
\item{Procedure: Tape the three mirrors together so that they form a triangular tube with the reflective sides facing in. Look through the kaleidoscope at any objects, especially colored beads or paper, and turn the scope to watch the pretty colors change!}
\end{itemize}

\subsection{Color wheel}
\begin{itemize}
\item{Preparation time: Half hour}
\item{Materials: white paper, colored pencils, Nido can lid, 1” screw, tape or glue@Construction: Cut the paper into a circle with the same diameter as the Nido cap. Using a pencil and straight-edge, divide the circle into seven pie slices and color each slice a single color from ROYGBIV (Red, orange, yellow, green, blue, indigo, violet). Drawing 14 or 21 slices is more effective, but seven works well.@Using a pencil or something else sharp, balance the Nido lid until you find its center of gravity. Mark it and carefully screw the screw through at that point, creating a kind of top. Tape or glue the colored paper to the top, colored side up, screw point down. Now you have a top with the seven rainbow colors on top.}
\item{Procedure: Review the colors with your students using the colors on the wheel/top/thing. Place the wheel on a desk and give it a good spin. If it is well-balanced, it should spin smoothly and all the colors run together to form white. Bask }
\item{Variation: (1) Wrap string around the screw so that both ends of the string stick out. Pull them quickly in opposite directions to get the top spinning quickly. (2) Poke two holes about 1 cm on either side of the center of the wheel (no screw). Loop a string through the two holes so that two lengths of string of equal length stick out on each side. Tie the loose ends on the one side and then hook your thumbs into each loop. Twist the strings by spinning the wheel. If you pull, the wheel will spin and re-twist itself in the other direction. This is a child’s toy in many villages, so get one of your students to help you.}
\item{Theory: For light, white is the presence of all colors (this is opposite for pigment). Theoretically, you could do this demo using only the three primary colors, but it might be harder to get the top to spin fast enough to lose all resolution of the colors. By using the seven rainbow colors, we make it easier for them to blend together; creating white as far as the eye is concerned.}
\end{itemize}

\subsection{Thin Film Interference}
\begin{itemize}
\item{Preparation time: 5 minutes}
\item{Materials: A small bowl or other dish, water, oil}
\item{Procedure: Pour water into the dish. Touch your finger to the oil, then to the surface of the water. A small amount of oil should be transferred to the surface of the water, where it will form a thin film. A colorful rainbow pattern should be visible in the thin film. Try looking at it from different angles. If you are having trouble seeing the colors try moving the dish into brighter light (direct sunlight works well), or using a dark-colored dish.}
\item{Theory: When light strikes the surface of the water, some is reflected off the top of the film of oil, and some is reflected from the oil-water interface. When the difference in the path length between these two paths is an integer number of wavelengths, light of that wavelength will be strongly seen. This gives rise to the rainbow pattern.}
\end{itemize}

\subsection{Water Prism}
\begin{itemize}
\item{Preparation Time: 1 minute}
\item{Materials: mirror, clear rectangular container, water, white light source}
\item{Procedure: Fill the container with water. On the inside of one of the sides, place the mirror with the reflective side facing in. In a dark place, shine a light through the opposite side of the container at an angle so that the light passes through the water, reflects off the mirror, and exits the container on the original side. The light leaving the container should be dispersed into the color spectrum.}
\item{Theory: Light refracts when entering a dense medium like water. As white light is refracted, each color that makes up the white light (the whole color spectrum) refracts at a different angle depending on its wavelength, and so a refracted ray creates a slight rainbow pattern. Normally this effect can be seen only partially as white light passes through water, but as you are refracting the water twice, once into the water and once back into air, the dispersion effect will have twice the magnitude.}
\end{itemize}

\subsection{Refraction of light through water}
\begin{itemize}
\item{Preparation Time: 5 minutes}
\item{Materials: cardstock or cardboard, jar, water, Nido or powdered soap, light source}
\item{Procedure: Cut a small hole, about half a cm, in the cardstock. Put some Nido or soap into the water in the jar so that it becomes cloudy. Shine the light through the hole in the card so that a thin beam can be seen in the cloudy water on the other side. Change the direction of the beam through the water to see the different refracted angles.}
\item{Theory: The Nido or soap provides particles in the water that will reflect light, clearly showing the path of the light through the water (picture headlights on a foggy day). Light slows down as it enters a denser medium, like water, from a less dense medium, like air. As such, the direction of the light changes in order to reduce the traveling time through the medium. This effect, called refraction, can be seen in the cloudy water.}
\end{itemize}

\subsection{Rectangular Prism}
\begin{itemize}
\item{Preparation Time: 1 minute}
\item{Materials: rectangular prism (available in a lab store for about 6,000/=), paper, cardboard, sewing pins or surgical needles, pencil, protractor}
\item{Procedure: Place the paper on the cardboard and secure it with staples or tape on the edges. Place the prism flat on the middle of the paper and trace its outline with the pencil. Draw an incident ray and its respective normal on the paper and then place two pins – one next to the prism and the other farther away – vertically on the line representing the incident ray.\\
When you look through the prism from the other side, you should see the two pins clearly. Line them up so that they look like one pin; now place two more pins on this side so that they also line up with the pins of the incident ray. Now you should have four pins creating two parallel lines, as shown here:\\
Trace the line connecting the two new pins all the way to the prism. This is the ray as it leaves the prism. Draw its respective normal. Now, inside the prism, you can connect the two rays with a line through the glass prism. Do this and measure the resulting angle of refraction. When this is done, you can calculate the index of refraction of glass.}
\item{Theory: An single ray of light incident on a surface between two media will be subject to Snell’s law:  where n1 is the index of refraction of the first medium (air, in this case), n2 is the index of refraction of the second medium (glass), i is the angle of incidence, and r is the angle of refraction. Since we can measure the two angles easily with a protractor on the paper, and we know the index of refraction in air to be 1.0, we can calculate the index of glass, around 1.52.\\
Since we do not have a point source of light, we use the pins to represent a single ray that would start at one and pass through the other. In a way, the four pins are one ray of light, at least for our purposes.}
\end{itemize}

\subsection{Pouring light}
\begin{itemize}
\item{Preparation Time: 5 minutes}
\item{Materials: Opaque container (Nido can), nail, flashlight, regular container, water}
\item{Procedure: Use the nail to poke a hole at the bottom of one side of the opaque container. Fill the container with water and allow the water to pour out into the other container. In a dark place, shine the flashlight down through the top of the opaque container: you will see the water glow as it is poured out.}
\item{Theory: The light is reflected at the surface of water (total internal reflection), so when it travels through the stream of pouring water, it continues to be reflected inside the stream until it reaches the container below. The light that does escape the pouring water is what we see as the glowing effect.}
\end{itemize}

\section{Lenses}
subsection{Focusing an Image through a Convex Lens}
\begin{itemize}
\item{Preparation Time: 1 minute}
\item{Materials: convex lens (magnifying glass on a Swiss army knife works well), white paper or screen, tissue paper (the paper used to wrap the Rexa toilet paper is perfect), pen, point light source (your headlamp, desk lamp, etc.), optional retort stands}
\item{Procedure: cut a piece of tissue paper to fit over your light source. Flatten this paper in a book overnight if necessary. Draw a thick arrow on this tissue paper and tape it over your light source. With students, set up the light source to shine directly on a white screen or paper about half a meter away. The distance depends on how strong the light is). Move the magnifying glass/convex lens back and forth between the light and screen until the image of the arrow is focused on the screen. Measure the distances from the lens to the screen and lens to the light source. Now you can calculate the focal length of the lens.}
\item{Variation: Fry some bugs with sunlight! If the sun is strong enough, you should be able to get paper to smoke, and no one likes siafu anyway.}
\item{Theory: The lens equation is given as 1/f = 1/u + 1/v where f is the focal length of the lens, u is the distance from the object to the lens, and v is the distance from the focused image to the lens. In our case, the object is out light source and arrow, and the image is on the white screen. By focusing the image, we set u and v, allowing us to calculate f.}
\end{itemize}

\section{Measurement}

\subsection{Measurement Errors}
\begin{itemize}
\item{Preparation time: 1 minute}
\item{Materials: Meter sticks or stopwatches}
\item{Procedure: Ask for several students to volunteer to help. If using stopwatches, tell them to measure the time between two claps that you will give. Clap once, at which time the students should start their stopwatches. After a period of several seconds, clap a second time, at which point the students should stop their stopwatches. Make a simple table of their results, including several intervals. Each student should have a slightly different measurement. As they were all measuring the same event, this shows that their measurements contain errors.\\
Alternately, place a chalk mark on the wall at a height of more than 1 meter above the floor. Give several students a meter stick, and ask them to measure the height of the mark. Again, their answers should all be slightly different, because measurements always contain an error.\\
For this demonstration, it may sometimes happen that the first student will report a certain value, and then all of the following students will agree with the first value, regardless of what they have measured. The students should not all get the same value. To prevent this false agreement, it may be necessary to have each student first write their measurement on a small piece of paper, and then hand all of the papers to you.}
\end{itemize}

\subsection{Beam balance}
\begin{itemize}
\item{Preparation Time: half hour}
\item{Materials: Coat hanger, retort stand or other support, cardboard, pen, two water bottle bottoms, string, tape}
\item{Procedure: Hang the coat hanger from the retort stand or support so that it is free to swing. Cut out a strip of cardboard with a single centerline and tape it to the stand upright. Cut out a cardboard pointer and tape it to the coat hanger so that when the hanger hangs level, the pointer lines up exactly with the centerline on the cardboard strip. Any swinging of the hanger should cause the pointer to drift from the line. Now hang the water bottle bottoms with string from each corner of the hanger (you can bend the hanger a bit so that the string does not slide in; the bottles act as scale pans. Make sure that the hanger still hangs level when nothing is placed in either scale pan; calibrate with extra bending or mass as necessary. With a little bit of fidgeting, you should have a decent beam balance ready for use!}
\item{Theory: Beam balances do not need to be calibrated to specific masses; as long as they indicate clearly when two masses are equal, it is enough. If you have a set of known, graduated masses, you can do specific measurements.}
\end{itemize}

\subsection{Spring balance}
\begin{itemize}
\item{Preparation time: 1 hour}
\item{Materials: pen springs, paper, ruler, known masses, pen, eye-hooks, glue}
\item{Construction: Hang the spring(s) from an eye-hook in whatever frame you choose. From the bottom of the springs, hang the other eye-hook such that any weight can be hung again from it. On the bottom eye-hook place a pointer (paper, needle, etc.) facing sideways, then glue the paper over the frame so that the springs are free to move up and down and the pointer always points to some point on the paper (toothpaste boxes, glycerin boxes, etc. work well). Mark the pointer’s position when the springs hang freely, then when they hold 1 gram, 2 grams, etc.\\
You now have a spring balance, though you will have to do your own work to make it smooth and structural depending on your materials. You can measure mass, acceleration due to gravity, and the spring constant. Every spring has a constant.}
\end{itemize}

\section{Magnetism}

\subsection{Suspended Magnet Compass}
\begin{itemize}
\item{Preparation Time: 1 minute}
\item{Materials: thread, bar magnet, Optional: second magnet}
\item{Procedure: Tie the thread around the bar magnet’s center so that it hangs horizontally and is free to spin. Allow it to settle and you will see that it points north and south. Turn it away and allow it to settle again. Rotate your hand and the magnet will stay facing north and south. If you have a second magnet, pass it by the suspended magnet and watch the suspended magnet try to face the other magnet. Take the magnet away and the suspended magnet will return to its original direction.}
\item{Theory: A magnet will naturally align itself with the Earth’s magnetic field. Usually there is too much friction for this to happen, but a suspended magnet is free to face North and South. Even if you try to confuse it by turning it or by bringing another magnet close, it will eventually align itself with the earth’s field.}
\end{itemize}

\subsection{Magnetic Dip Gauge}
\begin{itemize}
\item{Preparation Time: 15 minutes}
\item{Materials: magnet, sewing needle, cork, two pins, paper, pen, cardboard or metal strip}
\item{Procedure: Push the two pins into the ends of the cork to create an axle. Push the sewing needle through the cork perpendicular to the axle pins so that the needle rolls end-over-end when you roll the cork/pins between your fingers. Adjust the needle so that it rests horizontally when the cork is free to pivot (equilibrium). Use the magnet to magnetize the needle without changing its position in the cork.\\
Bend the metal or cardboard strip into a U-shape to create a stand for the cork and pins. Rest the pins on each style of the U-stand so that the needle is free to rotate vertically. If you like, cut out a semicircular piece of paper and label the angles 0 – 90 degrees on it; tape or glue this to the stand. The needle will rotate, or dip, to point in the vertical direction of the earth’s magnetic field.}
\item{Theory: Earth’s magnetic field is not level across the surface of the earth: it goes into or out of the ground at an angle depending on the latitude. The angle of the field relative to the surface is called Magnetic Dip and is measured with this needle.}
\end{itemize}

\subsection{Mapping Magnetic Fields}
\begin{itemize}
\item{Preparation Time: 1 minute}
\item{Materials: bar or horseshoe magnet, iron wool, piece of white paper}
\item{Procedure: Place the magnet on a table and the paper over the magnet. Using your thumb and forefinger, rub the iron wool above the paper. Small pieces of iron should fall onto the paper, gradually mapping out the field of the magnet below. Move the wool around as you do this to try to show the field in a wide area. If the magnet is too strong, put some space between it and the paper. Try this with two magnets, showing attraction and repulsion between the poles. Note that the field is strongest at the poles.}
\item{Variation: Pour the filings into a container of viscous fluid (play around with glycerin and others). Shake the container so the filings are distributed around the fluid. Hold a magnet next to the container; the filings will arrange themselves into the 3D pattern of the field.}
\item{Theory: Magnetic fields extend from the North Pole of a magnet to any South Pole. Iron filings are small enough that they can form patterns in any magnetic field, showing the shape and the relative strength and various points on the field. If the field is strong enough, the filings will also form a 3D structure.}
\end{itemize}

\subsection{Pin compass}
\begin{itemize}
\item{Preparation Time: 1 minute}
\item{Materials: small pin, magnet, small dish of water}
\item{Procedure: Magnetize the pin by stroking it with one pole of the magnet; use this time also to review methods of magnetization. Place the pin gently on the surface of the water so that it does not sink (you can review surface tension here if you like), watch as it rotates to face north and south.}
\item{Theory: The earth is a magnet and its field lines can be seen using a compass, as a compass itself is a magnet and will align itself with any magnetic field. By magnetizing the pin, you make it into a compass needle, which will naturally align with the earth’s field, and the water allows it to pivot freely.}
\end{itemize}

\subsection{Magnetizing a Nail}
\begin{itemize}
\item{Preparation Time: 1 minute}
\item{Materials: nail, insulated wire (speaker wire), 2 or more D-cell batteries}
\item{Procedure: coil the middle of the wire around the nail to create a solenoid. Connect the two ends of the wire to the battery. The nail and connectors will become hot and the nail will become magnetized. You can use it to pick up staples, paper clips, etc.}
\item{Theory: The moving electric charge in the wire solenoid creates a magnetic field in the nail (use the RHR), aligning the “domains” in north-south. The stronger the current is, the stronger the magnetic field and therefore the stronger the magnet. If you use another material, you will find that the magnet is not as strong as the iron nail.}
\end{itemize}

\section{Newton’s Laws}

\subsection{Tin Can Piñata}
\begin{itemize}
\item{Preparation Time: 5 minutes}
\item{Materials: Two cans or buckets, sand, string or rope, stick}
\item{Procedure: Hang the two cans/buckets from the stick with the string so that they hang at equal lengths. Pour a small amount of sand in one can and a large amount in the other. Support the stick between two desks and start the cans swinging. Have students stop each can, feeling the difference in the force it takes to stop the almost empty can as opposed to the full can.}
\item{Theory: Inertia, or momentum, of an object is directly proportional to its mass. The full can, therefore, has more inertia and will tend to continue its motion more than the empty can. You can also offer to throw a piece of chalk or a desk to a student. They usually choose the chalk.}
\end{itemize}

\subsection{Magic Card Trick}
\begin{itemize}
\item{Preparation Time: none}
\item{Materials: empty soda bottle, card, heavy coin}
\item{Procedure: Place the card over the mouth of the bottle and let the coin rest on top. Invite students to try to remove the card without moving the coin. Most will not be able to do it. Flick the card quickly from the side; it should fly off the bottle, leaving the coin resting neatly on top of the bottle.}
\item{Theory: The coin has inertia, meaning it will resist any changes to its motion. Despite the friction from the card pulling the coin off the bottle, the coin will remain in place. This is also a good demonstration of impulse.}
\end{itemize}

\subsection{Balloon Rocket}
\begin{itemize}
\item{Preparation Time: 0 minutes – easy: 15 minutes – advanced}
\item{Materials: easy: balloon; advanced: also 2 m (or longer) string, nails, paper, tape, 1 large rubber band, paper clip}
\item{Procedure:
\begin{itemize}
\item{Easy – Inflate the balloon by blowing into it. When it is big, release the balloon. It will fly around the room.}
\item{Advanced – Cut paper into a strip about 5 cm by 10 cm. Roll the paper strip into a cylinder 5cm long, with a small diameter, maybe 0.5 cm. Tape the cylinder so it stays, and attach the rubber band with tape. Put the string through the cylinder. Attach the ends of the string to nails in the ceiling, or perhaps stretched between 2 retort stands (or even have students holding the ends) so that the string is horizontal. Put the paper clip on the string. Inflate the balloon and then use the rubber band to hold the big part to the paper, and attach the mouth of the balloon to the paper clip. Release the balloon and it will shoot across the string. This demonstrates the same principles as the “easy” version above, but because the balloon goes in a straight line, it is somewhat easier to see.}
\end{itemize}
} % Procedure
\item{Theory: The balloon pushes the air out, so there is an equal and opposite force of the air pushing the balloon. Momentum is conserved; as the air goes backwards, the balloon goes forwards.}
\end{itemize}

\subsection{Bottle Rocket}
\begin{itemize}
\item{Preparation Time: 1 hour}
\item{Materials: empty 500 ml water bottle, nail, rubber stopper, straight pin, bicycle pump, needle attachment for pump (the type used to fill a football), tape, old pen, rigid straight wire (approx 1 meter), water.}
\item{Construction: Make a small round hole (between 0.5 and 1.0 cm in diameter) in the lid of the water bottle. One easy way to do this is to heat the head of a nail until it is hot, and then use it to melt a hole in the lid. Cut a round piece of the rubber stopper so that it can be used to stop this hole. The stopper should form a good seal in this hole, but it should be possible to push the stopper through the hole by exerting some force on it. Pierce the stopper with a straight pin (if may help to heat the straight pin first) so that you can pass the needle attachment for a bicycle pump through the stopper (see figure 1, page 6). You should be able to put the stopper in the hole inside of the lid, and insert the needle attachment through the stopper so that you can increase the pressure inside of the bottle.@Disassemble a pen and cut the body so that you have two hollow cylindrical pieces approximately 3cm long each. Affix them to the side of the bottle using adhesive tape. They should be in a straight line with each other.}
\item{Procedure: This demonstration should be done outside. Insert the rigid straight wire into the ground. Fill approximately half the bottle with water. Put the stopper on the inside of the lid. Put the needle attachment through the stopper. Put the lid on the water bottle and tighten. Pass the rigid wire through the pen cylinders, and lower the bottle to the ground (see figure 2, page 6) Pump the bicycle pump. Once the pressure in the bottle becomes great enough, the stopper will be forced out of the bottle, and the rocket will fly into the air. It should be possible to reach a height of 10 meters or more.}
\item{Theory: When the stopper leaves the bottle, pressurized air forces water out of the bottom of the bottle at a high speed. Just as with the matchstick rocket and the balloon rocket, this results in a reaction force forwards on the rocket. As with the matchstick rocket and balloon rocket, we can also consider this from the perspective of conservation of momentum.\\
N.B.: After the bottle rocket fires, find the rocket and note the thick white fog that appears inside of the bottle. This is caused because the rapid expansion of the air during firing is adiabatic, causing cooling, lowering the temperature of the gas below its dew point.\\
Figure 1\\
Figure 2}
\end{itemize}

\subsection{Matchstick Rocket}
\begin{itemize}
\item{Preparation time: 5 minutes}
\item{Materials: Matches, straight pin, metal foil, scissors, small rock}
\item{Construction:
\begin{itemize}
\item{For this demonstration, I have found that the metal foil from underneath the lid of a Blue Band container works best. If using this foil, make sure you have cleaned off any Blue Band that may have adhered to it. Normal aluminum foil works as well.}
\item{Different brands of matches work better or worse. Best results have been found with Lucky brand matches, although others have also worked well (the current record for Kasuku matches is 3m). You should attempt this demonstration once or twice yourself with a certain type of matches before doing it in from of the class.}
\item{Cut out a rectangular piece of foil approximately 2cm x 4cm. Place the straight pin along the length of the match, with the point of the pin touching the match head (see figure 1). Wrap the foil tightly around the match and pin, with about half of the foil extending past the tip of the match (see figure 2). Fold the extra foil securely down over the tip of the match (see figure 3). Remove the straight pin. You have now finished constructing a matchstick rocket.}
\end{itemize}
} % Construction
\item{Procedure: Prop the matchstick rocket on two small, smooth rocks so that it is at approximately a 45° angle. Light another match and hold it underneath for several seconds (see figure 4). The heat will cause the matchstick rocket to ignite. It should fire for a distance of between one and five meters.\\
Once you have practiced on several matchstick rockets, you should be able to make several of them per minute. It is therefore easy and worthwhile to bring several of them to class, so that you can repeat the demonstration several times. This also gives you an opportunity to invite several students to launch a rocket themselves. If time permits, have a contest to see who can get the best distance on a rocket.}
\item{Theory: When the matchstick rocket ignites, rapidly expanding hot gases are produced. These are only able to escape by following the pathway left behind by the straight pin. The hot gases are forced backwards from the rocket at a high speed. Newton’s Third Law of Motion tells us that for every force there is an equal and opposite counter-force. Because the hot gases are being forced backwards, there must be a counter-force pushing the rocket forwards.\\
Alternately, we can consider Conservation of Momentum. Initially, the matchstick rocket is at rest. Once it ignites, hot gases develop a backwards velocity. Because momentum must be conserved, some other part of the system must develop a forward velocity. Thus, the rocket will fly forwards.}
\end{itemize}

\section{Projectile Motion}

\subsection{Object Toss}
\begin{itemize}
\item{Preparation time: none}
\item{Materials: Any object(s)}
\item{Procedure: When teaching projectile motion, it is productive to throw objects in the classroom. This is useful, and extremely simple. Almost any object may be used. In the past, I have used the keys from my pocket, lemons from the lemon tree next to our classroom, small pieces of chalk, and my coffee cup.\\
One good demonstration consists of repeatedly throwing an object vertically up in the air and then catching it when it returns to your hand. Point out that when you first throw it up, it has an upward velocity. As it moves up, the velocity becomes less. At the top of its trajectory, it momentarily has a zero velocity. After that, it gains a downward velocity, at first a small one and then increasing in magnitude.\\
If you are walking across the classroom at a constant rate while performing this demonstration, you can additionally show that the projectile continues to move horizontally at the same rate, matching your motion. This shows that the horizontal velocity of a projectile is a constant.}
\end{itemize}

\section{Pressure}

\subsection{Hydraulic Press}
\begin{itemize}
\item{Preparation Time: 15 minutes}
\item{Materials: Two syringes of different sizes (50 ml and 20 ml work well), thin rubber tubing, water}
\item{Procedure: Fill the larger syringe with water and attach one end of the rubber tubing to its end. Attach the other end of the tubing to the smaller syringe (the plunger should be inserted all the way in the smaller syringe). Pushing the plunger of the larger syringe will cause the plunger of the smaller syringe to go out, and vice-versa. You will notice that it is easier to push the plunger of the small syringe than that of the larger syringe.}
\item{Theory: Pressure is equal to force per area, and the pressure is distributed equally throughout a liquid. As such, the pressure at one plunger must be equal to the pressure at the other plunger. Setting the two ratios equal, we can see that a small force over a small area can overcome a large force over a large area.}
\end{itemize}

\subsection{Holey Bottle 1}
\begin{itemize}
\item{Preparation time: 5 minutes}
\item{Materials: empty 1.5 liter water bottle, water}
\item{Construction: Pierce three or more small (less than 0.5cm) neat holes in the water bottle, at different vertical heights. An easy way to do this is to use firm but gently pressure with a metal needle from a syringe. Make sure to pierce the bottle in parts that are vertical, not parts that slope in or out.}
\item{Procedure: Fill the bottle with water and place it on the ground or on a table. Water will pour out of the bottle through the small holes. Note that the streams of water strike the ground at different distances from the bottle. At any given time, the hole that is closest to one-half of the height of the water level should strike the ground or table at the greatest distance from the bottle.}
\item{Theory: To understand this we must consider both Bernoulli’s Principle and Projectile Motion. Bernoulli’s Principle tells us that the horizontal speed of each stream of water varies as the square root of the depth from the surface of the water. Projectile Motion tells us that the distance reached by a stream of water is proportional to the velocity of the water and to the time before the water strikes the ground. The time before the water strikes the ground is proportional to the square root of the height above the ground. Combining these two factors, we find that the maximum distance will be reached for a stream of water that is halfway between the ground and the surface.}
\end{itemize}

\subsection{Holey Bottle 2}
\begin{itemize}
\item{Preparation Time: 10 minutes}
\item{Materials: Water bottle, pin or small nail, water, bucket (for catching water)}
\item{Procedure: In even intervals around the base of the water bottle, poke small holes with the pin or nail. Try to get an even distribution and the same size hole all around. Fill the bottle with water and watch the water leave each hole with the same force. Blowing into the bottle will help illustrate the equality of the pressure in all directions.}
\item{Theory: Pressure in a fluid acts equally in all directions, therefore the water being forced out the bottom should feel the same amount of pressure and shoot the same distance.}
\end{itemize}

\subsection{Straw Fountain}
\begin{itemize}
\item{Preparation Time: 10 minutes}
\item{Materials: 0.5 liter water bottle with cap, water, straw, glue}
\item{Procedure: Poke a hole with the diameter of the straw in the cap of the water bottle with a hot nail or pin. Insert the straw so that it extends almost to the bottom of the water bottle and leaves enough sticking out for your mouth. Secure it with glue so that it is airtight. When the glue is dry, fill the bottle about half way with water and close the cap with the straw inside. Have a student blow as hard as they can through the straw into the water. When they run out of air and stop blowing, they will get a nice spray in the face. }
\item{Theory: By blowing into the bottle, you greatly increase the pressure inside. When you finish blowing, the pressure will try to equilibrate by forcing the pressure back out through the straw. There is nowhere for the water to go but out.}
\end{itemize}

\subsection{Siphon}
\begin{itemize}
\item{Preparation Time: 1 minute}
\item{Materials: two containers, half meter of rubber tubing/IV line/feeding tube, water}
\item{Procedure: Place one jar with water on a table and the other empty jar on a chair just below the table. Place one end of the tubing into the water and the other in your mouth. Suck on the tube until the water starts coming out and place the end of the tube into the empty beaker, holding the middle of the tube at the level of your mouth. The water will continue to flow from the water jar to the empty jar, despite the water’s initial uphill climb.}
\item{Theory: By sucking on the tubing, you create low pressure on that side. The slightly higher pressure (atmospheric) at the water will cause the water to continue to travel as long as the pressure difference is enough to overcome gravity. If you raise the middle of the tube too high, the water will stop flowing.}
\end{itemize}

\subsection{Balloon Pop}
\begin{itemize}
\item{Preparation Time: 20 minutes}
\item{Materials: piece of wood, nails, water balloons, water}
\item{Procedure: Put one nail through the board in one place and a large cluster of closely spaced nails in another place, all pointing up. Fill a balloon with water. As you lower the balloon onto the single or few nails, the balloon eventually pops. Fill another balloon with water and slowly lower it onto the cluster of nails. It should not pop.}
\item{Theory: As area of a force increases, pressure decreases. Therefore, as more nails are added and the area of the force (the weight of the balloon) increases, the pressure decreases and the balloon does not pop. Or, it takes more force to pop.}
\item{Alternative: hang the balloon from a spring balance as you lower it (by holding the spring balance) onto the nails. The difference in weight will allow you to calculate the force needed to pop the balloon.}
\end{itemize}

\subsection{Potato Poke}
\begin{itemize}
\item{Preparation Time: none}
\item{Materials: some straws, potato}
\item{Procedure: Take a straw and jab it into the potato. The straw should bend easily leaving the potato unharmed. Now place your thumb firmly over one end of a straw and jab the other end into the potato. This time the straw should enter the potato quite easily.}
\item{Theory: The straw is weaker than the potato and so will bend rather than break the potato’s skin. But, with your thumb plugging the back of the straw, the air inside the straw cannot leave and instead pushes out against the sides of the straw. As the straw strikes the potato, it cannot bend with the air pressure inside and so instead can poke through the skin into the potato.}
\end{itemize}

\subsection{Straw Elevator}
\begin{itemize}
\item{Preparation Time: none}
\item{Materials: two straws, container, water}
\item{Procedure: Fill the container with water and insert one straw so that it stands vertically in the water. Using the other straw, blow across the opening of the vertical straw; the water level in the straw will rise.}
\item{Theory: Bernoulli’s Principle states that moving air causes low pressure; the air passing in a stream over the vertical straw creates low pressure and therefore a pressure differential between the bottom of the straw (the water) and the top. The water will move towards the lower pressure, moving up the straw.}
\end{itemize}

\subsection{Reverse Air Pump}
\begin{itemize}
\item{Preparation Time: varies, about 1 hour}
\item{Materials: Bicycle pump (the tall, metal kind), short piece of rubber tubing fitted to pump valves, utility knife, tightening sleeves, extra valve}
\item{Procedure: There are two parts of the pump that control the direction of airflow: the first is a diaphragm inside the pump and the second is a ball valve at the base of the pump in the hose.
\begin{enumerate}
\item{You need to open the pump and pull out the ‘dipstick’ with the diaphragm attached. At the bottom, there should be a diaphragm with holes around the top, a metal disc the same diameter as the diaphragm, and a few nuts and washers to keep it all together. In its normal configuration, the diaphragm is pulled down by friction away from the disc when the pump handle is pulled up, allowing air to enter the pump freely. When the pump handle is pushed in, the diaphragm is forced against the disc, restricting any back airflow, and forcing all the air forward through the hose. Switch the position and direction of the diaphragm and disc so that it has the opposite effect when the pump handle is pulled in or out.}
\item{Next, you need to cut open the hose at the base of the pump and find the valve with the small bead inside. Normally, when air is forced forward through the valve, the bead does not restrict any airflow. When air tries to go back through the pump, the bead blocks the valve and stops any airflow. Switch the direction of the valve.}
\item{From here, you need to reattach the hose to the pump. You may need to get another nozzle to attach to the pump, attaching the hose with reversed valve with the extra bit of rubber tubing. It depends on your pump, but if you have made it this far, you will find a way to make it work. Tightening sleeves will come in handy here to make sure no air is lost after all this cutting and jury-rigging.}
\end{enumerate}
} % Procedure
\item{Applications: This suction pump is great for showing the gas laws and boiling points: suck the air out of a jar of water and watch the water boil, you could also kill stuff in the jar this way, but that is just morbid, and possibly cool, or that sound travels through a medium.}
\end{itemize}

\subsection{Atmospheric Pressure}
\begin{itemize}
\item{Preparation Time: 5 minutes}
\item{Materials: Water bottle, pin and/or nail, water}
\item{Procedure: Using an empty water bottle (bigger is better), poke four or five small holes (0.5 cm) in the bottom with the pin and then the nail. Fill the bottle about half way with water, allowing it to spill out through the holes in the bottom. While the students are watching, seal the cap on the bottle. The water will cease to pour out of the bottom despite the holes and rather predictable effect of gravity. When the gasps of wonder die down, discuss the following:}
\item{Theory: The pressure of the water combined with the pull of gravity is enough to cause the water to pour through the holes in the bottle when the cap is not sealed. When the cap is on tight, however, the combined high air pressure outside the bottle and low air pressure inside the bottle creates enough of an upward force on the water to counter the downward force of gravity.}
\end{itemize}

\section{Properties of Matter}

\subsection{Water drops}
\begin{itemize}
\item{Preparation time: none}
\item{Materials: Water dropper or syringe}
\item{Procedure: Slowly drip water from the water dropper or syringe and point out that before a drop falls; it will hang suspended by its surface tension. Explain that as the drop becomes larger, its weight increases until surface tension is insufficient to support it, at which point it falls.}
\end{itemize}

\subsection{Blowing bubbles}
\begin{itemize}
\item{Preparation time: 5 minutes}
\item{Materials: Thin piece of wire (approximately 30cm), water, detergent, glycerin (optional)}
\item{Procedure: Bend the wire into a loop 2 to 3 cm in diameter. Continue to bend the wire so that it circles around the circumference of this circle several times. Leave a straight piece several centimeters long to use as a handle. This is the “bubble blower”. Dip the circular part of the bubble blower into a strong solution of detergent (regular powdered laundry detergent works well) mixed with glycerin. When you remove the bubble blower from the solution, a thin film should remain across the circle. Gently blow through the center of the circle. With a little practice, you should be able to cause a spherical bubble to separate from the blower and float away.}
\item{Theory: The detergent causes the surface tension in the solution to be slightly variable. In areas of higher concentration of detergent, the surface tension is lower. In order for the films and bubbles to be stable, the surface tension near the top must be slightly higher than at the bottom. As the detergent molecules are heavier than water, they tend to sink towards the bottom of the film, accomplishing this.\\
When you blow through the bubble blower, we can see that then tension is pulling it back towards a flat surface. Once an independent bubble is formed, we see that it forms a nearly perfect sphere. This is because the surface is under tension. This tension forces the bubble to form the shape with the minimum surface area, a sphere. It is also worth noting that both the film that stays on the bubble blower and the bubbles themselves appear to have small rainbows of colors in them. This is caused by thin-film interference.}
\end{itemize}

\subsection{Pin Float}
\begin{itemize}
\item{Preparation time: none}
\item{Materials: A cup or small dish, a straight pin, water, detergent}
\item{Procedure: Make sure the cup or dish is clean, and has no soap or detergent residue. Fill the cup or dish with clean water. Carefully place the straight pin on the surface of the water, being careful not to break the surface. If done properly, it should be possible to get the straight pin to remain suspended on the surface (see also floating compass). Next, sprinkle a small amount of detergent onto the water. The pin should sink to the bottom.}
\item{Theory: When the straight pin is placed on the surface, it causes the surface of the water to bend downwards. This means that the surface tension of the water is pulling the pin upwards. Although the pin is denser than water, and would normally sink, this surface tension is enough to support the weight of the pin. When detergent is sprinkled onto the surface of the water, it lowers the surface tension of the water. The surface tension is no longer strong enough to hold up the pin, so the pin sinks.}
\end{itemize}

\subsection{Pepper Float}
\begin{itemize}
\item{Preparation time: none}
\item{Materials: A cup or dish, water, ground black pepper, soap or detergent}
\item{Procedure: Make sure that the cup or dish is clean, and has no soap or detergent residue. Fill the cup or dish with clean water. Sprinkle ground black pepper over the surface of the water in a way that the pepper is distributed evenly and covers the whole surface. Next, apply a small amount of soap or detergent to one finger. Touch this finger to the surface of the water in the center of the cup or dish. The pepper will flee your finger, and all run to the sides of the cup or dish.}
\item{Theory: When you touch your finger to the surface, you introduce a small amount of soap or detergent, lowering the surface tension at that point. The surface of the water is now unbalanced – the surface tension near the edge is pulling the surface outwards more strongly than the surface tension at the center is pulling the surface inwards. As there is a net force on the surface outwards towards the edge, the surface moves, pulling the pepper along with it to the edges of the cup or dish.}
\end{itemize}

\subsection{Water Dome}
\begin{itemize}
\item{Materials: Coin, water, syringe or eyedropper}
\item{Preparation Time: none}
\item{Procedure: Place a coin flat on the table. Using the syringe or eyedropper, carefully drop individual water drops onto the coin. With some practice, you should be able to get quite a few drops onto the coin before the water spills over, creating a dome of water.}
\item{Theory: The surface tension of the water holds it together against the force of gravity, which is trying to pull the water off the coin.}
\end{itemize}

\subsection{Pinching Water}
\begin{itemize}
\item{Preparation Time: 10 minutes}
\item{Materials: 0.5 liter bottle, water, pin or small nail}
\item{Procedure: At the bottom of the side of the can or bottle, poke five small holes close together with the syringe needle or nail. Be careful not to let the holes overlap or be too far apart. Pour water into the bottle and allow the water to start flowing out of the holes at the bottom. Using your thumb and forefinger, pinch the streams of water together so that they form a single stream (this takes some practice). To undo your great work of creation, pass your hand in front of the holes and five streams will appear again.}
\item{Theory: Water has a tendency to “cling” to itself due to its surface tension and cohesion. As you bring the streams together, you allow the water to stick to itself forming a single stream. Passing your hand in front again stops the flow of water at the holes and allows it to start again, which it will do in five streams.}
\end{itemize}

\subsection{Lemonade}
\begin{itemize}
\item{Preparation Time: 5 minutes}
\item{Materials: lemon, drinking water, pitcher}
\item{Procedure: Make lemonade by putting lemon wedges (oranges also work) into the pitcher and adding about a liter of water. Let it sit for a couple hours, then drink and enjoy! Adding sugar or honey is recommended.}
\item{Theory: The citrus flavor of the lemons will gradually spread throughout the water, though no force is apparent. This process is called diffusion. See the Transport topic in the Biology section for more activities involving diffusion.}
\end{itemize}

\section{Rotation}
subsection{Object Spin}
\begin{itemize}
\item{Preparation Time: 1 minute}
\item{Materials: toilet paper tube, string, or rope 2 m, two objects of any kind (masses)}
\item{Procedure: Pass the half rope through the cardboard tube and tie each end to a mass of some kind. Holding the tube vertically in your hand, get the rope end on top spinning over your head. As you increase the speed of the spinning end, the rope should slide up through the tube, creating a longer length of rope spinning and shorter length hanging below the tube. Decrease the speed and allow the rope to settle back through the tube.}
\item{Theory: Centripetal force is related to the rotational speed of the object, so as the speed increases, the force with which it pulls outward increases. As this happens, the centripetal force of the spinning object overcomes the gravitational force on the hanging object, pulling it up.}
\end{itemize}

\subsection{Twenty Shillings of Equilibrium}
\begin{itemize}
\item{Preparation Time: none}
\item{Materials: Meter rule, handful of 20/= coins}
\item{Procedure: Balance the meter rule on your forefinger at the 50 cm mark. Discuss the idea of equilibrium and equal moments. Place one coin at the 1 m mark and watch the rule rotate and the coin fall, demonstrating a lack of equilibrium. Now replace the coin and add another coin to the 0 m mark. The meter rule should stay steady. For more numerical fun, add two coins at the 25 cm mark, four coins at the 12.5 cm mark, and even eight coins at 6.25 cm if you dare! Have students calculate the necessary lengths or masses (number of coins – we do not need actual units here since the masses are equal) while you perform the daring feats in front.}
\item{Theory: The moment of a force is equal to the force times the distance from the pivot. The pivot is your finger and the forces are all at even intervals. If the total moment in the clockwise direction is equal to the moment in the anticlockwise direction, you have equilibrium. You can add or remove coins, change distances, etc. while still keeping the moments equal.}
\end{itemize}

\subsection{Windmills}
\begin{itemize}
\item{Preparation Time: 15 minutes}
\item{Materials: thin cardboard or cardstock, scissors, pen, colored pencils/markers if desired, glue, paper fastener or thumb tack, straw or stick}
\item{Procedure: Use the following illustration (enlarge it); copy it onto a piece of thin cardboard or cardstock.}
\begin{enumerate}
\item{Cut along the lines and make holes with a pencil or pen.}
\item{Bend the four corners together into the center and glue them in place.}
\item{Push the fastener or tack through the center hole into a straw or stick.}
\end{enumerate}
\item{Reference: This demo was published in the Science Lab Kit by Silver Dolphin Books in 1997, compiled by Brenda Walpole}
\end{itemize}

\subsection{Helicopters}
\begin{itemize}
\item{Preparation Time: 15 minutes}
\item{Materials: paper, scissors, paper clip}
\item{Procedure: Copy the following design onto a piece of paper. Cut along the solid lines and fold along the dotted lines, attaching the paper clip to the bottom. Drop the helicopter with the paperclip down and watch it spin!
*This demo was published in the Science Lab Kit by Silver Dolphin Books in 1997, compiled by Brenda Walpole}
\end{itemize}

\subsection{Gyroscope}
\begin{itemize}
\item{Preparation time: 30 minutes}
\item{Materials: rubber strip (at least 4 pieces, about 40cm long), weights (at least 4, ½ kg is good), bicycle wheel, rope}
\item{Construction: The weights are used to increase the moment of inertia of the bicycle wheel. Use the rubber strip to tie firmly the weights to the bicycle wheel. Space them evenly so that the weight is equally distributed. Using four ½ kg weights works well, six works even better. Then, tie the rope in a loop to something strong enough to support the bicycle wheel. Arrange it so that the loop is about chest-high.}
\item{Procedure: Hold the axle of the bicycle wheel and have a student spin the wheel very fast. Then place one end of the axle in the loop of rope. The wheel will remain vertical, but will turn in a circle around the rope. In addition, you can have students hold the spinning bicycle wheel. If they move it, they should be able to feel that if the axis stays in the same direction, it is easy to move it, but it is difficult to change the direction of the axis of rotation. You can also balance the axle of the wheel on the end of one finger as the wheel spins around your arm.}
\item{Theory: The bicycle wheel is a gyroscope. By adding the weights to the rim and not the center, you increase the moment of inertia greatly, which makes the angular velocity nice and slow. It also increases the angular momentum of the spinning wheel so that it takes more torque to change the direction of the axis of rotation when the students try to move it.}
\end{itemize}

\subsection{Door Tug-o-War}
\begin{itemize}
\item{Preparation time: none}
\item{Materials: a strong door}
\item{Procedure: Get 2 students. One is going to push against the door near the hinge; one will push the other way near the other side (handle) of the door. The one pushing near the edge of the door will find it easy to push the door her way.}
\item{Theory: The student that pushes farther from the axis of rotation can exert less force, but still produce a greater torque, or moment of force, than the one pushing close to the hinge, because.}
\end{itemize}

\subsection{Spinning Eggs}
\begin{itemize}
\item{Preparation time: none}
\item{Materials: 2 eggs (1 boiled, 1 fresh)}
\item{Procedure: Spin the eggs on their sides. If you briefly stop the fresh egg, then release it, it will begin to spin again by itself. However, if you briefly stop the boiled egg, it will remain stopped.}
\item{Theory: The boiled egg is a rigid body, so when you stop its shell, you stop the whole egg. The fresh egg is not a rigid body. It has liquid inside, so when you stop the shell, the inside continues spinning. When you release it, the inside has enough angular momentum to start the whole thing spinning again.}
\end{itemize}

\subsection{Race of Rigid Bodies}
\begin{itemize}
\item{Preparation time: 5 minutes}
\item{Materials: Wide ramp, several different rolling objects (empty can or bottle, small ball, coin, etc)}
\item{Procedure: Roll the several objects down the ramp. Observe that objects with a larger ratio of radius of gyration to actual radius will roll more slowly. Point out which objects roll slower and which faster. Use this to demonstrate radii of gyration.}
\item{Theory: Rotational momentum depends on the distribution of mass along the radius of an object. A solid object will therefore roll differently than a hollow object.}
\end{itemize}

\section{Simple Harmonic Motion}

\subsection{Simple Oscillator}
\begin{itemize}
\item{Preparation time: 1 minute}
\item{Materials: Spring, thread or piece of rubber strip, several weights}
\item{Procedure: Attach the spring or rubber strip to your weight. The weight could be anything: a laboratory weight, a set of keys, or a small padlock. Start the weight oscillating, while explaining to the students how simple harmonic motion works. Add more weight (more keys, another padlock) and observe that there is no change to the period. Now increase or decrease the length of the pendulum and observe any changes to the period. You can tabulate the results for different masses and lengths (keeping one thing constant each time) so that students can see experimentally the dependence of period on length or mass.}
\item{Theory: The period of a pendulum depends on the length of the pendulum (neglecting air resistance), so no change should be noticed if the mass is changed.}
\end{itemize}

\subsection{Bottle Sine Graph}
\begin{itemize}
\item{Preparation time: 5 minutes}
\item{Materials: Empty water bottle, string (approximately 0.5m), water}
\item{Construction: Remove the cap of a water bottle and make a small hole in the center of the bottom with a syringe needle or a nail. Tie a string around the top of the water bottle.}
\item{Procedure: Fill the water bottle. Swing it as a pendulum from left to right while walking forwards. Water will pour out from the bottom of the water bottle in a thin stream, leaving a wet mark on the floor, which creates a graph of its position.}
\item{Theory: As you walk forward, you cause the forward direction to be the time axis. The bottle swings left to right, leaving a watery record of where it has been. Because a swinging pendulum executes simple harmonic motion, this demonstration allows us to see that simple harmonic motion has the shape of a sine curve.}
\end{itemize}

\subsection{Barton’s Pendulums}
\begin{itemize}
\item{Preparation time: 5 minutes}
\item{Materials: Several pieces of string, one large weight (approximately 0.5kg), several small weights}
\item{Construction: Suspend a piece of string horizontally between two fixed objects. Hang the various weights from different points along the string. Each of the small weights should hang from a string of different length. The large weight should hang from a string of similar length to one of the small weights.}
\item{Procedure: Start the large weight swinging. Tell the students to take note of how this affects the behavior of the smaller weights. You should find that the small weight hanging from a string of the same length as the large one exhibits the largest oscillation.}
\item{Theory: The large weight acts as a driving force. Each small weight can swing as a simple harmonic oscillator. We know that a driving force will have the largest effect on a simple harmonic oscillator if the driving force is operating at the natural frequency of the oscillator. When the lengths of the two pendulums are the same, their frequencies are the same. You should be able to get “harmonics” going if you measure the lengths accurately (see string instruments).}
\end{itemize}

\section{Sound}

\subsection{Sound in a Medium}
\begin{itemize}
\item{Preparation Time: half hour}
\item{Materials: Large jar with lid, glue, bicycle pump needle, string, cell phone, vacuum pump (see Reverse Pump)}
\item{Procedure: Poke a small hole in the jar lid and insert the pump needle with at least 1 cm above the lid. Secure the needle with glue, rubber, whatever you need to ensure that it is airtight. Program your cell phone to play something repeatedly at full volume. Hang the phone by the string in the jar so that it is not touching the sides; close the lid on the jar (if the glue is dry) and listen for the phone. You should still be able to hear the phone. Attach the vacuum pump from Reverse Air Pump to the needle on top of the jar and start pumping out the air. You should hear the sound of the phone decrease until it is not heard at all.}
\item{Theory: Sound requires a medium to travel. The denser the medium, the faster sound will travel. Without a medium, there is nothing to vibrate and therefore no sound. By removing the air in the jar, you are removing any material medium and the sound will not be able to travel beyond the cell phone speaker itself.}
\end{itemize}

\subsection{Spoon bell}
\begin{itemize}
\item{Preparation Time: 1 minute}
\item{Materials: spoon, string 1 m}
\item{Procedure: Tie the spoon into the middle of the length of string so that it will hang freely when you hold the string ends. Have a student hold the string ends to his or her temples or the bone just under his or her ear as you strike the spoon with a pen or other object. The student should hear a clear, loud sound.}
\item{Theory: The vibrations of the spoon propagate up the string and into the student’s head. Bone, especially around the temples and outer ear, resonates readily in response to sound.}
\end{itemize}

\subsection{One-String Guitar}
\begin{itemize}
\item{Preparation Time: 5 minutes}
\item{Materials: String or thin steel wire, two clothes clips, any mass, tape}
\item{Procedure: Secure the two clips to the table so that one is close to the edge. Stretch the string or wire across the two clips, securing it at one end by tying or clamping, and hanging the other end over the second clip and over the edge of the table. Attach some mass to this free end so that it pulls the string taught and produces a clear pitch when the string is plucked. Play around by changing the length or tension of the string and hear the different pitches. See if you can play a song!}
\item{Variation: Tape some paper just under the string and mark the ‘frets’ as you find them. The 1st harmonic should be at half the length, and so on.}
\item{Theory: The frequency of a standing wave on a string depends on the tension of the string, its length, and it’s mass per unit length. As you are not changing the string material, you do not need to worry about the last one. Increasing mass (tension) on the string will raise the pitch, as will shortening the length. Most students have seen guitars played and will notice what is going on.}
\end{itemize}

\subsection{Bottle Amplifier}
\begin{itemize}
\item{Preparation Time: 10 minutes}
\item{Materials: Plastic water bottle, string or thread, match or small stick}
\item{Procedure: Poke a small hole in the bottom of the bottle and string one end of the thread through the hole. Tie the end on the inside to the match or small stick so that it cannot be pulled back through the hole. Pull the string taught and have a student hold the top of the bottle. Pluck the string and hear the nice loud sound! Play around with plucking just the string vs. the string and bottle together. Try it with the cap on or off.}
\item{Theory: The vibration of the string causes the bottle itself to vibrate. Rather than hearing just the sound of the string vibrating, we hear the sound of the bottle, which produces noticeably greater amplitude.}
\end{itemize}

\subsection{Transverse Waves on a String}
\begin{itemize}
\item{Preparation time: depends, but in any case a long time}
\item{Materials: Show a design to a fabricator/welder and let them decide this. You can supply thin string, a small pulley, and a weight.}
\item{Construction: Using whatever driving device available. (I used a bicycle, like the men who pedal a bike wheel to drive a grinder), drive a piston with a very small amplitude (1 mm is fine). Whomever you find to do this will have their own way of doing this, but the easiest thing to do is just an offset axle, where the axle being driven jogs to one side a small amount. When you have a piston which can be driven at a very small amplitude by a bike wheel, car motor, etc., attach a string to the top of the piston and hang the other end of the string over a pulley about two meters (varies) away, suspended by a weight. Now you have a string that is driven at whatever frequency you choose.}
\item{Procedure: Pedal the bicycle or turn on the motor and increase the speed (frequency) until you see the fundamental on the string, a standing wave with one antinode and two nodes – the ends of the string. Chat about that for a minute, then increase the frequency until you get the first harmonic, then the second harmonic, etc., until you run out of juice in one way or another.}
\item{Variation: Drive the string with a speaker connected to a single-tone generator. This could be a simple circuit, in fact, allowing you to combine two of the biggest physics topics ever! Use a rheostat to vary the frequency of the circuit, ergo the speaker.}
\item{Theory: Every string has a natural frequency at which it will vibrate with ease, meaning with the greatest possible amplitude. This is called the fundamental (and is directly related to the fundamental as known in music theory, since all harmonics which follow are the octave, 5th, 4th, 3rd, etc.) and is the simplest standing wave. Doubling the frequency will give you the 1st harmonic (octave), which is the next simplest standing wave. All harmonics which follow are closer in frequency and become gradually more complex, but might be difficult to do on this machine unless you have a super-high gear ratio on the bike wheel or a speedy car motor.}
\end{itemize}

\subsection{Musical rubber strip}
\begin{itemize}
\item{Preparation time: none}
\item{Materials: a length of rubber strip}
\item{Procedure: Stretch the rubber strip taught and pluck it. It should produce a musical note. Demonstrate that increasing the tension but keeping the length the same gives a higher note. Demonstrate that keeping the tension the same but increasing the length gives a lower note. Allude to tuning a guitar, which many students will have seen in church.}
\end{itemize}

\subsection{Musical Soda Bottle}
\begin{itemize}
\item{Preparation time: none}
\item{Materials: 2 soda bottles of the same type and size, water}
\item{Procedure: By blowing over the top of a soda bottle it is possible to create a musical note. Add water and blow again several times to demonstrate that the higher the water level in the bottle, the higher the pitch of the note produced. Empty one bottle entirely, and in the other add enough water to achieve a depth of approximately one millimeter. Ask for a volunteer to help at this point. Blow over one bottle to produce a note. Ask the volunteer to blow over the other bottle to produce a note. Point out that the two notes sound almost identical. Now blow over your bottle at the same time as the volunteer. A beat frequency should be heard.}
\item{Theory: Because the soda bottle is open at the top and closed at the bottom, it acts as a half-open pipe, and produces notes with a wavelength of four times the height of the column of air in the bottle. Thus, by adding water, we shorten the height of the column of air, shortening the wavelength and increasing the frequency. When two soda bottles with slightly different heights of water are blown, they produce slightly different frequency notes, and so a beat frequency can be heard.}
\end{itemize}

\subsection{Doppler Whirl}
\begin{itemize}
\item{Preparation time: 5 minutes}
\item{Materials: string of length 1 or 2 meters, mobile phone, sock}
\item{Procedure: You will need a mobile phone that can be programmed with user generated ring tones. Program a ring tone that consists of one note repeated for a period of at least 20 seconds. Demonstrate to the class that the ring tone consists of just the one note. Now place the phone in the sock, tie it to the string, and swing the string rapidly around your head so that the phone moves in a large circle around you. As the phone moves towards the students, they will hear the pitch increased, and as it moves away, they will hear the pitch decreased, because of the Doppler Effect. Note that for the person swinging the phone, their phone neither approaches nor moves away from their ears, but circles around them. For them, there will be little or no discernable Doppler Effect.}
\item{Theory: Sound waves are pressure waves, so they depend on the medium through which they travel as well as the motion of the source. If the source of sound is moving, the sound waves in front of the source become compressed (much like they are being pushed), which translates as higher frequency or shorter wavelength. The sound waves behind the source are extended (much like they are being stretched behind), so the frequency is lower or wavelength longer. A higher or lower frequency is heard as higher or lower pitch.}
\end{itemize}

\section{Simple circuits}

\subsection{Circuit Board}
\begin{itemize}
\item{Preparation Time: 1 hour}
\item{Materials: Piece of flat wood, staples or small nails, hammer, broken radio case, glue, any circuit components}
\item{Procedure: Draw out a grid on the wood with squares about 5-6 cm on a side. At each grid intersection, gently tap in a staple or small nail. From the radio, take the battery casing with its terminals and clips and glue it to one side of the wood; this will be your power supply. Using any configuration you like, set up a circuit on the board using the pins as wire holders. This makes the circuit easier to handle and see.}
\end{itemize}

\subsection{Breadboards}
\begin{itemize}
\item{Preparation Time: 3-4 hours}
\item{Materials: shower sandal (new), knife, glue, sewing needle, metal strips or aluminum foil (0.5 - 1 cm wide, 5 - 10 cm long), sharpie or marker, simple circuit components}
\item{Construction: Remove the sole of the sandal and cut the bottom layer (about 0.5 cm; keep this piece whole for later) off the sandal. Inlay the metal strips at angles into the other section of the sandal in the arrangement shown below. These are the wires of your breadboard. The two long, thin sections are your power strips; the larger section is the board itself where circuit components will be placed. The angle of the metal strips allows the components to remain in contact with the strips under the constriction of the rubber sandal, but if another configuration works better, do that.\\
The diagram shows the layout of a typical breadboard; change this as necessary. Replace the cut-off section of sandal; you will need to cut slots for the metal strips to fit snugly. Using the sewing needle, punch thin holes into the bottom of the sandal along the lines of the metal strips, about 1 cm apart. Use a sharpie or marker to indicate the positions of the holes and the outlines of the different sections on the breadboard, as shown above. On the power strips, label one line as positive and the other as negative. Use glue to keep the pieces of sandal together. Now your breadboard is done; some modification may be necessary depending on the resources available.\\
As for electrical components, broken radios, cell phone chargers, old computers, etc., can be stripped for parts. Resistors, transistors, capacitors (parallel plate or cylindrical), diodes, variable resistors (rheostats), switches, fans, LEDs, heat syncs, speakers, and wires can be found easily, even in villages without electricity. If you are stuck, drop a few shillings at the Broken Stuff shop in town.}
\item{Procedure: Using your new broken radios, pull out the various components and place them in the breadboard as the circuit you desire. Connect the negative and positive ends of this circuit into the power strips, and the appropriate terminals of the power strips to some batteries, or an accumulator. If you smell burning resistors, that is another lesson. If not, then you have a circuit to play with.}
\item{Theory: Students spend plenty of time staring at circuit drawings on the board and sometimes become fairly adept at analyzing them, but when shown a real circuit they cannot tell parallel from series. When teaching simple circuits, accompany any real circuit with a drawing for students to follow.}
\end{itemize}

\section{Simple Machines}

\subsection{Bottle Cap Gearworks}
\begin{itemize}
\item{Preparation Time: 30 minutes}
\item{Materials: handful of bottle caps, pliers, nails, small piece of flat wood, hammer}
\item{Procedure: Find the exact center of each bottle cap and poke a hole through it for the nail. Bend the edges of the bottle caps in so that the ridges along the sides will act as gear teeth when the cap rotates. Nail the caps into the wood at even intervals so that they can freely rotate and in turn cause others to rotate. Make different configurations and note the direction of rotation from one gear to another.}
\end{itemize}

\subsection{Pulleys and Inclined Planes}
\begin{itemize}
\item{Preparation Time: 15 minutes}
\item{Materials: thread spool or water bottle, cardboard, thread: meter rule, spring or spring balance, various masses, stiff wire}
\item{Procedure – Pulleys: A thread spool works well as a pulley, but you can also cut out the ridged section of a water bottle and insert a circle of cardboard into the center as a support. Bend the stiff wire through the hole of the pulley so it can rotate easily without sliding off. Tie the masses to the thread and drape it over the pulley. You can make any fixed or moveable pulley with these resources.}
\item{Procedure – Inclined Planes: Prop up a meter rule at an angle. Hang a mass from a spring or spring balance and drag it up the slope. Measure the extension of the spring (or weight) for the mass when it hangs freely, and again as it moves up the slope. If friction is low, there will be a noticeable decrease in spring extension from the free-hanging mass to the mass on the slope.}
\end{itemize}

\section{Static Electricity}

\subsection{Salt and Pepper Trick}
\begin{itemize}
\item{Preparation Time: 1 minute}
\item{Materials: salt, pepper flakes, pen, dish}
\item{Procedure: Mix a spoonful of salt with a spoonful of pepper and place it on a piece of paper or dish. Charge the pen by rubbing it on your hair or a piece of cloth and hold it over the salt and pepper. Which flakes jump to the pen?}
\item{Theory: Both salt and pepper will be attracted to the pen, but the salt is too heavy to move so only the pepper will make the jump.}
\end{itemize}

\subsection{Electrostatics}
\begin{itemize}
\item{Preparation Time: 5 minutes}
\item{Materials: Plastic ruler and piece of nylon cloth, a glass object and silk cloth, or a latex balloon and piece of fur (or hair), small pieces of metal foil, thread}
\item{Procedure: Rub the plastic ruler against the piece of nylon cloth. This transfers electrons between the two items, producing an electrostatic charge. If the piece of nylon cloth is small, try suspending it from a thread near the ruler. As the two items have opposite charges, they attract each other, causing the nylon to lean towards the ruler.\\
Crumple a piece of foil into a small ball, and suspend it from a thread. Bring the charged ruler near to the foil ball. The charge on the ruler should cause an induced dipole in the foil, which is in turn attracted to the charge on the ruler, causing the foil to lean towards the ruler.\\
If you rub the ruler on two different small pieces of cloth, try suspending the two cloths near each other. As they have the same charge, they will repel and lean away from each other.\\
N.B.: The above can be performed by rubbing a plastic ruler on nylon cloth, or by rubbing glass on silk, or by rubbing latex on fur. Some clothing is made out of nylon. Silk is commonly found in the liner to suit jackets. Other combinations of items can also produce static electric charges. It is best to try these on your own before showing them in front of class.\\
This demonstration is best performed in a room with no wind or air currents, which will make it difficult to see the objects leaning towards each other. The static charges will last for a longer time if there is low humidity and a low amount of dust. On humid or dusty days, the static charges will discharge faster. This is a good alternative to the Gold Leaf Electroscope, which is rather expensive and unnecessary.}
\end{itemize}

\section{Thermodynamics}

\subsection{Copper Coil Candle Snuffer}
\begin{itemize}
\item{Preparation Time: 5 minutes}
\item{Materials: thick copper wire about 40 cm, candle, match}
\item{Procedure: Bend the copper wire into a spiral coil, leaving a length enough for a handle. It should be in the shape of a candle snuffer but clearly open. Light a candle and then put out the flame with your new snuffer.}
\item{Theory: Metal, especially copper, conducts heat readily. By putting the copper coil over a flame, you allow the copper to conduct all of the heat away from the flame, careful not to hurt your hand, depriving the flame of its own heat source.}
\end{itemize}

\subsection{Conduction of heat by different materials}
\begin{itemize}
\item{Preparation Time: 2 minutes}
\item{Materials: wooden stick, metal rod, candle, match}
\item{Procedure: Light the candle and set the stick and rod to rest with one end in the flame (the stick should not light on fire if you just grabbed it from outside, but you can dampen it to be sure). Teach your lesson for a couple minutes and then check to see if it is ready. Have students touch the end (the end NOT in the flame) of each and determine which is hotter. The metal should be significantly warmer than the stick.}
\item{Variation: Drip candle wax at even intervals along both the wooden stick and the metal rod. As heat is conducted along each, the wax will melt and drop off. You should see a significant difference between rate of melting of the stick and metal. You can also stick beans into the wax before it dries, to get a more dramatic effect when the wax melts.}
\item{Theory: Heat is conducted through metal much more efficiently than through wood; therefore the end of the metal rod will become hotter faster than the stick. You should be able to feel it easily, and if the candle is hot enough the metal will be almost too hot to hold.}
\end{itemize}

\subsection{Hot Water Hold}
\begin{itemize}
\item{Preparation time: 10 minutes}
\item{Materials: 3 beakers or drinking glasses, a thermos of very hot water 3 metal coins, a piece of cloth, a piece of rubber strip}
\item{Procedure: Pour hot water into the three beakers. Ask for three students to help with the demonstration. One student will hold his glass using rubber strip to protect against the heat. One will use fabric. One will use metal coins. Tell the students they can put down their beaker if it becomes too hot.}
\item{Theory: Metal is a good conductor of heat, and so we expect that the student using metal coins will only be able to last a short time. rubber strip is a poor conductor of heat (a good insulator), and so will protect its student’s hands for a longer time. Thus, we expect the student using rubber strip to last a long time.}
\end{itemize}

\subsection{Thermal Windmills}
\begin{itemize}
\item{Preparation Time: 1 minute}
\item{Materials: Windmills from another demo “Windmills,” candle, match}
\item{Procedure: If you have already created the windmill, set it up so that it spins horizontally about 10 cm above a candle. Light the candle and watch the windmill rotate.}
\item{Theory: As air is heated, it rises, displacing the cooler air above it. The heated air from the candle will force the windmill to rotate horizontally just as wind would cause it to rotate vertically.}
\end{itemize}

\subsection{Sawdust Water Currents}
\begin{itemize}
\item{Preparation Time: 1 minute}
\item{Materials: sawdust, water, beaker, heat source}
\item{Procedure: Fill the beaker with water and pour in a handful of sawdust so that the sawdust spreads out through the water. Heat the water over a candle or jiko and watch the sawdust cycle through the water from top to bottom.}
\item{Theory: As water is heated, it moves up, displacing the cooler water above it. In this way, heated water continually cycles. As water heats, it moves up and cools, whereby it is later displaced by newly heated water moving up. The sawdust follows the currents in the water so you can clearly see the cycle.}
\end{itemize}

\subsection{Specific Heats}
\begin{itemize}
\item{Preparation Time: 5 minutes}
\item{Materials: thermometer, water, any liquid, measuring cylinder, small pot, glass container or jar, heat source}
\item{Procedure: Measure a known volume of the liquid into a glass container. Heat the water in the pot over a jiko or stove until it is significantly warmer than the other liquid. Measure the volume of the water in the measuring cylinder. Before mixing the liquids, measure each temperature and record it. Now pour the hot water into the other liquid and wait for the temperature of the mixture to equalize. When the temperature levels off, measure and record it. With this data, you can calculate the specific heat capacity of the liquid.}
\item{Theory: Specific heat capacity is given by where H is the heat needed to raise a mass m a temperature T. Since the liquid and water are being mixed, the same amount of energy used to raise the liquid’s temperature is lost by the water to cool it down. We can set the heat of water Hw equal to the heat of the liquid Hl. The masses of the substances are known (by using the mole equations), and you measured the changes in temperature with the thermometer. The specific heat capacity of water is 4200 J/kgK, so you can solve for the specific heat capacity of the liquid.}
\end{itemize}

\subsection{Expansion of Liquids: Moving Colors}
\begin{itemize}
\item{Preparation Time: 15 minutes}
\item{Materials: 0.5 liter water bottle, water, food coloring, metal pot, heat source, straw, knife, glue}
\item{Procedure: Cut a small hole in the bottle cap for the straw to fit through. Insert the straw most of the way and glue it so that it is airtight. Fill the bottle most of the way with water and add some food coloring, then close the cap. Place the bottle into a metal pot with more water and heat the metal pot. As the temperature increases, the level of colored water in the straw will increase. When the level is high enough and the students can see clearly what is happening, take the bottle out of the metal pot and dip it in cold water. Watch the level in the straw drop quickly!}
\item{Theory: As with solids and gases, liquids expand when heated. In a sealed bottle with a straw, the liquid must expand as the temperature increases and can only move up the straw.}
\end{itemize}

\subsection{Expansion of Gases: Oil Elevator}
\begin{itemize}
\item{Preparation Time: 15 minutes}
\item{Materials: bottle with cap or flask with stopper, cooking oil, straw, glue}
\item{Procedure: Create the same bottle-straw configuration as in the Expansion of Liquids demo, or use a flask with a stopper with a glass tube if you have it. Close the cap tightly and carefully pour a drop of oil into the top of the straw. The oil should stop in the straw before reaching the bottom. Heat the bottle a little and watch the oil drop move back up the straw. If you have a glass flask, you should be able to heat it with just your hands.}
\item{Theory: Gases respond much more to heat than solids or liquids, and will expand noticeably with even a small amount of heat. By slightly heating the bottle, you raise the temperature, and therefore the pressure, of the air inside. As the pressure increases, the air pushes the oil up the straw.}
\end{itemize}

\subsection{Expansion of Solids: Screw and Loop}
\begin{itemize}
\item{Preparation Time: 5 minutes}
\item{Materials: screw, length of thick-ish metal wire about 10 cm, heat source, tongs}
\item{Procedure: Bend the wire into a loop such that the head of the screw will just fit through. Demonstrate that the screw fits, then heat the screw in a candle or jiko for a minute or so. Try to pass the screw through the loop again; demonstrate that it no longer fits! Try alternately heating the loop and screw, then let them cool and see if the screw fits again.}
\item{Theory: Metals expand when heated. When you heat the screw, the diameter of the head increases slightly. It is not noticeable to the naked eye, but it becomes obvious when the screw becomes too large to fit into the loop.}
\end{itemize}

\section{Uniform Circular Motion}

\subsection{Conical Pendulum}
\begin{itemize}
\item{Preparation time: 1 minute}
\item{Materials: string (about 0.5 m), small weight (e.g. pendulum bob, rock)}
\item{Procedure: Tie the weight to one end of the string. Hold the other end of the string and gently swing the weight in a horizontal circle.}
\item{Theory: This can help students to see what is meant by the angle beta, to show that the length of the string is different from the radius of the circle, and to help them think about the forces involved—that there is only tension and gravity, but that the tension has a vertical component equal and opposite to gravity, and a horizontal component that provides the centripetal acceleration.}
\end{itemize}

\subsection{Bicycle Turn}
\begin{itemize}
\item{Preparation time: none}
\item{Materials: bicycle}
\item{Procedure: Ride the bicycle in a circle by leaning in on direction rather than turning the handlebars.}
\item{Theory: It is possible to turn a bicycle just by leaning in one direction. When leaning, the normal force has two components: the vertical component is equal and opposite of the gravitational force and the horizontal component provides a centripetal acceleration, causing the change in direction of velocity: you go in a circle.}
\end{itemize}

\section{Waves}

\subsection{Ripple tank}
\begin{itemize}
\item{Preparation time: 1-2 hours}
\item{Materials: pane of glass (40 cm x 60 cm is average), wood frame, caulk or some sealant, straight-edge, pen, lamp or torch, white paper.}
\item{Construction: Make the ripple tank itself using an old window in its frame, or have a craftsman make a sealed window to your specifications (bigger is better, but use what is available). Seal the glass and frame with caulk or some equivalent so that water will not leak through it. Prop this shallow tank up on stands or books about a foot off the table; place the lamp above the tank facing straight down and the paper about 30 cm directly below the tank.}
\item{Procedure: Level the tank and fill it with water 5 mm deep. Turn on the lamp so that any variation in the water’s surface shows up as a shadow on the white paper underneath. Create plane waves using a ruler or circular waves using a pen. You can also attach a thin strip of metal so that half hangs over the tank and half outside. Attach a pen to the inside half and flick the metal strip. If the masses on either side of the tank frame are equal, the strip should oscillate fairly easily. Observe diffraction, reflection, etc. by placing objects in the tank. All phenomena will be visible on the paper and can be measured there.}
\item{Theory: Water waves, while different from sound or light waves, show the same properties of propagation, reflection, diffraction, etc. as any other wave. The varying depth of the water due to oscillating crests and troughs creates areas of bright and dark spots from the lamp above. Any behavior of the waves can be seen clearly and even measured: wavelength, frequency, and wave velocity.}
\end{itemize}

\subsection{String Waves}
\begin{itemize}
\item{Preparation time: none}
\item{Materials: at least 2 meters of string}
\item{Procedure: We can show several different types of waves using just a string. First, lay the string out along a table. By snapping your hand, you can create a transverse pulse wave that will travel along the length of the string. Now tie one end of the string to a rigid object and repeat. You should see the pulse reflect from the fixed end and come back towards you. Now wave the string up and down so as to create a standing wave. By changing the frequency with which you move your hand, you should be able to create at least two different modes.}
\end{itemize}

\section{Work}

\subsection{Work as Heat, Part A}
\begin{itemize}
\item{Preparation time: 5 minutes}
\item{Materials: thin strip of metal, pliers}
\item{Procedure: Take a piece of metal. Use a set of pliers to bend the metal back and forth. Feel the temperature of the metal.}
\item{Theory: Work can manifest itself in a variety of ways. One of the most common ways is the rise in temperature. By moving the metal back and forth, you are doing work on the metal. This work is converted into heat. This heat is evidenced by the rise in temperature in the metal. }
\end{itemize}

\subsection{Work as Heat, Part B}
\begin{itemize}
\item{Preparation time: 0 minutes}
\item{Materials: radio antennas, old or new }
\item{Procedure: The radio antennas operate in a telescopic motion. Pull the radio antenna in and out for one full minute. Do not break the antenna in this movement. Observe the temperature of the antenna after the work is over.}
\item{Theory: Again, you are doing work on the radio antenna by moving it in an out quickly. Through this action, the antenna heats up. This is the evidence of the work you have been doing. Work is defined as force times distance or. In this case, the force is the effort required to move the antenna in and out while the distance is the length of the antenna.}
\end{itemize}

\subsection{Work as Light}
\begin{itemize}
\item{Preparation time: 0 minutes}
\item{Materials: duct tape, or other tape that holds together tightly.}
\item{Procedure: Cut two pieces of duct tape. Press the ends of the bottom pieces of tape together but allow the top pieces of tape to be apart. Hold tightly to both pieces of tape at the top, and quickly rip them apart. Observe the blue light when the tape comes apart.}
\item{Theory: Pulling the tape apart quickly creates a faint blue light. It is best to observe this light at night since it is so faint. In this activity, this is the work being done to pull the tape apart. Unlike the previous activities, this work is released as light. This phenomenon as where work manifests itself as light is called triboluminescence. This is the same phenomenon that causes the green light when snapping wintergreen mints.}
\end{itemize}


% That's it!
\end{document}
