\chapter{Recycling organic waste}

Organic chemicals are often expensive 
to purchase and difficult to dispose. 
Every effort should be made to collect organic wastes and recycle them. 
For the purpose of this discussion, 
organic chemicals are liquids insoluble in water, 
e.g. 
kerosene, 
ether, 
ethyl ethanoate (ethyl acetate), 
etc.

Mixtures of multiple organic wastes 
require fractional distillation to separate. 
This is difficult and dangerous without the right equipment. 
Generally, 
if none of the organic chemicals in the mixture are particularly dangerous – 
see the section on Dangerous Chemical – 
it is best to label the mixture “mixed organic solvents, 
does not contain benzene or chlorinated hydrocarbons” 
and keep it for future use as a generic solvent or for solubility activities.

Mixtures of a single organic waste and water are inherently unstable, 
and given enough time will separate out into two layers. 
If the organic layer is on the bottom, 
it is probably di-, 
tri-, 
or tetrachloromethane, 
all dangerous chemicals. 
Follow the instructions in Dangerous Chemicals. 
If the organic layer is on the top, 
simply decant it off. 
You might do this in two steps – 
the first to separate only water from organic mixed with some water, 
and the second to separate from the latter fraction pure organic 
from a small volume that remains a mixture. 
Then the water can be discarded, 
the organic saved, 
and the small residual mixture left open to the air to evaporate.

Often, 
mixtures of organic and aqueous waste 
contain a solute dissolved in both solvents. 
The solvent is said to be distributed 
or partitioned between these two layers. 
Examples of compounds that partition between an organic 
and an aqueous layer are organic acids, 
like ethanoic acid (acetic acid) and succinic acid, 
and iodine when the aqueous layer is rich in iodide 
(usually potassium iodide). 
To reuse the organic layer it is necessary to first remove the solute.

If the solute is an organic acid, 
add a small amount of indicator to the mixture 
and then sodium hydroxide solution, 
shaking vigorously from time to time. 
The sodium hydroxide will react with the organic acid 
in the aqueous layer, 
converting it to the salt. 
As the concentration of the acid in the aqueous layer decreases, 
the distribution equilibrium will ``push'' acid dissolved 
in the organic layer into the aqueous layer, 
where it too is converted to salt. 
Eventually, 
all the organic acid will be converted to its conjugate base salt, 
which is only soluble in the aqueous layer, 
and the indicator will show that the aqueous layer 
is alkaline even after much shaking. 
Now the organic layer may be run off as above.

If the solute is iodine, 
the organic layer should have a color due to the iodine, 
and thus it will be straightforward 
to know when the iodine is fully removed. 
If there is no color, 
add starch to give a black color to the aqueous layer. 
Then add ascorbic acid (crushed vitamin C tablets) 
to the mixture and shake vigorously 
until either the organic layer returns to its normal color 
or the starch-blackened aqueous layer turns colorless. 
At this point all of the iodine will have been reduced to iodide, 
soluble only in the aqueous layer. 
The clean organic layer may be run off as above. 
Sodium thiosulfate may be used instead of ascorbic acid.

If you require the final organic to be of quite high purity, 
repeat the treatment. 
A small amount of residual water 
may also be removed with use of a drying agent, 
such as anhydrous sodium sulfate or calcium chloride.

