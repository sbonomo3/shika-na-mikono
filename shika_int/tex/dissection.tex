\chapter{Dissection}

%==============================================================================
\section{Preparation of Specimens}

Unless you want students to observe a beating heart, 
dead specimens are much easier to work with than unconscious ones. 
This also removes the problem of stunned animals waking up 
in the middle of their dissection.

\begin{itemize}

\item{Flowers and other plant parts: 
No preparation required as long as the samples are relatively fresh. 
Store samples in closed plastic bags to minimize drying. 
If you intend to keep them for more than a day or two, 
submerge the bags in cold water to slow the rate of molding.}

\item{Insects: Kill with household aerosol insecticide. 
Use specimens within one day of collection, 
unless you have refrigeration or freezer.}

\item{Fish: Keep living until the day of the dissection. 
Then remove from water until they suffocate. 
Use immediately after death.}

\item{Frogs: Able to breathe above and below water, 
frogs are hard to starve of oxygen. 
One option is to seal them in a container of methylated spirits 
and then rinse the dead bodies with water prior to dissection.}

\item{Reptiles, birds, and mammals: For most organ systems, 
you can kill the animal by blunt trauma without ruining the lesson. 
Students can even bring animals caught and killed in homes. 
Snakes should be decapitated along with enough of the body 
to remove the fangs and venom sacks. 
Bury these deeply. 
Do not use animals killed by poison, 
or those that were found dead. 
For completely undamaged specimens, 
enclose the live animal in a cage (or a tin with adequate holes) 
and submerge in a bucket of water until drowned.}

\item{Living specimens: 
If you really want to see that heart beating, use chloroform. 
This can be transferred from bottle to specimen jar via cotton ball, 
or perhaps made in situ by the reaction between propanone (acteone) and bleach. 
We have not yet attempted the latter – 
if you do, remember that the products are poisonous gases; 
indeed, that is the point. Note that if you use too little chloroform, 
the animal will feel the blade opening it up. 
If you use way too little, it may start squirming. 
If you use too much chloroform, however, you will simply kill the animal – 
you might as well have drowned it.}

\end{itemize}

%==============================================================================
\section{Tools}

\begin{itemize}

\item{Scalpels are the best tool for this job. 
They are very sharp and deliver a great deal of force to their point. 
This allows students to made clean cuts with minimal pressure. 
The next best things are homemade scalpels, 
razor blades attached to a handle to ensure a firm command of the blade. 
If the blade is dull or floppy, the students will probably push too hard, 
and may cut themselves when the skin finally gives and the blade slips.}

\item{Pins should be sharp and strong. 
Unused needles from new disposable syringes are an easy option.}

\item{Dissection trays can be prepared 
by making a 1cm thick layer of wax on the bottom of a shallow tray or bowl. 
This surface will readily accept pins and is easy to clean.}

\end{itemize}

%==============================================================================
\section{Procedure}

This varies by species. The internet has many resources and there are many good books with very detailed instructions – alas, this manual is not yet one of them. A crude method follows:

Position the specimen on its back and make a clean, symmetric, and shallow incision down the full length of the underside. Make additional perpendicular cuts at the top and bottom of the torso for an overall “I” shape. These cuts should only just penetrate the body cavity. Open up skin “doors” you have created, pinning them back onto the dissection tray. Pick an organ system – circulation, digestion, nervous, etc – and, with the aid perhaps of a good drawing, remove other material to focus on the target anatomy. You can teach many systems from one specimen – start with the most ventral (front) and move to the most dorsal (back).

Encourage students to sketch at various steps in the process. Also encourage them to identify anatomy for themselves, perhaps with the aid of thought provoking questions and discussion in groups.

%==============================================================================
\section{Cleanup and Carcass Disposal}

Wash all blades, pins, and trays with soapy water. 
Rinse all tools to remove the soap 
and then soak for about fifteen minutes in bleach water. 
When finished, rinse again in ordinary water.

Bury all carcasses in a deep pit, below the reach of dogs. 
You may also add kerosene and burn, 
but this smells bad and costs money.
