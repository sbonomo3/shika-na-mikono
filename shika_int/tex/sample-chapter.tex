% This is a sample chapter file.
% Note that lines beginning with "%" are comments.
% Comments have no effect on the code.
% So we can say whatever and it will have no effect.

% We start by declaring a chapter, named Example
\chapter{Example}

% In our chapter, there might be sections. Here is the first, called Introduction
\section{Introduction}

% The Introduction section has two sub-sections, Alpha and Beta.

\subsection{Alpha}

\subsection{Beta}

% The is no text in sub section Alpha, but in Beta let us say something.
% If we want words, we just type them.

Blah blah blah

% We can add a return in the middle of the sentence, and it will not change anything

% In the final document, the text below looks exactly the same as the text above.
Blah blah 
blah

% Often, we want to make a list. There are two kinds of lists we use.
% First, we could simply itemize things, like a bulleted list:
\begin{itemize}

% And then we declare each item.
\item{This is the first item. Note that the item doesn't end
no matter how many return breaks we add 

until there is a curly bracket to close it --> }

\item{This is another item}

% If our itemizing has reached an end, we should say so:
\end{itemize]

Some text to break things up.

% We can also enumerate our items, giving them a 1, 2, 3...
\begin{enumerate}

\item{This is the first item}

\item{Remember to close everything with curly brackets}

\item{And don't forger to declare the end of your enumeration, when reached}

\end{enumerate}

% This is enough for most of Shika! Some more advanced things follow...
