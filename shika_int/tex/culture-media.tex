\chapter{Preparation of Culture Media}

\section{Introduction}

In microbiology, there are two basic types of media: solid agar media and a liquid broth media. From these, many types of media can be made. Generally, exact amounts of ingredients are not needed so if you want to make some agar plates or liquid cultures try with the resources you have. The recipes listed are a guideline to help you get started.

%==============================================================================
\section{Media Recipes}

%------------------------------------------------------------------------------
\subsection{Basic Agar (1.5\%)}

\begin{itemize}

\item{15 g/L agar\\
it is like gelatin or if you can find seaweed you can grind it up}

\item{10 g/L nutrient source\\
e.g.sugar, starch (potatoes), beans fruits like mango and papaya}

\item{1-2 g/L salts and phosphates\\
this varies with what you want to grow — experiment! (table salt is usually fine)}

\item{1 L water}

\end{itemize}

Add and mix all the ingredients together and heat until boiling. Boil for \~15 minutes and make sure all the gelatin/agar is dissolved. Pour liquid into Petri plates (15-20~mL each). The plates should solidify \~45°C. Cover and keep agar side up in a cool place if possible. If the plates do not solidify, try adding more gelatin or corn starch to thicken it up. You can also pour agar into test tubes/syringes to do oxygen tests (aerobic vs. anaerobic)

%------------------------------------------------------------------------------
\subsection{Blood Agar}

\begin{itemize}

\item{15 g/L agar/gelatin/ground sea weed}

\item{10 g/L nutrient source}

\item{15 mL sheep’s blood (other organizisms also work)}

\item{1 L water}

\end{itemize}

Heat and boil agar, nutrient source and water for 15 minutes. After liquid has ceded (\~45°C (when you can leave your hand on the flask for a few seconds) add in blood until the mixture is blood red. Swirl in and pour into plates.

%------------------------------------------------------------------------------
\subsection{Liquid Broths}

\begin{itemize}

\item{10~g nutrient source}

\item{1~L water}

\item{1-2~g salts/phosphates}

Mix together, heat, and boil. Distribute in test tubes.

\end{itemize}

%==============================================================================
\section{Things you can do after media preparation}

\begin{itemize}

\item{Agar-streaked plates! Swab something (back of throat, nose, belly button, door handle, etc) and gently rub onto the agar. Try not to gouge the agar.}

\item{You can also do experiments to test the effects of salt concentrations, temperature, and nutrient concentrations.}

\item{After all the plates solidify, incubate them at around 25-30C. Ideally the temperature remains constant. Check the plates after 24 hours for growth.}

\item{For liquid broths you can inoculate test tubes with a sample from the environment. Incubate and check like agar plates. If there is growth the liquid will be turbid instead of clear like a control tube with only broth.}

\item{You can use liquid cultures for wet mounts under microscopes as samples for agar plates or to allow students to see the difference between growth and no growth.}

\end{itemize}

%==============================================================================
\section{What to use if you do not have plates or test tubes}

\begin{itemize}

\item{Use old water bottles or old plastic packaging for plates}

\item{Use anything rigid and heavy for covered, e.g. building tiles}

\item{Sealed/closed plastic syringes for test tubes}

\item{Try to keep materials as sterile as possible but do not worry if there is contamination. Use contamination as a learning experience. Penicillin was contamination and it became a wonder drug.}

\end{itemize}

\section{Things to do once you have cultures}

\begin{itemize}

\item{Take a sample from agar plate and drop hydrogen peroxide on it. Does it bubble? (Yes, it has catalase)}

\item{Extract DNA from \textit{E. coli.}}

\item{Fermentation = use a liquid broth with peptone, acid-base indicator like phenol red, and inverted tube to trap gas and 0.5 – 1.0\% of carbohydrate you want to test. If fermentation occurs (phenol red), the broth will turn yellow and gas should be collected in the tube. If the tube remains red, you can test for glucose production by adding a few drops of methyl orange. If the pH is below 4.4, it will remain red. If the pH is above 6.0, it will turn yellow.}

\end{itemize}

%==============================================================================
\section{Guide to Identifying Common Microorganisms}

\begin{itemize}

\item{\textit{Pseudomonas aeroginosa}: is green and smells like grape jelly (can grow in disinfectant)}

\item{\textit{Serratia marcescens}: grows pink-red between 25-32°C (will be white otherwise)}

\item{\textit{Escherichia coli}: pale white/yellow, smells like inole}

\item{\textit{Proteus spp}: swarm on plates and smell like urine and brownies}

\item{\textit{Bacillus subtillis}: pale beige, smells a bit sweet}

\item{\textit{Vibrio cholera}: smells like buttery popcorn}

\item{Staph vs Strep: Staph is catalase (+), strep is (-)}

\end{itemize}
