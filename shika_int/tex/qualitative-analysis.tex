\chapter{Qualitative Analysis}
\label{cha:qualana}
Of all branches of chemistry studied in secondary schools, 
qualitative analysis bears the most resemblance to alchemy. 
Students receive a salt, 
a source of heat, 
a variety of dangerous chemicals, 
and sometimes a few pages of paper listing 
qualitative analysis procedures. 
Students then follow the procedures to identify their salt. 
There is great potential for both harm and mess. 
At the same time, 
students have an opportunity to independently perform 
a large variety of chemical reactions, 
and do observe some remarkable transformations.

For the chemistry teacher or lab manager, 
qualitative analysis is a useful process for identifying unknown salts, 
thus permitting their use. 
The process is time consuming and requires a wide variety of reagents. 
Many, 
however, 
are common in chemistry labs, 
and many can be found from local sources.

This section contains the following:
\begin{itemize}
\item{Overview of qualitative analysis}
\item{Description of each step}
\item{List of bench reagents with recommended concentrations}
\item{Hazards and cleanliness}
\item{Qualitative analysis with limited resources}
\end{itemize}

%==============================================================================
\section{Overview of Qualitative Analysis}
The goal of qualitative analysis is to identify an unknown salt, 
both the cation and the anion. 
Generally, 
these are identified separately 
although often knowing one helps interpret 
the results of tests for the other. 
This guide is written for the requirements of 
ordinary level qualitative analysis in Tanzania -- 
the chemistry applies everywhere 
although the process may have to be adapted 
to different expectations and requirements. 
In Tanzania, 
students are confronted with binary salts made from the following ions:
\begin{itemize}
\item{Cations: NH$_{4}^{+}$, 
Ca$_{2}^{+}$, 
Fe$_{2}^{+}$, 
Fe$_{3}^{+}$, 
Cu$_{2}^{+}$, 
Zn$_{2}^{+}$, 
Pb$_{2}^{+}$, 
Na$^{+}$}
\item{Anions: CO$_{3}^{2-}$, 
HCO$_{3}^{-}$, 
NO$_{3}^{-}$, 
SO$_{4}^{2-}$, 
Cl$^{-}$}
\end{itemize}
At present, 
ordinary level students receive only one salt at a time. 
Advanced level students may receive mixtures.

The ions are identified by following a series of ten steps, 
divided into three stages. 
These are:
\begin{itemize}
\item{Preliminary tests:
These tests use the solid salt. 
They are: appearance, 
flame test, 
action of heat, 
action of dilute H$_{2}$SO$_{4}$, 
action of concentrated H$_{2}$SO$_{4}$, 
and solubility.}
\item{Tests in solution:
The compound should be dissolved in water before carrying out these tests. 
If it is not soluble in water, 
use dilute HNO$_{3}$ to dissolve the compound. 
The tests in solution involve addition of NaOH and NH$_{3}$.}
\item{Confirmatory tests:
These tests confirm that the conclusions students draw from the previous steps. 
By the time your students start the confirmatory tests, 
they should have a good idea of what cation and anion are present. 
Have students do one confirmatory test for the cation they believe is present, 
and one for the anion you believe is present. 
Even if several confirmatory tests are listed, 
students only need to do one. 
When identifying an unlabelled container, 
however, 
you might be moved to try several, 
especially if you are new to this process.}
\end{itemize}

%==============================================================================
\section{Description of Each Step}
\subsection{Appearance}
Students observe three properties: the color of the salt, 
the physical form of the salt (whether it is powdery or crystalline), 
and the salt’s smell. 
The color shows whether a transition metal such as copper 
or iron is present. 
A powdery salt contains carbonate or hydrogen carbonate; 
a crystalline salt does not. 
Students should not touch the salt; 
they can tell if it is powder or crystalline with their eyes. 
A salt with a sharp smell like stale urine contains ammonium ions. 
Not all ammonium salts have a strong smell. 
Students should not inhale the salt or put it near their nose. 
If an ammonia smell is present, 
the smell will be obvious.

\subsection{Flame test}
A solid sample of the salt is placed in a non-luminous flame. 
The best flames are, 
in order, 
a bunsen burner, 
an alcohol infused heavy oil burner 
(see \nameref{sec:heatsources} in \nameref{cha:labequip}), 
a spirit burner, 
a butane lighter, 
and a kerosene stove without the outer cover 
to allow access to the blue flame. 
Some salts produce a characteristic flame color: 
copper produces blue or green flames and calcium produces brick red flames. 
Sodium and lead can be problematic: sodium produces a golden yellow flame, 
which can be hard to distinguish 
from the ordinary flame of some kerosene stoves, 
and the bluish-white flame of lead is a rare find at lower temperatures, 
though is obvious in a Bunsen burner.

\subsection{Action of heat}
The purpose of this test is to decompose the salt. 
Students should observe two things: the formation of gas or a residue. 
Damp pieces of blue and red litmus paper should be held 
over the test tube to test the acidity or basicity of any gas formed.

Gases can be problematic. 
Gases that are easy to identify are nitrogen dioxide 
(which is brown) and ammonia (which is basic). 
Most sulfates do not decompose to sulfur dioxide when heated, 
so students should not expect a gas if sulfate is present. 
Carbonates decompose, 
but carbon dioxide does not usually change the color of litmus paper. 
Chlorides do not decompose when heated. 
A test that produces no identifiable gas is inconclusive: a carbonate, 
sulfate, 
or chloride are possible anions. 
Do not teach the students to leave it blank and fill it in later – 
this is lying, 
and terrible science. 
Tell them to write what they see and what it means – no gas observed, 
so a sulfate of chloride may be present..

If the salt makes a crackling sound while heating, 
then either sodium chloride or lead (II) nitrate is present. 
None of the other salts used in qualitative analysis have this property.

\subsection{Action of dilute H$_{2}$SO$_{4}$}
This step tests for carbonates. 
The student should add a few drops of dilute sulfuric acid 
(0.5 -- 1~M) to the solid sample. 
If effervescence (bubbles) is observed, 
then carbonates or hydrogen carbonates are present. 
If there is no effervescence, 
then they are absent. 
To confirm that the gas is carbon dioxide, 
students can pass it through lime water 
(hard) or show that it will extinguish a glowing splint (easy).

\subsection{Action of concentrated H$_{2}$SO$_{4}$}
If no identifiable gas was formed due to the action of heat, 
the addition of concentrated H$_{2}$SO$_{4}$ will help students 
distinguish whether a chloride or sulfate is present. 
The student should just cover a small sample 
with concentrated sulfuric acid (5~M, 
or battery acid -- not the full 18~M sulfuric acid from a stock bottle. 
This is unnecessarily dangerous) and heat gently. 
Hold a piece of blue litmus above the mouth of the test tube. 
If no gas forms, 
then sulfate anion is present. 
If a colorless, 
acidic gas forms, 
the blue litmus will turn red. 
This means that HCl gas formed from the addition of concentrated sulfur acid. 
This means that the anion is chloride. 
Unfortunately, 
students want to see a gas, 
and will often boil the acid solution 
until the litmus paper changes color due to 
acidic fumes of the sulfuric acid rather than the decomposition of the salt. 
Advise them to remove the test tube from the heat before the acid boils -- 
if no gas has formed by this point, 
than it is unlikely that a gas will ever form. 
Further, 
boiling concentrated acids efficiently courts injury.

SAFETY: if the addition of dilute acid identified 
a carbonate or hydrogen carbonate, 
students should add very little concentrated acid -- 
just a drop or two. 
The addition of more concentrated sulfuric acid 
to a powder carbonate can cause such violent effervescence 
that concentrated acid is ejected from the test tube. 
This can blind.

\subsection{Solubility}
Students test whether the salt is soluble in water, 
and observe the color of the solution. 
If the salt is not soluble in cold water, 
they test its solubility in warm water by heating the mixture. 
Only two commonly used salts 
(lead chloride and calcium sulfate) are insoluble in cold water 
but soluble in warm water. 

Most qualitative analysis sheets have a table of solubility on the last page. 
Students can use this table to cross check their final results. 
For example, 
if students identified a soluble salt as calcium carbonate, 
then they made a mistake and should repeat some of their tests.

Better than tables, 
teach students basic rules of solubility:
All nitrates and hydrogen carbonates are soluble
All Group I (sodium, 
potassium, 
etc) and ammonium salts are soluble (sodium borate is an exception)
Most chlorides are soluble (silver and lead chlorides are exceptions, 
though the latter is soluble in hot water)
Carbonates of metals outside of Group I are generally insoluble 
(note that aluminum and iron carbonate do not exist)
Lead sulfate is insoluble and calcium sulfate is soluble only in hot water. 
Magnesium sulfate is completely soluble 
while sulfates of the Group II metals heavier than calcium 
(strontium and barium) are insoluble.

\subsection{Addition of NaOH solution}
NaOH solution is added to a solution of the salt. 
The purpose of this test is to identify the cation present. 
A small amount of NaOH solution should be added first 
(to identify the color of the precipitate), 
then a large amount should be added (to test if the precipitate 
forms a hydroxide complex soluble in excess sodium hydroxide solution). 
Blue precipitates indicate copper cations. 
Green precipitates suggest iron (II) 
and red/brown precipitates suggest iron (III). 
White precipitates insoluble in excess hydroxide are from calcium – 
this is often observed when hydroxide is added to ordinary hard water. 
White precipitates soluble in excess hydroxide are either zinc or lead. 
If no precipitate forms, 
the cation is probably sodium or ammonium.

\subsection{Addition of NH$_{3}$ solution}
NH$_{3}$ solution is added to a solution of the salt. 
The purpose of this test is to identify the cation. 
As with NaOH, 
a small amount of solution should be added first, 
followed by a large amount. 
The results similarly divide the cations into groups 
based on the presence and nature of the precipitate – 
importantly zinc produces a precipitate that dissolves in excess ammonia 
while the precipitate from lead remains. 
Note that NH$_{3}$ solution degrades over time 
due to evaporation of the ammonia. 
Store this reagent in plastic water bottles with screw-on caps 
and test the solution yourself before using it with students.

\subsection{Confirmatory tests}
Emphasize to students that they need to carry out 
only one confirmatory test for the cation, 
and one for the anion. 
If the test gives the expected result, 
then they can be sure that the ion they have identified is present. 
If the test does not give the expected result, 
they have probably made a mistake, 
and they should revisit the results of their previous tests 
and choose a different possibility to test.

%==============================================================================
\section{List of Bench Reagents}

The reagents required for most qualitative analysis practicals are:
\begin{itemize}
\item{dilute (about 0.5~M) sulfuric acid, 
H$_{2}$SO$_{4}$}
\item{concentrated (about 5~M) sulfuric acid, 
H$_{2}$SO$_{4}$}
\item{sodium hydroxide solution, 
NaOH}
\item{ammonia solution, 
ammonium hydroxide solution, 
NH$_{3}$}
\item{dilute nitric acid (about 1~M), 
HNO$_{3}$}
\end{itemize}

The reagents used for confirmatory tests are:
\begin{itemize}
\item{Ammonium ethandioate, 
ammonium oxalate, 
COO(NH$_{4}$)$_{2} \cdot$2H$_{2}$O}
\item{Ammonium thiocyanate, 
NH$_{4}$SCN, 
or potassium thiocyanate, 
KSCN}
\item{Barium chloride, 
BaCl$_{2}$, 
or barium nitrate, 
BaNO$_{3}$}
\item{Copper turnings, 
small pieces of copper metal}
\item{Iron (II) sulfate, 
FeSO$_{4}$}
\item{Magnesium sulfate, 
MgSO$_{4}$}
\item{Manganese (IV) oxide, 
manganese dioxide, 
MnO$_{2}$}
\item{Potassium chromate, 
KCrO$_{4}$}
\item{Potassium hexacyanoferrate (II), 
potassium ferrocyanide, 
K$_{4}$Fe(CN)$_{6} \cdot$3H$_{2}$O}
\item{Potassium hexacyanoferrate (III), 
potassium ferricyanide, 
K$_{3}$Fe(CN)$_{6}$}
\item{Potassium iodide, 
KI}
\item{Silver nitrate, 
AgNO$_{3}$}
\end{itemize}

Other reagents which may be required in qualitative analysis:
\begin{itemize}
\item{Ethanoic (acetic) acid (CH$_{3}$COOH).}\\
\textit{Glacial (concentrated) acetic acid 
is sometimes used in a confirmatory test for calcium.}
\item{Lead ethanoate (acetate) (Pb(CH$_{3}$COO)$_{2} \cdot$3H$_{2}$O)}\\
\textit{Sometimes used as a confirmatory test for sulfates.}
\end{itemize}
You do not need all the confirmatory test reagents for a single practical. 
At minimum, 
you only need the reagents required to confirm the ions 
that are present in that practical. 
However, 
it is a good idea to have the other solutions around -- 
students will often ask for them, 
and it is better to let the students realize their errors 
by carrying out these confirmatory tests 
than to simply tell them that they are wrong. 
In addition, 
the more clever students may infer the presence of a certain ion 
simply by looking at the solutions you have placed on their bench.

%==============================================================================	
\section{Hazards and Cleanliness}
Qualitative analysis practicals are full of hazards, 
from open flames to concentrated acids. 
To reduce the risk of accidents, 
teach students how to use their flame source 
before the day of the practical, 
especially if you are using Bunsen burners. 
Most students have never used gas before, 
and do not know the basic safety precautions involved in using gas. 
If you have choice about what salts are offered, 
do away with those requiring concentrated acid.

Teach students to hold their test tubes at an angle when they heat them. 
Test tubes should be pointed away from the student holding them 
and from other students. 
This will prevent injuries due to splashing chemicals, 
and will also minimize inhalation of any gases produced.

Teach students never to fill test tubes 
or any other container more than half. 
That way, 
they minimize spills and boiling over of chemicals during heating. 
In addition, 
this also prevents bumping in the test tubes 
(when a gas bubble forms suddenly), 
which can cause dangerous spray.

Teach students that if they get chemicals on their hands, 
they should wash them off immediately, 
without asking for permission first. 
Some students have been taught to wait for a teacher's permission 
before doing anything in the lab, 
even if concentrated acid is burning their hands. 
On the first day, 
give them permission to wash their hands 
if they ever spill chemicals on them. 
Also, 
teach students to tell you immediately when chemicals are spilled. 
Sometimes they hide chemical spills for fear of punishment. 
Do not punish them for spills -- legitimate accidents happen. 
Do punish them for unsafe behavior of any kind, 
even if it does not result in an accident. 
See the \nameref{sec:basiclabrules} in \nameref{cha:labsafety}.

Practicals involving nitrates, 
chlorides, 
ammonium compounds, 
and some sulfates produce harmful gases. 
Open the lab windows to maximize airflow. 
Kerosene stoves also produce noxious fumes -- 
all the more reason to consider an alternative flame source. 
If students feel dizzy or sick from the fumes, 
let them go outside to recover.

Make absolutely sure that students clean their tables 
and glassware before they leave. 
Walking into a lab sprinkled with mystery salts 
and test tubes of unlabelled liquids is annoying and possibly dangerous. 
Qualitative analysis experiments can leave residues in test tubes 
that are difficult to clean with brushes alone. 
To remove stubborn residues, 
pour a little dilute nitric acid into the test tube. 
The acid should dissolve the precipitate 
and leave a clean test tube behind. 
Remember that the utility of nitric acid -- 
that it will dissolve almost anything -- is also a serious hazard.

For additional advice, 
refer to the chapters on \nameref{cha:labsafety} and \nameref{cha:dangerchem}.

%==============================================================================
\section{Qualitative Analysis with Limited Resources}

Qualitative analysis can seem impossible for schools 
without a traditional laboratory. 
There are so many specific chemicals 
and so many complicated procedures. 
To the contrary, 
students can be well prepared for the national exam 
using very basic resources. 
This section offers some general suggestions as well as 
a specific procedure for teaching qualitative analysis.

\subsection{General Suggestions}
\begin{itemize}
\item{Heat sources – Alcohol infused heavy oil burners 
cost nothing to make and consume only a small amount of fuel. 
They give a non-luminous flame ideal for flame tests 
and still produce enough heat for the other tests. 
See \nameref{sec:heatsources} in \nameref{cha:labequip}.}
\item{Test tubes – Most of the tests do not involve heating, 
so students may perform these experiments in plastic tubes 
made from disposable plastic syringes. 
Many of the tests requiring heating use the salt in solution 
and thus can be performed 
by holding the plastic test tube in a hot water bath. 
For more information, 
read the entries for \nameref{sec:testtubes} 
and \nameref{sec:hotwaterbathes} in \nameref{cha:labequip}.
For the action of heat test, 
salts may be heating in metal spoons to observe residue products, 
although it is difficult to test the gases produced. 
Do not wait for test tubes to start teaching qualitative analysis. 
Just try to find at least one borosilicate (Pyrex, 
Borosil) test tube per student before the national exams.}
\item{Litmus paper – Make your own. 
See \nameref{sec:indicator} in \nameref{cha:sourcesofchemicals}.}
\item{Low cost sources of chemicals. 
Many chemicals have low cost alternatives. 
For any specific chemical, 
read its entry in \nameref{cha:sourcesofchemicals}.}
\item{Share expensive chemicals among many schools. 
A single container of potassium ferrocyanide, 
for example, 
can supply ten or even twenty schools for several years. 
Schools might consider bartering 10~g of one chemical for 10~g of another. 
Schools without any expensive chemicals could produce distilled water (see \nameref{sub:distillwater}) and exchange this for 10~g samples of expensive salts.}
\end{itemize} 
 
%==============================================================================
\section{Performing Qualitative Analysis in the Basic Chemistry Laboratory}

For each step in the qualitative analysis procedure, 
we recommend a procedure using low cost materials 
and offer low cost compounds for students to use to learn this step.

\subsection{Appearance}
\begin{itemize}
\item{Procedure: place salt on a piece of plastic cut from a water bottle}
\item{Example salts:}
\begin{itemize}
\item{White salts = sodium chloride, 
sodium carbonate, 
sodium hydrogen carbonate}
\item{Blue salts = copper sulfate}
\item{Yellow/Green salts = iron sulfate}
\item{Powder salts = sodium carbonate, 
sodium hydrogen carbonate}
\item{Crystal salts = sodium chloride, 
copper sulfate}
\item{Ammonia smell = calcium ammonium nitrate}
\end{itemize}
\item{Reagents: none}
\end{itemize}

\subsection{Flame test}
\begin{itemize}
\item{Procedure: clean an iron wire or metal spoon handle 
with steel wool and use it to hold a small amount 
over a non-luminous flame.}
\item{Example salts:}
\begin{itemize}
\item{blue/green flame = copper sulfate}
\item{golden yellow flame = sodium chloride, 
sodium carbonate, 
sodium hydrogen carbonate}
\item{brick red flame = calcium sulfate}
\item{no flame color = iron sulfate}
\item{crackling sound = sodium chloride}
\end{itemize}
\item{Reagents: none}
\end{itemize}

\subsection{Action of heat}
\begin{itemize}
\item{Procedure: place a small quantity of the salt 
in a metal spoon and heat over a flame}
\item{Example salts:} 
\begin{itemize}
\item{blue to white dehydration = copper sulfate}
\item{black residue = copper sulfate}
\item{yellow to white residue = zinc carbonate or zinc sulfate} 
\item{red to dark residue = iron sulfate}
\end{itemize}
\item{Reagents: none}
\end{itemize}

\subsection{Action of Dilute Acid} 
\begin{itemize}
\item{Procedure: place a small quantity of the salt in a test tube, 
plastic or glass, 
and add dilute sulfuric acid drop-wise}
\item{Example salts:}
\begin{itemize}
\item{effervescence = sodium carbonate, 
sodium hydrogen carbonate}
\item{no effect = sodium chloride, 
calcium sulfate}
\end{itemize}
\item{Reagents: prepare dilute sulfuric acid 
by adding one part battery acid to nine parts water (e.g. 
fill a 1.5~L bottle half full with ordinary water, 
add 150~mL of battery acid, 
and then fill the bottle the rest of the way 
with water.in 900~mL of water)}
\end{itemize}

\subsection{Action of Concentrated Acid}
\begin{itemize}
\item{Procedure: place a small quantity of the salt in a (plastic) test tube, 
add concentrated sulfuric until the sample is submerged, 
and heat in a water bath. 
Hold blue litmus over the tube while heating.}
\item{Example salts:}
\begin{itemize}
\item{acidic gas produced: sodium chloride}
\item{no gas produced: calcium sulfate}
\end{itemize}
\item{Reagents: use battery acid for concentrated sulfuric acid}
\end{itemize}

\subsection{Solubility}
\begin{itemize}
\item{Procedure: add a small quantity of the salt in a test tube 
and then fill it half way with distilled water. 
If the salt does not dissolve in cold water, 
heat in a water bath.}
\item{Example salts:}
\begin{itemize}
\item{Soluble in cold water to form a clear solution = sodium chloride}
\item{Soluble in cold water to form a blue solution = copper sulfate}
\item{Soluble in cold water to form a solution of another color = iron sulfate}
\item{Soluble only in hot water = calcium sulfate} 
\item{Insoluble in water = copper carbonate}
\end{itemize}
\item{Reagents: rain water works as distilled water. 
If no rain water is available, 
tap water may be distilled with a tea kettle, 
rubber tubing, 
and some water bottles or a bucket.}
\end{itemize}

\subsection{Making a Solution}

To make a solution, 
dilute nitric acid is required for lead carbonate, 
lead sulfate and lead chloride 
while dilute hydrochloric acid will suffice 
for calcium carbonate and calcium sulfate. 
See the entry for \nameref{sec:hydroacid} in \nameref{cha:sourcesofchemicals}. 
Dilute battery acid will suffice for all other insoluble salts. 

\subsection{Action of sodium hydroxide solution}
\begin{itemize}
\item{Procedure: add about 1~mL of a solution of the salt to a test tube, 
plastic or glass, 
and then add sodium hydroxide solution drop wise.}
\item{Example salts:}
\begin{itemize}
\item{no effect = sodium chloride, 
sodium carbonate}
\item{blue precipitate, 
insoluble in excess = copper sulfate}
\item{red precipitate, 
insoluble in excess = iron sulfate}
\item{white precipitate insoluble in excess = calcium sulfate}
\item{white precipitate, 
soluble in excess NaOH = zinc sulfate, 
zinc carbonate}
\end{itemize}
\item{Reagents: to prepare 1~M sodium hydroxide solution, 
dissolve 4~g of caustic soda in 100~mL of distilled water.}
\end{itemize}

\subsection{Action of Ammonia Solution}
In theory, 
ammonia may be prepared, 
with difficulty, 
by mixing sodium hydroxide (caustic soda) 
and calcium ammonium nitrate (a fertilizer) in solution 
and distilling the mixture. 
A liquid rich in dissolved ammonia will be produced 
as the distillate (ammonia liquor). 
This is potentially dangerous, 
especially if the connections are not well sealed. 
Pass exit gases through water to prevent filling the room 
with ammonia fumes. 
Bear in mind also that ammonia is corrosive, 
so things that start well sealed might not remain so. 

Fortunately, 
the procedure and results of this step are similar 
to that of adding sodium hydroxide. 
Therefore, 
students can learn about this technique 
and how to recognize results without the use of ammonia. 
Then the school can find a small quantity for mock and the national exam.
Note: a small quantity can be used for many tests 
if students learn to use a small quantity for each test!

\subsection{Confirmatory Tests for the Cation}

\subsubsection{Ammonium}
\begin{itemize}
\item{Example salt: calcium ammonium nitrate}
\item{Procedure: add sodium hydroxide solution and heat in a water bath}
\item{Confirming result: smell of ammonia} 
\item{Reagents: NaOH solution as used in Step 7 above}
\end{itemize}

\subsubsection{Calcium}
\begin{itemize}
\item{Example salt: calcium sulfate}

\item{Procedure: Two options
\begin{enumerate}
\item{flame test}
\item{addition of NaOH solution}
\end{enumerate}
} % Procedure

\item{Confirming results:
\begin{enumerate}
\item{brick red flame}
\item{white precipitate insoluble in excess}
\end{enumerate}
} % Confirming results

\item{Reagents:
\begin{enumerate}
\item{none}
\item{NaOH solution from Step 7}
\end{enumerate}
} % Reagents

\end{itemize} % Calcium

\subsubsection{Copper}
\begin{itemize}
\item{Example salt: copper sulfate}
\item{Procedure: flame test}
\item{Confirming result: blue/green flame}
\item{Reagents: none}
\end{itemize}

\subsubsection{Iron (II)}
\begin{itemize}
\item{Example salt: locally manufactured iron sulfate 
(kept away from water and air)}
\item{Procedure: addition of sodium hydroxide solution 
and then transfer of precipitate to the table surface}
\item{Confirming result: green precipitate 
that oxidizes to brown when exposed to air}
\item{Reagent: sodium hydroxide solution from Step 7}
\end{itemize}

\subsubsection{Iron (III)}
\begin{itemize}
\item{Example salt: locally manufactured iron sulfate 
(oxidized by water and air)}
\item{Procedure: addition of sodium ethanoate solution}
\item{Confirming result: blood red solution}
\item{Reagent: see \nameref{sec:sodiumeth} in \nameref{cha:sourcesofchemicals}}
\end{itemize}

\subsubsection{Lead}
\begin{itemize}
\item{Example salt: no local sources for safe manufacture, 
consider purchasing lead nitrate}

\item{Procedure: Three options
\begin{enumerate}
\item{flame test} 
\item{addition of dilute sulfuric acid}
\item{addition of potassium iodide solution}
\end{enumerate}
} % Procedure

\item{Confirming results:
\begin{enumerate}
\item{blue/white flame}
\item{white precipitate}
\item{yellow precipite that dissolves when heated and reforms when cold}
\end{enumerate}
} % Confirming results

\item{Reagents:
\begin{enumerate}
\item{none but a very hot flame, e.g. Bunsen burner, is required} 
\item{dilute sulfuric acid used in Step 5 above}
\item{see \nameref{sec:potiodide} in \nameref{cha:sourcesofchemicals}}
\end{enumerate}
} % Reagents

\end{itemize} % Lead

\subsubsection{Sodium}
\begin{itemize}
\item{Example salt: sodium chloride, 
sodium carbonate, 
sodium hydrogen carbonate}
\item{Procedure: flame test}
\item{Confirming result: golden yellow flame}
\item{Reagents: none}
\end{itemize}

\subsubsection{Zinc}
\begin{itemize}
\item{Example salt: locally manugactured zinc carbonate 
or zinc sulfate}
\item{Procedure: addition of potassium ferrocyanide solution}
\item{Confirming result: gelatinous gray precipitate}
\item{Reagents: no local source of potassium ferrocyanide -- 
consider collaborating with many schools to share a container; 
only a very small quantity is required}
\end{itemize}

\subsection{Confirmatory Tests for the Anion} 

\subsubsection{Hydrogen carbonate}
\begin{itemize}
\item{Example salt: sodium hydrogen carbonate}
\item{Procedure: add magnesium sulfate solution 
and then boil in a water bath}
\item{Confirming result: white precipitate forms only after boiling}
\item{Reagent: dissolve Epsom salts in distilled water 
(see \nameref{sec:magsulfate} in \nameref{cha:sourcesofchemicals})}
\end{itemize}

\subsubsection{Carbonate}
\begin{itemize}
\item{Example salt: sodium carbonate}
\item{Procedure for soluble salts: addition of magnesium sulfate solution}
\item{Confirming result: white precipitate forming in cold solution}
\item{Reagent: dissolve Epsom salts in distilled water 
(see \nameref{sec:magsulfate} in \nameref{cha:sourcesofchemicals})}
Note that insoluble salts that effervesce 
with dilute acid are likely carbonates. 
None of the other anions described here produce gas with dilute acid 
(all hydrogen carbonates are soluble).
\end{itemize}

\subsubsection{Chloride}
\begin{itemize}
\item{Example salt: sodium chloride}
\item{Procedure: Three Options
\begin{enumerate}
\item{addition of silver nitrate solution}
\item{addition of manganese IV oxide and concentrated sulfuric acid 
followed by heating in a water bath}
\item{addition of weak acidified potassium permangante solution 
followed by heating in a water bath}
\end{enumerate}
} % Procedure
\item{Confirming results:
\begin{enumerate}
\item{white precipitate of silver chloride} 
\item{production of chlorine gas that bleaches litmus} 
\item{decolorization of permanganate}
\end{enumerate}
} % Confirming results
\item{Reagents:
\begin{enumerate}
\item{silver nitrate has no local source 
but may be shared among many schools as only a small amount is required; 
also see \nameref{cha:recyclesilver}.}
\item{Manganese dioxide may be purified from used batteries 
(see \nameref{cha:sourcesofchemicals}) and battery acid is concentrated sulfuric acid. 
This method is useful because of its low cost, 
but remember that chlorine gas is poisonous! 
Students should use very little sample salt in this test.}
\item{Prepare a solution of potassium permanganate, 
dilute with distilled water until the color is light pink, 
and then add about 1 percent of the solution's volume in battery acid. 
Note that this solution will cause lead to precipitate, 
and will also be decolorized by iron II, 
so it is not a perfect substitute for silver nitrate. 
This final option may not be recognized by examination boards.}
\end{enumerate}
} % Reagents
\end{itemize}

\subsubsection{Sulfate}
\begin{itemize}
\item{Example salt: copper sulfate, 
calcium sulfate, 
iron sulfate}
\item{Procedure: addition of a few drops of a solution of lead nitrate, 
barium nitrate, 
or barium chloride}
\item{Confirming result: white precipitate}
\item{Reagents: none of these chemicals have local sources. 
Because lead nitrate is also an example salt, 
it is the most useful and the best to buy. 
The ideal strategy is to share one of these chemicals among many schools. 
Remember that all are toxic.}
\end{itemize}
