\chapter{Substituting Chemicals in Volumetric Analysis}
\label{cha:subchemvolana}
\section{Theory}

The volumetric analysis practical exercises sometimes call for expensive chemicals, for example potassium hydroxide or oxalic acid. As the purpose of exercises and exams is to train or test the ability of the students and not the resources of the school, it is possible to use different chemicals as long as the solutions are calibrated to give equivalent results. For example, if the instructions call for a potassium hydroxide solution, you can use sodium hydroxide to prepare this solution. It will not affect the results of the practical -- if you make the correct calibration. How to calibrate solutions when substituting chemicals is the subject of this section.

Technically, only two chemicals are required to perform any volumetric analysis practical: one strong acid and one strong base. The least expensive options are sulfuric acid, as battery acid, and sodium hydroxide, as caustic soda. To substitute one chemical for another in volumetric analysis, the resulting solution must have the same normality (N).

\begin{itemize}

\item{For all monoprotic acids (HCl, ethanoic acid), the normality is the molarity.\\
\textit{Example: 0.1~M ethanoic acid = 0.1~N ethanoic acid}}
\item{For diprotic acids (sulfuric acid, ethandiotic acid), the normality is twice the molarity, because each molecule of diprotic acid brings two molecules of $\mathrm{H}^{+}$.\\
\textit{Example: 0.5~M sulfuric acid = 1.0~N sulfuric acid}}
\item{For the hydroxides and hydrogen carbonates used in ordinary level (NaOH, KOH, NaHCO$_{3}$), the normality is the molarity.\\
\textit{Example: 0.08~M KOH = 0.08~N KOH}}
\item{For the carbonates most commonly used ($\mathrm{Na}_2\mathrm{CO}_3$, $\mathrm{Na}_2\mathrm{CO}_3 /dot 10\mathrm{H}_2\mathrm{O}$, $\mathrm{K}_2\mathrm{CO}_3$), the normality is twice the molarity.\\
\textit{Example: $0.4 M \mathrm{Na}_2\mathrm{CO}_3 = 0.8 N \mathrm{Na}_2\mathrm{CO}_3$}}

\end{itemize}

\section{Substitution Calculations}

When instructions describe solutions in terms of molarity, calculating the molarity of the substitution is relatively simple. For example, suppose we want to use sulfuric acid to make a 0.2~M solution of ethanoic acid. 0.2~M ethanoic acid is 0.2~N ethanoic acid which will titrate the same as 0.2~N sulfuric acid. 0.2 N sulfuric acid is 0.1~M sulfuric acid, and thus we need to prepare 0.1~M sulfuric acid.

When instructions describe solutions in terms of concentration ($^g/_L$), we just need to add an extra conversion step. For example, suppose we want to use sodium hydroxide to make a $14.3 ^g/_L$ solution of sodium carbonate decahydrate. $14.3 ^g/_L$ sodium carboante decahydrate is 0.05~M sodium carbonate decahydrate which is 0.1~N sodium carbonate decahydrate. This will titrate the same as 0.1~N sodium hydroxide, which is 0.1~M sodium hydroxide or $4 ^g/_L$ sodium hydroxide, and thus we need to prepare $4 ^g/_L$ sodium hydroxide to have a solution that will titrate identically to $14.3 ^g/_L$ sodium carbonate decahydrate.

\section{Common Substitutions}
\label{sec:commonsubs}
To simplify future calculations, we have prepared general conversions for the most common chemicals used in volumetric analysis. Remember to check all final solutions with relative standardization to ensure that they indeed give the correct results.

\begin{center}
\begin{tabular}{| p{2cm} | p{2cm} | p{5cm} | p{5cm} | p{5cm} |}
\hline

\textbf{Required Chemical} & 
\textbf{Low Cost Alternative} & 
\textbf{Substiution Method} & 
\textbf{Molarity Example} & 
\textbf{Concentration Example} \\ \hline

Hydrochloric Acid & 
Sulfuric Acid (Battery Acid) & 
If you are required to prepare an X molarity solution of HCl, prepane a $X \times 0.5$ molarity solution of battery acid & 
The instructions call for 0.12~M HCl. Instead, prepare 0.06~M sulfuric acid & 
 \\ \hline

Ethanoic (Acetic) Acid & 
Sulfuric Acid (Battery Acid) & 
If you are required to prepare an M molarity solution of ethanoic acid, prepare a $M \times 0.5$ molarity solution of sulfuric acid & 
The instructions call for 0.10~M ethanoic acid. Prepare 0.05~M sulfuric acid. & 
 \\ \hline

Ethandioic (Oxalic) Acid dihydrate (C$_{2}$H$_{2}$O$_{4} \cdot$2H$_{2}$O) & 
Sulfuric Acid (Battery Acid) & 
If you are required to prepare an M molarity solution of ethandioic acid, prepare an M molarity solution of sulfuric acid. If you are required to prepare a C concentration solution of ethandioic acid, prepare a $^C/_{126}$ molarity solution of sulfuric acid. & 
The instructions call for 0.075~M ethandioic acid. Prepare 0.075~M sulfuric acid. & 
The instructions call for $6.3 ^g/_L$ ethandioic acid. Prepare 0.05~M sulfuric acid. \\ \hline

Potassium Hydroxide & 
Sodium Hydroxide (Caustic Soda) & 
For M molarity potassium hydroxide, make M molarity sodium hydroxide. For C concentration potassium hydroxide, make $C \times ^{40}/_{56}$ concentration sodium hydroxide. & 
The instructions call for 0.1~M potassium hydroxide. Just prepare 0.1~M sodium hydroxide. &
The instructions call for $2.8 ^g/_L$ potassium hydroxide. Prepare $2 ^g/_L$ sodium hydroxide. \\ \hline

Anhydrous Sodium Carbonate & 
Sodium Carbonate Decahydrate (Soda Ash) &  
For M molarity anhydrous sodium carbonate, make M molarity sodium carbonate decahydrate. For C concentration anhydrous sodium carbonate, make $C \times ^{286}/_{106}$ sodium carbonate decahydrate. & 
The instructions call for 0.09~M anhydrous sodium carbonate. Make 0.09~M sodium carbonate decahyrate. & 
The instructions call for $5.3 ^g/_L$ anhydrous sodium carbonate. Make 14.3 g/L sodium carbonate decahydrate. \\ \hline

Anhydrous Sodium Carbonate & 
Sodium Hydroxide (caustic soda) & 
For M molarity anhydrous sodium carbonate, make $M \times 2$ molarity sodium hydroxide. For C concentration anhydrous sodium carbonate, make $C \times 2 \times ^{40}/_{106}$ sodium hydroxide. & 
The instructions call for 0.09~M anhydrous sodium carbonate. Make 0.18~M sodium hydroxide. & 
The instructions call for $5.3 ^g/_L$ anhydrous sodium carbonate. 4.0 g/L sodium hydroxide. \\ \hline

Sodium Carbonate Decahydrate (Na2CO3∙10H2O) &
sodium hydroxide (caustic soda) &
For M molarity sodium carbonate ecahydrate, make $M \times 2$ molarity sodium hydroxide. For C concentration sodium carbonate decahydrate, make $C \times 2 \times ^{40}/{286}$ sodium hydroxide. &
The instructions call for 0.09~M sodium carbonate decahydrate. Make 0.18~M sodium hydroxide. &
The instructions call for 14.3 g/L sodium carbonate decahydrate. Make 4.0 g/L sodium hydroxide. \\ \hline

Anhydrous Potassium Carbonate & 
Sodium Carbonate decahydrate (Soda Ash) & 
For M molarity potassium carbonate, make M molarity sodium carbonate decahydrate. For C concentration potassium carbonate, make $C \times ^{286}/_{122}$ concentration sodium carbonate. & 
The instructions call for 0.08~M anhydrous potassium carbonate. Prepare 0.08~M sodium carbonate decahydrate. & 
The instructions call for $6.1 ^g/_L$ anhydrous potassium carbonate. Prepare $14.3 ^g/_L$ sodium carbonate decahydrate. \\ \hline

Anhydrous Potassium Carbonate & 
Sodium Hydroxide (caustic soda) & 
For M molarity potassium carbonate, make $M \times 2$ molarity sodium hydroxide. For C concentration potassium carbonate, make $C \times 2 \times ^{40}/{122}$ concentration sodium hydroxide. & 
The instructions call for 0.08~M anhydrous potassium carbonate. Prepare 0.16~M sodium hydroxide. & 
The instructions call for $6.1 ^g/_L$ anhydrous potassium carbonate. Prepare $4.0 ^g/_L$ sodium hydroxide. \\ \hline

\end{tabular}
\end{center}

\section{Additional Notes}

\begin{itemize}

\item{In volumetric analysis experiments with two indicators, it is not possible to substitute one chemical for another as the acid/base dissociation constant is critical and specific for each chemical. It is still possible to substitute sodium carbonate decahydrate for anhydrous sodium carbonate with the above conversion.}

\item{These substitutions only work for volumetric analysis. In qualitative analysis, the nature of the chemical matters. If the instructions call for sodium carbonate, you cannot provide sodium hydroxide and expect the students to get the right answer!}

\end{itemize}
