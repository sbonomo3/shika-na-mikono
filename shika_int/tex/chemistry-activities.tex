\chapter{Chemistry Activities}

\section{Black Light}

A black light is a lamp whose light has wavelengths between 400 and 315 nm. They have some mercury atoms surrounded by helium so that when a high voltage is applied, the lamp produces photons in the ultraviolet spectrum. This system will release UVA and UVB light. UVB light is cancer causing, because camphors painted on the inside of the bulb absorb the UVB light and fluoresce visible purple, hence the color of these lights.@Black lamps are useful because some chemical compounds fluoresce when hit with UV light. Fortunately, in Tanzania black lights are very common – just about any purple tube light in town. If some white object looks unnaturally white, the light hitting it is probably from a UV lamp. Ask the store or bar owner where the light was purchased, and track one down. A small lamp should not be very expensive. @Black lights are particularly useful for helping students understand and explore the concept of light waves, especially as lasers and other optical devices are more difficult to obtain. The properties of fluorescence helps to explain that there are indeed different types of lights. Under normal light, the objects behave normally; under UV light, things change. This section is devoted to helping students experience different types of light and to talk about electronic transitions.

\subsection{Secret Messages}
\begin{itemize}
\item{Preparation time: 0 minutes}
\item{Materials: black light, yellow highlighters}
\item{Procedure: Let students write on their arms with a yellow highlighter. The writing should be difficult to read over dark skin pigments. In a dark place, turn on a black light and bring it close to the highlighter writing. The writing will be very easy and clear to read}
\item{Theory: When some compounds have light shown on them, they can fluoresce. Fluorescence is a series of electron transitions. Much like Balmer series in the hydrogen spectrum, fluorescence is electrons moving from high energy levels to low energy levels and releasing the extra energy as light. In the Balmer series with a hydrogen atom, the photons are released from a single electron movement, from the fourth energy level to the second, for example. Fluorescence has intermediate steps that release photons in the visible spectrum. The electrons absorb light in from photons in the UV spectrum. If the electrons move from one energy level and then drop to the initial energy level, then the photons released would have the same wavelength as the UV light. In other words, it would not be possible to see the light released from the electron transition. That is not the case since some compounds show pretty intense light. What actually occur are intermediate energy transitions; the electrons fall to a nearby energy level before falling down to the initial energy level. Each of these transitions last for a short time and the photons released from these shorter transitions may have wavelengths in the visible spectrum. This is the reason there are some interesting colors in different objects.}
\end{itemize}

\subsection{Chlorophyll Colors}
\begin{itemize}
\item{Preparation time: 10 minutes}
\item{Materials: green leaf, konyagi, 2 jam jars, pen, black light}
\item{Procedure: We need to extract the chlorophyll from a green leaf. Tear the green leaf into small pieces and place into a jam jar. Add enough konyagi to cover the leaves. Using the bottom end of a pen, mash the leaves in the alcohol. The idea is to break the plant cellular structure so that the chlorophyll comes out. After mashing for five minutes, let the leaves sit in the alcohol for another five minutes. Pour off the liquid but remove any green leafy pieces. Take the liquid and place near a black light. The solution will glow red.}
\item{Theory: Chlorophyll is the compound that lets plant absorb visible light for as energy in a cell. This energy is stored as glucose. This is due to a large modified conjugated heme group with a magnesium atom. A conjugated system is a system of bonding orbitals that have similar energies. This allows the electrons to travel not just in its own bond, but also into other adjacent bonds of similar energies. The electrons are free to travel and this allows them to absorb different wavelengths of light and even fluoresce. This modified heme group not only absorbs normal visible light, but it also fluoresces. Under UV light, the green chlorophyll produces a red color. The ability for the electrons to move so easily under UV light is the exact reason plants can convert solar energy into sugars. By moving electrons around, the plant can store electrons for future use in the form of starches.}
\end{itemize}

\subsection{Quinine Blue, Part A – The Tonic}
\begin{itemize}
\item{Preparation time: 5 minutes}
\item{Materials: bottle of tonic water, black light}
\item{Procedure: Under visible light, the tonic water is clear. Place the bottle of tonic water near the black light. The tonic water will glow a slight blue color.}
\item{Theory: One of the flavorings of tonic water is quinine. Quinine is a common medicine against malaria. Quinine has a conjugated system of pi-orbitals that give the molecule molecular orbitals with electron transitions able to fluoresce when hit by UV. This fluorescence is in the blue spectrum. Note that there are no electron transitions that are forced by visible light – hence quinine has no visible color under normal light.}
\end{itemize}

\subsection{Quinine Blue, Part B – Blue Water}
\begin{itemize}
\item{Preparation time: 20 minutes}
\item{Materials: water, clear plastic water bottle, tablet of quinine, black light}
\item{Procedure: Dissolve one or two tablets of quinine in some clean water. Shake well and let the solution sit to allow the tablets to dissolve completely. Place the bottle near a black light. It will glow blue.}
\item{Theory: The quinine from the medicine tablets dissolve readily in solution. The solution glows a bright blue color. This is the same color as the tonic water. The water is has no color in visible light, but when the UV light is applied, the quinine fluoresces brightly.}
\end{itemize}

\subsection{Petroleum Jelly Glow}
\begin{itemize}
\item{Preparation time: 0 minutes}
\item{Materials: petroleum jelly, black light}
\item{Procedure: Let the students put some petroleum jelly on their hands and then have them walk over to the black light. Their hands will glow slightly purple.}
\item{Theory: Petroleum jelly consists of long hydrocarbon chains that are one of the last products of crude oil distillation. Its molecular mass is very high and it has some conjugation in its structure. For reasons similar to those of quinine, these conjugated systems absorb UV light and fluoresce a purple light. }
\end{itemize}

\subsection{Alum Crystals Fluorescence }
\begin{itemize}
\item{Preparation time: 3 hours}
\item{Materials: same supplies from Growing Giant Crystals, fluorescent highlighters, black light}
\item{Procedure: Follow the same procedure as Growing Giant Crystals. When adding the alum to the hot water, add the ink of a water-soluble fluorescent highlighter. When the crystal is formed, let it dry completely. Once dry, place near a black light. The crystal should glow in the UV light.}
\item{Theory: As the crystal cools, some impurities will be trapped inside the crystal. This time, we want impurities trapped in the crystals. The fluorescing agent from the highlighters will be trapped inside the alum crystals. Even though it is inside of the crystals, it will still fluoresce when placed near a black light. The alum crystal will glow the color of the highlighter when it fluoresces. It is a good comparison to have two highlighters of the same color: one for use in making the crystal and one to show that the fluorescing light of the crystal is the same as the highlighter.}
\end{itemize}

\section{Chromatography}

\subsection{Paper Chromatography}
\begin{itemize}
\item{Preparation time: 5 minutes}
\item{Materials: Jam jar with lid, white paper, colored markers, alcohol either methylated spirits or isopropyl}
\item{Procedure: Cut a thin piece of paper so it forms a 6 cm by 2 cm rectangle. One cm from end of a paper, put a mark with the marker. Put 10 mL of alcohol in the bottom of the jam jar. We want to add just enough to have a small layer of solvent at the bottom, which does not reach the marker spot. Add the paper, and cover the lid. After a few minutes, the solvent will climb the paper and separate out the inks in the marker.}
\item{Theory: Chromatography is the process where compounds are separated by different solvents. In this case, the alcohol acts as the solvent. The solutes, the different pigments that combine to give the marker its color, dissolve in the alcohol. However, only some of the compounds in the marker dissolve in the alcohol. As the alcohol climbs up the paper, it will carry the dissolved compounds. Depending on how well the compounds in the marker ink dissolve in the alcohol the colored compounds will move up to different heights on the paper. The better the compound dissolves in the alcohol, the farther the ink will move. With markers, at least four different colors can be separated out from the marker ink. }
\end{itemize}

\subsection{Chalk Chromatography}
\begin{itemize}
\item{Preparation time:  5 minutes}
\item{Materials: Jam Jar with lid, white chalk, colored markers, alcohol either methylated spirits or isopropyl}
\item{Procedure: Follow the same procedure as Paper Chromatography, except instead of using paper, use a piece of chalk. Take the piece of chalk out when the solvent has reached the top of the chalk and let it dry. The chalk remains usable for writing on blackboards.}
\item{Theory: This style of chromatography has two distinct phases: a stationary solid phase and a mobile liquid phase. The solid phase is chalk, which is calcium sulfate, an ionic solid. The mobile phase is the alcohol, a non-polar liquid. The solute, the marker ink, has both polar and non polar compounds. The polar compounds are attracted to the ionic chalk while the non-polar compounds dissolve in the alcohol. As the alcohol moves up the piece of the chalk, the different colored compounds will move up the piece of chalk. These colored compounds will not make the chalk colored, but the outside will have a rainbow appearance.}
\end{itemize}

\section{Clouds}

\subsection{Cloud in a Jar}
\begin{itemize}
\item{Preparation time: 10 minutes}
\item{Materials: 1 wide mouth glass jar, 1 latex glove, water, matches}
\item{Procedure: In the bottom of a glass jar, fill with a small amount of water. Take the glove and put it so the fingers are inside of the jar. Seal the jar with the glove. Put your hand into the glove and pull the glove out. Be sure not to break the seal. Quickly open the seal on the glove and drop in a lit match. Quickly cover the seal with the glove just like before, fingers pointing in the jar. Pull the glove out of the jar once again and observe the cloud inside the jar.}
\item{Theory: The clouds in the sky are formed when water vapor is cooled enough to form tiny water droplets. When moist, cool air rises to a higher altitude, it cools water droplets form and aggregate to form clouds. In this activity, we duplicate this same process by causing air in a bottle to rapidly cool. By putting the glove on the jar and pulling it out, the volume of the gas inside the container increases. Some of the water on the bottom of the jar turns into a gas and the temperature drops. This primes the jar. Dropping a match in the jar creates smoke and other particles that act as nucleation sites for the rain to form. Pulling out the glove a second time lowers the temperature by decreasing the volume enough to start forming clouds.}
\end{itemize}

\section{Colligative properties}

\subsection{Boiling Point Elevation, Part A – Salt Water}
\begin{itemize}
\item{Preparation time: 20 minutes}
\item{Materials: water, salt, battery acid, heat source, thermometers, 3 metal pots}
\item{Procedure: In one metal pot, add water only. In the second pot, make a salt solution. The more concentrated the better. Heat both pots and record the temperature when they start to boil. }
\item{Theory: Pure solutions boil at a lower temperature than solutions that have dissolved salts. The first pot acts as the pure solution of water. It should boil around 100 C. Depending on the amount of salt added, the second pot will boil slightly higher, around 102-105 C. The impurities of salt prevent the water from boiling at its normal temperature. It increases the temperature required to boil. }
\end{itemize}

\subsection{Boiling Point Elevation, Part B - Electrolytes}
\begin{itemize}
\item{Preparation time: 20 minutes}
\item{Materials: glucose, salt, water, balance, jam jars, heat source}
\item{Procedure: In two jam jars, fill each with 150 mL of water. To one, add 30 grams of salt. The other, add 90 grams of glucose. It may look like there is too much glucose, but it will dissolve in the water upon heating. Heat both solutions slowly and record the temperature at which they boil.}
\item{Theory: Colligative properties depend not on the actual makeup of the impurities, but the number of impurity particles. In this variant, we are adding about 0.5 moles of salt to one jar and 0.5 moles sugar to the second. Upon heating, the salt solution will boil at 102 C while the sugar solution will boil at 101 C. This discrepancy is due to the fact that salt is a strong electrolyte and glucose is a non electrolyte. This means that salt will dissolve completely in water to form two ions: a sodium and a chlorine. Glucose does not dissociate. Comparing the number of particles in both the salt and sugar solution, we can see that there is twice the number of impurity particles in the salt solution compared to the sugar solution. This means that the salt solution boils at a higher temperature than the sugar solution.}
\end{itemize}

\section{Combustion}

\subsection{Products of Combustion, Part A – H2O}
\begin{itemize}
\item{Preparation time: 5 minutes}
\item{Materials: A glass jam jar or the bottom half of a plastic water bottle, a candle, water}
\item{Procedure: Find a flame or ignition source. A candle works best for this activity. Take the beaker fill it with water. Place this over the candle not to touch the flame just above the tip of the flame. The gases from the candle should collect bottom on the container. After a minute or two, condensation will form on the outside of the glass. This shows one of the products of organic combustion: water.}
\item{Theory: When organic compounds burn in excess air, they form two main compounds: carbon dioxide and water according to the formula:. If we collect the gases from the candle, we are collecting carbon dioxide gas and water vapor. As the water cools on contact with the cold container, it will undergo a phase transition from gaseous to liquid. In this activity, the water vapor is captured on the jar. It cools and then condenses on the side of the container.}
\end{itemize}

\subsection{Products of Combustion, Part B – CO2}
\begin{itemize}
\item{Preparation time: 5 minutes}
\item{Materials: A glass jam jar or the bottom half of a plastic water bottle, a larger container like a plastic bucket or metal bowl, a candle}
\item{Procedure: Now we are going to collect the other product of combustion, carbon dioxide. For this, we will place our candle on an upside down container. Then place these two inside of a larger container. After ten to thirty minutes, we have collected enough carbon dioxide. Remove the candle from the container and place it next to the large container. Now pour the gas products gently from the large beaker on top of the candle so that there is no rush of air. The candle should gently go out.}
\item{Theory: From the reaction, carbon dioxide is produced from combustion. When compared to air, carbon dioxide has a greater density. By placing the candle on in a container, the carbon dioxide produced falls down to the bottom of the container instead of dispersing away. Now, if carbon dioxide displaces the oxygen that is normally used for combustion, the organic compounds can no longer combust. By pouring the carbon dioxide on the candle, we prevent any oxygen participating in combustion and the candle goes out, since the flame is the sign of combustion of a candle.}
\end{itemize}

\subsection{Reactants of Combustion, Part A – O2}
\begin{itemize}
\item{Preparation time: 5 minutes}
\item{Materials: a clear glass jam jar or the bottom half of a plastic water bottle, a candle}
\item{Procedure: Place the candle on the table so it burns freely. Turn the container upside and place over the candle on the table. A transparent container works best for this activity. This traps the gases inside the container. After a minute, the candle should go out.}
\item{Theory:  Combustion requires oxygen,. If there is no more oxygen for use in the combustion reaction, the combustion ends. When we trap the candle under our beaker, no more oxygen can come in from the surrounding air. That means we have a set amount of oxygen inside the upside down container. As the candle burns, it will consume all the oxygen trapped in the candle. When there is no more oxygen to burn, the reaction stops and the candle go out. Note that before the candle goes out, a lot of smoke is produced. This is because as oxygen becomes scarce the combustion becomes incomplete and unburned hydrocarbons are formed.}
\end{itemize}

\subsection{Reactants of Combustion, Part B}
\begin{itemize}
\item{Preparation time: 5 minutes}
\item{Materials: radio antenna tube, kerosene single wick burner (kibatari)}
\item{Procedure: Light the burner on the table. Take a radio antenna tube and place it at an angle inside the bottom part of the flame. The gases from the flame will travel up the pen. Take a match and ignite the flame at the end of the ratio antenna tube. }
\item{Theory: The purpose of this activity is to show that the hydrocarbons in combustion reaction are in the gaseous state, not the liquid state. We can show this with a kerosene flame. When kerosene burns, the liquid changes to the gaseous state. Through the entire length of the flame, the gases are burning, not the liquid kerosene. By providing an alternate path for the gases to travel, up the radio antenna, these gases do not burn immediately. Now at the end of the tube, we can light a match and ignite the gases as they reach the end of the tube. There is no liquid kerosene traveling in the tube to burn in the second flame, only the gas from the first flame. This shows that it is the hydrocarbons, or kerosene in the case of a burner, are gases not liquids. }
\end{itemize}

\subsection{Burning Money}
\begin{itemize}
\item{Preparation time: 10 minutes}
\item{Materials: jam jars, methylated spirits, water, matches, paper money or ordinary paper, tongs}
\item{Procedure: Make a mixture of 3 parts methylated spirits to 2 parts water. Soak the money in the mixture. Remove the note with the tongs and take a match to it. The bill should flame up and after 5 seconds drop the bill into the extra water. If the blue flames are difficult to see, repeat the activity in a darker space.}
\item{Theory: Methylated spirits are a mixture of ethanol and water. Ethanol is a very volatile compound while water is much less volatile. This means that the flash point of ethanol is low compared to paper. The water regulates the temperature of the flame such that it is higher than the flash point of ethanol and lower than the flash point of paper. This is why we can take a flame to the bill and let it burn for a short period of time without any damage to the bill. The ethanol is burning at a low temperature while the water protects the bill from combusting. However if there is a lot of ethanol and it burns for a long time, the water will evaporate away and then the bill will start burning. Do not be a criminal; just like in the United States, destruction of currency is a crime. Do not let the note to start burning.}
\end{itemize}

\subsection{Rusting of Steel Wool, Part A}
\begin{itemize}
\item{Preparation time: 15 minutes}
\item{Materials: Iron wool, plastic syringe with the top fused shut or a graduated cylinder, a dish, water}
\item{Procedure: A plastic 10 mL syringe is highly recommended for this procedure. Remove the metal needle and the plunger part of the syringe leaving the graduated shell. Then use a flame to fuse closed the narrow opening where the needle joined. If the steel wool has a detergent coating, wash it off before use with soapy water. Then take a piece of steel wool, wet it in water, and stuff it up the shell so it stays in place at the top. Hold the shell upright and place the syringe shell onto of a dish of water. Wait for three days, checking each day to replace the water lost due to evaporation. The water should move up the syringe.}
\item{Theory: Rusting is a form of combustion. When iron is in contact with both water and oxygen, it will rust in a complicated set of reactions. This is the formation of two products, iron (III) oxide, Fe2O3, and iron (III) hydroxide, Fe(OH)3. Inside the syringe, there is a trapped volume of oxygen. The reaction consumes the oxygen as time goes on. This will lower the pressure inside of the syringe and atmospheric pressure pushes the water up the syringe shell.@	For the more advanced, this activity can be used to show the percent by volume, and then percent by mole, of oxygen in the atmosphere. Prior to the reaction, measure the volume of the steel wool. Find this by water displacement in a fused 10 mL syringe shell. Then place the steel wool in the syringe and ensure the water level is in the gradations of the syringe volume. Record the initial volume. The reaction finishes when the water layer stops rising. Record the final volume. By comparing the initial volume minus the volume of the steel wool and the volume of oxygen used indicated by the rise in the water level, we can see the composition of oxygen in air as a ratio of volumes. Further, since Avogadro’s law states volume is proportional to the number of moles, the percent of oxygen in the atmosphere by volume is also the percent of oxygen in the atmosphere by moles. }
\end{itemize}

\subsection{Rusting of Steel Wool Part B – Hard Water}
\begin{itemize}
\item{Preparation time: 30 minutes}
\item{Materials: Iron wool, 2 plastic syringes with top fused shut, 2 dishes, hard water, soft water}
\item{Procedure: This procedure works well if the local water is hard water. Hard water may be prepared by dissolving magnesium sulfate in ordinary water. Use the procedure from Rusting of Steel Wool, Part A with hard water. Repeat the same procedure with one small change: use soft water at the same time. Soft water can be found from collecting the water when distilling water in a teapot or rainwater. After sufficient time passes, the rusting with hard water will have the water at a higher level than the rusting with soft water.}
\item{Theory: Just like Rusting of Steel Wool, Part A, the reaction consumes the air inside of the syringe. The difference between the two rusting variations is the hard water and soft water. Hard water contains dissolved Ca2+ (and Mg2+) ions; soft water has none. These ions will react with the iron (III) hydroxides to form calcium hydroxides (and Mg) and iron calcium oxide hydroxides. Due to these side reactions, the presence of Ca2+ ions speeds up the rusting reactions. As it speeds up the reactions, the oxygen is consumed faster and the atmospheric pressure compensates by pushing more water up the syringe shell. This is reason that the hard water rusting will have a higher water level than the soft water.}
\end{itemize}

\subsection{Rusting of Steel Wool, Part C – Salt Water}
\begin{itemize}
\item{Preparation time: 10 minutes}
\item{Materials: table salt, iron wool, two syringe shells, distilled or soft water}
\item{Procedure: Follow the same procedure to from Rusting of Steel Wool, Part A for one of the syringe shells. For the other, make a concentrated salt solution. Use this salt solution in place of the normal water. Observe which iron will rust first.}
\item{Theory: This activity tests how iron rusts in the presence of other ions. Rusting is not a completely understood reaction. This variant of the activity tests the effect of electrochemical forces, the presence of salt, as it affects the rate of rusting.}
\end{itemize}

\subsection{Rusting of Steel Wool, Part D – Heat of Rusting}
\begin{itemize}
\item{Preparation time: 30 minutes}
\item{Materials: a thermometer, iron wool, vinegar, a container with cap large enough to hold the thermometer like a 1 or 1.5 L empty plastic water bottle}
\item{Procedure: Place the thermometer in the empty plastic water bottle. Wait 10 minutes for the temperature to stabilize. Take a small piece of steel wool soak it in vinegar while the temperature stabilizes. Record the temperature and remove the thermometer. Squeeze the iron wool to release any vinegar. Wrap the bulb of the thermometer with the iron wool that was soaked vinegar, replace them in the plastic water bottle, and cap. Record the temperature every 10 minutes for the next 2 hours. As time proceeds, the temperature will rise.}
\item{Theory: The rusting of iron is an exothermic reaction. An exothermic reaction is a chemical change that is accompanied with the release of heat. In this particular activity, the rusting of the iron releases heat that is recorded by the thermometer. The rise in temperature is a representation of the exothermic nature of this particular reaction. In order to have the reaction proceed quickly and without problems, vinegar is used to clean the surface allowing the iron metal to react easily, especially if there is any detergent coating the iron as wells as any surface iron oxides.}
\end{itemize}

\subsection{Rust prevention}
\begin{itemize}
\item{Preparation time: 30 minutes}
\item{Materials: oil, water, syringes shells fused at the top, iron (non galvanized) nails, paint (if desired), syringe plungers}
\item{Procedure: In this activity, different aspects of the rusting process will be examined. Each part of this activity requires a syringe shell with the top fused. These are all recommended activities but any combination of the aspects maybe conducted. First, make a control syringe. Place the nail in the syringe shell and cover with water. If the nail has some rust on it to begin with, use some iron wool and rub the nail to remove the rust. Stand the syringe shell upright so to leave the nail is submerged in the water but open to the air. Once we have our control, other aspects can be tested of the rusting reaction. Examine each experiment each day for a week for rust on the nail. Here are six different variations to test:

\begin{enumerate}
\item{Place a nail in a syringe and fill with water. Leave open to air. This is the control.}
\item{Place a nail in a syringe and fill with water. Place the plunger back to seal the syringe to prevent oxygen from entering the shell.}
\item{Place a nail in a syringe and cover with oil. Leave open to air.}
\item{Place a nail in a syringe and cover with oil. Replace plunger to seal the syringe.}
\item{Place a nail in a syringe cover with water. Make an oil layer with a thickness of at least 2 cm.}
\item{Take a nail and paint the nail to cover the metal completely. Now place in a syringe and fill with water.}
\end{enumerate}
} % Procedure

\item{Theory: The reaction for the rusting of iron consists of a reaction between iron, oxygen, and water. In this activity, each variety tests a different aspect of the rusting reaction. Experiment 1 is the control experiment. This will be the normal rusting of the iron nail. Experiment 2 now prevents any oxygen from participating in the rusting reaction. This nail should have little or no rust. Experiment 3 allows oxygen to encounter the oil for rusting; however, there is no water since it is filled with oil. In addition, oxygen migrates through oil very poorly that also limits the available oxygen for rusting on the surface of the nail. There should be little or no rusting on the nail. Experiment 4 has no oxygen or water available for rusting. This nail should have no rusting. Experiment 5 allows water available to react, but now there is an oil layer. This oil layer prevents oxygen from entering the water layer and rusting the iron nail. This nail should have little or no rust. Experiment 6 is much like the control, but since we have painted the nail, there is no iron available to react since it is protected by the paint. In each of these experiments, the availability of the iron, oxygen, and water change and the effects of the lack of availability of each component slow down or even prevent the rusting reaction.}
\end{itemize}

\subsection{Oxidation of Iron, Part A – Burning in Air}
\begin{itemize}
\item{Preparation time: 0 minutes}
\item{Materials: lighter, fresh iron wool, iron oxide or rusted iron wool from a previous activity}
\item{Procedure: With a lighter, place the rusted steel wool in the flame from the lighter. There no be no reaction or change other than a rise in temperature. Now, do this activity again with fresh iron wool. The iron wool should burn brightly and oxidize in the air readily, producing a bright light, much like fireworks. Lastly, let the iron wool cool for 2 minutes. Place the iron wool back into a lighter flame. There should be no reaction this time.}
\item{Theory: Iron, when heated in the presence oxygen, oxidizes to iron (III) oxide. The reaction is as follows:. With the rusted iron wool, the iron metal has undergone oxidation to form the iron (III) oxide and iron (III) hydroxides. If the rusted iron wool is heated, there is no reaction since the iron is already been oxidized. The iron metal has all been consumed. The entire iron wool will burn and produce a bright light from the oxidization. Now as the iron wool burns and then cools, it should turn a darker color that shows the presence of iron (III) oxide. This means that there is no more iron available to oxidize. This activity is a good example of the different pathways of oxidation. Oxidization can occur through rusting or it can be direct oxidization in air. }
\end{itemize}

\subsection{Oxidation of Iron, Part B – New Mass}
\begin{itemize}
\item{Preparation time: 5 minutes}
\item{Materials: fresh iron wool, lighter, balance}
\item{Procedure: Take a piece of iron wool and find its mass. Record the mass. Place the iron wool in the lighter flame as to oxidize it according to the procedure in Oxidation of Iron, Part A. Let the iron wool cool to room temperature. Record the mass a second time. The mass of the oxidized iron will be larger than the fresh iron wool}
\item{Theory: From the reaction, the iron wool is just iron metal. As it undergoes oxidation, the iron reacts to form iron (III) oxide. The molecular mass of iron (III) oxide is larger than iron metal due to the additional oxygen. Since the number of moles of iron is not changing, the mass of the entire solid is increasing from the addition of the oxygen.}
\end{itemize}

\subsection{Oxidation of Iron, Part C – In the World}
\begin{itemize}
\item{Preparation time: 30 minutes}
\item{Materials: Nearby metalworker}
\item{Procedure: Arrange a field trip to a nearby metalworker who grinds metal. Observe the technician grind the metal and watch the sparks fly from the metal being grinded. }
\item{Theory: The sparks caused by the metal grinder is the same reaction from Oxidation of Iron, Part A. Bits of iron metal, heated from the mechanical work from the grinder, oxidize as they fly through the oxygen-rich air. The piece being ground does not glow because the rest of the metal conducts the heat away from the grinding site fast enough to keep the working surface below the temperature of combustion. Nevertheless, evidence of the heat is often visible on the finished piece.}
\end{itemize}

\subsection{Combustion of Meats}
\begin{itemize}
\item{Preparation time: 30 minutes}
\item{Materials: 1/4 kg of meat, two burners (electric, charcoal, or kerosene), two metal bowls or pans, one for frying, one for boiling water}
\item{Procedure: Cut the meat into small cubes. Cut enough for 3 pieces for each participant or student. Take one piece for each student and boil it in water for 30 minutes. Take a second piece of meat for each student and sear it in oil. This means to fry, or grill, the meat until the surface of the meat browns and dark flakes appear in the oil, but not until it burns. Take the third piece of meat and fry, or grill, it in oil until it blackens or burns and turns into a black color. Have each student eat a piece of meat, one boiled, one fried (or grilled), one blackened. Ask them to identify the difference in flavors and textures. Of course, this experiment should be done in an ordinary classroom, not the school lab. You would not want to encourage eating in the lab.}
\item{Theory: Meats and other foods are a large system of organic compounds. Specifically, meats have amines on the surface. These amines react with reducing sugars, like glucose and fructose, in the meat and form a large variety of different compounds that all accompany with a good taste and smell. This is why fried meats, and even grilled meats, taste so good. This is due to the way they are cooked. Now if we continue to fry or grill the meat past the point where the meat is browned, the amines and other surface compounds of the meat combust until they are carbon solid, or like charcoal. If the students eat the fried meat, it should taste good. The blackened meat should taste bad or like charcoal. The difference between the two is the extent that the compounds of the meat combust; browned meat is partial combustion or reactions while blackened meat is approaching complete combustion. This particular chemical reaction of the browning of meat is called the Maillard reaction.//
The boiled meat however, does not have the same Maillard reaction. The Maillard reaction happens between the surface chemicals of the meat under heat. The boiled meat allows for these compounds to dissolve in the boiling water. This means that the Maillard reaction does not occur. This can be seen by first looking at the surface of the boiled meat. It is not the same as the fried or grilled meat; in fact, it should look dark and maybe grey but not brown. If eaten, the boiled meat does not have the same flavor as the fried or grilled meat. This is due to the absence of the Maillard reaction.}
\end{itemize}

\section{Crystal Growing}

State of matter transitions are fun and interesting to show. One of the more fun, intriguing and hands on example of changing states of matter is dissolving solids and forming them again. Different crystals will form different shapes. The smooth sides with sharp edges are something that is just too neat not to make and touch. Plus, the students get a big kick out of growing their own crystals to take home. There is a variety of different solids to try to make crystals from cheap and locally available materials.

\subsection{Salt Crystals}
\begin{itemize}
\item{Preparation Time: 0 minutes}
\item{Materials: Table salt, water, container/beaker}
\item{Procedure: Add 19g of table salt (NaCl) to 50 mL water. The actual amount of salt or water does not really matter; just add salt until the solution is saturated. Transfer the salt solution into a wide mouth container or a pan: the larger the surface area that the solution has, the faster it will evaporate. As the salt evaporates, crystals of sodium chloride will form. Normally, table salt is a very fine grain. These crystals will make slightly larger crystals where the cubic habit can easily be seen with a simple water drop magnifier.}

\item{Theory: Solutions are what happens when we mix two or more components dissolve into another. One does not disappear into the other; both components are still there although not seen. An aqueous solution cannot be separated by normal physical means. The two can be separated by letting the water evaporate. The water leaves, but the NaCl does evaporate. It is left behind and forms a solid. Ionic solids tend to form nice crystal lattices, and table salt is no exception. This activity can be used to explain a variety of different topics:

\begin{itemize}
\item{Crystallization: This is a perfect example at the difference between freezing and crystallizing. Freezing occurs when a temperature of a liquid drops until the molecules have so little energy that the intermolecular forces begin to hold them together. We can think of it as the molecules are free to move as a liquid but as the temperature drops, the molecules slow down until they are stuck in place. Crystallization occurs when we have ions that have a greater stability by surrounding themselves in more ions. Instead of one sodium atom bonding to one chlorine atom, the crystal finds its stability by allowing the sodium ion to bond to many chlorine atoms (six in this case). The same happens for the chlorine atom. This extra stability helps the molecules to hold together and form the cubic shape.}
\item{Difference between covalent and ionic compounds: Covalent compounds tend to have low intermolecular forces and are more volatile. This means they easily turn into a gas or evaporate. Ionic compounds on the other hand do not turn into gases easily. Relatively large electrostatic forces, the forces between positive and negative ions, hold the individual ions together so they cannot easily break away from each other. Water is an example of a covalent compound. It is somewhat volatile: if you leave some water in the sun, it will evaporate. Salt or NaCl, is an example of an ionic compound. It is not volatile and stays behind when the water evaporates.}
\end{itemize}
}
\end{itemize}

\subsection{Epsom Salt Crystals}
\begin{itemize}
\item{Preparation Time: 10 minutes}
\item{Materials: Epsom salt, water, glass jam jar, 1 small square of glass or plastic}
\item{Procedure: Add 1g of Epsom salt (MgSO4) to 5 mL water. Heat the solution until all the crystals dissolve. Pour the Epsom salt solution onto a small square piece of glass or plastic: the larger area the solutions occupy, the thinner a layer will form and the faster it will evaporate. As the solution evaporates, it leaves behind crystals. Use a water magnifier, from the activity in the biology section Water Magnifier, to look at the crystals. MgSO4 forms a lattice of needlelike crystals. Another method to see the crystals is to form them on a piece of clear glass. Shine a light over the crystals to project their image onto a piece of paper. The shape of the crystals can be easily seen. }
\item{Theory: The same processes are involved in the formation of MgSO4 crystals. However, we can use this slightly altered procedure to talk about different things.
\begin{itemize}
\item{Temperature dependent solubility: For many solutes, the temperature of the solvent plays a big role in solubility. In the case of these ionic compounds, a slight increase in the temperature of water dramatically increases the amount salt that will dissolve. By heating the water, we are able to dissolve more Epsom salt in the water so that when it evaporates it will form crystals large enough for us to see.}
\item{Crystal Lattice Shapes: Due the difference size of the ions combined with their charges, they will form different shapes as a crystal. NaCl forms a normal cube. MgSO4 forms Monoclinic shape. In fact, we can follow the same procedure to make different crystal shapes from different compounds. 
\begin{itemize}
\item{Sodium thiosulfate: Hexagonal shape}
\item{Magnesium Sulfate heptahydrate: Monoclinic}
\item{Sodium Nitrate or Saltpeter: Tetragonal}
\item{NaCl: Cubic}
\item{FeSO4: Tetragonal}
\item{CuSO4: Triclinic}
\end{itemize}
} %crystal lattice shapes
\end{itemize}
} % theory
\end{itemize}

\subsection{Watching Rapid Crystal Growth}
\begin{itemize}
\item{Preparation Time: 15 minutes}
\item{Materials: Sodium Thiosulfate Na2S2O3∙10H20, water, test tube or thin glass container}
\item{Procedure: In a test tube add the thiosulfate until the bottom 3 or 4 cm of the tube are filled with the salt. Gently heat the test tube until all the crystals have dissolved. All the crystals need to be dissolved and there should be no stray crystals on the side of the container. Put aside to cool. Once cool, add a seed crystal of thiosulfate. A crystal structure should form very quickly and spread throughout the test tube. It might be possible to have this rapid crystal growth a plastic syringe test tube.}
\item{Theory: In this activity, students make crystals rapidly from a super saturated solution. This means that we have more solute dissolved in the solvent than normally allowed, usually by increasing the solubility with heat and then cooling. Here we have extra thiosulfate dissolved in the water and it is ready to form crystals. However, crystals cannot be formed unless there is a nucleation point. This is just a spot for the crystals to start growing from. It can be a piece of dust, a rough surface in the test tube, or another crystal. When we add our seed crystal, the supersaturated solution will grow on the crystal outwards until enough thiosulfate precipitates so that the solution changes from supersaturated to just saturated. Beyond just a visually stunning activity, this activity is a good way to talk about saturated and super saturated solutions. }
\end{itemize}

\subsection{Growing Giant Crystals}
\begin{itemize}
\item{Preparation Time: 30 minutes}
\item{Materials: Alum, water, container of decent volume like a large plastic water bottle or bucket}
\item{Procedure: Making a giant crystal is easy using the ideas we have talked about so far. We are going to increase the temperature to increase the solubility and we are going to make a supersaturated solution crystallize out alum with a seed crystal. A note on alum; There are many times of alum. Commonly available in Tanzania is a white alum, KFe(SO4)2. First, find a container that you want to grow your crystal in. Start with a beaker or a jam jar. Tie a seed crystal into some string. Nylon dental floss is perfect but really any will work. Tie this string to a stick or a pencil such that when placed over the jar, the crystal will hang in the solution but will not touch the bottom. Remove the string. Use enough water such that the jar will be 2/3 filled. Heat this water and add alum slowly and stirring until completely dissolved. Add until no more will dissolve. Pour this solution into the jar. Place the seed crystal on the string into the solution so that it is hanging in the solution. After the solution cools, 1 to 2 hours, or overnight, you will have a much larger crystal than before. }
\item{Theory: This crystallization, we want to cause the crystals to grow on the seed crystal so that we can have a large crystal. The more alum you have dissolved we can precipitate more on our crystal. As expected, the more alum and the larger the container we have, the larger the crystal we can make. With this heat treatment procedure, some crystal deposits on the bottom of the container are expected. Remember that crystals precipitate on a nucleation point. The rougher the surface, crystals can form. Therefore, some containers make poor large crystal forming vessels. Metal or ceramic containers have a quite rough surface, which allows nucleation. A better container would be either glass or a plastic container. A tried and true vessel with the least amount of precipitate on the bottom of the container is plastic water bottles. These bottles come in different volumes, some as large as 12 L. Using a 6 L water bottle, a 4 kg giant alum crystal has been made.//
There are a few tricks to grow really large crystals:
\begin{itemize}
\item{Make sure the alum is clean. The dirt and impurities will provide sites for the alum to crystallize. We want it to crystallize where we want on the seed crystal not on the dirt or impurities that can provide nucleation sites.}
\item{Ensure the solution is saturated at the heated temperature or some of the alum will stay dissolved in the solution}
\item{Grow the crystal in a plastic container}
\item{Cover the container with a piece of cloth to prevent dust and other contaminants to find their way into the crystal solution}
\end{itemize}
If there are impurities found in alum, this is a good activity of the different ways to purify compounds. Crystallization of the alum crystals will purify itself by pushing out many of the contaminants. Especially when a seed crystal is used, a large crystal can be formed while leaving the impurities at the bottom of the container. However, not all the impurities can be removed. A pure alum crystal should be transparent but crystals made with a heated solution rather than evaporated are formed at a faster rate. This means the crystal is purifying by forming the crystal, but the rate is too fast prevent impurities. A cloudy alum crystal signifies the presence of water trapped inside the crystal lattice. }
\end{itemize}

\subsection{Hot Ice, Part A - Production}
\begin{itemize}
\item{Preparation time: 2 hours}
\item{Materials: baking soda, vinegar, heat source}
\item{Procedure: In an aluminum bowl, mix 1 liter of vinegar with 1 boxes of baking soda. Do not mix them quickly or you will have a volcano on your hands. After it has been completely mixed, boil the mixture. Boil until the total volume is reduced to 100 or 150 mL. Or, boil until a crystal skin covers the surface of the mixture. This may take upwards of an hour. It is not a problem is discoloration occurs. Once finished, cover the bowl to prevent evaporation. Ensure there are no stray crystals in the liquid. Move this mixture into a refrigerator to chill. Save any stray crystals.}
\item{Theory: Mixing baking soda and vinegar is a fairly common reaction and much repeated in this text. The reaction is. This solid can do some neat things, most notably that we can supercool it as a liquid in a normal refrigerator.}
\end{itemize}

\subsection{Hot Ice, Part B – Super Cool}
\begin{itemize}
\item{Preparation time: 10 minutes}
\item{Materials: sodium acetate supercooled solution from Hot Ice, Part A - Production, a small sodium acetate crystal}
\item{Procedure: Carefully remove the cooled sodium acetate liquid from the refrigerator. Be careful not to disturb it greatly or let anything fall into the liquid. Once ready, drop a sodium acetate crystal into the liquid. Crystallization will being immediately branching out from the crystal that we dropped in. Feel the container and the newly formed sodium acetate crystal structure.}
\item{Theory: Heat of crystallization is usually very hard to explain. In this activity, we get both an example of supercooling and heat of crystallization. Supercooling is where a liquid is cooled past the point it normally turns into a solid but is still a liquid. Any agitation or seed crystal will immediately cause the crystallization to begin. Heat of crystallization is the heat change when crystals form. Usually, this is an exothermic process. Once the crystallization begins, the heat of crystallization starts to heat up the container.}
\end{itemize}

\subsection{Hot Ice, Part C – Climbing Crystals}
\begin{itemize}
\item{Preparation time: 5 minutes}
\item{Materials: sodium acetate supercooled solution from Hot Ice, Part A - Production, a shallow container}
\item{Procedure: Carefully remove the cooled sodium acetate liquid from the refrigerator. Be careful not to disturb it greatly or let anything fall into the liquid. Once ready, gently pour the supercool sodium acetate solution into the second container. The crystals will start forming on contact, and if you pour slowly enough, will start to grow up the stream towards the main container of sodium acetate.}
\item{Theory: Supercooled solutions are ready to start crystallization at any moment. Any of the slightest disturbance or change in temperature can make crystals fall out of solution. We are using this fact to start the crystallization: through the combined factors of the change in temperature and the violent act of pouring this liquid into another container, crystallization occurs immediately. It is so ready for crystallization that it starts to grow up the pouring stream.}
\end{itemize}

\subsection{Hot Ice, Part D - Recycling}
\begin{itemize}
\item{Preparation time: 0 minutes}
\item{Materials: crystal solutions or crystals from the earlier Hot Ice, Part B – Supercool or Hot Ice, Part C – Climbing Crystals activities.}
\item{Procedure: Collect the remains of the crystal structures or solutions and add just enough water to dissolve the crystals. Heat the solution to boil away excess water until the consistency is much like in Hot Ice, Part A and refrigerate once again.}
\item{Theory: Crystallization is a reversible reaction. This fact can be demonstrated by the sheer fact that the chemicals used in these activities is recyclable is good to show. A reminder for safety: this chemical does not pose a danger for students to handle. It is not poisonous or harmful. Many cases it is a food additive. Further, the heat generation is not enough to burn students. In many reusable heat packs, this is actually the reaction heats the packs.}
\end{itemize}

\section{Density and Polarity of Liquids}

\subsection{Exploring Density, Part A - Temperature Dependent Density}
\begin{itemize}
\item{Preparation time: 20 minutes}
\item{Materials: 2 jam jars, 1 narrow clear glass container, water, two different food colorings, ice or refrigerator, electric or kerosene burner, 1 syringe with metal needle}
\item{Procedure: Place 200 mL of water in each jam jar. Place a few drops of the two different food colors in each jar, for example make one blue and one red. Take one jar and add ice or place it in the refrigerator. Heat the second jar for 20 minutes. Take a second clear, glass, tube or jar and add 50 mL of the cold solution. Using a syringe, take a few mL of the hot solution and add gently to the cold solution. This takes some practice to ensure the water does not mix. Use the metal needle to add one drop at a time so that the liquid runs down the side of the container. Two layers will form. }
\item{Theory: The density of all liquids is temperature dependent. Generally, they expand on heating and contract on cooling. Since density is defined as mass divided by volume, as a liquid is heated the volume increases, decreasing the density. As a liquid cools, the volume contracts, increasing the density. Therefore, in this case the hot water has a smaller density than the cold water. By adding the water carefully to prevent mixing, two layers can be seen in the water: an upper, hot water layer and a lower, cold-water layer. Keep in mind that these two layers are temporary. Water mixes perfectly with other water so as time goes on, the two liquids will mix and even out the temperatures.}
\end{itemize}

\subsection{Exploring Density, Part B - Densities of Different Water Solutions}
\begin{itemize}
\item{Preparation time: 15 minutes}
\item{Materials: 3 jam jars, 1 narrow clear glass container, water, table salt, sugar, three different food colorings, syringe with metal needle}
\item{Procedure: Place 200 mL of water into each of the jam jars. Place a few drops different food colorings in each jam jar. In one jar, place 1 spoon of table salt. In a second jar, place 1 spoon of sugar. In the third jar, add nothing. Mix to ensure all solutes are dissolved completely. Take a syringe and to the narrow container add ten mL of the salt-water solution. Carefully, add drop-by-drop ten mL of the sugar solution to the container in such a way to prevent mixing. Use a syringe needle to aid this process. Now, carefully add drop-by-drop ten mL of pure water. Adding these together requires practice to ensure the layers do not mix. Three layers should form. }
\item{Theory: Dissolving solutes in water will change the density of the liquid. Density is defined as mass divided by volume. If we use the same volume of each solution, the density of each depends on the mass of each solution. The mass of each solution depends both on the mass of the water and the mass of the solute dissolved in it. The pure water solution has no solute dissolved while the other two solutions have extra components dissolved: salt and sugar. One spoon of salt has a different mass than 1 spoon of sugar. In fact, the spoon of salt weighs more than the sugar. Therefore the salt solution will be the heaviest, then sugar, and then pure water will be on top.}
\end{itemize}

\subsection{Floating Eggs}
\begin{itemize}
\item{Preparation time: 5 minutes}
\item{Materials: 1 uncooked egg, 1 jam jar, water, salt}
\item{Procedure: Fill jam jar with water. Place egg in the water. The egg will sink. Add salt until the egg floats. }
\item{Theory: The density of an egg is greater than water. This is why the egg will sink. Since density is defined as mass divided by volume, the density of water can be changed by dissolving extra mass, salt in this case. As more and more salt dissolves in the water, the densities increases until the density of the water is greater than the egg and the egg floats to the surface.//
This is the same reason why it is more difficult to swim in fresh water than salt or ocean water. The extra salt or ions in the ocean water increase its density and making the body more buoyant. Since those ions are much fewer in fresh water, the density of fresh water is greater than salt water. The Dead Sea in the Middle East has so much salt in the water that people do not swim in the water; they just float. Lake Natron in Tanzania probably has similar properties.}
\end{itemize}

\subsection{Density Tower, Part A - Production}
\begin{itemize}
\item{Preparation time: 15 minutes}
\item{Materials: liquid soap, water, honey, cooking oil, propanone (acetone), kerosene, glycerin, methylated spirits, motor oil, 1 tall clear container. Volumes of each depend on size of the container.}
\item{Procedure: Carefully pour one the liquids down the side of the container. Pour it in a way that a small volume slowly goes down the side of the container in a gentle, nonviolent manner. Pour a second liquid in the same manner so that it gently rests on top of the other liquid. Some of the liquids will rise and some will fall. Be very careful so as to prevent mixing of the layers when pouring. This may take some practice. Further, some compounds will mix if poured on top of each other. It may be necessary to have intermediate layers to see which ones will mix even if poured very carefully. }
\item{Theory: Each component has its own density depending upon the composition of the liquid. Some sugar, oil, or other component will give a different density. These liquids will float or rise on top of each other depending on the other liquids density. In other words, the least dense sample will rise to the top of the tower and the compound with the least density will sink to the bottom. In this example, the glycerin is the densest sample while kerosene is the least dense. Experiment with the order of adding the different liquids; the order is important. If you add a layer that dissolves in another layer right on top of each other, like water and methylated spirits, they will combine to form one layer. To get a dramatic ordering seen in this diagram, start first with water. Add kerosene and then add cooking oil. If added properly, the oil will settle in between the water and the kerosene layer. This step is difficult, but possible. If it is not working, add oil first followed by kerosene. Add methylated spirits, and it will sink in between the oil and kerosene layer. Lastly, add glycerin and it will fall through both layers.}
\end{itemize}

\subsection{Density Tower, Part B – Adding More}
\begin{itemize}
\item{Preparation time: 5 minutes}
\item{Materials: The column of liquids of different densities from Density Tower, Part A - Production, small pieces of different materials like metal, wax, plastic, wood, cork, orange peel, sand, or anything that is locally available.}
\item{Procedure: Use the density tower from Density Tower, Part A - Production. Now, it is possible to compare the relative densities of solid materials. Drop in the materials piece by piece, carefully to prevent mixing of the layers. It is recommended to use food coloring with the water to see the layer better. The materials will float or sink to the layers that are close to their density.  }
\item{Theory: The solids will float or sink above or below the liquids. The flotation of each component is a direct example of differing densities. The less dense compound will rise to the surface while the denser will sink. In other words, a solid will float above layers with greater densities and sink below layers with smaller densities. This tower can show us the relative density of different materials we can find locally. For example, the matchstick floats on top of the methylated spirits level. This shows us that the density of the matchstick is greater than kerosene but is less dense than methylated spirits. In fact, since it floats above the methylated spirit layer, this tells us that the matchstick is less dense than all the layers below it, like water. Alternatively, all the layers below it are denser than the matchstick. It is possible to test for this by taking some water by itself and drop in a matchstick. The matchstick will float. Wax floats in between the methylated spirits layer and the cooking oil layer. This shows us that the density of the wax is greater than methylated spirits, but less than cooking oil. Try this with different plastics, as they have differing densities. In fact, this is how many plastic recyclers separate out different plastics from waste.}
\end{itemize}

\subsection{Density Tower, Part C- Mixing}
\begin{itemize}
\item{Preparation time: 0 minutes}
\item{Materials: column of liquids different densities from Density Tower, Part A - Production}
\item{Procedure: Place a lid, a stopper, or even a thumb over to the top of the tower so that it is leak proof. Shake the tower. All the layers will come together in one big emulsion. After some time, the layers will separate into two layers.}
\item{Theory: All the liquids are subject to intermolecular forces. The nature of those forces is related to the nature of the molecules. Components with charged groups are polar compounds, like water. Components without charged groups are called non-polar compounds. Liquids will mix together if the polarities are similar. This gives rise to the common rule of thumb ‘like dissolves like’, polar compounds dissolve polar compounds, and non-polar compounds dissolve non-polar. In the density tower from Density Tower, Part A, there are 3 non-polar layers: kerosene, glycerin, cooking oil. There are 2 polar layers: methylated spirits and water. When the layers are all mixed, they form an emulsion. An emulsion is a mixture of non-polar and polar compounds. The non-polar compounds form micelles to dissolve in the polar water layers. This happens when there is a molecule that is part way between being polar and non-polar. However, this is not the most stable organization. Slowly, the non-polar and polar compounds will separate yielding only two layers after a few minutes. The less dense, or upper layer, is the non-polar layer. The lower layer is the aqueous layer. Now, if there are the extra materials from Density Tower, Part B, there is no problem. A good set of density activities is to do each part of the Density Tower activities in consecutive order.}
\end{itemize}

\section{Dilution}

\subsection{Dilution, Part A – Salt Dilution}
\begin{itemize}
\item{Preparation time: 10 minutes}
\item{Materials: Salt, syringe, salt, water, 5 jam jars}
\item{Procedure: In one jam jar, place 100 mL of water and dissolve enough salt to make a saturated solution. With a syringe remove a 10 mL aliquot of the salt-water solution and put it into another jam jar. Add 90 mL of clean water. Now, take 10 mL of this diluted solution and put it into another jam jar with 90 mL of water. Repeat this procedure 3 more times. Taste each solution. In this activity, students will be eating so do not conduct this experiment in the laboratory, but in a classroom. Remember, no eating or drinking in the lab. For extra safety, utilize fresh syringes. Open them from their plastic wrapper and remove their needles.}
\item{Theory: When we are diluting, we have two things we need to think about: the solute and the solvent. The solute is the compound that dissolves into another compound, the solvent. For this activity, salt is the solute and water is the solvent. Dilution is the process where we start with a specific ratio of solute to solvent, and we add solvent to make that ratio of solute to solvent smaller. When we take 10 mL of our solution and add 90 mL of water, we have a specific amount of saltwater in the syringe and we add water. This changes the ratio of salt and water that we have in the jam jar. This is dilution. We can test this by tasting each of the jam jars. The first jar will be super salty. The second jar, third, forth, and fifth jar will have an increasing smaller ratio of salt to water. By tasting each jar, we can see that the salt is being diluted because they will be less and less salty than the previous jar.}
\end{itemize}

\subsection{Dilution, Part B – Color Dilution}
\begin{itemize}
\item{Preparation time: 10 minutes}
\item{Materials: food coloring, water, syringe, 5 jam jars}
\item{Procedure: Follow the same procedure from Dilution, Part A with one exception. Instead of using salt, use food coloring. As the dilution happens, the color will dilute little by little. This activity, unlike Dilution, Part A, involves no eating or drinking. This activity can be done in the classroom or the laboratory.}
\item{Theory: The explanation is the same as Dilution, Part A. The only difference is that the observed property is color, not taste.}
\end{itemize}

\subsection{Cement Making, Part A - Dilution}
\begin{itemize}
\item{Preparation time: 3 hours}
\item{Materials: cement, sand, water, plastic water bottles, large plastic container}
\item{Procedure: Place 1 volume of cement is a large plastic container or wheelbarrow. The actual volume of cement used is not too important, but use the same volume when adding sand. To that, add 1 volume of sand. Add water to make a paste and pour off into a plastic water bottle with the top cut off. This bottle acts as a mold for the cement. Then, add a second volume of sand to the large plastic container. Pour into a plastic water bottle. Add a third volume of sand, and then pour into a water bottle. Repeat this procedure until you have added 12 volumes of sand. Let the cement dry overnight and cut off the plastic water bottle. Label and keep each different piece of cement.}
\item{Theory: Making cement is an exercise in creating a proper ratio of cement to sand. This is very similar to dilution. In fact, by adding sand 1 volume at a time, we are making cement of different concentrations. Each concentration of cement is a little more diluted from the previous one. This gives the cement different textures and properties. The ratio of waterproof cement is a 1:4 of cement to sand. 1:7 is the standard ratio for normal cement. As the ratio becomes 1:10 and higher, the cement does not have enough binder to effectively hold the cement to the sand in one block. In fact, this ratio tends to flake very easily. This activity highlights this dilution of the cement. Progressively adding more and more sand will make the ratio of the cement to salt grower much larger.}
\end{itemize}

\subsection{Cement Making, Part B - Erosion}
\begin{itemize}
\item{Preparation time: 10 minutes}
\item{Materials: cement made from different ratios from Cement Making, Part A - Dilution}
\item{Procedure: Take each of the cement pieces and place them outside to bear the elements. Record each week the status of each piece of cement.}
\item{Theory: The ratio of the binder, cement, to sand decreases through each dilution. This means that the strength that holds the cement together decreases as the ratio of cement to sand increases. We can see this by leaving all the different pieces of cement outside to erode. The strongest pieces of cement will resist erosion the most. The pieces of cement that have a 1:10 ratio of cement to sand will erode very easily. This is why most cement blocks look like they are melting when it rains. The cement is too diluted to resist erosion effectively. Over the course of a year, the cement that has a 1:10 or 1:12 ratio will erode while the other pieces of cement will not erode.}
\end{itemize}

\section{Electrochemistry}

\subsection{Displacement Reactions, Part A - Metal Reactivity}
\begin{itemize}
\item{Preparation time: 10 minutes}
\item{Materials: jam jars or transparent reaction vessels, variety of chemicals, magnesium metal (optional), magnesium sulfate (Epsom salt), zinc metal from an unused D cell battery, zinc sulfate, iron nail, iron (II) or (III) sulfate, copper wire, copper (II) sulfate, silver nitrate (optional)}
\item{Procedure: Place a piece of metal from the list of metals at the bottom of a jam jar. Then pour solution of a different chemical from the list of solutions in the jam jar to cover the metal. In five or ten minutes, a precipitate will form on the metal if the proper combinations of chemicals are used. For each of these solutions, a 1\% by mass solution is more than enough to see the precipitates. There is a variety of combinations to try:
\begin{itemize}
\item{Metals: magnesium, zinc, iron, copper}
\item{Solutions: magnesium sulfate, zinc sulfate, iron (II) or (III) sulfate, copper (II) sulfate, silver nitrate}
\end{itemize}
A sample activity: take an iron nail. Place at the bottom of the jam jar. Cover the nail with copper (II) sulfate. In 5 minutes a reddish brown precipitate of copper will form. Repeat this activity but instead of copper (II) sulfate, use magnesium sulfate. There will be no precipitate this time.}
\item{Theory: This activity is a direct example of displacement reactions. The metal ion in the solution will reduce and the metal solid will oxidize if organized the proper combination is used. Oxidation means that the compound loses electrons and reduction means that the compound gains electrons. In the first sample activity, the iron metal will oxidize to form iron (II) ions in solution and the copper (II) ions will reduce to form copper metal, precipitating out on the nail. This is not rust, but copper precipitate. It is easily rubbed off to show that it is copper. When the activity is repeated with magnesium sulfate, nothing will happen. This is an example table of electrochemical potentials: ReactionPotential Table//
In this table, E refers to electrode potential. Electrode potential is the ability of metal ions to reduce when compared to the reduction of hydrogen. Another way to think about this is how strongly metals grab electrons in electrochemical reactions. The stronger metal always wins. The values are sometimes negative because zero was assigned to the electrode potential of hydrogen. Remember that metal ions are in the solution, not on the piece of metal.//
We can use this table by the thinking about the reduction potentials. A negative number means that the reaction proceeds in the negative direction. They can be calculated by the formula . By looking at the electrode potential for the cells, it shows which reaction will occur and which will not. To simplify this table, look for the ion and the metal. If the metal is higher on the table than the ion, the ion will reduce and the metal will oxidize.//
This is effect is important because it is the science behind the reactivity series. The reactivity series describes how the metals will interact. However, the reactivity series does not explain the science behind the reactions. This electrochemical table both describes the series and explains the science behind it.}
\end{itemize}

\subsection{Displacement Reactions, Part B – Reactivity Series}
\begin{itemize}
\item{Preparation time: 30 minutes}
\item{Materials: Magnesium ribbon, zinc sulfate solution, iron (II) sulfate solution, copper (II) sulfate solution, silver nitrate solution}
\item{Procedure: The same procedure as Displacement Reactions, Part A – Metal Reactivity with a small change: instead of doing just one displacement in the jam jar, we are going to do all the displacements one by one. In a jam jar, place a strip of magnesium ribbon. Cover with zinc sulfate solution. Wait one day to allow all the zinc to reduce. Carefully remove the ribbon, now coated with zinc, and transfer to a clean jam jar. Cover the ribbon coated with zinc with a solution of iron (II) sulfate. The zinc will oxidize and the iron reduces as it plates out on the ribbon. Now add some copper (II) sulfate and wait a day. The iron will now oxidize while the copper reduces to copper metal. Finally, add some silver nitrate and wait a day. The copper will oxidize back to copper (II) ions and the silver will precipitate out.}
\item{Theory: This specific procedure is very interesting one to follow. This is a good series of reactions to show the reversibility of the electrochemical reactions. In one reaction, the metal will reduce. However, when we add another metal, that very same metal will oxidize.}
\end{itemize}

\subsection{Electrolysis of Water, Part A – The H and the O}
\begin{itemize}
\item{Preparation time: 30 minutes}
\item{Materials: 3 or 4 live d cell batteries, 2 dead d cell batteries, water, Epsom salt or baking soda, wires, 1 LED light for testing the circuit}
\item{Procedure: Make a salt solution with Epsom salt or baking powder. The actual concentration does not matter, but if it is too dilute, it will cause some difficulties in the electrolysis. Take apart the two dead d cell batteries and collect the two graphite electrodes. Place the live batteries in series and connect two wires: one for the positive end and one for the negative. Run the positive line to one electrode and the negative one to the other electrode. Place both electrodes in the Epson salt or baking soda solution without touching each other. Bubbles should form at both electrodes. If no bubbles form, this may be caused from a few different problems. First, check if the batteries are indeed live. Second, use the LED light to check each connection. Third, ensure the solution is concentrated enough to allow the electrons to move between electrodes. }
\item{Theory: In this particular cell, the reaction follows the hydrolysis of water. In cells like this one, the cathode is the electrode where oxidation occurs and the anode is the electrode that reduction occurs. At the cathode, hydrogen reduces according the half cell reaction:. This is the production of hydrogen gas in the hydrolysis of the water. At the anode, the oxidation of the hydroxide ion occurs. The half-cell reaction is. The hydroxide oxidizes to from water and oxygen gas. This can be seen at the anode. The purpose of the magnesium sulfate is to provide a salt in order to allow electrons to flow through the solution and complete the circuit. Since water has poor electrical conductivity, a strong electrolyte needs to be added in order to raise the conductivity. In this activity, baking soda or Epsom salt act as that electrolyte. Further, these electrolytes will not influence the reaction.}
\end{itemize}

\subsection{Electrolysis of Water, Part B – Capturing H and O}
\begin{itemize}
\item{Preparation time: 60 to 90 minutes. }
\item{Materials: Same as Electrolysis of Water, Part A – the H and the O, two syringe shells.}
\item{Procedure: The procedure is the same as Hydrolysis of Water, Part A with one exception. In this activity the gases, both hydrogen and oxygen, can be captured. The needle and the plunger are not needed. Run a wire through the top of the syringe shell to the batteries. Seal the top of the shell to prevent any gas from leaking. Repeat for the second syringe shell. Ensure the electrode is held in place by either gluing or using the wire for the cell. Place both syringe shells and electrodes in the Epsom salt solution. The gas collection is very visible by looking at the displacement of the solution in the syringes. }
\item{Theory: Hydrolysis of water is now captured in this activity. Hydrogen gas is collected at the cathode and oxygen is collected at the anode. If the volumes of both the hydrogen and the oxygen are measured, the hydrogen gas will have twice the volume of the oxygen gas. This is due to the complete reaction for the hydrolysis of water. The reaction for the hydrolysis of water is found by adding the half-cell reaction at the cathode with the half-cell reaction at the anode. Adding the two half-cell reactions yields the reaction. In fact, this reaction can be demonstrated visually by looking at the production of the two gases. Since the reaction states that for 1 mole of oxygen gas, 2 moles of hydrogen gas are produced, expect that the volume of hydrogen will be twice that of oxygen. Historically, this was one of the proofs that water consists of both hydrogen and oxygen. Further, it also confirmed that the ratio of hydrogen to oxygen in water molecules is 2 to 1.//
The purpose of the syringes is to collect the gas to be tested at a later time. Carefully remove the syringes keeping in mind that hydrogen is less dense than air and oxygen has about the same density as air. The test for hydrogen gas is the pop test. Take a match or a piece of burning paper and put it in the hydrogen gas. The hydrogen combusts in air, causing a pop sound. To test for oxygen through the glowing splint test; lower a glowing piece of paper in and see if the paper relights. It is possible the syringes might melt. }
\end{itemize}

\subsection{Electrolysis of Salt Water; Production of Chlorine}
\begin{itemize}
\item{Preparation time: 30 minutes}
\item{Materials: Same as Electrolysis of Water, Part A – the H and the O, table salt  }
\item{Procedure:  it is same as Electrolysis of Water, Part A – the H and the O, with one change: table salt instead of Epsom salt or baking soda. On the cathode, hydrogen will form. On the anode, chlorine gas is produced. Chlorine is a green gas that is poisonous with a pungent smell.}
\item{Theory: The half-cell reaction for the cathode is. The half cell reaction is. The overall electrochemical reaction is. In the Electrolysis of Water, Part A, the reducing agent was hydroxide ion but in this activity, chlorine gas is the reducing agent. It is possible to capture the chlorine gas by using the same apparatus as Electrolysis of Water, Part B. Keep in mind safety. Do not produce chlorine in a small, poorly ventilated room. Further, do not breathe chlorine gas directly.//
As the chlorine gas reacts with the hydroxide ion in the water, it will form hypochlorite (ClO-). Together with the sodium cations from the salt, this is sodium hypochlorite, the active ingredient in bleach. Industrial bleach production is the same process at much larger scale.}
\end{itemize}

\subsection{Electroplating Copper}
\begin{itemize}
\item{Preparation time: 10 minutes}
\item{Materials: 3 to 6 live batteries, metal object to be plated like a spoon, copper metal, wires, baking soda or Epsom salt, LED light for testing connections}
\item{Procedure: Connect the metal object to be copper plated to a wire. Run this wire to batteries. Run a wire to the other side of the batteries and connect this to a piece of copper. If using copper wire as the source of copper, remove the insulating coating to allow the copper metal to have contact with the solution. Place a salt solution, Epson salt or baking soda, in a container. Submerge the object to be copper plated and the copper metal in the solution. In a short time, copper should appear on the metal object. If run for a long time, this reaction will completely plate the metal object and the copper metal will disappear. If there is no reaction occurring in the cell, use a LED light to test each individual connection to ensure a close circuit.}
\item{Theory: In the previous activities of electrolytic cells, a chemical in the solution either oxidizes or reduces to form the product. In this particular cell, the copper metal at the anode will first undergo oxidation to form copper ions, Cu2+. The copper ions will migrate towards the cathode where reduction takes place. The copper ions gain electrons to form copper metal once again. The copper will form on the cathode object, and the reddish brown color of copper metal becomes quite visible. With enough time, the entire object will be covered with copper metal. Given enough time and battery power, the entire copper metal will react, disappear, and plate out on the object. In the sample set up, a spoon will be copper plated. For this activity to work, any electrically conducting object will work: a graphite electrode, a piece of metal, a spoon, etc. A normal household utensil is recommended. To clean the object, the copper metal will just rub off. This is the process for all different types of metal plating. Chrome plating uses a cell like the one used in this activity except chromium is used instead of copper. A process called galvanization can counter rusting. Galvanization is the electroplating of a metal, which does not rust, on top of another metal that is prone to rusting. For example, galvanized nails are iron nails with zinc electroplated onto the surface of the iron.}
\end{itemize}

\section{Flame Tests}

\subsection{Flame Test, Part A – Colored Flames}
\begin{itemize}
\item{Preparation time: 5 minutes}
\item{Materials: lighter, salts of different metals, like CuSO4}
\item{Procedure: Take a pinch of salt and sprinkle it into the flame of the lighter. Watch the flame; its color will change depending on metal in the salt.}
\item{Theory: The electrons in metal atoms move to different orbitals (energy levels) when heated. As the metal cools, the electrons fall to a lower orbital. When the electrons fall to a lower level, they release the difference in energy as a photon. Sometimes these photons are in the visible spectrum. This gives a color to the flame. Different metals give different colors because the wavelength of the photons depends on the energy between orbitals in the atom. Electronic configurations and atomic charges lead to the differing energies between the orbitals giving rise to the different colors.
\begin{itemize}
\item{Lithium compounds give a red color when placed into a flame. See the instructions for extraction lithium from old cell phone batteries in Chemicals in Shika na Mikono Volume 1.}
\item{Calcium hypochlorite is bleaching powder. When placed into a flame, it gives an orange color. Calcium hydroxide should also produce an orange color. Gypsum is calcium sulfate.}
\item{Sodium compounds, like salt or sodium carbonate, give a yellow color to a flame. Even though a lighter has a yellow part of the flame, there is a greater yellow color.}
\item{Boric Acid is composed of boron, which gives a green color when it burns.}
\item{Borax is sodium borate, so when in the flame test it gives a yellowish green color. The yellow is from sodium and green from the boron.}
\item{Copper sulfate gives a green color in the flame while copper chloride gives a blue color.}
\item{A mixture of 3 parts potassium sulfate and 1 part potassium nitrate gives a purple color. The purple flame from the potassium. This mixture of potassium sulfate and potassium nitrate is necessary because potassium sulfate does not burn by itself and potassium nitrate is saltpeter, which is explosive by itself. However, Potassium chloride gives a purple flame by itself.}
\item{Magnesium ribbon, solid magnesium, also burns with a white flame.}
\item{Aluminum also burns white. For aluminum, it will not burn unless aluminum solid is used. Aluminum oxide does not give a color due to the protective oxide on the outside of the metal.}
item{Zinc metal gives a green color in a flame.}
\end{itemize}
} % Theory
\end{itemize}

\subsection{Flame Test, Part B – Colored Fireballs}
\begin{itemize}
\item{Preparation time: 10 minutes}
\item{Materials: iron wool, balloon, battery acid, salt for flame test. Matches}
\item{Procedure: Put some salt in a balloon. This is the salt for the flame test. Use the balloon and fill it with hydrogen gas; use a water bottle with iron wool and battery acid to make hydrogen and let it flow through an IV tube into the balloon with the salt. Once the balloon has some hydrogen, bring it to a flame. See Hydrogen Production, Part B.}
\item{Theory: Instead of using a flame from a lighter, we can burn hydrogen gas to make the flame test occur. This variation on the activity is fun because when the hydrogen balloon combusts, it will give a colored flame. }
\end{itemize}

\subsection{Flame Test, Part C - Spritzer}
\begin{itemize}
\item{Preparation time: 20 minutes}
\item{Materials: hard alcohol of at least 35\% alcohol by volume (methylated spirits will work well), salts for flame test, spray bottle, any high temperature open flame, like motopoa, Bunsen, or kerosene burner.}
\item{Procedure: Put some salt inside a spray bottle. Dissolve it with some hard alcohol. Spray this solution into a burner. There should be a flame test.}
\item{Theory: This variation of the flame test uses alcohol as a medium for the metal ions. As the alcohol burns, it leaves behind the metal ions that produce the color in the flames. }
\end{itemize}

\section{Fun Liquids}

\subsection{Making Borax}
\begin{itemize}
\item{Preparation time: 20 minutes}
\item{Materials: boric acid, either in powder form or as a liquid found in bottles (see the procedure for more explanation), baking soda}
\item{Procedure: Dissolve some powdered boric acid the smallest amount of water possible. Carefully add baking soda without adding excess. Let the white precipitate settle. Decant off the liquid leaving behind the powder. Let it dry. This is sodium borate or borax. If there is no powdered boric acid available, it is possible to buy antifungal boric acid solutions. These are also good sources of boric acid. Add, without adding to excess, baking soda to the liquid. There will be no effervescence. Look for precipitate formation. Adding baking soda slowly so it dissolves and white precipitate forms. When excess baking soda is added, the soda will dissolve but no new white precipitate will form. Let the precipitate settle, decant, and let the solid dry. }
\item{Theory: Borax is sodium borate. This is formed easily by neutralizing the boric acid with the baking soda since the carbonate will form carbon dioxide. Use this solid whenever one of these activities calls for borax.}
\end{itemize}

\subsection{Slime}
\begin{itemize}
\item{Preparation time: 20 minutes}
\item{Materials: warm water, hot water, glue, food coloring, metal spoon, borax from above}
\item{Procedure: Dissolve a small spoonful of borax into 250 mL of hot water. Stir to dissolve. In another container, mix 500 mL of office or clear glue with 500 mL of warm water with a few drops of food coloring. With a metal spoon beat the borax solution into the glue solution. It will be difficult at first, but will become easier with more work.}
\item{Theory: This is a fun polymer to make since it has such a unique texture. Enjoy playing around with this substance; it is safe to touch. This is an example of a polymer, one that may be formed from everyday materials. When you are done playing, store the slime in an airtight container. Do not eat, and do no throw down a drain. To dispose, put it in the trash.}
\end{itemize}

\subsection{Flubber}
\begin{itemize}
\item{Preparation time: 20 minutes}
\item{Materials: borax, warm water, glue, food coloring}
\item{Procedure: In one container, mix 300 mL of water with 500 mL of glue and some food coloring. In a second container, mix 300 mL of water with 3 spoonfuls of borax. Mix each container thoroughly and then gently mix together. The flubber will be sticky at first. Let the excess liquid drain, and begin to play.}
\item{Theory: This is a variant of the Slime activity. The difference between these two is the ratio of borax to glue. By changing this ratio, the texture and consistency of the flubber will also change. When you are done playing, store in an airtight container. Do not eat, and do no throw down a drain. Put it in the trash.}
\end{itemize}

\subsection{Making Oobleck}
\begin{itemize}
\item{Preparation time: 10 minutes}
\item{Materials: cornstarch, water, newspaper, food coloring, shallow mixing bowls, hammer}
\item{Procedure: This activity is difficult to have a defined procedure, since it relies so much on the quality of cornstarch. Begin by putting 200 g of cornstarch in a mixing bowl. Slowly add water mixing the entire time. It is better to use your hands for this process. Add water until the consistency becomes something like honey and that it gives a little resistance and tears on the surface of the mixture. Add food coloring here if you want it to be colored. Use newspaper on all the surfaces to prevent a mess.}
\item{Theory: Oobleck is a non-Newtonian fluid. It is a suspension of starch molecules with water. It is not like a sugar water solution where the sugar dissolves in the water, the starch particles are suspended in the water. If you do not stir long enough, the cornstarch will settle to the bottom.@	Use this procedure for making oobleck for the following activities. It is very important to remember not to pour this suspension down the drain. It will clog drains. It also tastes bad, so do not eat it, and throw it away when you are done with it.}
\end{itemize}

\subsection{Exploring oobleck}
\begin{itemize}
\item{Preparation time: 0 minutes}
\item{Materials: Oobleck from Making Oobleck, bowl, hammer}
\item{Procedure: In the mixing bowl, gently move your fingers through the oobleck. Observe the texture. Grab a handful of oobleck and quickly squeeze it. It will become like a rock. Release the pressure and the oobleck will flow out of the hands like a liquid. Slap the oobleck and observe the texture. Remove all hands from the oobleck and hit it with a hammer. The hammer will bounce on the surface. Do it again with the hammer, but instead of swinging it, let it gently rest on the oobleck and watch it sink.}
\item{Theory: All fluids have a property called a viscosity constant. This describes the fluidity of the liquid. When this viscosity constant remains constant, the liquid is described as a Newtonian liquid. Oobleck, on the other hand, changes viscosity when different amounts of pressure are applied. If a student pokes the oobleck or hits it with a hammer or applies a large force in a small amount of time, the oobleck increases its viscosity greatly and stays in place. If a student applies no pressure or pours it gently, the viscosity becomes similar to water or honey. When a fluid acts like this, it is called a dilatant material. \\
Oobleck is believed to behave this way due to kinetics between the starches and the water. When sitting still the granules of starch are surrounded by water. The surface tension of the water keeps it from completely flowing out of the spaces between the granules. The cushion of water provides quite a bit of lubrication and allows the granules to move freely. But, if the movement is abrupt, the water is squeezed out from between the granules and the friction between them increases rather dramatically. This makes the oobleck becomes almost solid. }
\end{itemize}

\subsection{Walking on Water}
\begin{itemize}
\item{Preparation time: 30 minutes}
\item{Materials: lots of cornstarch, water, big shallow pool or basin, kangas}
\item{Procedure: Make enough oobleck to fill a bottom of a basin or pool. Place some kangas around the basin. When it is ready, take off your shoes and walk on the oobleck. If you walk quickly, you will walk on the top easily. If you walk slowly, you will sink in the oobleck.}
\item{Theory: This is basically the oobleck activity from earlier, but it is on a larger scale. The students will be amazed when you can walk on water.}
\end{itemize}

\subsection{Dancing Oobleck}
\begin{itemize}
\item{Preparation time: 10 minutes}
\item{Materials: oobleck from Making Oobleck, a speaker, plastic wrap or a plastic bag}
\item{Procedure: Turn a speaker so that its speaker is facing straight up. Cover with plastic wrap or a plastic bag. Be absolutely sure nothing can leak or fall into the speaker. Pour on some oobleck. Turn on the speaker and play some music that has some deep base notes. The oobleck will bounce to the beat.}
\item{Theory: As the music plays, the sound from the speakers transfers its kinetic energy into the oobleck. Due to the nature of its composition, the oobleck will respond to the music by bouncing to the beat.}
\end{itemize}

\section{Gas Laws}

\subsection{Boyle’s Law, Part A – Syringe}
\begin{itemize}
\item{Preparation time: 0 minutes}
\item{Materials: One syringe of any size minus metal needle}
\item{Procedure: Fill the syringe with air until the end of the graduations. Place a finger at the tip of the syringe to create a seal. Press the plunger as far as possible. Make a competition with the students to see which person can decrease the volume the greatest. It should be easy to decrease the volume most of the way but impossible by human means to completely squeeze out the air.}
\item{Theory: Boyle’s law states that the pressure of a gas at a constant temperature is inversely proportional to the volume. As the pressure increases, the volume decreases. As the pressure decreases, the volume increases. As the plunger pushes down on the gas, the volume decreases. As the progressively decreases, the pressure to push the plunger progressively increases. The pressure becomes so great that it is hard to puss the plunger in all the way. }
\end{itemize}

\subsection{Boyle’s Law, Part B – Cartesian Diver}
\begin{itemize}
\item{Preparation time: 5 minutes}
\item{Materials: 1 clear plastic water bottle with cap that forms a good seal, syringe, some weights like small nuts or nails, water}
\item{Procedure: Fill the water bottle completely with water. The water should be at the brim. Place the syringe, with the inside loaded with some weights and some air, carefully in the top of the bottle. The wings on the syringe may need to be cut in order to make the syringe fit through the bottleneck. It might bob out of the top of the bottle a little. Seal the bottle with the cap. Squeeze the bottle. The syringe should sink. Release the pressure and the syringe rises again. If the force required to squeeze the bottle in order to make the syringe sink is too great, there are two things to be done. First, ensure that the water is completely to the brim of the bottle. Second, adjust the size of the bulb on the syringe to minimize the volume of air. Find a new syringe if the syringe leaks water on the inside.}
\item{Theory: An object will float or sink depending on its density relative to the liquid it is in. In this case, if the syringe is less dense than water, it floats. If the density becomes greater than the density of water, it will sink. This is exactly what is happening inside the syringe. Inside, there is some air trapped on the inside. This gas is subject to Boyles’ law. Applying pressure to the bottle increases the pressure pushing on the syringe. This pressure makes the volume of the syringe contract. Since density is defined as the mass over the volume, by squeezing the bottle the density changes. The mass does not change, but the volume of the syringe decreases because the volume of air is compressed. As the volume decreases, the density of the syringe increases. If the density increases sufficiently, the syringe sinks. This is also known as a Cartesian Diver.}
\end{itemize}

\subsection{Boyle’s Law, Part C – Filling a Balloon}
\begin{itemize}
\item{Preparation time: 0 minutes}
\item{Materials: 1 bottle, 1 balloon}
\item{Procedure: First, blow up the balloon to stretch out the balloon and show that there are no holes. Release all air. Stretch the balloon such that it hangs in the bottle. Have students try blow up the balloon inside the bottle. It is impossible for a normal person to fill this bottle.}
\item{Theory: This is another good example of Boyle’s Law. Usually when balloons are used, we think of the gas inside the balloon. However, this time we are concerned with the air inside of the bottle. By filling the balloon, the air of the balloon increases. This means that the volume of the air inside the bottle decreases. In order to decrease the volume of the air inside the bottle, Boyle’s Law says that the pressure needs to increase. The normal human’s lungs cannot blow enough air at a high enough pressure to fill the balloon inside the bottles. }
\end{itemize}

\subsection{Boyle’s Law, Part D – Sucking a Balloon}
\begin{itemize}
\item{Preparation time: 10 minutes}
\item{Materials: balloon, plastic water bottle, straw}
\item{Procedure: In the plastic water bottle, put a straw through part of the wall. Seal it up so that it does not leak air. Place a balloon over the mouth of the bottle so it falls into the bottle. Use the straw to suck air out of the bottle to have the balloon fill with air.}
\item{Theory: As we suck the air out of the bottle, the volume of the air inside of the bottle gets smaller due to Boyle’s Law. The atmospheric pressure compensates by pushing the balloon into the bottle, which fills up with air.}
\end{itemize}

\subsection{Charles’ Law, Part A – Coin Cap}
\begin{itemize}
\item{Preparation time: 30 minutes}
\item{Materials: 2 coins, 2 bottles, some way to cool the air in one bottle either through refrigerator or ice.}
\item{Procedure: Take a coin and a bottle. Place the coin over the mouth of the bottle so it covers the entire opening. This is the control bottle. In a second bottle, place the bottle coin a refrigerator with the coin next to it. After 25 minutes, the air inside of the bottle cools down. Remove the bottle from the refrigerator and immediately cover with a coin as before. If no refrigeration is available, take some ice water or cold water and pour into the bottle. Swirl and mix the cold water to ensure the bottle is cold. Pour out the water and cover with the coin. Let the two bottles sit next to each other. After a short time, the coin on the refrigerator bottle will be blown off the top. }
\item{Theory: Charles’ law states that temperature is directly proportional to volume. As the temperature increases, the volume increases. As temperature decreases, the volume decreases. In this activity, the temperature of the first bottle remains constant so nothing happens. However, the air in the second bottle is at a lower temperature so it has less volume. When the temperature increases, the volume of the air expands in volume. This is shown by the coin being pushed off the lid of the container. The air expands but the coin stands in the way. The air pushes the coin so that it is possible to expand further in volume.}
\end{itemize}

\subsection{Charles’ Law, Part B – Spray Time}
\begin{itemize}
\item{Preparation time: 15 minutes}
\item{Materials: 1 can of non-CFC aerosol spray (e.g. Rungu insect repellent), 1 balloon.}
\item{Procedure: Place the plastic bag or balloon to act as a container over the mouth of the spray container. Use the container and spray it into a balloon. If the balloon is too small, use a funnel. The spray will liquefy and be cold inside the balloon. Tie the balloon. As the liquid warms up to room temperature, it will change from a liquid to a gas. Students should be able to hear and feel it boiling. Further, as the gas heats up, the balloon will expand in size.}
\item{Theory: Charles’ Law states that temperature of a gas at constant pressure is directly proportional to volume. Inside of the spray cans, there is a chemical held under high pressure. Phase diagrams show that gases under high pressure become liquids. The pressure is released and the temperature cools. This is called Joule-Thompson effect. It is an adiabatic expansion. However, by spraying long enough the temperature will cool down to the point that the chemicals will change back to a liquid. This liquid can be transferred to the balloon where it changes back into a gas quickly. This is where Charles’ Law comes into play. As the gas comes to room temperature, the volume of the trapped gas will increase.\\
This activity also works quite well for showing phase transitions. As liquids change to a gas, they do not disappear. They still exist even though they may not be seen. Here, the liquid changes to a gaseous state, which accompanies the expansion of the balloon. The size of the balloon is the direct representation of the liquid molecules that have evaporated. The mass of the balloon will also be different than that of one filled with normal air.}
\end{itemize}

\subsection{Charles’ Law, Part C – Bottle Crush}
\begin{itemize}
\item{Preparation time: 10 minutes}
\item{Materials: water bottle, boiling water}
\item{Procedure: pour some boiling water into the water bottle. Cap the bottle and shake to make sure all the air in the bottle is heated from the hot water. Open the bottle and pour out the liquid. Recap the bottle. After a short time, the bottle will contract.}
\item{Theory: Charles’ law states that volume is proportional to temperature. By capping the hot air inside of the water bottle, the volume of the air inside the bottle will decrease as the temperature of the gas cools off. As the volume of the air reduces, the atmospheric pressure crushes the plastic water bottle.}
\end{itemize}

\subsection{Egg Suck}
\begin{itemize}
\item{Preparation time: 0 minutes}
\item{Materials: 1 pealed egg, 1 glass bottle with a narrow mouth, like a konyagi bottle, matches, small piece of paper}
\item{Procedure: With the egg ready to cap the alcohol bottle, use a match to light the piece of paper on fire. Drop the paper into the alcohol container. Let it burn for a second, and then cap the bottle with the egg. The egg should be sucked slowly into the bottle if the egg is not too large. If it does not pull into the bottle, you can try again but use petroleum jelly on the mouth. Even if the egg is not sucked into the bottle completely, there will be a suction holding the egg to the bottle. It is possible to lift the bottle by the egg.}
\item{Theory: The burning match and paper heat the air inside the bottle. When we cap the bottle with the egg, we seal the air inside of the bottle. This air sealed inside is at a higher temperature than the surroundings. As the bottle cools down, the pressure of the air inside the bottle decreases. This is a direct example of Charles’s law. As the pressure inside drops, the atmospheric pressure still pushes down onto the egg. If pressure difference is sufficient, the egg will be pushed slowly into the bottle. }
\end{itemize}

\subsection{Avogadro’s Law, Part A – The Breath}
\begin{itemize}
\item{Preparation time: 0 minutes}
\item{Materials: 2 balloons}
\item{Procedure: Take one balloon and blow one breaths worth of air into it. Tie it off. Take the second balloon. This time, blow two breaths worth of air into the second balloon. Tie it off. The second balloon will be larger than the first. }
\item{Theory: Avogadro’s Law states that for a gas at constant temperature and pressure, the volume is directly proportional to the number of moles. Moles are a reference for the number of particles, or it can be thought of the amount of stuff. In other words, this law basically says the volume of the gas is related to the number of gas molecules. Here, one breath of balloon has fewer air molecules than the balloon of two breaths. Therefore, the volume of the one-breath balloon is smaller than the volume of the two-breath balloon.}
\end{itemize}

\subsection{Avogadro’s Law, Part B – The Spray}
\begin{itemize}
\item{Preparation time: 10 minutes}
\item{Materials: 2 balloons, 1 can of non CFC aerosol spray.}
\item{Procedure: This is the same procedure as Charles’ Law, Part B – Spray Time, except that the amount of spray is controlled. Once the spray starts to become a liquid, spray it for 5 seconds longer. Tie off the balloon. Repeat the same except extend the spray time to 10 or 15 seconds. Tie off the balloon. The second balloon will be larger.}
\item{Theory: The longer the spray is collected, the more of the liquid is collected. As more of the liquids collects, it means that more of the molecules are placed inside of the balloon. When the liquid evaporates and expands, the balloon with more molecules will have the larger volume. This is expected from Avogadro’s Law. The larger balloon should be heavier.}
\end{itemize}

\section{Gas Production}

\subsection{Hydrogen Production, Part A – In the Bottle}
\begin{itemize}
\item{Preparation time: 10 minutes}
\item{Materials: steel wool, citric acid, plastic water bottle, matches}
\item{Procedure: Put some steel wool in the bottom of a plastic water bottle. Pour in enough concentrated citric acid solution to cover the iron. Crush the bottle and cap. The bottle will refill. If using citric acid, you can simply invert the bottle, drain the acid into a container, and bring a lit match to the open bottom. There should be a pop sound. If using battery acid, or any strong acid, do not do this experiment in this way! The acid can burn hands, clothing, eyes, etc. Instead, do Part B below.}
\item{Theory: Hydrogen is an easy and fun gas to make. Steel wool, iron, reacts with acid based on the following reaction (sulfuric acid is used as a model): . By crushing the bottle, it is easy to see that the bottle is filling with a gas. The most common test for hydrogen is the ‘pop’ test. This test involves taking a lit splint, match, or burning piece of paper and taking it to the hydrogen. In this activity, open the lid, and have a student bring the match to the mouth of the bottle. Do this quickly since hydrogen is lighter than air. Every can hear a definite pop sound. This pop sound comes from the combustion of hydrogen in air,. Do not just throw out the waste solution; we need to neutralize any excess acid with baking soda. Also, this activity is a little more dangerous so take caution in deciding which students should be doing this activity.}
\end{itemize}

\subsection{Hydrogen Production, Part B – In a Balloon}
\begin{itemize}
\item{Preparation time: 10 minutes}
\item{Materials: steel wool, battery acid or citric acid, plastic water bottle, IV line, balloon. matches}
\item{Procedure: Run an I.V. line through the cap of the water bottle. On the other end of the line, connect a balloon. Follow the instructions from Hydrogen Production, Part A but instead of crushing the bottle, cap the bottle with the cap with the IV line. In this activity, the balloon will fill with hydrogen. Once the balloon fills with hydrogen, bring a match near the surface of the balloon. This balloon will combust in a small, but flashy flame. This also can be quite loud, depending on the size of the balloon. Note that concentrated citric acid is much safer than battery acid, but the reaction happens more slowly. In some balloons, the hydrogen will leaks faster than it is produced.}
\item{Theory: This activity is very similar to the previous one, except the hydrogen is moved from the bottle to the balloon, making it portable, and also removing the risk of pouring it out. If you tie the end of the balloon, the hydrogen can be captured for a short period of time. It may be possible to create a balloon that floats, or at least sinks less quickly than a normal balloon, by first crushing the bottle and filling the balloon with enough hydrogen. This is not an effective way to store hydrogen for another day as the molecules diffuse out of the balloon over time.}
\end{itemize}

\subsection{Oxygen Production, Part A – Manganese Catalysis}
\begin{itemize}
\item{Preparation time: 10 minutes}
\item{Materials: D cell battery, old or new, hammer (or rock), hydrogen peroxide, match and paper, plastic water bottle}
\item{Procedure: Smash open the battery. Inside there is a black powder. This black powder is manganese (IV) oxide, MnO2. It may be necessary to peal back a metal casing with a pair of pliers. Be sure to clean them afterwards since manganese dioxide is corrosive to metal. Scoop out the powder and put in the bottom of the plastic water bottle. Pour in some hydrogen peroxide, maybe 20 mL at most. Crush the bottle and cap. The bottle will inflate with oxygen gas. Test for oxygen using a glowing splint.}
\item{Theory: Hydrogen peroxide is a rather reactive compound. It can decompose into water and oxygen rather easily. The reaction is. The manganese acts as an inorganic catalyst. This means that the MnO2 speeds up the production of oxygen. As the reaction proceeds, the bottle will fill with a gas. To test for oxygen, there is a test called the glowing splint. Take a piece of paper, roll it like a cigarette, and ignite the end. Once it is burning for a few seconds, blow it out; we are looking for the glowing red color in the paper. Once you have this, slowly lower this glowing red part of the paper into the bottle. The paper will reignite. Remove the paper, blow out the flame, and lower the paper back into the bottle. It should reignite a second time.}
\end{itemize}

\subsection{Oxygen Production, Part B – Yeast Catalysis}
\begin{itemize}
\item{Preparation time: 10 minutes}
\item{Materials: --}
\item{Procedure: --}
\item{Theory: See Catalase Enzyme, Part A in the biology section for procedure for making oxygen using yeast and hydrogen peroxide. }
\end{itemize}

\subsection{Oxygen Production, Part C - Elephant Tooth Paste}
\begin{itemize}
\item{Preparation time: 0 minutes}
\item{Materials: fused syringe shell, powdered soap, yeast hydrogen peroxide, food coloring}
\item{Procedure: In the bottom of a syringe shell, mix some powdered soap and yeast together. If desired, let a drop of food coloring down the inside of the syringe shell. Pour into some 5 mL of hydrogen peroxide. Bubbles will start coming out of the syringe}
\item{Theory: Catalase decomposes hydrogen peroxide to form water and oxygen gas. The soap (dissolved in the water from the peroxide solution) traps the gas in bubbles. These bubbles build up upon each other slowly forcing them out of the syringe shell. If you use food coloring in the process, the bubbles will have a color stripe in the side of the bubbles. This will make the stream of bubbles coming out of the syringe look much like tooth paste. Of course, this is not toothpaste, and hazardous to put in your mouth.}
\end{itemize}

\subsection{Carbon Dioxide Production, Part A – Baking Soda}
\begin{itemize}
\item{Preparation time: 10 minutes}
\item{Materials: 1 jam jar, baking soda, candle, citric acid solution, plastic bottle lid.}
\item{Procedure: In a jam jar or a beaker, pour in 50 mL of water with a spoonful or two of baking soda. Put the candle sitting on the bottle lid, so that the candle is not sitting in the liquid. Light the candle, and add citric acid solution to the water. Bubbles will form and with enough time, the carbon dioxide from the reaction will put out the candle.}
\item{Theory: Baking soda is a chemical called bicarbonate of soda or sodium bicarbonate. This compound, NaHCO3, reacts with an acid to form carbon dioxide. The reaction is a 2 part reaction. First, the carbonate reacts with the acid:. H2CO3 is called carbonic acid. This acid quickly decomposes in water to form carbon dioxide and water:. This is the source of the carbon dioxide that will put out the candle. The carbon dioxide puts out the candle because in order for combustion to occur, oxygen must be present. Carbon dioxide is heavier than air and oxygen, and so as carbon dioxide accumulates, oxygen and other gases are displaced up, eventually out of the reach of the candle.}
\end{itemize}

\subsection{Carbon Dioxide Production, Part B – CO2 Balloon}
\begin{itemize}
\item{Preparation time: 10 minutes}
\item{Materials: 2 plastic water bottles, baking soda, citric acid, IV line, cement or lime water, water}
\item{Procedure: Follow the same procedure from Hydrogen Production, Part B – In a Balloon to put an IV line through a plastic water bottle lid. In a water bottle, put some cement in some water. This makes limewater. Decant to harvest the liquid to leave the solid behind. In a second water bottle, mix some water and baking soda. Add some citric acid and cap the bottle with the lid that has the IV line through it. Run the IV line into the limewater. The limewater will turn cloudy.}
\item{Theory: Another common test for carbon dioxide is called the ‘limewater test’. This is where carbon dioxide is bubbled through a limewater solution. Lime water is calcium oxide dissolved in water. As the carbon dioxide bubbles through the water, it reacts with the calcium to form a white, insoluble precipitate, calcium carbonate according the reaction:. This insoluble precipitate makes the solution appear cloudy. }
\end{itemize}

\section{Iron Chemistry}

\subsection{Activitying Ferrous Acetate}
\begin{itemize}
\item{Preparation time: 10 minutes}
\item{Materials: steel wool, vinegar, heat source}
\item{Procedure: Boil some steel wool in vinegar. The steel wool will dissolve in the vinegar forming ferrous acetate.}
\item{Theory: As the iron dissolves, it is giving up two electrons to for iron (II) acetate. Save this solution for later activities. Label as Ferrous Acetate.}
\end{itemize}

\subsection{Making Ferric Acetate}
\begin{itemize}
\item{Preparation time: 10 minutes}
\item{Materials: Ferrous Acetate, hydrogen peroxide}
\item{Procedure: Take a solution of ferrous acetate and add hydrogen peroxide. Save this solution for later activities. Label as Ferric Acetate.}
\item{Theory: Hydrogen peroxide is an oxidizing agent. In this case, the peroxide oxidizes iron (II) acetate to form iron (III) acetate. This is a brown solution.}
\end{itemize}

\subsection{Iron and Tea}
\begin{itemize}
\item{Preparation time: 5 minutes}
\item{Materials: ferrous acetate, ferric acetate, tea, jam jars}
\item{Procedure: Make some light tea. Do not make the tea too concentrated or it will be too hard to see the reaction. Separate the tea into two jars. Add ferrous acetate and ferric oxide to the tea.}
\item{Theory: The ferrous oxide will not react with the tea; however, the ferric oxide will react with the tannins in the tea. In tea, there are tannates, and these will displace the acetates. Iron (III) tannate produces a black precipitate.}
\end{itemize}

\subsection{Iron and Tartar}
\begin{itemize}
\item{Preparation time: 5 minutes}
\item{Materials: ferrous acetate, ferric oxide, cream of tartar or raisins, jam jars}
\item{Procedure: Add cream of tartar to two jam jars. Add ferrous acetate and ferric acetate to the jam jars.}
\item{Theory: Cream of tartar is the weak organic acid, tartaric acid. Ferrous acetate, again, does not react with the tartaric acid. However, the ferric acetate will react. The tartarate ion displaces the acetate ion. This forms a green iron (III) tartarate compound, ferric tartarate. If there is no cream of tartar available, raisins can be used. Add the raisins to barely enough water and let the acid seep out over night. Remove the raisins and continue the process. A common source of tartaric acid is grapes. It is possible that baobab fruit may contain tartaric acid. Experiment with these fruits; however, we are unsure if they will work.}
\end{itemize}

\subsection{Iron and Blood}
\begin{itemize}
\item{Preparation time: 20 minutes}
\item{Materials: Ferrous acetate, ferric acetate, dilute ammonia, hydrogen peroxide, jam jars}
\item{Procedure: In one jam jar, mix ferrous acetate and dilute ammonia together. Then add a small amount of hydrogen peroxide. Repeat this procedure for ferric acetate.}
\item{Theory: In this activity, the ammonia displaces acetate on the ferrous acetate. However, it does not react with the iron; the ammonia raises the pH to the point forming iron (II) hydroxide. Adding the peroxide forms iron (II) oxide and the solution should turn red. For the ferric acetate, the ammonia raises the pH to form insoluble iron (III) hydroxide. Adding the peroxide creates iron (III) hydroxide, a second red solution.}
\end{itemize}

\section{Miscible and Immiscible Liquids}

\subsection{Mixing Compounds, Part A – Solids and Liquids}
\begin{itemize}
\item{Preparation time: 15 minutes}
\item{Materials: 2 jam jars, sugar, water}
\item{Procedure: Add 50 mL of water to one jar and mark the level of the water. Repeat and make the same mark on the second jam jar. Then add another 50 mL to the second jar and make the second mark, thus showing 100 mL. Empty the water and wipe the jars dry. The actual volumes do not matter, just make sure one jar has a mark at 1 volume and the other jar has 2 marks at 1 volume and 2 volumes. In the first jar, add water until the mark. In the second jar, add sugar until the first mark. Then add the water in the first jar to the second jar with the sugar. }
\item{Theory: It seems logical that adding 50 mL of water and 50 mL of sugar yields 100 mL of total solution. However, this is not the case. Adding 50 mL water and 50 mL of sugar gives us a volume less than 100 mL. This is because there are intermolecular forces between sugar and water that draw the molecules closer together. These forces make the total volume of the solution contract just a little. This activity shows that 50 mL plus 50 mL does not always equal 100 mL. }
\end{itemize}

\subsection{Mixing Compounds, Part B - Liquids}
\begin{itemize}
\item{Preparation time: 15 minutes}
\item{Materials: 2 jam jars, methylated spirits, water}
\item{Procedure: Same as Mixing Compounds, Part A – Solids and Liquids  with one change. Instead of using sugar, use methylated spirits. }
\item{Theory: Intermolecular forces occur between many different molecules. Water and sugar is just one example. Methylated spirits have the same type of intermolecular forces involved in the contraction of the volume like the sugar. This is a useful activity to follow up if you have some clever students that recognize that 50 mL of sugar is not exactly 50 mL since there is sometimes space between sugar grains. In this activity, both components are liquids so there should be no question to the overall volume. However, the contraction will be smaller so thinner containers will show the change in volume much easier.}
\end{itemize}

\subsection{Lava Lamp}
\begin{itemize}
\item{Preparation time: 10 minutes}
\item{Materials: Clear plastic water bottle, water, food coloring, oil, effervescing antacid tablets, flashlight,}
\item{Procedure: Fill the bottom 10 cm of a water bottle with water. Add a few drops of food coloring. Fill rest of the bottle with oil. Drop in an effervescing antacid tablet. Cap and put a flashlight underneath the bottle. Observe the colors and the movement of the liquids.}
\item{Theory: Oil is a compound that is hydrophobic. This means that it repels water due to its nature. Oil is a long non-polar hydrocarbon, while water is a small polar compound. This means that the water cannot mix with the oil layer. This is why there are two layers on mixing oil and water. Adding the effervescing antacid tablets dissolve and release carbon dioxide in the water layer. The carbon dioxide dissolves in the water and forms small bubbles of carbon dioxide. These bubbles trap small amounts of food coloring. These bubbles rise since they have a much lower density than water. When the bubble reaches the surface, the carbon dioxide escapes and the colored water bubble falls down through the oil layer.}
\end{itemize}

\subsection{Magic Milk}
\begin{itemize}
\item{Preparation time: 5 minutes}
\item{Materials: Milk, different food coloring, powdered soap, cotton ball or q tip, shallow dish or a plate}
\item{Procedure: Pour in just enough milk to cover the plate or the bowl. Use food coloring and place a few drops around the plate of the milk. Soak the cotton ball or q tip in some detergent water. Touch the center of the milk plate with the cotton ball or q tip. The colors will start to move and swirl towards the center.}
\item{Theory: Milk is made up of fats and different proteins. These are non polar molecules. Food coloring is a water solution with some coloring compounds. The water solution and the non polar milk barely mix. When the q tip soaked in detergent touches the milk, it leaves behind soap. It is a compound that is both polar on one end and non polar on the other end. When this compound mixes with the milk, some interesting things occur. The milk and the soap intermingle forming micelles. In addition, the surface tension of the water in the milk also breaks. This allows the food coloring to move around in the milk. In other words, the coloring will start to move in the milk mixing with the other colors. It forms a milk color wheel. Do not drink this milk mixture.}
\end{itemize}

\section{Organic Chemistry}

\subsection{Showing the presence of C in sugar}
\begin{itemize}
\item{Preparation time: 10 minutes}
\item{Materials: metal wire with an end coiled so as to hold a metal soda cap or a spoon, soda cap with plastic removed leaving only the metal or a steel spoon, sugar, candle}
\item{Procedure: Bend some metal wire so as to make a cap holder so the handle is a distance from the flame source. Wrap the end of the metal wire into a circle so it can hold a metal soda bottle cap. Place the cap in the metal holder and fill it with a little sugar, or just use a spoon. Using the metal holder, place the metal cap above the flame of the candle. As the cap heats up it will heat the sugar. Heating the sugar will partially burn it before turning completely to carbon and carbon dioxide. This partial combustion is visible as the sugar burns it turns brown and then black.}
\item{Theory:  As sugar burns in air, it partially combusts leaving behind carbon solid and other carbon compounds before the changing into carbon dioxide. These carbon compounds have a brown color and the carbon solid compounds have a black color. As the sugar heats, it will combust until we have a brown color, then to a black color. This black color is solid carbon. This is the same color as charcoal, which is another example of solid carbon. @	The browning of the sugar is called caramallization. As the sugar breaks down into smaller and different saccarides, they bring a very delicious taste. This is the process that is used very much in making different candies, especially toffees, brittles, and caramels. In fact, this is how many candies get their brown or dark color.}
\end{itemize}

\subsection{Emulsification, Part A – Two Layers}
\begin{itemize}
\item{Preparation time: 5 minute}
\item{Materials: 1 jam jar, kerosene, water, powder soap}
\item{Procedure: In the jam jar, mix 20 mL of water and 20 mL of kerosene. Observe the effects when mixing these two liquids.}
\item{Theory: kerosene is a hydrophobic compound. This means that kerosene repels water. This happens because long carbon chains compose kerosene which makes it non polar. Water is a polar compound. Polar compounds dissolve in polar compounds. Non polar compounds dissolve in non polar compounds. Non polar compounds will not dissolve in polar compounds.}
\end{itemize}

\subsection{Emulsification, Part B – Now One Layer}
\begin{itemize}
\item{Preparation time: 5 minutes}
\item{Materials: 1 jam jar, kerosene, water, powder soap}
\item{Procedure: Measure 5 mL of water and 5 mL of kerosene into a jam jar. Two layers are visible from the separation between kerosene and water. Add a half or 1 gram of powdered soap. Shake to mix. The two layers will form one cloudy layer. After a long time, the layers will separate.}
\item{Theory: When a non-polar and a polar liquid come in contact, they will form two layers. The intermolecular forces push each other away. Some compounds, like soap, can be in both layers. In the case of soaps, these compounds have both a charged and a non-charged end. This allows part of the compound to exist in a polar layer, part to exist in the non-polar layer. In fact, they form micelles; small balls of hydrophobic pockets with the polar ends on the outside of the ball. This allows the polar and non-polar to mix. This arrangement is unstable. Over time, it will separate out into two layers. The addition of the soap allows the molecules to mix giving plenty of time before the layers separate. Salad dressing and mayonnaise are examples of emulsions.}
\end{itemize}

\subsection{Converting Soaps Into Lipids or Fats}
\begin{itemize}
\item{Preparation time: 5 minutes}
\item{Materials: powdered soap, battery acid, water, jam jar}
\item{Procedure: take ½ of a spoon of powdered soap and dilute it to about 50 mL in a jam jar. Add about 2 mL of battery acid. Bubbles of oil will form on the surface of the water. }
\item{Theory: Soaps are actually long chains of hydrocarbons with a deprotonated carboxylic acid group on the end. Many from living sources oils, like cooking oil, are long chains of hydrocarbons with a protonated carboxylic acid group on the end. This protonated group allows soaps to cause oil and water to mix: it is both hydrophobic on one end and hydrophilic on the other. One end can dissolve in water while the other ends dissolves in the oil layer. This allows soap to be soluble in both water and oil solutions. In fact, this is why soaps are used to clean off grease, oil, and other organic solvents. The organic solvents are hydrophobic and the soap can dissolve them. The acid protonates the carboxylic acid to revert back to a fat structure. The soap no longer has a polar end, and is completely hydrophobic. It collects on the top and looks like oil. It is oil, and not suitable for cooking. Oil is a triglyceride: a molecule with three long hydrocarbon chains. The soap does not have the same structure. In fact, what were made are palmitic acids and stearic acids. }
\end{itemize}

\subsection{Difference Between Ionic and Covalent Compounds}
\begin{itemize}
\item{Preparation time: 0 minutes}
\item{Materials: 1 lighter, 1 spoon, citric acid, salt}
\item{Procedure: Place a small amount of citric acid the spoon. Heat the spoon. It should bubble first, and then later turn brown then black as the compound thermally decomposes and then pyrolyzes. Repeat with salt. Heating has no effect on salt.}
\item{Theory: Carboxylic acids decompose to carbon dioxide on heating. Heating citric acid will cause the three carboxylic acid groups to decompose quickly to carbon dioxide. This is shown by the bubbles produced. Further heating will decompose the compound to solid carbon, which may itself burn off. When the experiment is repeated with salt, heating will cause no change in the salt. This is because salt is an ionic solid. This activity shows the difference between ionic and covalent compounds and the different interactions. Covalent compounds tend to have covalent bonds and weak intermolecular forces. In order to burn a compound, the bonds that compose the chemical must be break. These covalent compounds have their bonds decompose at relatively low temperatures; the bonds are broken by heating with a lighter. Ionic compounds have strong intermolecular forces that hold the molecules together. Heating salt with a lighter has no effect. In order to break the strong intermolecular forces in ionic compounds requires a great deal of heat, almost 1000 C. By showing that citric acids burns with a lighter and salt does not, students discover the difference between covalent and ionic compounds.}
\end{itemize}

\subsection{Cracking Household Oil}
\begin{itemize}
\item{Preparation time: 20 minutes}
\item{Materials: test tube, steel wool, two burners, oil}
\item{Procedure: Take a narrow a test tube and place it at an angle. At the bottom, place some normal household oil. At the top of the jar or test tube, stuff in some steel wool so it does not move. Place the tube through the stopper or lid and then put it firmly on the test tube so the only way for gas to exit is through the metal tube. Heat the steel wool until very hot. Then place a second burner under the oil so both the oil and the wool is being heated. After a minute or two, take a flame to the end of the tube. The gases escaping the tube can sustain a flame.}
\item{Theory: Cracking is a process where long hydrocarbon chains are heated so they are broken into smaller chains of molecules. Household oil is a triglyceride: a compound that has a glycerin backbone that has 3 long hydrocarbon arms. These long hydrocarbons range between 15 and 30 carbons long. These hydrocarbon arms can be broken easily by heating them in the presence of a catalyst, iron wool in this case. As they break down into smaller hydrocarbons, they become more volatile and combust easily. The long hydrocarbons will break down into hydrocarbon chains of 3 to 8 carbons. These will burn in air supporting a flame at the end of the tube.}
\end{itemize}

\subsection{Different Petroleum Products}
\begin{itemize}
\item{Preparation time: 10 minutes}
\item{Materials: syringe with metal needle, 3 jam jars, and as many as possible of: petrol, diesel, car lubricants, greases, petroleum jelly, kerosene, asphalt, tar, butane from a lighter}
\item{Procedure: Place some petrol, kerosene, diesel, and any other petroleum product in different jam jars. Compare the different properties of each of these compounds; look at density, viscosity, volatility, flammability, and more. Then use a syringe to remove the butane from a lighter. There is usually a small hole where a needle can be inserted to add more butane gas. However, it is possible to open this hole with a syringe needle and let the butane gas escape. It is a liquid in the lighter since it is under high pressure. As you use a syringe to release the butane, the pressure drops and it becomes a gas.}
\item{Theory: We use many petroleum products everyday. Crude oil is a black thick mixture of different hydrocarbons. To get different petroleum products, the crude oil is cracked and distilled to separate compounds depending on the number of carbon atoms. Butane is an early distillate since it has 4 carbons it distills easily. Petrol is an 8 carbon distillate. Kerosene has 12 to 15 carbons. Diesel has 15 to 25 carbons. Petroleum jelly is not a small hydrocarbon, but rather very long chains of varying length that does not distill easily. It is one of the last products from distilling crude oil. If you want to make some mock crude oil to show students, mix road tar with kerosene until you have a viscous liquid.}
\end{itemize}

\section{Plastics and Polymers}

\subsection{Differences in Plastics, Part A – Different Densities}
\begin{itemize}
\item{Preparation time: 5 minutes}
\item{Materials: different plastic objects, water, clear container}
\item{Procedure: Fill container halfway with water. Place different plastic pieces in the water. Depending on the composition of the plastic, some will float some will sink. }
\item{Theory: Plastics are made by polymerizing different monomers and therefore have difference densities. Some of these plastics have densities that are denser than water and will sink. There are a handful of common plastics and are identified by the number on the recycling label. Number one is polyetheneteraphthalate (PETE), two is high density polyethene (HDPE, crinkly plastic bags), three is polyvinylcholoride (PCV, plastic pipes), four is low density polyethene (LDPE), five is polypropene (PP), six is polystyrene (PS, brittle hard plastic cups, Styrofoam), and seven is anything else. }
\end{itemize}

\subsection{Differences in Plastics, Part B – Plastic and the Tower}
\begin{itemize}
\item{Preparation time: 10 minutes}
\item{Materials: density tower from Density Tower, Part A – Different Densities, different plastic objects}
\item{Procedure: Using the density tower, place different pieces of plastic and watch to see what layer the plastic sits in between.}
\item{Theory: The density tower can tell us a little more information about the different densities of the plastics. By looking at which layer the plastic settles at, we can deduce information about the density of each plastic.}
\end{itemize}

\subsection{Strength of Polymers, Part A – Floss Pull}
\begin{itemize}
\item{Preparation time: 5 minutes}
\item{Materials: dental floss}
\item{Procedure: Give each student a 1 meter long length of dental floss. Without using their teeth or a knife, let them try to break the floss.}
\item{Theory: Dental floss is made from a few possible polymers: nylon, Teflon, or polyethylene. Polymers tend to be very strong compounds. Unless a student has help in terms of a sharp edge, they will not be able to break the dental floss unless the floss is of the very cheap variety.}
\end{itemize}

\subsection{Strength of Polymers, Part B – Tearing Bottles}
\begin{itemize}
\item{Preparation time: 5 minutes}
\item{Materials: plastic water bottles}
\item{Procedure: Give each student a plastic water bottle. Without using a sharp edge, have the students try to tear apart the plastic bottles.}
\item{Theory: This is another example of the strength of plastics. This is one of the reasons that we use plastic for many different containers. It is light and very strong. If the same container were made out of glass or metal, it would be too heavy to use. Also, glass breaks too easy and metal reacts with many compounds. Plastic, on the other hand, does not react. These three properties make plastic perfect containers for different compounds. }
\end{itemize}

\subsection{Flammability of Plastic Compounds}
\begin{itemize}
\item{Preparation time: 10 minutes}
\item{Materials: samples of different plastic compounds}
\item{Procedure: Take small samples of plastic compounds and put them in a flame. They should rapidly ignite. Do not breathe in the fumes. Do not use PCV (plastic pipes) or any rubber compounds.}
\item{Theory: While plastic is very strong, light, and nonreactive, it does have one downside. Since it is a polymer made from different carbon compounds, plastics will combust when exposed to a flame. Plastics will contract and change shape under heat until they catch fire. Further, many polymers form toxic compounds when they burn. Therefore it is important to not to breathe the plastic fumes.}
\end{itemize}

\subsection{Non-Reactivity of Plastic Compounds}
\begin{itemize}
\item{Preparation time: 15 minutes}
\item{Materials: 5 different plastic containers, battery acid, caustic soda, iodine, water}
\item{Procedure: Put some battery acid in a plastic container. Do the same with iodine, caustic soda, and normal water. Further, bury one plastic bottle in the ground. After 1 month, check each bottle to see if they have reacted with anything.}
\item{Theory: Plastics are very nonreactive. After 1 month, no plastic bottle will have reacted. They will not react with an acid, a base, or an oxidant. Bacteria and other compounds that decompose different organic material cannot react with plastic either. Since they cannot react with the plastic, the bacteria do not break down the plastic bottle.\\
This is one of the reasons it is bad for the environment to through plastic bottles out of the bus window. Since the bottles do not break down in the environment, they will just remain there. This is why there are big piles of plastic water bottles on the side of the road. They will not break down or decompose so proper disposal of plastics is important.}
\end{itemize}

\section{Reactions and Kinetics}

\subsection{Kinetics of Baking Soda and Battery Acid}
\begin{itemize}
\item{Preparation time: 30 minutes}
\item{Materials: citric acid solution, sodium bicarbonate or baking soda, one plastic bottle, one syringe, water, stop watch}
\item{Procedure: Prepare the reaction container prior to the experiment. Take a lid and place a syringe needle through the lid. Seal the needle-cap connection with glue to prevent any leakage. Do not glue the cap to the bottle. Inside the bottle, place 50 mL of water and 1 g of baking soda. Mix well. Use the syringe to measure out 1 mL of citric acid solution. Connect the syringe to the needle. Start the stopwatch when adding the acid. Let go of the syringe; the plunger will rise as the reaction proceeds. Record the time required until the plunger stops moving or until it reaches the top. Repeat the experiment with different amounts of baking soda and battery acid. Repeat the experiment by changing one variable at a time: diluting the battery acid, increasing mass of baking soda, or increasing the amount of initial water. }
\item{Theory: Given the reaction, the reaction rate can be monitored by the production of a product. In this case, the formation of carbon dioxide is a way to monitor this reaction. Carbon dioxide gas forms and fills the container, as more and more gas is produced, the syringe will be pushed outwards according to Avogadro’s Law. By recording the time required and volume displaced in the syringe, we are studying the kinetics of the reaction. This analysis is much more qualitative than quantitative. Compare the rate of the bubble formation and the syringe rise speed in all the variations. }
\end{itemize}

\subsection{Exothermic and Endothermic Reactions}
\begin{itemize}
\item{Preparation time: 10 minutes}
\item{Materials: Supplies for Making Hydrogen, Part A, Kinetics of Baking Soda and Battery Acid}
\item{Procedure: Make both of these activities side by side. Instead of testing for hydrogen or observing the kinetics of the reaction, feel both bottles as the reaction proceeds. One bottle will feel warm while the other will get cold. }
\item{Theory: This activity provides concrete examples of both thermodynamic types of reactions: exothermic and endothermic. The formation for hydrogen from iron and sulfuric acid is a very exothermic reaction. This means that as the reaction proceeds, it releases energy into its surroundings, and the temperature of the reacting mixture increases. Students often get confused. Exothermic is the conversion of chemical potential energy in the chemicals to thermal energy that transfers this energy to the surrounding environment. In these activities, the reaction takes place in a water environment. This means as the energy transfers from the reaction to the water, the water will heat up. In contrast to the reaction of battery acid and iron, the formation of carbon dioxide from baking soda is endothermic. This reaction absorbs the thermal energy from surrounding environment, the water, and converts it to chemical potential energy necessary to drive this particular reaction. As thermal energy is being consumed and converted to chemical potential energy, the temperature of the reacting mixture decreases.}
\end{itemize}

\subsection{Inorganic and Organic Catalysts}
\begin{itemize}
\item{Preparation time: 20 minutes}
\item{Materials: D cell battery, yeast, water, heat source, 4 20 mL syringe shells, water, balloons, hydrogen peroxide}
\item{Procedure: In all syringe shells, put in 5 mL of water. Add yeast to two syringe shells and Add manganese dioxide from the inside of the D cell batteries. See Producing Oxygen, Part A for more information. Then, place 1 syringe with manganese dioxide and 1 syringe with yeast into a boiling water bath until they boil. In this activity, we want to have 4 variants: 1 syringe with yeast, 1 syringe with MnO2, 1 syringe with boiled yeast and 1 syringe with boiled MnO2. Add 5 mL of hydrogen peroxide to one and cap it. Time how long it takes for the balloon to fill with oxygen until it gets to its maximum size. Repeat for this procedure with the remaining three balloons. }
\item{Theory: Catalysts speed up the rate of reactions. Catalysts can be split into two categories: biological catalysts and inorganic catalysts. Biological catalysts are also known as enzymes. In the decomposition of hydrogen peroxide to oxygen, both manganese (IV) oxide from batteries and catalase from yeast catalyze this reaction. In this activity, it is possible to identify the difference between the two different types of catalysts. The two syringes that are not boiled act just like normal; they both quickly form oxygen in the balloons. However, the two syringes that are boiled do not act the same. Boiling the yeast destroys its catalase. This slows down the production of oxygen since the catalase does not catalyze the reaction. The balloon will not fill or slowly fill at best, and most of that filling will be from thermal decomposition of the hydrogen peroxide caused by the hot syringe. In contrast, manganese dioxide is not degraded. If anything, the higher temperature will speed up the reaction. This balloon will fill normally if not faster than normal. This activity shows one of the differences between biological and inorganic catalysts. Biological catalysts are susceptible to variations in the environment that kill or destroy biological activity.}
\end{itemize}

\section{Soda Science}

\subsection{Pressure Dependent Solubility of Sodas}
\begin{itemize}
\item{Preparation time: 0 minutes}
\item{Materials: unopened soda bottle}
\item{Procedure: Quickly and carefully, open soda bottle. You should see some vapor escape.}
\item{Theory: When sodas are bottled, the carbon dioxide is added by pumping CO2 at a pressure of 5 atmospheres. Henry’s law tells us that as we increase the pressure, the amount of gas dissolved in the water also increases. As we release the pressure by opening the bottle, the CO2 and trapped water vapor dissolved in the liquid come out and become a gas again.}
\end{itemize}

\subsection{Dissolved Carbon Dioxide in Soda, Part A – The Opening}
\begin{itemize}
\item{Preparation time: 0 minutes}
\item{Materials: unopened soda bottle}
\item{Procedure: Open the soda bottle. Watch how long it takes for the soda to stop releases carbon dioxide (when bubbles stop forming).}
\item{Theory: To make a soda, the soda solution has carbon dioxide gas pumped in at 4 or 5 atmospheres of pressure. Henry’s Law states that when we have a solution that is under pressure from a gas, the solution will dissolve some of the gas. In this case, the soda dissolves some of the carbon dioxide from the pressure. Removing the top of a soda allows this pressure to be relieved. This means that the carbon dioxide in the solution will slowly migrate to the surface of the soda and escape to the atmosphere. However, this process is not always quick. There is some small intermolecular attraction between carbon dioxide and the water. The carbon dioxide slowly breaks this attraction between water and carbon dioxide and escapes in the air. This is why sodas will have dissolved carbonation for some time after they are opened. However, this dissolved carbonation is not in an equilibrium state: equilibrium is achieved when most if not all of the dissolved carbon dioxide escapes. The equilibrium state for sodas occurs when the soda becomes flat.}
\end{itemize}

\subsection{Dissolved Carbon Dioxide in Soda, Part B – Shaking It}
\begin{itemize}
\item{Preparation time: 0 minutes}
\item{Materials: unopened soda bottle}
\item{Procedure: Shake the soda bottle vigorously prior to opening. Be careful opening this bottle since it will make a big mess.}
\item{Theory: The process for carbon dioxide to escape from the soda is happening immediately after production. The air pocket in between the lid and soda is filled with carbon dioxide from the production. By shaking the soda, some of the carbon dioxide that previously escaped redissolves in the liquid temporarily. The carbon dioxide is not stable in the states and forms larger bubbles. On opening the lid, these bubbles rush to the surface to escape the solution. The soda will come exploding out the mouth of the soda from these redissolved bubbles. }
\end{itemize}

\subsection{Dissolved Carbon Dioxide in Soda, Part C – Salt and Soda}
\begin{itemize}
\item{Preparation time: 0 minutes}
\item{Materials: unopened soda bottle, salt}
\item{Procedure: Open the soda bottle. Let the carbon dioxide escape. Add salt to remove the rest of the carbon dioxide.}
\item{Theory: The carbon dioxide trapped in the soda is held weakly by the intermolecular attraction between water and carbon dioxide. This weak attraction slowly lets the carbon dioxide escape. Adding salt changes the dynamics of this attraction. Salt is a strong electrolyte and water has greater attraction to the salt molecules than the carbon dioxide molecules. On adding salt, the water will find itself moving towards the salt and pushing the carbon dioxide together. As the carbon dioxide combines and forms larger bubbles, these bubbles rise to the surface of the soda. It is possible to push all the carbon dioxide out of a soda through the addition of salt.}
\end{itemize}

\section{Solubility}

\subsection{Unsaturated and Saturated Solutions, Part A – More, More, More}
\begin{itemize}
\item{Preparation time: 0 minutes}
\item{Materials: sugar or salt, jam jar, water}
\item{Procedure: Fill the jam jar halfway with water. Add a few grains of salt and mix until it is all dissolved. Let the students taste the solution. It should taste faintly like salt.}
\item{Theory: There are three different types of solutions when concerned with solubility: unsaturated solutions, saturated solutions, and supersaturated solutions. An unsaturated solution is a solution that has a little solute dissolved but it is able to dissolve more. In this case, the solution is unsaturated because there are only a few grains of salt in the water; the water can easily dissolve more.}
\end{itemize}

\subsection{Unsaturated and Saturated Solutions, Part B – Just Enough}
\begin{itemize}
\item{Preparation time: 0 minutes}
\item{Materials: same as Unsaturated and Saturated Solutions, Part A – More, More, More}
\item{Procedure: Using a second jam jar filled with the same amount of water, slowly add salt until no more salt dissolves in the water. Let the students taste the water. It will strongly taste like salt.}
\item{Theory: In this part, we are making a saturated solution. This is a solution that has dissolved all the salt that is possible. No more salt will dissolve in the solution. The taste of this solution is a much more salty because more salt has dissolved in the solution.}
\end{itemize}

\subsection{Unsaturated and Saturated Solutions, Part C – Too Much}
\begin{itemize}
\item{Preparation time: 0 minutes}
\item{Materials: same as Unsaturated and Saturated Solutions, Part A – More, More, More}
\item{Procedure: Using a third jam jar filled with the same amount of water, add salt until no more salt dissolves. Then add more salt so that a small mound sits at the bottom. Let the students taste the salt solution.}
\item{Theory:  Once a solution is saturated, it will dissolve no more solute. In this case, we have made a saturated solution of salt. By adding more and more salt, we cannot make more salt dissolve. The salt just sits on the bottom of the jar. By tasting the salt solution, it will taste exactly like the saturated salt solution from Unsaturated and Saturated Solutions, Part B. This is because we taste only the salt dissolved in the solution. The extra salt does not make the water taste more salty; it stays undissolved. }
\end{itemize}

\subsection{Solubility of Salt and Sugar, Part A – Waiting Game}
\begin{itemize}
\item{Preparation time: 5 minutes}
\item{Materials: jam jars, salt, sugar, water}
\item{Procedure: Fill each jar half full with water. Add ½ spoon of salt in the first container and ½ spoon of sugar in the second container. Carefully watch how long it takes, without stirring, to dissolve the salt and sugar.}
\item{Theory: Salt and sugar both dissolve in water, but at different rates. This is due to the strength of the differing intermolecular forces. Salt is a strong electrolyte and the polar water molecules easily respond to the addition of ions from the salt. The intermolecular forces between the polar water and ions from salt allow salt to dissolve easily in the water. Sugar on the other hand is a non-electrolyte. It does not form ions in solution. However, there are some intermolecular forces between the atoms in sugar and water. It will dissolve, but it will take a much longer time to dissolve.}
\end{itemize}

\subsection{Solubility of Salt and Sugar, Part B – There is No Spoon}
\begin{itemize}
\item{Preparation time: 5 minutes}
\item{Materials: same as Solubility of Salt and Sugar, Part A, spoons}
\item{Procedure: Follow the same procedure as Solubility of Salt and Sugar, Part A – Waiting Game. Instead of watching the salt and sugar dissolve without stirring, now stir each solution at the same rate. Watch how long it takes for the salt and sugar to dissolve.}
\item{Theory: Water can only dissolve a certain amount of salt and sugar in one specific area. If the water is not stirred, the water surrounding the salt and sugar will dissolve all that it can. Once it has dissolved all the salt or sugar that is possible, no more can be dissolved. When no more salt or sugar can dissolve, the water is full of the chemicals or it can be thought as locally saturated. As time goes on, the dissolved salt and sugar will migrate to the rest of the container allowing the water around the salt or sugar to dissolve more. This process will continue, albeit slowly, until all the salt and sugar has dissolved. The purpose of the stirring is to speed up the migration of dissolved salt and sugar as well as to move the locally saturated water away from the salt and sugar allowing unsaturated water to dissolve more salt or sugar.}
\end{itemize}

\subsection{Snow Globes}
\begin{itemize}
\item{Preparation time: 1 day}
\item{Materials: borax from Making Borax, clear glass jam jar with lid, water}
\item{Procedure: Into a clean dry jam jar, place a small amount of borax. The actual amount does not matter. Fill the jam jar with water until the brim. Seal tightly. Shake and watch the borax float like snow.}
\item{Theory: Borax is a sparingly soluble salt. In water, it has similar texture to actual snow. Shaking the jar causes the liquid to start moving around. Stopping the jar does not stop the inertia of the water inside the jam jar. This inertia will make sure that borax will keep on moving even when the jar stops. To make the snow globe a little more interesting, glue a toy or a figure inside the jar. Dying the water a color will make it even more interesting. }
\end{itemize}

\section{Visualizing Chemical Reactions}

\subsection{Dancing Reactions}
\begin{itemize}
\item{Preparation time: 10 minutes}
\item{Materials: bottle caps, matches}
\item{Procedure: For a given reaction, use different soda caps as atoms for both the reactants and products. Have students practice balancing the reaction based on having a proper number of bottle caps for each atom. Use the matches for bonds. It may be helpful to use a marker to put letters on each bottle cap for ease in remembering which bottle cap is which atom and/or to code them by color.}
\item{Theory: Balancing reactions is sometimes hard for students to understand. Using bottle caps and matches, it makes the process for balancing reactions much easier. By making it a visual experience, the balancing act becomes much more accessible for students.}
\end{itemize}

\subsection{Modeling Kit}
\begin{itemize}
\item{Preparation time: 1 hour}
\item{Materials: different colored sandals, toothpicks, marker}
\item{Procedure: Cut out small circles of the different sandals. Use a marker to label each color to a specific atom. Use a toothpick to connect different atoms to form molecules. Remember to make these models geometrically correct; carbon, for example, usually makes four bonds that radiate with a tetrahedral structure.}
\item{Theory: VSEPR Theory is can be very hard to students to visualize, especially the three dimensional aspects. Therefore, using cut out sandals allows students to make the different structures of the chemicals. Further, this is helpful to learn about crystal structure. For example, making the crystal structure of graphite is easy to make with these.}
\end{itemize}

\section{Water Chemistry}

\subsection{Distilling Water}
\label{sub:distillwater}
\begin{itemize}
\item{Preparation time: 30 minutes}
\item{Materials: 1 tea kettle, tea leaves or food coloring, plastic tubing, 1 plastic water bottle, 1 container for collecting distilled water}
\item{Procedure: In the teakettle, fill it with water and place some tealeaves. Make the tea so it has a dark color. Connect the pouring spout of a teakettle with some I.V line. If the tubing is too small for the spout, use a syringe shell to connect the two. Do not cut the I.V. line short; just let it be the entire length. Place a container at the end of the tube to collect the condensed water. Boil the teakettle after some time, start collecting water. }
\item{Theory: Distillation is a purification method of miscible liquids or solutions. In this situation, there is tea leaves dissolved in the water. The coloration of the water indicates this. However, the water can be purified since the leaves will not undergo a phase transition to the gaseous state. The water vapor is pure water; the tea is left behind. As the vapor goes down the line, it cools down to below the vaporization point and becomes a liquid again. By collecting this water, it is easy to see that it is pure water and has left behind all the tea in the kettle. @	This is the basic principle behind the distillation of crude oil to make common organic solvents like petrol, airplane fuel, kerosene, lubricants, motor oils, even petroleum jelly. Crude oil is a mixture of a variety of different components that all have different vapor points. Careful heating allows the separation and collection of all the components of oil. These separate out and have different physical and chemical properties based on their composition or molecular weight. The lightest components separate first, while the heaviest components are last.\\
Distilling on a larger scale is a good way to purify water for laboratory use when normal tap water is contaminated with many salts. This is also the principles behind hard alcohols. They first begin as beer or wine or other fermented liquid and then they are distilled to collect and concentrate the alcohol. You may use this apparatus to produce clear spirits from methylated spirits. These are very useful for making POP solution for chemistry practicals. Note that this method if anything concentrates the poisonous methanol in the spirits, so even the clear spirit is toxic to drink.}
\end{itemize}

\subsection{Laboratory Distillation of Water}
\begin{itemize}
\item{Preparation time: 2 hours}
\item{Materials: 1 Large tea kettle, plastic tubing, a 20 liter bucket, water, super glue, container to collect water.}
\item{Procedure: Find a large kettle. There are many large kettles available. Many times people will carry a large kettle and sell coffee in towns. Find one of those large, 5 liter kettles. Connect thick plastic tubing to the kettle and run this tubing into the side of a large 20 liter bucket. This bucket acts as the condenser. Coil the tubing in the bottom of the bucket and let the tubing exit the bottom of the bucket. Put a container underneath the tubing to collect distilled water. Use super glue to seal the tubing connections with the bucket on the inside and outside of the container. When the glue dries, fill the kettle, fill the bucket condenser with normal water, and apply a heat source and start collecting distilled water.}
\item{Theory: In Distilling Water, the IV line is so narrow that the air surrounding the tube can cool the gaseous water inside of it. There is no reason to have a condenser. However, scaling up this project will involve a condenser. The purpose of a condenser is to transfer the heat from the gaseous vapors from inside the tube into something outside the tube. In this case, we use water surrounding the tubing to cool the water inside the tubing. This way, the only thing that comes out from the plastic tubing is condensed water and not water vapor. It is important to have the tubing coil around in the bucket and exit at the bottom so that the distilled water, once condensed, can use gravity to pull itself out of the bucket into the collection. In extended distilling processes, the water in the bucket condenser can get hot. It may be necessary to remove the hot water and add new cold water.\\
The water collected from this activity will be distilled water. This water is perfect for use in creating bench reagents, specifically the ones in the qualitative analysis practicals. Remember, this process is unnecessary for volumetric analysis.}
\end{itemize}

\subsection{Hard Water, Part A - Foaming}
\begin{itemize}
\item{Preparation time: 10 minutes}
\item{Materials: powdered soap, 2 small jam jars, tap water, pure water from Distilling Water}
\item{Procedure: Fill one jar with some tap water and another with some distilled water. Add a big pinch of powder soap. Try to make bubbles in the water by capping and shaking. The amount of foam depends on the water: hard water forms almost no foam, soft water produces a little foam, and distilled water will foam greatly.}
\item{Theory: Water has different compounds dissolved in it. Water from the ground has lots of minerals and ions that dissolved from the earth. These ions and minerals are not necessary bad for a person’s health; rather distilled water has no taste since there is a lack of those minerals and ions. There are three general classifications of water: hard, soft, and distilled. Hard water is water that has lots ions dissolved in it, especially calcium and magnesium. Soft water has a much smaller concentration of ions, but it still has ions nonetheless. Distilled water has no dissolved ions or minerals. The purpose of adding soap is to see the formation of bubbles. Bubbles are an arrangement of the long hydrocarbon chains from the soap. The chains need to organize themselves in a very particular manner and the ions prevent these chains from forming properly. Hard water has many of these ions dissolved in water and they prevent the formation of the bubbles. Soft water has fewer ions, but enough to inhibit some bubble formation. Distilled water has no ions dissolved, and the bubbles form unimpeded.\\
This affects people in regions with hard water. Since people add soap until there is a good foamy mixture to wash dishes, people with hard water have to add more soap. If there is soft water, less soap is required to make a foamy mixture. If the water available is soft water, make hard water by dissolving Epsom salt in the water – this adds magnesium ions, which artificially makes the water hard.}
\end{itemize}

\subsection{Hard Water, Part B - Removal}
\begin{itemize}
\item{Preparation time: 10 minutes}
\item{Materials: 1 object with hard water remains to be removed, citric, water, bucket}
\item{Procedure: Place the object with lime scale or hard water remains in a bucket. Add a one to one ratio of vinegar and water. Add enough to cover the object. Let the object sit overnight. The hard water remains will dissolve. Calcium and magnesium carbonates dissolve easily in vinegar. If there remains some precipitate, soak the object in a more concentrated citric acid solution over night. It may require soaking in diluted battery acid to remove all the remains.}
\item{Theory: Carbonates are sparingly soluble salts. This means that they tend to form solids or precipitates. However, a little does dissolve. When they precipitate out, it is difficult to dissolve them back into water. Their solubility does depend on pH however. Adding an acid, vinegar, citric, or battery acid, decreases the pH that increases the solubility of lime scale or hard water remains. In other words, the acid helps to dissolve the hard water deposit.}
\end{itemize}

\subsection{Polar Water, Part A – Comb Attraction}
\begin{itemize}
\item{Preparation time: 5 minutes}
\item{Materials: water, plastic water bottle with cap, comb, hair}
\item{Procedure: Put a small hole in the cap of a water bottle. Applying firm, constant pressure with a syringe needle on the lid is the easiest method; however take care to be safe. After the hole is in the cap, run the comb through someone’s hair. Turn the plastic water bottle over and squeeze gently to force a thin but continuous water stream at a 45 degree angle. Bring the comb towards the water stream and the water will bend.}
\item{Theory: Water is a polar molecule. It is made up of 2 hydrogens and 1 oxygen. The hydrogens form a positive end and the oxygen forms the negative end. When a comb that has been statically charged comes near water, there are some electrostatic forces. The static forces on the comb pull on the water. The effect of this force is visible by observing the bending of the water stream when the comb comes near, and the water returning to its normal stream when the comb moves away. The bending occurs only because water is a polar molecule and is attracted to the static charge on the comb.}
\end{itemize}

\subsection{Polar Water, Part B – The Dribble}
\begin{itemize}
\item{Preparation time: 10 minutes}
\item{Materials: water, glass container, plastic container}
\item{Procedure: Fill both containers with water. Carefully, and barely pour from one container to the next container. Look to see if the water dribbles down the side of the container.}
\item{Theory: Glass and plastic are made of two different materials and have themselves two different chemical properties. Glass is a polar compound while plastic is a non-polar compound. Between compounds, intermolecular forces play an important role in how compounds hold together. Intermolecular forces between polar and polar compounds are strong, and they will hold onto each other. On the other hand, intermolecular forces between polar and non-polar compounds are very weak, and they will not hold onto each other. Since water is a polar molecule, it will adhere to the glass. This is the reason that water tends to dribble down the sides of glass containers making a mess when pouring. However, since water is polar and plastic non-polar, there is little to no attraction between water and plastic. This means that there will be no water sticking to the plastic dribbling down the side of the plastic. This is the same reason kerosene dribbles down the side of plastic containers but pours freely from glass.}
\end{itemize}

\subsection{Expansion of Ice, Part A – Breaking the Bottle}
\begin{itemize}
\item{Preparation time: 5 minutes}
\item{Materials: 1 small water bottle, water, freezer}
\item{Procedure: Fill a water bottle full of water. Cap the bottle. Place this bottle in the freezer. Check on the bottle the next day.}
\item{Theory: Water is an interesting chemical since its density actually increases as it changes from a liquid phase to the solid phase. Normally, that is not the case with other compounds. In this activity, the water freezes into ice and increases in volume. Since the bottle is capped, there is no extra volume for the water to expand and will actually break the plastic water bottle. }
\end{itemize}

\subsection{Expansion of Ice, Part B - Floatation}
\begin{itemize}
\item{Preparation time: 0 minutes}
\item{Materials: ice, water, jam jar}
\item{Procedure: Put some water in the jam jar and add a piece of ice. Observe if the ice floats or sinks.}
\item{Theory: The density of water changes when it freezes to make ice. Density is defined as mass divided by volume. In ice, the volume increases as the water freezes but the mass does not change. This makes the density of ice smaller than normal water. Since the density is less when water freezes, adding ice to water means that the ice will float on the surface of the water. This indicates that the density of ice is less than that of normal water.}
\end{itemize}

\subsection{Pressure Melting Ice}
\begin{itemize}
\item{Preparation time: 10 minutes}
\item{Materials: Ice, thin plastic string like fishing line or dental floss, two weights like rocks or cans}
\item{Procedure: Take a meter long piece of thin plastic line and tie the two cans or rocks on the end. Drape the string over the ice and let the weights pull down on the ice. The line will move through the ice without the ice obviously melting.}
\item{Theory: Another interesting property of water is that ice melts on the application of pressure. In this activity, the plastic line applies pressure to the ice from the weights. This in turn, applies specific pressure on the ice, melting it only directly under the line. The string moves through the newly melted water, and then the water refreezes above the string. This is a fun activity because it appears that the line is moving through the ice magically, but really, the ice is melting under pressure, and then refreezing when the pressure is alleviated by the string moving on down through the ice.}
\end{itemize}
