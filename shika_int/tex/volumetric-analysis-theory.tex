\chapter{Volumetric Analysis Theory}

Most examples of volumetric analysis involve acid-base reactions, so first is a bit of acid-base theory.

\section{Acids, Bases, and pH}

The Bronsted-Lowery definition of an acid is a substance that provides $\mathrm{H}^{+}$ to a solution while a base is a substance that removes $\mathrm{H}^{+}$ from a solution.

It is important to remember that in a water solution, $\mathrm{H}^{+}$ does not exist. Rather, $\mathrm{H}^{+}$ binds with water to form the hydronium ion, $ \mathrm{H}_3 \mathrm{O}^{+} $ .

\[ \mathrm{H}^{+} + \mathrm{H}_2 \mathrm{O} \longrightarrow \mathrm{H}_3 \mathrm{O}^{+} \]

pH is defined as the power of the hydronium ion concentration. To find the pH of a solution:

\[ \mathrm{pH} = \log{[\mathrm{H}^{+}]_{aq}} \]

Pure water has $ 10^{7} $ moles of $\mathrm{H}_3 \mathrm{O}^{+}$ per liter, or $ \mathrm{pH} = 7 $. This is because some water molecules are always reversibly reacting with each other to form hydronium and hydroxide:

\[ 2\mathrm{H}_2\mathrm{O} \longleftrightarrow \mathrm{H}_3 \mathrm{O}^{+} + \mathrm{OH}^{-} \]

Acids increase the amount of $\mathrm{H}_3 \mathrm{O}^{+}$. By increasing the concentration of hydronium ion, the power of the concentration increases to a less negative number, and thus the solution will have a smaller pH. Bases decrease the amount of H3O+ and thus basic (alkaline) solutions have pH greater than 7.

\section{Types of Acids and Bases}

\subsection{Strong Acids}

Strong acids are acids that dissociate completely to provide $\mathrm{H}^{+}$. One can approximate the molarity of $\mathrm{H}^{+}$ (or $\mathrm{H}_3 \mathrm{O}^{+}$) as the molarity of the acid. For example, a solution of 1~M HCl has one mole of $\mathrm{H}_3 \mathrm{O}^{+}$ per liter of solution (pH 0); most of the molecules of HCl have dissociated and the $\mathrm{H}^{+}$ has reacted with water to form $\mathrm{H}_3\mathrm{O}^{+}$.

\[ \mathrm{HCl} + \mathrm{H}_2\mathrm{O} \longrightarrow \mathrm{H}_3 \mathrm{O}^{+} + \mathrm{Cl}^{-} \]

The most common strong acids are sulfuric acid ($\mathrm{H}_2\mathrm{SO}_4$), hydrochloric acid (HCl), and nitic acid ($\mathrm{HNO}_3$).

\subsection{Weak Acids}

Weak acids, however, are reticent to contribute $\mathrm{H}^{+}$ to solution. For example, in a solution of ethanoic acid, an equilibrium forms where only one in 250 ethanoic acid molecules dissociates to form $\mathrm{H}_3 \mathrm{O}^{+}$.

\[ \mathrm{CH}_3\mathrm{COOH} + \mathrm{H}_2\mathrm{O} \longleftrightarrow \mathrm{H}_3 \mathrm{O}^{+} + \mathrm{CH}_3\mathrm{COO}^{-} \]

The most common weak acids are ethanoic acid or acetic acid ($\mathrm{CH}_3\mathrm{COOH}$), ethandioic acid or oxalic acid ($\mathrm{C}_2\mathrm{H}_2\mathrm{O}_4$), and citric acid ($\mathrm{COOHCH}_2\mathrm{COH(COOH)CH}_2\mathrm{COOH}$).

One mole of hydrochloric acid and one mole of ethanoic acid both require the same amount of base for neutralization. The difference is how the pH of the solution changes during the titration. When hydrochloric acid is titrated, the pH remains very low until right before the endpoint when it jumps to alkaline. When ethanoic acid is titrated, the pH gradually rises through a range of acidic pH's and then jumps at the endpoint. This is why methyl orange cannot be used for titrations with weak acids – see Properties and Preparation of Indicators.

\subsection{Strong Bases}

Strong bases are bases that either dissociate completely in solution to form $\mathrm{OH}^{-}$ which reacts to remove $\mathrm{H}_3\mathrm{O}^{+}$. The common strong are sodium hydroxide, NaOH, and potassium hydroxide, KOH.

\subsection{Weak Bases}

Weak bases form an equilibrium with water where only a few of the molecules react to remove $\mathrm{H}_3\mathrm{O}^{+}$. Common weak bases include ammonia (ammonium hydroxide), soluble carbonates, $\mathrm{CO}_3^{2-}$ and all hydrogen carbonates, $\mathrm{HCO}_3^{-}$.

Much like strong and weak acids, both strong and weak bases readily react with acids to neutralize them. As with acids, weak bases will form a buffered solution that changes pH gradually whereas strong bases will change pH abruptly when the base is neutralized fully.

\section{Volumetric Analysis}

Volumetric Analysis is a method to find the concentration (molarity) of a solution of a known chemical by comparing it with the known concentration of a solution of another chemical known to react with the first.

For example, to find the concentration of a solution of citric acid, one might use a 0.1~M solution of sodium hydroxide because sodium hydroxide is known to react with citric acid.

The most common kinds of volumetric analysis are for acid-base reactions and oxidation-reduction reactions. Acid-base reactions require use of an indicator, a chemical that changes color at a known pH. Some oxidation-reduction reactions require an indicator, often starch solution, although many are self-indicating, that is one of the chemicals itself has a color. For more about indicators, read Properties and Preparation of Indicators. For more on the specific technique of volumetric analysis, read Traditional Volumetric Analysis Technique if you have burettes and Volumetric Analysis Without Burettes if you do not.

The process of volumetric analysis is often called \textit{titration.}
