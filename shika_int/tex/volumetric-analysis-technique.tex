\chapter{Traditional Volumetric Analysis Technique}
\label{cha:volanatech}
\section{Burettes}

In most acid-base titrations, the acid comes from the burette, although sometimes the burette holds the base. Prior to use, the student should thoroughly wash the burette to remove any residue from previous use. Then, the student should close the stopcock and add about 5ml of the solution that they will use in the burette. With their thumb over the open end of the burette they should make sure the solution covers every surface of the burette. They should then run this solution out into a waste container. This step is to replace the residue of water from the first washing with a layer of the titration solution. If students do not perform this step, the water reside will dilute their titration solution.

Most burettes have a volume of 50 mL. The 0 mL mark is at the top, and the 50 mL mark is at the bottom. This is because the burette tells you the volume of solution used, not the volume of solution present. If you start at 0 mL, and finish at 20 mL, then you have used 20 mL of acid.

Many burettes do not have stopcocks. Instead, they have a piece of rubber tubing at the bottom, which has a glass tip inserted into it. Either a metal clip is used to hold the rubber tubing closed or there is a small bead in the tubing around which fluid may pass when the tube is squeezed at that point. Broken burettes can often be repaired; see the section on Repairing Burettes.

\section{Reading measurements}

\begin{itemize}

\item{Always read burettes at eye-level. If the burette is clamped to a stand, remove it from the stand so you can hold it at eye-level. Or move the stand.}

\item{Always read from the bottom of the meniscus. Students often forget this; it helps to remind them at the beginning of a practical. In plastic apparatus, there is often no meniscus.}

\item{Burettes are accurate to 2 decimal places. Many times, students are taught that the last number should be either 5 or 0, like 15.55 or 15.50. This is incorrect – students should estimate the fluid level in the burette to the nearest 0.01 mL.}

\end{itemize}

\section{Titration Procedure}

\begin{itemize}

\item{Clean the burette with water. Then rinse it with the solution you will be using for titration.}

\item{Fill the burette with the solution. Allow a little solution to run out of the tip until the top of the fluid is at either 0.00 mL exactly or any value below. An initial volume of 1.32 mL is completely acceptable, at least from a scientific point of view. Your country may have specific expectations for marking exams.}

\item{Record the initial burette reading.}

\item{Use a syringe to transfer the other solution into a conical flask. Record the volume moved by the syringe.}

\item{If you are using indicator, add a few drops to the conical flask. For acid-base indicators, the less indicator used the better. In order to change color the indicator itself must react with some of the fluid from the burette. This consumes more chemical than is technically needed for neutralization; the additional chemicals required for titrating the indicator is called indicator error. One or two drops is really all you need. For starch indicator, use about 1 mL. The starch is not titrated, unlike acid-base indicators, so you can use more and often must to get a good color.}

\item{Slowly add solution from the burette to the conical flask. As you titrate, swirl the flask to mix. Do not shake it back and forth, because the solution in the flask will splatter onto the sides of the flask and thus will not be part of the neutralization reaction. Much the same, be careful to add the drops from the burette so they fall into the solution and are not stuck on the side of the flask. Stop titration when the indicator starts a permanent, slight color change. This is the endpoint. Again, the slightest change in color to the appropriate color indicates the endpoint, as long as the color remains after a few swirls.}

\item{Record the final burette reading.}

\end{itemize}

Titration is often done four times: a pilot followed by three trials. The purpose of the pilot is to find the approximate volume from the burette. The pilot is done quickly, and often overshoots the endpoint. In subsequent titration, use the results of the pilot to avoid overshooting while speeding up the work. For example, if the pilot gave an endpoint of 26 mL, add your volume rapidly from the burette until about 20 mL. Then add drop by drop until you find the endpoint.

The result from the pilot is not considered in calculations, as it is not expected to be accurate. Do not include it when finding the average volume or the variance.
