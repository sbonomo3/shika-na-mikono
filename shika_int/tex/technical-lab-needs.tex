\chapter{Specific Technical Needs of a School Laboratory}

Laboratories facilitate hands-on investigation of various phenomena. 
Every syllabus requires different topics for study, 
but a core of topics provide a good foundation for each subject.

\section{Basic Biology Laboratory}

A basic biology laboratory should allow the following investigations:
\begin{itemize}
\item{Collection, shelter, and observation of living specimens 
(plant, insect, fish, reptile, mammal)}
\item{Bacterial and fungal cultures}
\item{Preservation and dissection of dead specimens 
(plant, insect, fish, reptile, mammal; both whole and parts thereof)}
\item{Assembly and observation of miniature ecosystems}
\item{Low power microscopy}
\item{Diffusion and osmosis}
\item{Chemical tests of basic biological molecules 
(``biochemical tests'' / ``food tests'')}
\item{Chemical analysis of the products of animal and plant respiration}
\item{Non-invasive investigation of human systems 
(nervous, sensory, circulatory, muscular, parts of the digestive)}
\end{itemize}

Key materials are:
\begin{itemize}
\item{Containers, bottles, tubes, super glue}
\item{Plants, insects, fish, (safe) reptiles, and small mammals}
\item{Sugar, starch, protein source, fertilizer, salt, food coloring}
\item{Chemicals for preservation of specimens}
\item{Scalpels and pins}
\item{Low power microscopes (water drop microscopes, locally assembled)}
\item{Reagents for biochemical tests}
\item{Reagents for gas identification}
\item{Stopwatches}
\item{Heat sources}
\end{itemize}

\section{Basic Chemistry Laboratory}

A basic chemistry laboratory should allow for the following investigations:
\begin{itemize}
\item{Distinguishing compounds from mixtures, 
preparing chemical compounds, separating mixtures}
\item{Changes in the state of matter (melting/freezing, 
evaporation/condensation, sublimation/deposition, 
dissolution/crystallization)}
\item{Comparison of metals and non-metals}
\item{Comparison of covalent and ionic (electrovalent) compounds}
\item{Observing various elements and compounds 
and their reactivity with air, water, acids and bases}
\item{Acid/base, oxidation/reduction, and precipitation reactions}
\item{Energy changes from chemical reactions (thermochemistry, energetics)}
\item{Factors affecting the rates of chemical reactions (chemical kinetics)}
\item{Properties of gases (gas laws)}
\item{Preparation of gases (hydrogen, oxygen, carbon dioxide)}
\item{Electrochemical experiments 
(conductivity, electrolysis, electroplating, voltage generation)}
\item{Volumetric analysis (titration)}
\item{Identification of unknown salts (``qualitative analysis'')}
\item{Very basic organic reactions 
(e.g. preparation of ethanol by fermentation, 
oxidation of ethanol to ethanal)}
\end{itemize}

Key materials are:
\begin{itemize}
\item{Containers, bottles, tubes, balloons}
\item{Tools for measuring volume 
(calibrated plastic water bottles, plastic syringes)}
\item{Low cost balance (digital)}
\item{Heat sources and open non-luminous flames}
\item{Stopwatches}
\item{Power supplies (e.g. batteries) and wires}
\item{Wide variety of chemicals including metallic elements, 
non-metallic elements, solid covalent compounds, salts, 
acids, bases, redox reagents, indicators, 
and many chemicals for specific kinds of reactions}
\end{itemize}

\section{Basic Physics Laboratory}

A basic physics laboratory should allow for the following investigations:
\begin{itemize}
\item{Measuring volume, mass, and density of liquids and solid objects}
\item{Measuring time, velocity, acceleration}
\item{Gravitational acceleration, force, and friction}
\item{Mechanical tools (levers, pulleys, etc.)}
\item{Simple harmonic motion (pendulum, spring)}
\item{Temperature, heat capacity, and heat transfer 
(conduction, convection, radiation)}
\item{Waves (including water and sound)}
\item{Optical experiments (reflection, refraction, diffraction)}
\item{Electromagnetic experiments 
(conductivity, magnetic field lines, 
induction, motors, electrical generation)}
\item{Simple circuits (including resistors, capacitors, and switches)}
\end{itemize}

Key materials are:
\begin{itemize}
\item{A low cost balance (digital)}
\item{Tools for measuring volume}
\item{Containers, misc. objects, bottles, etc.}
\item{Stopwatches}
\item{Heat sources}
\item{Thermometers}
\item{String, springs, wire}
\item{Water, oil, sand, rocks}
\item{Mirrors, lenses, glass blocks, diffracting surfaces}
\item{Magnets}
\item{Power supply (e.g. batteries)}
\item{Inexpensive multimeters or locally made galvanometers}
\item{Electrical components}
\end{itemize}
