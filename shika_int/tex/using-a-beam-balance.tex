\chapter{Use of the Beam Balance}

\section{Measuring Mass}

A common tool for measuring mass is the triple beam balance. The name comes from the three parallel beams holding sliding weights, labeled in the diagram below. On one side of pivot point there is either a flat metal surface or a boom suspending a weighing tray. On the other side of the fulcrum, one the three parallel beams, are weights that the user slides closer or further to the pivot point. At the far end of the three beams is some kind of level indicator showing when the balance is in equilibrium or, if not, which side is too heavy.

\section{Calibration}

Calibrate the balance prior to use. Move all the sliding masses as far as they go towards the pivot point – the zero mass mark. There are usually small groves that the sliders will fit snugly in. Make sure they are in those groves – each slider except for the smallest should “click” into place. Take off any weight on the weighing tray and clean it completely. Look at the level indicator. There are two pieces. The right side not moving, but the left side of the level will move on addition of mass. The level shows the balance is calibrated when the level forms an unbroken horizontal line. If the balance is not level, there usually is a massive screw or a dial under the weigh pan. Turn it until the balance becomes level.

\section{Weighing Samples}

Triple beam balances are very accurate at measuring masses if used properly. Do not measure the chemicals directly on the metal weighing tray; use a piece of paper or glass. Many samples will react with the metal, permanently altering its mass and ruining the balance. Because the paper of glass you put the chemical on has mass, before adding any chemical you must weigh the paper or the glass first by itself. To weigh properly, move the sliders slowly until the balance becomes level or makes a horizontal line. Start with the smallest. If you reach the end before the balance equalizes, return the mass to its zero and start moving the next larger mass, one stop at a time. When the balance “tips,” move back one notch and again move the smallest slider until the balance is level. Record this mass by adding each of the slides together. The mass should be recorded to one decimal place beyond the units of the smallest lines on the balance. For example, if the lines each represent 0.1g, estimate the position of the slider to the nearest 0.01~g.

Sum the desired chemical mass with mass of the paper or glass you just measured. Move the sliders to this total mass. Now, slowly add the chemical onto the paper or glass until the beam balance becomes level. After weighing, transfer the chemical from the glass or paper into whatever will actually hold it. If you use a glass and plan for the sample to be dissolved, rinse the glass into your solution container to get every last bit of chemical into your solution. If you spill any chemical on the balance, clean it up immediately.

\section{Simplified Procedure}

\begin{itemize}

\item{Clean and calibrate balance}
\item{Use some paper or glass and move the sliders till level}
\item{Sum the mass of desired chemicals to the mass of the paper or watch glass.}
\item{Add the chemical until balance is level.}
\item{Transfer chemical to receiving container.}
\item{Clean up any spills}

\end{itemize}

\section{Other Important Tips}

Many times, you will need to measure small masses, less than 5 grams. Unfortunately, the beam balance is not as accurate when measuring such small masses, as movements in the air can cause the balance to err. To overcome this problem, place an additional mass on the weighing tray along with the paper so that the effective mass is much larger. If you are using a glass container, this step is probably unnecessary. If you add another object to the tray, make sure that there is enough space still for your chemical!

Wind is another difficulty – find a place to weigh where the air is still, perhaps in a closed room or behind some sort of obstacle or screen.

If you need many samples each the same weight, use papers of identical size and therefore mass. This allows you to keep the sliding masses in the same place for each weighing.

If you are measuring a deliquescent chemical – one that takes in water from the air, e.g. sodium hydroxide, iron (III) chloride, etc – work efficiently, but remain careful not to spill. Close the stock chemical bottle as soon as possible after use. Measure the chemical on glass rather than paper if possible as the paper often absorbs the solution that forms as the chemical deliquesces.

Finally, make sure that the volume of substance you are measuring will physically fit on your paper or glass. For volumes greater than 20g of most substances, consider using a beaker or plastic container. For volumes 100g or greater, you almost certainly need a wide mouthed and high walled vessel to hold it all. Look at the volume of substance in a contain of known mass to have an idea of how much space your sample will occupy.
