\chapter{Volumetric Analysis Without Burettes}

\section{Theory}

Burettes are not necessary to perform volumetric analysis with reasonable precision. Students may use plastic syringes in place of burettes. These should be the most precise syringes available, which as of late 2010 were the 10~mL NeoJect brand plastic syringes. These syringes are more accurate than the low cost glass pipettes that many school purchase. As the accuracy of the titration is no better than its least accurate instrument, a titration with two plastic syringes is more accurate than a titration with a burette and a cheap glass pipette.

If use of these syringes is new to you, please read Use of Plastic Syringes before proceeding.

To get maximum precision from plastic syringes, students should learn how to estimate values between the lines on the syringe body. The NeoJect syringes are marked with lines every 0.2~mL. Students should observe the top of the fluid and decide if it is on the line exactly, half way in between, or in between half way and one of the lines. This allows them to divide the space between lines into four parts, giving them a precision of 0.05~mL. Estimation between gradations is standard practice with scientific instruments; even students using burettes should estimate the fluid height between the lines to at least 0.05~mL. Syringes have the capacity to deliver the precision required by most if not all  national exams.

If students are using syringes in place of burettes, they require two syringes for the practical, one as a burette and a different one as the pipette. We recommend that you label the syringes, for example, on one syringe writing ‘Burette’ with a permanent pen to help students remember which is which.

\section{Titration Procedure without Burettes}

\begin{enumerate}

\item{Clean the ‘pipette’ syringe with water. Then rinse it with the acid or base solution you will be putting in the flask.}

\item{Use a syringe to transfer the required amount of acid or base to the flask. To do this transfer accurately, add first 1 mL of air to the syringe and then suck up the fluid to beyond the desired amount. Push back the plunger until the top of the fluid is exactly the volume required. Delivering the required volume to the flask may take multiple transfers with the single syringe. Record the total volume transferred to the flask as the ‘volume of pipette used’}

\item{Add one or two drops of indicator to the flask.}

\item{Clean the ‘burette’ syringe with water. Then rinse it with the acid or base solution you will be using to titrate.}

\item{Add 1~mL of air to the syringe and then suck up the acid or base solution to beyond the 10~mL mark. Slowly push back the plunger until the top of the fluid is exactly at the 10 mL line.}

\item{Slowly add the solution from the syringe to the flask. As you titrate, swirl the flask to mix. As described above, swirl instead of shaking to keep all of the liquid together. Make sure that each drop from the syringe hits the liquid rather than getting suck on the edge of the container. Stop titration when the indicator starts a permanent color change. Just as with a burette, this is the endpoint.}

\item{Often the volume required from the ‘burette’ is greater than 10~mL. This is no problem – after finishing the syringe students should simply fill it again as they did the first time and continue. On their rough paper (scratch paper), they should note that they have already consumed 10~mL.}

\end{enumerate}

\section{Table of Results when using syringes in place of burettes}

At present, many national exam marking boards expect students to use burettes. The obvious problem is that while the top line on a burette is 0~mL, the top of the syringe reads 10~mL. For students to get the marks their careful technique deserves, they must record their results in a manner consistent with traditional reporting. On rough paper, students should calculate the volume of solution used in their titration. This is easy -- if the syringe started at 10.00 mL and ended at 2.55~mL, the student used $10.00 \mathrm{mL} - 2.55 \mathrm{mL} = 7.45 \mathrm{mL}$ of solution. If the student used two full syringes and the third finished at 4.65~mL, then the student used $10.00 \mathrm{mL} - 4.65 \mathrm{mL} = 5.35 \mathrm{mL}$ in the last syringe plus 10~mL in each of the first two syringes, so $5.35 \mathrm{mL} + 10 \mathrm{mL} + 10 \mathrm{mL} = 25.35 \mathrm{mL}$ total.

In the Table of Results, the student should then write 25.35~mL for the Volume Used. If this volume had been used in a burette, the student would have found an initial volume of 0.00~mL and a final volume of 25.35~mL. The rest of the table should be filled in this manner. When using a syringe as a burette, the student should always write 0.00~mL for the Initial Volume and then for Final Volume they should write the total number they calculated for Volume Used. This method will ensure that the students gets the marks he or she deserves for careful titration – and likewise ensure that he or she loses the appropriate marks for mistakes.
