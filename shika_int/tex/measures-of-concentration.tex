\chapter{Measures of Concentration}

\section{Molarity (M)}

Molarity is the number of moles of substance per liter of SOLUTION. Note that molarity is not the number of moles of substance per liter of solvent (e.g. water), although practically these are very similar. A molar solution has a concentration of 1~M.

\section{Density and percent purity}

These measurements are used to find the concentration of stock acid solutions. The acid bottle should list two pieces of information: the density of the acid in $\frac{g}{cm^3}$ or $\frac{kg}{dm^3}$, and the percent purity of the acid. The percent purity tells you what portion of the density is due to the acid itself, and what portion is due to water or impurities. See the chapter on Calculating the Molarity of Bottled Reagents to see how this information is used to find molarity.

\section{Percent by mass}

The percent by mass of a solute (\% or $^w/_w$ or $^m/_m$) is the grams of the solute in 100~g of solution. Now, for most practicals, solutions do not need to be very precise. Thus it is acceptable to let the percent by mass just be the grams of solute in 100~ml of water. This makes these solutions much faster to prepare.

Such approximation may not suffice for more advanced work. Consider a 1\% by mass solution of copper (II) sulfate. This solution should contain 1~g of \ce{CuSO4} in 100~g of solution. This means that the mass of water is $100 g - 1 g = 99 g$. By assuming that the density of water before adding the solute is $1 ^g/_{mL}$, we find that 99~mL of water must be combined with 1~g of \ce{CuSO4} to make the solution. This difference matters if you are making, say, a solution of iron sulfate on which students will perform a redox titration. 

\section{Percent by volume (\% or $^v/_v$)}

Percent by volume is used to measure concentration for a mixture of a liquid chemical and water. It is equal to the volume of the chemical divided by the volume of the solution.

Example: What volume of pure ethanol must be used to make 500~mL of a 70\% ethanol solution?

70\% ethanol means 70~mL ethanol per 100~mL of solution. Thus, the required volume is:

\[ \mathrm{volume of pure ethanol} = \mathrm{total solution volume} \times \mathrm{desired fraction ethanol} \]
\[ V = 500ml \times 0.70 \]
\[ V = 350 mL \] 

\section{Normality (N)}

The normality of the solution is closely related to the molarity. For many solutions, the normality IS the molarity. Normality is generally used in older books to refer to acid and base solutions. Technically, it is the ``moles of equivalent'' per liter. So for an acid solution, it is the moles of \ce{H+} per liter of solution. For a base solution, it is the moles of \ce{H+} that may be neutralized per liter of solution. For example, 1~M \ce{HCl} has one mole of \ce{H+} per liter of solution. Thus 1~M \ce{HCl} is also 1~N. However, 1~M \ce{H2SO4} provides TWO moles of \ce{H+} per liter of solution, so 1~M \ce{H2SO4} is 2~N. In a similar vein, 1~M \ce{NaOH} is 1~N, but 1~M \ce{Na2CO3} is 2~N.

\section{Molality}

MolaRity is the number of moles of solute per liter of solution. MolaLity is the number of moles of solute per kilogram of SOLVENT. In dilute aqueous solutions, the molarity and the molality are almost the same.

\section{Some Notes on Calculations}

Many textbooks and student notebooks transcribed from them feature equations that range between novel and obtuse to the American eye. Here are two very common equations that you should be aware of, mostly because the teachers that mark exams expect students to use them.

First off, the equation that often defined molarity as $M = \frac{concentration}{molecular\mbox{ }mass}$. That is, molarity is equal to the concentration in grams per liter divided by the molecular mass of the solute (in grams per mole).

Second, the central equation for titration calculations: 

\[ \frac{(M_A)(V_A)}{(M_B)(V_B)} = \frac{n_A}{n_B} \]

A refers to the acid, B to the base, M to molarity, V to volume, and N to the stoichiometric coefficient of the acid/base in the reaction equation.

This said, there is a strong case to be made for teaching students equations that rely on an understanding of moles rather than encouraging them to memorize antiquated methods. The above equations essentially try to circumvent the need to think about moles. If you are teaching ordinary level, teach your students moles, and then show how the molarity and titrations equations come about from this unifying concept. If students can reduce every quantitative problem to moles, they will have a better understanding of the manipulations they are performing.
