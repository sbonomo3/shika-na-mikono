\chapter{Preservation of Specimens}

\section{Dead Specimens}
\begin{itemize}

\item{Mosses and lichens: 
Wrap in paper or keep in a closed container.}

\item{Plants and parts thereof: 
hang in the sun until dry. 
Alternatively, press the plants using absorbent material and a stack of books.}

\item{Insects: 
Leave exposed to air but out of reach by other insects 
until bacteria eat everything except the exoskeleton. 
If you want to preserve the soft tissue, 
store under methylated spirits.}

\item{Fish, worms, amphibians, and reptiles: 
Store in methylated spirits (will makes specimens brittle) 
or a 10\% formaldehyde solution (more poisonous and more expensive).}

\item{Parts of mammals (e.g. pig eyes, bovine reproductive organs): 
store in 10\% formaldehyde solution.}

\end{itemize}

\section{Skeletons}
Skin the animal and remove as much meat as possible. 
Bury the bones for several months. 
Exhume and assemble with wire and superglue.

\section{Living Specimens}
Be creative! 
Figure out what the animal will eat, 
who will feed it, what it will drink, where it can hide, 
how it can be observed, etc.
