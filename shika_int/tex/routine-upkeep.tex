\chapter{Routine Cleanup and Upkeep}
\label{cha:routineup}
Like gardens and children, 
laboratories require constant attention. 
The Second Law of Thermodynamics does not sleep. 
The following advice should keep you on the winning side 
of the struggle against entropy.

\section{Things to do immediately}
\begin{itemize}
\item{Remove broken glass from the floor. 
Use tools, 
like pieces of cardboard, 
not fingers!}
\item{Neutralize and wash up chemical spills}
\item{Replace chemical labels that have fallen off}
\end{itemize}
The person who made the mess should clean it up. 
Make sure they know how before they are in a position to make a mess. 
If they are unable (e.g. 
hurt), 
have someone else do it. 
Review the incident with everyone present focusing on 
how to prevent similar accidents in the future. 
Avoid blaming other people -- 
as the supervisor the accident is your fault; 
either you did not train someone well enough 
or your supervision of their behavior/technique was inadequate.

\section{Things to do right after every lab use}
\begin{itemize}
\item{Return stock containers of chemicals to the store area. 
Only teachers should move glass bottles of corrosive or toxic chemicals. 
Remember to carry these with two hands!}
\item{Transfer waste, 
including chemicals to be reused, 
into suitable storage containers}
\item{Return apparatus to their proper places}
\item{Put broken apparatus in a special place}
\item{Wash off all benches / tables}
\end{itemize}
The people who used the lab should do these things. 
If it is a lab class, 
the students should clean up the lab in that class period. 
If it is a group of teachers preparing experiments, 
the teachers should clean up their mess. 
Mess tends to grow with time, 
and no one wants to clean up someone else's mess.

\section{Things to do either right after lab use or later that same day}
\begin{itemize}
\item{Transfer chemicals to be reused into more permanent 
and well labeled storage containers.}
\item{Process all waste for disposal. 
See the instructions in \nameref{cha:wastedisp}}
\item{Remove all trash from the laboratory}
\end{itemize}
If done right after lab use, 
those who used the lab should do this work. 
If the work is done later 
anyone can take out the trash 
but waste should only be processed 
by someone who knows what (s)he is doing, 
and never working alone.

\section{Things to do every week}
\begin{itemize}
\item{Sweep and mop the floor. 
Note that this should be done with brooms and buckets of water, 
or long handled mops, 
not by pushing cloth on the floor directly with hands.}
\item{Wipe down the chemical storage area. 
Check for broken and leaking bottles.}
\item{Ensure that sinks (if present) are not clogged. 
If a sink is clogged, 
either unclog it immediately or prevent use of the sink 
by physically obstructing the basin and also writing a sign. 
Signs by themselves are often insufficient. 
Barriers with signs tend to get moved.}
\end{itemize}
You can do this work or you can train students to do it. 
Supervise their work while they are learning 
to make sure they use safe techniques. 
Ensure that students never work alone -- 
even for mopping at least two students must be present at all times. 
Students should not work in the chemical storage area 
without a teacher present.

\section{Things to do when you have time}
\begin{itemize}
\item{Unclog sinks}
\item{Repair broken apparatus}
\item{Rearrange materials -- 
make sure you plan enough time to finish the job!}
\end{itemize}
These are good projects to do together with students or other teachers.
