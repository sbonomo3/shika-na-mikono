\chapter{Use of a Plastic Syringe}

\section{Safety First}

\paragraph{Syringes are probably the best means of transferring specific volumes. They are also very safe – if used correctly. First, many syringes come with sterile needles in the same package. If this is the case, open the packages yourself and collect the needles. Never provide students with both syringes and needles. Syringe needles are designed to inject compounds into the bloodstream. Many laboratory chemicals can be very toxic if injected into the blood, and any injection done improperly carries significant risk of serious infection. Laboratory syringes should be used without the needles. If you decide to keep the syringe needles for tools (e.g. optical pins, dissection pins), store them in a well labeled container. If you decide to not keep the syringe needles, dispose of them in a sharps bin at a health center (best) or in a pit latrine.}

\paragraph{Laboratory syringes should never be used for anything other than work in the laboratory. They should never leave the laboratory. Do not let students play with the syringes like squirt guns or point them at students’ eyes even when empty. The mantra for all gun users – treat every gun like a loaded gun – should apply to syringes. They should be held with the nozzle pointed down.}

\paragraph{Anyone working with organic solvents or concentrated acids/bases should wear goggles, whether or not syringes are involved.}

\section{Measuring Volume}

There are two methods ways to use a syringe. The second is superior.

\subsection{Direct Measure}
Place the syringe in the solution you want to measure. Push the plunger completely in to remove all air. Draw the plunger back past the proper volume is measured. Use the front of the rubber plunger to read the volume measured. Slowly push the plunger in until the rubber reaches the desired volume. Remove the syringe from the liquid being measure and transfer the liquid to the desired receptacle.

This method is a poor way to use the syringe. First of all, it is difficult to remove all the air bubbles from syringes. You will push the plunger in and out many times and still not be free of the bubbles. Often students turn the syringe upside-down and try pushing the bubbles up and out. While this effectively removes air, this method is likely to eject chemicals out into a student’s eye. 

In addition, using the rubber stopper to measure is surprisingly difficult. It is hard to see the volume markings, and the curvature of the rubber can cause confusion. Also, the refractive index of water is different than air, introducing additional error. Finally, if this method is used to measure organic solvents or concentrated solvents, these chemicals will react with the rubber in the syringe. This will make the rubber sticky and difficult to draw in and out. This makes the likelihood of an accident even higher. Therefore, we do not recommend this method for measuring volume with a syringe.

\subsection{Air Bubble Method}
Before putting the syringe into the solution you want to measure, draw back the plunger so it hold about 1 mL of air. Now put the syringe in the solution. Draw the plunger back beyond the desired volume. This time, there will be a large air bubble between the rubber and the top of the solution. Hold the syringe about the liquid being measure and push down the plunger until the top of the liquid inside is at the desired volume. Make sure that the top level of the liquid is level with your eye to prevent parallax error. Hold the container of liquid up so liquid exiting the syringe does not fall a long distance and splatter. Transfer the measured volume to its receptacle. 

This method is the preferred manner of using a syringe. The air bubble allows for easier and more exact volume measurements. In addition, this method can be used with concentrated chemicals and organic solvents. The air bubble does not allow these chemicals to come in contact with the rubber, at least on the initial measure. The rubber will start to react with the residue, and without prompt cleaning this can destroy the syringe.
 
\section{Cleaning Syringes After Use}

Like all lab equipment, syringes need to be cleaned after use. Fill a beaker or other open mouth container with water. Draw water into the syringe and push it out. Repeat 2 or 3 times. If you used the syringe to measure an organic solvent, wash the syringe thoroughly in soapy water and then rinse in ordinary water until all the soap is removed.
