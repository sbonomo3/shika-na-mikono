\chapter{About This Book}

Just before the rains returned at the end of 2009, 
Peace Corps Tanzania recruited Leigh Carroll and me 
to conduct a training on laboratory methods 
for the new cohort of education Volunteers. 
We decided to write a hand-out for our presentation, 
summarizing ``everything that Volunteers needed to know about the lab.'' 
We quickly realized that this was a much larger undertaking 
than we could manage on our own 
and began recruiting other Peace Corps Volunteers. 
Thus the Shika na Mikono project was born.

As the first edition came together, several themes emerged. 
We believe that hands-on activities are essential for learning science. 
We believe that the use of local and low cost materials 
can enable any school to do these activities. 
Finally, we believe that for hands-on science to be successful, 
it must be safe. 
These three ideas -- interactive learning, equity, and safety -- 
remain the core of the Shika vision.

The first Shika na Mikono spread far beyond 
its original use in Peace Corps. 
Within six months of its publication, 
copies were used in several teacher trainings in Tanzania, 
hundreds of additional copies were locally printed in various regions, 
and the book was presented to the Tanzanian Minister of Education. 
Shika na Mikono became a project, a set of values, 
and a solid cadre of Peace Corps Volunteers 
joined in furthering these ideas in the field. 
Peace Corps gave me a third year extension to 
coordinate this work and to train both Peace Corps Volunteers 
and Tanzanians in these methods.

Given the enthusiastic response to the first edition 
and the significant work the project had done in the past year, 
we decided to produce a second edition, 
this one for the 2010 Peace Corps pre-service training. 
Once again, many people contributed to this effort, 
most especially PCVs Michael Rush, Kristen Grauer-Gray, and Peter McDonough 
-- without their excellent ideas and passionate hard work, 
this book would not exist.

The revised edition proved even more popular, 
and we began receiving requests for information 
from other Peace Corps posts, beyond Tanzania. 
Much of the value of the Tanzanian version of this book 
is its specificity -- 
chemicals sourced to an address in the capital, 
Swahili vocabulary for specific items and actions, 
explicit instructions for performing practical exercises 
for the national examinations. 
Nevertheless, we thought that some of the general information 
in the text would be of use abroad, 
and have prepared this version -- 
the first ``internationalized'' edition of Shika na Mikono. 
Our hope is that this book will equip teachers in other countries 
with some core ideas and inspire them to make 
a book of specific utility to their environment. 
PCVs Jessi Bond, Carolyn Rhodebeck, and Dylan Masters 
created additional content for this version. 
PCV Dave Berg provided the inspiration, instruction, 
and much of the labor to import this version into \LaTeX.

Many of the ideas for locally available materials 
come from or were inspired by the Source Books 
published by the Mzumbe Book Project, Morogoro, Tanzania. 
Several other ideas for locally available materials 
were developed at Bihawana Secondary by Mwl. Mohamed Mwijuma. 
PCVs Peter Finin and Gregor Passolt wrote 
a book on physics demonstrations in 2008 
that has been incorporated wholesale 
into the Hands-On Activities section of this book. 
My own knowledge of the laboratory was greatly increased 
by a brilliant if ancient book found on the shelves of Bihawana Secondary -- 
the cover and title pages with the title have long sense been lost, 
but the preface sites G.P.Rendle, M.D.W. Vokins and P.M.H. Davis as authors, 
and 1967 as the date of publication.

We are all grateful to our schools 
for giving us the opportunity to work in such supportive environments, 
the freedom to explore these ideas, 
and the time to document them. 
We have certainly benefited from the wisdom 
and creativity of many other teachers, 
both in this country and in America. 
Many of us working without reliable electricity 
or internet connections benefited enormously 
from the hospitality of the numerous expatriate families 
who sheltered us in town. 
We are all grateful to Peace Corps Tanzania for supporting our work, 
especially James Ogondiek and the now retired and much beloved Thomas Msuka, 
both of whom recognized early on the value of this project 
and advocated for us to undertake the work required to develop 
and spread the ideas in this book.

Most of all, we are grateful to our students, 
for it is their curiosity and enthusiasm that has motivated everything.\\[24pt]
Aron Walker\\
Bihawana Secondary, Dodoma\\
aronwalk@alum.mit.edu
