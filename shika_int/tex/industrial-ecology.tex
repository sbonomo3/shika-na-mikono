\chapter{Industrial Ecology in the Laboratory}

\textit{Industrial Ecology} is a manufacturing design philosophy 
where the byproducts of one industrial operation 
are used as input material for another. 
The philosophy may be applied to a school laboratory 
with similar economic and environmental benefits.

The science teacher generally plans the term in advance, 
and thus has a good understanding of the experiments students will perform. 
Each experiment has input reagents and output products. 
Normally, 
each of these inputs has to be purchased, 
sometimes at great expense, 
and each of these outputs has to be disposed of properly. 
When the term is analyzed in aggregate, 
however, 
there should be many occasions where the outputs of one experiment 
may serve as the inputs for another.

For example, 
students learning about exothermic reactions might 
dissolve sodium hydroxide in water and measure the temperature increase. 
The students might then use this solution 
of sodium hydroxide for a titration against a solution of ethanoic acid. 
The product of this titration will be 
perfectly balanced sodium ethanoate solution, 
which may be used in qualitative analysis for detecting iron (III) salts.

The maximize the opportunities for such pairings of inputs and outputs, 
the teacher should identify the reagents and byproducts of 
all activities planned for the term. 
Teachers may even coordinate between subjects - 
the reaction between citric acid and sodium carbonate 
to make carbon dioxide in chemistry class 
produces a sodium citrate solution that may be used 
to prepare Benedict's solution for biology class.
