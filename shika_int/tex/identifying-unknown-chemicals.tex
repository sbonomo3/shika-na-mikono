\chapter{Identifying Unknown Chemicals}
\label{cha:unknownchemicals}
Unlabelled chemicals are dangerous. 
If you do not know what the chemical is, 
then you do not know what to do if it spills, 
or how to safely get it out of your school.

%==============================================================================
\section{Identifying Bottles of Unknown Liquids}

Usually, 
these are:
Concentrated acids (sulfuric, 
hydrochloric, 
nitric, 
ethanoic)
Concentrated ammonia solution
Organic solvents including methanol, 
ethanol, 
isobutanol, 
propanone (acetone), 
diethyl ether, 
ethyl ethanoate (ethyl acetate), 
dichloromethane, 
trichloromethane (chloroform), 
tetrachloromethane (carbon tetrachloride), 
trichloroethene, 
benzene, 
chlorobenzene, 
toluene, 
xylene, 
and petroleum spirits

Distinguishing these chemicals is important, 
and relatively possible. 
Here is a procedure:

First, 
protect yourself against whatever it might be. 
Concentrated acids burn skin on contact and blind if they get in the eyes. 
Concentrated hydrochloric acid and concentrated ammonia solution 
release fumes that corrode the throat and lungs. 
Diethyl ether and propanone rapidly evaporate at room temperature 
and pose a significant flash fire hazard if opened near flame. 
Ingesting even a small amount of toxic carbon tetrachloride can be fatal, 
and benzene is a proven and serious carcinogen.

Why, 
you might say, 
should I even attempt this? Because sooner or later, 
someone will, 
and better it be someone with these instructions than without. 
But if you do not feel comfortable, 
call a friend who is more excited about this process.

Many precautions are available. 
Tie a cloth over your mouth and nose to mitigate inhalation. 
Find a pair of goggles or sunglasses to protect you eyes 
from any splash when opening the stopper in the bottle. 
Wear gloves or at least plastic bags on your hands. 
Neither will protect your hands for more than a second 
or a few against concentrated acids or some organics, 
but that second can be useful in this case. 
Thick rubber gloves are available (see \nameref{cha:labequip}) 
and offer greater protection. 
Regardless, 
have at the ready a bucket of water and a box of baking soda 
(bicarbonate of soda) to neutralize acid burns. 
Move the container outside and remain upwind. 
Have a small, 
dry, 
clean beaker ready to hold a sample.

Open the bottle. 
This may be as simple as unscrewing the top 
or there may be an internal stopper that requires prying off. 
Find a suitable tool, 
one that can pry under the cap but cut neither the cap nor you. 
A butter knife works well. 
Do not use your fingers.

When the bottle opens, 
look at the top. 
Are there white fumes? 
Is there an obvious smell that you can perceive 
from where you are standing? 
White fumes suggest hydrochloric acid 
and an intense smell could be ammonia (smells like stale urine), 
hydrochloric or ethanoic acid (both smell like vinegar), 
or an organic solvent (various odors).

If the contents smell obviously like ammonia, 
there is no need to further experimentation. 
Nothing else in schools smells even remotely like ammonia. 
Stopper that bottle and give it a good label.

Otherwise, 
carefully, 
pour a few cubic centimeters 
of the liquid into your sample beaker. 
As you pour the liquid, 
observe the viscosity. 
Concentrated acids are all noticeably more viscous than water, 
especially concentrated sulfuric acid. 
Propanone, 
on the other hand, 
is noticeably more fluid than water. 
Close the bottle and take the beaker 
to a safe place for experimentation.

Color is surprisingly useless in identifying unknown liquids 
because most readily take on color 
from even small amounts of contamination.

Rest the beaker on a sturdy surface. 
If you have already noticed an intense smell, 
leave the cloth on your face. 
If you have not yet noticed a smell, 
remove it.

%==============================================================================
\section{Test one: Add to water}

Fill a large, 
clean test tube half way with ordinary water. 
Alternatively, 
find the smallest beaker you have (probably 50~mL), 
and fill it about a quarter of the way with water. 
Carefully pour in a few drops of your unknown 
and observe what happens. 


If it does not mix with the water, 
instead forming a new (possibly quite small) layer on top, 
you have an organic solvent less dense than water, 
probably one of: isobutanol, 
diethyl ether, 
ethyl ethanoate, 
benzene, 
chlorobenzene, 
toluene, 
xylene, 
or petroleum spirits. 
If it does not mix with the water, 
instead sinking to form a distinct layer on the bottom, 
you have an organic solvent more dense than water, 
probably dichloromethane, 
chloroform, 
or carbon tetrachloride.

If your unknown does not mix with water, 
jump down to \nameref{sec:testorganic} on what to do with organics.

If the unknown seems to sink into the water but not mix completely, 
you probably have a concentrated acid. 
The test tube might even get a little warmer. 
You might also have a very concentrated solution of some other solute, 
left over from a previous experiment.

If the unknown seems to mix into the water like, 
well, 
water, 
you probably have an aqueous solution that is not very concentrated. 
It might be dilute acid, 
dilute hydroxide, 
hydrogen peroxide solution, 
etc. -- more work lies ahead.

%==============================================================================
\section{Test two: Is it an acid?}

This only applies to solutions that mix completely into water.

This is easy with a piece of blue litmus paper. 
Dip a corner down into the test tube or beaker. 
If it turns bright red, 
you probably have an acid, 
and if your liquid was noticeably viscous, 
a concentrated acid. 
If there is no change, 
move on to \nameref{sec:whatelse}.

Another option is universal indicator or universal pH paper. 
Prepare a 100-fold dilution of the original acid 
and test with the indicator. 
If the color is bright red, 
you must have a strong acid, 
like hydrochloric, 
sulfuric, 
or nitric acid. 
If the color is instead orange or yellow, 
you must have a weak acid, 
like ethanoic acid. 
If there is no universal indicator, 
you can show that something definitely is an acid 
if it causes methyl orange to turn from orange to red. 
However, 
if there is no color change, 
you might still have a weak acid, 
so you cannot use methyl orange to eliminate the possibility of an acid. 
You also cannot use POP to show that there is an acid, 
as both concentrated acid and tap water have the same effect on POP: 
none whatsoever.

If you do not have any litmus paper or other indicator, 
find another beaker and add 10-20~mL of ordinary water 
and dissolve a bit of baking soda (bicarbonate of soda). 
Carefully, 
with eye protection, 
add a few drops of your DILUTED unknown (from test one). 
If there are bubbles, 
you have an acid. 
Adding a concentrated acid directly to baking powder 
can cause such vigorous effervescence as to eject acid from the test tube.

%==============================================================================
\section{Test three: What kind of acid?}

%------------------------------------------------------------------------------
\subsection{Sulfuric acid}

\begin{itemize}

\item{Hints: obviously viscous, 
significantly denser than water, 
noticeable heat released on dilution, 
no smell}

\item{Confirmatory test: dip the wooden end of a match into the original solution. 
If the end appears to char, 
you have concentrated sulfuric acid. 
Another variant of this test is take some concentrated sulfuric acid 
and pour over some sugar in a beaker. 
After some time, 
a black color from carbon produced 
from the dehydration of sugar confirms sulfuric acid. 
Yes, 
the same thing happens to skin. 
The downside of this second test is that the beaker 
is almost impossible to clean.}

\item{Alternative test: Find or prepare a 0.1~M barium nitrate, 
barium chloride, 
or lead nitrate solution. 
In a test tube, 
add about one centimeter of your diluted sample 
and then a few drops of one of the above solutions. 
An instant, 
white, 
cloudy precipitate demonstrates that sulfate is present. 
To confirm that this is from sulfuric acid and not, 
say, 
your tap water, 
test in the same way the water you used for the dilution. 
Not much should happen. 
If your tap water contains sulfates, 
find some distilled (e.g. 
rain) water and remake the dilution.}
\end{itemize}

%------------------------------------------------------------------------------
\subsection{Hydrochloric acid}

\begin{itemize}

\item{Hints: white fumes, 
intense acidic smell similar to vinegar, 
more dense than water}

\item{Confirmatory test: 
prepare a dilute potassium permanganate solution. 
This should be pink in color, 
which might require significant dilution. 
Fill a test tube with a couple centimeters of your dilute solution 
and add the potassium permanganate solution drop wise. 
If the pink color is rendered colorless 
after mixing with your diluted sample, 
you probably have hydrochloric acid. 
This reaction makes small amounts of chlorine gas, 
but that poses much less risk than the hydrochloric acid fumes.}

\item{Alternative test: 
Find or prepare a 0.05~M or 0.1~M silver nitrate solution. 
Remember that this chemical is very expensive, 
so only make a small quantity. 
In a test tube, 
add about one centimeter of the water you used 
for diluting your sample 
and then a few drops of one of silver nitrate solution. 
An instant, 
white to gray, 
cloudy precipitate demonstrates that chloride is present. 
If this happens, 
your tap water contains chlorine 
and you will have to prepare another dilution 
using rain or distilled water. 
If the water you used for dilution lacks chlorine, 
add a centimeter of the diluted sample 
to a clean test tube and add a few drops of silver nitrate solution. 
The precipitate confirms that you have hydrochloric acid. 
Note that for this test to be effective, 
the hydrochloric acid must be diluted. 
Concentrated hydrochloride acid reacts with aqueous silver 
to form the [\ce{AgCl2}]-complex, 
which is soluble.}

\end{itemize}

%------------------------------------------------------------------------------
\subsection{Ethanoic (acetic) acid}

\begin{itemize}

\item{Confirmatory Test: This acid smells strongly of vinegar. 
If you have a definite vingar smell, 
it is probably ethanoic acid, 
but beware that concentrated HCl can have a similar smell. 
To confirm ethanoic acid, 
use some diluted acid from test one 
and add a small amount of baking soda until it is just neutralized. 
Do not add excess baking soda - neutralization is the goal. 
After neutralizing, 
add a small amount of iron (III) chloride or nitrate. 
A blood red solution of iron (III) acetate 
proves that the acid is ethanoic. 
Boiling the solution should form a red brown precipitate. 
If you do not have iron (III) salts but do have universal indicator, 
use the indicator method above 
for confirming that your unknown is a weak acid -- 
ethanoic is the only common weak acid that smells like vinegar.} 
\end{itemize}

%------------------------------------------------------------------------------
\subsection{Nitric acid}
A concentrated acid in a school that does not smell like vinegar 
and is not hydrochloric or sulfuric acid is very likely to be nitric acid. 

\begin{itemize}

\item{Confirmatory Tests: Take a wooden splinter 
or match stick and dip it in the concentrated acid. 
If the splinter turns yellow, 
the acid is nitric. 
A second confirmatory test is adding copper wire 
or turnings to the concentrated acid. 
A brown gas of nitrogen dioxide is formed. 
Do this confirmatory test in a well ventilated area.}

\item{Special note: if you suspect nitric acid, 
dip a piece of copper wire into the solution. 
If it comes back with a silvery coating, 
you have Million’s Reagent, 
mercury metal dissolved in nitric acid. 
This is highly toxic, 
very dangerous, 
and should never be used in a school. 
Label the bottle “Million’s Reagent, 
Contains \ce{Hg2+}, 
TOXIC, 
CORROSIVE, 
do not use, 
do not dump” along with similar warnings 
in any local language(s) and find a safe place to store it.}

\end{itemize}

%==============================================================================
\section{Test four: What kind of organic?}
\label{sec:testorganic}
Let us be honest. 
Distinguishing between different kinds of organic solvents 
is hard with the resources that are probably available. 
If the chemical is more dense than water 
and no one at the school claims that it is chloroform 
(trichloromethane) for the biology lab, 
there is no way to show that it is not carbon tetrachloride 
(tetrachloromethane), 
a toxic organic solvent responsible 
for the death students in several countries. 
Label the bottle ``Unknown organic solvent more dense than water, 
possibly carbon tetrachloride, 
TOXIC, 
never use, 
never dump,'' with similar warnings in any local language(s) 
and find a safe place to store it.

If the chemical is less dense than water 
and you are familiar with organic solvents, 
you might try a careful smell test.

If the unknown smells like strong booze and is soluble in water, 
it is probably ethanol or methanol. 
Do not drink it! -- methanol blinds. 
If it is bright red, 
it is probably Sudan III solution, 
for biology. 
Label and use it. 
If it is yellow or brown it might be iodine solution, 
see below in test five. 
If it is light purple or green or whatever 
the popular color in your country, 
it is probably methylated spirits, 
a mixture of about 70\% ethanol and 30\% water
with some impurities to make it undrinkable. 
Confirm this by showing that paper soaked in the chemical 
will burn with a blue flame but that paper soaked in a 50/50 
mixture of the chemical and water will not burn. 
If it is clear and someone at the school can assure 
that the contents are ethanol and not methanol, 
label the bottle ``ethanol'' and use it. 
If the bottle might be methanol, 
a poison, 
pour the contents into a large bucket 
and leave it in a place where no one will disturb it 
and where the fumes will not accumulate. 
Let it evaporate.

If the unknown smells like nail polish remover 
and is soluble in water, 
it is probably propanone (acetone). 
If you put a drop in a spoon it should evaporate relatively quickly. 
Label it ``Propanone, 
EXTREMELY FLAMMABLE'' and keep it around. 
If it is not soluble in water and smells like magic markers, 
it is probably diethyl ether or ethyl ethanoate (ethyl acetate). 
If you are familiar with organics, 
perhaps you can pick between these. 
Otherwise just label the bottle ``volatile organic solvent, 
insoluble in water, 
EXTREMELY FLAMMABLE'' and keep it around.

If the unknown has a sweet sickly smell it might be toluene. 
It also might be benzene. 
If you cannot further identify it and no one else can, 
label the bottle ``unknown non-volatile organic solvent 
less dense that water, 
possibly benzene, 
TOXIC, 
never use, 
never dump'' with similar warnings in any local language(s), 
and find a safe place to store it.

%==============================================================================
\section{Test five: What else?}
\label{sec:whatelse}
If you unknown is not an acid, 
not ammonia, 
and soluble in water, 
see what it smells like. 
If it smells like booze or nail polish remover, 
it could be methanol, 
ethanol, 
or acetone. 
See the above section on organics. 
If it does not have a smell, 
it is probably a solution left over from an earlier lab. 
These are not nearly as dangerous as concentrated acids 
or some volatile organics. 
However, 
be sure to use proper handing methods. 
Here are some possibilities:

%------------------------------------------------------------------------------
\subsection{Sodium hydroxide solution}

\begin{itemize}

\item{Hints: cloudy and a jammed stopper, 
but not always}

\item{Test: red litmus turns blue or POP pink.}

\item{What to do: sodium hydroxide is cheap when bought as caustic soda. 
Keep it around just for neutralizing acid wastes. 
If its presence disturbs you, 
add some indicator and then cheap acid until neutralization. 
After complete neutralization, 
dump.}

\end{itemize}

%------------------------------------------------------------------------------
\subsection{Hydrogen peroxide}

\begin{itemize}

\item{Hints: colorless liquid, 
in an opaque or dark bottle}

\item{Test: add a bit of acidified potassium permanganate solution. 
The potassium permanganate should turn colorless 
on mixing and bubbles of gas should be observed.}

\item{What to do: label and use. 
If you want to dispose of it for some reason, 
leave it in a bucket in the sun.}

\end{itemize}

%------------------------------------------------------------------------------
\subsection{Potassium permanganate solution}

\begin{itemize}

\item{Hints: intensely purple, 
pink after significant dilution}

\item{Test: to a very dilute solution, 
add crushed vitamin C (ascorbic acid) from a pharmacy. 
The solution should turn colorless.}

\item{What to do: test the pH with litmus paper 
or methyl orange to see if acid has been added. 
Then label ``(acidified) potassium permanganate'' and use. 
If you want to dispose of it, 
add crushed vitamin C until the color disappears 
and then pour into a pit latrine.}

\end{itemize}

%------------------------------------------------------------------------------
\subsection{Iodine solution}

\begin{itemize}

\item{Hints: brown color, 
smells like iodine tincture, 
and possibly also like ethanol}

\item{Test: to a dilute solution, 
add crushed vitamin C (ascorbic acid) from a pharmacy. 
The solution should turn colorless.}

\item{What to do: Put a centimeter of water in a test tube 
followed by a smaller quantity of cooking oil. 
Add a few drops of the solution, 
cap with your thumb and mix thoroughly for one minute. 
If two layers quickly separate, 
the iodine solution has been prepared without ethanol. 
If a cloudy mixture (an emulsion) forms, 
the iodine solution has been prepared with ethanol. 
Label the solution ``iodine solution (with ethanol)'' and use it.}

\end{itemize}

%------------------------------------------------------------------------------
\subsection{Potassium ferrocyanide solution}

\begin{itemize}

\item{Hints: light neon green or yellow color}

\item{Test: make a dilute solution of copper sulfate and add a few drops of the unknown. 
An instant, 
massive brown precipitate confirms potassium ferrocyanide solution.}

\item{What to do: Label and use. 
Do not dump while it remains useful.}

\end{itemize}

%==============================================================================
\section{Unidentifiable Liquids}

\ldots are worthless. 
In order to safely dump a liquid chemical, 
ensure the following are true:
The liquid is water soluble (otherwise see the organic section above)
The liquid is neutral pH (if acid, 
neutralize with bicarbonate of soda, 
if base neutralize with acid waste, 
citric acid, 
or, 
carefully, 
battery acid)
The liquid does not contain heavy metals (to a small sample, 
add dilute sulfuric acid drop-wise. 
A precipitate indicated lead or barium. 
Continue adding until additional precipitation stops. 
Then neutralize with bicarbonate of soda. 
The solids are safe for disposal in a pit latrine, 
but may clog sink pipes).
The liquid does not contain mercury (to a small sample, 
add sodium hydroxide solution until POP turns the solution pink. 
A yellow precipitate indicates mercury. 
Label the solution ``Contains \ce{Hg2+}, 
TOXIC, 
do not use, 
do not dump'' and store it in a safe place.)
Then, 
dilute the chemical in a large amount of water 
and dispose of it in a lab sink or pit latrine.

%==============================================================================
\section{Deliquescent Salts}

If you have a chemical in a container that seems meant 
for holding solids but the chemical looks like a thick liquid, 
you probably have a deliquescent salt that fully deliquesced. 
These solutions can be quite dangerous 
because they are maximally concentrated. 
Make sure that no unknown chemicals touch your skin, 
and wear goggles for this work. 
Then, 
do the following:

%------------------------------------------------------------------------------
\subsection{Test for a base}
The most common deliquescent salt is sodium hydroxide. 
Fill a test tube half way with water 
and a few drops of the unknown syrup 
followed by a few drops of POP indicator. 
If the solution turns pink, 
you almost certainly have either 
sodium hydroxide or potassium hydroxide. 
Dilute the liquid in at least 10 times its 
rough volume of water and titrate a sample against 1M acid. 
Find a plastic water bottle with a screw cap for your dilution and 
label it ``sodium or potassium hydroxide, 
$n$~M'', 
where $n$ is the molarity you measured in your titration.

%------------------------------------------------------------------------------
\subsection{Color}
If the liquid is not a base, 
it is probably a chloride or nitrate salt of one element or another. 
If it is colorless, 
the cation is probably in Group IIA (Ca, 
Sr, 
or Ba) or Group IIB (Zn, 
etc). 
Group IIA compounds have distinct flame test colors: Ca = orange red, 
Sr = bright red, 
Ba = apple green) while Zn has no flame test color. 
If it is red or brown, 
it is probably iron (III) nitrate or iron (III) chloride. 
If it is intensely pink, 
it might be cobalt. 
To identify the compound completely, 
you will have to perform qualitative inorganic analysis. 
An introduction to the art is 
in the Qualitative Analysis section of this manual, 
and more advanced methods are available on the internet 
and in some advanced chemistry books.

%------------------------------------------------------------------------------
\subsection{Check for mercury}
If the liquid is not a base, 
dilute a small sample in water 
and add sodium hydroxide solution until POP turns pink. 
If a bright yellow precipitate forms, 
you probably have a mercury salt. 
Transfer all of the compound to a sturdy container 
with a well-sealing lid, 
wash the original container with minimal water once 
and add the washings to the storage container. 
Then wash the original container 
and anything the liquid touched thoroughly. 
Label the new storage container ``Solution of unknown mercury salt, 
CONTAINS Hg!!, 
TOXIC! Do not use, 
Do not dump,'' along with appropriate warnings in any local language(s), 
and find a safe and secure place for long term storage.

%==============================================================================
\section{Identifying Unknown Solid Chemicals}

This is not nearly is important as identifying unknown liquids 
for two reasons. 
First, 
these chemicals are generally (though not always!) less dangerous, 
and second, 
accidental spills are less dramatic. 
The smallest containers are the most likely to hold dangerous chemicals, 
like mercury salts. 
It is best to just leave these ones alone.

What you can do is look at the solid 
and see if it matches any of the descriptions below.
Color is much more useful for identifying solids.

\begin{itemize}

\item{Bright orange crystals are likely a chromate 
or dichromate salt (toxic) or a ferricyanide salt (much less toxic). 
The later will form an intensely blue precipitate 
with a small amount of \ce{Fe2+}, 
perhaps from iron (II) sulfate. 
Chromates form a yellow solution that turns orange 
on addition of acid while dichromates for an orange solutions 
that turns yellow on addition of base.}

\item{Bright yellow, 
orange, 
or red powders might be lead or mercury compounds. 
These are poisonous, 
the latter very. 
It also might be methyl orange powder. 
Try to dissolve a small amount in water. 
Methyl orange will dissolve readily to give a bright orange solution, 
one that turns red in acid and yellow in base. 
Label the powder and keep it around. 
If the salt dissolves but does not seem to be methyl orange, 
add sodium hydroxide until POP changes color. 
A yellow precipitate suggests mercury. 
Label as with mercury compounds encountered above. 
Most lead compounds are not soluble, 
and will not form a color in solution. 
Other mercury compounds are also insoluble. 
Label a container that might be lead or mercury as 
``possible lead or mercury compound, 
POISON,'' and store it for the long haul.}

\item{A yellow powder insoluble in water may also be sulfur. 
It should smell like sulfur. 
A small amount will dissolve in kerosene, 
and the dry powder will melt when heated in a spoon 
over a flame and then burn with a blue flame -- 
producing sulfur dioxide, 
a poisonous gas. 
Do not heat an unknown yellow compound 
unless you are fairly sure it is sulfur.}

\item{Blue compounds are often copper salts. 
These should have a green flame test.}

\item{Purple crystals or flakes insoluble in water are probably iodine. 
Iodine will dissolve in kerosene to form a red solution.}

\item{One of the few green powders is nickel carbonate.}

\item{Pink wet looking crystals might be a cobalt compound. 
Heat them gently in a spoon and they should dehydrate to turn blue. 
The blue crystals should turn pink when dissolved in water. 
Cobalt is poisonous.}

\item{Crystals so purple they look brown or yellow 
are probably potassium permanganate. 
They should form an intensely purple solution in water. 
Confirm as with potassium permanganate solution above.}

\item{White crystals and powders are really hard to identify. 
Label them ``unknown white powder/crystals'' 
and move them to a safe and secure place.}

\item{Flat dull gray metallic ribbon about 5~mm wide 
and 1~mm thick is probably magnesium metal. 
It should turn shiny if polished with steel wool. 
It will also burn with a very intense white light 
if lit in either a Bunsen burner or gas cigarette lighter. 
Hold it with tongs, 
and do not stare at the light.}

\item{A metal stored under oil is probably sodium or potassium. 
If you are feeling adventurous, 
remove a sample and cut off a VERY small piece, 
perhaps 5~mm on a side. 
Both metals may be easily cut with a knife. 
Return the rest to the original container and seal it again. 
Then, 
add the piece of metal to an open container of water and stand back. 
Both react violently and generally send the piece of metal 
spinning around on a cushion of hydrogen gas. 
Potassium generally gets hot enough to ignite this gas 
which then burns with a lilac flame. 
If the hydrogen under sodium burns, 
it will be yellow. 
The water will become a solution of sodium or potassium hydroxide.}

\end{itemize}
