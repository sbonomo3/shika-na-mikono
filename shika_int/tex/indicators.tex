\chapter{Properties of Indicators}

\section{Acid-base indicators}
These indicators are chemicals that change colors in a specific pH range, which makes them suited to use in acid-base reactions. When the pH of changes from low pH to high pH or from high to low, the color of the solution changes. 

Four common acid-base indicators are methyl orange (MO), phenolphthalein (POP), bromothymol blue (BB), and universal indicator (U)

\begin{itemize}

\item{Methyl Orange, MO, is always used when titrating a strong acid against a weak base. The pH range of MO is 4.0-6.0 and thus no color change is observed until the base is completely neutralized. If you use MO with a weak acid, the color might start to change before completely neutralizing the acid.}

\item{Phenolphthalein, POP, is always used when titrating a weak acid against a strong base. The pH range of POP is 8.3-10.0, and thus no color change is observed until the weak acid is completely neutralized. If you use POP with a weak base, the color might start to change before completely neutralizing the base.}

\item{Bromothymol Blue, BB, is used in the same manner as methyl orange.}

\item{Universal indicator, U, is not suitable for volumetric analysis involving either weak acids or bases as it changes color continuously rather than in a limited pH range. It is very useful for tracking the pH continuously over a titration, perhaps by performing two titrations side by side, one with a standard indicator and another with universal indicator.}

\end{itemize}

Any indicator can be used when titrating a strong acid against a strong base. Universal indicator, however, will not produce very accurate results.

No indicator is suitable for titrating a weak acid against a weak base.

In some experiments, more than one indicator may be used in the same flask, for example when titrating a mixture of strong and weak acids or bases.

\subsection{Colors of Indicators}
The colors of the above indicators in acid and base are:

\begin{center}
\begin{tabular}{ c | c | c | c |}
\hline
Indicator & Acid & Neutral & Base \\ \hline
Methyl Orange & Red & Orange & Yellow \\ \hline
Phenolphthalein & Colorless & Colorless & Pink \\ \hline
Bromothymol Blue & Yellow & Blue & Blue \\ \hline
Universal Indicator & Red, Orange, Yellow & Yellow/Green & Green, Blue, Indigo  \\
\hline
\end{tabular}
\end{center}

Titration is finished when the indicator starts a permanent color change. For example, when methyl orange turns orange, the titration is finished. If students wait until methyl orange turns pink (or yellow) they have overshot the endpoint of the titration, and their volume will be incorrect. Likewise, POP indicates that the titration is finished when it turns light pink. If students wait until they have an intensely pink solution, they will use too much base and get the wrong answer. 

Note that light pink POP solutions may turn colorless if left for a few minutes. This is due to carbon dioxide in the air reacting to neutralize bases in solution.

\subsection{Note on technique}
Students should use as little acid-base indicator as possible. This is because some acid or base is required to react with the indicator so that it changes color. If a lot of indicator is used, students will add more acid or base than they need.

\section{Other indicators}
Starch indicator is used in oxidation-reduction titrations involving iodine. This is because iodine forms an intense blue to black colored complex in the presence of starch. Thus starch allows a very sensitive assessment of the presence of iodine in a solution.

It is important to add the starch indicator close to the end point when there is an acid present. The acid will cleave the starch and that will prevent the starch from working properly. Students using starch should use a pilot run to get an idea when to add the starch indicator.

\section{Preparation of Indicators}
\begin{itemize}

\item{Methyl orange (MO): if you have a balance, weigh out about 1 g of methyl orange powder and dissolve it in about 1 L of water. Store the solution in a plastic water bottle with a screw on cap and it will keep for years. If it gets thick and cloudy, add a bit more water and shake. If you do not have a balance, add half of a small tea spoon to a liter of water.}

\item{Phenolphthalein (POP): Dissolve about 0.2 g of phenolphthalein powder in 100 mL of pure ethanol; then add 100 mL water with constant stirring. If you use much more water than ethanol, solid phenolphthalein will precipitate. Store POP in a plastic water bottle with a screw on cap. We recommend making POP in smaller quantities than MO as it does not keep as well, mostly due to the evaporation of ethanol. If the solution develops a precipitate, add a bit of ethanol and shake. We do not recommend using purple methylated spirits as a source of ethanol for making POP. You can distill purple spirits to make clear spirits. For clear methylated spirits, use 140ml of spirit and 60ml of water, as spirits generally are already 30\% water.}

\item{Starch: place about 1 g of starch in 10 mL of water in a test tube. Mix well. Pour this suspension into 100 mL of boiling water and continue to boil for one minute or so. Alternatively, use the water leftover after boiling pasta or potatoes. If this is too concentrated, dilute it with regular water.}

\item{The authors have never prepared bromothymol blue or universal indicator from powder, but suspect their preparation is similar to methyl orange.}

\end{itemize}

Note that the exact mass of indicator used is not very important. You just need to use enough so that the color is clearly visible. Students use very little indicator in each titration, and a liter of indicator solution should last you a long time.
