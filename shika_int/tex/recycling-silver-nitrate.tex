\chapter{Recycling Silver Nitrate}
\label{cha:recyclesilver}
In many places, 
silver nitrate is the most expensive chemical 
used in a school laboratory. 
Silver nitrate is often used to confirm the presence of halide ions, 
which form insoluble precipitates with silver cations. 
The result of such tests are silver halide precipitates, 
themselves of little value.

To regenerate the silver nitrate from these silver halides 
you must first reduce the silver halides to silver metal 
and then dissolve the metal in nitric acid. 
This process is easiest and most efficient 
with a large amount of material, 
so consider accumulating silver waste in a central location 
for many terms and perhaps from many schools.

To reduce the silver halides, 
they must be in solution. 
Add aqueous ammonia solution to the silver halides until they dissolve. 
You have formed a soluble silver - ammonia complex. 
Add to the mixture a reducing agent. 
We have used both glucose and steel wool. 
Ascorbic acid, 
zinc metal, 
and sodium thiosulfate should in theory also work. 
Heat the mixture until a metallic silver precipitate forms. 
It is OK if the solution boils.

Once you believe all of the silver has precipitated as metal, 
decant the liquid, 
ideally filtering to separate all of the silver metal. 
Wash the silver metal in distilled (rain) water and filter again.

Before adding nitric acid, 
make sure that the silver is dry. 
Then, 
add concentrated nitric acid slowly. 
The goal is to dissolve most but not all of the silver metal. 
If you dissolve all of the metal, 
you may have residual nitric acid in your silver nitrate solution 
that will make it ineffective for many uses. 
Decant the solution into a dark bottle - 
silver nitrate decomposes in light - 
and save the residual silver metal for the next time you do this.
