\chapter{Biology Practicals}

\section{Introduction to Biology Practicals}

\subsection{Format}

Until 2008, NECTA biology practicals contained three questions. Question 1 was required, and was a food test. Students then chose to answer either question 2 or question 3. One of these questions was usually classification.

The format changed in 2008. Now, the practical contains two questions, and both are required. Food test and classification remain the most common questions, but sometimes only one of these two topics is on a given exam. The second question may cover one of a variety of topics, including respiration, transport, coordination, photosynthesis, and movement.

Each question is worth 25 marks.

\subsection{Common Practicals}
\begin{itemize}
\item{Food test: students must test a solution for starch, sugars, fats, and protein}
\item{Classification: students must name and classify specimens, then answer questions about their characteristics}
\item{Respiration: students use lime water to test air from the lungs for carbon dioxide}
\item{Transport: students investigate osmosis by placing leaf petioles or pieces of raw potato in solutions of different solute concentrations}
\item{Photosynthesis: students test a variegated leaf for starch to prove that chlorophyll is necessary for photosynthesis}
\item{Coordination: students look at themselves in the mirror and answer questions about the sense organs they see}
\item{Movement: students name bones and answer questions about their structure and position in the body}
\end{itemize}

Note: These are the most common practicals, but they are not necessarily the only practicals that can occur on the national exam. Biology practicals frequently change, and it is possible that a given exam will contain a new kind of question. Look through past NECTA practicals yourself to get an idea of the kind of questions that can occur

%==============================================================================
\section{Food Tests}

In this practical, students test a solution of unknown food substances for starch, protein, reducing sugars, non-reducing sugars, and fats/oils. They record their procedure, observation, and conclusions, then answer questions about nutrition and the digestive system.

This section contains the following:
\begin{itemize}
\item{Preparation of test solutions}
\item{Preparation of food solutions}
\item{How to carry out food tests}
\item{How to write a report}
\item{Sample practical with solutions}
\end{itemize}

\subsection{Test for Starch}

\subsubsection{Materials}
\begin{itemize}
\item{iodine tincture from the pharmacy (any brand)}
\item{tap or clean river water}
\end{itemize}

\subsubsection{Procedure to make 400 mL}
\begin{enumerate}
\item{Pour one bottle of iodine tincture into a 500 mL plastic water bottle.}
\item{Add about 400 mL of water.}
Cap the bottle and shake.
4)	Use a permanent pen to label the bottle:

IODINE SOLUTION FOR FOOD TESTS

	The solution may be stored in any plastic or glass bottle and will keep indefinitely.

Test for protein
The best test for protein is the Biuret test. This requires two solutions: 1\% CuSO4 and 1 M NaOH.

Materials
	copper sulfate
	sodium hydroxide
	tap or clean river water	

Procedure to make 1 liter copper sulfate solution:
1)	Use a small metal or plastic spoon (tea size) to transfer one level spoon of copper sulfate into 	a 1 or 1.5 liter water bottle.
2)	Add about one liter of water. The amount does not have to be exact.
3)	Cap the bottle and shake until the copper sulfate has completely dissolved.
4)	Use a permanent pen to label the bottle:


1% Copper (II) Sulfate Solution
For food tests

The solution may be stored in any plastic or glass bottle and will keep indefinitely.

Procedure to make 250 mL of sodium hydroxide solution:
1)	Use a small PLASTIC spoon (tea size) to transfer one level spoon of sodium hydroxide into a 500 mL 	water bottle. Caustic soda (sodium hydroxide) reacts with metal. DO NOT TOUCH.

**SAFETY NOTE: prepare ~100ml of citric acid or ethanoic acid solution to have available to neutralize sodium hydroxide spills on skin or lab tables. One spoon of citric acid in about 100 mL of water is a suitable concentration. Ethanoic acid solutions of the proper concentration are sold in food shops as vinegar.**

2)	Add about 250 mL of water to the bottle. In the new 500 mL Maji Africa bottles, this is the first 	straight line above the curving lines. The addition of water to sodium hydroxide gets HOT.
3)	Cap the bottle very well and shake to mix.
4)	Use a permanent pen to label the bottle:

~1M SODIUM HYDROXIDE SOLUTION FOR FOOD TESTS
CORROSIVE. Do NOT vomit if swallowed.
Neutralize spills with weak acid soolution. 

	The hydroxide solution will react with carbon dioxide in the air if the container is not well sealed. The solution should not be stored in glass bottles with glass stoppers for overnight or longer as these will stick. The solution may be stored in plastic bottles indefinitely.

Test for lipids
Materials
	iodine tincture from the pharmacy without ethanol / alcohol / “spiriti”
** The absence of ethanol is critical. The Mansor Daya brand in 30-40 mL glass bottles works well. **
	tap or clean river water 

Procedure to make 400 mL
1)	Pour one bottle of iodine tincture into a 500 mL plastic water bottle.
2)	Add 300 – 400 mL of water. The exact amount is not important, and may be estimated.
3)	Cap the bottle and shake.
4)	Use a permanent pen to label the bottle:


IODINE SOLUTION FOR FOOD TESTS
HARMFUL. Vomit if swallowed.

	The solution may be stored in any plastic or glass bottle and will keep indefinitely.

A note on theory:
	The traditional chemical test for fats and oils is Sudan III solution. Sudan III is a red pigment that is considerably more soluble in fat than in water. After addition to a sample, the Sudan III will move the less dense fat layer that separates above the water. As the Sudan III is concentrated in this thin layer, a red ring forms at the top of the sample.
	Iodine exhibits the same property as Sudan III – it is more soluble in fat than in water. After the addition of iodine, the sample must be shaken to transfer the iodine to the fat. Then a red ring will form as with Sudan III. A major advantage of iodine is that the red color only occurs if fat is present; whereas Sudan III is red even in water, iodine is yellow in aqueous solution. A simple comparison should demonstrate the obvious superiority of iodine over Sudan III.
	Ethanol causes oil and water to form an emulsion which then takes considerable time to separate again. This is the reason why Sudan III tests often take a long time to give results. This is also the reason why the iodine tincture used must be without ethanol. The Mansor Daya brand works well.


Test for reducing sugars with Benedict's solution
Test for reducing sugars with Benedict's solution
Materials
	sodium carbonate if available, otherwise sodium hydrogen carbonate
	citric acid
	copper sulfate
	tap or clean river water
	plastic water bottle with screw cap

Procedure to make 500 mL

	Combine the following into the bottle in order, shaking after each addition: about 500 mL water, five spoons of sodium carbonate, three spoons of citric acid, and one spoon of copper sulfate. If copper sulfate is expensive, you can use half a spoon. The spoon should be an average size tea spoon. All measurements with the spoon should be level to ensure that the volume measured is consistent. The final solution should have a bright blue color.

	If you do not have sodium carbonate, add about ten spoons of sodium hydrogen carbonate to the water and boil for about ten minutes. This will convert the sodium hydrogen carbonate to sodium carbonate and carbon dioxide – bubbles of gas should be observed. Then add three spoons of citric acid, shake, and finally add one half to one spoon of copper sulfate.

	Label the final solution:
	
BENEDICT'S SOLUTION FOR FOOD TESTS.

	The solution may be stored in any plastic or glass bottle and will keep indefinitely.

Test for non-reducing sugar
Materials
Benedict's solution, above
~1M sodium hydroxide solution, above under protein test
citric acid
tap or clean river water

	To perform this test, students first test their sample with Benedict's solution to eliminate the possibility of a reducing sugar. Then, they must use an acid solution to hydrolyze any non-reducing sugars and then a base solution to neutralize the sample solution. Then the students perform again the test for non-reducing sugars.

Procedure to make 500ml of acid solution
	1)	Put 500 mL of water into a plastic water bottle.
	2)	Add 50 g of citric acid.
	3)	Cap the bottle very well and shake to mix.
	4)	Use a permanent pen to label the bottle:

0.5 M CITRIC ACID FOR FOOD TESTS

	The solution may be stored indefinitely in any plastic or glass container.

\subsection{Preparation of Food Solutions}

Try to make food solutions colorless so that students cannot guess what is in them, and so that they can see the colors that form during food tests. You do not need to measure the ingredients of the solution, but make sure to test solutions before the practical.

Starch—the easiest solution is the water left from boiling pasta or potatoes. If you have been only ugali and rice lately, add some wheat or corn flour to boiling water. Let cool. Decant the solution or filter it with a tea filter to remove the largest particles. If you are in a hurry, you can also just mix flour with cold water, but then it will be obvious that flour is present.

Protein—Combine egg whites with water. If you do not have egg whites, you can use fresh or powder milk, although this will give the solution a white color and add a reducing sugar (lactose). The water used in boiling beans also contains some protein, but it may not be enough to see a color change.

Reducing sugar—The easiest is to buy glucose powder from a shop and dissolve it in water. You can also grind pieces of onion with water and filter the resulting solution.

Non-reducing sugar—dissolve sugar in water. Table sugar is sucrose, a non-reducing sugar.

Fat—add cooking oil to water. Shake immediately before use. Sunflour oil is best – avoid fats that solidify near room temperature.

\subsection{How to Carry Out Food Tests}

Starch—Add a few drops of iodine to the solution and shake well. A blue-black color forms if starch is present

Protein (Biuret test)—Add a few drops of 1 M NaOH to the solution and shake well. Then add a few drops of 1\% CuSO4 solution and shake. A violet color forms if protein is present. Sometimes the color takes a minute or two to appear. 

Reducing sugar—Place some food solution in a test tube, and add an equal volume of Benedict’s solution. Heat to boiling, then let cool. A brick red or orange precipitate forms if a reducing sugar is present.

Non-reducing sugar—Do the test for a reducing sugar using Benedict's solution. Notice that no reaction occurs. (Hopefully students will already have done the test for a reducing sugar, and won't need to repeat it again here.) Add a few drops of citric acid solution to the solution, then heat to boiling. Let solution cool. Add a few drops of 1 M NaOH, and shake well. Then, add some Benedict’s solution (equal in volume to the liquid in the test tube). Boil the solution, and let it cool. A brick red or orange precipitate forms if a non-reducing sugar is present.

Notes
1. reducing sugars get their name from their tendency to reduce copper(II)sulfate to copper(I)sulfate, which produces the orange color. Non-reducing sugars do not have this  power.
2. the function of the HCl is to convert the non-reducing sugar to a reducing sugar. The function of NaOH is to neutralize the citric acid to prevent it from reacting with Benedict's solution. Then, Benedict’s solution is used to test for a reducing sugar. Note that this means false positives are possible—if a reducing sugar is present in the solution, the test for a non-reducing sugar will be positive as well. 
3. if the solution contains both reducing and non-reducing sugars, the presence of a non-reducing sugar cannot be confirmed, because the test for reducing sugars using Benedict's solution will be positive before HCl and NaOH are added.
4. You can use sodium carbonate solution in place of sodium hydroxide. This is safer, but you have probably already provided sodium hydroxide for protein tests.  

Fats/oils—Add a few drops of Sudan III to the test solution. Shake the mixture and let it stand. A red ring will appear on the surface if fat is present. Sudan III is absorbed by fats and not water, so the red color of Sudan III makes the separation of fat and water visible.

You can also have your students do the grease spot test—rub a drop of solution onto a  
piece of paper, and let dry. A translucent spot forms if fat is present. This test is great for its simplicity, but is not  used on national exams.

Past practicals used Millon’s solution to test for proteins. This solution is made from dissolving mercury metal in nitric acid – never use it.

\subsection{How to Write a Report}
Food test data is reported in a table containing four columns: test for, procedure, observation, and inference. With the exception of the ‘test for’ column, data should be reported in full sentences written in past tense. The procedure should also be in passive voice. No, this is not the way professional scientists write. However, students here must use passive voice to get marks on the national exam.

Note that every column is worth marks on the exam. Even if students fail to do the food tests correctly, they can still get marks for writing what they are testing for and what the procedure should be.

See the sample practical below for an example of a report.

\subsection{Sample Food Test Practical}

You have been provided with Solution K. Carry out food test experiments to identify the food substances present in the solution.

(a) Record your experimental work as shown in the table below.

Test for
Procedure
Observations
Inferences

(b) Suggest two natural food substances from which solution K might have been prepared.
(c) What is the function of each of the food substances in solution K to human beings?
(d) For each food substance identified, name the enzyme and end product 
  of digestion taking place in the:
(i) Stomach
(ii) Duodenum
(e) What deficiency diseases are caused by a lack of the identified food substances?

SOLUTIONS
(Assume Solution K contains protein and starch.)
(a)
Test for
Procedure
Observations
Inferences
Protein
A few drops of NaOH solution were added to Solution K. The solution was shaken. Then a few drops of CuSO4 solution were added to Solution K, and the solution was shaken again. 
A violet color was observed.
Protein was present.
Fats/oils
A few drops of Sudan III solution were added to Solution K. The solution was shaken and then allowed to stand.
A red ring did not form at the surface.
Fats/oils were absent.
Reducing sugar
A small amount of Benedict’s solution was added to Solution K. The solution was boiled and allowed to cool.
There was no precipitate.
Reducing sugars were absent.

Non-reducing sugar
A small amount of dilute HCl was added to Solution K. The solution was boiled and allowed to cool. Then a small amount of NaOH solution was added, and the solution was shaken. Finally, a small amount of Benedict’s solution was added. The solution was boiled and let cool. 
There was no precipitate.
Non-reducing sugars were absent.
Starch
A few drops of iodine solution were added to Solution K, and the solution was shaken.
A blue-black color was observed.
Starch was present.

(b) Solution K could have been prepared from egg and maize. (Any non-processed food containing protein or starch is correct here).
(c) Starch provides energy to the body. Proteins are used in growth and tissue repair.
(d) 
Food substance
Location
Enzyme
End product of digestion
Protein
Stomach
Pepsin
Polypeptides

Duodenum
Trypsin
Amino acids
Starch
Duodenum
Pancreatic amylase
Maltose

(e) A deficiency of protein causes kwashiorkor. A deficiency of starch causes marasmus.

%==============================================================================
\section{Classification}

The classification practical requires students to identify specimens of animals, plants, and fungi. The students must write the common name, kingdom, phylum, and sometimes class of each specimen. They also answer questions about the characteristics and uses of the specimens.

This section contains the following:
\begin{itemize}
\item{Common specimens}
\item{Where to find specimens}
\item{Storage of specimens}
\item{Sample practical with solutions}
\item{Additional classification questions}
\end{itemize}

Common Specimens
Specimens in italics are less common.
Fungi
  Mushroom, yeast, bread mold
Plants
  Fern, moss, bean plant, bean seed, maize plant, maize seed, pine tree, cactus, sugar cane, Irish potato1, cypress tree2, acacia tree, hibiscus leaf, cassava
Animals
  Millipede, centipede, grasshopper, lizard, tilapia (fish)3, scorpion, frog, tapeworm, liver fluke, cockroach, spider

Where to Find Specimens
Start collecting specimens several months before the NECTA exams, as some specimens can be hard to find in the dry season.
Ask your students to bring specimens! Students are especially good at finding insects and other animals. You can  even pay primary school children to gather insects such as grasshopper and millipedes. 
Where to find specimens:
Ferns, hibiscus, pines, and cypresses are used in landscaping. Try looking near nice hotelis or guestis. Ferns should have sori (sporangia) on the underside of their leaves.
Moss often grows near water tanks and in shady corners of courtyards. It is hard to find in the dry season.
Sugarcane, Irish potato, cassava, tilapia, bean seeds, and maize seeds can be found at the market. Yeast is available at dukas and is called hamira. 
Mushrooms are hard to find in the dry season. However, they are available at grocery stores in large cities, and you may be able to find dried mushrooms at the market. You can also collect mushrooms in the rainy season and dry them yourself.
Tapeworms and liver flukes may be acquired from butchers. Find out where livestock is slaughtered and ask the butchers to look for worms (minyoo). Liver flukes are found in the bile ducts inside the liver, while tapeworms are found in the intestines. You can also try going to a livestock fair/market (mnada) or talking to the local meat inspector (mkaguzi wa nyama).
Grow your own bread mold. Just put some bread in a plastic bag and leave it in a warm place. But do it ahead of time—it can take two weeks to obtain bread mold with visible sporangia.

Storage of Specimens
	Insects and mushrooms can be dried and stored in jars. However, they become brittle and break easily.
	A 10\% solution of formaldehyde is the best way of storing specimens. Formaldehyde is often sold as a 40\% solution. It should be stored in glass jars and out of the sun. Check specimens periodically for evaporation. Formaldehyde works because it is toxic; handle carefully.
	In a pinch, a 70\% solution of ethanol can also be used to store insects, lizards, and worms. However, specimens sometimes decay in ethanol. 
­­
SAMPLE CLASSIFICATION PRACTICAL

You have been provided with specimens L, M, N, O, and P.
(a) Identify the specimens by their common names.
(b) Classify each specimen to the phylum level.	
(c) 	(i) Write the classes of specimens L and M.
(ii) List two observable differences between specimens L and M.
(d) Explain why specimen P cannot grow taller.
(e) Write down two distinctive characteristics of the phylum to which specimen O belongs.
(f) 	(i) List the modes of reproduction in specimens M and N.
(ii) What are two differences between these modes of reproduction?

SPECIMENS
L: maize plant
M: bean plant
N: yeast
O: millipede
P: moss
(b) 
Specimen
Kingdom
Phylum
L (maize plant)
Plantae
Angiospermophyta
M (bean plant)
Plantae
Angiospermophyta
N (yeast)
Fungi
Ascomycota
O (millipede)
Animalia
Arthropoda
P (moss)
Plantae
Bryophyta

 (c) (i)	Specimen L (maize plant): Monocotyledonae
	Specimen M (bean plant): Dicotyledonae
      (ii) 
Maize Plant
Bean Plant
(i) Parallel veins
(i) Net veins
(ii) Fibrous roots
(ii) Tap roots
*the answers to this question should be differences between monocots and dicots that the student can see by observing the plants with their naked eyes. Hence answers such as “vascular bundles in a ring” are not correct.

(d) Specimen P (moss) cannot grow taller because it has no xylem and phloem. If it grew taller, it would not be able to transport food and water throughout the plant.
(e) Characteristics of phylum Arthropoda:
	-jointed legs
	-segmented body
	-exoskeleton made of chitin
(f) (i) Specimen M (bean plant) reproduces by sexual reproduction. Specimen N (yeast) reproduces by asexual reproduction.
     (ii) 
Asexual reproduction
Sexual reproduction
(i) There is no genetic variation 
  between offspring.
(i) There is genetic variation 
  between offspring.
(ii) Requires one parent only.
(ii) Usually requires two parents.
(iii) No gametes are involved.
(iii) Involves fusion of two gametes.

Additional Classification Questions
Identify specimen X, Y, and Z by their common names.
Classify specimens X, Y, and Z to the class level
	(This means write the kingdom, phylum, and class.)
Write the observable features of specimen X.
List three observable differences/similarities between specimens X and Y.
State the economic importance of specimen X.
What characteristics are common among specimens X and Y?
Why are specimens X and Y placed in different classes/phyla/kingdoms?
Why are specimens X and Y classified under the same class/phylum/kingdom?
What distinctive features place specimen X in its respective kingdom/phylum/class?
How is specimen X adapted to its way of life? 
Suggest possible habitats for specimens X and Y.
Which specimen is a primary producer/parasite/decomposer?
For mushroom, yeast, bread mold, grasshopper, moss, tilapia, liver fluke, and tapeworm:
Draw and label a diagram of specimen X.

For tilapia:
Draw a big and well-labeled diagram showing a lateral view of specimen X.

For maize and bean:
Mention the type of pollination in specimen X [wind pollinated or insect pollinated]. How is specimen X adapted to this type of pollination?
Mention the type of germination [hypogeal or epigeal] in specimen X.
For bean seed:
List three observable features of specimen X and state their biological importance.
Split specimen X into two natural halves. Draw and label the half containing the embryo.

For fern:
Observe the underside of the leaves of specimen X
What is the name of the structures you have observed?
Give the function of the structures named above.
Draw specimen X and show the structures named above.
%==============================================================================
\section{Respiration}

The purpose of this practical is to investigate the properties of air exhaled from the lungs. This section contains the following:

\begin{itemize}
\item{Limewater (properties and preparation)}
\item{Substitutions with local chemicals}
\item{Apparatus}
\item{Cautions and advice when using traditional materials}
\item{Sample practical with solutions}
\end{itemize}

\subsection{Limewate}r
Limewater is a saturated solution of calcium hydroxide. It is used to test for carbon dioxide. When carbon dioxide is bubbled through limewater, the solution becomes cloudy. This is due to the precipitation of calcium carbonate by the reaction:



Limewater can be prepared from either calcium hydroxide or calcium oxide. Calcium oxide reacts with water to form calcium hydroxide, so either way you end up with a calcium hydroxide solution.

Two ways to prepare limewater:
combine about 3 g Ca(OH)2 or CaO with 250 mL water and let it sit overnight. Filter in the morning.
add about 3 g Ca(OH)2 or CaO to 250 mL of boiling water. Let boil about five minutes, then filter. 
The exact mass of calcium hydroxide used is not important. Just check whether some calcium hydroxide remains undissolved at the end—a sign that you have made a saturated solution.
Test limewater by blowing air into a sample with a straw. It should become cloudy. If it does not, then the concentration of Ca(OH)2 is too low.

Substitutions with Local Chemicals

Calcium hydroxide is available from building supply stores (chokaa) and calcium oxide is the primary ingredient in cement. Satisfactory lime water can be prepared by mixing three spoons of either of these in a plastic bottle of ordinary water and then decanting the solution. Do not waste money on analytic grade calcium hydroxide or calcium oxide.
 
Apparatus
	Traditionally this practical calls for delivery tubes, test tubes, and stoppers. These are totally unnecessary. Add the limewater to any small clear container and blow into it with a straw.

Cautions and Advice When Using Traditional Materials
	If you use a delivery tube and pass it through a rubber stopper, do not use a single-holed stopper. This is what the pictures on NECTA practicals suggest, but it is a terrible idea. A single-holed stopper has no space for air to escape. So when a student blows air into the solution, the pressure in the test tube increases. The high pressure air then pushes limewater up the straw into the student's mouth. Alternatively, the student blows the stopper out of the test tube. If you use a stopper, use a double-holed stopper so that the extra air has a place to escape.
	Is a glass delivery tube stuck in a rubber stopper? Do not yank on it. Just  soak the stopper in warm water for a few minutes. The rubber will soften and the tube will come out.
	Are your test tubes and delivery tubes cloudy after the practical? Clean them with dilute acid. This will dissolve any calcium carbonate that has been deposited on the glass.


SAMPLE RESPIRATION PRACTICAL

You have been provided with Solution B in a test tube. Use a delivery tube to breathe (exhale) into the solution until its color changes. (See diagram below.) 

What is the aim of this experiment?
What is Solution B?
(i) What changes did you observe after breathing into Solution B?	
(ii) What can you conclude from these changes?
Breathe out over the palm of your hand. What do you observe?
Breathe out over a mirror. What do you observe?
Using your observations in the three experiments above, list three properties of exhaled air.
Explain why exhaled air is different from inhaled air. Where do the substances you identified in exhaled air come from?
SOLUTIONS
The aim of this experiment is to test exhaled air for carbon dioxide.
Solution B is limewater.
(i) Solution B became cloudy (or milky).
(ii) Conclusion: exhaled air contains carbon dioxide.
Air breathed out over the palm of the hand is warm.
Droplets of water condense on the mirror.


Conclusions:
exhaled air contains carbon dioxide
exhaled air contains water
exhaled air is warm
	(g) Exhaled air contains the waste products of aerobic respiration. The carbon dioxide and 
	  water in exhaled air are products of respiration.

%==============================================================================
\section{Transport}

The purpose of this practical is to investigate osmosis by observing the changes in a leaf petiole placed in a hypotonic solution (water) and a hypertonic solution (water containing salt or sugar). 

This section contains the following:
\begin{itemize}
\item{Materials}
\item{Sample practical with solutions}
\item{Additional questions}
\end{itemize}

\subsection{Materials}
The petiole is the stalk which attaches a leaf to a branch. The papaya leaf petioles in this practical should be soft petioles from young leaves, not stiff petioles from older leaves. Cut the petioles into pieces, and give each student two pieces of about 6 cm in length. Cylinders cut from a raw potato  may be used instead of petioles.

The hypertonic solution may be made with by mixing either salt or sugar with water. The hypotonic solution is tap water.

\subsection{Sample Transport Practical}

\subsubsection{Instructions}

You have been provided with two pieces of a papaya leaf petiole, Solution A, and Solution B.
 
Use a razor blade to split the pieces of petiole longitudinally, up to a half of their length.  You should have four strips at one end of each petiole, while the other end remains intact. 
      (See diagram below).

Place one petiole in solution A, and place the other petiole in solution B. Let the petiole  sit for about ten minutes, then touch them to feel their hardness or softness.

Draw a sketch of each petiole after sitting in its respective solution for ten minutes.

Record your observations and explanations about the petioles in the table below.

%%%%%%%%%%%%%%%%%%%%TABLE
Solution
Observation
Explanation
A
B
%%%%%%%%%%%%%%%%%%%%TABLE

\subsubsection{Questions}

What was the aim of this experiment?
    (i) What was the biological process demonstrated by this experiment?
              (ii) What is the importance of this process to plants?
Which solution contained:
1. pure water 
2. a high concentration of solutes
What happened to the cells of the petioles in each solution? Illustrate your answer.
What would happen to the cells of the petioles in solution A if their cell walls were removed?

SOLUTIONS

(Assume Solution A is pure water, and Solution B is a concentrated solution of water and salt.)

(a) 
(b)

Solution
Observation
Explanation
A
The petiole became hard (turgid).
Water diffused into the petiole cells.
B
The petiole became soft (flaccid).
Water diffused out of the petiole cells.

(c) The aim of the experiment was to investigate the effect of osmosis on plant cells.
(d)       (i) The experiment demonstrated osmosis.
(ii) Importance of osmosis in plants: 
1. Water enters plant cells by osmosis so that they become turgid. Turgor helps support the plant and hold it upright.
2. Water diffuses into the xylem from the soil via osmosis.
(e)      (i) Pure water: Solution A. 
      (ii) High concentration of solutes: Solution B
(f)  *The diagrams below show the following cell parts (from outside moving in): cell wall, cell membrane, vacuole, and nucleus).

(g) The petiole cells would burst in Solution A if their cell walls were removed.

Additional Questions
	You can extend this experiment by giving students two pieces of meat in addition to the petioles. The piece of meat placed in pure water should expand and become soft due to the cells bursting. The piece placed in salt water should shrink and become hard due to water diffusing out of the cells. This experiment helps to teach the different effects of osmosis on plant and animal cells.
	If your school has a good microscope, try observing plant cells under the microscope after letting them sit in hypotonic and hypertonic solutions. 
	You can add critical thinking questions to the practical that require the student to use their knowledge of osmosis. For example:
Why does a freshwater fish die if it is placed in salt water?
Why do merchants spray vegetables with water in the market?
You can die if a doctor injects pure water into your bloodstream. Why?

%==============================================================================
\section{Photosynthesis}

The purpose of this practical is to prove that chlorophyll is required for photosynthesis. This is done by using iodine to test a variegated leaf for starch. The parts of the leaf containing chlorophyll are expected to contain starch, while the parts lacking chlorophyll are expected to lack starch.

This section contains the following:
\begin{itemize}
\item{Procedure}
\item{Cautions}
\item{Materials and where to find them}
\item{Sample practical with solutions}
\item{Additional practicals}
\end{itemize}

\subsection{Procedure}
Prepare the iodine solution used in food tests, that is, one unit of pharmacy iodine tincture for every nine units of water. For example, 10ml of iodine tincture in 90ml of water.
Prepare hot water bathes. The water should be boiling.
While the water gets hot, send the students to gather small leaves. The best have no waxy coating and are varigated (have sections without green).
The leaves should be boiled in the hot water bath for one minute.
Each group should then move its leaf into their test tube and cover it with methylated spirit.
Each group should then heat their test tube in a water bath. Over time, the leaf should decolorize and the methylated spirit will turn bight green. The chlorophyll has been extracted and moved to the spirit. A well chosen leaf should turn completely white, although this does not always happen.
After decolorization, dips the leaves briefly in the hot water.
For leaves that turn white, students should test them for starch with drops of iodine solution.

\subsection{Cautions}
-Ethanol is flammable! It should never be heated directly on a flame. Use a hot water bath—place a test tube or beaker of ethanol in a beaker or bowl of hot water and let it heat slowly. The boiling point of ethanol is lower than the boiling point of water, so it will start boiling before the water. If the ethanol does catch fire, cover the burning test tube with a petri dish or other non-flammable container to extinguish the flame.

\subsection{Materials and Where to Find Them}
-variegated leaf: this is a leaf that contains chlorophyll in some parts, but not in others. Often variegated leaves are green and white or green and red. Look at the flower beds around the school and at the teachers’ houses—they often contain variegated leaves. Test the leaves before the practical, as some kinds are too waxy to be decolorized by ethanol. Also, check for chlorophyll by looking at the underside of the leaves; the leaves you use have at least a small section of white on their undersides, signifying a lack of chlorophyll.
-source of heat: anything that boils water – Motopoa is best, followed by kerosene and charcoal
-ethanol: use the least expensive strong ethanol available; this is probably methylated spirits unless your village specializes in high proof gongo.

\subsection{Sample Photosynthsis Practical}

You have been provided with specimen G. 
Identify specimen G.
Make a sketch showing the color pattern of specimen G.
Carry out the following experiment:
1. Place specimen G in boiling water for one minute. 
2. Boil specimen G in ethanol using a hot water bath. Do not heat the ethanol directly on a flame. 
3. Remove specimen G from the ethanol. Dip it in hot water.
4. Spread specimen G on a white tile and drip iodine solution onto it. Use enough iodine to cover the entire specimen.
5. Make a sketch showing the color pattern of specimen G at the end of the experiment.
What was the aim of this experiment?
Why was specimen G
1. Boiled in water for one minute
2. Boiled in ethanol
3. Dipped in hot water at the end of the experiment
What was the purpose of the iodine solution? 
Why was the ethanol heated using a hot water bath?
What can you conclude from this experiment? Why?
	 
SOLUTIONS
Specimen G is a variegated leaf.
Drawing:
   

See diagram above.
The aim of this experiment was to investigate whether chlorophyll is required for photosynthesis.
Specimen G was:
1. boiled in water to kill the cells and stop all metabolic processes. 
2. boiled in ethanol to decolorize it (to remove the chlorophyll).
3. dipped in hot water to remove the ethanol. (If ethanol is left on the leaf it will become hard and brittle.)
The purpose of the iodine solution was to test for starch.
The ethanol was heated using a hot water bath because ethanol is flammable.
The experiment shows that chlorophyll is required for photosynthesis. We know this because the parts of the leaf containing chlorophyll also contained starch, which is a product of photosynthesis. Thus, the parts of the leaf containing chlorophyll performed photosynthesis. The parts of the leaf lacking chlorophyll lacked starch. Hence, these parts of the leaf did not perform photosynthesis.

Additional Practicals
-To test if light is required for photosynthesis:
	Take a live plant, and leave it in the dark for 24 hours to destarch all leaves. Then, cover some of its leaves with cardboard or aluminum foil, while leaving others uncovered. Let the plant sit in bright light for several hours. Give each group of students one leaf that was covered in cardboard, and one leaf that was uncovered. Have them use the procedure above to test for starch. They should find that the covered leaf contains no starch, while the uncovered leaf contains starch.
	(A cool variation on this experiment is to cover leaves with pieces of cardboard that have letters or pictures cut out of them. The area where the cardboard is cut out will perform photosynthesis and produce starch. When the students do a starch test, a blue-black letter or picture will appear on the leaf).

-To prove that oxygen is a product of photosynthesis
	This experiment requires a water plant. Basically, place a live water plant under water*, then cover it with an inverted funnel. Place an upside-down test tube filled with water on top of the funnel. Let the plant sit in bright light until the water in the test tube is displaced and the test tube fills with gas. Use a glowing splint to test the gas—if it is oxygen, it will relight the splint.

*Note: the TIE Biology practical manual suggests putting sodium bicarbonate (baking soda) in the water. 
