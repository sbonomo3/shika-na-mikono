\chapter{The Shika na Mikono Teaching Philosophy}

Science is the study of the natural world. 
To learn science, students must interact with the world around them. 
They must ask their own questions and seek their own answers. 
They must see things and they must grasp them in their hands; 
hence the name of this book: 
\textit{shika na mikono} is Swahili 
for ``grasping in hands.''

It is our fervent belief that every student in the world 
should perform science practical exercises. 
For too long we have heard complaints that schools lack the materials 
necessary for these exercises. 
This book attempts to make clear that students may perform 
science practicals at any school, 
most especially at those without traditional laboratories, starting today. 
Everything teachers need to create these hands-on learning experiences 
is available locally and/or at low cost.

Many national syllabi require practical exercises, 
often on their national examinations. 
This is good. 
Critically, however, we urge teachers to expand the scope of 
students' hands-on work beyond the practicals for the national exams. 
Every topic, every lesson may be a ``practical,'' 
not just a demonstration on the front bench 
but an opportunity for students to touch 
and manipulate and discover on their own.

In this vision, the science teacher becomes a guide, 
someone who can assemble parts of the natural world 
into a compelling lesson and ask the questions that help students see how things work. 
In this capacity, the science teacher remains a resource 
irreplaceable by the march of technology. 
Photocopy machines produce student editions of notes 
much more efficiently than teachers copying them to the board 
for students to copy again. 
Instructional films shown on low cost solar-powered projectors 
offer students articulate explanations and demonstrations. 
But no technology can replace the essential role of 
the modern science teacher: she is an architect, 
one who builds a space in which students can learn for themselves, 
and a shepherd, who tends to their learning through that discovery.

The aim of this book is to inspire and empower this sort of teaching. 
For years, many educators have bantered about the phrase 
``student-centered teaching.'' 
This sounds rather like patient-centered medicine -- 
anything else is simply absurd. 
The focus of a lesson must always be the experience of the student. 
To prepare such a lesson, the teacher should answer the following questions. 
What will the student do in class? 
How will she use her hands to interact with the world? 
What will the student observe with her own senses? 
Given these experiences, 
what questions will arise from the student's observations? 
Given these questions, 
how might the teacher respond to provoke further inquiry and 
critical thinking? 
How might the student's peers respond to build a common understanding? 
How might the student, through further observation and experimentation, 
arrive at the answer herself? 
Given these goals, 
what experiences will put her on the journey to that answer? 
What series of activities should be offered to her 
to facilitate that discovery? 
This is student-centered teaching -- 
a lesson plan crafted around the experience of the student, 
the internal, cognitive, 
and emotional experience of being in class that day.

The process of answering these questions involves several steps. 
The teacher must first organize the material that 
the student is to understand into a well-structured framework: 
logical, sequential, and hierarchical. 
Then, using this framework, the teacher should design activities 
for the student to discover each aspect of the material. 
These activities should be sequenced to expand understanding, 
moving from simple phenomena to the more complex, 
from the specific to the general. 
Discussion questions should seek first to uncover core phenomena 
and then to link each new insight with 
what the students already understand about their world. 
Targeted questions catalyze introspection, group discussions, 
and the realizations necessary for the students themselves 
to start articulating scientific theories. 
Once the students have discovered phenomena, 
linked them to pre-existing understanding, 
and begun articulating general theory, 
the teacher can help focus and form these articulations 
into the accepted vocabulary and nomenclature of modern science.

In this vision, 
the students learn material from their experience and 
their reflection on that experience. 
They believe in theories because they have demonstrated and 
articulated them for themselves. 
In this vision, 
science becomes the study of reality, 
an ever-growing understanding combined with a powerful set of mental tools 
to bear on all parts of life. 
What students learn in the classroom connects with and 
illuminates aspects of life at home, 
in the village, in town, and on the farm. 
The capacity that students gain to ask critical questions and 
seek their own answers empowers them well beyond 
high scores on formal assessments; 
the scientific mindset allows students to seek truth in all matters, 
and to invent solutions to the many challenges before them -- 
not just for test questions in school, 
but for the ones that really matter in life.

This ultimate achievement, 
that students gain something in the classroom with 
value beyond the limits of the school, 
is further incentive to embrace the style 
of teaching through hands-on activities using local materials 
as we advocate. 
Few students will encounter professional scientific instruments 
later in their lives; 
an understanding founded on exotic apparatus and 
imported high-end chemicals has little applicability to life after school. 
When students explore the world with readily available materials, 
however -- 
when they see parts of their own world appear in the classroom 
for focused experimentation and analysis -- 
they gain an understanding that bridges scientific theory and 
daily reality, 
that sheds light on the world beyond the laboratory, 
and that lets them wield scientific thinking anywhere.

Hands-on science education is possible anywhere. 
The materials we need are available in our villages and in our towns. 
The key ingredients in science education are not precision glassware, 
imported reagents, nor massive loans. 
The key ingredients are curiosity, creativity, 
and the ability of teachers to use what they already have 
to provide students with experiences 
that broaden their understanding of the world.

Finally, to realize this vision, 
we teachers must embrace questions; 
we must encourage students to ask about what they do not understand. 
Rather than answer these questions directly, 
whenever possible we should design experiments 
or ask questions in return that allow students 
to find answers for themselves. 
As role models, we also must embrace the limits of our own understanding. 
Often students ask questions to which we do not know the answer. 
This is a fundamental aspect of science education. 
Our job is to help students to understand the world better, 
to guide them in that discovery. 
Our job is not to know everything; this is neither necessary, 
nor is it possible, nor even desirable. 
When our students observe us confronting the unknown, 
when they see how we ask questions and 
perform experiments ourselves to seek out the truth, 
then they become more comfortable asking questions 
and seeking answers themselves. 
This experience helps them to understand the true power of science, 
that a person anywhere may always find the answer.

Let us gather the world around us and 
put it in the hands of our students, 
so they might understand how it works. 
Let us let them grasp it in their hands -- 
\textit{walishike na mikono yao.}
