\chapter{Preparation of Solutions without a Balance}

The procedure in the section on Relative Standardization allows us to do something seemingly impossible – prepare solutions for volumetric analysis that allow students to get perfect results without using either a balance or volumetric glassware in the preparation. All that you have to do is make two solutions that are close, and then use several cycles of relative standardization to prefect the molarity ratio. 

To measure volume, we can use marks on plastic water bottles as described in the entry for volumetric glassware in the Sources of Equipment section. What follows is an example of how rough solutions can be prepared in Tanzania based on the water bottles available in that country. We encourage people in other country to calibrate their water bottles and then to customize these instructions for the resources available to them.

\section{To make 0.05~M sulfuric acid (equivalent to 0.1~M HCl) for fifty students}
\begin{enumerate}
\item{Put 9.9 liters of water into a bucket. On the new 1.5~L Kilimanjaro water bottle, the bottom points of the crown embossed on the side correspond to 300~ml and the top of the mountain corresponds to 1.5~L. Therefore one can measure 9.9 liters by filling the bottle to the mountain top six times and then to the bottom points of the crown three times.}
\item{Add 110~mL of battery acid. This may be accomplished easily by filling a 10~mL plastic syringe eleven times. Please read the safety note in Dangerous Chemicals.}
\end{enumerate}

\section{To make 0.033~M citric acid (equivalent to 0.1~M HCl) for fifty students}
\begin{enumerate}
\item{Put 10 liters of water into a bucket. One the new 500~mL Kilimanjaro water bottle, the second straight line corresponds to 300~mL and the highest straight line corresponds to 400~mL. Therefore one can use the 1.5~L bottle six times to add nine liters and then use the 500~mL bottle to add one more liter, 400~mL + 300~mL + 300~mL.}
\item{Add 64~g of citric acid. In the absence of a balance, one can often have $^1/_8$ of a kilogram (125~g) measured in the market. Dissolved this in 20~L of water to produce a 0.033~M solution. Alternately, use a plastic syringe to find the volume of a plastic spoon. Fill the spoon with citric acid and push off any extra acid until there is a flat surface (like the water). Then use that spoon to add a total $38 \mathrm{cm}^3$ or mL of citric acid soda knowing the volume of each spoonful.}
\end{enumerate}

\section{To make 0.1~M sodium hydroxide for fifty students}
\begin{enumerate}
\item{Put 10 liters of water into a bucket. See the instructions above.}
\item{Add 40~g of caustic soda. In the absence of a balance, measure the volume of a spoon as above and add $19 \mathrm{cm}^3 \mathrm{or mL}$ of caustic soda. Please read the safety note in Dangerous Chemicals.}
\end{enumerate}

\section{To make 0.1~M sodium hydrogen carbonate for fifty students}
\begin{enumerate}
\item{Put 10 liters of water into a bucket. See the instructions above.}
\item{Add 84~g of bicarbonate of soda. In the absence of a balance, find the volume of a spoon as above and add $39 \mathrm{cm}^3 \mathrm{or mL}$ of bicarbonate of soda. Alternately, if 8.33 liters of solution is sufficient, measure this volume of water and then add one whole box of bicarbonate of soda. A box is 70~g.}
\end{enumerate}
